% ---------------------------------------------------------------------
% Podstawowe ustawienia i pakiety
% ---------------------------------------------------------------------
\RequirePackage[l2tabu, orthodox]{nag} % Wykrywa przestarzałe i niewłaściwe
% sposoby używania LaTeXa. Więcej jest w l2tabu English version.
\documentclass[a4paper,11pt]{article}
% {rozmiar papieru, rozmiar fontu}[klasa dokumentu]
\usepackage[MeX]{polski} % Polonizacja LaTeXa, bez niej będzie pracował
% w języku angielskim.
\usepackage[utf8]{inputenc} % Włączenie kodowania UTF-8, co daje dostęp
% do polskich znaków.
\usepackage{lmodern} % Wprowadza fonty Latin Modern.
\usepackage[T1]{fontenc} % Potrzebne do używania fontów Latin Modern.



% ------------------------------
% Podstawowe pakiety (niezwiązane z ustawieniami języka)
% ------------------------------
\usepackage{microtype} % Twierdzi, że poprawi rozmiar odstępów w tekście.
% \usepackage{graphicx} % Wprowadza bardzo potrzebne komendy do wstawiania
% grafiki.
% \usepackage{verbatim} % Poprawia otoczenie VERBATIME.
% \usepackage{textcomp} % Dodaje takie symbole jak stopnie Celsiusa,
% wprowadzane bezpośrednio w tekście.
\usepackage{vmargin} % Pozwala na prostą kontrolę rozmiaru marginesów,
% za pomocą komend poniżej. Rozmiar odstępów jest mierzony w calach.
% ------------------------------
% MARGINS
% ------------------------------
\setmarginsrb
{ 0.7in}  % left margin
{ 0.6in}  % top margin
{ 0.7in}  % right margin
{ 0.8in}  % bottom margin
{  20pt}  % head height
{0.25in}  % head sep
{   9pt}  % foot height
{ 0.3in}  % foot sep



% ------------------------------
% Często używane pakiety
% ------------------------------
\usepackage{csquotes} % Pozwala w prosty sposób wstawiać cytaty do tekstu.
\usepackage{xcolor} % Pozwala używać kolorowych czcionek (zapewne dużo
% więcej, ale ja nie potrafię nic o tym powiedzieć).



% ------------------------------
% Pakiety do tekstów z nauk przyrodniczych
% ------------------------------
\let\lll\undefined % Amsmath gryzie się z językiem pakietami do języka
% polskiego, bo oba definiują komendę \lll. Aby rozwiązać ten problem
% oddefiniowuję tę komendę, ale może tym samym pozbywam się dużego Ł.
\usepackage[intlimits]{amsmath} % Podstawowe wsparcie od American
% Mathematical Society (w skrócie AMS)
\usepackage{amsfonts, amssymb, amscd, amsthm} % Dalsze wsparcie od AMS
% \usepackage{siunitx} % Dla prostszego pisania jednostek fizycznych
\usepackage{upgreek} % Ładniejsze greckie litery
% Przykładowa składnia: pi = \uppi
\usepackage{slashed} % Pozwala w prosty sposób pisać slash Feynmana.
\usepackage{calrsfs} % Zmienia czcionkę kaligraficzną w \mathcal
% na ładniejszą. Może w innych miejscach robi to samo, ale o tym nic
% nie wiem.





% ---------------------------------------------------------------------
% Dodatkowe ustawienia dla języka polskiego
% ---------------------------------------------------------------------
\renewcommand{\thesection}{\arabic{section}.}
% Kropki po numerach rozdziału (polski zwyczaj topograficzny)
\renewcommand{\thesubsection}{\thesection\arabic{subsection}}
% Brak kropki po numerach podrozdziału



% ------------------------------
% Pakiety których pliki *.sty mają być w tym samym katalogu co ten plik
% ------------------------------
\usepackage{latexgeneralcommands}
\usepackage{mathcommands}



% ------------------------------
% Ustawienia różnych parametrów tekstu
% ------------------------------
\renewcommand{\baselinestretch}{1.1}

\renewcommand{\arraystretch}{1.4}  % Ustawienie szerokości odstępów między
% wierszami w tabelach.



% ------------------------------
% Pakiet „hyperref”
% Polecano by umieszczać go na końcu preambuły
% ------------------------------
\usepackage{hyperref} % Pozwala tworzyć hiperlinki i zamienia odwołania
% do bibliografii na hiperlinki










% ---------------------------------------------------------------------
% Tytuł, autor, data
\title{Filozofia \\
  {\Large błędy i~uwagi}}

\author{Kamil Ziemian}


% \date{}
% ---------------------------------------------------------------------










% ####################################################################
% Początek dokumentu
\begin{document}
% ####################################################################





% ######################################
\maketitle  % Tytuł całego tekstu
% ######################################





% ######################################
\section{Filozofia wraz z~świętą wiarą}

% \vspace{\spaceTwo}
% ######################################










% ######################################
\subsection{Dzieła bliższe Bogu}

\vspace{\spaceTwo}
% ######################################



% ############################
\Work{ % Autor i tytuł dzieła
  Jacques Maritain \\
  \textit{Trzej reformatorzy. Luter, Kartezjusz, Rousseau},
  \cite{MaritainTrzejReformatorzy2005} }

\vspace{0em}


% ##################
\CenterBoldFont{Uwagi do konkretnych stron}

\vspace{0em}


\StrWd{115}{9} Na~tą linią nie powinno być odstępu w~tekście, zaś~w~niej
samej brakuje wcięcia akapitu.

\vspace{\spaceFour}





\StrWg{119}{12} W~tej linii brakuje wcięcia akapitu.

\vspace{\spaceFour}





\StrWd{126}{11} W~tej linii brakuje wcięcia akapitu.

\vspace{\spaceFour}





\StrWd{147}{14} Na~tą linią nie powinno być odstępu w~tekście, zaś~w~niej
samej brakuje wcięcia akapitu.

\vspace{\spaceFour}





\Str{160} Przypis (100) prawie na~pewno podaje odniesieni do~nie tych
źródeł co~trzeba.

\vspace{\spaceFour}





\StrWg{170}{12} W~tej linii brakuje wcięcia akapitu.

\vspace{\spaceFour}





\StrWg{193}{3} W~tej linii brakuje wcięcia akapitu.

\vspace{\spaceFour}





\StrWg{194}{6} W~tej linii brakuje wcięcia akapitu.

\vspace{\spaceFour}





\StrWg{198}{9} W~tej linii brakuje wcięcia akapitu.

\vspace{\spaceFour}





\Str{209} Przypisy (140) i~(141) są~umieszczone w~taki sposób w~tekście,
że~wygląda to na~błąd. Nie wiem jednak jaka to~poprawić.

\vspace{\spaceFour}





\Str{233} Nie jestem w~stanie zrozumieć co~oznacza zdanie
„Luter do~tego stopnia zapoznaje prawdę, że~miłość jest w~nas wlanym
uczuciem w~samym życiu Boga i~Chrystusa, które otrzymuje w~zasłudze
krwi Chrystusowej.”

\vspace{\spaceFour}





% ##################
\newpage

\CenterBoldFont{Błędy}

\vspace{\spaceFive}


\begin{center}

  \begin{tabular}{|c|c|c|c|c|}
    \hline
    Strona & \multicolumn{2}{c|}{Wiersz} & Jest
                              & Powinno być \\ \cline{2-3}
    & Od góry & Od dołu & & \\
    \hline
    18  & 13 & & JózefTischner & Józef Tischner \\
    20  & &  9 & dobra & dobra” \\
        25  & 14 & & że~\ldots~za & że~za \\
    29  & &  4 & deorsum” & deorsum \\
    35  & &  4 & słabości\ldots & słabości. \\
    35  & &  2 & ( dalej & (dalej \\
    39  &  8 & & stanie & stanu \\
    40  & &  2 & małżeństwie,1522 & małżeństwie, 1522 \\
    40  & &  2 & X. & X \\
    42  & 13 & & świętej\ldots & świętej \\
    42  & &  1 & WA. & WA \\
    42  & &  1 & X. & X \\
    48  & 10 & & teologii?$^{ 27 }$ & teologii$^{ 27 }$? \\
    49  & &  3 & $^{ ( 16 ) }$Komentarz & $^{ ( 16 ) }$ Komentarz \\
    49  & &  3 & XL. & XL \\
    52  & &  1 & \textit{theol.} & \textit{theol.}, \\
    55  & &  1 & II. --~II & II \\
    65  & &  4 & VI & \textit{VI} \\
    66  & &  6 & XL,P & XL, P \\
    69  &  5 & & wskutek & nie wskutek \\
    70  & & 12 & --~Wystarczy & Wystarczy \\
    84  & &  2 & $^{ ( 59 ) }$\hspace{0.3em} Jacques-Benigne
           & $^{ ( 59 ) }$ Jacques-Benigne \\
    89  & &  1 & Pochwała Kartezjusza & \textit{Pochwała Kartezjusza} \\
    104 & &  4 & --~Nie & Nie \\
    122 & 11 & & niekształtowanej & nieukształtowanej \\
    133 & &  4 & (Math., IV,3) & Mat. IV, 3 \\
    133 & &  1 & (Teresa) & Teresa \\
    135 &  5 & & J & Jest \\
    136 & 14 & & ono zależne & ona zależna \\
    136 & &  9 & niż raczej & raczej niż \\
    137 & 12 & & \textit{Sturmund} & \textit{Sturm und} \\
    147 & &  8 & ludzki”$^{ ( 79 ) }$ . & ludzki”$^{ ( 79 ) }$. \\
    149 & &  1 & Ibid & Ibid. \\
    150 &  4 & & grobu?$^{ 62 }$. & grobu$^{ 62 }$? \\
    150 & &  1 & Heloizy” & Heloizy”. \\
    \hline
  \end{tabular}





  \newpage

  \begin{tabular}{|c|c|c|c|c|}
    \hline
    Strona & \multicolumn{2}{c|}{Wiersz} & Jest
                              & Powinno być \\ \cline{2-3}
    & Od góry & Od dołu & & \\
    \hline
    157 & &  1 & $^{ ( 98 ) }$Jean & $^{ ( 98 ) }$ Jean \\
    162 &  2 & & żyć & żyć” \\
    162 &  7 & & wcale: & wcale, \\
    162 &  7 & & najpierwszym & „najpierwszym \\
    168 & &  1 & II-II & I-II \\
    173 & &  8 & kija obręczy & kija lub obręczy \\
    174 & &  7 & „PIEKNE & <<PIĘKNE  % >>
    \\
    174 & &  7 & {\footnotesize MA}” & {\footnotesize MA}>>” \\
    174 & &  6 & LXXVII. & LXXVII \\
    174 & &  5 & Seilli\`{e}re,\textit{J.J.}
           & Seilli\`{e}re, \textit{J.J.} \\
    178 & &  7 & ludzkości\ldots & ludzkości. \\
    179 & &  4 & jeśli & „jeśli \\
    180 & &  1 & \textit{partes}) & \textit{partes}). \\
    181 & & 13 & pełniona & spełniona \\
    184 &  6 & & woli\ldots & woli. \\
    224 &  1 & & sprawiedliwości.”., & sprawiedliwości.”. \\
    224 &  2 & & Paryż & (Paryż \\
    224 & 12 & & z~którego & o~którą \\
    225 &  4 & & X. & X \\
    229 & 13 & & błyskawicami.. & błyskawicami \\
    233 & 11 & & XL & XL, \\
    234 & &  3 & owszem!, & owszem!”, \\
    236 & 11 & & Dobra & dobra \\
    243 & &  1 & 1767roku & 1767~roku \\
    \hline
  \end{tabular}

\end{center}

\vspace{\spaceTwo}


\noindent
\StrWd{23}{8} \\
\Jest  Człowiekwspółczesnyszamoczesięwewłasnymwnętrzuniczym \\
\Powin Człowiek współczesny szamocze się we własnym wnętrzu niczym \\
\StrWg{143}{6} \\
\Jest  Dostojewski Andre Gide'a będzie szukał \\
\Powin Dostojewski i~Andre Gide'a będą szukali \\
\StrWg{198}{13} \\
\Jest  romantyczno\dywiz naturalistyczny $^{ 74 }$ \\
\Powin romantyczno\dywiz naturalistyczny$^{ 74 }$ \\
\StrWd{223}{9} \\
\Jest  G{\footnotesize DYBYM NIE BYŁ DOKTOREM} \\
\Powin Gdybym nie był doktorem \\
\StrWd{240}{7} \\
\Jest  ODPOWIEDŹ NA~PIERWSZE \\
\Powin \textit{Odpowiedź na~pierwsze zarzuty} \\


% ############################










% ######################################
\newpage

\subsection{Dzieła obojętne lub~wrogie Bogu}

\vspace{\spaceTwo}
% ######################################



% ############################
\Work{ % Autor i tytuł dzieła
  Tomasz Herbich \\
  \textit{Pragnienie Królestwa. August Cieszkowski, Mikołaj Bierdiajew \\
  i~dwa oblicza mesjanizmu}, \cite{HerbichPragnienieKrolestwa2018}}

\vspace{0em}


% ##################
\CenterBoldFont{Uwagi}

\vspace{0em}


\StrWg{54}{9} Nie wiem czy zadanie „Pierwotnie jednak
Chrześcijaństwo zostało pojęte czysto zaprzeczenie.” zawiera błąd,
czy też wtedy pisało~się w~takim stylu.

\vspace{\spaceFour}





% \start





% ##################
\newpage

\CenterBoldFont{Błędy}

\vspace{\spaceFive}


\begin{center}

  \begin{tabular}{|c|c|c|c|c|}
    \hline
    & \multicolumn{2}{c|}{} & & \\
    Strona & \multicolumn{2}{c|}{Wiersz} & Jest
                              & Powinno być \\ \cline{2-3}
    & Od góry & Od dołu & & \\
    \hline
    20  & &  3 & Paprocki,Kęty & Paprocki, Kęty \\
    % & & & & \\
    % & & & &  \\
    43  & 12 & & \textit{nasz}nie & \textit{nasz} nie \\
    43  & &  8 & Pan!\ldots”$^{ 17 }$ & Pan!\ldots$^{ 17 }$ \\
    46  & 14 & & tenże,\textit{Augusta} & tenże, \textit{Augusta} \\
    61  & 13 & & \textit{Bog}a & \textit{Boga} \\
    % & & & & \\
    % & & & & \\
    % & & & & \\
    % & & & & \\
    % & & & & \\
    % & & & & \\
    \hline
  \end{tabular}

\end{center}

\vspace{\spaceTwo}


\noindent
\StrWd{38}{3} \\
\Jest  Cieszkowskiego$^{ 8 }$.Zdystansowanie \\
\Powin Cieszkowskiego$^{ 8 }$. Zdystansowanie \\




% ############################










% ######################################
\newpage

\section{Dzieła konkretnych filozofów}

\vspace{\spaceTwo}
% ######################################




% ######################################
% \newpage

\subsection{Gottlob Frege}

\vspace{\spaceTwo}
% ######################################

















% ######################################
\newpage

\subsection{Eric Voegelin}

\vspace{\spaceTwo}
% ######################################



% ############################
\Work{ % Autor i tytuł dzieła
  Eric Voegelin \\
  \textit{Nowa nauka polityki}, \cite{VoegelinNowaNaukaPolityki1992}}

\vspace{0em}


% ##################
\CenterBoldFont{Uwagi}

\vspace{0em}


Zapomniałem już na której stronie Voegelin wprowadza pojęcie
gnostyckiego snu, z~którego tłumaczeniem na język polski
jest pewien problem. Po
angielsku „sleep” określa stan człowieka, gdy ten śpi, lecz sny
które mogą człowieka najść określa~się słowem „dream”. Z~tego
względu w~języku angielskim jest jasne, czy Voegelin mówił o~„gnostic
sleep”, czyli stanie w~którym człowiek ma radykalnie osłabiony
kontakt z~rzeczywistością, czy też o~„gnostic dream”, czyli życiu
w~świecie dzikich fantazji. Tłumacz nie zaznaczając tej ważnej różnicy
w~tym wydaniu popełnił bardzo poważny błąd. Korzystając z~Internetu
starałem~się sprawdzić która rozumienie jest poprawne, nie mam
rozstrzygających informacji, lecz artykuł anglojęzyczny który
znalazłem, przywołując tę~myśl Voegelina, używał sformułowania
„gnostic dream”.

\vspace{\spaceFour}





% ##################
\CenterBoldFont{Uwagi do konkretnych stron}


\Str{64} Przeczytałem tłumaczenie tu przywoływanych fragmentów
\textit{Państwa} 368 c--d i~nie znalazłem w~nich stwierdzenia,
że~\textit{polis} to wielki człowiek, była za to stwierdzenie,
iż~jest większe niż człowiek, więc i~sprawiedliwość jest
w~nim większa i~łatwiejsza do zauważenia. W~momencie gdy Glaukon mówi,
że~w~pierwotnym państwie jest jedzenie który i~świnie jeść by~mogły,
Sokrates powiada, że~to jest państwo jakby zdrowe, musimy więc teraz
rozważyć państwo w~którym jest bardziej dostatnie, wtedy mówi, że~jest
ono jakby w~gorączce. Są~więc obecne metafory \textit{polis} jako czegoś
żywego, ale~nie wiem czy można~się posunąć do stwierdzenia, że~jest
ono wielki człowiekiem, może jednak coś mi umyka. Opowieść o~państwie
jako o~jednym ciele jest jednak znana już w~starożytności, choćby
w~postaci bajki
\href{https://en.wikipedia.org/wiki/Agrippa_Menenius_Lanatus_(consul_503_BC)}
{Menenius Agrypy}.

\vspace{\spaceFour}





\StrWd{144}{} Zdanie „W~\textit{Prawach} Platon odsunął prawdę
duszy na~odległość jej objawienia w~\textit{Państwie}.” jest tak trudne
do~zrozumienia, że~najprawdopodobniej jest to błąd tłumaczenia.





% ##################
\newpage

\CenterBoldFont{Błędy}


\begin{center}

  \begin{tabular}{|c|c|c|c|c|}
    \hline
    & \multicolumn{2}{c|}{} & & \\
    Strona & \multicolumn{2}{c|}{Wiersz} & Jest
                              & Powinno być \\ \cline{2-3}
    & Od góry & Od dołu & & \\
    \hline
    29  & 13 & & warunkowa & warunkową \\
    35  & &  9 & „ Filozof & „Filozof \\
    45  & 10 & & sposob & sposób \\
    64  & 15 & & do do & do \\
    70  &  5 & & ton & Platon \\
    113 &  5 & & pierwszy & drugi \\
    116 & & 11 & w znaczenie & znaczenie \\
    132 & 15 & & wspierać & wspierać się \\
    141 & & 16 & f formami & z~formami \\
    153 & &  3 & wdzięczną & niewdzięczną \\
    166 & & 15 & ładząca & zaprowadzająca ład \\
    % & & & & \\
    \hline
  \end{tabular}

\end{center}

\vspace{\spaceTwo}



% ############################










% ############################
\newpage

\Work{ % Autor i tytuł dzieła
  Eric Voegelin \\
  \textit{Izrael i~Objawienie}, \cite{VoegelinIzraelIObjawienie2014}}


% ##################
\CenterBoldFont{Uwagi do konkretnych stron}


\StrWd{28}{2} Mowa jest tu o~symbolizacja mikroantropicznej,
wydaje~się jednak, że~Voegelinowi chodziło o~symbolizację
makroantropiczną.

\vspace{\spaceFour}





\Str{54} W~przytoczonej tu~inskrypcji jest podane, że~„Enlil
zwrócił oczy kraju [\textit{kalama}] na~siebie”, podczas gdy na dole
strony Voegelin pisze, że~oczy całego kraju Sumerów zostały skierowane
na~Lugalzagesi. Czy jest to błąd tłumaczenie, czy~też Voegelin
pozwolił sobie na~taką interpretację tego fragmentu? Jeśli to drugie,
to~należy zauważyć, iż~ta interpretacja jest dość odległa od~tekstu,
choć nie oznacza to, że~jest niepoprawna. Ja~w~każdym razie bym jej
w~takiej formie nie przyjął.

\vspace{\spaceFour}





\StrWg{57}{2} Jest tu mowa, że~przed stworzeniem „niebiańskiej
ziemi” została stworzona „ziemska ziemia”, jednak kontekst
sugeruje, że~kolejność powinna być odwrotna.

\vspace{\spaceFour}





\Str{127} Na tej stronie pierwszy raz pojawia~się wspomniana
postać~N, ale~jej imię jest na~przemian pisane „N.” albo~„N” i~nie
wiadomo która wersja jest poprawna.

\vspace{\spaceFour}





\Str{132} Cytowany tu tekst o~N przychodzącym z~Wyspy Sławy,
wykazuje duże rozbieżność z~brzmieniem, do~którego odwołuje~się
w~swojej analizie Voegelin. Na przykład w~tekście cytowany jest Wyspa
Sławy, a~Voegelin pisze o~Wyspie Płomieni.

\vspace{\spaceFour}





\Str{147} Na~tej stronie cytowany jest fragment~V omawianego
tekstu, ale nigdzie nie jest chyba podane, jak ten fragment brzmi.

\vspace{\spaceFour}





\Str{180} Ta~strona wprowadziła trochę zamieszania do~mojego
rozumienia tekstu. Przywoływane~są \textit{Hymny do~Amona}, choć wydaje
mi~się do~tej pory w~kontekście egipskim była mowa o~\textit{Hymnach
  do~Atona} (str.~163) i~\textit{Hymnach Echnatona} (str.~171). Poza
tym, są to zapewne dwie różne nazwy tego samego zbioru tekstów.

Do~tego, tekst zdaje~się mówić, że~nowa forma egipskiej religijność,
jakoś związana z~\textit{Hymnami do~Amona}, była monoteistyczna, podczas
gdy na~stronie~173 Voegelin stwierdza, że~nawet stworzenie Atona,
nadal mieściło~się w~obrębie mitu politeistycznego.

\vspace{\spaceFour}





\StrWd{197}{10} Zdanie „Przestaniemy ufać tablicy~III, lecz
odsuniemy ją na~bok” brzmi nienaturalnie, to musi być jakaś pomyłka
tłumacza. Możliwe, że~miało być „Nie~przestaniemy ufać tablicy~III”.

\vspace{\spaceFour}





\StrWd{213}{3} W~wersie tym jest mowa o~autorach badań, lecz
chodzi raczej o~wynik pracy na Pismem~Św. autorów pracujących po
niewoli babilońskiej.

\vspace{\spaceFour}





\StrWd{229}{6} Użycie w~tekście polskim angielskiego słowa
„patchwork” nie pasuje stylu tłumaczenia. Lepiej byłoby znaleźć
polski odpowiednik tego zwrotu.

\vspace{\spaceFour}





\StrWg{241}{12} Rozdział 5~jest wyjątkowo krótki jak na tę
książkę i~nie dzieli się nad podrozdziały, więc odniesienie do
podrozdziału 5.2~jest błędne. Nie umiem jednak ustalić poprawnego
miejsca o~które chodzi Voegelinowi.

\vspace{\spaceFour}





\StrWg{312}{5} Po tej linie w~tekście powinie znajdować~się odstęp.

\vspace{\spaceFour}





\StrWg{370}{12} Napisane tu jest, że~po teopolis doszło do
wycofania~się porządku w~formie kosmologicznej, jednak na podstawie
tego co Voegelin pisał wcześniej logiczniejsza byłoby inne
stwierdzenie. Mianowicie, że~po okresie teopolis wraz z~ustanowieniem
królestwa wkracza do~Izraela porządek w~formie kosmologicznej.





% ##################
\newpage

\CenterBoldFont{Błędy}

\vspace{\spaceFive}


\begin{center}

  \begin{tabular}{|c|c|c|c|c|}
    \hline
    Strona & \multicolumn{2}{c|}{Wiersz} & Jest
                              & Powinno być \\ \cline{2-3}
    & Od góry & Od dołu & & \\
    \hline
    52  & 13 & & 2923 & 2123 \\
    54  & &  3 & Lugalzaggesi & Lugalzagesi \\
    56  & 13 & & przepływ & na~przepływ \\
    98  & &  4 & forma” & „forma” \\
    139 & &  7 & agnostyczną metafizykę & agnostycznej metafizyce \\
    165 & & 11 & sa & są \\
    202 & 12 & & mogła & nie mogła \\
    207 &  8 & & Izraela & izraelskiego \\
    278 & &  6 & siedemnastym & dwudziestym pierwszym \\
    291 & 12 & & 13 & 14 \\
    303 & &  8 & wydarł & wydarłem \\
    314 &  6 & & 11.13 & 11, 13 \\
    316 & &  4 & miał zaś & zaś \\
    340 & &  6 & 175 & 1175 \\
    341 &  1 & & wyjaśnić,przyjmując & wyjaśnić, przyjmując \\
    357 & &  1 & 8.17 & 8, 17 \\
    365 & &  8 & nie będziecie & będziecie \\
    375 & 10 & & [zbójeckie]{ } wyprawy & [zbójeckie] wyprawy \\
    377 & & 14 & zarzadzanie & zarządzanie \\
    % & & & & \\
    % & & & & \\
    % & & & & \\
    % & & & & \\
    \hline
  \end{tabular}

\end{center}

\vspace{\spaceTwo}



% ############################










% ######################################
\newpage
\section{Filozofia analityczna}

\vspace{\spaceTwo}
% ######################################



% ############################
\Work{ % Autorzy i tytuł dzieła
  Gertud Elisabeth M. Anscombe, Peter Thomas Geach \\
  \textit{Trzej filozofowie}, \cite{AnscombeGeachTrzejFilozofowie1981}}


% ##################
\CenterBoldFont{Uwagi do konkretnych stron}


\StrWg{161}{11} W~tej linii brakuje wcięcia akapitu.

\vspace{\spaceFour}





\Str{172--173} Kwestia tego czym jest funkcja wprowadza tu
pewne zamieszanie. Geach krytykuje tu to, że~Frege mówi o~funkcji
$2 \cdot \xi^{ 2 } + \xi$, „$2 \cdot \xi^{ 2 } + \xi$” określa jako
jej nazwę, by~zaraz potem samemu mówić o~„funkcji
$2 \cdot \xi^{ 2 } + \xi$”. Należałoby tą~kwestię w~jakiś sposób
wyjaśnić.

\vspace{\spaceFour}





\Str{174} Wydaje mi~się, że~twierdzenie „$\vdash 2 \cdot x^{ 2 } + x > -1$”, powinno
być poprawnie zapisywane jako „$\vdash( 2 \cdot x^{ 2 } + x > -1)$”.

\vspace{\spaceFour}





\StrWd{174}{17} Zdanie „co~Frege miałby błędnie wprowadzić do~liczebnika”
jest tak trudne do~zrozumienia, że~jest to zapewne błąd tłumaczenia.

\vspace{\spaceFour}





\StrWd{193}{17} Nie jestem pewien, czy~poprawnie powinno tutaj
być „zwierzęciem lądowym”, jak jest obecnie, czy~„zwierzęciem
domowym”, co~jest bardziej spójne z~poprzednimi przykładami.

\vspace{\spaceFour}





% ##################
\newpage

\CenterBoldFont{Błędy}

\vspace{\spaceFive}


\begin{center}

  \begin{tabular}{|c|c|c|c|c|}
    \hline
    Strona & \multicolumn{2}{c|}{Wiersz} & Jest
                              & Powinno być \\ \cline{2-3}
    & Od góry & Od dołu & & \\
    \hline
    15  & & 11 & jak~X & jak~x \\
    161 & & 12 & $\vdash$({ } jeżeli & $\vdash$(jeżeli \\
    161 & & 11 & $\vdash$ (nie $m$) & $\vdash$(nie $m$) \\
    161 & & 11 & $\vdash$ ($m${ } lub $p$) & $\vdash$($m$ lub $p$) \\
    161 & & 10 & $\vdash$ (jeżeli & $\vdash$(jeżeli \\
    161 & &  9 & „$\vdash$ ($m$ lub $p$”)
           & „$\vdash$($m$ lub $p$)” \\
    162 &  1 & & $\vdash$ (nie & $\vdash$(nie \\
    162 &  2 & & $\vdash$ (jeżeli & $\vdash$(jeżeli \\
    162 &  3 & & $\vdash$ (nie & $\vdash$(nie \\
    162 & 15 & & \textit{znaczenie} * & \textit{znaczenie}* \\
    162 & & 12 & „$\vdash m$” & „nie ($\vdash m)$” \\
    162 & &  9 & „$\vdash$ (nie $m$)” & „$\vdash$(nie $m$)” \\
    163 & & 15 & Przed Fregem & U~Fregego \\
    169 &  1 & & obowiązywałby & nie obowiązywałby \\
    169 & 15 & & \textbf{(p}sem & (psem \\
    172 & & 15 & „$2 \cdot 2^{ 2 } \, +$”
           & „$2 \cdot ( \:\: )^{ 2 } + ( \:\: )$” \\
    172 & &  5 & $2 \cdot 7^{ 2 } + 7$” & „$2 \cdot 7^{ 2 } + 7$” \\
    177 & & 13 & $\bigcirc$ & $0$ \\
    179 & 21 & & $\textrm{M}_{ \beta }\, \varphi( \beta )$”
           & „$\textrm{M}_{ \beta }\, \varphi( \beta )$” \\
    180 &  5 & & $=\!\! X \hat{ \;\, } \alpha'$
           & $= x \hat{ \;\, } \alpha'$ \\
    183 &  1 & & (Zdania & Zdania \\
    184 & 19 & & „$\vdash$ $( 2^{ 4 } = 4^{ 2 })$”
           & „$\vdash$$( 2^{ 4 } = 4^{ 2 })$” \\
    184 & & 13 & „$2^{ 4 } = 4^{2}$).” & „$2^{ 4 } = 4^{2}$”. \\
    191 &  6 & & \ldots” & \ldots”. \\
    \hline
  \end{tabular}

\end{center}

\vspace{\spaceTwo}



% ############################










% ######################################
\newpage
\section{Historia myśli i~idei}

\vspace{\spaceTwo}
% ######################################



% ############################
\Work{ % Autor i tytuł dzieła
  R\'{e}mi Brague \\
  \textit{Mądrość świata. Historia ludzkiego doświadczenia Wszechświata},
  \cite{BragueMadroscSwiata2021}}


% ##################
\CenterBoldFont{Uwagi do konkretnych stron}


\Str{3} Tytuł książki „Mądrość świata” nie wydrukowała~się
w~pełni dobrze. Czcionka jest trochę rozmyta.

\vspace{\spaceFour}





\noindent
Omówienie każde filozofa powinno zawierać listę dzieł
polecanych do przeczytania, aby~czytelnik wiedział, gdzie najlepiej
zetknąć~się z~tym wszystkim co~zawiera~się w~sposobie filozofowania
danego myśliciela. Z~rzeczami takimi jak konkretne sposoby dowodzenia,
używane argumenty, wplatane anegdoty, żarty, etc.





% ##################
\CenterBoldFont{Błędy}

\vspace{\spaceTwo}


\noindent
\textbf{Grzbiet} \\
\Jest  R \'{e}mi {\small Brague} \\
\Powin R\'{e}mi Brague \\
\textbf{Tylna okładka} \\
\Jest  R\'{e}mi {\small Brague} \\
\Powin R\'{e}mi Brague \\


% ############################










% % ######################################
% \newpage
% \section{Filozofia i~naturalny świat}

% \vspace{\spaceTwo}
% % ######################################



% % ############################
% \Work{ % Autor i tytuł dzieła
%   Roger Scruton \\
%   „Zielona filozofia. Jak poważnie myśleć o naszej planecie”,
%   \cite{ScrutonZielonaFilozofia2017} }


% % ##################
% \CenterBoldFont{Uwagi do konkretnych stron}


% \start \StrWg{XV}{15} W~tej linii brakuje numeru przypisu. Numerem tym
% oczywiści powinno być „1”.

% \vspace{\spaceFour}





% % ##################
% \CenterBoldFont{Błędy}


% \begin{center}

%   \begin{tabular}{|c|c|c|c|c|}
%     \hline
%     & \multicolumn{2}{c|}{} & & \\
%     Strona & \multicolumn{2}{c|}{Wiersz} & Jest
%                               & Powinno być \\ \cline{2-3}
%     & Od góry & Od dołu & & \\
%     \hline
%     14 & &  9 & \textit{i Agricultural} & \textit{and Agricultural} \\
%     18 & &  1 & tłum.]. & tłum.]). \\
%     %     & & & & \\
%     %     & & & & \\
%     %     & & & & \\
%     %     & & & & \\
%     %     & & & & \\
%     \hline
%   \end{tabular}

% \end{center}


% \vspace{\spaceTwo}
% % ############################










% ######################################
\newpage
\section{Filozofia, polityka i~historia}

\vspace{\spaceTwo}
% ######################################



% ############################
\Work{ % Autor i tytuł dzieła
  Samuel Taylor Coleridge \\
  \textit{O~konstytucji Kościoła i~państwa. Wybór pism}, \cite{}}


% ##################
\CenterBoldFont{Uwagi do konkretnych stron}


\StrWg{XV}{15} W~tej linii brakuje numeru przypisu. Numerem tym
oczywiści powinno być „1”.

\vspace{\spaceFour}





% % ##################
% \newpage
% \CenterBoldFont{Błędy}


% \begin{center}

%   \begin{tabular}{|c|c|c|c|c|}
%     \hline
%     & \multicolumn{2}{c|}{} & & \\
%     Strona & \multicolumn{2}{c|}{Wiersz} & Jest
%                               & Powinno być \\ \cline{2-3}
%     & Od góry & Od dołu & & \\
%     \hline
%     %     & & & & \\
%     %     & & & & \\
%     %     & & & & \\
%     %     & & & & \\
%     %     & & & & \\
%     %     & & & & \\
%     %     & & & & \\
%     \hline
%   \end{tabular}

% \end{center}


\vspace{\spaceTwo}
% ############################










% ############################
\Work{ % Autor i tytuł dzieła
  Christian Meier \\
  \textit{Powstanie polityczności u~Greków},
  \cite{MeierPowstaniePolitycznosciUGrekow2012}}

\vspace{0em}


% ##################
\CenterBoldFont{Uwagi do konkretnych stron}

\vspace{0em}


\Str{39} Na tej stronie brakuje odstępu między tekstem głównym,
a~przypisami.

\vspace{\spaceFour}





% ##################
\CenterBoldFont{Błędy}

\vspace{\spaceFive}


\begin{center}

  \begin{tabular}{|c|c|c|c|c|}
    \hline
    & \multicolumn{2}{c|}{} & & \\
    Strona & \multicolumn{2}{c|}{Wiersz} & Jest
                              & Powinno być \\ \cline{2-3}
    & Od góry & Od dołu & & \\
    \hline
    21  & &  2 & \textit{pism}a & \textit{pisma} \\
    40  & &  8 & stało & stało~się \\
    43  & &  7 & inrensywne & intensywne \\
    73  & &  4 & and & \textit{and} \\
    76  &  9 & & Odpowiadał~on & Odpowiadała~ona \\
    80  & &  2 & \textit{Groeth} & \textit{Growth} \\
    % 87???
    % & & & & \\
    % & & & & \\
    % & & & & \\
    % & & & & \\
    % & & & & \\
    % & & & & \\
    % & & & & \\
    % & & & & \\
    % & & & & \\
    \hline
  \end{tabular}

\end{center}

\vspace{\spaceTwo}

% ############################










% ######################################
\newpage

\section{Filozofia i~nauki szczegółowe}

\vspace{\spaceTwo}
% ######################################





% ######################################
\subsection{Filozofia, teologia i~społeczeństwo}

\vspace{\spaceTwo}
% ######################################



% ############################
\Work{ % Autor i tytuł dzieła
  John Milbank \\
  \textit{Przekroczyć rozum sekularny. Teologia i~teoria społeczna},
  \cite{MilbankPrzekroczycRozumSekularny2020}}


% ##################
\CenterBoldFont{Uwagi do konkretnych stron}


\Str{1 i~3} Marginesy tych stron są za małe, tak że tytuł książki
w~sposób nieprzyjemny zahacza o~brzeg strony.

\vspace{\spaceFour}





\StrWg{63}{4} Czy w~tej linii zamiast słowa „response”, nie powinno
być „reakcje”?

\vspace{\spaceFour}



% \start \StrWd{114}{15}

% \vspace{\spaceFour}



% \start \StrWg{116}{10--11}

% \vspace{\spaceFour}





% ##################
\newpage

\CenterBoldFont{Błędy}

\vspace{\spaceFive}


\begin{center}

  \begin{tabular}{|c|c|c|c|c|}
    \hline
    & \multicolumn{2}{c|}{} & & \\
    Strona & \multicolumn{2}{c|}{Wiersz} & Jest
                              & Powinno być \\ \cline{2-3}
    & Od góry & Od dołu & & \\
    \hline
    23  &  1 & & O’Regana)$^{ 3 }$ & O’Regana$^{ 3 }$) \\
    25  & &  4 & \textit{theory} & \textit{Theory} \\
    32  &  8 & & o~nieokreślonych & nieokreślonych \\
    40  &  2 & & \textit{moralności} ) & \textit{moralności}) \\
    62  & &  1 & \textit{Cambridge 1996} & Cambridge 1996  \\
    %  & & &  &  \\
    %  & &  &  &  \\
    %  & &  &  &  \\
    %  & & &  &  \\
    %  & &  &  &  \\
    \hline
  \end{tabular}

\end{center}

\vspace{\spaceTwo}



% ############################










% ######################################
\newpage
\section{Analizy filozofii i~kultur}

\vspace{\spaceTwo}
% ######################################



% ############################
\Work{ % Autor i tytuł dzieła
  Reinhart Koselleck \\
  \textit{Krytyka i~kryzys. Studium patogenezy świata mieszczańskiego},
  \cite{KoselleckKrytykaIKryzys2015}}


% ##################
\CenterBoldFont{Uwagi do konkretnych stron}


\StrWg{5}{4} W~tej linii brakuje liczby 39, oznaczającej stronę
na~której rozpoczyna~się wstęp.

\vspace{\spaceFour}





\Str{39} Jest tu mowa o~dwóch mocarstwach światowych, Ameryce
i~Rosji, lecz~użycie w~tym kontekście tej drugiej nazwy jest
problematyczne. O~ile Ameryka to pojęcie używane zamiennie z~Stanami
Zjednoczonymi, to~Rosja i~Związek Sowiecki nie~są ze sobą tożsame.

\vspace{\spaceFour}





\StrWd{114}{15} Książka Kosellecka została pierwszy raz wydana
w~1959, czyli zbyt wcześnie by~mogła cytować wydawnictwo z~1993 roku.
Możliwe jednak, że~jest to data publikacji zbioru w~którym praca Tarna
o~Aleksandrze Macedońskim ukazała~się w~kolejnym przedruku i~akurat
ten odnośnik do~tego artykułu został wybrany w~którymś z~kolejnych
wydań. Jednak nie jestem w~stanie rozstrzygnąć tego z~całą pewnością.

\vspace{\spaceFour}





\StrWg{116}{10--11} Podzielenie „natury?” na~te dwie linie jest nie tylko
niepoprawne, ale~również bardzo źle wygląda.

\vspace{\spaceFour}





% ##################
\CenterBoldFont{Błędy}

\vspace{\spaceFive}


\begin{center}

  \begin{tabular}{|c|c|c|c|c|}
    \hline
    & \multicolumn{2}{c|}{} & & \\
    Strona & \multicolumn{2}{c|}{Wiersz} & Jest
                              & Powinno być \\ \cline{2-3}
    & Od góry & Od dołu & & \\
    \hline
    65  & &  5 & dalekowzrocznego działania & działania dalekowzrocznego \\
    71  & &  4 & and & \textit{and} \\
    110 & & 14 & jdnym & jednym \\
    115 &  4 & & występują & występując \\
    115 &  5 & & uznając & uznają \\
    137 & 15 & & więc na & więc \\
    184 & & 11 & Neuf & Les Neuf \\
    233 & &  5 & x$\rho\iota\tau\iota\chi\acute{\eta}$ & $\chi\rho\iota\tau\iota\chi\acute{\eta}$ \\
    260 & & 11 & z~której & w~której \\
    331 & & 12 & wladcy & władcy \\
    \hline
  \end{tabular}

\end{center}

\vspace{\spaceTwo}



% ############################










% ############################
\Work{ % Autor i tytuł dzieła
  Nakamura Hajime \\
  \textit{Systemy myślenia ludów Wschodu} \\
  \textit{Indie, Chiny, Tybet, Japonia},
  \cite{NakamuraSystemyMysleniaLudowWschodu2005}}

\vspace{0em}


% ##################
\CenterBoldFont{Uwagi do konkretnych stron}

\vspace{0em}


\Str{23} Nakamura powinien tu jawnie napisać, co rozumiem przez
„sposoby myślenia” i~„systemy myślenia”.

\vspace{\spaceFour}





\StrWd{90}{13} W~polskim tłumaczeniu użyto zwrotu „relacja
wyróżniającego~się i~wiedzącego”, który jest dziwny i~trudny
do~zrozumienia.

\vspace{\spaceFour}





\StrWd{101}{20} Ten fragment pozostawia niejasnym,
czy~działanie kogoś innego jest manifestowane, jako przedłużenie
czynności własnego „ja” u~ludzi Zachodu czy~u~Indusów.

\vspace{\spaceFour}





\StrWg{116}{6} Po~tej linii nie powinno być pionowego odstępu w~tekście.

\vspace{\spaceFour}





\StrWg{559}{6} W~tej linii nie~powinno być wcięcia.





% ##################
\CenterBoldFont{Błędy}

\vspace{\spaceFive}


\begin{center}

  \begin{tabular}{|c|c|c|c|c|}
    \hline
    Strona & \multicolumn{2}{c|}{Wiersz} & Jest
                              & Powinno być \\ \cline{2-3}
    & Od góry & Od dołu & & \\
    \hline
    69  & &  2 & jedynie rzeczownikowi & rzeczownikowi \\
    106 & 14 & & można & nie można \\
    % & & & & \\
    % & & & & \\
    % & & & & \\
    % & & & & \\
    559 & 20 & & naukowego]. & naukowego].] \\
    \hline
  \end{tabular}

\end{center}


\vspace{\spaceTwo}
% ############################










% ######################################
\newpage
\section{Historia filozofii}

\vspace{\spaceTwo}
% ######################################



% ############################
\Work{ % Autor i tytuł dzieła
  Frederick Copleston S. J. \\
  \textit{Historia filozofii. Tom~I: Grecja i~Rzym},
  \cite{CoplestonHistoriaFilozofiiTomI2004}}


% ##################
\CenterBoldFont{Uwagi}


\noindent
Omówienie każdego filozofa powinno zawierać listę dzieł
polecanych do przeczytania, aby~czytelnik wiedział, gdzie najlepiej
zetknąć~się z~tym wszystkim co~zawiera~się w~sposobie filozofowania
danego myśliciela. Z~rzeczami takimi jak konkretne sposoby dowodzenia,
używane argumenty, wplatane anegdoty, żarty, etc.

\vspace{\spaceThree}





% ##################
\CenterBoldFont{Uwagi do konkretnych stron}


\Str{32} W~polskiej tłumaczeniu Diogenesa Laertios, opisuje
styl Anaksymenesa jako prosty i~niewyszukany, zaś w~wersji angielskiej
jako „pure unmixed”, co należałoby by przetłumaczyć raczej jako
„czysty i~pozbawiony obcych naleciałości”, jednak to tłumaczenie
również jest niesatysfakcjonujące.

\vspace{\spaceFour}





\Str{33} Według tego co tu napisano, Anaksymanes uważał,
że~istnieje jedne pierścień okalający, zawierający i~ogień i~zimno,
wewnątrz niego zaś~jest powietrze. Nie potrafię sobie wyobrazić, jak
taki pierścień miałby wyglądać.

\vspace{\spaceFour}





\Str{33} Należałoby pójść śladem wydania angielskiego i~ostatni
akapit tej strony, który rozpoczyna podsumowanie filozofii jońskiej,
oddzielić graficznie od omówienia myśli Anaksymenesa.





% ##################
\CenterBoldFont{Błędy}


\begin{center}

  \begin{tabular}{|c|c|c|c|c|}
    \hline
    & \multicolumn{2}{c|}{} & & \\
    Strona & \multicolumn{2}{c|}{Wiersz} & Jest
                              & Powinno być \\ \cline{2-3}
    & Od góry & Od dołu & & \\
    \hline
    16  & &  3 & poglądów & spojrzenia na świat \\
    32  & 16 & & \ldots Mówi & \ldots mówi \\
    37  & & 18 & zgodę & niezgodę \\
    38  &  5 & & zwykłą & tylko \\
    % & & & & \\
    % & & & & \\
    61 & & 12 & do~Schelling & to~Schelling \\
    82 &  8 & & do~kraju & w~kraju \\
    % & & & & \\
    % & & & & \\
    \hline
  \end{tabular}

\end{center}


\noindent
\StrWg{11}{17} \\
\Jest  że~podmiotu w~nie większym stopniu nie~można sprowadzić
do~przedmiotu niż~przedmiotu do~podmiotu \\
\Powin że~podmiot nie może być w~większym stopniu sprowadzony
do~przedmiotu niż~przedmiot do~podmiotu \\
\StrWg{32}{3} \\
\Jest  Każdy jest zniszczalny, jednakże, jak~się wydaje, istnieje ich
nieograniczone liczba w~tym samy czasie, światów zaczynających istnieć
dzięki wiecznemu ruchowi. \\
\Powin Każdy jest zniszczalny, jednakże, wydaje~się, że~nieskończona ich
liczba istnieje w~tej samej chwili, światów powstających dzięki
wiecznemu ruchowi. \\
\StrWg{32}{25} \\
\Jest  poprzedzając koncepcję \\
\Powin dochodząc do koncepcji \\
\StrWd{61}{9} \\
\Jest  Nie~mógł raczej być utożsamiany z~Jednym, nie~mogło~się też
zdarzyć by~ktoś to czynił zbyt dosłownie. \\
\Powin Nie może być utożsamiony z~Jednym, nie mogło~się też zdarzyć, by
ktoś robił to dosłownie.


\vspace{\spaceTwo}
% ############################










% ######################################
\newpage
\section{Filozofia nauki i~filozofujący naukowcy}

\vspace{\spaceTwo}
% ######################################



% ############################
\Work{ % Autor i tytuł dzieła
  Roger Penrose \\
  \textit{Moda, wiara i~fantazja w~nowej fizyce Wszechświata},
  \cite{PenroseModaWiaraIFantazja2017}}

\vspace{0em}

% ##################
\CenterBoldFont{Uwagi do konkretnych stron}

\vspace{0em}


\StrWd{72}{3--2} Ponieważ kwantowa teoria pola zajmuje~się
głównie procesami relatywistycznymi, naturalne jest przyjęcie, że~ruch
wszystkich rozważanych cząstek~są relatywistyczne. W~takim wypadku nie
powinno~się upraszczać sprawy do~tego stopnia, by~przyjmować, iż~pęd
to~prędkość cząstki razy jej masa. Rozumiem, że~zapewne Penrose chciał
uprościć tekst na~potrzebę niespecjalistów, ale~według mnie poszedł
za~daleko.

\vspace{\spaceFour}





\Str{78} Należy według mnie zatrzymać~się chwilę nad
stwierdzeniem, że~wartość masy „ubranej” cząstki\footnote{Dla
  prostoty mówię tu o~masie nie zaś o~ładunku, ale~dla niego
  powinno~się dać przeprowadzić analogiczne rozumowanie.}
otrzymuje~się doświadczeń, nie zaś z~teorii. Jeśli weźmiemy teorię
masywnego pola swobodnego, która nie wymaga renormalizacji, to i~tak
zawiera ona parametr masy, który można wyznaczyć tylko z~obserwacji.
Problem teorii kryje~się więc gdzieś indziej.

To~na co liczylibyśmy, to~że znając masę gołej cząstki, jesteśmy
w~stanie policzyć masę cząstki „ubranej”, co okazuje~się nieprawdą,
musimy więc wziąć~ją z~danych obserwacyjnych. Właśnie porażka w~tym
miejscu teorii wywołuje spory niesmak. Drugim powodem do~tego niesmaku
jest to, że~należy w~ten sposób zastępować wyrażenia nieskończone,
co~jest zdecydowanie bardzo „uwłaczającą” czynnością.

\vspace{\spaceFour}





\StrWd{659}{5} Na~podstawie bibliografii nie jestem w~stanie
zidentyfikować pozycji kryjącej~się za~akronimem DDR, która
wielokrotnie jest przywoływana w~tej książce.

\vspace{\spaceFour}





\Str{661} Może to być już przesadny pedantyzm, ale~nie podoba
mi~się stwierdzenie, że~aby otrzymać wykres funkcji odwrotnej
wystarczy zamienić miejscami osie. Ryzyko niezrozumienia jest według
mnie zbyt duże. To co naprawdę trzeba zrobić, to obrócić \textit{cały}
wykres w~taki sposób, by osie zamieniły~się miejscami.

\vspace{\spaceFour}





\Str{673} Zamiast liczb całkowitych~$\Zbb$ należałoby całą
dyskusję oprzeć na~liczbach naturalnych~$\Zbb$. Uczyniłoby to~między
innymi Rys.~A-3 bardziej zrozumiałym.

\vspace{\spaceFour}





\StrWd{683}{10} Nie powinno~się łamać wzorów matematycznych
na~końcu linii w~taki sposób, jak tu zostało złamane wyrażenie
$\langle \vecubold | \vecvbold \rangle$.

\vspace{\spaceFour}





\Str{718} Rysunek A-26 (b) jest narysowany błędnie. Po~lewej stronie punktu
zgięcia wstęgi M\"{o}biusa, który wypada na~poziomie cięcia tożsamościowo
równego zeru, widać dwa ciąg
przerywanych linii, symbolizujące cięcie. Po~pierwsze nie może być
dwóch takich przerywanych linii, bo rysunek zawiera tylko jedno
cięcie. Po~drugie, cięcia te opuszczają wstęgę M\"{o}biusa, co jest
niemożliwe, bowiem gładkość cięcia tego zabrania, by~powstał taki
skok. Po~trzecie, po~lewej stronie punktu zagięcia, wykres cięcia
znajduje~się na niewidocznej stronie wstęgi, przysłoniętej przez jej
zagięty fragment.

\vspace{\spaceFour}





\Str{725} Nie rozumiem, jaka jest logika Penrosa, by~w~tym
miejscu powoływać~się na~wzór~$e^{ i \theta }$, skoro w~dodatku~A.9
na~stronie~735 będzie tłumaczył zupełnie od~zera co~to są liczby
zespolone. Skutkiem tego, osoba, która nie wie czym są liczby
zespolone musiałaby czytać, dodatki nie po~kolei. Skoro jednak taka
osoba nie wie, czym~są liczby zespolone, zapewne widząc ten wzór nie
będzie wiedziała, gdzie szukać jego wyjaśnienia.

Wystarczyłoby dodać na~końcu tego tekstu uwagę „Po~informacje o~tym
wzorze zob.~A.9”\footnote{To~może nie jest najlepsze sformułowanie,
  ale~nie wiem jak to~zrobić lepiej.}, ale~niczego takiego tu nie ma.

\vspace{\spaceFour}





\Str{747} Pojęcie \textbf{funkcji holomorficznej} zdaje~się tu
pojawiać pierwszy raz, w~skutek czego jest zupełnie niezrozumiałe
dla~osoby niezorientowanej w~analizie zespolonej.





% ##################
\CenterBoldFont{Błędy}

\vspace{\spaceFive}


\begin{center}

  \begin{tabular}{|c|c|c|c|c|}
    \hline
    Strona & \multicolumn{2}{c|}{Wiersz} & Jest
                              & Powinno być \\ \cline{2-3}
    & Od góry & Od dołu & & \\
    \hline
    4   &  5 & & \textit{in~the~Universe} & \textit{of~the~Universe} \\
    107 & &  7 & mieszczącej~się & mieszczących~się \\
    % & & & & \\
    % & & & & \\
    684 &  1 & & $+\, a \langle \vecubold | \vecvbold \rangle$
           & $= a \langle \vecubold | \vecvbold \rangle$ \\
    686 &  6 & & $\ldots,$ & $\ldots$ \\
    688 & &  8 & $i \neq j$ & $i \neq j$, \\
    688 & &  3 & $\ldots,$ & $\ldots$ \\
    689 & &  4 & $( \vecrhobold_{ 1 },\; \ldots, \vecrhobold_{ n } )$
           & $( \vecrhobold_{ 1 }, \ldots, \vecrhobold_{ n } )$ \\
    711 & & 10 & A-15(c) & A-15(b) \\
    731 &  6 & & płaszczyzny & z~płaszczyzny \\
    732 &  2 & & $\infty^{ k \infty }$ & $\infty^{ 2 k \infty }$ \\
    733 &  4 & & $\infty^{ k \infty }$ & $\infty^{ 2 k \infty }$ \\
    733 &  7 & & $s = 1$ & $s = 2$ \\
    733 &  7 & & $s = k$ & $s = 2k$ \\
    733 & 11 & & $\Rbb^{ f + k }$ & $\Rbb^{ f + s }$ \\
    740 &  7 & & $( z - b_{ 2 } )( z - b_{ n } )$
           & $( z - b_{ 2 } ) \cdot \ldots \cdot ( z - b_{ n } )$ \\
           % & & & & \\
           % & & & & \\
           % & & & & \\
    \hline
  \end{tabular}

\end{center}

\vspace{\spaceTwo}


\StrWg{668}{7} \\
\Jest $X_{ i } = x_{ i } + A_{ i } ( i = 1, 2, \ldots, n )$ \\
\Powin $X_{ i } = x_{ i } + A_{ i }, \quad i = 1, 2, \ldots, n.$ \\
\StrWd{703}{11} \\[0.2em]
\Jest $\sqrt{ ( X - x )^{ 2 } + ( Y - y )^{ 2 } + ( Z - z )^{ 2 } .) }$
\\[0.4em]
\Powin $\sqrt{ ( X - x )^{ 2 } + ( Y - y )^{ 2 } + ( Z - z )^{ 2 } }.)$



% ############################










% #####################################################################
% #####################################################################
% Bibliografia

\bibliographystyle{plalpha}

\bibliography{DEUSPhilBooks}{}





% ############################

% Koniec dokumentu
\end{document}
