% Autor: Kamil Ziemian

% ---------------------------------------------------------------------
% Podstawowe ustawienia i pakiety
% ---------------------------------------------------------------------
\RequirePackage[l2tabu, orthodox]{nag}  % Wykrywa przestarzałe i niewłaściwe
% sposoby używania LaTeXa. Więcej jest w l2tabu English version.
\documentclass[a4paper,11pt]{article}
% {rozmiar papieru, rozmiar fontu}[klasa dokumentu]
\usepackage[MeX]{polski}  % Polonizacja LaTeXa, bez niej będzie pracował
% w języku angielskim.
\usepackage[utf8]{inputenc}  % Włączenie kodowania UTF-8, co daje dostęp
% do polskich znaków.
\usepackage{lmodern}  % Wprowadza fonty Latin Modern.
\usepackage[T1]{fontenc}  % Potrzebne do używania fontów Latin Modern.



% ------------------------------
% Podstawowe pakiety (niezwiązane z ustawieniami języka)
% ------------------------------
\usepackage{microtype}  % Twierdzi, że poprawi rozmiar odstępów w tekście.
% \usepackage{graphicx}  % Wprowadza bardzo potrzebne komendy do wstawiania
% % grafiki.
\usepackage{vmargin}  % Pozwala na prostą kontrolę rozmiaru marginesów,
% za pomocą komend poniżej. Rozmiar odstępów jest mierzony w calach.
% ------------------------------
% MARGINS
% ------------------------------
\setmarginsrb
{ 0.7in} % left margin
{ 0.6in} % top margin
{ 0.7in} % right margin
{ 0.8in} % bottom margin
{  20pt} % head height
{0.25in} % head sep
{   9pt} % foot height
{ 0.3in} % foot sep



% ------------------------------
% Często używane pakiety
% ------------------------------
\usepackage{xcolor}  % Pozwala używać kolorowych czcionek (zapewne dużo
% więcej, ale ja nie potrafię nic o tym powiedzieć).



% ------------------------------
% Pakiety których pliki *.sty mają być w tym samym katalogu co ten plik
% ------------------------------
\usepackage{latexgeneralcommands}



% ---------------------------------------------------------------------
% Dodatkowe ustawienia dla języka polskiego
% ---------------------------------------------------------------------
\renewcommand{\thesection}{\arabic{section}.}
% Kropki po numerach rozdziału (polski zwyczaj topograficzny)
\renewcommand{\thesubsection}{\thesection\arabic{subsection}}
% Brak kropki po numerach podrozdziału



% ------------------------------
% Ustawienia różnych parametrów tekstu
% ------------------------------
\renewcommand{\arraystretch}{1.2}  % Ustawienie szerokości odstępów między
% wierszami w tabelach.





% ------------------------------
% Pakiet „hyperref”
% Polecano by umieszczać go na końcu preambuły
% ------------------------------
\usepackage{hyperref}  % Pozwala tworzyć hiperlinki i zamienia odwołania
% do bibliografii na hiperlinki.










% ---------------------------------------------------------------------
% Tytuł, autor, data
\title{Historia Polski~-- błędy i~uwagi}

% \author{}
% \date{}
% ---------------------------------------------------------------------










% ####################################################################
% Początek dokumentu
\begin{document}
% ####################################################################





% ######################################
\maketitle  % Tytuł całego tekstu
% ######################################





% ######################################
\section{Syntezy historii Polski}

\vspace{\spaceTwo}
% ######################################



% ############################
\Work{ % Autor i tytuł dzieła
  Norman Davies \\
  „Boże igrzysko. Historia Polski”, \cite{DaviesBozeIgrzysko2010} }


% ##################
\CenterBoldFont{Uwagi do konkretnych stron}


\start \StrWg{4}{20} Treść tego wersu „[uzupełnić na~ozalidzie]”,
to~musi być wynik naprawdę okropnego błędu edytorskiego.

% \vspace{\spaceFour}





% ##################
\CenterBoldFont{Błędy}


\begin{center}

  \begin{tabular}{|c|c|c|c|c|}
    \hline
    & \multicolumn{2}{c|}{} & & \\
    Strona & \multicolumn{2}{c|}{Wiersz} & Jest
                              & Powinno być \\ \cline{2-3}
    & Od góry & Od dołu & & \\
    \hline
    4   &  4 & & Volume~I & Volume~II \\
    % & & & & \\
    % & & & & \\
    % & & & & \\
    % & & & & \\
    % & & & & \\
    % & & & & \\
    % & & & & \\
    % & & & & \\
    % & & & & \\
    % & & & & \\
    % & & & & \\
    \hline
  \end{tabular}





  % \begin{tabular}{|c|c|c|c|c|}
  %   \hline
  %   & \multicolumn{2}{c|}{} & & \\
  %   Strona & \multicolumn{2}{c|}{Wiersz} & Jest
  %   & Powinno być \\ \cline{2-3}
  %   & Od góry & Od dołu & & \\
  %   \hline
  %   %   & & & & \\
  %   %   & & & & \\
  %   %   & & & & \\
  %   %   & & & & \\
  %   %   & & & & \\
  %   %   & & & & \\
  %   %   & & & & \\
  %   %   & & & & \\
  %   %   & & & & \\
  %   %   & & & & \\
  %   %   & & & & \\
  %   %   & & & & \\
  %   %   & & & & \\
  %   %   & & & & \\
  %   %   & & & & \\
  %   %   & & & & \\
  %   %   & & & & \\
  %   %   & & & & \\
  %   %   & & & & \\
  %   %   & & & & \\
  %   %   & & & & \\
  %   %   & & & & \\
  %   %   & & & & \\
  %   %   & & & & \\
  %   %   & & & & \\
  %   %   & & & & \\
  %   %   & & & & \\
  %   %   & & & & \\
  %   %   & & & & \\
  %   %   & & & & \\
  %   %   & & & & \\
  %   %   & & & & \\
  %   %   & & & & \\
  %   %   & & & & \\
  %   %   & & & & \\
  %   %   & & & & \\
  %   %   & & & & \\
  %   %   & & & & \\
  %   \hline
  % \end{tabular}





  % \begin{tabular}{|c|c|c|c|c|}
  %   \hline
  %   & \multicolumn{2}{c|}{} & & \\
  %   Strona & \multicolumn{2}{c|}{Wiersz} & Jest
  %   & Powinno być \\ \cline{2-3}
  %   & Od góry & Od dołu & & \\
  %   \hline
  %   %   & & & & \\
  %   %   & & & & \\
  %   %   & & & & \\
  %   %   & & & & \\
  %   %   & & & & \\
  %   %   & & & & \\
  %   %   & & & & \\
  %   %   & & & & \\
  %   %   & & & & \\
  %   %   & & & & \\
  %   %   & & & & \\
  %   %   & & & & \\
  %   %   & & & & \\
  %   %   & & & & \\
  %   %   & & & & \\
  %   %   & & & & \\
  %   %   & & & & \\
  %   %   & & & & \\
  %   %   & & & & \\
  %   %   & & & & \\
  %   %   & & & & \\
  %   %   & & & & \\
  %   %   & & & & \\
  %   %   & & & & \\
  %   %   & & & & \\
  %   %   & & & & \\
  %   %   & & & & \\
  %   %   & & & & \\
  %   %   & & & & \\
  %   %   & & & & \\
  %   %   & & & & \\
  %   %   & & & & \\
  %   %   & & & & \\
  %   %   & & & & \\
  %   %   & & & & \\
  %   %   & & & & \\
  %   %   & & & & \\
  %   %   & & & & \\
  %   \hline
  % \end{tabular}

\end{center}


\vspace{\spaceTwo}
% ############################










% ############################
\Work{ % Tytuł i autor dzieła
  Norman Davies \\
  „Serce Europy”, \cite{DaviesSerceEuropy2014} }


% ##################
\CenterBoldFont{Uwagi do konkretnych stron}


\start \Str{16} Davies pisze tu o~tym, że~rzut trójwymiarowej Ziemi
na~dwuwymiarową mapę musi pociągać za sobą zniekształcenia
odwzorowywanego terenu, jednak mnie~się wydaje, że~źle zrozumiał,
gdzie leży problem. Powierzchnię Ziemi można uważać za dwuwymiarową,
tak jak mapę i~to raczej nie jest problemem w~kartografii.

Jednocześnie wiadomo na mapie albo kąty albo kształt lądów nie mogą
być wiernie oddane. Jest to jednak związane z~czymś innym niż wymiar,
mianowicie z~Theorema Egregium Gaussa. Jako szczególny przypadek
wynika z~niego, że~takie zniekształcenia przy przekształcaniu
dwuwymiarowej powierzchni w~inną dwuwymiarową powierzchnię muszą się
pojawić, jeśli mają one różne krzywizny Gaussa, dla~płaszczyzny zaś
jest ona równa 0, a~dla sfery $\frac{ 1 }{ r^{ 2 } }$.

\vspace{\spaceFour}



\start \StrWg{17}{12} W~tym miejscu Davies stosuje często spotykany
anachronizm pisząc o~Watykanie jako metonimie Ojca Świętego
i~najwyższych władz Kościoła. Do~czasu zajęcia przez wojska Victora
Emanuela~II Państwa Kościelnego w~1870~roku, władza doczesna papieża
nie ograniczała~się tylko do~tej dzielnicy Rzymu, lecz przez wieki
dotyczyła ogromnych połaci Półwyspu Apenińskiego. Dlatego trafniej
w~kontekście 1765~r. byłoby pisać o~decyzji Rzymu.

\vspace{\spaceFour}



\start \Str{28} Wydaje mi~się, że~nazwanie wszystkich wymienionych
tu~przez Davies państw dyktaturami, jest błędne. Państwo Watykańskie
i~Tybet Dalajlamy nie są demokracjami, ale~to za~mało, aby uznać je
za~dyktatury, bo~dyktatura jest pojęciem czysto negatywnym
i~piętnującym.

\vspace{\spaceFour}



\start \StrWg{52}{10} Zostawienie w~tym wersie słowa „Batman”
pisanego z~dużej litery, jest chyba błędem tłumacza. Sugeruje to,
że~chodzi o~Człowieka-Nietoperza, jednanego z~najbardziej
znanych superbohaterów amerykańskich, jednak jest bardziej
prawdopodobne, że~angielskie słowo „batman” występuje to~w~znaczeniu
„ordynans”.





% % ##################
% \CenterBoldFont{Błędy}


% \begin{center}

%   \begin{tabular}{|c|c|c|c|c|}
%     \hline
%     & \multicolumn{2}{c|}{} & & \\
%     Strona & \multicolumn{2}{c|}{Wiersz} & Jest
%                               & Powinno być \\ \cline{2-3}
%     & Od góry & Od dołu & & \\
%     \hline
%     & & & & \\
%     & & & & \\
%     & & & & \\
%     & & & & \\
%     \hline
%   \end{tabular}

% \end{center}


\vspace{\spaceTwo}
% ############################










% ############################
\Work{ % Autor i tytuł dzieła
  Andrzej Nowak \\
  „Dzieje Polski. Tom~I do~1202: Skąd nasz ród”,
  \cite{NowakDziejePolskiVolI2014} }


% ##################
\CenterBoldFont{Uwagi}


\start \textbf{Strona tytułowa.} W informacjach o~autorze jest podane,
że~był redaktorem naczelnym „ARCANA” w~latach 1994--2012, lecz
prawidłowy okresem są chyba lata 1995--2012.

\vspace{\spaceFour}



\start \Str{62} W~ostatnim paragrafie jest mowa o~czterech królów
Słowian, ale~wymienionych jest tylko trzech.

\vspace{\spaceFour}



% POPRAW
\start \StrWg{85}{16} „Nie było innej drogi do~Europy w~końcu X~w.
jak poprzez chrzest, nie~było dalej w~Europie innej drogi do humanizmu
jak poprzez chrześcijaństwo.” To zdanie chyba najlepiej oddaje
problem z~postrzeganiem chrześcijaństwa nie tylko przez Nowaka,
ale~i~przez przytłaczającą większość, jeśli nie~całość, polskiej myśli
patriotycznej. Chrześcijaństwo, nawet nie rzymskokatolickim,
jest tylko środkiem do osiągnięcia doczesnych, świeckich, ziemskich
celów, takich jak dostanie~się do~„Europy”, albo kultywowanie
humanizmu, nie~zaś wiarą pochodzącą od Boga i~jedyną drogą do życia
wiecznego. To~zaś zredukowanie Boga i~wiary, do~doczesnych korzyści,
to~straszliwe zło.

\vspace{\spaceFour}



\start \StrWd{92}{21--20} Nie potrafię zrozumieć, czy chodziło o~to,
że~Otton~II przejściowo zdobył Akwizgran, czy że~miejsce to było
przejściowo stolicą cesarską.

\vspace{\spaceFour}



\start \Str{96} W~swoim wykładzie z~cyklu
\href{https://www.youtube.com/watch?v=QovVLT2fitc}{„Filary Polskości:
  Mieszko i~Bolesław”} Nowak znacznie wyraźniej niż w~tej książce,
pokazał cały cynizm polityczny Mieszka~I. O~możliwości takiego
spojrzenia na tego władcę Nowak mówi tam otwarcie jednocześnie,
zarówno na wykładzie jaki i~książce, próbuje przykryć ten cynizm,
nazywając działania Mieszka „majstersztykiem polityki polskiej”.

\vspace{\spaceFour}



\start \StrWg{144}{24} Jest tu napisane, że~Miecław był cześnikiem,
jednak jaka była nadworna funkcja cześnika jest wyjaśnione dopiero
na~stronie~150.

\vspace{\spaceFour}



\start \StrWd{163}{17--15} Zdanie „Rozpoczyna od~modlitwy
do~św.~Piotra, patrona Stolicy Apostolskiej w~Rzymie i~zarazem
Gertrudowego syna, jadącego właśnie do~Rzymu, do~papieża.” można
zrozumieć w~ten sposób, że~Gertruda modli~się do swojego syna, co jest
nonsensowne. Prawdopodobnie miało być „zarazem patrona Gerturdowego
syna”, co czyni całe zdanie zrozumiałym i~sensownym.





% ##################
\CenterBoldFont{Błędy}


\begin{center}

  \begin{tabular}{|c|c|c|c|c|}
    \hline
    & \multicolumn{2}{c|}{} & & \\
    Strona & \multicolumn{2}{c|}{Wiersz} & Jest
                              & Powinno być \\ \cline{2-3}
    & Od góry & Od dołu & & \\
    \hline
    41  &  3 & & W X w. jeszcze & Jeszcze w X w. \\
    53  & &  1 & Księga Wyjścia & Księga Rodzaju \\
    61  & 10 & & do dziejów & dla dziejów \\
    100 & 14 & & Kto & Kto to \\
    169 & 17 & & mnie & do mnie \\
    202 & &  1 & posiąść & przesiąść \\
    236 & 14 & & „Gall”). & „Gall”. \\
    % & & & & \\
    % & & & & \\
    % & & & & \\
    \hline
  \end{tabular}

\end{center}


\noindent
\StrWg{61}{8} \\
\Jest  z~Rocznika kapituły krakowskiej dawnego \\
\Powin z~dawnego Rocznika kapituły krakowskiej \\

\vspace{\spaceTwo}
% ############################










% ######################################
\newpage
\section{Dzieje Polski z lat 1795--1914}

\vspace{\spaceTwo}
% ######################################



% ############################
\Work{ % Autor i tytuł dzieła
  Red. Andrzej Nowak \\
  „Historie Polski w~XIX wieku. Kominy, ludzie i~obłoki: \\
  modernizacja i~kultura. Tom~I”, \cite{HistoriaPolskiXIXVolI2013} }


% ##################
\CenterBoldFont{Błędy}


\begin{center}

  \begin{tabular}{|c|c|c|c|c|}
    \hline
    & \multicolumn{2}{c|}{} & & \\
    Strona & \multicolumn{2}{c|}{Wiersz} & Jest
                              & Powinno być \\ \cline{2-3}
    & Od góry & Od dołu & & \\
    \hline
    16 & & 15 & równości równość & równości \\
    % & & & & \\
    % & & & & \\
    % & & & & \\
    % & & & & \\
    \hline
  \end{tabular}

\end{center}

\vspace{\spaceTwo}
% ############################










% ######################################
\newpage
\section{Dzieje Polski po~roku 1914}

% \vspace{\spaceTwo}
% ######################################



% ############################
\Work{ % Autor i tytuł dzieła
  Paweł Zyzak \\
  „Efekt domina. Czy Ameryka obaliła komunizm w~Polsce? \\
  Tom~I”, \cite{ZyzakEfektDominaVolI2016} }


% ##################
\CenterBoldFont{Błędy}


\begin{center}

  \begin{tabular}{|c|c|c|c|c|}
    \hline
    & \multicolumn{2}{c|}{} & & \\
    Strona & \multicolumn{2}{c|}{Wiersz} & Jest
                              & Powinno być \\ \cline{2-3}
    & Od góry & Od dołu & & \\
    \hline
    11  &  4 & & Labor & \textit{Labor} \\
    11  & & 15 & in & \textit{in} \\
    12  &  3 & & \textit{Konfederacja} & Konfederacja \\
    12  & 23 & & CSS & CSSA \\
    13  & 21 & & Development & \textit{Development} \\
    13  & 22 & & The & \textit{The} \\
    13  & & 11 & The~International Rescue
           & \textit{The~International Rescue} \\
    14  & &  3 & Office & \textit{Office} \\
    16  & 15 & & I~Rozwoju & i~Rozwoju \\
    16  & & 14 & \textit{Labour} & \textit{Trade Union} \\
    17  &  5 & & miała obejmować & obejmować \\
    31  & &  2 & \textit{WCFL} & \textit{WCL} \\
    32  & &  7 & podziałało & podziałało~to \\
    37  & 16 & & \textit{Konfekcji Damskiej} & Konfekcji Damskiej \\
    41  &  3 & & polskiego.~ZRK & polskiego~ZRK \\
    47  & &  1 & Katyń & \textit{Katyń} \\
    48  & & 12 & za-oceanicznych & zaoceanicznych \\
    % & & & & \\
    % & & & & \\
    % & & & & \\
    % & & & & \\
    % & & & & \\
    % & & & & \\
    % & & & & \\
    % & & & & \\
    % & & & & \\
    % & & & & \\
    % & & & & \\
    % & & & & \\
    % & & & & \\
    % & & & & \\
    % & & & & \\
    % & & & & \\
    % & & & & \\
    % & & & & \\
    % & & & & \\
    % & & & & \\
    % & & & & \\
    % & & & & \\
    % & & & & \\
    \hline
  \end{tabular}

\end{center}


\noindent
\textbf{Tyla okładka, wiersz 17.} \\
\Jest Sorosa ,Williama \\
\Powin Sorosa, Williama \\
\StrWd{11}{16} \\
\Jest  Amerian Seafarers Union \\
\Powin \textit{Amerian Seafarers Union} \\
\StrWd{11}{5} \\
\Jest  Council~of Economic Advisers \\
\Powin \textit{Council~of Economic Advisers} \\
\StrWd{11}{1} \\
\Jest  Council on~Foregin Relations \\
\Powin \textit{Council on~Foregin Relations} \\
\StrWg{12}{11} \\
\Jest  \textit{Congress}) --~Kanadyjski Kongres Związków Zawodowych \\
\Powin \textit{Congress} --~Kanadyjski Kongres Związków Zawodowych) \\
\StrWd{12}{18} \\
\Jest  Emergency Committe for~Aid to~Poland \\
\Powin \textit{Emergency Committe for~Aid to~Poland} \\
\StrWd{12}{8} \\
\Jest  (\textit{Froce Ouvri\'{e}re} ) \\
\Powin (\textit{Froce Ouvri\'{e}re} --~Główna Konfederacja Pracy --~Siły
Pracy) \\
\StrWg{13}{10} \\
\Jest  Generalized System~of Preferences \\
\Powin \textit{Generalized System~of Preferences} \\
\StrWd{13}{4} \\
\Jest  \textit{Leuven}) \\
\Powin \textit{Leuven} --~Kotlicki Ośrodek Dokumentacyjny i~Badań
Katolickich Uniwersytetu Leuven \\
\StrWg{15}{1} \\
\Jest  Polish American Congress Charitable Foundation \\
\Powin \textit{Polish American Congress Charitable Foundation} \\
\StrWg{15}{3} \\
\Jest  Polish-American Enterprise Fund \\
\Powin \textit{Polish-American Enterprise Fund} \\
\StrWg{15}{10} \\
\Jest  Postal, Telegraph and~Telephone International \\
\Powin \textit{Postal, Telegraph and~Telephone International} \\
\StrWg{43}{2} \\
\Jest  \textit{Univeristy~of Illinois at~Urbana Champaign, 1998} \\
\Powin Univeristy~of Illinois at~Urbana-Champaign, 1998 \\

\vspace{\spaceTwo}
% ############################










% ############################
\Work{ % Autor i tytuł dzieła
  Paweł Zyzak \\
  „Efekt domina. Czy Ameryka obaliła komunizm w~Polsce? \\
  Tom~II”, \cite{ZyzakEfektDominaVolII2016} }


% ##################
\CenterBoldFont{Uwagi}


\start \StrWg{653}{3} Nie jest podana data wydania książki
Dubinsky'ego i~Raskina. W~skutek tego daty tej brakuje również
w~pierwszym tomie, str.~39, wiersz ósmy od~dołu.





% ##################
\CenterBoldFont{Błędy}


\begin{center}

  \begin{tabular}{|c|c|c|c|c|}
    \hline
    & \multicolumn{2}{c|}{} & & \\
    Strona & \multicolumn{2}{c|}{Wiersz} & Jest
                              & Powinno być \\ \cline{2-3}
    & Od góry & Od dołu & & \\
    \hline
    % & & & & \\
    % & & & & \\
    % & & & & \\
    % & & & & \\
    % & & & & \\
    % & & & & \\
    % & & & & \\
    % & & & & \\
    % & & & & \\
    % & & & & \\
    % & & & & \\
    % & & & & \\
    % & & & & \\
    % & & & & \\
    % & & & & \\
    % & & & & \\
    % & & & & \\
    % & & & & \\
    % & & & & \\
    % & & & & \\
    % & & & & \\
    % & & & & \\
    664 & & 22 & Katyń & \textit{Katyń} \\
    % & & & & \\
    % & & & & \\
    \hline
  \end{tabular}

\end{center}


\vspace{\spaceTwo}
% ############################










% ######################################
\subsection{Lata 1914--1939}

\vspace{\spaceThree}
% ######################################



% ############################
\Work{ % Tytuł i autor dzieła
  Wojciech Roszkowski \\
  „Najnowsza historia Polski: 1914--1939”,
  \cite{RoszkowskiNajnowszaHistoriaPolski1914-1939Wyd2011} }


% ##################
\CenterBoldFont{Uwagi}


\start Karygodną, i~to~niezależnie od uznawanej metodologi pisania
prac historycznych, cechą całego tego wydania „Najnowszej historii
Polski”, jest nieumieszczenie w~każdej tomie listy używanych
skrótów. Należy dodać, że~jeśli skrót został wprowadzony w~jednym
tomie, to~nie jest już wyjaśniany w~następnych, co~dodatkowo
komplikuje sprawę.

\vspace{\spaceFour}



\start Ciekawym wydaje~się zauważanie, że~w~tej książce Roszkowski
zrealizował chyba idealnie, jedno z~założeń zaprojektowanego przez
piłsudczyków programu edukacji, przyjętego po reformie
jędrzejewiczowskiej (1932): sprowadzenia lat 1918--1920 wyłączenie
do~tematu walki o~granice. Więcej na ten temat \\
w~Andrzej Chojnowski „Kwestia patriotyzmu w~poszukiwaniach
programowych obozu piłsudczykowskiego”, str.~136
\cite{RedKloczkowskiPatriotyzmPolakow2006}.

\vspace{\spaceFour}



\start \Str{26} Podany tu opis przyczyn wybuchu I~Wojny Światowej,
zwłaszcza bardzo silne stwierdzenie, że~Austro-Węgry
wypowiedziały wojnę Serbii pod naciskiem Niemiec, warto skonfrontować
z~tym co pisze M. Gilbert w~swojej książce na temat tego przedziwnego
wydarzenia \cite{GilbertPierwszaWojnaSwiatowa2003}.

\vspace{\spaceFour}



\start \StrWd{78}{4} Cudzysłów otwarty w~tym wierszu nigdy nie został
zamknięty, przez co~nie wiadomo, gdzie~się kończy cytat.

\vspace{\spaceFour}



\start \StrWd{425}{6} Przy nazwisku „Unrag Józef” nie ma podanej
strony na~której~się ta postać pojawia.

\vspace{\spaceFour}





% ##################
\CenterBoldFont{Błędy}


\begin{center}

  \begin{tabular}{|c|c|c|c|c|}
    \hline
    & \multicolumn{2}{c|}{} & & \\
    Strona & \multicolumn{2}{c|}{Wiersz} & Jest
                              & Powinno być \\ \cline{2-3}
    & Od góry & Od dołu & & \\
    \hline
    10  & & 13 & przyszłe & przyszle \\
    20  & &  9 & sita & siła \\
    27  & & 16 & z agrozić & zagrozić \\
    29  & 21 & & Bąjończyków & Bajończyków \\
    30  & 21 & & uczestniczyło w~niej & wśród jej członków było \\
    31  & &  8 & Hans Beseler & Hans von Beseler \\
    36  &  3 & & POW & POW. \\
    37  & 12 & & LLOYDA & LOYDA \\
    47 & & 6 & przed nadchodzącą zimą & nadchodzącej zimy \\
    50  &  7 & & H.Wereszycki & H. Wereszycki \\
    50  &  8 & & R.Dmowski & R.~Dmowski \\
    50  &  8 & & J.Molenda & J.~Molenda \\
    50  &  8 & & \textit{Pibudnczcy} & \textit{Piłsudczycy} \\
    50  & 21 & & \textit{Pobki} & \textit{Polski} \\
    68  & 14 & & j~ednolitego & jednolitego \\
    73  & & 13 & 1920 R & 1920 R. \\
    & & 17 & W braku & Z braku \\
    % & & & & \\
    % & & & & \\
    % & & & & \\
    % & & & & \\
    % & & & & \\
    % & & & & \\
    % & & & & \\
    % & & & & \\
    % & & & & \\
    % & & & & \\
    411 &  4 & & \textit{50} & 50 \\
    411 &  7 & & \textit{210, 211} & 210, 211 \\
    411 &  9 & & \textit{113, 114, 170, 259} & 113, 114, 170, 259 \\
    411 & 14 & & \textit{114} & 14 \\
    411 & 18 & & \textit{389} & 389 \\
    411 & &  6 & \textit{390} & 390 \\
    411 & &  5 & \textit{391} & 391 \\
    411 & &  3 & \textit{14}  & 14 \\
    411 & &  3 & \textit{260} & 260 \\
    412 &  2 & & \textit{390, 391} & 390, 391 \\
    412 &  3 & & \textit{390, 391} & 390, 391 \\
    412 &  5 & & \textit{50}  & 50 \\
    412 &  7 & & \textit{210} & 210 \\
    412 & 13 & & \textit{308} & 308 \\
    412 & 13 & & \textit{171} & 171 \\
    412 & 15 & & \textit{171} & 171 \\
    412 & 19 & & \textit{211} & 211 \\
    412 & 20 & & \textit{308} & 308 \\
    412 & & 15 & \textit{113} & 113 \\
    412 & &  9 & \textit{391} & 391 \\
    412 & &  1 & \textit{14}  & 14 \\
    412 & &  1 & \textit{113} & 113 \\
    413 &  2 & & \textit{260} & 260 \\
    \hline
  \end{tabular}





  \begin{tabular}{|c|c|c|c|c|}
    \hline
    & \multicolumn{2}{c|}{} & & \\
    Strona & \multicolumn{2}{c|}{Wiersz} & Jest
                              & Powinno być \\ \cline{2-3}
    & Od góry & Od dołu & & \\
    \hline
    413 &  4 & & \textit{307} & 307 \\
    413 &  8 & & \textit{50}  &  50\\
    413 & 12 & & \textit{171, 210, 308, 309, 390}
           & 171, 210, 308, 309, 390 \\
    413 & 13 & & \textit{391} & 391 \\
    413 & 15 & & \textit{390} & 390 \\
    413 & 15 & & \textit{51}  & 51 \\
    413 & 17 & & \textit{171} & 171 \\
    413 & 19 & & \textit{390} & 390 \\
    413 & & 20 & \textit{390} & 390 \\
    413 & & 18 & \textit{307} & 307 \\
    413 & & 17 & \textit{114} & 114 \\
    413 & & 10 & \textit{113} & 113 \\
    413 & &  7 & \textit{390} & 390 \\
    413 & &  4 & \textit{50}  & 50 \\
    413 & &  3 & \textit{260} & 260 \\
    414 &  2 & & \textit{260} & 260 \\
    414 &  3 & & \textit{170} & 170 \\
    414 &  6 & & \textit{114} & 114 \\
    414 &  7 & & \textit{170} & 170 \\
    414 & 11 & & \textit{210} & 210 \\
    414 & 12 & & \textit{211} & 211 \\
    414 & & 19 & \textit{51}  & 51 \\
    414 & & 16 & \textit{210} & 210 \\
    414 & & 15 & \textit{113} & 113 \\
    414 & & 10 & \textit{390} & 390 \\
    414 & &  6 & \textit{307} & 307 \\
    414 & &  2 & \textit{260} & 260 \\
    % & & & & \\
    % & & & & \\
    % & & & & \\
    % & & & & \\
    % & & & & \\
    % & & & & \\
    % & & & & \\
    % & & & & \\
    % & & & & \\
    % & & & & \\
    % & & & & \\
    % & & & & \\
    % & & & & \\
    % & & & & \\
    % & & & & \\
    % & & & & \\
    % & & & & \\
    % & & & & \\
    % & & & & \\
    % & & & & \\
    % & & & & \\
    % & & & & \\
    % & & & & \\
    % & & & & \\
    % & & & & \\
    % & & & & \\
    % & & & & \\
    % & & & & \\
    % & & & & \\
    % & & & & \\
    % & & & & \\
    % & & & & \\
    % & & & & \\
    % & & & & \\
    % & & & & \\
    \hline
  \end{tabular}





  % \begin{tabular}{|c|c|c|c|c|}
  %   \hline
  %   & \multicolumn{2}{c|}{} & & \\
  %         %   Strona & \multicolumn{2}{c|}{Wiersz} & Jest
  %         %   & Powinno być \\ \cline{2-3}
  %         %   & Od góry & Od dołu & & \\
  %         %   \hline
  %   %   & & & & \\
  %   %   & & & & \\
  %   %   & & & & \\
  %   %   & & & & \\
  %   %   & & & & \\
  %   %   & & & & \\
  %   %   & & & & \\
  %   %   & & & & \\
  %   %   & & & & \\
  %   %   & & & & \\
  %   %   & & & & \\
  %   %   & & & & \\
  %   %   & & & & \\
  %   %   & & & & \\
  %   %   & & & & \\
  %   %   & & & & \\
  %   %   & & & & \\
  %   %   & & & & \\
  %   %   & & & & \\
  %   %   & & & & \\
  %   %   & & & & \\
  %   %   & & & & \\
  %   %   & & & & \\
  %   %   & & & & \\
  %   %   & & & & \\
  %   %   & & & & \\
  %   %   & & & & \\
  %   %   & & & & \\
  %   %   & & & & \\
  %   %   & & & & \\
  %   %   & & & & \\
  %   %   & & & & \\
  %   %   & & & & \\
  %   %   & & & & \\
  %   %   & & & & \\
  %   %   & & & & \\
  %   %   & & & & \\
  %   %   & & & & \\
  %   \hline
  % \end{tabular}





  \begin{tabular}{|c|c|c|c|c|}
    \hline
    & \multicolumn{2}{c|}{} & & \\
    Strona & \multicolumn{2}{c|}{Wiersz} & Jest
                              & Powinno być \\ \cline{2-3}
    & Od góry & Od dołu & & \\
    \hline
    425 &  8 & & \textit{170} & 170 \\
    425 & 13 & & \textit{50}  & 50 \\
    425 & 20 & & Adolf25 & Adolf 25 \\
    425 & 23 & & \textit{170} & 170 \\
    425 & & 16 & \textit{114, 170, 211, 260} & 114, 170, 211, 260 \\
    425 & & 15 & \textit{308, 391} & 308, 391 \\
    425 & & 14 & \textit{389} & 389 \\
    425 & & 13 & \textit{113, 170, 171} & 113, 170, 171 \\
    425 & & 12 & \textit{308} & 308 \\
    425 & &  8 & \textit{50}  &  50 \\
    425 & &  3 & \textit{389} & 389 \\
    426 &  2 & & \textit{114} & 114 \\
    426 &  9 & & \textit{50}  &  50 \\
    426 & 14 & & \textit{211} & 211 \\
    426 & 17 & & \textit{50}  &  50 \\
    426 & 18 & & \textit{114} & 114 \\
    426 & 21 & & \textit{259} & 259 \\
    426 & 22 & & \textit{260} & 260 \\
    426 & 23 & & \textit{308} & 308 \\
    426 & &  6 & \textit{50}  &  50 \\
    426 & &  5 & \textit{170} & 170 \\
    426 & &  3 & \textit{170} & 170 \\
    426 & &  2 & \textit{210, 211} & 210, 211 \\
    427 &  4 & & \textit{389} & 389 \\
    427 &  7 & & \textit{170, 391} & 170, 391 \\
    427 &  8 & & \textit{114} & 114 \\
    \hline
  \end{tabular}

\end{center}


\noindent
\StrWd{47}{6} \\
\Jest  nadchodzącą zimą \\
\Powin nadchodzącej zimy \\

\vspace{\spaceTwo}
% ############################










% ######################################
\subsection{Lata 1939--1989}

\vspace{\spaceThree}
% ######################################



% ############################
\Work{ % Redaktorzy i tytuł dzieła
  Redakcja i~opracowanie Adam Dziurok, Filip Musiał \\
  „Instrukcje, wytyczne, pisma Departamentu IV~Ministerstwa Spraw
  Wewnętrznych z~lat 1962--1989. Wybór dokumentów”,
  \cite{RedDziurokMusialInstrukcjeWytycznePisma2017} }


% ##################
\CenterBoldFont{Uwagi do konkretnych stron}


\start \Str{905} Nazwiska pisane czcionką prostą (antykwą?) należą
do~bohaterów historii omawianej w~tym tomie, zaś~te pisane kursywą
do~badaczy i~historyków.


\vspace{\spaceTwo}
% ############################










% ##############################
\Work{ % Redaktor i tytuł dzieła
  Red. Piotr Franaszka \\
  „Granice kompromisu. Naukowcy wobec aparatu władzy ludowej”,
  \cite{RedFranaszekGraniceKompromisu2015} }


% ##################
\CenterBoldFont{Błędy}


\begin{center}

  \begin{tabular}{|c|c|c|c|c|}
    \hline
    & \multicolumn{2}{c|}{} & & \\
    Strona & \multicolumn{2}{c|}{Wiersz} & Jest
                              & Powinno być \\ \cline{2-3}
    & Od góry & Od dołu & & \\
    \hline
    9  & &  2 & A.Dziuba & A.~Dziuba \\
    17 & 14 & & ZMP$^{ 33 }$\ldots & ZMP$^{ 33 }$. \\
    % & & & & \\
    % & & & & \\
    % & & & & \\
    \hline
  \end{tabular}

\end{center}


\vspace{\spaceTwo}
% ############################










% ############################
\Work{ % Autor i tytuł dzieła
  Wojciech Roszkowski \\
  „Najnowsza historia Polski: 1939--1945”,
  \cite{RoszkowskiNajnowszaHistoriaPolski1939-1945Wyd2011} }


% ##################
\CenterBoldFont{Uwagi}





% ##################
\CenterBoldFont{Błędy}


\begin{center}

  \begin{tabular}{|c|c|c|c|c|}
    \hline
    & \multicolumn{2}{c|}{} & & \\
    Strona & \multicolumn{2}{c|}{Wiersz} & Jest
                              & Powinno być \\ \cline{2-3}
    & Od góry & Od dołu & & \\
    \hline
    10 & & 5 & Wisy & Wisły \\
    % & & & & \\
    % & & & & \\
    % & & & & \\
    \hline
  \end{tabular}

\end{center}

\vspace{\spaceTwo}






% ############################
\Work{ % Autor i tytuł dzieła
  Wojciech Roszkowski \\
  „Najnowsza historia Polski: 1980--1989”,
  \cite{RoszkowskiNajnowszaHistoriaPolski1980-1989Wyd2011} }


% ##################
\CenterBoldFont{Uwagi}


\start \Str{8} Jest to jeden z~największych przykładów straszliwie
suchej, pozbawiającej wydarzenia z~przeszłości realności, a~także nie
pozwalającej~się zorientować w~symbolach o~ogromnej wadze,
historiografii pozytywistycznej, chyba szkoły niemieckiej, jaką
reprezentuje Roszkowski. Jeśli dobrze wywnioskowałem z~tego
wystąpienia
\href{https://www.youtube.com/watch?v=6B93_3CCMac}{Sławomira
  Cenckiewicza}, to słynny skok Wałęsy przez płot, wedle niego był to
w~istocie mur, miał miejsce w~opisywanym tu dniu 22~sierpnia 1980~r.
Wałęsa musiał przeskoczyć ten mur, właśnie dlatego, że~spóźnił~się na
główne otwarcie stoczni. Tylko pomarzyć jak~by to opisał Paul Johnson.

\vspace{\spaceFour}



\start \Str{16} Choć nie~czytałem poprzednich tomów, i~być może jest
tam informacja o~tym kim jest Karol Modzelewski, to w~tym tomie jest
wymieniony tylko dwa razy, i~z tego powodu powinna być podana jakaś
informacja o~nim, by~czytelnik wiedział kto jest autorem tak wiele
znaczącej nazwy jak „Solidarność”.

\vspace{\spaceFour}



\start \StrWd{23}{6} Cudzysłów otwarty w~tym wierszu nigdy nie został
zamknięty, przez co~nie wiadomo, gdzie~się kończy cytat.

\vspace{\spaceFour}



\start \StrWd{26}{6} Nie jest podane co było tematem omawianej
tu~narady sztabowej.


% Błędy:\\
% \begin{center}
%   \begin{tabular}{|c|c|c|c|c|}
%     \hline
%     & \multicolumn{2}{c|}{} & & \\
%           %     Strona & \multicolumn{2}{c|}{Wiersz} & Jest
%     %     & Powinno być \\ \cline{2-3}
%           %     & Od góry & Od dołu & & \\
%     %     \hline
%     & & & & \\
%     & & & & \\ \hline
%   \end{tabular}
% \end{center}

\vspace{\spaceTwo}
% ############################










% ############################
\Work{ % Redaktorzy i tytuł dzieła
  Redakcja naukowa Mirosław Sikora; współpraca Piotr Fuglewicz \\
  „High-tech za~żelazną kurtyną: elektronika, komputery
  i~systemy sterowania w~PRL”, \cite{SikoraFuglewiczHighTech2017} }


% \CenterTB{Uwagi}

% \start \StrWd{26}{6} Nie jest podane co było tematem omawianej
% tu~narady sztabowej.


% ##################
\CenterBoldFont{Błędy}


\begin{center}

  \begin{tabular}{|c|c|c|c|c|}
    \hline
    & \multicolumn{2}{c|}{} & & \\
    Strona & \multicolumn{2}{c|}{Wiersz} & Jest
                              & Powinno być \\ \cline{2-3}
    & od góry & Od dołu & & \\
    \hline
    4   & &  7 & przeciwko & Przeciwko \\
    22  & 13 & & o\ldots{} fantastyce & o~fantastyce \\ % Czy to aby na pewno
    % błąd?
    50  & & 15 & tej uchwały & tej \\
    % & & & & \\
    % & & & & \\
    % & & & & \\
    \hline
  \end{tabular}

\end{center}

\vspace{\spaceTwo}
% ############################










% ######################################
\section{Dzieje Polski po~1989~r.}

\vspace{\spaceTwo}
% ######################################



% ############################
\Work{ % Autorzy i tytuł dzieła
  Jan Kofman, Wojciech Roszkowski \\
  „Transformacja i~postkomunizm”,
  \cite{KofmanRoszkowskiTransformacjaIPostkomunizm1999} }


% ##################
\CenterBoldFont{Uwagi}


\start \Str{42} Przeliczyłem za~pomocą komputera przytoczone tu~dane
i~otrzymany wynik nie zawsze pokrywał~się z~tym co zostało podane
w~ostatniej kolumnie. Dokładniej, wyniki różniły~się zawsze, jednak
w~wielu wypadkach był to zapewne wynik przyjętego sposobu
zaokrąglania, dlatego w~błędach odnotowałem tylko te przypadki,
gdy~różnica była rzędu procenta lub~większa.

\vspace{\spaceFour}


% ##################
\CenterBoldFont{Błędy}


\begin{center}

  \begin{tabular}{|c|c|c|c|c|}
    \hline
    & \multicolumn{2}{c|}{} & & \\
    Strona & \multicolumn{2}{c|}{Wiersz} & Jest
                              & Powinno być \\ \cline{2-3}
    & od góry & od dołu & & \\
    \hline
    13  & &  4 & o~ekspansji & do~ekspansji \\
    16  &  9 & & marntrawstwem & marnotrawstwem \\
    18  &  2 & & jego do & do~jego \\
    26  & & 19 & dziewięćdziesięciokrotne & czterdziestokrotnie \\
    26  & &  5 & 1998 & 1988 \\
    42  & 11 & & 80,2 & 78,5 \\
    42  & 14 & & (107,4) & (108,3) \\
    42  & & 11 & 77,8 &  76,6 \\
    42  & &  8 & (35,7) & (37,4) \\
    % & & & & \\
    % & & & & \\
    % & & & & \\
    % & & & & \\
    % & & & & \\
    % & & & & \\
    \hline
  \end{tabular}

\end{center}

\vspace{\spaceTwo}
% ############################










% ############################
\Work{ % Autor i tytuł dzieła
  Wojciech Roszkowski \\
  „Najnowsza historia Polski: 1989--2011”,
  \cite{RoszkowskiNajnowszaHistoriaPolski1989-2011Wyd2011} }


% ##################
\CenterBoldFont{Uwagi}


% ##################
\CenterBoldFont{Błędy}


\begin{center}

  \begin{tabular}{|c|c|c|c|c|}
    \hline
    & \multicolumn{2}{c|}{} & & \\
    Strona & \multicolumn{2}{c|}{Wiersz} & Jest
                              & Powinno być \\ \cline{2-3}
    & Od góry & Od dołu & & \\
    \hline
    23 & &  2 & na~naturalnym & za~naturalnym \\
    38 & & 11 & IX & XI \\
    45 & &  2 & przez & przed \\
    46 & & 15 & 2,5\%\% & 2,5\% \\
    65 & & 16 & wygrali & wyciągnęli \\
    % & & & & \\
    % & & & & \\
    % & & & & \\
    % & & & & \\
    % & & & & \\
    % & & & & \\
    % & & & & \\
    \hline
  \end{tabular}

\end{center}


\vspace{\spaceTwo}
% ############################










% ######################################
\section{Historia Polski~-- zbiory artykułów}

\vspace{\spaceTwo}
% ######################################



% ############################
\Work{ % Autor i tytuł dzieła
  Janusz Cisek \\
  „Oskar Halecki. Historyk~-- Szermierz Wolności”,
  \cite{CisekOskarHalecki2009} }


% ##################
\CenterBoldFont{Uwagi}


\start \Str{5} Tytuł części „Historyk Kościoła” jest trochę
nieadekwatna, bowiem dwa z~trzech zamieszczonych tu artykułów dotyczą
historii Kościoła w~Polsce, co jest bardzo małym wycinkiem z~tego
zagadnienia.

\vspace{\spaceFour}



\start \StrWg{48}{4} Artykuł ten mógł~się rzeczywiście ukazać
w~wymienionym tu roku 1963, jednak spodziewałbym~się raczej tego,
że~został opublikowany w~1966 roku.

\vspace{\spaceFour}



\start \StrWg{70}{7} Słowa „i~jego pielgrzymki” mogą być wynikiem
błędu, w~poprawnej wersji powinny brzmieć „i~jego pielgrzymi”. Mogą
też mieć następujący sens. Poeci romantyczni byli „wieszczami”
narodu, jak i~w~szczególny sposób, czasów jego pielgrzymki.





% ##################
\CenterBoldFont{Błędy}


\begin{center}

  \begin{tabular}{|c|c|c|c|c|}
    \hline
    & \multicolumn{2}{c|}{} & & \\
    Strona & \multicolumn{2}{c|}{Wiersz} & Jest
                              & Powinno być \\ \cline{2-3}
    & Od góry & Od dołu & & \\
    \hline
    44  & 14 & & Honorowej: & Honorowej, \\
    90  & 10 & & W~braku & Z~braku \\
    93  & & 12 & Rzym. & Rzym, \\
    % & & & & \\
    % & & & & \\
    % & & & & \\
    % & & & & \\
    % & & & & \\
    % & & & & \\
    % & & & & \\
    % & & & & \\
    % & & & & \\
    % & & & & \\
    % & & & & \\
    % & & & & \\
    % & & & & \\
    % & & & & \\
    % & & & & \\
    % & & & & \\
    % & & & & \\
    % & & & & \\
    % & & & & \\
    % & & & & \\
    % & & & & \\
    \hline
  \end{tabular}

\end{center}


\vspace{\spaceTwo}
% ############################










% ####################################################################
% ####################################################################
% Bibliografia
\bibliographystyle{plalpha}

\bibliography{HistoryBooks,VariousFieldsBooks}{}





% ############################

% Koniec dokumentu
\end{document}