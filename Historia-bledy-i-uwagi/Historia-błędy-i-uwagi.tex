% Autor: Kamil Ziemian

% --------------------------------------------------------------------
% Podstawowe ustawienia i pakiety
% --------------------------------------------------------------------
\RequirePackage[l2tabu, orthodox]{nag} % Wykrywa przestarzałe i niewłaściwe
% sposoby używania LaTeXa. Więcej jest w l2tabu English version.
\documentclass[a4paper,11pt]{article}
% {rozmiar papieru, rozmiar fontu}[klasa dokumentu]
\usepackage[MeX]{polski} % Polonizacja LaTeXa, bez niej będzie pracował
% w języku angielskim.
\usepackage[utf8]{inputenc} % Włączenie kodowania UTF-8, co daje dostęp
% do polskich znaków.
\usepackage{lmodern} % Wprowadza fonty Latin Modern.
\usepackage[T1]{fontenc} % Potrzebne do używania fontów Latin Modern.



% ----------------------------
% Podstawowe pakiety (niezwiązane z ustawieniami języka)
% ----------------------------
\usepackage{microtype} % Twierdzi, że poprawi rozmiar odstępów w tekście.
\usepackage{graphicx} % Wprowadza bardzo potrzebne komendy do wstawiania
% grafiki.
\usepackage{verbatim} % Poprawia otoczenie VERBATIME.
\usepackage{textcomp} % Dodaje takie symbole jak stopnie Celsiusa,
% wprowadzane bezpośrednio w tekście.
\usepackage{vmargin} % Pozwala na prostą kontrolę rozmiaru marginesów,
% za pomocą komend poniżej. Rozmiar odstępów jest mierzony w calach.
% ----------------------------
% MARGINS
% ----------------------------
\setmarginsrb
{ 0.7in} % left margin
{ 0.6in} % top margin
{ 0.7in} % right margin
{ 0.8in} % bottom margin
{  20pt} % head height
{0.25in} % head sep
{   9pt} % foot height
{ 0.3in} % foot sep



% ------------------------------
% Często używane pakiety
% ------------------------------
\usepackage{csquotes} % Pozwala w prosty sposób wstawiać cytaty do tekstu.
\usepackage{xcolor} % Pozwala używać kolorowych czcionek (zapewne dużo
% więcej, ale ja nie potrafię nic o tym powiedzieć).



% ------------------------------
% Często używane pakiety
% ------------------------------
\usepackage{csquotes} % Pozwala w prosty sposób wstawiać cytaty do tekstu.
\usepackage{xcolor} % Pozwala używać kolorowych czcionek (zapewne dużo
% więcej, ale ja nie potrafię nic o tym powiedzieć).



% ----------------------------
% Pakiety napisane przez użytkownika.
% Mają być w tym samym katalogu to ten plik .tex
% ----------------------------
\usepackage{latexshortcuts}



% --------------------------------------------------------------------
% Dodatkowe ustawienia dla języka polskiego
% --------------------------------------------------------------------
\renewcommand{\thesection}{\arabic{section}.}
% Kropki po numerach rozdziału (polski zwyczaj topograficzny)
\renewcommand{\thesubsection}{\thesection\arabic{subsection}}
% Brak kropki po numerach podrozdziału



% ----------------------------
% Ustawienia różnych parametrów tekstu
% ----------------------------
\renewcommand{\arraystretch}{1.2} % Ustawienie szerokości odstępów między
% wierszami w tabelach.






% Koniec komend
% ############################





% ----------------------------
% Pakiet "hyperref"
% Polecano by umieszczać go na końcu preambuły.
% ----------------------------
\usepackage{hyperref} % Pozwala tworzyć hiperlinki i zamienia odwołania
% do bibliografii na hiperlinki.





% ####################################################################
% Początek dokumentu
\begin{document}
% ####################################################################



% ######################################
\Main{Historia, błędy i~uwagi} % Tytuł całego tekstu

\vspace{\spaceTwo} \vspace{\spaceThree}
% ######################################



% ############################
\Field{Historia świętej wiary} % Nazwa dziedziny

\vspace{\spaceTwo} % \vspace{\spaceThree}
% ############################



% ##################
\Work{ % Autor i tytuł dzieła
  Richard Butterwick \\
  ,,Polska Rewolucja a~Kościół Katolicki 1788--1792'',
  \cite{ButterwickPolskaRewolucjaAKosciolKatolicki12} }



\CenterTB{Uwagi}

\start \Str{28} Euzebiusz \\


% \CenterTB{Błędy}
% \begin{center}
%   \begin{tabular}{|c|c|c|c|c|}
%     \hline
%     & \multicolumn{2}{c|}{} & & \\
%     Strona & \multicolumn{2}{c|}{Wiersz}& Jest & Powinno być \\ \cline{2-3}
%     & Od góry & Od dołu &  &  \\ \hline
%     & & & & \\
%     & & & & \\
%     & & & & \\
%     & & & & \\
%     %     \hline
%   \end{tabular}


\vspace{\spaceTwo}





% ##################
\newpage
\Work{ % Autor i tytuł dzieła
  Warren H.~Carroll \\
  ,,Historia Chrześcijaństwa. Tom~I: Narodziny Chrześcijaństwa'',
  \cite{CarrollHistoriaChrzecijanstwaTomI09} }

\CenterTB{Błędy}
\begin{center}
  \begin{tabular}{|c|c|c|c|c|}
    \hline
    & \multicolumn{2}{c|}{} & & \\
    Strona & \multicolumn{2}{c|}{Wiersz} & Jest
                              & Powinno być \\ \cline{2-3}
    & Od góry & Od dołu & & \\
    \hline
    % & & & & \\
    % & & & & \\
    % & & & & \\
    % & & & & \\
    % & & & & \\
    % & & & & \\
    % & & & & \\
    % & & & & \\
    % & & & & \\
    % & & & & \\
    % & & & & \\
    % & & & & \\
    % & & & & \\
    % & & & & \\
    % & & & & \\
    % & & & & \\
    % & & & & \\
    % & & & & \\
    561 & & 19 & \emph{Bible, a~Historical}
           & \emph{Bible: A~Historical} \\
    562 & 11 & & \emph{Desert, a~History} & \emph{Desert: A~History} \\
    562 & 15 & & \emph{Canaan: the~Ras} & \emph{Canaan: The~Ras} \\
    562 & & 14 & \emph{Judaea} & \emph{Judea} \\
    562 & & 13 & \emph{Israel, from} & \emph{Israel: From} \\
    562 & &  4 & \emph{Covenant, a~Study} & \emph{Covenant: A~Study} \\
    % & & & & \\
    % & & & & \\
    % & & & & \\
    % & & & & \\
    % & & & & \\
    \hline
  \end{tabular}
\end{center}

\vspace{\spaceTwo}





% ##################
\Work{ % Autor i tytuł dzieła
  Warren H.~Carroll \\
  ,,Historia Chrześcijaństwa. Tom~II: Budowanie Chrześcijaństwa'',
  \cite{CarrollHistoriaChrzecijanstwaTomII10} }


\CenterTB{Uwagi}

\start Tłumaczenie podtytułu tego tomu ,,Budowanie Chrześcijaństwa''
jest wyjątkowo niezręczne. Należy zwrócić uwagę, że~Carroll nadał
swojemu cyklowi tytuł ,,History~of Christendom'' nie ,,History~of
Christianity''. ,,Chrisitianity'' tłumaczy się prosto jako
,,chrześcijaństwo'', ,,Chistendom'' nie ma chyba odpowiednika w~języku
polski, w~tym przypadku zaś można jego sens chyba wyjaśnić, jako
wspólnotę ludzi, której sposób życia definiuje chrześcijaństwo.
W~szczególności ,,Christendom'' oznacza również sens polityczny, jako
zbioru państw, które~są połączone wspólną wiarą chrześcijańską i~tym
samym powinny działać jak różne członki jednego ciała.

Jakkolwiek więc tłumaczenie tytułu całego cyklu jako ,,Historia
Chrześcijaństwa'' ma~sens, to tego podtytułu jako ,,Budowanie
Chrześcijaństwa'' już nie. Sugeruje bowiem, że~religia chrześcijańska
była budowana, podczas gdy ona została już wzniesiona przez Chrystusa,
zaś budowane było właśnie ,,Christendom'', wspólnota ludzka żyjąca jej
prawami.

\CenterTB{Błędy}
\begin{center}
  \begin{tabular}{|c|c|c|c|c|}
    \hline
    & \multicolumn{2}{c|}{} & & \\
    Strona & \multicolumn{2}{c|}{Wiersz} & Jest
                              & Powinno być \\ \cline{2-3}
    & Od góry & Od dołu & & \\
    \hline
    % & & & & \\
    % & & & & \\
    % & & & & \\
    % & & & & \\
    % & & & & \\
    585 & &  7 & London. & London \\
    586 &  5 & & Struggle & \emph{Struggle} \\
    586 & 15 & & \emph{Chalcedon} & \emph{Chalcedon} \\
    586 & & 15 & \emph{1} & \emph{the~First} \\
    586 & & 10 & Danielou, Jean & Danielou Jean \\
    586 & & 10 & Henri Marrou & Marrou Henri \\
    587 & 13 & & \emph{A.D.} & \emph{A.D.}, \\
    587 & & 17 & \emph{Jerome,} & \emph{Jerome:} \\
    587 & &  8 & London, & London \\
    587 & &  2 & \emph{Moesia, a~History} & \emph{Moesia: History} \\
    588 &  2 & & \emph{Arthur, a~History} & \emph{Arthur: A~History} \\
    588 &  4 & & \emph{Invasion; the Making}
           & \emph{Invasion: The making} \\
    588 & 15 & & \emph{God; the Life} & \emph{God: The Life} \\
    588 & & 18 & \emph{Britain s} & \emph{Britain's} \\
    588 & & 17 & \emph{Chalcedon, a~Historical}
           & \emph{Chalcedon: A~Historical} \\
    590 & 14 & & \emph{Constantinople; Ecclesiastical}
           & \emph{Constantinople: Ecclesiastical} \\
    591 & & 12 & London, & London \\
    592 & 16 & & \emph{Lyons, Churchman} & \emph{Lyons: Churchman} \\
    592 & &  9 & \emph{Great, the~King} & \emph{Great: The~King} \\
    592 & &  8 & \emph{Canterbury; a~Study} & \emph{Canterbury: A~Study} \\
    592 & &  3 & \emph{Slavs; Saints} & \emph{Slavs: Saints} \\
    593 &  7 & & \emph{Empire; the~Arabs} & \emph{Empire: The~Arabs} \\
    593 & 12 & & \emph{Byzantium: the~Imperial}
           & \emph{Byzantium: The~Imperial} \\
    593 & 14 & & \emph{Kings; Their} & \emph{Kings: Their} \\
    593 & 16 & & \emph{England; a~History} & \emph{England: A~History} \\
    593 & 20 & & \emph{Great: the~Truth} & \emph{Great: The~Truth} \\
    593 & & 14 & \emph{State; the~Period} & \emph{State: The~Period} \\
    593 & & 12 & \emph{Dragon; Alfred} & \emph{Dragon: Alfred} \\
    593 & &  5 & \emph{St.~Peter; the~Birth}
           & \emph{St.~Peter: The~Birth} \\
    594 & 12 & & \emph{Dublin: the~History} & \emph{Dublin: The~History} \\
    594 & &  9 & \emph{Desiderius; Montecassino}
           & \emph{Desiderius: Montecassino} \\
    594 & &  1 & \emph{Empire; the~Arabs} & \emph{Empire: The~Arabs} \\
    595 &  1 & & \emph{Rufus; an~Investigation}
           & \emph{Rufus: An~Investigation} \\
    595 &  6 & & \emph{Byzantium: the~Imperial}
           & \emph{Byzantium: The~Imperial} \\
    595 &  8 & & \emph{England; a~History} & \emph{England: A~History} \\
    595 & 11 & & \emph{Kings; Their} & \emph{Kings: Their} \\
    \hline
  \end{tabular}

  \begin{tabular}{|c|c|c|c|c|}
    \hline
    & \multicolumn{2}{c|}{} & & \\
    Strona & \multicolumn{2}{c|}{Wiersz}& Jest & Powinno być \\ \cline{2-3}
    & Od góry & Od dołu &  &  \\ \hline
    595 & 16 & & \emph{State: the~Period} & \emph{State: The~Period} \\
    595 & 19 & & \emph{Tancred: a~Study} & \emph{Tancred: A~Study} \\
    595 & & 10 & \emph{Saint Peter; the~Reception}
           & \emph{Saint Peter: The~Reception} \\
    596 &  6 & & \emph{440} & 440 \\
    % Popraw dalsze błędy w indeksie
    % & & & & \\
    % & & & & \\
    % & & & & \\
    % & & & & \\
    % & & & & \\
    \hline
  \end{tabular}
\end{center}
\noi
\StrWd{3}{4} \\
\Jest www. WydawnictwoWektory.pl \\
\Pow  www.WydawnictwoWektory.pl \\

\vspace{\spaceTwo}





% ##################
\Work{ % Autor i tytuł dzieła
  Warren H.~Carroll \\
  ,,Historia Chrześcijaństwa. Tom~IV: Podział Chrześcijaństwa'',
  \cite{CarrollHistoriaChrzecijanstwaTomIV11} }


\CenterTB{Błędy}
\begin{center}
  \begin{tabular}{|c|c|c|c|c|}
    \hline
    & \multicolumn{2}{c|}{} & & \\
    Strona & \multicolumn{2}{c|}{Wiersz} & Jest
                              & Powinno być \\ \cline{2-3}
    & Od góry & Od dołu & & \\
    \hline
    % & & & & \\
    % & & & & \\
    % & & & & \\
    % & & & & \\
    % & & & & \\
    % & & & & \\
    % & & & & \\
    % & & & & \\
    % & & & & \\
    % & & & & \\
    % & & & & \\
    % & & & & \\
    % & & & & \\
    % & & & & \\
    % & & & & \\
    % & & & & \\
    % & & & & \\
    % & & & & \\
    % & & & & \\
    % & & & & \\
    % & & & & \\
    817 & 12 & & Hilaire. & Hilaire, \\
    817 & &  7 & Henrich. & Henrich, \\
    817 & &  2 & Anthony. & Anthony, \\
    818 & &  4 & E.H. & E.H., \\
    818 & &  1 & Philippe. & Philippe, \\
    819 & &  9 & 1913 & 1913. \\
    820 & &  1 & 1992.. & 1992. \\
    823 &  2 & & \emph{1621--9} & \emph{1621--1629} \\
    823 &  6 & & \emph{1520--21} & \emph{1520--1521} \\
    823 & 17 & & (red.). & (red.), \\
    825 &  5 & & Charles. & Charles, \\
    825 & 11 & & John., & John, \\
    825 & & 15 & \emph{World; Our} & \emph{World: Our} \\
    825 & & 14 & \emph{the~Sea; the~Treasure} & \emph{the~Sea:
                                                The~Treasure} \\
    825 & &  5 & Carlos. & Carlos, \\
    825 & &  2 & Parkman, Francis. & Parkman Francis, \\
    825 & &  1 & Francis. & Francis, \\
    826 &  7 & & \emph{Letters} & \emph{Times} \\
    826 & 12 & & St.~Louis. & St.~Louis \\
    826 & &  8 & \emph{leyasu} & \emph{Ieyasu} \\
    826 & &  4 & R.S. & R.S., \\
    \hline
  \end{tabular}
\end{center}

\vspace{\spaceTwo}





% ##################
\Work{ % Autor i tytuł dzieła
  Warren H.~Carroll, Anne W. Carroll \\
  ,,Historia Chrześcijaństwa. Tom~VI: Kryzys Chrześcijaństwa'',
  \cite{CarrollHistoriaChrzecijanstwaTomVI14} }


\CenterTB{Uwagi}

\start \Str{12} Wcięcia wszystkich akapitów poza pierwszy~są zbyt
duże.

\vspace{\spaceFour}


\start \StrWd{31}{4--2} Zdanie ,,Jestem zobowiązany Jamesowi
H.~Billingtonowi, \emph{Fire In the~Minds~of Man}, wielkiemu
historykowi myśli rewolucyjnej'' po polsku brzmi źle i~jest trochę bez
sensu. Nie wiem jednak jak je~poprawić.

\vspace{\spaceFour}


\start \Str{33} Jest dziwne, że~Lamennais jest tu nazwany ,,wielkim,
choć czasami błądzącym, francuskim duchownym'', skoro sama ta książka
podaje na~43 stronie, że~odrzuci on najpierw wiarę katolicką, potem
zaś chrześcijaństwo. Możliwe, że~ta nielogiczność jest wyniki
pośmiertnej edycji i~uzupełniania tego dzieła oraz pracy tłumacza.

\vspace{\spaceFour}


\start \Str{54} Pisze tu, że~bitwa pod Nowym Orleanem była decydującym
momentem w~Wojnie~1812 roku, powołując~się na książkę Paula Johnsona
\emph{Birth~of the~Modern}. Jednak w~tej pozycji Johnson przedstawia
zupełnie inną wersję wydarzeń. Bitwa ta rozegrała~się już po zawarciu
pokoju w~Londynie \red{Sprawdź miasto}, ale~przed tym jak statek
z~informacją o~tym dotarła do~USA, jej przebieg nie doprowadził jednak
do~kontynuacji działań wojennych. Tym samym, konkluduje Johnson, nie
wpłynęła na zawarcie pokój, ale~bardzo na~jego recepcję. Amerykanie
mogli~się bowiem czuć zwycięzcami wojny jako, że~wygrali ostatnią jej
bitwę.

\vspace{\spaceFour}


\start \Str{63} Możliwe, że~informacje podane na tej i~na następnych
stronach dotyczące Ameryki Łacińskiej są poprawne, jednak napisane są
w~sposób pełen luk i~niejasności. Na~przykład na dole tej strony jest
podane, że~Martin skapitulował przed Monteverdim i~wyjechał
z~Wenezueli, zaraz potem zaś~został zdradzony, aresztowany i~wysłany
przez Bolivara do~Hiszpanii w~zamian za paszport, który umożliwi mu
przyjazd do~Starego Kraju. Wydaje~się mało prawdopodobne, by~Bolivar
mógł aresztować Martina, gdyby ten opuścił już Wenezuelę.

Poza tym, nie ma żadnego jasnego stwierdzenia, że~Bolivar wykorzystał
paszport i~udał~się do~Hiszpanii. Zaraz po~informacji, że~zdobył ten
dokument przenosimy~się do Trujillo dnia 15~czerwca 1813, co może
oznaczać miasto w~Hiszpania, ale~też jedno z~wielu o~takiej nazwie
w~Ameryce Południowej. Pierwszym pewnym miejsce w~którym go potem
widzimy, jest wenezuelska Barcelona.

\vspace{\spaceFour}


\start \StrWd{67}{8} Po~tej linii powinien nastąpić odstęp między
przypisami.

\vspace{\spaceFour}


\start \Str{76} Następcą zmarłego w~1820~roku Jerzego~III
Hanowerskiego był jego najstarszy syn Jerzy~IV Hanowerski panujący
w~latach 1820--1830. Dopiero po~nim panował w~latach 1830--1837
panował Wilhelm~IV, który był młodszym synem Jerzego~III, a~nie jego
dalekim krewnym. Z~tego tej karygodnej pomyłki wszelkie dalsze
odniesienia do~działań tego monarchy mogą być błędnie przypisanymi mu
aktami Jerzego~IV, bądź źle umieszczone w~czasie.

\vspace{\spaceFour}


\start \StrWd{83}{20--17} Zdanie ,,Tak samo było w~przypadku Lenina,
kolejnego wielkiego przywódcy rewolucji, który wychował~się w~pobożnej
chrześcijańskiej rodzinie, a~fakt, że~wedle jego własnego świadectwa,
utracił wiarę w~wieku szesnastu lat, nie miał na~to żadnego wpływu.''
źle brzmi i~bardzo trudno zrozumieć myśl jaką w~tym kontekście miało
przekazywać.

\vspace{\spaceFour}


\start \Str{110} Na~tej stronie jest podane, że~gdy~w~1914 roku
zamordowano arcyksięcia Franciszka Ferdynanda i~jego żonę Zofię,
Franciszkowi Józefowi wyrwał~się raz jedyny okrzyk ,,Nie oszczędzono
mi niczego!'', podczas gdy na~stronie~115 jest napisane, iż~wykrzyknął
on ,,Nie oszczędzono mi niczego na~tej ziemni'' w~momencie,
gdy~dowiedział~się o~zamordowaniu swojej żony Elżbiety. Te~dwa
fragmenty zdają~się sobie przeczyć.

\vspace{\spaceFour}


\start \Str{125} W~drugim paragrafie na~tej stronie jest trochę
zamieszani. Na~początku jest mowa o~zebraniu 87 osób szwajcarskim
Vevey. Na~samym jego końcu jest mowa o~głosowaniu w~kortezach i~ilości
głosów jaka tam padła, co~nie ma chyba nic wspólnego z~tym zebraniem
i~ilością osób która na nim była, nie~pamiętam zaś aby w~tej książce
była podana ilość osób zasiadających w~kortezach.

\vspace{\spaceFour}


\start \StrWd{126}{8} Nie wiem czemu w~tej linii umieszczono słowa
\emph{Dios! Patria! Fueros! Rey!}

\vspace{\spaceFour}


\start \Str{135} Fragment utworu poety Grillparzera o~marszałku
Radetzkim jest tu cytowany z~innego źródła niż na~następnej stronie.
Nie jest to żaden błąd, jedynie trochę to dziwne.

\vspace{\spaceFour}


\start \Str{145} Dwa ostatnie paragrafy nie~mają wcięcia w~tekście.

\vspace{\spaceFour}


\start \Str{147} Stwierdzenie, że~to święty Piotr ustanowił papiestwo
i~, ten błąd jest szczególnie karygodny, Kościół jest sprzeczne
z~wiarą katolicką. Zapewne jest to herezja, lecz nie jestem na tyle
kompetentny by~stwierdzić to na 100\%. Jeśli jest to herezja, to
wątpię by obarczała sumienie Carrolla, który zapewne po prostu
popełnił głupi błąd pisząc te słowa.

\vspace{\spaceFour}


\start \Str{151} Przynajmniej w~mojej opinii na~tej stronie panuje
pewne zamieszanie. Nie potrafię na~przykład z~całą pewnością
stwierdzić, które z~wydarzeń opisanych w~ostatnim paragrafie
odnoszą~się do~pierwszego synodu, a~które do drugiego.

\vspace{\spaceFour}


\start \StrWd{165}{14--12} Sens zdania ,,Wielu opuszczało ojczyznę,
wypływając do~USA z~niewielkich portów, a~ich nazwiska przetrwały
tylko w~lokalnej tradycji.'' jest następujący. Pamięć o~tym, kto
wówczas wypłynął do~Stanów Zjednoczonych zachowała~się w~lokalnej
tradycji ustnej, ale~nie w~dokumentach z~tamtej epoki. W~tym sensie
ich nazwiska nie przetrwały w~źródłach, nie należy jednak przez to
rozumieć, że~ich nazwiska zniknęły z~użycia, co taka forma tego zdania
może sugerować.

\vspace{\spaceFour}


\start \Str{173} Mam problem ze zrozumieniem opisanych tu powodów
wybuchu wojny francusko\dywiz pruskiej. Dlaczego niby informacja
o~tym, że~Niemcy obrażają Francuzów wysłana do~króla Prus Wilhelma
miała spowodować wypowiedzenie wojny przez Napoleona~III.

\vspace{\spaceFour}


\start \Str{218} Na~dole strony pozostawiono puste miejsce, które
powinien zajmować tekst z~następnej strony.

\vspace{\spaceFour}


\start \StrWd{225}{3} Po tej linii następuje za~duży odstęp.

\vspace{\spaceFour}


\start \Str{264} Dwa pierwsze paragrafy są źle sformatowane.

\vspace{\spaceFour}


\start \Str{274} Na~dole strony pozostawiono puste miejsce, które
powinien zajmować tekst z~następnej strony.

\vspace{\spaceFour}


\start \Str{277} Należy sprawdzić, czy w~czasie Powstania Tajpingów
nie zginęło na~polach bitew więcej osób, niż podczas I~Wojny
Światowej. Uwaga którą tu poczynił Carroll\footnote{Myślę, że~Anne
  W.~Carroll zgodziłaby~się na~przyznanie autorstwa jej mężowi
  Warrenowi.}, należy mieć na uwadze czytając to~co pisze
on~o~I~Wojnie Światowej na~stronach 867 i~873.

\vspace{\spaceFour}


\start \StrWd{299}{1} Czcionka w~tej linii jest za~duża.

\vspace{\spaceFour}


\start \StrWd{305}{4} Imię ojca Rasputina Efima, na~str.~313 jest
pisane ,,Jefim''.

\vspace{\spaceFour}


\start \StrWd{352}{1} Czcionka w~tej linii jest za~duża.

\vspace{\spaceFour}


\start \Str{355} W~pierwszym paragrafie jest mowa o~głosowaniu które
zakończyło się wynikiem siedem do~pięciu, później zaś, że~decyzja
o~pokoju z~Niemcami przeszła stosunkiem siedem do~czterech. Najpewniej
w~obu przypadkach mowa jest o~tym samym głosowaniu i~jeden z~podanych
wyników jest błędny.

\vspace{\spaceFour}


\start \Str{383} Jeśli niczego nie przeoczyłem, to w~tym miejscu
ostatni raz jest mowa o~Denikinie i~jego armii, gdy wycofują~się
na~Kubań i~Krym. Nie dowiadujemy~się tym samym jakie były ich dalsze
losy.

\vspace{\spaceFour}


\start \Str{397} Ponieważ Polska, zapewne tak samo, jak kraje
nadbałtyckie, nie istniała w~1914~r., jest nieprawdopodobne, by
w~memorandum Erzberga była mowa o~nich jako o~sąsiadujących
z~Niemcami. Należy~się domyślać, że~Erzberg chciał włączenia
wszystkich ziem które można było uznać za w~jakimś sensie polskie,
analogicznie dla~państw nadbałtyckich, do~Cesarskich Niemiec po
wygranej wojnie.

\vspace{\spaceFour}


\start \StrWg{416}{22} Po tej linii powinien być większy odstęp.

\vspace{\spaceFour}


\start \StrWd{432}{8} Na~podstawie wcześniejszej części książki nie
jestem w~stanie powiedzieć o~co chodziło w~sprawie nadużyć w~Gruzji.

\vspace{\spaceFour}


\start \Str{435} Na~dole strony pozostawiono puste miejsce, które
powinien zajmować tekst z~następnej strony.

\vspace{\spaceFour}


\start \StrWd{437}{8} Wydaje mi~się, że~spotkałem~się z~wersją,
iż~Trocki został zabity ciosem czekanem. Należy to sprawdzić jeszcze
w~jakiejś innej pracy.

\vspace{\spaceFour}


\start \StrWg{441}{1--2} Szacunki Carrollów, że~w~Chinach żyła jedna
trzecia ludności świata, budzą pewne moje wątpliwości. Po~pierwsze
należałoby ustalić o~jakim okresie czasu mowa, po~drugie należałoby
sprawdzić, jak rzeczywiście przedstawiał~się stosunek ludności Chin do
ludności świata.

\vspace{\spaceFour}


\start \StrWg{445}{17} Deng Xiaoping żył w~latach 1904--1997, zaś za
moment przejęcia jego władzy po~Mao Zedongu, który zmarł w~1976 roku,
należy chyba przyjąć rok~1978. Xiaoping miał więc wtedy nie
dziewięćdziesiąt lecz siedemdziesiąt cztery lata.

\vspace{\spaceFour}


\start \StrWg{451}{8} Tu~można powtórzyć wątpliwości odnośnie podanej
ludności~Chin i~jej udziału w~ludności świata, które~są w~komentarzu
do~strony~441.

\vspace{\spaceFour}


\start \StrWd{451}{6} Ten wiersz jest źle wcięty.

\vspace{\spaceFour}


\start \Str{456} Na~dole strony pozostawiono puste miejsce, które
powinien chyba zajmować tekst z~następnej strony. Choć w~tym wypadku
możliwe jest, że~obecny wybór jest lepszy.

\vspace{\spaceFour}


\start \StrWd{456}{11} Znak ,,*'' jest w~tej linii za mały.

\vspace{\spaceFour}


\start \StrWd{466}{18} Ponieważ ten cytata zaczyna~się z~małej litery,
do~tego zaraz następuje znak~,,\ld'', co sugeruje, że~ten cytat został
błędnie przytoczony. Jednak nie wiem jak go~poprawić.

\vspace{\spaceFour}


\start \StrWd{466}{1} To~odwołanie bibliograficzne jest niedokończone.

\vspace{\spaceFour}


\start \StrWd{478}{5--6} W~mojej opinii te~dwie linie~są źle sformatowane.

\vspace{\spaceFour}


\start \StrWd{867}{17} Linia jest źle zedytowana. Drugie zdanie w~tej
linii jest początkiem następnej pozycji w~bibliografii, powinna więc
być zgodnie z~tym sformatowana.

\vspace{\spaceFour}



\newpage
\CenterTB{Błędy}
\begin{center}
  \begin{tabular}{|c|c|c|c|c|}
    \hline
    & \multicolumn{2}{c|}{} & & \\
    Strona & \multicolumn{2}{c|}{Wiersz} & Jest
                              & Powinno być \\ \cline{2-3}
    & Od góry & Od dołu & & \\
    \hline
    7   & &  4 & wszystko$^{ * }$ & wszystko \\
    7   & &  3 & Rekonkwiście$^{ * }$ & Rekonkwiście \\
    23  & & 10 & \emph{Vhutch} & \emph{Church} \\
    24  & & 25 & Zbawiciela$^{ *^{ * } }$ & Zbawiciela$^{ ** }$ \\
    25  & &  9 & 1919 & 1819 \\
    32  & 11 & & dostosowania & do~stosowania \\
    50  & & 12 & za~panowania & rozpoczęta za~panowania \\
    55  & & 12 & piętnstu & piętnastu \\
    55  & &  7 & interesy & interesy Południa \\
    67  & 17 & & bezbożności'')$^{ 31 }$ & bezbożności''$^{ 31 }$) \\
    67  &  8 & & północy & południa \\
    68  & 21 & & siom & siłom \\
    85  &  1 & & Herald'' Tribune'' & Herald'' \\
    96  & &  2 & W.H. Warren & W.H. Carroll \\
    97  & &  7 & ,,tak uważamy'' & ,,Tak uważamy'' \\
    104 & & 17 & i~związku & i~w~związku \\
    104 & &  2 & W.H.~Warren & W.H.~Carroll \\
    105 & & 11 & Counter-Revolution & Counter-Revolution'' \\
    105 & &  5 & W.H.~Warren & W.H.~Carroll \\
    116 & & 15 & stał~się był & stał~się \\
    117 & & 23 & wyd.3,Boston & wyd.~3, Boston \\
    121 &  3 & & tonizowały & uspokajały \\
    126 & &  4 & Pampelunie. & Pampelunie). \\
    127 & 10 & & aż~przez & potem aż~przez \\
    135 & & 12 & doskonalej & doskonałej \\
    137 &  6 & & go & je \\
    139 & &  5 & \emph{1833} & \emph{1883} \\
    141 & &  8 & tom~VI, rozdział~XIV & rozdział~VIII, \\
    141 & &  7 & P\emph{olitical} & \emph{Political} \\
    141 & &  5 & rozdział zatytułowany & rozdział~II, \\
    150 &  5 & & dogmatach; & dogmatach, \\
    151 & &  8 & torturom... & torturom. \\
    151 & &  7 & miasta.. & miasta. \\
    152 & 22 & & i~i & i \\
    165 & &  7 & 2003) & 2003 \\
    170 & 16 & & potomek & bratanek \\
    170 & & 10 & potomka, ,,F\"{u}hrera & potomka ,,F\"{u}hrera \\
    \hline
  \end{tabular}

  \begin{tabular}{|c|c|c|c|c|}
    \hline
    & \multicolumn{2}{c|}{} & & \\
    Strona & \multicolumn{2}{c|}{Wiersz} & Jest
                              & Powinno być \\ \cline{2-3}
    & Od góry & Od dołu & & \\
    \hline
    171 & &  7 & skrajnym wręcz & wręcz skrajnym \\
    177 &  3 & & ,,byliście & ,,Byliście \\
    187 & 21 & & roku~Na & roku. Na \\
    190 & &  1 & Karl Marx & \emph{Karl Marx} \\
    194 & 12 & & etc.. & etc. \\
    194 & &  8 & spikerze & mówcy \\
    197 & &  7 & 1987) & 1987 \\
    198 & 16 & & wojny; & wojny \\
    199 & &  2 & 210 & 210. \\
    208 &  7 & & jest & jest natomiast \\
    208 & &  5 & wschodni, wschodni & zachodni, wschodni \\
    210 & &  6 & Pratt,, & Pratt, \\
    210 & &  5 & Carrol & Carroll \\
    223 & & 20 & dna & dnia \\
    228 & &  4 & 2005) & 2005 \\
    229 & & 14 & \emph{Westrn} & \emph{Western} \\
    233 & & 11 & piaty & piąty \\
    235 & 16 & & rzecz & Rzecz \\
    235 & & 15 & Uranu & Urana \\
    237 & &  1 & 1954) & 1954 \\
    239 & 12 & & wyrazili & nie~wyrazili \\
    239 & 15 & & z~zatem & a~zatem \\
    243 &  5 & & uczony; & uczony. \\
    243 &  5 & & roku1743 & roku 1743 \\
    248 & &  1 & 2008) & 2008 \\
    265 & & 13 & si & się \\
    267 & &  3 & (1944 ) & (1944) \\
    270 & &  1 & \emph{s.} & s. \\
    273 & &  3 & 1944 & 1994 \\
    294 &  7 & & % ,,
                 światu''... & światu... \\
    299 & & 11 & Kołłnotaj & Kołłontaj \\
    299 & &  1 & 1988) & 1988 \\
    303 & &  4 & % ,,
                 \emph{opończa}'' & \emph{opończa} \\
    308 & & 19 & niszczysz & Niszczysz \\
    309 & 22 & & warstw & wszystkich warstw \\
    317 & &  3 & \emph{Kerensky; the} & \emph{Kerensky: The} \\
    320 & &  3 & Habsburg & \emph{Habsburg} \\
    324 & &  8 & eserowcow & eserowców \\
    \hline
  \end{tabular}

  \begin{tabular}{|c|c|c|c|c|}
    \hline
    & \multicolumn{2}{c|}{} & & \\
    Strona & \multicolumn{2}{c|}{Wiersz} & Jest
                              & Powinno być \\ \cline{2-3}
    & Od góry & Od dołu & & \\
    \hline
    327 &  7 & & w~coraz & ludzie w~coraz \\
    330 & &  1 & \emph{Wtnesses} & \emph{Witnesses} \\
    332 & & & Piotrogrodu,. & Piotrogrodu. \\ % Popraw tą linię
    338 & 22 & & roboty!, & roboty! \\
    349 & &  1 & \emph{s.} & s. \\
    351 &  4 & & 1917--1921 & 1914--1922 \\
    365 & &  5 & miasta ; & miasta; \\
    373 & 19 & & rok & rok. \\
    379 & &  2 & 1989) & 1989 \\
    380 &  7 & & zlej & złej \\
    380 & &  6 & destruktywna & destruktywną \\
    381 & &  1 & 1951) & 1951 \\
    385 & 10 & & dopływem & odpływem \\
    387 &  4 & & 1915--1922 & 1914--1922 \\
    387 & &  2 & 1989) & 1989 \\
    392 & &  3 & \emph{s.} & s. \\
    393 & &  6 & \emph{s.} & s. \\
    393 & &  2 & \emph{s.} & s. \\
    394 &  7 & & z & z~dala \\
    394 & &  1 & \emph{XV} , & \emph{XV}, \\
    396 & & 19 & Leonowi XII & Leonowi XIII \\
    408 & &  2 & London1971 & London 1971 \\
    409 & &  1 & \emph{Glory; Poland} & \emph{Glory: Poland} \\
    421 & & 17 & terroru\ld. & terroru\ld \\
    424 & &  5 & Radzieckiej.Trzon & Radzieckiej. Trzon \\
    431 & &  2 & \emph{s.} & s. \\
    432 & &  1 & \emph{s.} & s. \\
    433 & &  2 & \emph{s.} & s. \\
    435 &  2 & & krajem.. & krajem. \\
    436 & &  1 & \emph{s.} & s. \\
    442 &  6 & & imperium & imperium Czang \\
    443 &  7 & & doobra & dobra \\
    446 & 10 & & Baun & Braun \\
    448 & 15 & & Hunan Jiangxi. & Huan i~Jangxi \\
    457 & &  3 & \emph{war} & \emph{War} \\
    457 & &  2 & \emph{war} & \emph{War} \\
    459 & &  2 & \emph{s.} & s. \\
    460 & &  2 & \emph{s.} & s. \\
    \hline
  \end{tabular}

  \begin{tabular}{|c|c|c|c|c|}
    \hline
    & \multicolumn{2}{c|}{} & & \\
    Strona & \multicolumn{2}{c|}{Wiersz} & Jest
                              & Powinno być \\ \cline{2-3}
    & Od góry & Od dołu & & \\
    \hline
    461 & & 11 & \emph{s.} & s. \\
    462 & &  4 & \emph{Pro; Modern} & \emph{Pro. Modern} \\
    463 & &  7 & \emph{s.} & s. \\
    464 & &  8 & \emph{s.} & s. \\
    464 & &  3 & \emph{s.} & s. \\
    465 & &  9 & ,,Viva Cristo Rey'' & \emph{Viva Cristo Rey} \\
    465 & &  8 & 194. 199. & 194, 199. \\
    468 &  6 & & \emph{Altars; Baltimore's} & \emph{Altars. Baltimore's} \\
    473 & 14 & & klepsydrze & klepsydrze. \\
    479 & &  6 & \emph{s.} & s. \\
    % & & & & \\
    % & & & & \\
    % & & & & \\
    487 & & 14 & John a.~Ryan & John A.~Ryan \\
    % & & & & \\
    % & & & & \\
    \hline
  \end{tabular}

  \begin{tabular}{|c|c|c|c|c|}
    \hline
    & \multicolumn{2}{c|}{} & & \\
    Strona & \multicolumn{2}{c|}{Wiersz} & Jest
                              & Powinno być \\ \cline{2-3}
    & Od góry & Od dołu & & \\
    \hline
    553 & &  3 & \emph{1939--1940} , & \emph{1939--1940}, \\
    575 & &  3 & 1997) & 1997 \\
    % & & & & \\
    % & & & & \\
    858 &  5 & & Pio Non (bł.~Pius~IX): & \emph{Pio Non (bł.~Pius~IX):} \\
    858 & 19 & & \emph{ofCatholic} & \emph{of Catholic} \\
    858 & 19 & & \emph{History,}(St.~Louis & \emph{History} (St.~Louis \\
    % Sprawdź czy poprzedni wers jest poprawny
    859 & 13 & & portugalskiej & portugalskiej. \\
    860 &  5 & & DuffDavid. & Duff David \\
    861 &  7 & & państwa.. & państwa. \\
    861 & 18 & & wyd.. & wyd. \\
    862 &  6 & & S. John Brown & S., \emph{John Brown} \\
    862 & &  2 & Jen. & Jen, \\
    864 &  5 & & York, & York \\
    864 & 21 & & York, & York \\
    866 & &  2 & FDR & \emph{FDR} \\
    867 & 15 & & York, & York \\
    867 & &  2 & York, & York \\
    868 &  3 & & 2004.. & 2004. \\
    868 & 15 & & York, & York \\
    868 & 23 & & York, & York \\
    869 & 24 & & \emph{Denikin} & \emph{Denikin.} \\
    869 & & 12 & wojskowości.. & wojskowości \\
    870 &  8 & & 1937) & 1937). \\
    871 & 13 & & 1939 1961 & 1939, 1961 \\
    871 & 17 & & jedneaj & jednej \\
    873 &  1 & & York, & York \\
    873 &  9 & & York, & York \\
    873 & 10 & & York, & York \\
    873 & &  5 & 1958 1966 & 1958, 1966 \\
    874 & &  6 & Najlepsze I~najbardziej & Najlepsze i~najbardziej \\
    \hline
  \end{tabular}
\end{center}
\noi
\StrWg{103}{5} \\
\Jest mianem krucjaty (\emph{la~cruzada} ) określali \\
\Pow  określali mianem krucjaty (\emph{la~cruzada}) \\
\StrWd{170}{10} \\
\Jest ,,F\"{u}hrera z~Poczdamu'', ojca Fryderyka Wielkiego \\
\Pow ,,F\"{u}hrera z~Poczdamu'', Fryderyka Williama~I, ojca Fryderyka
Wielkiego \\
\StrWd{228}{5} \\
\Jest The~Victory~of Reason: How Christianity Led to Freedom,
Capitalism and~Western Success \\
\Pow \emph{The~Victory~of Reason: How Christianity Led to Freedom,
  Capitalism and~Western Success} \\
\StrWg{234}{18} \\
\Jest zapoczątkowujących teorię indukcji elektromagnetycznej \\
\Pow  które doprowadziły do~powstania teorii indukcji elektromagnetycznej \\
\StrWd{237}{1} \\
\Jest Ford: The~Times, the~Man, and~the~Company \\
\Pow  \emph{Ford: The~Times, the~Man, and~the~Company} \\
\StrWd{246}{9} \\
\Jest Alexander Graham Bell and~the~Passion for~Invention \\
\Pow  \emph{Alexander Graham Bell and~the~Passion for~Invention} \\
\StrWd{299}{2} \\
\Jest Three Who Made a~Revolution \\
\Pow  \emph{Three Who Made a~Revolution} \\
\StrWd{383}{18} \\
\Jest Kołczak \\
\Pow  Kołczak doszedł do wniosku \\


\vspace{\spaceTwo}





% ##################
\Work{ % Autor i tytuł dzieła
  Łukasz Czarnecki \\
  ,,Konstantynopol~626'', \cite{CzarneckiKonstantynopol17} }


\CenterTB{Uwagi}

\start \StrWg{58}{18} Mam wątpliwość czy~wszystkie słowa wyróżnione
tu~kursywą~są częścią cytowanego fragmentu.

% \vspace{\spaceFour}


\CenterTB{Błędy}
\begin{center}
  \begin{tabular}{|c|c|c|c|c|}
    \hline
    & \multicolumn{2}{c|}{} & & \\
    Strona & \multicolumn{2}{c|}{Wiersz} & Jest
                              & Powinno być \\ \cline{2-3}
    & Od góry & Od dołu & & \\
    \hline
    19  &  6 & & w~legła gruzach & legła w~gruzach \\
    19  & &  5 & \emph{TheChronicle} & \emph{The~Chronicle} \\
    22  & & 16 & na~bizantyńską & bizantyńską \\
    32  & &  2 & \emph{wiary, islam} & \emph{wiary. Islam} \\
    68  & &  1 & Dzieje Bizancjum & \emph{Dzieje Bizancjum} \\
    121 & & 13 & przed & przed nim \\
    126 & &  2 & The~Armenian & \emph{The~Armenian} \\
    143 & &  4 & wroga$^{ \textrm{\emph{20}} }$. & wroga$^{ 20 }$. \\
    178 &  5 & & tylko & ma~tylko \\
    202 & & 15 & \emph{wiary, islam} & \emph{wiary. Islam} \\
    203 &  3 & & \emph{history} & \emph{History} \\
    \hline
  \end{tabular}
\end{center}

\vspace{\spaceTwo}





% ##################
\Work{ % Autor i tytuł książki
  Red. E. Guerriero, M. Impagliazzo \\
  ,,Najnowsza historia Kościoła. Katolicy i~kościoły chrześcijańskie
  w~czasie pontyfikatu Jana Pawła II (1978--2005)'',
  \cite{GuerrieroImpagliazzoNajnowszaHistoriaKosciola06} }


% \CenterTB{Uwagi.}

% \start \StrWd{9}{6} Zamieszczony tu komentarz odnośnie słowa

\CenterTB{Błędy}
\begin{center}
  \begin{tabular}{|c|c|c|c|c|}
    \hline
    & \multicolumn{2}{c|}{} & & \\
    Strona & \multicolumn{2}{c|}{Wiersz} & Jest
                              & Powinno być \\ \cline{2-3}
    & Od góry & Od dołu & & \\
    \hline
    6  & 10 & & religia miały & nauka miały \\
    6  & & 10 & do & od \\
    7  & & 11 & dużo & duże \\
    14 & &  3 & zgodne & zgadzające~się \\
    30 & &  4 & Afryki & Ameryki Południowej \\
    51 & & 17 & śś. & św. \\
    63 & 9 & & 1987 & 1986 \\
    \hline
  \end{tabular}
\end{center}

\vspace{\spaceTwo}





% ##################
\Work{ % Autor i tytuł dzieła
  Ks. Bogusław Kumor \\
  ,,Historia Kościoła. Tom~I: Starożytność chrześcijańska'',
  \cite{KumorHistoriaKosciolaTomI03} }


\CenterTB{Błędy}
\begin{center}
  \begin{tabular}{|c|c|c|c|c|}
    \hline
    & \multicolumn{2}{c|}{} & & \\
    Strona & \multicolumn{2}{c|}{Wiersz} & Jest
                              & Powinno być \\ \cline{2-3}
    & Od góry & Od dołu & & \\
    \hline
    14 & 11 & & rzeciwieństwie & przeciwieństwie \\
    14 & 15 & & jednej formy & jedną formę \\
    % & & & & \\
    % & & & & \\
    % & & & & \\
    % & & & & \\
    % & & & & \\
    % & & & & \\
    \hline
  \end{tabular}
\end{center}

\vspace{\spaceTwo}





% ########################################
% \newpage
% \Field{Historia świecka}

% \vspace{\spaceThree}









% ######################################
\newpage
\Field{Dzieje lat 1789--1914} % Nazwa dziedziny

\vspace{\spaceTwo} \vspace{\spaceThree}
% ######################################



% ##################
\Work{ % Autor i tytuł dzieła
  James M. McPherson \\
  ,,Battle Cry~of Freedom. Historia Wojny Secesyjnej'',
  \cite{McPhersonBattleCryOfFreedom16} }


\CenterTB{Uwagi}

\start W~książce angielskie określenie ,,Civil War'' zostało
przełożone jako Wojna Secesyjna, jednak znacznie lepszym tłumaczeniem
jest ,,Wojna Domowa'' i~jej będę w~razie potrzeby używał.

\vspace{\spaceFour}


\start W~języku częściej mówi~się o~,,American Revolution''
niż~,,American War~of Independence'', jednak w~Polsce przyjęło~się
błędnie przekładać oba te określenia jako ,,Wojna o~Niepodległość''.
Należałoby dosłownie tłumaczyć pierwsze określenie jako ,,Amerykańską
Rewolucję'', bądź ,,Rewolucję Amerykańską'' i~tej wersji będę w razie
potrzeby używał.

\vspace{\spaceFour}


\start \StrWd{23}{4} Podane tu~stwierdzenie, że~Amerykanie w~1850~r.
zapełnili imperium nabyte w~1803 od~Napoleona wydaje~się mocno
przesadzone. Zauważmy, że~zgodnie z~tym co pisze dalej McPherson
populacja Stanów Zjednoczonych dopiero w~kilka lat później
prześcignęła Wielką Brytanię, której powierzchnia jest mniejsza niż
dzisiejszego stanu Michigan, i~tylko dwa i~pół razy większa od~tej
stanu Luiziany. Zakładając, że~mowa jest o~liczebności ludności
mieszkającej w~samej Wielkiej Brytanii, nie zaś liczonej razem
z~koloniami, pokazuje to jak małe było wtedy zaludnienie ogromnych
terenów Ameryki Północnej. Ciężko więc twierdzić, że~ziemie te zostały
zapełnione, trafniej byłoby powiedzieć, iż~zostały zaludnione.

% \vspace{\spaceFour}

\CenterTB{Błędy}
\begin{center}
  \begin{tabular}{|c|c|c|c|c|}
    \hline
    & \multicolumn{2}{c|}{} & & \\
    Strona & \multicolumn{2}{c|}{Wiersz} & Jest
                              & Powinno być \\ \cline{2-3}
    & Od góry & Od dołu & & \\
    \hline
    8   & &  2 & rally ,round & rally, round \\
    14  & & 21 & Stowarzyszeń & ,,Stowarzyszeń  % ''
    \\
    17  & &  6 & sprzeczności.. & sprzeczności. \\
    23  & & 11 & niewolnictwo jednak & jednak niewolnictwo \\
    25  & & 13 & \emph{Encclopedia~od} & \emph{Encyclopedia~of} \\
    26  & 17 & & 1960~r. & 1860~r. \\
    30  &  7 & & 1854~r. & w~1854~r. \\
    33  & & 19 & codzienne & ,,codzienne  % ''
    \\
    40  & & 15 & społeczeństwie. & społeczeństwie''. \\
    % & & & & \\
    % & & & & \\
    % & & & & \\
    % & & & & \\
    % & & & & \\
    % & & & & \\
    % & & & & \\
    % & & & & \\
    % & & & & \\
    \hline
  \end{tabular}
\end{center}
\noi
\StrWd{31}{22} \\
\Jest Podczas gdy nie odrzucali tezy \\
\Pow  Nie~odrzucając tezy \\
\StrWg{35}{13} \\
\Jest i~okresami bezrobocia, spowodowanymi recesją \\
\Pow  i~okresów bezrobocia, spowodowanych recesją \\

\vspace{\spaceTwo}





% ########################################
\newpage
\Field{Świat po~1914~r.}

\vspace{\spaceTwo} \vspace{\spaceThree}
% ########################################



% ##################
\Work{ % Autor i tytuł dzieła
  Martin Gilbert \\
  ,,Pierwsza wojna światowa'', \cite{GilbertPierwszaWojnaSwiatowa03} }


\CenterTB{Uwagi}

\start \StrWg{21}{1} Ponieważ książka ta został pierwotnie wydania
w~1994 roku, przynajmniej prawa autorskie Martina Gilberta zostały
wtedy zatwierdzone, a~data powstania ,,Wstępu'' to~20~czerwca
1994~roku, pojawia~się problem. Przytaczana wypowiedź z~wojny
na~terenie Bośni, nie mogła pochodzić z~26 grudnia 1996~roku.
Najpewniej chodzi tu~o~26~grudnia 1993 roku, ale~pewności mieć
nie~mogę.

\vspace{\spaceFour}


\start \StrWg{70}{2} Nie wiem kto popełnił błąd, żołnierz, autor
czy~tłumacz, ale to zdanie o~martwym doboszu jest bez sensu.

\vspace{\spaceFour}


\start \StrWd{70}{5} To zdanie jest na~pewno źle przetłumaczone,
ale~nie~wiem jakie je poprawić.


\CenterTB{Błędy}
\begin{center}
  \begin{tabular}{|c|c|c|c|c|}
    \hline
    & \multicolumn{2}{c|}{} & & \\
    Strona & \multicolumn{2}{c|}{Wiersz} & Jest
                              & Powinno być \\ \cline{2-3}
    & Od góry & Od dołu & & \\
    \hline
    68  & &  2 & Wielka Brytania & Rosja \\
    69  &  2 & & kulturowo & kulturowo'' \\
    69  & 16 & & 1 sierpnia & 12 sierpnia \\
    70  &  8 & & Pułk Feuchtingera, kiedy & Kiedy pułk Feuchtingera \\
    70  & &  4 & Sir Edward Gray, kiedy & Kiedy sir Edward Gray \\
    122 & & 10 & dal & dał \\
    177 & 15 & & ,,Walcie, aż lufy pękną''.
           & <<Walcie, aż lufy pękną>>''. \\
           % & & & & \\
    \hline
  \end{tabular}
\end{center}

\vspace{\spaceTwo}





% ##################
\Work{ % Autor i tytuł dzieła
  Paul Johnson \\
  ,,Historia świata XX wieku, od~Rewolucji Październikowej \\
  do~<<Solidarności>>. Tom~I'',
  \cite{JohnsonHistoriaSwiataXXWiekuTomI09} }


\CenterTB{Uwagi}

\start \Str{17--18} Terminy id, ego, superego nie zostały wprowadzone
przez Freuda, lecz przez jego tłumacza, bądź tłumaczy, na~język
angielski. Sam Freud używał zwykłych słów z~języka niemieckiego:
das~Es, Ich, Uberich. (Powinno to być omówione w książce Burzyńskiej
i~Markowskiego \cite{BM09}).

\vspace{\spaceFour}


\start \Str{56} Książka Keynesa \emph{Ekonomiczne konsekwencje
  pokoju}, nie mogła ukazać~się pod koniec 1917 roku.
Najprawdopodobniej chodzi tu o~koniec roku 1919. Powinno~się tu też
znaleźć obszerniejsze omówienie treści tej książki.

\vspace{\spaceFour}


\start \Str{63} Głosowanie nad traktatem o~którym tu mowa
nie~odbyło~się w~marcu 1919 roku, lecz w~marcu 1920 r.

\vspace{\spaceFour}


\start \Str{72} Wyrażoną tu opinię, że~Polska skorzystała z~obawy
Wielkiej Brytanii przed zalewem bolszewizmu, warto skonfrontować z~tym
co wielokrotnie mówił
\href{https://www.youtube.com/watch?v=yfQ7rpq_irA}{Andrzej Nowak} i~co
opisał w~,,Pierwszej zdradzie zachodu''.

\vspace{\spaceFour}


\start \Str{90} Wydaje~się, że~opisany tu~ciąg przemówień Lenina
i~relacja Krupskiej o~tym jak położył~się spać bez słowa, dotyczą
wydarzeń z~jednego dnia, tego którego Lenin wrócił on do Rosji. Jeśli
to prawda powinno to zostać lepiej zaznaczone w~tekście, w~chwili
obecnej, nie jest to w~pełni jasne.

\vspace{\spaceFour}


\start \Str{189} Opis udziału niemieckich wojskowych w~zawieszeniu
przez Niemcy broni w~I~Wojnie Światowej, powinien być bardziej
wyczerpujący, w~tym momencie jest zbyt zwięzły, aby był jasny.

\vspace{\spaceFour}


\start \Str{189} W~tym miejscu po~raz pierwszy zostaje użyte
określenie ,,Alianci'' na~członków Ententy. Jest to~chyba anachronizm,
którego nie powinno~się stosować jako, że~nazwa ,,Aliantów'' jest
powszechnie przyjęta dla~sojuszu z~II, a~nie z~I, Wojny Światowej.

\vspace{\spaceFour}


\start \Str{223} Drugie zdanie drugiego paragrafu na~tej stronie jest
źle skonstruowane, nie wiem jednak jak je poprawić. Mimo tego, tego
jego sens jest jasny.

\vspace{\spaceFour}


\start \Str{232} Ludendorff spada tu jak z~nieba, aby zostać naczelnym
wodzem w~rządzie Hitlera, powołanym podczas puczu monachijskiego.
Warto byłoby napisać skąd on~się w~ogóle w~tym miejscu wziął.

\vspace{\spaceFour}


\start \Str{237--238} Tekst byłby znacznie bardziej logiczny, gdy
zamiast zdania ,,Poincar\'{e} manifestował arystokratyczną pogardę
dla~wulgarności klasy średniej i~francuskiego braku równowagi
emocjonalnej'', było ,,Poincar\'{e} manifestował arystokratyczną
pogardę dla~wulgarności klasy średniej i~francuski brak równowagi
emocjonalnej''.

\vspace{\spaceFour}


\start \Str{241} Stwierdzenie o~kurczącej~się populacji Francji jest
mało udane, bowiem jak zaraz potem Johnson wskazuje, populacja ta
w~rzeczywistości rosła. Problemem jest to, że~rosła ona bardzo słabo
w~porównaniu z~innymi państwami i~tym samym malał stosunek liczby
mieszkańców Francji do mieszkańców innych krajów w~Europie.

\vspace{\spaceFour}


\start \Str{262} Na tej stronie pojawia~się po~raz pierwszy
sformułowanie ,,teoria spisku'', którą lepiej byłoby zastąpić
przyjętym w~języku polskim terminem ,,teoria spiskowa''. Użycie
takiego słownictwa wynika zapewne z~tego, że~jeśli dobrze rozumiem,
książkę przetłumaczono jeszcze w~latach 80 XX~w., kiedy nie było
jeszcze w~języku polskim ustalonej nazwy na~to~zjawisko.

\vspace{\spaceFour}


\start \Str{288}

\vspace{\spaceFour}


\start \Str{304} W~2016~r. byłem na wykładzie na~temat chrześcijaństwa
w~Japonii w~XVI~wieku. Choć sam prowadzący przyznawał, że~w~tej
historii pewne kluczowe punkty są do~dziś niezrozumiałe, to
jednocześnie w~świetle wszystkich rzeczy o~jakich mówił, stwierdzenie,
że~chrześcijaństwo zostało odrzucone w~skutek kłótni misjonarzy, jest
gigantycznym uproszczeniem, a~może nawet wypaczeniem, historii.

\vspace{\spaceFour}


\start \Str{303} Czytając ten fragment odniosłem wrażenie, że~cesarz
Meiji był stary człowiekiem, gdy sprawował swą władzę,
w~rzeczywistości jednak objął formalnie panowanie, gdy miał 15 lat,
zmarł zaś w~wieku 60 lat. Jego następca cesarz Yoshihito urodził~się,
gdy miał on 27 lat, więc można wykluczyć wpływ wieku Meiji na stan
zdrowia jego następcy, co ten fragment mógł sugerować.

\vspace{\spaceFour}


\start \Str{305}

\vspace{\spaceFour}


\start \Str{310} Rok 1944 jako data przystąpienia Japonii do II~Wojny
Światowej jest błędny, od 1937~r. prowadziła już drugą wojnę
chińsko\dywiz japońską, zaś w~grudniu 1941 roku dokonała ataku na
Stany Zjednoczone. Jest to zapewne kolejna w~tej książce literówka,
nie wiem jednak jak~ją poprawić.

\vspace{\spaceFour}


\start \Str{345} Brak numeru strony w~prawy górnym rogu.

\vspace{\spaceFour}


\start \Str{345} Opisane tu wydarzenia, okupacja Korei przez
Japończyków i~ich reakcja na~sytuację w~Chinach powinny być
przedstawione szerzej. W~obecnej chwili przez swoją zwięzłość jest to
dosyć niejasne i~chaotyczne.

\vspace{\spaceFour}


\start \Str{454} Nie rozumiem dlaczego Stalin chował za~plecy prawą
rękę, skoro jego lewa była uszkodzona. Czy to dlatego, że~nie był
w~stanie schować lewej ręki za~plecami?

\vspace{\spaceFour}


\start \Str{457} W~tekście brak odwołania do~przypisu 11.

\vspace{\spaceFour}


\start \Str{459}

\vspace{\spaceFour}


\start \Str{465}

\vspace{\spaceFour}


\start \Str{477--478}

\newpage
\CenterTB{Błędy}
\begin{center}
  \begin{tabular}{|c|c|c|c|c|}
    \hline
    & \multicolumn{2}{c|}{} & & \\
    Strona & \multicolumn{2}{c|}{Wiersz} & Jest
                              & Powinno być \\ \cline{2-3}
    & Od góry & Od dołu & & \\
    \hline
    15  & 17 & & Mendla & prac Mendla \\
    28  &  8 & & cywilizacyjne & cywilizowane \\
    36  & 10 & & zastąpić & zaspokoić \\
    51  & & & Jedyny & Jeden \\ % Dokończ.
    54  & 11 & & [dotyczących planu) & [dotyczących planu] \\
    & & & ] & \\
    55  & &  1 & M. Keynes & J. M. Keynes \\
    64  &  2 & & kształt ów & ów kształt \\
    65  & & 14 & późniejszy doradca & doradca \\
    68  &  5 & & szśćdziesiątych & sześćdziesiątych \\
    82  &  6 & & Ghandi: & Ghandi. \\
    89  &  5 & & odbyć & przebiegać \\
    111 & 12 & & Ogó1norosyjski & Ogólnorosyjski \\
    138 & & 10 & socjalrewolucjonistom & socjalrewolucjoniści \\
    142 & &  8 & poszczegó1nymi & poszczególnymi \\
    154 & 11 & & spadl & spadł \\
    155 & & 15 & przemysł kluczowy & kluczowe gałęzie przemysłu \\
    160 & 14 & & ,,wolność pracy'' & <<wolność pracy>>'' \\
    161 & &  5 & upijać'' & upijać''. \\
    185 & & 16 & można & nie~można \\
    190 & 17 & & więc równe & równe \\
    221 &  3 & & był & nie był \\
    243 &  8 & & nie & nie została \\
    248 & &  3 & 1853 & 1853 -- przy. red.] \\
    252 & &  4 & Enqu\'{e}te sur la~monarchie
           & \emph{Enqu\'{e}te sur la~monarchie} \\
    259 & 14 & & Algierczyków & Algierczykom \\
    281 & &  2 & red. & red.] \\
    292 & 15 & & Jesteśmy & ,,Jesteśmy  % ''
    \\
    314 & &  6 & Korupcja & korupcja \\
    336 &  6 & & wiec & więc \\
    351 &  5 & & podejrzanego & ,,podejrzanego  % ''
    \\
    382 & 11 & & wzrosty & wzrosły \\
    392 & 10 & & dały & dawały \\
    393 & &  2 & Steffens,{ }{ }\emph{Individualism}
           & Steffens, \emph{Individualism} \\
    408 & & 13 & problem6w & problemów \\
    423 &  9 & & ulega kwestii & podlega dyskusji \\
    426 & &  7 & zalegle & zaległe \\
    \hline
  \end{tabular}
\end{center}

\begin{center}
  \begin{tabular}{|c|c|c|c|c|}
    \hline
    & \multicolumn{2}{c|}{} & & \\
    Strona & \multicolumn{2}{c|}{Wiersz} & Jest
                              & Powinno być \\ \cline{2-3}
    & Od góry & Od dołu & & \\
    \hline
    427 &  4 & & Ministerstwa. Zdrowia & Ministerstwa Zdrowia \\
    436 &  1 & & wielkim & ,,wielkim  % ''
    \\
    436 & & 13 & to jedynie & jedynie \\
    444 &  5 & & zaspokoić & uspokoić \\
    445 & & 10 & nie żądające & żądające \\
    449 &  9 & & Białym. Domu & Biały Domu \\
    457 &  9 & & Problemy & ,,Problemy  % ''
    \\
    461 & & 10 & miłosierny!$^{ 20 }$ & miłosierny!''$^{ 20 }$. \\
    % & & & & \\
    % & & & & \\
    % & & & & \\
    \hline
  \end{tabular}
\end{center}
\noi \\
\tb{Okładka} \\
\Jest ''Solidarności'' \\
\Pow  ,,Solidarności'' \\
\Str{1} \\
\Jest \tb{Historia świata} \\
\Pow  \tb{Historia świata XX wieku} \\
\Str{3} \\
\Jest \tb{Historia świata} \\
\Pow  \tb{Historia świata XX wieku} \\

\vspace{\spaceTwo}





% ######################################
\newpage
\Field{Świat po~1945~r.}

\vspace{\spaceTwo} \vspace{\spaceThree}
% ######################################



% ##################
\Work{
  Tony Judt \\
  ,,Powojnie. Historia Europy od~roku 1945'', \cite{JudtPowojnie16} }


\CenterTB{Uwagi}

\start \Str{28} Stwierdzenie, że~aż do lat trzydziestych~XIX~w. babcie
hiszpańskie straszyły dzieci Napoleonem, jest trochę niezręczne. Wojny
napoleońskie skończyły~się dopiero w~1815 roku, więc chodzi tu
o~wydarzenia sprzed 25--40 lat, co nie wydaje~się obecnie zbyt długim
czasem, choć możliwe, że~w~XIX wieku taka długa pamięć była czymś
niezwykłym. Jeśli to właśnie Judy chciał przekazać, to można było
to~zdanie sformułować lepiej.

\vspace{\spaceFour}


\start \StrWd{102}{8} Nie potrafię zrozumieć co~w~tym kontekście miało
znaczyć zdanie ,,co~wyjątkowo miało~się okazać nieskuteczne''.

\vspace{\spaceFour}


\start \Str{111} Warto byłoby podać trochę więcej informacji o~zimie
roku 1947, aby~pozwolić czytelnikom poczuć jej siłę. Kilka zadań
podających dokładnie temperaturę panującą wtedy w~Europie i~czas przez
jaki~się utrzymywała, byłoby zupełnie wystarczające.

\vspace{\spaceFour}


\start \StrWd{125}{7} Należałoby podać, w~jakiej walucie~są wyrażone,
przedstawione tu wydatki.

\vspace{\spaceFour}


\start \Str{131} W~ostatnim akapicie na~tej stronie jest mowa
o~frontach ludowych i~narodowych w~taki sposób, że~nie można zrozumieć
o~co tak naprawdę chodzi.

\vspace{\spaceFour}


\start \StrWg{150}{20} Zdanie ,,zbliża~się czas wielkich zawirowań
--~a~tym samym konieczność określenia przez Związek Radziecki
wynikających z~tego korzyści'' nie jest zbyt dobrze skonstruowane
i~trochę niezrozumiałe.

\vspace{\spaceFour}


\start \StrWg{159}{20} Zdanie ,,pozwoli Niemcom gnić, dopóki owoce
niemieckiej urazy i~beznadziei nie wpadną mu do~koszyka'' nie jest
ani~zbyt jasne, ani~nie brzmi zbyt dobrze w~języku polskim.

\vspace{\spaceFour}


\start \StrWg{174}{18} Słowa Milovana Dżilasa warto byłoby opatrzyć
komentarzem. W~tym momencie ich brzmienie jest trochę dziwne, a~sens
niepewny.

\vspace{\spaceFour}


\start \StrWg{197}{8} W~tym wierszu jest mowa o~przeżyciu przez
Brytyjczyków Pierwszej~Wojny Światowej, ale bardzo możliwe, że~jest
to~błąd i~tak naprawdę chodzi o~Drugą.

\vspace{\spaceFour}


\start \Str{204} Przypis konsultanta wydania polskiego, który obecnie
znajduje~się na~końcu zdania w~wierszu 21 od~góry, powinienem
znajdować~się na końcu zdania w~wierszy drugim od~góry.

\vspace{\spaceFour}


\start \StrWg{258}{17} Z~kontekstu ciężko wywnioskować, kim było
,,dwóch lewicowych członków ruchu oporu''. Ten fragment powinien być
poprawiony.

\vspace{\spaceFour}


\start \StrWd{282}{3} Na~końcu książki nie ma odniesienia do przypisu
z~tej linii.

\vspace{\spaceFour}


\start \StrWd{417}{19--18} O~Luckym Luku, czyli na polski Mającym
Szczęście Łukaszu, można stwierdzić wiele, ale~nie to,
że~jest~,,nieszczęsny''. Również uznanie tego komiksu za belgijski,
jest dla mnie kontrowersyjne.

\vspace{\spaceFour}


\start \Str{474} Raymond Aron był zapewne całe życie antykomunistą,
ale~ponieważ w~jednym z~wydań \emph{Opium dla~intelektualistów}
napisał, że~książkę można traktować jako marksistowską krytykę pewnych
zjawisk, jego stosunek do~intelektualnego dziedzictwa marksizmu,
pozostaje sprawą otwartą.

\vspace{\spaceFour}


\start \Str{510} Choć ,,Dziady. Część III'' zostały napisane po
Powstaniu Listopadowym, to jednak jego akcja rozgrywa~się kilka lat
przed tym wydarzeniem i~nie dotyczy losów powstańców, lecz losów ludzi
uciskanych przez władzę cara. Co~zresztą w~1968 roku również brzmiało
bardzo współcześnie. Należy zaznaczyć, że~trzeciej części ,,Dziadów''
nie można sprowadzić tylko do tego wątku, choć jest on jednym
z~najważniejszych.

\vspace{\spaceFour}


\start \StrWg{513}{7} W~1970 roku protesty o~szczególnych
konsekwencjach miały miejsce w~Szczecinie, nie można więc ich
ograniczać tylko do Gdańska.

\vspace{\spaceFour}


\start \StrWg{526}{14} Wydaje mi~się, że rozumiem sens zdania
o~optymistycznym zapatrzeniu w~postindustrialne wyobcowanie
i~bezduszność, ale według mnie nie powinno~się pisać w~taki przewrotny
i~skomplikowany sposób, aby czytelnik nie zgubił~się.

\vspace{\spaceFour}


\start \StrWg{556}{8} Nie rozumiem co miały znaczyć słowa
,,dla~obojętnych skrajów ruchu robotniczego''.

\vspace{\spaceFour}


\start \Str{561} Nie wiem czy mogę~się w~pełni zgodzić ze
stwierdzeniem, że~Foucault był w~głębi duszy racjonalistą. Możliwe,
lecz zapewne była to specyficzna odmiana racjonalizmu.

\vspace{\spaceFour}


\start \StrWd{585}{18} Panków to dzielnica Berlina, gdzie
w~początkowym okresie istnienia NRD~znajdowały się rezydencje władz
tego kraju.

\vspace{\spaceFour}


\start \StrWg{602}{13} Na~końcu książki nie ma odniesienia do przypisu
z~tej linii.

\vspace{\spaceFour}


\start \StrWd{655}{16--14} Zdanie ,,przeczyć fundamentalnemu
powinowactwu demokratycznego państwa opiekuńczego (bez względu na~to,
jak bardzo niewystarczająco) z~kolektywistycznym planem komunizmu''
jest napisane w~niezrozumiały sposób.

\vspace{\spaceFour}


\start \StrWd{681}{4--3} Nie rozumiem co miało dokładnie znaczyć
zdanie ,,Epoka zastoju Leonida Breżniewa (Michaił Gorbaczow)
żywiła~się wieloma złudzeniami''.

\vspace{\spaceFour}

\start \StrWd{681}{2} W~Polsce przyjęła~się pisonia tego nazwiska
,,Potiomkin'' nie jak w~krajach anglojęzycznych ,,Potemkin''.

\vspace{\spaceFour}


\start \StrWd{693}{16} Na~końcu książki nie ma odniesienia do przypisu
z~tej linii.

\vspace{\spaceFour}


\start \Str{694} Bill Clinton miał 46 lat, gdy został prezydentem USA
w~1993 roku, był więc młodszy od Michaił Gorbaczow który miał 54, gdy
został w~1985 roku Sekretarzem Generalnym KPZR. Jednak nie można
twierdzić, że~Gorbaczow był młodszy od każdego prezydenta USA do
Clintona, bowiem najmłodszym prezydentem do roku 2017, jest Theodore
Roosevelt który miał tylko 42 lata, gdy~objął ten urząd w~1901 roku.

\vspace{\spaceFour}


\start \StrWd{703}{18} Słowo ,,niezrównany'' brzmi trochę dziwnie
w~tym kontekście. Może powinno być ,,nierówny''.

\vspace{\spaceFour}


\start \StrWd{718}{2} Zwrot ,,bezwarunkowe niezrozumienie'' jest
dziwny i~ciężki do zrozumienia. Po~angielsku brzmiał zapewne
,,unconditional misunderstanding'', warto się zastanowić nad jego
lepszym tłumaczeniem.

\vspace{\spaceFour}


\start \StrWd{734}{19} NRD pojawia~się w~tym wersie dość
niespodziewanie, może chodziło o~Czechosłowację?

\vspace{\spaceFour}


\start \StrWd{742}{10} Na~końcu książki nie ma odniesienia do przypisu
z~tej linii.

\vspace{\spaceFour}


\start \StrWd{757}{6--5} Zdanie ,,którego rządzący byli
komunistycznymi satrapami, przejęli kontrolę nad tym obszarem'', brzmi
jakoś niezręcznie. Może dałoby~się je sformułować lepiej?

\vspace{\spaceFour}


\start \StrWd{782}{15} Powinno tu być wyjaśnione czym~są prawa
ciągnięcia.

\vspace{\spaceFour}


\start \StrWd{805}{9} Na~końcu książki nie ma odniesienia do przypisu
z~tej linii.

\vspace{\spaceFour}


\start \StrWd{823}{12-10} Zdanie ,,W~kraju było teraz więcej osób
mówiących po~holendersku niż~francusku (w~stosunku trzy do~dwóch),
które produkowały na~głowę mieszkańca i~zarabiał więcej.'' źle brzmi
i~nie od razu zrozumiałe.

\vspace{\spaceFour}


\start \tb{Wkładka~1, str.~6, u~góry.} Dla ułatwienia czytelnikom
orientacji, warto byłoby napisać, o~jaki akt stworzenia tu chodzi.

\vspace{\spaceFour}


\start \tb{Wkładka~1, str.~6, u~dołu.} Cytowane są tu te same słowa
Clementa Attlee co na stronie~127, jednak te dwie wersje nie są
identyczne.


\CenterTB{Błędy}

\nopagebreak

\begin{center}
  \begin{tabular}{|c|c|c|c|c|}
    \hline
    & \multicolumn{2}{c|}{} & & \\
    Strona & \multicolumn{2}{c|}{Wiersz} & Jest
                              & Powinno być \\ \cline{2-3}
    & Od góry & Od dołu & & \\
    \hline
    22  & 16 & & w~nich żyli & żyli w~nich \\
    34  & 10 & & --supermani & --~supermani \\
    35  & & 17 & prądu & braku prądu \\
    35  & & 16 & dość & lecz dość \\
    58  & &  7 & postępującą & postępującej \\
    58  & &  6 & degeneracją & degeneracji \\
    99  & 10 & & to & za~to \\
    127 & 11 & & \emph{metody} & \emph{metody dozwolone} \\
    131 & 13 & & radziecką & bolszewicką \\
    185 & 19 & & być może & może \\
    199 & 11 & & przeregulowanej prawnie & prawnie przeregulowanej \\
    199 & 18 & & miastem & miasto \\
    203 & &  4 & Europy Środkowej & na~Europę Środkową \\
    264 & 13 & & Odstawiliśmy & ,,Odstawiliśmy  % ''
    \\
    356 & 16 & & męża & ojca \\
    380 & 12 & & k~o~n~t~r~rewolucji & k~o~n~t~r~r~e~w~o~l~u~c~j~i \\
    393 & 23 & & zwyczajowo tradycyjnie & zwyczajowo i~tradycyjnie \\
    409 & & 12 & manipulować. & manipulować''. \\
    409 & & 11 & dziesięcioleciu''. & dziesięcioleciu. \\
    430 & &  3 & specjalnością.: & specjalnością: \\
    457 & 15 & & (czy & czy \\
    457 & &  2 & wino. & wino''. \\
    478 & &  9 & dziewięćdziesiątych & sześćdziesiątych \\
    490 & & 14 & je! & je!'' \\
    509 & & 18 & uczeni & uczelni \\
    526 & &  5 & polityka & że~polityka \\
    555 & &  8 & się wydaje & wydaje~się \\
    564 &  9 & & and & i \\
    565 &  8 & & przeglądzie & w~przeglądzie \\
    582 & & 11 & Nemiec & Niemiec \\
    \hline
  \end{tabular}
\end{center}

\begin{center}
  \begin{tabular}{|c|c|c|c|c|}
    \hline
    & \multicolumn{2}{c|}{} & & \\
    Strona & \multicolumn{2}{c|}{Wiersz} & Jest
                              & Powinno być \\ \cline{2-3}
    & Od góry & Od dołu & & \\
    \hline
    596 & &  4 & Andreas) & Andreas \\
    611 &  5 & & 1976 & 1975 \\
    676 &  1 & & zresztą nie była & nie była zresztą \\
    682 & 12 & & zawłaszczenie odśrodkowego & odśrodkowe zawłaszczenie \\
    691 & &  3 & a~nawet & nawet \\
    707 &  2 & & na nowo & nowe \\
    710 & 12 & & pierwszy hotel & hotel \\
    711 & 19 & & był & nie~był \\
    718 &  2 & & działania & pracy \\
    793 &  7 & & przez & wobec \\
    793 & &  8 & międzynarodową... & międzynarodową. \\
    793 & &  3 & nawet & on nawet \\
    798 & &  6 & walczyć & wlec~się \\
    815 & 15 & & \emph{Beatrice Webb (1925)} & Beatrice Webb (1925) \\
    838 &  3 & & w~związku z~tym & w~tym czasie \\
    937 & & 13 & Tabu & Różne tabu \\
    967 & & 11 & góry(Schuman & góry (Schuman \\
    976 &  6 & & praktycznie & w~praktyce \\
    977 & &  7 & wpław & wpłat \\
    \hline
  \end{tabular}
\end{center}
\noi
\StrWd{110}{6} \\
\Jest na~samą perspektywę pokoju \\
\Pow  na~samą myśl o~perspektywie pokoju \\
\StrWg{417}{15} \\
\Jest którymi~się w~nich chwalono \\
\Pow  w~których~się nimi chwalono \\
\StrWg{555}{4} \\
\Jest ,,nowy patriotyzm'' za~granicą \\
\Pow  ,,nowy patriotyzm'' \\
\StrWd{617}{2} \\
\Jest w~hiszpańskich przedsiębiorstwach \\
\Pow  hiszpańskich przedsiębiorstw \\
\StrWd{698}{4} \\
\Jest który rozpada~się w~wartości 37~miliardów rozpadów na~sekundę \\
\Pow  w~którym dochodzi do~37~miliardów rozpadów na~sekundę \\

\vspace{\spaceTwo}





% ######################################
\newpage
\Field{Świat po~1989~r.}

\vspace{\spaceTwo} \vspace{\spaceThree}
% ######################################



% ##################
\Work{
  J. Kofman, W. Roszkowski \\
  ,,Transformacja i postkomunizm'', \cite{KR99} }


\CenterTB{Błędy}

\begin{center}
  \begin{tabular}{|c|c|c|c|c|}
    \hline
    & \multicolumn{2}{c|}{} & & \\
    Strona & \multicolumn{2}{c|}{Wiersz} & Jest
                              & Powinno być \\ \cline{2-3}
    & Od góry & Od dołu & & \\
    \hline
    10 & & 17 & jedynie & jedynej \\
    13 & &  4 & o ekspansji & do ekspansji \\
    16 &  9 & & marntrawstwem & marnotrawstwem \\
    % & & & & \\
    % & & & & \\
    \hline
  \end{tabular}
\end{center}

\vspace{\spaceTwo}





% ##################
\Work{
  Philipp Ther \\
  ,,Nowy ład na starym kontynencie. Historia neoliberalnej \\
  Europy'', \cite{TherNowyLad15} }


\CenterTB{Uwagi}

\start \Str{120} Jest tu mowa, że~Lech Wałęsa był pierwszym
prezydentem demokratycznej polski. Trzeba sprawdzić, czy ten tytuł nie
powinien przypaść gen.~Wojciechowi Jaruzelskiemu.

% \vspace{\spaceFour}


\CenterTB{Błędy} \nopagebreak
\begin{center}
  \begin{tabular}{|c|c|c|c|c|}
    \hline
    & \multicolumn{2}{c|}{} & & \\
    Strona & \multicolumn{2}{c|}{Wiersz} & Jest
                              & Powinno być \\ \cline{2-3}
    & Od góry & Od dołu & & \\
    \hline
    13  & &  5 & Federalnej, & Federalnej. \\
    14  & 16 & & ,,ruskich'' & o~,,ruskich'' \\
    16  &  6 & & o & jako \\
    30  & 12 & & wschodniego & zachodniego \\
    33  &  2 & & \emph{prospects}'' & \emph{prospects} \\
    41  & 13 & & po1917 & po~1917 \\
    43  &  3 & & po1918 & po~1918 \\
    49  & &  2 & \emph{1989},,  % ''
           & \emph{1989}, \\
    50  & & 13 & doprowadziło & nie~doprowadziło \\
    51  & &  1 & 61-65,71 & 61--65, 71 \\
    57  & &  2 & 1985--1988 & \emph{1985--1988} \\
    60  & &  7 & NRD.nCi & NRD. Ci \\
    60  & &  1 & \emph{War},: & \emph{War}, \\
    % & & & & \\
    62  & &  9 & przypieczętowa & przypieczętowano \\
    71  & &  8 & Niemczami~i & Niemcami~niż \\
    80  & & 17 & z~Lewobrzeżną & Lewobrzeżną \\
    85  & & 11 & w & o \\
    88  & &  6 & \tb{\emph{Anders}} & Anders \\
    109 &  1 & & & Tilly'emu \\
    124 & 11 & & 5) & 5). \\
    146 &  1 & & ( ang. & (ang. \\
    % & & & & \\
    % & & & & \\
    % & & & & \\
    % & & & & \\
    \hline
  \end{tabular}
\end{center}
\noi
\StrWd{30}{22} \\
\Jest metropolii''$^{ 14 }$. Natomiast w~regionalnej \\
\Pow  metropolii''$^{ 14 }$, w~regionalnej \\

\vspace{\spaceTwo}





% ############################
\newpage
\Field{Różne dzieła historyczne}

\vspace{\spaceTwo} \vspace{\spaceThree}
% ############################



% ####################
\Work{
  Christopher A. Ferrara \\
  ,,Liberty: The God That Failed'', \cite{Ferrara12} }


\CenterTB{Błędy}

\begin{center}
  \begin{tabular}{|c|c|c|c|c|}
    \hline
    & \multicolumn{2}{c|}{} & & \\
    Strona & \multicolumn{2}{c|}{Wiersz} & Jest
                              & Powinno być \\ \cline{2-3}
    & Od góry & Od dołu & & \\
    \hline
    23  &  3 & & in & in an other \\
    40  & &  1 & \emph{St. Saint} & \emph{Saint} \\
    207 & &  2 & Government & \emph{Government} \\
    227 & 11 & & doing-not & doing--not \\
    % & & & & \\
    % & & & & \\
    % & & & & \\
    \hline
  \end{tabular}
\end{center}

\vspace{\spaceTwo}





% ##################
\newpage

\Work{
  Andrew Gordon \\
  ,,Nowożytna historia Japonii'',
  \cite{GordonNowozytnaHistoriaJaponii10} }


\CenterTB{Uwagi}

\start \StrWg{46}{3} Jest tu podana ilość mieszkańców Tokio w~1720~r.,
ale~do 1868~r. to miasto nosiło nazwę~Edo.

\vspace{\spaceFour}


\start \Str{83} Nie rozumiem na czym polegała dewaluacja złotych monet
przeprowadzona przez \emph{bakufu}, ani czemu wywołało to zwiększenie
podaży pieniądza i~inflację.

\vspace{\spaceFour}


\start \StrWg{85}{10} Nie jest napisane kim jest ten potężny
reformator i~wróg cudzoziemców.

\vspace{\spaceFour}


\start \Str{88} Powinno zostać tu dokładniej opisany zamach na Iiego
oraz jawnie napisane, czy zginął on w~tym zamachu, bądź w~skutek
niego. Z~kontekstu wynika, że~Iiego zginął.

\vspace{\spaceFour}


\start \Str{91}{15} Sens tego zdania miał być chyba następujący. Gdyby
rząd \emph{bakufu} przetrwał, to w~skutek jego reform powstałby system
zbliżony do tego, który stworzyła restauracja Meiji.

\vspace{\spaceFour}


\start \StrWd{134}{13--11} Nie jest wyjaśnione która grupa
z~wymienionych grup odziedziczyła z~okresu Tokugawów zachowania
ksenofobiczne.

\vspace{\spaceFour}


\start \StrWd{424}{11} Zgodnie z~tym co było napisane na~stronie~417
kobieta ta raczej nie fałszowała banknotów, lecz~potwierdzenia
depozytu z~lokalnej agencji kredytowej.

\vspace{\spaceFour}


\start \StrWg{425}{5} W~tym miejscu jest mowa o~sytuacji gospodarczej
Japonii na przełomie lat osiemdziesiątych i~dziewięćdziesiątych XX
wieku, należy więc zwrócić uwagę, że~Unia Europejska istnieje w~sensie
formalnym od~1~listopada 1993~roku. Może~się więc zdarzyć, że~podana
tu nadwyżka handlowa odnosi~się do~czasu, gdy Unia Europejska jeszcze
formalnie nie istniał i~użycie jej nazwy jest formą skrótu myślowego.

\vspace{\spaceFour}


\start \Str{432} Podany tu średni czas czas trwania kadencji premierów
Japonii są błędne bądź problematyczne. Według nich w~latach 1955--1989
było 12 premierów, przyjmując więc, że~pierwszym z~tej dwunastki jest
Hatoyama Ichir\^{o}\footnote{Hatoyama został premierem jeszcze w~1954
  roku, ale~ponieważ urzędował cały 1955 rok, uznałem, że~należy go
  wliczyć do tej listy. Do~obliczeń włączyłem pełen czas jego
  kadencji, co wydaje~się rozsądne, bo~powinno to dać mniej powodów
  do~zamieszania.}, a~ostatnim Takeshita Noboru, średni czas ich
kadencji to 2.9 roku, podczas, gdy mediana to tylko 2.2 roku. Jeśli
zaś rozważymy kwartyl górny $3/4$ to wynosi\footnote{Kwartyl górny
  wybrałem w~ten sposób, że~powyżej niego jest 25\% populacji, jego
  samego zaś umieszczam pośród pozostałych 75\%.} on~3.4. Wszystkie te
liczby są niższe od~podanej tu~średniej~3.7 roku.

Zauważmy jednak, że~gdyby przyjąć tak jak w~książce pisze, że~między
1955 a~1989 rokiem upłynęły 44 lata, jawny błąd, to średnia czasu
urzędowania rzeczywiście wychodzi 3.7 roku. Ja~przyjąłem, że~skoro
$1989 - 1955 = 34$ to taką długość należy przypisać temu okresowi,
oznacza to bowiem tylko kilkumiesięczny błąd w~sumie długości
kadencji.

Co do czasu urzędowania premierów w~latach 1989--2000, to aby uzyskać
liczbę 10 premierów, należy liczyć od~Takeshita Noboru, który zaczął
kadencję jeszcze w~1987~r., do~Mori Yoshir\^{o}, który zakończył ją
w~2001~r. Przyjęcie więc, że~dziesięciu premierów, sprawowało urząd
dwanaście lat, jest po~prostu bardzo niedokładnym postawieniem sprawy,
by~nie powiedzieć niechlujnym.

\CenterTB{Błędy}
\begin{center}
  \begin{tabular}{|c|c|c|c|c|}
    \hline
    & \multicolumn{2}{c|}{} & & \\
    Strona & \multicolumn{2}{c|}{Wiersz} & Jest
                              & Powinno być \\ \cline{2-3}
    & Od góry & Od dołu & & \\
    \hline
    7   & 17 & & 398 & 298 \\
    29  &  5 & & nstąpiły & nastąpiły \\
    \hline
  \end{tabular}

  \begin{tabular}{|c|c|c|c|c|}
    \hline
    & \multicolumn{2}{c|}{} & & \\
    Strona & \multicolumn{2}{c|}{Wiersz} & Jest
                              & Powinno być \\ \cline{2-3}
    & Od góry & Od dołu & & \\
    \hline
    7   & 17 & & 398 & 298 \\
    29  &  5 & & nstąpiły & nastąpiły \\
    50  & 11 & & przepychały & przepychało \\
    50  & 19 & & sześć & sześćdziesiąt \\
    67  &  3 & & Kaitokud\^{o} & nad Kaitokud\^{o} \\
    77  &  4 & & prądów & tych prądów \\
    95  &  6 & & zamach & zamachem \\
    116 & & 12 & 1887 & 1867 \\
    123 & 17 & & ich właśnie & ich \\
    123 & & 10 & to & o~tym to \\
    146 &  4 & & 23~6475 & 236~475 \\
    156 &  1 & & osiemnastowieczni & dziewiętnastowieczni \\
    160 & 12 & & temat & łamach \\
    166 & 10 & & legacji & delegacji \\
    168 & & 5 & 265 & 264 \\
    190 & 11 & & kich & ich \\
    199 & & 7 & ,,kastowości  % ''
           & ,,kastowości'' \\
    220 & &  4 & dawały & dodawały \\
    323 & 22 & & ojczyzny & Japonii \\
    326 & & 21 & główną & jednak główną \\
    368 & &  7 & wstrzymywania & utrzymywania \\
    380 & 25 & & pracy kluby & pracy, kluby \\
    432 & 16 & & czterdzieści cztery & trzydzieści cztery \\
    465 &  8 & & 56 & 256 \\
    465 & & 18 & Tokyo1963 & Tokyo 1963 \\
    \hline
  \end{tabular}
\end{center}
\noi
\StrWd{240}{7} \\
\Jest gospodarczej i~militarnej przewagi \\
\Pow  o~gospodarczą i~militarną przewagę \\

\vspace{\spaceTwo}





% ##################
\Work{
  Paul Johnson \\
  ,,Narodziny nowoczesności'', \cite{JohnsonNarodzinyNowoczesnoci95} }


\CenterTB{Błędy}
\begin{center}
  \begin{tabular}{|c|c|c|c|c|}
    \hline
    & \multicolumn{2}{c|}{} & & \\
    Strona & \multicolumn{2}{c|}{Wiersz} & Jest
                              & Powinno być \\ \cline{2-3}
    & Od góry & Od dołu & & \\
    \hline
    29  &  2 & & cali, członie & cali, o członie \\
    142 &  1 & & Barbaji & Barbajowi \\
    142 & & 14 & w nową operą & z nową operą \\
    345 & 15 & & XIX & XVIII \\
    345 & 18 & & od & na od \\
    409 &  8 & & sposób & nie sposób \\
    % & & & & \\
    % & & & & \\
    % & & & & \\
    \hline
  \end{tabular}
\end{center}

\vspace{\spaceTwo}





% ############################
\newpage
\Field{Biografie}

\vspace{\spaceTwo} \vspace{\spaceThree}
% ############################



% ##################
\Work{
  Sławomir Cenckiewicz \\
  ,,Anna Solidarność'', \cite{CenckiewiczAnnaSolidarnosc10} }


\CenterTB{Błędy}
\begin{center}
  \begin{tabular}{|c|c|c|c|c|}
    \hline
    & \multicolumn{2}{c|}{} & & \\
    Strona & \multicolumn{2}{c|}{Wiersz} & Jest
                              & Powinno być \\ \cline{2-3}
    & Od góry & Od dołu & & \\
    \hline
    21 & 10 & & zajęłam & zajęła \\
    % & & & & \\
    \hline
  \end{tabular}
\end{center}

\vspace{\spaceTwo}





% ##################
\Work{
  Masha Gessen \\
  ,,Putin. Człowiek bez twarzy'',
  \cite{GessenPutinCzlowiekBezTwarzy12} }


\CenterTB{Uwagi}

\start \StrWd{9}{6} Zamieszczony tu komentarz odnośnie słowa
,,lustracja'', które ma wedle niego pochodzić z~greki, jest zapewne
wynikiem niedbałości tłumacza. Polskie słowo ,,lustracja'', pochodzi
najpewniej od słowa ,,lustro'', które wydaje~się w~ogóle nie związane
z~greką. Prawdopodobnie ten fragment został przetłumaczony
mechanicznie, bez~refleksji, że~w~języku polskim, w~przeciwieństwie
do~oryginału, ten związek etymologiczny nie zachodzi.

\vspace{\spaceFour}


\start \Str{75--76} W~przedstawionej tu opowieści jest pewna
niekonsekwencja. Na~75 stronie pisze, że~Putin był w~tłumie
drezdeńczyków nacierających na budynek Stasi, czyli musiał
znajdować~się na zewnątrz. Jednak na~następnej stronie pisze,
że~wyszedł do owego tłumu na zewnątrz, więc musiał znajdować~się
w~środku budynku. Nigdzie nie jest napisane, jak i~dlaczego opuścił
tłum i~wszedł do siedziby Stasi.

\vspace{\spaceFour}


\start \Str{93} Sacharow urodził~się w~1921 r., Gorbaczow zaś w~1931,
w~1989 mięli więc odpowiednio 68 i~58 lat. Nazwanie Gorbaczowa młody
to pewne nadużycie, wynikające zapewne z~kontrastu między schorowany,
bliski śmierci Sacharowem, a~pełnym energii Michaiłem.

\vspace{\spaceFour}


\start \Str{101} \tb{Akapit trzeci.} Powinno tu być jawniej napisane,
że~wracamy do historii Putina.

\vspace{\spaceFour}


\start \tb{Tylna okładka, wiersz 5 (od dołu).} Masha Gessen
urodziła~się w~1967~r. więc w~latach 1981--1991 miła od~14 do~24 lat,
jest więc wysoce nieprawdopodobne, by~w~tym okresie pracowała
w~Stanach Zjednoczonych. Zapewne chodziło o~to, że~wówczas tam
mieszkała.


\CenterTB{Błędy}
\begin{center}
  \begin{tabular}{|c|c|c|c|c|}
    \hline
    & \multicolumn{2}{c|}{} & & \\
    Strona & \multicolumn{2}{c|}{Wiersz} & Jest
                              & Powinno być \\ \cline{2-3}
    & Od góry & Od dołu & & \\
    \hline
    32 &  2 & & zdawali się nie & nie zdawali się \\
    48 &  7 & & przed wyznaczeniem & po wyznaczeniu \\
    59 & & 11 & założyciela & twórcy \\
    63 &  4 & & karierze$^{ 34 }$ & karierze \\
    63 &  6 & & n i c h''. & n i c h''$^{ 34 }$. \\
    63 &  9 & & twarze$^{ 35 }$ & twarze \\
    63 & 10 & & znaczenie''. & znaczenie''$^{ 35 }$. \\
    63 & & 17 & ważnego$^{ 36 }$ & ważnego \\
    63 & & 13 & KGB''. & KGB''$^{ 36 }$. \\
    64 &  2 & & międzyludzkich<<$^{ 37 }$ & międzyludzkich<< \\
    64 &  5 & & międzyludzkich'' & międzyludzkich''$^{ 37 }$ \\
    65 &  4 & & przyjaciółko$^{ 40 }$ & przyjaciółko \\
    78 &  5 & & robić$^{ 68 }$ & \\
    78 &  7 & & błędy?'' & błędy?''$^{ 68 }$ \\
    % & & & & \\
    % & & & & \\
    \hline
  \end{tabular}
\end{center}

\vspace{\spaceTwo}





% ############################
\newpage
\Field{Eseje i~publicystyka}

\vspace{\spaceTwo} \vspace{\spaceThree}
% ############################



% ##################
\Work{ % Autor i tytuł dzieła
  Andrzej Nowak \\
  ,,Strachy i lachy. Przemiany polskiej pamięci 1982-2012'',
  \cite{Nowak12} }


\CenterTB{Uwagi}

\start \Str{47} T.~S.~Eliot jest na tej stronie nazwany ,,wielkim
poetą katolickim'', acz z~tego co wiem do Kościoła nigdy nie
przyszedł, zamiast tego dołączył do jakiegoś wyznania
anglokatolickiego. Zaś użycie przymiotnika ,,wielki'' w~odniesieniu to
tego poety, którego twórczości nie da~się czytać, jest już na~pewno
błędem.

\vspace{\spaceTwo}





% ##################
\Work{ % Autor i tytuł dzieła
  Andrzej Nowak \\
  ,,Intelektualna historia III~RP. Rozmowy z~lat 1990--2012'',
  \cite{NowakIntelektualnaHistoriaIIIRP13} }


\CenterTB{Błędy}
\begin{center}
  \begin{tabular}{|c|c|c|c|c|}
    \hline
    & \multicolumn{2}{c|}{} & & \\
    Strona & \multicolumn{2}{c|}{Wiersz} & Jest
                              & Powinno być \\ \cline{2-3}
    & Od góry & Od dołu & & \\
    \hline
    195 &  6 & & pomocą & pomocy \\
    579 & &  3 & osób. & osób, \\
    580 & & 10 & od & do \\
    665 &  7 & & The National Interest'' & ,,The National Interest'' \\
    % & & & & \\
    % & & & & \\
    \hline
  \end{tabular}
\end{center}
\noi \\
\tb{Przednia okładka, wiersz 14.} \\
\Jest \emph{ImperologicalStudies.APolishPerspective}(2011);
\emph{Czaswalki} \\
\Pow \emph{Imperological Studies. A Polish Perspective} (2011);
\emph{Czas walki} \\
\tb{Przednia okładka, wiersz 10 (od dołu).} \\
\Jest \ldots w~Brnie \\
\Pow  w~Brnie \\

\vspace{\spaceTwo}





% ##################
\Work{ % Autor i tytuł dzieła
  Andrzej Nowak \\
  ,,Historia i~polityka'', \cite{NowakHistoriaIPolityka16} }


\CenterTB{Uwagi}

\start \Str{} Nowak popełni tu pewien błąd pisząc o~grze komputerowej
,,Dzikie Pola'', jest to standardowa stołowa gra RPG i~nie ma nic
wspólnego z~komputerem.


\CenterTB{Błędy}
\begin{center}
  \begin{tabular}{|c|c|c|c|c|}
    \hline
    & \multicolumn{2}{c|}{} & & \\
    Strona & \multicolumn{2}{c|}{Wiersz} & Jest
                              & Powinno być \\ \cline{2-3}
    & Od góry & Od dołu & & \\
    \hline
    6   &  2 & & 448 & 484 \\
    18  & 18 & & ,, kult & ,,kult \\
    % & & & & \\
    % & & & & \\
    % & & & & \\
    % & & & & \\
    99  & &  2 & (1918--1920) & (1918--2008) \\
    133 & & 23 & 1981 & 1918 \\
    135 & &  5 & 361 (przytoczony & 361. Przytoczony \\
    % & & & & \\
    % & & & & \\
    \hline
  \end{tabular}
\end{center}

\vspace{\spaceTwo}





% ####################################################################
% ####################################################################
% Bibliografia
\bibliographystyle{alpha} \bibliography{Bibliography}{}


% ############################

% Koniec dokumentu
\end{document}