% ------------------------------------------------------------------------------------------------------------------
% Basic configuration and packages
% ------------------------------------------------------------------------------------------------------------------
% Package for discovering wrong and outdated usage of LaTeX.
% More information to be found in l2tabu English version.
\RequirePackage[l2tabu, orthodox]{nag}
% Class of LaTeX document: {size of paper, size of font}[document class]
\documentclass[a4paper,11pt]{article}



% ------------------------------------------------------
% Packages not tied to particular normal language
% ------------------------------------------------------
% This package should improved spaces in the text
\usepackage{microtype}
% Add few important symbols, like text Celcius degree
\usepackage{textcomp}



% ------------------------------------------------------
% Polonization of LaTeX document
% ------------------------------------------------------
% Basic polonization of the text
\usepackage[MeX]{polski}
% Switching on UTF-8 encoding
\usepackage[utf8]{inputenc}
% Adding font Latin Modern
\usepackage{lmodern}
% Package is need for fonts Latin Modern
\usepackage[T1]{fontenc}



% ------------------------------------------------------
% Setting margins
% ------------------------------------------------------
\usepackage[a4paper, total={14cm, 25cm}]{geometry}



% ------------------------------------------------------
% Setting vertical spaces in the text
% ------------------------------------------------------
% Setting space between lines
\renewcommand{\baselinestretch}{1.1}

% Setting space between lines in tables
\renewcommand{\arraystretch}{1.4}



% ------------------------------------------------------
% Packages for scientific papers
% ------------------------------------------------------
% Switching off \lll symbol, that I guess is representing letter "Ł"
% It collide with `amsmath' package's command with the same name
\let\lll\undefined
% Basic package from American Mathematical Society (AMS)
\usepackage[intlimits]{amsmath}
% Equations are numbered separately in every section
\numberwithin{equation}{section}

% Other very useful packages from AMS
\usepackage{amsfonts}
\usepackage{amssymb}
\usepackage{amscd}
\usepackage{amsthm}

% Package with better looking calligraphy fonts
\usepackage{calrsfs}

% Package with better looking greek letters
% Example of use: pi -> \uppi
\usepackage{upgreek}
% Improving look of lambda letter
\let\oldlambda\Lambda
\renewcommand{\lambda}{\uplambda}





% ------------------------------------------------------
% BibLaTeX
% ------------------------------------------------------
% Package biblatex, with biber as its backend, allow us to handle
% bibliography entries that use Unicode symbols outside ASCII
\usepackage[
language=polish,
backend=biber,
style=alphabetic,
url=false,
eprint=true,
]{biblatex}

\addbibresource{Grafika-i-animacja-komputerowa-Bibliography.bib}





% ------------------------------------------------------
% Defining new environments (?)
% ------------------------------------------------------
% Defining enviroment "Wniosek"
\newtheorem{corollary}{Wniosek}
\newtheorem{definition}{Definicja}
\newtheorem{theorem}{Twierdzenie}





% ------------------------------------------------------
% Local packages
% ------------------------------------------------------
% Package containing various command useful for working with a text
\usepackage{./Local-packages/general-commands}
% Package containing commands and other code useful for working with
% mathematical text
% \usepackage{math-commands}





% ------------------------------------------------------
% Package "hyperref"
% They advised to put it on the end of preambule
% ------------------------------------------------------
% It allows you to use hyperlinks in the text
\usepackage{hyperref}










% ------------------------------------------------------------------------------------------------------------------
% Title and author of the text
\title{Grafika i~animacja komputerowa \\
  {\Large Błędy i~uwagi}}

\author{Kamil Ziemian}


% \date{}
% ------------------------------------------------------------------------------------------------------------------










% ####################################################################
\begin{document}
% ####################################################################





% ######################################
% Beginning of the document
\maketitle
% ######################################





% % ######################################
% % Table of content
% \tableofcontents
% % ######################################





% ######################################
\section{Rick Parent \textit{Animacja komputerowa. Algorytmy
    i~techniki},
  \parencite{Parent-Animacja-komputerowa-Algorytmy-i-techniki-Pub-2012}}

\label{sec:Parent-Animacja-komputerowa}
% ######################################


% % ##################
% \CenterBoldFont{}

% \vspace{0em}
% % ##################


\noindent

% \VerSpaceFour





\noindent
% \textbf{Str. 5, rysunek 1.6: Fazy kompilacji.} Lepiej narysować ostatnią
% strzałkę.







% ##################
\CenterBoldFont{Uwagi do~konkretnych stron}



\noindent
\Str{2--3} Parent dyskutuje tutaj bardzo ważne z~punktu widzenia teorii
zagadnienie, dlaczego ciąg nieruchomych obrazów jest postrzegany przez
człowieka jako obraz ruchomy? Szczególnie ważne w~tym kontekście jest
zdanie „Ostatnio związek przyczynowy z~mechanizmem (fizjologicznego) trwania
wrażenia wzrokowego został zakwestionowany i~postrzeganie ruchu zostało
powiązane z~(psychologicznym) mechanizmem znanym jako \textit{zjawisko fi};
widoczny ruch jest określany jako \textit{ruch beta}”, po czym Autor cytuje
trzy artykuły. Dwa z~nich to prace Josepha Andersona i~Barbary Anderson
\textit{The~Myth~of Persistance~of Vision} (pl. \textit{Mit trwania wrażenia
  wzrokowego})
\parencite{Anderson-Fisher-The-Myth-of-Persistence-of-Vision-Pub-1978}
i~\textit{The~Myth~of Persistance~of Vison Revisited}
\parencite{Anderson-Anderson-The-Myth-of-Persistence-of-Vision-ETC-Pub-1993}.

Tutaj pojawia~się problem zgodności treści tych artykułów z~tekstem książki.
Andersonowie odrzucają bowiem jawnie, zarówno trwanie wrażenia wzrokowego,
jak i~zjawisko fi jako możliwe wytłumaczenia postrzegania ciągu nieruchomych
obrazów jako ruchomego obrazu. Według nich, cf.~str.~10
\parencite{Anderson-Anderson-The-Myth-of-Persistence-of-Vision-ETC-Pub-1993},
ruch należy powiązać z~zjawiskiem znanym jako \textbf{krótkozasięgowy
  ruch pozorny} (ang. \textit{short-range apperant motion}). W~chwili
publikacji tego artykułu w~1993 roku, badanie zjawiska krótkozasięgowego
pozornego ruchu, wciąż było dalekie od~zakończenia, jednak z~artykułu
Andersonów jasno wynika, że~w~ich opinii teoria filmu musi być powiązana
z~dalszymi badaniami naukowymi nad fizjologią i~psychologią ludzkiego
widzenia.

\VerSpaceFour





\noindent
\Str{3--4} Omawiana tutaj różnica między częstotliwością odświeżania,
a~próbkowania może chyba zostać wyjaśniona w~następujący sposób. Jeśli
najpierw będzie wyświetlany dany obraz przez $0.45$ sekundy, następnie
będzie $0.1$ przerwy w~wyświetlaniu (przykładowo, widoczny będzie
wtedy tylko czarny ekran), a~następnie ten sam obraz będzie wyświetlany
przez kolejne $0.45$ sekundy, to częstotliwość odświeżania będzie wynosiła
dwa obrazy na sekundę. Jednocześnie prędkość próbkowania będzie wynosiła
jeden obraz na sekundę.

\VerSpaceFour



\noindent
\StrWierszeDol{8}{11--12} Czytamy tutaj, że~„Każdy z~planów może poruszać~się
w~sześciu kierunkach (w~lewo, w~prawo, w~górę, w~dół, w~przód i~w~tył)”.
W~tym zadaniu pierwszy raz stykamy~się z~potencjalnym problemem w~tłumaczeniu
tej książki, gdyż użyte tu słowo „kierunek” w~oryginale prawie na~pewno
brzmiało „direction”. Angielskie słowo „direction” odpowiada w~zasadzie
polskiemu pojęciu „unormowany wektor kierunkowy” i~rzeczywiście, kamera
omawiana w~tym fragmencie mogła~się poruszać w~sposób wyznaczony przez
takie sześć wektorów kierunkowych.

Jednocześnie może to kolidować z~znaczeniem słowa „kierunek” w~technicznym
języku polskim, gdzie słowo to oznacza prostą po~której w~danej chwili
odbywa~się ruch\footnote{Dla uproszczenia przyjęliśmy, że~ruch musi~się
  odbywać po krzywej gładkiej, więc taka prosta zawsze istnieje.}. Wobec
tego dla każdego kierunku możliwe są dwa rodzaje ruchu, które odróżniamy
poprzez podanie zwrotu ruchu. Przykładowo jeśli weźmiemy jako kierunek
prostą pionową, to możliwy jest zwrot ruchu do~góry i~w~dół. Przyjmując to
znaczenie słowa „kierunek” powiedzielibyśmy, że~omawiana w~tym przykładzie
kamera może~się poruszać „wzdłuż trzech kierunków”, nie „w~sześciu
kierunkach”.

Jednak by uniknąć dalszych problemów, będziemy od teraz przyjmować, że~słowo
„kierunek” używane w~tej książce ma ten sam sens, co angielski termin
„direction”.

\VerSpaceFour





































% ##################
\newpage

\CenterBoldFont{Błędy}


\begin{center}

  \begin{tabular}{|c|c|c|c|c|}
    \hline
    Strona & \multicolumn{2}{c|}{Wiersz} & Jest
    & Powinno być \\ \cline{2-3}
    & Od góry & Od dołu & & \\
    \hline
    XVIII & \hphantom{0}3 & & Założyłem~się & Założyłem \\
    \hphantom{0}1 & \hphantom{0}8 & & na~\textit{Ulicy} & w~\textit{Ulicy} \\
    \hphantom{0}2 & & \hphantom{0}2 & \textit{Thesaurusa}$^{ \, *) }$
    & \textit{Thesaurusa}$^{ \, *) }$, \\
    10 & 19 & & w~jej & przy \\
    19 & \hphantom{0}3 & & syntezą & syntezy \\
    % & & & & \\
    % & & & & \\
    % & & & & \\
    % & & & & \\
    % & & & & \\
    \hline
  \end{tabular}

\end{center}

\VerSpaceTwo


\noindent
\StrWierszGora{12}{14} \\
\Jest Wyolbrzymianie, urok, dobry rysunek, oraz podążanie za~ruchem
i~zazębienie~się \\
\PowinnoByc \textit{Wyolbrzymianie}, \textit{urok}, \textit{dobry rysunek},
oraz \textit{podążanie za~ruchem i~zazębienie~się} \\
\StrWierszeDol{12}{14} \\
\Jest Wyprzedzenie i~inscenizacja \\
\PowinnoByc \textit{Wyprzedzenie} i~\textit{inscenizacja} \\
\StrWierszeDol{12}{5} \\
\Jest Rysowanie progresywne oraz według klatek kluczowych \\
\PowinnoByc \textit{Rysowanie progresywne oraz według klatek kluczowych} \\
% \textbf{} \\
% \Jest  \\
% \PowinnoByc  \\

% ############################










% ####################################################################
% ####################################################################
% Bibliography

\printbibliography





% ############################
% End of the document

\end{document}
