% ------------------------------------------------------------------------------------------------------------------
% Basic configuration and packages
% ------------------------------------------------------------------------------------------------------------------
% Package for discovering wrong and outdated usage of LaTeX.
% More information to be found in l2tabu English version.
\RequirePackage[l2tabu, orthodox]{nag}
% Class of LaTeX document: {size of paper, size of font}[document class]
\documentclass[a4paper,11pt]{article}



% ------------------------------------------------------
% Packages not tied to particular normal language
% ------------------------------------------------------
% This package should improved spaces in the text
\usepackage{microtype}
% Add few important symbols, like text Celcius degree
\usepackage{textcomp}



% ------------------------------------------------------
% Polonization of LaTeX document
% ------------------------------------------------------
% Basic polonization of the text
\usepackage[MeX]{polski}
% Switching on UTF-8 encoding
\usepackage[utf8]{inputenc}
% Adding font Latin Modern
\usepackage{lmodern}
% Package is need for fonts Latin Modern
\usepackage[T1]{fontenc}



% ------------------------------------------------------
% Setting margins
% ------------------------------------------------------
\usepackage[a4paper, total={14cm, 25cm}]{geometry}



% ------------------------------------------------------
% Setting vertical spaces in the text
% ------------------------------------------------------
% Setting space between lines
\renewcommand{\baselinestretch}{1.1}

% Setting space between lines in tables
\renewcommand{\arraystretch}{1.4}



% ------------------------------------------------------
% Packages for scientific papers
% ------------------------------------------------------
% Switching off \lll symbol, that I guess is representing letter "Ł"
% It collide with `amsmath' package's command with the same name
\let\lll\undefined
% Basic package from American Mathematical Society (AMS)
\usepackage[intlimits]{amsmath}
% Equations are numbered separately in every section
\numberwithin{equation}{section}

% Other very useful packages from AMS
\usepackage{amsfonts}
\usepackage{amssymb}
\usepackage{amscd}
\usepackage{amsthm}

% Package with better looking calligraphy fonts
\usepackage{calrsfs}

% Package with better looking greek letters
% Example of use: pi -> \uppi
\usepackage{upgreek}
% Improving look of lambda letter
\let\oldlambda\Lambda
\renewcommand{\lambda}{\uplambda}




% ------------------------------------------------------
% BibLaTeX
% ------------------------------------------------------
% Package biblatex, with biber as its backend, allow us to handle
% bibliography entries that use Unicode symbols outside ASCII
\usepackage[
language=polish,
backend=biber,
style=alphabetic,
url=false,
eprint=true,
]{biblatex}

\addbibresource{Kompilatory-Bibliography.bib}





% ------------------------------------------------------
% Defining new environments (?)
% ------------------------------------------------------
% Defining enviroment "Wniosek"
\newtheorem{corollary}{Wniosek}
\newtheorem{definition}{Definicja}
\newtheorem{theorem}{Twierdzenie}





% ------------------------------------------------------
% Local packages
% ------------------------------------------------------
% Package containing various command useful for working with a text
\usepackage{./Local-packages/general-commands}
% Package containing commands and other code useful for working with
% mathematical text
% \usepackage{math-commands}





% ------------------------------------------------------
% Package "hyperref"
% They advised to put it on the end of preambule
% ------------------------------------------------------
% It allows you to use hyperlinks in the text
\usepackage{hyperref}










% ------------------------------------------------------------------------------------------------------------------
% Title and author of the text
\title{Kompilatory \\
  {\Large Błędy i~uwagi}}

\author{Kamil Ziemian}


% \date{}
% ------------------------------------------------------------------------------------------------------------------










% ####################################################################
\begin{document}
% ####################################################################





% ######################################
% Beginning of the document
\maketitle
% ######################################





% % ######################################
% % Table of content
% \tableofcontents
% % ######################################





% ######################################
\section{Alfred V.~Aho, Monica S.~Lam, Ravi Sethi,
  Jeffrey D.~Ullam \textit{Kompilatory. Reguły, metody,
    narzędzia},
  \parencite{Aho-Lam-Sethi-Ullman-Kompilatory-Pub-2019}}

\vspace{0em}
% ######################################


% ##################
\CenterBoldFont{Uwagi do rozdziału~1}

\vspace{0em}
% ##################


\noindent
Rozdział ten byłby prostszy do zrozumienia, gdyby wprowadzono tu jawnie
pojęcie translatora. Przez \textbf{translator} będziemy rozumieć program,
który tłumaczy tekst w~języku~A na równoważny tekst w~języku~B. Choć ta
definicja nie jest specjalnie precyzyjna, to na potrzeby tego rozdziału
jest ona zupełnie wystarczająca.

Dysponując nią, możemy stwierdzić, że~zarówno kompilator jak
i~interpreter~są rodzajami translatorów. Wydaje mi~się, że~interpreter można
teraz określić, jako translator, który czytam program linia po linii i~na
podstawie tego produkuje kod maszynowy, który jednak nie jest zapisywany
do żadnego pliku. Choć definicja ta wygląda rozsądnie, to jednak nie jestem
pewien, czy jest wystarczająco poprawna.















% ##################
\CenterBoldFont{Uwagi do~konkretnych stron}


% \noindent
% \Str{1}

% \VerSpaceFour





































% ##################
\newpage

\CenterBoldFont{Błędy}


\begin{center}

  \begin{tabular}{|c|c|c|c|c|}
    \hline
    Strona & \multicolumn{2}{c|}{Wiersz} & Jest
                              & Powinno być \\ \cline{2-3}
    & Od góry & Od dołu & & \\
    \hline
    XXI & \hphantom{0}5 & & szanowane & szacowne \\
    XXI & 14 & & o~wielu rozdziałach & do~wielu rozdziałów \\
    % & & & & \\
    % & & & & \\
    % & & & & \\
    % & & & & \\
    % & & & & \\
    % & & & & \\
    % & & & & \\
    % & & & & \\
    % & & & & \\
    % & & & & \\
    \hline
  \end{tabular}

\end{center}

\VerSpaceTwo


\noindent
\StrWierszGora{XXI}{8} ??? \\
\Jest błędy, a~co najważniejsze, \\
\PowinnoByc błędy oraz, co najważniejsze, \\
\StrWierszeGora{XXX}{17 i~18} \\
\Jest kwestie do rozstrzygnięcia, które \\
\PowinnoByc te zagadnienia, które musi rozstrzygnąć projektant, a~które \\

% ############################










% ####################################################################
% ####################################################################
% Bibliography

\printbibliography





% ############################
% End of the document

\end{document}
