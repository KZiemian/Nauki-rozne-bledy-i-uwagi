% ---------------------------------------------------------------------
% Basic configuration and packages
% ---------------------------------------------------------------------
% Package for discovering wrong and outdated usage of LaTeX.
% More information to be found in l2tabu English version.
\RequirePackage[l2tabu, orthodox]{nag}
% Class of LaTeX document: {size of paper, size of font}[document class]
\documentclass[a4paper,11pt]{article}



% ---------------------------------------
% Packages not tied to particular normal language
% ---------------------------------------
% This package should improved spaces in the text.
\usepackage{microtype}
% Add few important symbols, like text Celcius degree
\usepackage{textcomp}



% ---------------------------------------
% Polonization of LaTeX document
% ---------------------------------------
% Basic polonization of the text
\usepackage[MeX]{polski}
% Switching on UTF-8 encoding
\usepackage[utf8]{inputenc}
% Adding font Latin Modern
\usepackage{lmodern}
% Package is need for fonts Latin Modern
\usepackage[T1]{fontenc}



% ---------------------------------------
% Setting margins
% ---------------------------------------
\usepackage[a4paper, total={14cm, 25cm}]{geometry}



% ---------------------------------------
% Setting vertical spaces in the text
% ---------------------------------------
% Setting space between lines
\renewcommand{\baselinestretch}{1.1}

% Setting space between lines in tables
\renewcommand{\arraystretch}{1.4}



% ---------------------------------------
% Packages for scientific papers
% ---------------------------------------
% Switching off \lll symbol, that I guess is representing letter ``Ł''.
% It collide with `amsmath' package's command with the same name
\let\lll\undefined
% Basic package from American Mathematical Society (AMS)
\usepackage[intlimits]{amsmath}
% Equations are numbered separately in every section.
\numberwithin{equation}{section}





% ------------------------------
% Private packages
% You need to put them in the same directory as .tex file
% ------------------------------
% Contains various command useful for working with a text
\usepackage{latexgeneralcommands}





% ------------------------------
% Package ``hyperref''
% They advised to put it on the end of preambule
% ------------------------------
% It allows you to use hyperlinks in the text
\usepackage{hyperref}










% ------------------------------------------------------------------------------------
% Defining title and author of the text
\title{Systemy operacyjne \\
  {\Large Błędy i~uwagi}}

\author{Kamil Ziemian}


% \date{}
% ------------------------------------------------------------------------------------










% ####################################################################
\begin{document}
% ####################################################################





% ######################################
% Title of the text
\maketitle
% ######################################





% ######################################
\section{Andrew S.~Tanenbaum \textit{Systemy operacyjne},
  \cite{Tannenbaum-Systemy-operacyjne-Wydanie-III-2013}}

\label{sec:Tanenbaum-Systemy-operacyjne}
% ######################################


% ##################
\CenterBoldFont{Uwagi do~konkretnych stron}


\noindent
\StrWierszeGora{31}{16--17} Stwierdzenie „tom ma 1000 stron (więcej niż ta
książka)” odnosi~się zapewne do wydania angielskiego, bo polskie wydanie ma
ponad 1200~stron. Szkoda, że~w~wydaniu polskim nie opatrzono tego tekstu
przypisem.





% ##################
\newpage

\CenterBoldFont{Błędy}


\begin{center}

  \begin{tabular}{|c|c|c|c|c|}
    \hline
    Strona & \multicolumn{2}{c|}{Wiersz} & Jest
    & Powinno być \\ \cline{2-3}
           & Od góry & Od dołu & & \\
    \hline
    39 & & \hphantom{0}2 & częściowych & cząstkowych \\
    44 & 21 & & ciasteczka & bułeczki \\
    % & & & & \\
    % & & & & \\
    % & & & & \\
    % & & & & \\
    % & & & & \\
    % & & & & \\
    \hline
  \end{tabular}

\end{center}

\VerSpaceSix


% ######################################










% ####################################################################
% ####################################################################
% Bibliography

\bibliographystyle{plalpha}

\bibliography{ComputerScienceBooks}{}





% ############################

% End of the document
\end{document}
