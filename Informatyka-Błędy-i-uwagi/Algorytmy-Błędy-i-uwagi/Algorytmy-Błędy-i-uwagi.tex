% ------------------------------------------------------------------------------------------------------------------
% Basic configuration and packages
% ------------------------------------------------------------------------------------------------------------------
% Package for discovering wrong and outdated usage of LaTeX.
% More information to be found in l2tabu English version.
\RequirePackage[l2tabu, orthodox]{nag}
% Class of LaTeX document: {size of paper, size of font}[document class]
\documentclass[a4paper,11pt]{article}



% ------------------------------------------------------
% Packages not tied to particular normal language
% ------------------------------------------------------
% This package should improved spaces in the text
\usepackage{microtype}
% Add few important symbols, like text Celcius degree
\usepackage{textcomp}



% ------------------------------------------------------
% Polonization of LaTeX document
% ------------------------------------------------------
% Basic polonization of the text
\usepackage[MeX]{polski}
% Switching on UTF-8 encoding
\usepackage[utf8]{inputenc}
% Adding font Latin Modern
\usepackage{lmodern}
% Package is need for fonts Latin Modern
\usepackage[T1]{fontenc}



% ------------------------------------------------------
% Setting margins
% ------------------------------------------------------
\usepackage[a4paper, total={14cm, 25cm}]{geometry}



% ------------------------------------------------------
% Setting vertical spaces in the text
% ------------------------------------------------------
% Setting space between lines
\renewcommand{\baselinestretch}{1.1}

% Setting space between lines in tables
\renewcommand{\arraystretch}{1.4}



% ------------------------------------------------------
% Packages for scientific papers
% ------------------------------------------------------
% Switching off \lll symbol, that I guess is representing letter "Ł"
% It collide with `amsmath' package's command with the same name
\let\lll\undefined
% Basic package from American Mathematical Society (AMS)
\usepackage[intlimits]{amsmath}
% Equations are numbered separately in every section
\numberwithin{equation}{section}

% Other very useful packages from AMS
\usepackage{amsfonts}
\usepackage{amssymb}
\usepackage{amscd}
\usepackage{amsthm}

% Package with better looking calligraphy fonts
\usepackage{calrsfs}

% Package with better looking greek letters
% Example of use: pi -> \uppi
\usepackage{upgreek}
% Improving look of lambda letter
\let\oldlambda\Lambda
\renewcommand{\lambda}{\uplambda}




% ------------------------------------------------------
% BibLaTeX
% ------------------------------------------------------
% Package biblatex, with biber as its backend, allow us to handle
% bibliography entries that use Unicode symbols outside ASCII
\usepackage[
language=polish,
backend=biber,
style=alphabetic,
url=false,
eprint=true,
]{biblatex}

\addbibresource{LogikaITeoriaMnogosciBibliography.bib}





% ------------------------------------------------------
% Defining new environments (?)
% ------------------------------------------------------
% Defining enviroment "Wniosek"
\newtheorem{corollary}{Wniosek}
\newtheorem{definition}{Definicja}
\newtheorem{theorem}{Twierdzenie}





% ------------------------------------------------------
% Private packages
% You need to put them in the same directory as .tex file
% ------------------------------------------------------
% Contains various command useful for working with a text
\usepackage{latexgeneralcommands}
% Contains definitions useful for working with mathematical text
\usepackage{mathcommands}





% ------------------------------------------------------
% Package "hyperref"
% They advised to put it on the end of preambule
% ------------------------------------------------------
% It allows you to use hyperlinks in the text
\usepackage{hyperref}










% ------------------------------------------------------------------------------------------------------------------
% Title and author of the text
\title{Algorytmy \\
  {\Large Błędy i~uwagi}}

\author{Kamil Ziemian}


% \date{}
% ------------------------------------------------------------------------------------------------------------------










% ####################################################################
\begin{document}
% ####################################################################





% ######################################
\maketitle  % Tytuł całego tekstu
% ######################################





% ######################################
\section{ % Autor i tytuł dzieła
  Thomas H.~Cormen, Charles E.~Leiserson, Ronald L.~Rivest, Clifford Stein \\
  \textit{Wprowadzenie do~algorytmów},
  \cite{CormenAtAlWprowadzenieDoAlgorytmow2022}}

\vspace{0em}


% ##################
\CenterBoldFont{Uwagi}

\vspace{0em}


\noindent W~nauce o~algorytmach w~jej dzisiejszym stanie ważne miejsce
zajmuje badanie następujące pytanie. Jeśli algorytm pobiera $n \in \Nbb$
danych wejściowych, jak można oszacować od góry i~od dołu ilość operacji
jaką algorytm wykona by uzyskać pożądanych rezultat? W~innym sformułowaniu
zamiast pytać~się ilość operacji, pytamy~się o~oszacowanie czasu jaki zajmie
wykonanie algorytmu, jednak to zagadnienie zwykle jest sprowadzane do tego
samego problemu matematycznego, który różni~się od poprzedniego tylko
wartością pewnych stałych liczbowych.

Prowadzi to w~naturalny sposób do badania funkcji typu
\begin{equation}
  \label{eq:CormenAtAl-WprowadzenieDoAlgorytmow-01}
  f : \Nbb \to \Rbb.
\end{equation}
Ponieważ jednak nie ma sensu mówić, przynajmniej w~normalnych warunkach, ani
o~ujemnej liczbie kroków jakie wykona algorytm, ani o~ujemny czasie jego
trwania, więc możemy~się ograniczyć do badania funkcji o~wartościach
nieujemnych.
\begin{equation}
  \label{eq:CormenAtAl-WprowadzenieDoAlgorytmow-02}
  f : \Nbb \to \Rbb_{ + }, \quad
  \Rbb_{ + } := \{ x \in \Rbb \, | \, x \geq 0 \}.
\end{equation}
Przypadek gdy $f( n_{ 0 } ) = 0$, dla pewnego $n_{ 0 }$, czyli algorytm
w~ogóle nie będzie musiał niczego robić, wydaje~się dziwny, ale~dla
uproszczenia rozważań warto dopuścić występowanie takich funkcji. Opisują
one przypadek, gdy rozwiązanie jest od razu dane, więc nie musimy nic robić
by je uzyskać.

Ze względu na to, że~ilość operacji wykonywanych przez algorytm przy
dowolnej ilości danych wejściowych, też jest liczbą naturalną,
a~przynajmniej piszącemu te słowa nie jest znany
przypadek, gdzie byłoby inaczej, to możemy ograniczyć rozważaną klasę
funkcji. Mianowicie do funkcji, które są postaci
\begin{equation}
  \label{eq:CormenAtAl-WprowadzenieDoAlgorytmow-03}
  f( n ) = C g( n ), \qquad
  C > 0, \quad
  g : \Nbb \to \Nbb.
\end{equation}
W~szczególności sama $f$ może przyjmować tylko wartości naturalne.

Wydaje~się, że~dla autorów tej książki było tak oczywiste, że~rozważane
funkcje są nieujemne, iż~zapomnieli oni napisać jawnie, że~należy przyjmować,
iż~wszystkie rozważane funkcje są nieujemne, chyba że wskazano inaczej. Tym
samym podają wyniki, które są prawdziwe tylko gdy operujemy funkcjami
nieujemnymi, ale~nigdzie nie jest to napisane jawnie.

Przyjrzyjmy~się choćby dodatkowi~A w~tej książce. Na jego początku jest mowa
o~sumach i~szeregach które posiadają wyrazy ujemne, lecz potem prawie bez
wyjątku analizowane są tylko szeregi o~wyrazach dodatnich. Należałoby
sprawdzić, czy~któreś z~podanych w~tym rozdziale twierdzeń nie jest
prawdziwe, tylko dla takich sum oraz szeregów, bo inaczej czytelnik może
zostać wprowadzony w~błąd.

Warte uwagi jest to, że~w~tym dodatku liczby ujemne prawie~się nie
pojawiają, a~nawet gdy~się pojawiają, to może to być rezultatem z~rozkładu
pewnej liczby \textit{dodatniej} na różnicę dwóch liczb. Tak jest na
stronie~1173, gdzie analizowana jest suma teleskopowa o~wyrazie ogólnym
\begin{equation}
  \label{eq:CormenAtAl-WprowadzenieDoAlgorytmow-04}
  a_{ k } = \frac{ 1 }{ k ( k + 1 ) } =
  \frac{ 1 }{ k } - \frac{ 1 }{ k + 1 }.
\end{equation}

Innym założeniem które mogło ujść autorom niezauważenie, jest przyjęcie,
że~dla każdej funkcji $f : \Nbb \to \Rbb_{ + }$ zachodzi
\begin{equation}
  \label{eq:CormenAtAl-WprowadzenieDoAlgorytmow-05}
  \forall \, n \in \Nbb, \quad
  f( n ) \geq 1,
\end{equation}
tym samym za pomocą notacji asymptotycznej wprowadzonej na stronie ??? tej
książki możemy zapisać
\begin{equation}
  \label{eq:CormenAtAl-WprowadzenieDoAlgorytmow-06}
  1 = O\left( f( n ) \right).
\end{equation}
Założenie to jest naturalne, bowiem należy~się spodziewać, iż dla dowolnej
ilości danych wyjściowych, może z~wyjątkiem $n = 1$, każdy algorytm wykona
co najmniej jedną operację. Problem ile zajmie czas wykonanie danego
algorytmu jest bardziej subtelny, bowiem zależy od tego w~jakich jednostkach
mierzymy czas wykonania operacji, lecz można przypuszczać, iż~autorzy
książki i~tutaj przyjęli warunek analogiczny
do~\eqref{eq:CormenAtAl-WprowadzenieDoAlgorytmow-05}. Po prostu dlatego,
że~argument „każdy algorytm wykonuje co najmniej jedną operację”, ma tak
ogromną siłę perswazji.

Łatwo jednak podać funkcję, która nie spełnia warunku
\eqref{eq:CormenAtAl-WprowadzenieDoAlgorytmow-06}. Weźmy
\begin{equation}
  \label{eq:CormenAtAl-WprowadzenieDoAlgorytmow-07}
  f( n ) =
  \begin{cases}
    1, & n = 0, \\
    \frac{ 1 }{ n }, & n \geq 1,
  \end{cases}
\end{equation}
dla której mamy
\begin{equation}
  \label{eq:CormenAtAl-WprowadzenieDoAlgorytmow-08}
  1 \neq O\!\left( \tfrac{ 1 }{ n } \right), \quad
  \frac{ 1 }{ n } = O( 1 ).
\end{equation}
Dla ścisłości rozważań należy więc warunek
\eqref{eq:CormenAtAl-WprowadzenieDoAlgorytmow-05} przyjąć jawnie.

\VerSpaceFour





\noindent
W~tym dziele pojawia~się termin „asymptotycznie dokładne oszacowanie funkcji
$f( n )$”. Wedle tego co zrozumiałem, znalezienie takiego asymptotycznie
dokładnego oszacowania takiej funkcji $g( n )$, że~zachodzi
\begin{equation}
  \label{eq:CormenAtAl-WprowadzenieDoAlgorytmow-09}
  f( n ) = \Theta\big( g( n ) \big).
\end{equation}








% ##################
\CenterBoldFont{Uwagi do~konkretnych stron}


\noindent
\StrWierszeGora{4}{7--8} Zbigniew Rudy uświadomił mi, że jest pewien problem
z~podanym tu stwierdzeniem „algorytm jest ciągiem kroków obliczeniowych
przekształcających dane wejściowe w~wyjściowe”. Chodzi o~to, że~słowo
„ciąg” sugeruje, że~kroki obliczeniowe wykonuje~się jeden po drugim
w~określonej kolejności, co nie jest prawdą dla tak ważnych
dziś\footnote{Piszę to w~2023 roku.} algorytmów równoległych. W~ich przypadku
przynajmniej część obliczeń wykonuje~się w~tym samym czasie w~różnych
jednostkach obliczeniowych. W~szczególności nie jest ustalone, który krok
obliczeń równoległych wykona~się przed którym. Dodatkowo jeśli przy
pierwszym wykonywaniu algorytmu równoległego najpierw wykona~się krok~$A$,
a~potem~$B$, to przy następnym jego wykonywaniu może~się zdarzyć, iż~krok~$B$
wykona~się przed krokiem $A$.

\VerSpaceFour




















\noindent
\Str{1171} Trzeba poważnie przemyśleć co to znaczy, że~w~zależności
\begin{equation}
  \label{eq:CormenAtAl-WprowadzenieDoAlgorytmow-10}
  \sum_{ k = 1 }^{ n } \Theta\big( f( k ) \big) =
  \Theta\left( \sum_{ k = 1 }^{ n } f( k ) \right),
\end{equation}
symbol $\Theta$ po lewej stronie oznacza oszacowanie asymptotyczne
w~zmiennej~$k$, zaś po prawej w~zmiennej $n$. Wydaje mi~się, że~coś jest
z~tym nie w~porządku.

\VerSpaceFour





\noindent
\Str{1175--1176} Przeanalizujmy jeszcze raz dokładnie, błędny dowód fałszywej
zależności $\sum_{ k = 1 }^{ n } = O( n )$, zaczynając od ścisłej rachunków.

Dla $n = 1$ nam
\begin{equation}
  \label{eq:CormenAtAl-WprowadzenieDoAlgorytmow-11}
  \sum_{ k = 1 }^{ 1 } = 1,
\end{equation}
dla $n = 2$ dostajemy
\begin{equation}
  \label{eq:CormenAtAl-WprowadzenieDoAlgorytmow-12}
  \sum_{ k = 1 }^{ 2 } k = \sum_{ k = 1 }^{ 1 } k + 2 = 1 + 2 = 1.5 \cdot 2,
\end{equation}
w~trzecim kroku mamy
\begin{equation}
  \label{eq:CormenAtAl-WprowadzenieDoAlgorytmow-13}
  \sum_{ k = 1 }^{ 3 } k = \sum_{ k = 1 }^{ 2 } k + 3 = 1.5 \cdot 2 + 3 = 2 \cdot 3,
\end{equation}
zaś w~czwartym
\begin{equation}
  \label{eq:CormenAtAl-WprowadzenieDoAlgorytmow-14}
  \sum_{ k = 1 }^{ 4 } k = 2 \cdot 3 + 4 = 2.5 \cdot 4.
\end{equation}

Przyjmijmy teraz, że~mamy stałą $c_{ n }$, taką~że zachodzi
\begin{equation}
  \label{eq:CormenAtAl-WprowadzenieDoAlgorytmow-15}
  \sum_{ k = 1 }^{ n } k = c_{ n } n.
\end{equation}
Stąd dostajemy
\begin{equation}
  \label{eq:CormenAtAl-WprowadzenieDoAlgorytmow-16}
  \sum_{ k = 1 }^{ n + 1 } k = \sum_{ k = 1 }^{ n } k + n + 1 = c_{ n } n + n + 1 =
  \frac{ ( c_{ n } + 1 ) n + 1 }{ n + 1 } ( n + 1 ),
\end{equation}
czyli
\begin{equation}
  \label{eq:CormenAtAl-WprowadzenieDoAlgorytmow-17}
  c_{ n + 1 } = \frac{ ( c_{ n } + 1 ) n + 1 }{ n + 1 },
\end{equation}
przy czym $c_{ 1 } = 1$. Można sprawdzić, że~rozwiązaniem tego problemu,
dobrze znanym ze wzoru na sumę postępu arytmetycznego, jest
$c_{ n } = 0.5 ( n + 1 )$, czyli inaczej mówiąc
\begin{equation}
  \label{eq:CormenAtAl-WprowadzenieDoAlgorytmow-18}
  c_{ n + 1 } = c_{ n } + 0.5,
\end{equation}
co jest zgodne z~wzorami
\eqref{eq:CormenAtAl-WprowadzenieDoAlgorytmow-11}-\eqref{eq:CormenAtAl-WprowadzenieDoAlgorytmow-14}.
Widzimy więc, że~nie istnieje taka liczba $C$ by dla każdego $n$ zachodziło
\begin{equation}
  \label{eq:CormenAtAl-WprowadzenieDoAlgorytmow-19}
  \sum_{ k = 1 }^{ n } k \leq C n.
\end{equation}

Co więc było źródłem błędu w~podanym tu niepoprawnym dowodzie? Otóż w~kroku
indukcyjnym zawsze~dowodzimy naszego twierdzenia $T( n + 1 )$ przy
założeniu, że~prawdziwe jest twierdzenie $T( n )$, dla pewnego
\textit{ustalonego}~$n$. Zaś dla ustalonego $n$ zawsze istnieje liczba
$C \geq 0$, taka~że zachodzi
\begin{equation}
  \label{eq:CormenAtAl-WprowadzenieDoAlgorytmow-20}
  \sum_{ k = 1 }^{ n } k \leq C n.
\end{equation}
Natomiast notacja asymptotyczna, jak sama nazwa wskazuje, mówi o~zachowaniu
asymptotycznym funkcji, czyli, że~funkcja spełnia pewną własność dla
wszystkich $n \geq n_{ 0 }$, gdzie $n_{ 0 }$ jest pewną ustaloną liczbą. Zatem
gdy dowodzimy kroku indukcyjnego możemy z~twierdzenia $T( n )$ przy
ustalonym $n$ wywnioskować, że~nierówność typu
\eqref{eq:CormenAtAl-WprowadzenieDoAlgorytmow-19} jest prawdziwa dla
ustalonego $n$, ewentualnie dla liczb $0, 1, 2, \ldots, n$, gdzie $n$ jest
cały czas ustalone. Widzimy więc, że~na podstawie założenie $T( n )$
w~kroku indukcyjnym nie jesteśmy w~stanie nic powiedzieć o~zachowaniu
asymptotycznym danej funkcji, a~właśnie takiego rozumowania użyliśmy
w~przytoczonym tu dowodzie.

Płynie stąd wniosek, że~w~dowodach indukcyjnych najlepiej całkowicie unikać
notacji asymptotycznej. W~ostateczność, z~dużą dozą ostrożności można
korzystać w~nich z~relacji asymptotycznych, które zostały udowodnione innymi
metodami.

To co zostało powiedziane powyżej, nie przeczy oczywiście temu, że~stosując
w~sposób niedozwolony symbole asymptotyczne w~dowodach indukcyjnych otrzymano
wiele poprawnych twierdzeń. Takie sytuacje zapewne zdarzają~się dość
często.

\VerSpaceFour





\noindent
\Str{1178} Zatrzymajmy~się na chwilę nad tym w~jaki sposób oszacowano od dołu
sumę szeregu arytmetycznego dla~$n$ parzystego. Ponieważ wszystkie wyrazy
rozważanego szeregu są nieujemne, więc będziemy szukać tylko nieujemnej
funkcji $g( n )$, ograniczającej ją od dołu. By ją znaleźć rozkładamy szereg
na sumę szeregów:
\begin{equation}
  \label{eq:CormenAtAl-WprowadzenieDoAlgorytmow-21}
  \sum_{ k = 1 }^{ n } k =
  \sum_{ k = 1 }^{ n / 2 } k + \sum_{ k = ( n / 2 ) + 1 }^{ n } k \geq
  \sum_{ k = 1 }^{ n / 2 } 0 + \sum_{ k = ( n / 2 ) + 1 }^{ n } \frac{ n }{ 2 } =
  \left( \frac{ n }{ 2 } \right)^{ 2 }.
\end{equation}
Zwróćmy uwagę na to, że~oszacowania od dołu \textit{nie} otrzymujemy poprzez
zastąpienie wyrazów każdej sumy poprzez najmniejszy z~nich. W~takiej
sytuacji wyrazy pierwszej sumy musielibyśmy ograniczyć przez~$1$,
zaś drugiej przez $\frac{ n }{ 2 } + 1$, jednak dla wygody rachunkowej
użyliśmy odpowiednio $0$ i~$\frac{ n }{ 2 }$, czyli wielkości o~jeden
mniejszych od najmniejszego wyrazów danej sumy. Zwróćmy też uwagę na to,
że~gdybyśmy ograniczyli wyrazy pierwszej sumy od dołu przez $1$, to
otrzymalibyśmy oszacowanie
\begin{equation}
  \label{eq:CormenAtAl-WprowadzenieDoAlgorytmow-22}
  \sum_{ k = 1 }^{ n } \geq
  \left( \frac{ n }{ 2 } \right)^{ 2 } + \frac{ n }{ 2 },
\end{equation}
które prowadzi do tego samego ograniczenia asymptotycznego $\Omega( n^{ 2 } )$.

Z~naszego punktu widzenia nie ma to znaczenia, czy wyrazy konkretnej sumy
ograniczamy od dołu przez najmniejszy z~nich, czy przez wyraz mniejszy od
niego o~ile prowadzi to do zadowalającego dolnego oszacowania zachowania
szeregu przez funkcję nieujemną.

\VerSpaceFour





\noindent
\Str{1178} Przeliczmy tutaj oszacowanie od~dołu postępu arytmetycznego
dla przypadku $n$~nieparzystego. Wówczas $\frac{ n - 1 }{ 2 }$
i~$\frac{ n + 1 }{ 2 }$ są liczbami naturalnymi, tym samym możemy przepisać
rozpatrywaną sumę w~następujący sposób.
\begin{equation}
  \label{eq:CormenAtAl-WprowadzenieDoAlgorytmow-23}
  \begin{split}
    \sum_{ k = 1 }^{ n } k
    &=
      \sum_{ k = 1 }^{ ( n - 1 ) / 2 } k + \sum_{ k = ( n + 1 ) / 2 }^{ n } k \geq
      \sum_{ k = 1 }^{ ( n - 1 ) / 2 } 0 + \sum_{ k = ( n + 1 ) / 2 }^{ n } k \geq
      \sum_{ k = ( n + 1 ) / 2 }^{ n } \frac{ n + 1 }{ 2 } = \\[0.5em]
    &= \left( \frac{ n + 1 }{ 2 } \right)^{ 2 } =
      \frac{ n^{ 2 } + 2n + 1 }{ 4 } >
      \left( \frac{ n }{ 2 } \right )^{ 2 } = \Omega( n^{ 2 } ).
  \end{split}
\end{equation}
Aby otrzymać oszacowanie każdej z~sum z~osobna, w~pierwszej z~nich
ograniczyliśmy wszystkie jej wyrazy od dołu przez~$0$. Ten wybór oszacowania
od dołu omówiliśmy już powyżej. W~przypadku drugiej sumy ograniczyliśmy
od~dołu wszystkie jej wyrazy przez najmniejszy spośród nich, gdyż w~tym
wypadku nie uznaliśmy to za bardziej naturalny wybór, zaś alternatywne
oszacowania nie prowadzą do prostszych rachunków.

By otrzymać ostatnią nierówność
w~\eqref{eq:CormenAtAl-WprowadzenieDoAlgorytmow-23} skorzystaliśmy ponadto
z~oczywistego faktu, że~ponieważ $n \geq 1$, to zachodzi
\begin{equation}
  \label{eq:CormenAtAl-WprowadzenieDoAlgorytmow-24}
  \frac{ 2n + 1 }{ 4 } \geq \frac{ 3 }{ 4 } > 0.
\end{equation}

\VerSpaceFour





\noindent
\Str{1187} Należy nadmienić, że~n-tki są uporządkowane tak samo jak para
uporządkowana. Tym samym
\begin{equation}
  \label{eq:CormenAtAl-WprowadzenieDoAlgorytmow-25}
  \begin{split}
    &( a_{ 1 }, a_{ 2 }, \ldots, a_{ i - 1 }, a_{ i }, a_{ i + 1 }, \ldots, a_{ j - 1 },
    a_{ j }, a_{ j + 1 }, \ldots, a_{ n } ) \neq \\
    &\neq ( a_{ 1 }, a_{ 2 }, \ldots, a_{ i - 1 }, a_{ j }, a_{ i + 1 }, \ldots,
    a_{ j - 1 }, a_{ i }, a_{ j + 1 }, \ldots, a_{ n } ),
  \end{split}
\end{equation}
jeśli $a_{ i } \neq a_{ j }$.

\VerSpaceFour





\noindent
\Str{1193} Na tej stronie należałoby wspomnieć, że~zarówno dla grafu
skierowanego jak i~nieskierowanego krawędź łączącą wierzchołki $u, v \in V$
będziemy oznaczać tym samym symbolem $( u, v )$. Choć na poziomie notacji
nie rozróżniamy krawędzi tych dwóch typów grafu, należy pamiętać, że~dla
grafu nieskierowanego zawsze zachodzi
\begin{equation}
  \label{eq:CormenAtAl-WprowadzenieDoAlgorytmow-26}
  ( u, v ) \equiv ( v, u ).
\end{equation}

\VerSpaceFour





\noindent
\Str{1194} Zgodnie z~informacjami podanymi na tej stronie, jeśli w~grafie
skierowanym istnieje pętla wokół wierzchołka $v \in V$, czyli krawędź
$( v, v )$, to uwzględniamy ją zarówno przy obliczaniu stopnia wyjściowego
jak i~stopnia wejściowego tego wierzchołka. Ten fakt nie jest wcale
oczywisty.

\VerSpaceFour











% ##################
\newpage

\CenterBoldFont{Błędy}


\begin{center}

  \begin{tabular}{|c|c|c|c|c|}
    \hline
    Strona & \multicolumn{2}{c|}{Wiersz} & Jest
                              & Powinno być \\ \cline{2-3}
    & Od góry & Od dołu & & \\
    \hline
    1195 & 12 & & spójnej & zbioru krawędzi spójnej \\
    % & & & & \\
    % & & & & \\
    % & & & & \\
    % & & & & \\
    % & & & & \\
    % & & & & \\
    % & & & & \\
    % & & & & \\
    % & & & & \\
    % & & & & \\
    % & & & & \\
    % & & & & \\
    \hline
  \end{tabular}

\end{center}

% \vspace{\spaceTwo}



% ############################










% ####################################################################
% ####################################################################
% Bibliography

\printbibliography





% ############################
% End of the document

\end{document}
