% Autor: Kamil Ziemian
% Korekta: Wojciech Dyba

% ---------------------------------------------------------------------
% Podstawowe ustawienia i pakiety
% ---------------------------------------------------------------------
\RequirePackage[l2tabu, orthodox]{nag} % Wykrywa przestarzałe i niewłaściwe
% sposoby używania LaTeXa. Więcej jest w l2tabu English version.
\documentclass[a4paper,11pt]{article}
% {rozmiar papieru, rozmiar fontu}[klasa dokumentu]
\usepackage[MeX]{polski} % Polonizacja LaTeXa, bez niej będzie pracował
% w języku angielskim.
\usepackage[utf8]{inputenc} % Włączenie kodowania UTF-8, co daje dostęp
% do polskich znaków.
\usepackage{lmodern} % Wprowadza fonty Latin Modern.
\usepackage[T1]{fontenc} % Potrzebne do używania fontów Latin Modern.



% ------------------------------
% Podstawowe pakiety (niezwiązane z ustawieniami języka)
% ------------------------------
\usepackage{microtype} % Twierdzi, że poprawi rozmiar odstępów w tekście.
\usepackage{graphicx} % Wprowadza bardzo potrzebne komendy do wstawiania
% grafiki.
\usepackage{verbatim} % Poprawia otoczenie VERBATIME.
\usepackage{textcomp} % Dodaje takie symbole jak stopnie Celsiusa,
% wprowadzane bezpośrednio w tekście.
\usepackage{vmargin} % Pozwala na prostą kontrolę rozmiaru marginesów,
% za pomocą komend poniżej. Rozmiar odstępów jest mierzony w calach.
% ------------------------------
% MARGINS
% ------------------------------
\setmarginsrb
{ 0.7in} % left margin
{ 0.6in} % top margin
{ 0.7in} % right margin
{ 0.8in} % bottom margin
{  20pt} % head height
{0.25in} % head sep
{   9pt} % foot height
{ 0.3in} % foot sep



% ------------------------------
% Często przydatne pakiety
% ------------------------------
\usepackage{csquotes} % Pozwala w prosty sposób wstawiać cytaty do tekstu.
\usepackage{xcolor} % Pozwala używać kolorowych czcionek (zapewne dużo
% więcej, ale ja nie potrafię nic o tym powiedzieć).



% ------------------------------
% Pakiety do tekstów z nauk przyrodniczych
% ------------------------------
\let\lll\undefined % Amsmath gryzie się z językiem pakietami do języka
% polskiego, bo oba definiują komendę \lll. Aby rozwiązać ten problem
% oddefiniowuję tę komendę, ale może tym samym pozbywam się dużego Ł.
\usepackage[intlimits]{amsmath} % Podstawowe wsparcie od American
% Mathematical Society (w skrócie AMS)
\usepackage{amsfonts, amssymb, amscd, amsthm} % Dalsze wsparcie od AMS
% \usepackage{siunitx} % Dla prostszego pisania jednostek fizycznych
\usepackage{upgreek} % Ładniejsze greckie litery
% Przykładowa składnia: pi = \uppi
\usepackage{slashed} % Pozwala w prosty sposób pisać slash Feynmana.
\usepackage{calrsfs} % Zmienia czcionkę kaligraficzną w \mathcal
% na ładniejszą. Może w innych miejscach robi to samo, ale o tym nic
% nie wiem.



% ##########
% Tworzenie otoczeń "Twierdzenie", "Definicja", "Lemat", etc.
\newtheorem{theorem}{Twierdzenie}  % Komenda wprowadzająca otoczenie
% „theorem” do pisania twierdzeń matematycznych
\newtheorem{definition}{Definicja}  % Analogicznie jak powyżej
\newtheorem{corollary}{Wniosek}



% ---------------------------------------
% Pakiety napisane przez użytkownika.
% Mają być w tym samym katalogu to ten plik .tex
% ---------------------------------------
\usepackage{latexgeneralcommands}
\usepackage{mathcommands}
% \usepackage{calculuscommands}
% \usepackage{SchwartzBooksCommands}  % Pakiet napisany m.in. dla tego pliku.



% ---------------------------------------------------------------------
% Dodatkowe ustawienia dla języka polskiego
% ---------------------------------------------------------------------
\renewcommand{\thesection}{\arabic{section}.}
% Kropki po numerach rozdziału (polski zwyczaj topograficzny)
\renewcommand{\thesubsection}{\thesection\arabic{subsection}}
% Brak kropki po numerach podrozdziału



% ------------------------------
% Ustawienia różnych parametrów tekstu
% ------------------------------
\renewcommand{\arraystretch}{1.2} % Ustawienie szerokości odstępów między
% wierszami w tabelach.



% ------------------------------
% Pakiet "hyperref"
% Polecano by umieszczać go na końcu preambuły.
% ------------------------------
\usepackage{hyperref} % Pozwala tworzyć hiperlinki i zamienia odwołania
% do bibliografii na hiperlinki.










% ---------------------------------------------------------------------
% Tytuł
\title{Elektrodynamika Clerka Maxwella~-- błędy i~uwagi}

% \author{}
% \date{}
% ---------------------------------------------------------------------










% ####################################################################
\begin{document}
% ####################################################################





% ######################################
\maketitle % Tytuł całego tekstu
% ######################################





% ######################################
\section{Standardowe wykłady elektrodynamiki Clerka Maxwella}
% Tytuł danego działu

\vspace{\spaceTwo}
% \vspace{\spaceThree}

% ######################################



% ############################
\Work{ % Autor i tytuł dzieła
  J. D. Jackson \\
  „Elektrodynamik klasyczna”,
  \cite{JacksonElektrodynamikaKlasyczna1987} }


% ##################
\CenterBoldFont{Uwagi}


Dyskusja elektrostatyki powinna się zacząć od dyskusji problemu układu
odniesienia.

\vspace{\spaceFour}



Aby zapewnić fizyczną konsytencje teorii na początku rozdziału I
powinno zostać przyjęte, że w rozważanych przypadkach nie ma obecnych
pól magnetycznych. Nie jest to minimalny warunek konsytencji teorii,
ale najbardziej naturalny.

\vspace{\spaceFour}




% ##################
\CenterBoldFont{Uwagi do konkretnych stron}

% \noi \tb{Konkretne strony}

% \vspace{\spaceFour}


\Str{47} Powinna tu być zamieszczona dyskusja problemu określenia wartości
pola elektrycznego, w~punkcie w~którym znajduje się ładunek punktowy.





% ##################
\CenterBoldFont{Błędy}


\begin{center}

  \begin{tabular}{|c|c|c|c|c|}
    \hline
    & \multicolumn{2}{c|}{} & & \\
    Strona & \multicolumn{2}{c|}{Wiersz} & Jest
                              & Powinno być \\ \cline{2-3}
    & Od góry & Od dołu & & \\
    \hline
    % & & & & \\
    % & & & & \\
    % & & & & \\
    58 & 5 & & $\rho( \bold{ x }' ) \nabla^{ 2 } \bigg( \frac{ 1 }{ \sqrt{ r^{ 2 } + a^{ 2 } } } \bigg) \, d^{ 3 } x'$ & $\int \rho( \bold{ x }' ) \nabla^{ 2 } \bigg( \frac{ 1 }{ \sqrt{ r^{ 2 } + a^{ 2 } } } \bigg) \, d^{ 3 } x'$ \\
    \hline
  \end{tabular}

\end{center}


\noindent
\StrWg{25}{7} \\
\Jest  w~równaniach Maxwella niesymetrycznie jedynie w~pierwszych dwóch
równaniach. \\
\Powin nie występują symetrycznie w~równaniach Maxwella, są obecne jedynie
w~dwóch pierwszych równaniach. \\

Str. 51. po wewnętrznej stronie, Pristley w analogii

Str. 65. $\displaystyle w = \frac{ q^{ 2 }_{ 1 } }{ 8 \pi | \vecxbold - \vecxbold_{ 1 } |^{ 4 } } + \frac{ q^{ 2 }_{ 1 } }{ 8 \pi | \vecxbold - \vecxbold_{ 2 } |^{ 4 } } + \frac{ q_{ 1 } q_{ 2 } \, ( \vecxbold - \vecxbold_{ 1 } ) \cdot ( \vecxbold - \vecxbold_{ 2 } ) }{ 4 \pi | \vecxbold - \vecxbold_{ 1 } |^{ 3 } | \vecxbold - \vecxbold_{ 2 } |^{ 3 } }$.

Str. 114. $\displaystyle Y_{ l m }( \theta, \varphi ) = \sqrt{ \frac{ ( 2l + 1 ) ( l - m )! }{ 4 \pi ( l + m )! } } P^{ m }_{ l }( \cos \theta ) e^{ i m \varphi }$

\vspace{\spaceTwo}
% ############################










% ######################################
\section{Fizyka klasycznych cząstek naładowanych}
% Tytuł danego działu

\vspace{\spaceTwo}
% \vspace{\spaceThree}

% ######################################



% ############################
\Work{ % Autor i tytuł dzieła
  Fritz Rohrlich \\
  „Klasyczna teoria cząstek naładowanych”,
  \cite{RohrlichKlasycznaTeoriaCzastekNaladowanych1981} }


% ##################
\CenterBoldFont{Uwagi}





% ##################
\CenterBoldFont{Uwagi do konkretnych stron}





% ##################
\CenterBoldFont{Błędy}


\begin{center}

  \begin{tabular}{|c|c|c|c|c|}
    \hline
    & \multicolumn{2}{c|}{} & & \\
    Strona & \multicolumn{2}{c|}{Wiersz} & Jest & Powinno być \\ \cline{2-3}
    & Od góry & Od dołu &  &  \\ \hline
    % & & & & \\
    24 & 7 & & $k \vecrbold$ & $-k \vecrbold$ \\
    25 & & 14 & $\vecEbold \times \frac{ \vecvbold }{ c } \times \vecBbold$
           & $\vecEbold + \frac{ \vecvbold }{ c } \times \vecBbold$ \\
    25 & 4 & & \emph{Electrodynamics},John & \emph{Electrodynamics}, John \\
    27 & 10 & & $v / c^{ 2 }$ & $v^{ 2 } / c^{ 2 }$ \\
    % & & & & \\
    % & & & & \\
    \hline
  \end{tabular}

\end{center}


\vspace{\spaceTwo}
% ############################









% #####################################################################
% #####################################################################
% Bibliografia
\bibliographystyle{plalpha}

\bibliography{PhilNaturBooks}{}





% ############################

% Koniec dokumentu
\end{document}
