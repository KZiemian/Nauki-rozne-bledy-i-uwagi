% ---------------------------------------------------------------------
% Podstawowe ustawienia i pakiety
% ---------------------------------------------------------------------
\RequirePackage[l2tabu, orthodox]{nag}  % Wykrywa przestarzałe i niewłaściwe
% sposoby używania LaTeXa. Więcej jest w l2tabu English version.
\documentclass[a4paper,11pt]{article}
% {rozmiar papieru, rozmiar fontu}[klasa dokumentu]
\usepackage[MeX]{polski}  % Polonizacja LaTeXa, bez niej będzie pracował
% w języku angielskim.
\usepackage[utf8]{inputenc}  % Włączenie kodowania UTF-8, co daje dostęp
% do polskich znaków.
\usepackage{lmodern}  % Wprowadza fonty Latin Modern.
\usepackage[T1]{fontenc}  % Potrzebne do używania fontów Latin Modern.



% ------------------------------
% Podstawowe pakiety (niezwiązane z ustawieniami języka)
% ------------------------------
\usepackage{microtype}  % Twierdzi, że poprawi rozmiar odstępów w tekście.
% \usepackage{graphicx}  % Wprowadza bardzo potrzebne komendy do wstawiania
% grafiki.
% \usepackage{verbatim}  % Poprawia otoczenie VERBATIME.
% \usepackage{textcomp}  % Dodaje takie symbole jak stopnie Celsiusa,
% wprowadzane bezpośrednio w tekście.
\usepackage{vmargin}  % Pozwala na prostą kontrolę rozmiaru marginesów,
% za pomocą komend poniżej. Rozmiar odstępów jest mierzony w calach.
% ------------------------------
% MARGINS
% ------------------------------
\setmarginsrb
{ 0.7in}  % left margin
{ 0.6in}  % top margin
{ 0.7in}  % right margin
{ 0.8in}  % bottom margin
{  20pt}  % head height
{0.25in}  % head sep
{   9pt}  % foot height
{ 0.3in}  % foot sep



% ------------------------------
% Często przydatne pakiety
% ------------------------------
% \usepackage{csquotes}  % Pozwala w prosty sposób wstawiać cytaty do tekstu.
\usepackage{xcolor}  % Pozwala używać kolorowych czcionek (zapewne dużo
% więcej, ale ja nie potrafię nic o tym powiedzieć).



% ------------------------------
% Pakiety do tekstów z nauk przyrodniczych
% ------------------------------
\let\lll\undefined  % Amsmath gryzie się z językiem pakietami do języka
% polskiego, bo oba definiują komendę \lll. Aby rozwiązać ten problem
% oddefiniowuję tę komendę, ale może tym samym pozbywam się dużego Ł.
\usepackage[intlimits]{amsmath}  % Podstawowe wsparcie od American
% Mathematical Society (w skrócie AMS)
\usepackage{amsfonts, amssymb, amscd, amsthm}  % Dalsze wsparcie od AMS
\usepackage{siunitx}  % Do prostszego pisania jednostek fizycznych
\usepackage{upgreek}  % Ładniejsze greckie litery
% Przykładowa składnia: pi = \uppi
% \usepackage{slashed}  % Pozwala w prosty sposób pisać slash Feynmana.
\usepackage{calrsfs}  % Zmienia czcionkę kaligraficzną w \mathcal
% na ładniejszą. Może w innych miejscach robi to samo, ale o tym nic
% nie wiem.



% ------------------------------
% Tworzenie środowisk (?) „Twierdzenie”, „Definicja”, „Lemat”, etc.
% ------------------------------
\newtheorem{theorem}{Twierdzenie}  % Komenda wprowadzająca otoczenie
% „theorem” do pisania twierdzeń matematycznych
\newtheorem{definition}{Definicja}  % Analogicznie jak powyżej
\newtheorem{corollary}{Wniosek}



% ------------------------------
% Pakiety napisane przez użytkownika.
% Mają być w tym samym katalogu to ten plik .tex
% ------------------------------
\usepackage{latexgeneralcommands}
\usepackage{mathcommands}



% ---------------------------------------------------------------------
% Dodatkowe ustawienia dla języka polskiego
% ---------------------------------------------------------------------
\renewcommand{\thesection}{\arabic{section}.}
% Kropki po numerach rozdziału (polski zwyczaj topograficzny)
\renewcommand{\thesubsection}{\thesection\arabic{subsection}}
% Brak kropki po numerach podrozdziału



% ------------------------------
% Ustawienia różnych parametrów tekstu
% ------------------------------
\renewcommand{\baselinestretch}{1.1}

% Ustawienie szerokości odstępów między wierszami w tabelach.
\renewcommand{\arraystretch}{1.4}



% ------------------------------
% Pakiet "hyperref"
% Polecano by umieszczać go na końcu preambuły.
% ------------------------------
\usepackage{hyperref}  % Pozwala tworzyć hiperlinki i zamienia odwołania
% do bibliografii na hiperlinki.










% ---------------------------------------------------------------------
% Tytuł, autor, data
\title{Analizy danych eksperymentalnych \\
  {\Large Błędy i~uwagi}}

\author{Kamil Ziemian}


% \date{}
% ---------------------------------------------------------------------










% ####################################################################
\begin{document}
% ####################################################################





% ######################################
\maketitle % Tytuł całego tekstu
% ######################################





% ############################
\Work{ % Redaktor i tytuł dzieła
  Red. Henryk Szydłowski \\
  \textit{Teoria pomiarów},
  \cite{RedSzydlowskiTeoriaPomiarow1981}}

\vspace{0em}


% ##################
\CenterBoldFont{Uwagi do~konkretnych stron}

\vspace{0em}

\noindent
W~notatkach do tej książki, jeśli nie powiedziano inaczej, przez wartość
numeryczną będziemy zawsze rozumieli konkretną wartość liczbową danej
wielkości, niezależnie od tego, czy posiada ona wymiar czy nie. Jeśli więc
mamy daną masę $m = 10 \, \si{kg}$, to przez jej wartość numeryczną będziemy
rozumieć $10 \, \si{kg}$. Koncepcję liczb bezwymiarowych i~posiadających
wymiar omówiliśmy w~notatkach\footnote{W~chwili obecnej plik z~tymi notatkami
  dostępny jest pod adresem
  \href{https://github.com/KZiemian/Nauki-rozne-bledy-i-uwagi/blob/main/Fizyka-b\%C5\%82\%C4\%99dy-i-uwagi/Mechanika-Newtona-b\%C5\%82\%C4\%99dy-i-uwagi/Mechanika-Newtona-b\%C5\%82\%C4\%99dy-i-uwagi.tex}{https://github.com/KZiemian/Nauki-rozne-bledy-i-uwagi/blob/main/Fizyka-b\%C5\%82\%C4\%99dy-i-uwagi/Mechanika-Newtona-b\%C5\%82\%C4\%99dy-i-uwagi/Mechanika-Newtona-b\%C5\%82\%C4\%99dy-i-uwagi.tex}.}
noszących tytuł \textit{Mechanika Newtona, błędy i~uwagi}, więc nie będziemy
tu do tego wracać.





% ##################
\CenterBoldFont{Uwagi do~konkretnych stron}

\vspace{0em}


\noindent
\Str{12} Według informacji zawartych w~tej książce, jeśli dokonujemy pomiaru
na~skali przedziałowej, to jedyne operacje jakie są dopuszczone na wynikach
pomiarów to ich dodawanie i~odejmowanie. Jednak jako przykład skali
przedziałowej podana jest skala temperatur, co sugeruje, że~kilka innych
operacji też powinno być dopuszczonych. Przeanalizujemy teraz tą kwestię.

Zacznijmy od oczywistego stwierdzenia, że~aby operować wielkościami ze skali
przedziałowej, takimi jak temperatura, pomiar nie zawsze jest konieczny.
Całkowicie normalną rzeczą w~fizyce jest rozważanie obiektów o~danej~$T$,
które nie istnieją materialnie, ale~są obiektem naszych rozmyślań. Dotyczy
to nie tylko wielkości które należą do~pewnej skali przedziałowej,
ale~wszystkich innych wielkości, które możemy przypisać obiektom materialnym.
Jest też jasne, że~w~końcu chcemy odnieść nasze rozważania do materialnego
świata i~wtedy pomiar jest niezbędny.

Przejdźmy teraz do operacji, które możemy wykonywać na wielkościach ze skali
przedziałowej, za~wzorcowy ich przykład biorąc temperaturę. Wiemy już,
że~możemy je dodawać i~odejmować, acz książka zabrania nam ich mnożenia
i~dzielenia. Wydaje~się jednak, że~choć nie ma sensu fizycznego mnożenie
temperatur, to mam sens dzielenie dwóch różnicy temperatur. Przyjmijmy,
że~$T_{ 1 } = 10 {}^{ \circ }\si{C}$, $T_{ 2 } = 20 {}^{ \circ }\si{C}$,
$T_{ 3 } = 30 {}^{ \circ }\si{C}$, $T_{ 4 } = 50 {}^{ \circ }\si{C}$, wówczas nie widać
przeciwwskazań by obliczyć iloraz
\begin{equation}
  \label{eq:RedSzydlowski-Teoria-pomiarow-01}
  \frac{ T_{ 4 } - T_{ 3 } }{ T_{ 2 } - T_{ 1 } } = 2.
\end{equation}
Zależność ta wyraża prosty fakt, że~różnica temperatur $T_{ 4 } - T_{ 3 }$
jest dwa razy większa, niż różnica $T_{ 2 } - T_{ 1 }$. To, że początek skali
Celsjusza został wybrany w~sposób arbitralny, nie wpływa na ten wynik, bo
różnica temperatur nie zależy od wyboru miejsca na skali któremu
przypisaliśmy wartość~$0$.

Wzór \eqref{eq:RedSzydlowski-Teoria-pomiarow-01} można również przytoczyć na
poparcie tezy, że~różnicę temperatur można pomnożyć przez liczbę
bezwymiarową. Jeżeli mamy liczbę bezwymiarową, powiedzmy $0.5$, to
jak~się wydaje, stwierdzenie, iż~zachodzi
\begin{equation}
  \label{eq:RedSzydlowski-Teoria-pomiarow-02}
  0.5 \cdot ( T_{ 4 } - T_{ 3 } ) = T_{ 2 } - T_{ 1 },
\end{equation}
jest sensowne. Wyraża ona fakt, że~różnica temperatur między $T_{ 2 }$
i~$T_{ 1 }$ jest równa połowie różnicy temperatur między $T_{ 4 }$ i~$T_{ 3 }$.
Ponieważ zaś dzielenie przez liczbę bezwymiarową to mnożenie przez liczbę
do~niej odwrotną, więc jeśli mnożenie różnicy temperatur przez taką liczbę
jest sensowne, to również dzielenie tej różnicy przez taką liczbę posiada
sens.

Należy~się zastanowić, czy przeprowadzone wyżej rozumowanie jest poprawne,
już teraz jednak należy wspomnieć, że~gdyby było, to pozwoliłoby wyjaśnić
następujący fakt. Załóżmy, że mamy dwa litry wody, jeden o~temperaturze
$T_{ 1 }$, drugi o~temperaturze $T_{ 2 }$, przy czym przyjmujemy,
że~$T_{ 1 } < T_{ 2 }$. Jeśli zmieszamy je w~taki sposób, żeby nie dostarczyć
ani nie odprowadzić do układu żadnej energii, to jak wiemy z~termodynamiki
(zacytować jakieś źródło????) temperatura $T_{ \textrm{res} }$ takiej
mieszaniny będzie wynosić
\begin{equation}
  \label{eq:RedSzydlowski-Teoria-pomiarow-03}
  T_{ \textrm{res} } = \frac{ T_{ 1 } + T_{ 2 } }{ 2 }.
\end{equation}
Jak można uzasadnić to, że~w~tym wzorze dzielimy \textit{sumę} temperatur
przez liczbę bezwymiarową? Można to uzasadnić tym, że~temperatura tej
mieszaniny jest tak naprawdę dana przez wzór
\begin{equation}
  \label{eq:RedSzydlowski-Teoria-pomiarow-04}
  T_{ \textrm{res} } = T_{ 1 } + 0.5 \cdot ( T_{ 2 } - T_{ 1 } ).
\end{equation}

Można argumentować, że~ten wzór ma więcej sensu fizycznego, niż wzór
\eqref{eq:RedSzydlowski-Teoria-pomiarow-03}, jego sens można bowiem wyjaśnić
w~następujący sposób. Temperatura mieszaniny $T_{ \textrm{res} }$ jest większa
od~mniejszej z~temperatur $T_{ 1 }$, $T_{ 2 }$ o~połowę różnicy między nimi.
Choć wartość numeryczna $T_{ \textrm{res} }$ zależy od wyboru miejsca na skali
temperatur któremu przypisujemy wartość zerową, to we wzorze
\eqref{eq:RedSzydlowski-Teoria-pomiarow-04} jedynie wartość $T_{ 1 }$ zależy
od tego wyboru. Taki więc sposób wyznaczania temperatury mieszaniny traktuje
osobno wielkości arbitralne, zależne od~wyboru punktu zerowego na skali,
a~więc pozbawione głębszego sensu, od tych które są od niego
niezależne i~jako takie mają niepodważalny sens fizyczny.

W~takiej sytuacji wzór \eqref{eq:RedSzydlowski-Teoria-pomiarow-04}
należałoby uważać za przekształcenie wzoru
\eqref{eq:RedSzydlowski-Teoria-pomiarow-03}, które jest dopuszczalne
w~momencie gdy wybierzemy konkretną skalę temperatur, na mocy własności
liczb rzeczywistych. Nie jestem tego pewien, ale wydaje mi~się, że~przejścia
od wzoru \eqref{eq:RedSzydlowski-Teoria-pomiarow-03} do
\eqref{eq:RedSzydlowski-Teoria-pomiarow-04} możemy dokonać, bo~pozwalają na
to własności liczb rzeczywistych. Korzystając więc z~tych własności
przekształcamy wzór który ma więcej treści fizycznej, we wzór który prowadzi
do tych samych numerycznych wartości, ale który jest wygodniejszy przy
przeprowadzaniu obliczeń i~łatwiejszy do zapamiętania.

Na~koniec należy wspomnieć, że~w~języku codziennym mówimy na przykład,
iż~temperatura $20 \, {}^{ \circ }\si{C}$ jest dwa razy większa od temperatury
$10 \, {}^{ \circ }\si{C}$, co nie stanowi żadnego problemu, choć jest
niepoprawne z~punktu widzenia teorii wartości na skali przedziałowej. Język
dnia codziennego ma~swoje własne prawa, odmienne od praw języka teorii
naukowej.

{\Large Ponowne czytanie zacząć od strony 12.}
























% ##################
\newpage

\CenterBoldFont{Błędy}


\begin{center}

  \begin{tabular}{|c|c|c|c|c|}
    \hline
    Strona & \multicolumn{2}{c|}{Wiersz} & Jest
                              & Powinno być \\ \cline{2-3}
    & Od góry & Od dołu & & \\
    \hline
    % & & & & \\
    % & & & & \\
    % & & & & \\
    % & & & & \\
    % & & & & \\
    % & & & & \\
    % & & & & \\
    % & & & & \\
    \hline
  \end{tabular}

\end{center}

\vspace{\VerSpaceSix}


% ############################












% #####################################################################
% #####################################################################
% Bibliografia

\bibliographystyle{plalpha}

\bibliography{PhysicsBooks}{}





% ############################

% Koniec dokumentu
\end{document}
