% Autor: Kamil Ziemian

% ---------------------------------------------------------------------
% Podstawowe ustawienia i pakiety
% ---------------------------------------------------------------------
\RequirePackage[l2tabu, orthodox]{nag}  % Wykrywa przestarzałe i niewłaściwe
% sposoby używania LaTeXa. Więcej jest w l2tabu English version.
\documentclass[a4paper,11pt]{article}
% {rozmiar papieru, rozmiar fontu}[klasa dokumentu]
\usepackage[MeX]{polski}  % Polonizacja LaTeXa, bez niej będzie pracował
% w języku angielskim.
\usepackage[utf8]{inputenc}  % Włączenie kodowania UTF-8, co daje dostęp
% do polskich znaków.
\usepackage{lmodern}  % Wprowadza fonty Latin Modern.
\usepackage[T1]{fontenc}  % Potrzebne do używania fontów Latin Modern.



% ------------------------------
% Podstawowe pakiety (niezwiązane z ustawieniami języka)
% ------------------------------
\usepackage{microtype}  % Twierdzi, że poprawi rozmiar odstępów w tekście.
% \usepackage{graphicx}  % Wprowadza bardzo potrzebne komendy do wstawiania
% % grafiki.
% \usepackage{verbatim}  % Poprawia otoczenie VERBATIME.
% \usepackage{textcomp}  % Dodaje takie symbole jak stopnie Celsiusa,
% % wprowadzane bezpośrednio w tekście.
\usepackage{vmargin}  % Pozwala na prostą kontrolę rozmiaru marginesów,
% za pomocą komend poniżej. Rozmiar odstępów jest mierzony w calach.
% ------------------------------
% MARGINS
% ------------------------------
\setmarginsrb
{ 0.7in}  % left margin
{ 0.6in}  % top margin
{ 0.7in}  % right margin
{ 0.8in}  % bottom margin
{  20pt}  % head height
{0.25in}  % head sep
{   9pt}  % foot height
{ 0.3in}  % foot sep



% ------------------------------
% Często używane pakiety
% ------------------------------
% \usepackage{csquotes}  % Pozwala w prosty sposób wstawiać cytaty do tekstu.
\usepackage{xcolor}  % Pozwala używać kolorowych czcionek (zapewne dużo
% więcej, ale ja nie potrafię nic o tym powiedzieć).


% ------------------------------
% Pakiety do tekstów z nauk przyrodniczych
% ------------------------------
\let\lll\undefined  % Amsmath gryzie się z językiem pakietami do języka
% polskiego, bo oba definiują komendę \lll. Aby rozwiązać ten problem
% oddefiniowuję tę komendę, ale może tym samym pozbywam się dużego Ł.
\usepackage[intlimits]{amsmath}  % Podstawowe wsparcie od American
% Mathematical Society (w skrócie AMS)
\usepackage{amsfonts, amssymb, amscd, amsthm}  % Dalsze wsparcie od AMS
% \usepackage{siunitx}  % Do prostszego pisania jednostek fizycznych
\usepackage{upgreek}  % Ładniejsze greckie litery
% Przykładowa składnia: pi = \uppi
% \usepackage{slashed}  % Pozwala w prosty sposób pisać slash Feynmana.
\usepackage{calrsfs}  % Zmienia czcionkę kaligraficzną w \mathcal
% na ładniejszą. Może w innych miejscach robi to samo, ale o tym nic
% nie wiem.



% ------------------------------
% Pakiety napisane przez użytkownika.
% Mają być w tym samym katalogu to ten plik .tex
% ------------------------------
\usepackage{latexgeneralcommands}
\usepackage{mathcommands}

\usepackage{tensor}



% ---------------------------------------------------------------------
% Dodatkowe ustawienia dla języka polskiego
% ---------------------------------------------------------------------
\renewcommand{\thesection}{\arabic{section}.}
% Kropki po numerach rozdziału (polski zwyczaj topograficzny)
\renewcommand{\thesubsection}{\thesection\arabic{subsection}}
% Brak kropki po numerach podrozdziału



% ------------------------------
% Ustawienia różnych parametrów tekstu
% ------------------------------
\renewcommand{\arraystretch}{1.2}  % Ustawienie szerokości odstępów między
% wierszami w tabelach.





% ------------------------------
% Pakiet „hyperref”
% Polecano by umieszczać go na końcu preambuły.
% ------------------------------
\usepackage{hyperref}  % Pozwala tworzyć hiperlinki i zamienia odwołania
% do bibliografii na hiperlinki.










% ---------------------------------------------------------------------
% Tytuł, autor, data
\title{QFT~-- błędy i~uwagi}

% \author{}
% \date{}
% ---------------------------------------------------------------------










% ####################################################################
% Początek dokumentu
\begin{document}
% ####################################################################





% ######################################
\maketitle % Tytuł całego tekstu
% ######################################




% ############################
\Work{ % Autor i tytuł dzieła
  Silvan S. Schweber \\
  „An~Introduction to~Relativistic Quantum Field Theory”,
  \cite{SchewberAnIntroductionToRelativisticQuantumFieldTheory2005} }


% ##################
\CenterBoldFont{Uwagi do konkretnych stron}


\start \Str{4} Dyskusja roli pomiaru w~mechanice kwantowej jest
niesatysfakcjonująca, jednak nigdzie jeszcze nie znalazłem dobrej
dyskusji tego problemu, zaś~Schwebera należy pochwalić za zwięzłość
i~klarowność wywodu oraz~jasne stwierdzenie, że~pojęcie pomiaru jest
fundamentalne zarówno dla sformułowania jak i~interpretacji mechaniki
kwantowej.

Jednak problem tego czym jest pomiar nie jest w~ogóle postawiony.
Dlaczego niektóre oddziaływania z~układem wywołują kolaps wektora
stanu, inne nie? Dlaczego w~konsekwencji tego niektóre
oddziaływania~są pomiarem, a~inne nie? Dlaczego, w~końcu, układ
zaburzony przez pomiar który dał wartość obserwabli $a'$, skolapsuje
do wektora $| a' \rangle \langle a' | \Psi \rangle$, a~nie do jakiegoś
innego? Na to pytanie mechanika kwantowa nie udziela odpowiedzi
i~dopóki rozwiązanie tego problemu nie zostanie znalezione, mechanika
kwantowa nie będzie satysfakcjonującą teorią.

Drugim, mniej znaczącym problemem, jest to czy prawie wszystkie
pomiary można na poziomie podstawowym opisać jako zderzenie cząstek?
Których w~takim razie nie można do takiego procesu sprowadzić? Czy na
przykład kiedy czytam ten tekst, to pomiarem jaki wykonałem było
rozproszenie fotonu na cząstkach w~mojej siatkówce? Moja skromna
wiedza z~biologii sugeruje, że~jednak coś jeszcze.

\vspace{\spaceFour}



\start \Str{5} Muszę~się stanowczo nie~zgodzić z~twierdzeniem,
że~znaczenie procesów rozpraszania dla fizyki teoretycznej polega na
tym, że~nie jest konieczne rozumienie sensu fizycznego (lub jak kto
woli, interpretacja) funkcji falowej, gdy cząstki są~blisko siebie
i~silnie oddziałują. Głównym celem nauki jest zrozumienie przyrody,
więc poznanie fizycznego znaczenia funkcji falowej w~takiej procesie
jest jednym z~głównych problemów kwantowej teorii pola.

Stwierdzenie, że~nie musimy tego robić dla procesów rozpraszania
i~dlatego są one ważne, jest tylko dowodem na to, iż~jako fizycy
ponieśliśmy porażkę i~uznaliśmy ją za sukces.

\vspace{\spaceFour}



\start \Str{9}

\vspace{\spaceFour}



\start \Str{10} Ciągłe używane notacji Diraca, które zaczyna~się
na~tej stronie bardzie zaciemnia mi, niż rozjaśnia, zrozumienie
działania operatorów położenia i~pędu w~konkretnej reprezentacji.

\vspace{\spaceFour}



\start \Str{13}

\vspace{\spaceFour}





% ##################
\CenterBoldFont{Błędy}


\begin{center}
  \begin{tabular}{|c|c|c|c|c|}
    \hline
    & \multicolumn{2}{c|}{} & & \\
    Strona & \multicolumn{2}{c|}{Wiersz} & Jest
                              & Powinno być \\ \cline{2-3}
    & Od góry & Od dołu &  &  \\
    \hline
    7 & 13 & & $| \; t_{ 0 } )$ & $| t_{ 0 } \rangle$ \\
    7 & & 8 & $U^{ -1 }( t_{ 1 }, t_{ 0 } )$
           & $U^{ -1 }( t_{ 0 }, t_{ 1 } )$ \\
    7 & & 9 & (8\textbf{)} & (8) \\
    8 & & 4 & $V( t ) H_{ S } V( t ) \cdot V^{ -1 }( t )$
           & $V( t ) H_{ S }$ \\
    10 & & 8 & $\delta^{ ( 3 ) }( \vecqbold' - \vecqbold'' )$
           & $\delta^{ ( 3 ) }( \vecqbold'' - \vecqbold' )$ \\
    73 & 10 & & $\epsilon_{ i j }{}^{ 4 } \Gamma_{ k }{}' F \Gamma_{ k }$
           & $\epsilon_{ i j }{}^{ 4 } \Gamma_{ i }{}' F \Gamma_{ i }$ \\
    % & & & & \\
    % & & & & \\
    % & & & & \\
    % & & & & \\
    \hline
  \end{tabular}
\end{center}


\noindent
Str. 22. \ldots ($i = 1, 2,\cdots n$)\ldots


\vspace{\spaceTwo}
% ############################










% ######################################
\section{Trylogia Weinberga}

\vspace{\spaceTwo}

% ######################################



% ############################
\Work{ % Autor i tytuł dzieła
  Steven Weinberg \\
  „Teoria pól kwantowych. Podstawy”,
  \cite{WeinbergTeoriaPolKwantowychPodstawy2012} }


% ##################
\CenterBoldFont{Uwagi do konkretnych stron}


\Str{79} Z faktu, że istnieje macierz odwrotna do
$\eta_{ \mu \nu } \Lcal^{ \mu }_{ \rho }$????!!!!!





% ##################
\CenterBoldFont{Błędy}


\begin{center}

  \begin{tabular}{|c|c|c|c|c|}
    \hline
    & \multicolumn{2}{c|}{} & & \\
    Strona & \multicolumn{2}{c|}{Wiersz} & Jest
                              & Powinno być \\ \cline{2-3}
    & Od góry & Od dołu & & \\
    \hline
    31 & 4 & & $-c^{ 2 } \hbar^{ 2 }$ & $+c^{ 2 } \hbar^{ 2 }$ \\
    52 & 14 & & $b^{ \dagger }( \veckbold ) \exp( i \omega_{ \veckbold } t )$
           & $b^{ \dagger }( \veckbold ) \exp( +i \omega_{ \veckbold } t )$ \\
    59 & & 4 & Heinsenberg [72[ & Heinsenberg [72] \\ % Zła pisownia.
    60 & 5 & & przyspieszone & przyśpieszone \\
    70 & 2 & & 1961. & 1961). \\
    79 & 3 & & $\eta_{ \mu \nu } dx^{ ' \mu } dx^{ ' \nu }$
           & $\eta_{ \mu \nu } dx'^{ \mu } dx'^{ \nu }$ \\
    79 & 5 & & $\eta_{ \mu \nu } \frac{ \partial x^{ ' \mu } }{ \partial x^{ \rho } }
               \frac{ \partial x^{ ' \nu } }{ \partial x^{ \sigma } }$
           & $\eta_{ \mu \nu } \frac{ \partial x'^{ \mu } }{ \partial x^{ \rho } }
             \frac{ \partial x'^{ \nu } }{ \partial x^{ \sigma } }$ \\
    86 & & 2 & $\Lcal_{ \rho }^{ -1 \mu } P^{ \rho }$
           & $\tensor[]{ ( \Lcal^{ -1 } ) }{ ^\mu_\rho } P^{ \rho }$ \\
    142 & & 14 & powloce & powłoce \\
    \hline
  \end{tabular}





  % \begin{tabular}{|c|c|c|c|c|}
  %   \hline
  %   & \multicolumn{2}{c|}{} & & \\
  %   Strona & \multicolumn{2}{c|}{Wiersz} & Jest
  %   & Powinno być \\ \cline{2-3}
  %   & Od góry & Od dołu & & \\
  %   \hline
  %     %   & & & & \\
  %     %   & & & & \\
  %     %   & & & & \\
  %     %   & & & & \\
  %     %   & & & & \\
  %     %   & & & & \\
  %   \hline
  % \end{tabular}

\end{center}


\noindent
\Str{85} \ldots \\
\Str{86} Równania (2.5.1) i\ldots \\
\Str{86} równanie na dole \\
\StrWg{281}{1} \\
\Jest
$S_{ \vecpbold_{ 1 }', \sigma_{ 1 }', n_{ 1 }'; \, \vecpbold_{ 2 }',
  \sigma_{ 2 }', n_{ 2 }';\; \cdots, \; \vecpbold_{ 1 }, \sigma_{ 1 }, n_{
    1 } ; \, \ldots { p }_{ 2 },
  \sigma_{ 2 }, n_{ 2 }; \, \cdots }$ \\[0.5em]
\Powin
$S_{ \vecpbold_{ 1 }', \sigma_{ 1 }', n_{ 1 }'; \, \vecpbold_{ 2 }',
  \sigma_{ 2 }', n_{ 2 }';\; \cdots; \; \vecpbold_{ 1 }, \sigma_{ 1 },
  n_{ 1 } ; \, \vecpbold_{ 2 }, \sigma_{ 2 }, n_{ 2 }; \, \cdots }$ \\


\vspace{\spaceTwo}
% ############################










% ############################
\Work{ % Autor i tytuł dzieła
  Steven Weinberg \\
  „Teoria pól kwantowych. Tom II: Nowoczesne zastosowanie”,
  \cite{WeinbergTeoriaPolKwantowychNowoczesneZastosowania1999} }


% ##################
\CenterBoldFont{Uwagi do konkretnych stron}





% ##################
\CenterBoldFont{Błędy}

\begin{center}

  \begin{tabular}{|c|c|c|c|c|}
    \hline
    & \multicolumn{2}{c|}{} & & \\
    Strona & \multicolumn{2}{c|}{Wiersz} & Jest
                              & Powinno być \\ \cline{2-3}
    & Od góry & Od dołu & & \\
    \hline
    & & & & \\
    % & & & & \\
    % & & & & \\
    % & & & & \\
    % & & & & \\
    % & & & & \\
    % & & & & \\
    % & & & & \\
    % & & & & \\
    % & & & & \\
    % & & & & \\
    % & & & & \\
    % & & & & \\
    % & & & & \\
    % & & & & \\
    % & & & & \\
    % & & & & \\
    % & & & & \\
    % & & & & \\
    % & & & & \\
    % & & & & \\
    % & & & & \\
    % & & & & \\
    % & & & & \\
    % & & & & \\
    % & & & & \\
    % & & & & \\
    % & & & & \\
    % & & & & \\
    % & & & & \\
    % & & & & \\
    % & & & & \\
    % & & & & \\
    % & & & & \\
    % & & & & \\
    % & & & & \\
    % & & & & \\
    % & & & & \\
    \hline
  \end{tabular}





  % \begin{tabular}{|c|c|c|c|c|}
  %   \hline
  %   & \multicolumn{2}{c|}{} & & \\
  %   Strona & \multicolumn{2}{c|}{Wiersz} & Jest
  %   & Powinno być \\ \cline{2-3}
  %   & Od góry & Od dołu & & \\
  %   \hline
  %   %   & & & & \\
  %   %   & & & & \\
  %   %   & & & & \\
  %   %   & & & & \\
  %   %   & & & & \\
  %   %   & & & & \\
  %   %   & & & & \\
  %   %   & & & & \\
  %   %   & & & & \\
  %   %   & & & & \\
  %   %   & & & & \\
  %   %   & & & & \\
  %   %   & & & & \\
  %   %   & & & & \\
  %   %   & & & & \\
  %   %   & & & & \\
  %   %   & & & & \\
  %   %   & & & & \\
  %   %   & & & & \\
  %   %   & & & & \\
  %   %   & & & & \\
  %   %   & & & & \\
  %   %   & & & & \\
  %   %   & & & & \\
  %   %   & & & & \\
  %   %   & & & & \\
  %   %   & & & & \\
  %   %   & & & & \\
  %   %   & & & & \\
  %   %   & & & & \\
  %   %   & & & & \\
  %   %   & & & & \\
  %   %   & & & & \\
  %   %   & & & & \\
  %   %   & & & & \\
  %   %   & & & & \\
  %   %   & & & & \\
  %   %   & & & & \\
  %   \hline
  % \end{tabular}

\end{center}


\noindent
% \Jest  Selected Writings~of Alexandra Kollontai \\
% \Powin \emph{Selected Writings~of Alexandra Kollontai} \\

% \StrWd{11}{16} \\
% \Jest  Amerian Seafarers Union \\
% \Powin \emph{Amerian Seafarers Union} \\



% \StrWd{11}{5} \\
% \Jest  Council~of Economic Advisers \\
% \Powin \emph{Council~of Economic Advisers} \\
% \StrWd{11}{1} \\
% \Jest  Council on~Foregin Relations \\
% \Powin \emph{Council on~Foregin Relations} \\
% \StrWg{12}{11} \\
% \Jest \emph{Congress}) --~Kanadyjski Kongres Związków Zawodowych \\
% \Powin  \emph{Congress} --~Kanadyjski Kongres Związków Zawodowych) \\
% \StrWd{12}{18} \\
% \Jest Emergency Committe for~Aid to~Poland \\
% \Powin  \emph{Emergency Committe for~Aid to~Poland} \\
% \StrWd{12}{8} \\
% \Jest (\emph{Froce Ouvri\'{e}re} ) \\
% \Powin (\emph{Froce Ouvri\'{e}re} --~Główna Konfederacja Pracy
% --~Siły
% Pracy) \\
% \StrWg{13}{10} \\
% \Jest Generalized System~of Preferences \\
% \Powin  \emph{Generalized System~of Preferences} \\
% \StrWd{13}{4} \\
% \Jest \emph{Leuven}) \\
% \Powin \emph{Leuven} --~Kotlicki Ośrodek Dokumentacyjny i~Badań
% Katolickich Uniwersytetu Leuven \\
% \StrWg{15}{1} \\
% \Jest Polish American Congress Charitable Foundation \\
% \Powin  \emph{Polish American Congress Charitable Foundation} \\
% \StrWg{15}{3} \\
% \Jest Polish-American Enterprise Fund \\
% \Powin  \emph{Polish-American Enterprise Fund} \\
% \StrWg{15}{10} \\
% \Jest Postal, Telegraph and~Telephone International \\
% \Powin  \emph{Postal, Telegraph and~Telephone International} \\
% \StrWg{43}{2} \\
% \Jest \emph{Univeristy~of Illinois at~Urbana Champaign, 1998} \\
% \Powin  Univeristy~of Illinois at~Urbana-Champaign, 1998 \\


\vspace{\spaceTwo}
% ############################


































% ####################################################################
% ####################################################################
% Bibliografia
\bibliographystyle{plalpha}

\bibliography{PhilNaturBooks}{}





% ############################

% Koniec dokumentu
\end{document}