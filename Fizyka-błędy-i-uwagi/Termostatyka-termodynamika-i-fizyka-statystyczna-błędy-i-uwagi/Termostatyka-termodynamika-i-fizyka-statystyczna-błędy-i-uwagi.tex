% ---------------------------------------------------------------------
% Podstawowe ustawienia i pakiety
% ---------------------------------------------------------------------
\RequirePackage[l2tabu, orthodox]{nag} % Wykrywa przestarzałe i niewłaściwe
% sposoby używania LaTeXa. Więcej jest w l2tabu English version.
\documentclass[a4paper,11pt]{article}
% {rozmiar papieru, rozmiar fontu}[klasa dokumentu]
\usepackage[MeX]{polski} % Polonizacja LaTeXa, bez niej będzie pracował
% w języku angielskim.
\usepackage[utf8]{inputenc} % Włączenie kodowania UTF-8, co daje dostęp
% do polskich znaków.
\usepackage{lmodern} % Wprowadza fonty Latin Modern.
\usepackage[T1]{fontenc} % Potrzebne do używania fontów Latin Modern.



% ------------------------------
% Podstawowe pakiety (niezwiązane z ustawieniami języka)
% ------------------------------
\usepackage{microtype} % Twierdzi, że poprawi rozmiar odstępów w tekście.
\usepackage{graphicx} % Wprowadza bardzo potrzebne komendy do wstawiania
% grafiki.
\usepackage{verbatim} % Poprawia otoczenie VERBATIME.
\usepackage{textcomp} % Dodaje takie symbole jak stopnie Celsiusa,
% wprowadzane bezpośrednio w tekście.
\usepackage{vmargin} % Pozwala na prostą kontrolę rozmiaru marginesów,
% za pomocą komend poniżej. Rozmiar odstępów jest mierzony w calach.
% ------------------------------
% MARGINS
% ------------------------------
\setmarginsrb
{ 0.7in}  % left margin
{ 0.6in}  % top margin
{ 0.7in}  % right margin
{ 0.8in}  % bottom margin
{  20pt}  % head height
{0.25in}  % head sep
{   9pt}  % foot height
{ 0.3in}  % foot sep



% ------------------------------
% Często przydatne pakiety
% ------------------------------
% \usepackage{csquotes} % Pozwala w prosty sposób wstawiać cytaty do tekstu.
% \usepackage{xcolor} % Pozwala używać kolorowych czcionek (zapewne dużo
% % więcej, ale ja nie potrafię nic o tym powiedzieć).



% ------------------------------
% Pakiety do tekstów z nauk przyrodniczych
% ------------------------------
\let\lll\undefined % Amsmath gryzie się z językiem pakietami do języka
% polskiego, bo oba definiują komendę \lll. Aby rozwiązać ten problem
% oddefiniowuję tę komendę, ale może tym samym pozbywam się dużego Ł.
\usepackage[intlimits]{amsmath} % Podstawowe wsparcie od American
% Mathematical Society (w skrócie AMS)
\usepackage{amsfonts, amssymb, amscd, amsthm} % Dalsze wsparcie od AMS
% \usepackage{siunitx} % Dla prostszego pisania jednostek fizycznych
\usepackage{upgreek} % Ładniejsze greckie litery
% Przykładowa składnia: pi = \uppi
\usepackage{slashed} % Pozwala w prosty sposób pisać slash Feynmana.
\usepackage{calrsfs} % Zmienia czcionkę kaligraficzną w \mathcal
% na ładniejszą. Może w innych miejscach robi to samo, ale o tym nic
% nie wiem.



% ------------------------------
% Tworzenie środowisk (?) „Twierdzenie”, „Definicja”, „Lemat”, etc.
% ------------------------------
% Komenda wprowadzająca otoczenie „theorem” do pisania twierdzeń
% matematycznych.
\newtheorem{theorem}{Twierdzenie}
% Analogicznie jak powyżej
\newtheorem{definition}{Definicja}
\newtheorem{corollary}{Wniosek}



% ---------------------------------------
% Pakiety napisane przez użytkownika.
% Mają być w tym samym katalogu to ten plik .tex
% ---------------------------------------
\usepackage{latexgeneralcommands}
\usepackage{mathcommands}



% ---------------------------------------------------------------------
% Dodatkowe ustawienia dla języka polskiego
% ---------------------------------------------------------------------
\renewcommand{\thesection}{\arabic{section}.}
% Kropki po numerach rozdziału (polski zwyczaj topograficzny)
\renewcommand{\thesubsection}{\thesection\arabic{subsection}}
% Brak kropki po numerach podrozdziału



% ------------------------------
% Ustawienia różnych parametrów tekstu
% ------------------------------
\renewcommand{\baselinestretch}{1.1}

% Ustawienie szerokości odstępów między wierszami w tabelach.
\renewcommand{\arraystretch}{1.4}



% ------------------------------
% Pakiet „hyperref”
% Polecano by umieszczać go na końcu preambuły.
% ------------------------------
\usepackage{hyperref} % Pozwala tworzyć hiperlinki i zamienia odwołania
% do bibliografii na hiperlinki.










% ---------------------------------------------------------------------
% Tytuł, autor, data
\title{Termostatyka, termodynamika i~fizyka statystyczna \\
  {\Large Błędy i~uwagi}}

\author{Kamil Ziemian}


% \date{}
% ---------------------------------------------------------------------










% ####################################################################
\begin{document}
% ####################################################################





% ######################################
\maketitle % Tytuł całego tekstu
% ######################################





% ######################################
\section{Termostatyka}

\vspace{\spaceTwo}
% ######################################



% ####################
\Work{ % Autor i tytuł dzieła
  Józef Werle \\
  \textit{Termodynamika fenomenologiczna},
  \cite{WerleTermodynamikaFenomenologiczna1957}}

\vspace{0em}


% ##################
\CenterBoldFont{Uwagi do~konkretnych stron}

\vspace{0em}


\noindent
\Str{15} Werle wprowadza jak strasznie zamieszanie pojęciowe,
twierdząc na tej samej stronie, że~objętość jest zarówno parametrem
wewnętrzny jak i~zewnętrznym. Powinno~się jawnie stwierdzić, że~mogą
istnieć parametry zarówno wewnętrzne, jak i~zewnętrzne układu,
albo~że~są to wykluczające~się kategorie i~nigdy nie wprowadzać jednej
wielkości do~nich obu.

Dobrego wyjaśnienie podał Kacper Zalewski. Zmienne zewnętrzne to takie
na którymi mamy kontrolę. \red{A czy nad objętością metalu mamy
  kontrolę i tym podobne rozterki? Dokończ.}

\vspace{\spaceFour}



\noindent
\Str{17}





% ##################
\noindent

\CenterBoldFont{Błędy}


\begin{center}

  \begin{tabular}{|c|c|c|c|c|}
    \hline
    Strona & \multicolumn{2}{c|}{Wiersz} & Jest
                              & Powinno być \\ \cline{2-3}
    & Od góry & Od dołu & & \\
    \hline
    5   & &  2 & odwracalnych & nieodwracalnych \\
    15  & & 13 & lśedzenia & śledzenia \\
    % & & & & \\
    % & & & & \\
    % & & & & \\
    % & & & & \\
    % & & & & \\
    % & & & & \\
    % & & & & \\
    % & & & & \\
    % & & & & \\
    \hline
  \end{tabular}

\end{center}

\vspace{\spaceTwo}


% ############################










% ############################
\Work{ % Autor i tytuł dzieła
  Kerson Huang \\
  \textit{Mechanika statystyczna}, \cite{HuangMechanikaStatystyczna1987}}

\vspace{0em}


% ##################
\CenterBoldFont{Uwagi}

\vspace{0em}


\noindent
???? Brak dyskusji zależności wykładanej teorii od układu odniesienia.

\vspace{\spaceFour}





\noindent
???? Jaka jest struktura matematyczna przestrzeni stanów termodynamicznych
(krótko: przestrzeni termodynamicznej)?

\vspace{\spaceFour}





\noindent
??? Przedstawiona tu dyskusja termodynamiki zostawia ogromną ilość pytań,
zarówno fizycznych jak i matematycznych, bez odpowiedzi.

\vspace{\spaceFour}





\noindent
???? Prawdopodobieństwo zaistnienia danego stanu w części poświęconej fizyce
statystycznej nie zostało wyczerpująco omówione.





% ##################
\CenterBoldFont{Uwagi do~konkretnych stron}


\noindent
\Str{12} To, że równanie
\begin{equation}
  \label{eq:Huang-Mechanika-statystyczna-01}
  f( P, V, T ) = 0
\end{equation}
definiuje nam powierzchnię w~$\Rbb^{ 3 }$ jest intuicyjnie jasne. Trzeba by
oczywiście doprecyzować jakie własności posiada funkcja, czy jest na
przykład różniczkowalna i~wtedy możemy mówić o~powierzchni różniczkowalnej,
etc.

Jednak jeden problem wydaje~się być zupełnie nieporuszony. Czy przy
ustalonej wartości dwóch parametrów, powiedzmy $P_{ 1 }$ i~$V_{ 1 }$ jest
dopuszczalna więcej niż jedna wartość temperatury $T$? Inaczej mówiąc, czy
równanie $f( P_{ 1 }, V_{ 1 }, T ) = 0$ dopuszcza tylko jedno rozwiązanie na
$T$, czy też może ich być więcej: $T_{ 1 }$, $T_{ 2 }$, $T_{ 3 }$, \ldots

Jak powszechnie wiadomo równanie sfery w~trzech wymiarach
\begin{equation}
  \label{eq:Huang-Mechanika-statyczna-02}
  f( x, y, z ) = x^{ 2 } + y^{ 2 } + z^{ 2 } - 1 = 0,
\end{equation}
ma dla $x_{ 1 }$, $y_{ 2 }$ spełniających związek
$( x_{ 1 } )^{ 2 } + ( y_{ 1 } )^{ 2 } = 1$ dokładnie jedno rozwiązanie
$z = 0$. Jeżeli $( x_{ 1 } )^{ 2 } + ( y_{ 1 } )^{ 2 } < 1$ rozwiązania są
dokładnie $z = \sqrt{ ( x_{ 1 } )^{ 2 } + ( y_{ 1 } )^{ 2 } }$
i~$z = -\sqrt{ ( x_{ 1 } )^{ 2 } + ( y_{ 1 } )^{ 2 } }$. W~pozostałych
przypadkach, $( x_{ 1 } )^{ 2 } + ( y_{ 1 } )^{ 2 } > 1$, nie ma żadnych
rozwiązań. Przykład ten pokazuje więc, że~kwestia tego jaką powierzchnię
wyznacza równanie \eqref{eq:Huang-Mechanika-statyczna-01} nie jest wcale
banalna ani z~matematycznego, ani z~fizycznego punktu widzenia.










% ##################
\CenterBoldFont{Błędy}


\begin{center}

  \begin{tabular}{|c|c|c|c|c|}
    \hline
    Strona & \multicolumn{2}{c|}{Wiersz} & Jest
                              & Powinno być \\ \cline{2-3}
    & Od góry & Od dołu & & \\
    \hline
    12  & & 12 & $f( p, V, T )$ & $f( P, V, T )$ \\
    %     & & & & \\
    %     & & & & \\
    \hline
  \end{tabular}

\end{center}

\vspace{\spaceTwo}


Str. 135-136. dla których gęstość zależy od $( p, q )$ tylko przez hamiltonian \\


% ############################










% ##############################
\Work{ % Autor i tytuł dzieła
  Józef Werle ,,Termodynamika fenomenologiczna'', \cite{Wer57}. }

\vspace{0em}


% ##################
\CenterBoldFont{Uwagi do~konkretnych stron}

\vspace{0em}


\noindent
\Str{15} Werle wprowadza jak strasznie zamieszanie pojęciowe,
twierdząc na tej samej stronie, że~objętość jest zarówno parametrem
wewnętrzny jak i~zewnętrznym. Powinno~się jawnie stwierdzić, że~mogą
istnieć parametry zarówno wewnętrzne, jak i~zewnętrzne układu,
albo~że~są to wykluczające~się kategorie i~nigdy nie wprowadzać
jednej wielkości do~nich obu.

Trzeba~się zastanowić, jak jest naprawdę. \Dok

\vspace{\spaceFour}





\noindent
\Str{17}


% ##################
\CenterBoldFont{Błędy}


\begin{center}

  \begin{tabular}{|c|c|c|c|c|}
    \hline
    Strona & \multicolumn{2}{c|}{Wiersz} & Jest
    & Powinno być \\ \cline{2-3}
    & Od góry & Od dołu &  &  \\ \hline
    5 & & 2 & odwracalnych & nieodwracalnych \\
    15 & & 13 & lśedzenia & śledzenia \\
    % & & & & \\
    % & & & & \\
    % & & & & \\
    % & & & & \\
    % & & & & \\
    % & & & & \\
    % & & & & \\
    % & & & & \\
    & & & & \\ \hline
  \end{tabular}

\end{center}

\vspace{\spaceTwo}




% ############################










% #####################################################################
% #####################################################################
% Bibliografia

\bibliographystyle{plalpha}

\bibliography{PhilNaturBooks}{}





% ############################

% Koniec dokumentu
\end{document}
