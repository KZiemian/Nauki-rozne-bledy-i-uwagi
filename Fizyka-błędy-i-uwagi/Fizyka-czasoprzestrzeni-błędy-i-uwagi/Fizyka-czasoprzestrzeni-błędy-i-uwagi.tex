% ---------------------------------------------------------------------
% Basic configuration and packages
% ---------------------------------------------------------------------
% Package for discovering wrong and outdated usage of LaTeX.
% More information to be found in l2tabu English version.
\RequirePackage[l2tabu, orthodox]{nag}
% Class of LaTeX document: {size of paper, size of font}[document class]
\documentclass[a4paper,11pt]{article}



% ---------------------------------------
% Packages not tied to particular normal language
% ---------------------------------------
% This package should improved spaces in the text.
\usepackage{microtype}
% Add few important symbols, like text Celcius degree
\usepackage{textcomp}



% ---------------------------------------
% Polonization of LaTeX document
% ---------------------------------------
% Basic polonization of the text
\usepackage[MeX]{polski}
% Switching on UTF-8 encoding
\usepackage[utf8]{inputenc}
% Adding font Latin Modern
\usepackage{lmodern}
% Package is need for fonts Latin Modern
\usepackage[T1]{fontenc}



% ---------------------------------------
% Setting margins
% ---------------------------------------
\usepackage[a4paper, total={14cm, 25cm}]{geometry}
% Package for easy settings of margins. Unit of measurement is inch.
% \usepackage{vmargin}
% \setmarginsrb
% { 0.7in}  % left margin
% { 0.6in}  % top margin
% { 0.7in}  % right margin
% { 0.8in}  % bottom margin
% {  20pt}  % head height
% {0.25in}  % head sep
% {   9pt}  % foot height
% { 0.3in}  % foot sep



% ---------------------------------------
% Setting vertical spaces in the text
% ---------------------------------------
% Setting space between lines
\renewcommand{\baselinestretch}{1.1}

% Setting space between lines in tables
\renewcommand{\arraystretch}{1.4}



% ---------------------------------------
% Packages for scientific papers
% ---------------------------------------
% Switching off \lll symbol, that I guess is representing letter ``Ł''.
% It collide with `amsmath' package's command with the same name
\let\lll\undefined
% Basic package from American Mathematical Society (AMS)
\usepackage[intlimits]{amsmath}
% Equations are numbered separately in every section.
\numberwithin{equation}{section}

% Other very useful packages from AMS
\usepackage{amsfonts, amssymb, amscd, amsthm}

% Package with better looking calligraphy fonts
\usepackage{calrsfs}

% Package with better looking greek letters
% Example of use: pi -> \uppi
\usepackage{upgreek}
% Improving look of lambda letter
\let\oldlambda\Lambda
\renewcommand{\lambda}{\uplambda}

% Package for better support of tensor notation
\usepackage{tensor}





% ---------------------------------------
% Defining new environments (?)
% ---------------------------------------
% Defining enviroment ``Wniosek''
\newtheorem{corollary}{Wniosek}
\newtheorem{definition}{Definicja}
\newtheorem{theorem}{Twierdzenie}





% ------------------------------
% Private packages
% You need to put them in the same directory as .tex file
% ------------------------------
% Contains various command useful for working with a text
\usepackage{latexgeneralcommands}
% Contains definitions useful for working with mathematical text
\usepackage{mathcommands}





% ------------------------------
% Package ``hyperref''
% They advised to put it on the end of preambule
% ------------------------------
% It allows you to use hyperlinks in the text
\usepackage{hyperref}










% ---------------------------------------------------------------------
% Tytuł, autor, data
\title{Fizyka czasoprzestrzenni \\
  {\Large Błędy i~uwagi}}

\author{Kamil Ziemian}


% \date{}
% ---------------------------------------------------------------------










% ####################################################################
\begin{document}
% ####################################################################





% ######################################
% Title of the text
\maketitle
% ######################################





% ######################################
\section{Wojciech Kopczyński, Andrzej Trautman
  \textit{Czasoprzestrzeń i~grawitacja},
  \cite{KopczynskiTrautmanCzasoprzetrzenIGrawitacja1984}}

\label{sec:Uwagi-ogolne}
% ######################################



% ##################
\CenterBoldFont{Uwagi do~konkretnych stron}

\vspace{0em}


\noindent
\Str{33} Według mi~się, że~$l$ przebiega tylko zakres $0, n - 1$, ale~to
trzeba sprawdzić.

\VerSpaceFour





\noindent
\Str{37-38}

\VerSpaceFour





\noindent
\Str{51}





% ##################
\newpage

\CenterBoldFont{Błędy}


\begin{center}

  \begin{tabular}{|c|c|c|c|c|}
    \hline
    Strona & \multicolumn{2}{c|}{Wiersz} & Jest
                              & Powinno być \\ \cline{2-3}
    & Od góry & Od dołu & & \\
    \hline
    9   & 13 & & względności : & względności: \\
    12  & &  5 & Galileusza & Galileusza. \\
    25  & &  6 & teorii~) & teorii) \\
    28  & &  5 & sposób~? & sposób? \\
    34  & &  9 & zasada : & zasada: \\
    % & & & & \\
    % & & & & \\
    % & & & & \\
    % & & & & \\
    \hline
  \end{tabular}

\end{center}

\VerSpaceSix


% ############################







% ######################################
\section{Wald}

\label{sec:Uwagi-ogolne}
% ######################################



Jakieś błędy są na stronach 476, 482, 482, 482, 483



% ##################
\newpage

\CenterBoldFont{Błędy}


\begin{center}

  \begin{tabular}{|c|c|c|c|c|}
    \hline
    Strona & \multicolumn{2}{c|}{Wiersz} & Jest
                              & Powinno być \\ \cline{2-3}
    & Od góry & Od dołu & & \\
    \hline
    % & & & & \\
    24 & 2 & & w $\Ocal$ & w $\overline{ \Ocal }$ \\
    24 & 9 & & widzenia $\Ocal$ & widzenia $\overline{ \Ocal }$ \\
    41 & & 10 & zagara & zegara \\
    82 & & 1 & $( a \quad b \quad \ldots )$ & $( a \: b \: \ldots )$ \\
    83 & 2 & & $( a \quad b \quad \ldots )$ & $( a \: b \: \ldots )$ \\
    143 & & 4 & $B\indices{^\mu_{ v ; \, \beta }}$ & $B\indices{ ^\mu_{ \nu, \, \beta } }$ \\
    % & & & & \\
    \hline
  \end{tabular}

\end{center}

\VerSpaceSix


% ############################









% ############################
\section{Bernard F. Schutz,
  \textit{Wstęp do ogólnej teorii względności},
  \cite{SchutzWstepDoOgolnejTeoriiWzglednosci2002}}

\vspace{0em}


% ##################
\CenterBoldFont{Uwagi do konkretnych stron}

\vspace{0em}


\noindent
\Str{35} Wyprowadzenie transformacji Lorentza nie wydaje się poprawne.
Brakuje argumentu który by implikował relację $\sigma = \alpha$.

\VerSpaceFour





\noindent
\Str{62} Dowód, że foton ma zerową masę powinien być opatrzony większym
komentarzem, opiera się on bowiem na interpretacji składowych czterowektora
podanej dla cząstki z~niezerową masą. Dlatego nie można jej tak po prostu
przenieść dla czterowektora świetlnego. Z drugiej strony sama interpretacja
składowych czterowektora o niezerowej masie, nie została potraktowana jako
postulat lub uzasadniona fizycznie, lecz po prostu podana. Może tu też warto
byłoby dodać jakiś komentarz.

\VerSpaceFour





\noindent
\Str{106} Warto byłoby dodać komentarz wyjaśniający dlaczego
wielkości termodynamiczne (określane tu jako skalarne) definiuje~się
zawsze w~układzie CW \textsc{cw} danego fragmentu płynu.

\VerSpaceFour





\noindent
\Str{108} Zdanie „gdy przewodzone jest ciepło, energia będzie niosła pęd”
wymaga głębszego zastanowienia.





% ##################
\newpage

\CenterBoldFont{Błędy}


\begin{center}

  \begin{tabular}{|c|c|c|c|c|}
    \hline
    Strona & \multicolumn{2}{c|}{Wiersz} & Jest
                              & Powinno być \\ \cline{2-3}
    & Od góry & Od dołu & & \\
    \hline
    % & & & & \\
    24 & 2 & & w $\Ocal$ & w $\overline{ \Ocal }$ \\
    24 & 9 & & widzenia $\Ocal$ & widzenia $\overline{ \Ocal }$ \\
    41 & & 10 & zagara & zegara \\
    82 & & 1 & $( a \quad b \quad \ldots )$ & $( a \: b \: \ldots )$ \\
    83 & 2 & & $( a \quad b \quad \ldots )$ & $( a \: b \: \ldots )$ \\
    143 & & 4 & $B\indices{^\mu_{ v ; \, \beta }}$ & $B\indices{ ^\mu_{ \nu, \, \beta } }$ \\
    % & & & & \\
    \hline
  \end{tabular}

\end{center}

\VerSpaceSix


% ############################










% ############################
\section{S.W. Hawking, G.F.R. Ellis,
  \textit{The Large Scale Structure of Space-Time},
  \cite{HawkingEllisLargeScaleStructureOfSpaceTime1973}}


% ##################
\CenterBoldFont{Błędy}


\begin{center}

  \begin{tabular}{|c|c|c|c|c|}
    \hline
    Strona & \multicolumn{2}{c|}{Wiersz} & Jest
    & Powinno być \\ \cline{2-3}
    & Od góry & Od dołu & & \\
    \hline
    % & & & & \\
    24 & 9 & & $( y, 0 )$ & $( 0, y )$ \\
    31 & & 13 & $\tensor[]{\Phi}{ _{c'}^{ c' } }$
    & $\tensor[]{\Phi}{ _{c'}^{ c } }$ \\
    31 & & 12 & $E_{ b }$ & $\Ebold_{ b }$ \\
    31 & & 10 & $E_{ b' }$ & $\Ebold_{ b' }$ \\
    31 & & 6 & $E_{ b' }$ & $\Ebold_{ b' }$ \\
    % & & & & \\
    \hline
  \end{tabular}

\end{center}

\VerSpaceSix


% ############################










% ############################
\section{Richard P. Feynman,
  \textit{Wykłady z grawitacji},
  wydanie \romannumeral1, FWG }

\vspace{0em}


% ##################
\CenterBoldFont{Błędy}

\vspace{0em}


Str. 7. \ldots wynoszący,
$\sqrt{ ( 4 \pi e^{ 2 } / \hbar c) } = 0.31$\ldots ????????????

% \end{center}

\VerSpaceSix



% ############################










% #####################################################################
% #####################################################################
% Bibliografia

\bibliographystyle{plalpha}

\bibliography{PhysicsBooks}{}





% ############################

% Koniec dokumentu
\end{document}
