% Autor: Kamil Ziemian

% ---------------------------------------------------------------------
% Podstawowe ustawienia i pakiety
% ---------------------------------------------------------------------
\RequirePackage[l2tabu, orthodox]{nag}  % Wykrywa przestarzałe i niewłaściwe
% sposoby używania LaTeXa. Więcej jest w l2tabu English version.
\documentclass[a4paper,11pt]{article}
% {rozmiar papieru, rozmiar fontu}[klasa dokumentu]
\usepackage[MeX]{polski}  % Polonizacja LaTeXa, bez niej będzie pracował
% w języku angielskim.
\usepackage[utf8]{inputenc}  % Włączenie kodowania UTF-8, co daje dostęp
% do polskich znaków.
\usepackage{lmodern}  % Wprowadza fonty Latin Modern.
\usepackage[T1]{fontenc}  % Potrzebne do używania fontów Latin Modern.



% ------------------------------
% Podstawowe pakiety (niezwiązane z ustawieniami języka)
% ------------------------------
\usepackage{microtype}  % Twierdzi, że poprawi rozmiar odstępów w tekście.
\usepackage{graphicx}  % Wprowadza bardzo potrzebne komendy do wstawiania
% grafiki.
\usepackage{verbatim}  % Poprawia otoczenie VERBATIME.
\usepackage{textcomp}  % Dodaje takie symbole jak stopnie Celsiusa,
% wprowadzane bezpośrednio w tekście.
\usepackage{vmargin}  % Pozwala na prostą kontrolę rozmiaru marginesów,
% za pomocą komend poniżej. Rozmiar odstępów jest mierzony w calach.
% ------------------------------
% MARGINS
% ------------------------------
\setmarginsrb
{ 0.7in}  % left margin
{ 0.6in}  % top margin
{ 0.7in}  % right margin
{ 0.8in}  % bottom margin
{  20pt}  % head height
{0.25in}  % head sep
{   9pt}  % foot height
{ 0.3in}  % foot sep



% ------------------------------
% Często przydatne pakiety
% ------------------------------
\usepackage{csquotes}  % Pozwala w prosty sposób wstawiać cytaty do tekstu.
\usepackage{xcolor}  % Pozwala używać kolorowych czcionek (zapewne dużo
% więcej, ale ja nie potrafię nic o tym powiedzieć).



% ------------------------------
% Pakiety do tekstów z nauk przyrodniczych
% ------------------------------
\let\lll\undefined  % Amsmath gryzie się z językiem pakietami do języka
% polskiego, bo oba definiują komendę \lll. Aby rozwiązać ten problem
% oddefiniowuję tę komendę, ale może tym samym pozbywam się dużego Ł.
\usepackage[intlimits]{amsmath}  % Podstawowe wsparcie od American
% Mathematical Society (w skrócie AMS)
\usepackage{amsfonts, amssymb, amscd, amsthm}  % Dalsze wsparcie od AMS
% \usepackage{siunitx}  % Do prostszego pisania jednostek fizycznych
\usepackage{upgreek}  % Ładniejsze greckie litery
% Przykładowa składnia: pi = \uppi
\usepackage{slashed}  % Pozwala w prosty sposób pisać slash Feynmana.
\usepackage{calrsfs}  % Zmienia czcionkę kaligraficzną w \mathcal
% na ładniejszą. Może w innych miejscach robi to samo, ale o tym nic
% nie wiem.



% ------------------------------
% Tworzenie otoczeń „Twierdzenie”, „Definicja”, „Lemat”, etc.
\newtheorem{twr}{Twierdzenie}  % Komenda wprowadzająca otoczenie
% „twr” do pisania twierdzeń matematycznych
\newtheorem{defin}{Definicja}  % Analogicznie jak powyżej
\newtheorem{wni}{Wniosek}



% ------------------------------
% Pakiety napisane przez użytkownika.
% Mają być w tym samym katalogu to ten plik .tex
% ------------------------------
% \usepackage{reedsimon}  % Pakiet napisany konkretnie dla tego pliku.
\usepackage{latexgeneralcommands}
\usepackage{mathcommands}

\usepackage{tensor}



% ---------------------------------------------------------------------
% Dodatkowe ustawienia dla języka polskiego
% ---------------------------------------------------------------------
\renewcommand{\thesection}{\arabic{section}.}
% Kropki po numerach rozdziału (polski zwyczaj topograficzny)
\renewcommand{\thesubsection}{\thesection\arabic{subsection}}
% Brak kropki po numerach podrozdziału



% ------------------------------
% Ustawienia różnych parametrów tekstu
% ------------------------------
\renewcommand{\arraystretch}{1.2}  % Ustawienie szerokości odstępów między
% wierszami w tabelach.



% ------------------------------
% Pakiet „hyperref”
% Polecano by umieszczać go na końcu preambuły.
% ------------------------------
\usepackage{hyperref}  % Pozwala tworzyć hiperlinki i zamienia odwołania
% do bibliografii na hiperlinki.










% ---------------------------------------------------------------------
% Tytuł, autor, data
\title{Fizyka czasoprzestrzenni \\
  Błędy i~uwagi}

% \author{}
% \date{}
% ---------------------------------------------------------------------










% ####################################################################
% Początek dokumentu
\begin{document}
% ####################################################################





% ######################################
\maketitle % Tytuł całego tekstu
% ######################################

% „P\ldots” oznacza, że w wydaniu ,,\ldots'' błąd został poprawiony.\\





% ############################
\Work{ % Autor i tytuł dzieła
  Wojciech Kopczyński, Andrzej Trautman \\
  „Czasoprzestrzeń i~grawitacja”,
  \cite{KopczynskiTrautmanCzasoprzetrzenIGrawitacja1984} }



% ##################
\CenterBoldFont{Uwagi do konkretnych stron}


\start \Str{33} Według mi~się, że~$l$ przebiega tylko zakres
$0, n - 1$, ale~to trzeba sprawdzić.

\vspace{\spaceFour}



\start \Str{37-38}

\vspace{\spaceFour}



\start \Str{51}





% ##################
\CenterBoldFont{Błędy}


\begin{center}

  \begin{tabular}{|c|c|c|c|c|}
    \hline
    & \multicolumn{2}{c|}{} & & \\
    Strona & \multicolumn{2}{c|}{Wiersz} & Jest
                              & Powinno być \\ \cline{2-3}
    & Od góry & Od dołu & & \\
    \hline
    9   & 13 & & względności : & względności: \\
    12  & &  5 & Galileusza & Galileusza. \\
    25  & &  6 & teorii~) & teorii) \\
    28  & &  5 & sposób~? & sposób? \\
    34  & &  9 & zasada : & zasada: \\
    % & & & & \\
    % & & & & \\
    % & & & & \\
    % & & & & \\
    \hline
  \end{tabular}

\end{center}


\vspace{\spaceTwo}
% ############################










% ############################
\Work{ % Autor i tytuł dzieła
  Bernard F. Schutz \\
  „Wstęp do ogólnej teorii względności”,
  \cite{SchutzWstepDoOgolnejTeoriiWzglednosci2002} }


% ##################
\CenterBoldFont{Uwagi do konkretnych stron}


\start \Str{35} Wyprowadzenie transformacji Lorentza nie wydaje się
poprawne. Brakuje argumentu który by implikował relację $\sigma = \alpha$.

\vspace{\spaceFour}



\start \Str{62} Dowód, że foton ma zerową masę powinien być
opatrzony większym komentarzem, opiera się on bowiem na
interpretacji składowych czterowektora podanej dla cząstki z~niezerową masą. Dlatego nie można jej tak po prostu przenieść dla
czterowektora świetlnego. Z drugiej strony sama interpretacja
składowych czterowektora o niezerowej masie, nie została
potraktowana jako postulat lub uzasadniona fizycznie, lecz po prostu
podana. Może tu też warto byłoby dodać jakiś komentarz.

\vspace{\spaceFour}



\start \Str{106} Warto byłoby dodać komentarz wyjaśniający dlaczego
wielkości termodynamiczne (określane tu jako skalarne) definiuje~się
zawsze w układzie CW danego fragmentu płynu.

\vspace{\spaceFour}



\start \Str{108} Zdanie „gdy przewodzone jest ciepło, energia
będzie niosła pęd” wymaga głębszego zastanowienia.





% ##################
\CenterBoldFont{Błędy}


\begin{center}

  \begin{tabular}{|c|c|c|c|c|}
    \hline
    & \multicolumn{2}{c|}{} & & \\
    Strona & \multicolumn{2}{c|}{Wiersz} & Jest
                              & Powinno być \\ \cline{2-3}
    & Od góry & Od dołu & & \\
    \hline
    % & & & & \\
    24 & 2 & & w $\Ocal$ & w $\overline{ \Ocal }$ \\
    24 & 9 & & widzenia $\Ocal$ & widzenia $\overline{ \Ocal }$ \\
    41 & & 10 & zagara & zegara \\
    82 & & 1 & $( a \quad b \quad \ldots )$ & $( a \: b \: \ldots )$ \\
    83 & 2 & & $( a \quad b \quad \ldots )$ & $( a \: b \: \ldots )$ \\
    143 & & 4 & $B\indices{^\mu_\nu_{;\, \beta}}$ & $B\indices{^\mu_\nu_{,\, \beta}}$ \\
    % & & & & \\
    \hline
  \end{tabular}

\end{center}


\vspace{\spaceTwo}
% ############################










% ############################
\Work{ % Autor i tytuł dzieła
  S. W. Hawking, G. F. R. Ellis \\
  „The Large Scale Structure of Space-Time”,
  \cite{HawkingEllisLargeScaleStructureOfSpaceTime1973} }


% ##################
\CenterBoldFont{Błędy}


\begin{center}

  \begin{tabular}{|c|c|c|c|c|}
    \hline
    & \multicolumn{2}{c|}{} & & \\
    Strona & \multicolumn{2}{c|}{Wiersz} & Jest & Powinno być \\ \cline{2-3}
    & Od góry & Od dołu &  &  \\
    \hline
    % & & & & \\
    24 & 9 & & $( y, 0 )$ & $( 0, y )$ \\
    31 & & 13 & $\tensor[]{\Phi}{ _{c'}^{c'} }$ & $\tensor[]{\Phi}{ _{c'}^{c} }$ \\
    31 & & 12 & $E_{ b }$ & $\Ebold_{ b }$ \\
    31 & & 10 & $E_{ b' }$ & $\Ebold_{ b' }$ \\
    31 & & 6 & $E_{ b' }$ & $\Ebold_{ b' }$ \\
    % & & & & \\
    \hline
  \end{tabular}

\end{center}


\vspace{\spaceTwo}
% ############################










% ############################
\Work{ % Autor i tytuł dzieła
  Richard P. Feynman \\
  „Wykłady z grawitacji”, \\
  wydanie \romannumeral1, FWG }


% ##################
\CenterBoldFont{Błędy}


Str. 7. \ldots wynoszący,
$\sqrt{ ( 4 \pi e^{ 2 } / \hbar c) } = 0.31$\ldots ????????????

% \end{center}


\vspace{\spaceTwo}
% ############################










% #####################################################################
% #####################################################################
% Bibliografia
\bibliographystyle{alpha} \bibliography{PhilNaturBooks}{}


% ############################

% Koniec dokumentu
\end{document}
