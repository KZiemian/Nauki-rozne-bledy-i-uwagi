% Autor: Kamil Ziemian

% ---------------------------------------------------------------------
% Podstawowe ustawienia i pakiety
% ---------------------------------------------------------------------
\RequirePackage[l2tabu, orthodox]{nag}  % Wykrywa przestarzałe i niewłaściwe
% sposoby używania LaTeXa. Więcej jest w l2tabu English version.
\documentclass[a4paper,11pt]{article}
% {rozmiar papieru, rozmiar fontu}[klasa dokumentu]
\usepackage[MeX]{polski}  % Polonizacja LaTeXa, bez niej będzie pracował
% w języku angielskim.
\usepackage[utf8]{inputenc}  % Włączenie kodowania UTF-8, co daje dostęp
% do polskich znaków.
\usepackage{lmodern}  % Wprowadza fonty Latin Modern.
\usepackage[T1]{fontenc}  % Potrzebne do używania fontów Latin Modern.



% ------------------------------
% Podstawowe pakiety (niezwiązane z ustawieniami języka)
% ------------------------------
\usepackage{microtype}  % Twierdzi, że poprawi rozmiar odstępów w tekście.
\usepackage{graphicx}  % Wprowadza bardzo potrzebne komendy do wstawiania
% grafiki.
\usepackage{verbatim}  % Poprawia otoczenie VERBATIME.
\usepackage{textcomp}  % Dodaje takie symbole jak stopnie Celsiusa,
% wprowadzane bezpośrednio w tekście.
\usepackage{vmargin}  % Pozwala na prostą kontrolę rozmiaru marginesów,
% za pomocą komend poniżej. Rozmiar odstępów jest mierzony w calach.
% ------------------------------
% MARGINS
% ------------------------------
\setmarginsrb
{ 0.7in}  % left margin
{ 0.6in}  % top margin
{ 0.7in}  % right margin
{ 0.8in}  % bottom margin
{  20pt}  % head height
{0.25in}  % head sep
{   9pt}  % foot height
{ 0.3in}  % foot sep



% ------------------------------
% Często przydatne pakiety
% ------------------------------
\usepackage{csquotes}  % Pozwala w prosty sposób wstawiać cytaty do tekstu.
\usepackage{xcolor}  % Pozwala używać kolorowych czcionek (zapewne dużo
% więcej, ale ja nie potrafię nic o tym powiedzieć).



% ------------------------------
% Pakiety do tekstów z nauk przyrodniczych
% ------------------------------
\let\lll\undefined  % Amsmath gryzie się z językiem pakietami do języka
% polskiego, bo oba definiują komendę \lll. Aby rozwiązać ten problem
% oddefiniowuję tę komendę, ale może tym samym pozbywam się dużego Ł.
\usepackage[intlimits]{amsmath}  % Podstawowe wsparcie od American
% Mathematical Society (w skrócie AMS)
\usepackage{amsfonts, amssymb, amscd, amsthm}  % Dalsze wsparcie od AMS
% \usepackage{siunitx}  % Do prostszego pisania jednostek fizycznych
\usepackage{upgreek}  % Ładniejsze greckie litery
% Przykładowa składnia: pi = \uppi
\usepackage{slashed}  % Pozwala w prosty sposób pisać slash Feynmana.
\usepackage{calrsfs}  % Zmienia czcionkę kaligraficzną w \mathcal
% na ładniejszą. Może w innych miejscach robi to samo, ale o tym nic
% nie wiem.



% ------------------------------
% Tworzenie otoczeń „Twierdzenie”, „Definicja”, „Lemat”, etc.
\newtheorem{twr}{Twierdzenie}  % Komenda wprowadzająca otoczenie
% „twr” do pisania twierdzeń matematycznych
\newtheorem{defin}{Definicja}  % Analogicznie jak powyżej
\newtheorem{wni}{Wniosek}



% ------------------------------
% Pakiety napisane przez użytkownika.
% Mają być w tym samym katalogu to ten plik .tex
% ------------------------------
% \usepackage{reedsimon}  % Pakiet napisany konkretnie dla tego pliku.
\usepackage{latexgeneralcommands}
\usepackage{mathcommands}

% \usepackage{mechanika}  % Pakiet napisany konkretnie dla tego pliku.

% \usepackage{tensor}



% ---------------------------------------------------------------------
% Dodatkowe ustawienia dla języka polskiego
% ---------------------------------------------------------------------
\renewcommand{\thesection}{\arabic{section}.}
% Kropki po numerach rozdziału (polski zwyczaj topograficzny)
\renewcommand{\thesubsection}{\thesection\arabic{subsection}}
% Brak kropki po numerach podrozdziału



% ------------------------------
% Ustawienia różnych parametrów tekstu
% ------------------------------
\renewcommand{\arraystretch}{1.2}  % Ustawienie szerokości odstępów między
% wierszami w tabelach.



% ------------------------------
% Pakiet „hyperref”
% Polecano by umieszczać go na końcu preambuły.
% ------------------------------
\usepackage{hyperref}  % Pozwala tworzyć hiperlinki i zamienia odwołania
% do bibliografii na hiperlinki.










% ---------------------------------------------------------------------
% Tytuł, autor, data
\title{Mechanika Newtona~-- błędy i~uwagi}

% \author{}
% \date{}
% ---------------------------------------------------------------------










% ####################################################################
\begin{document}
% ####################################################################





% ######################################
\maketitle % Tytuł całego tekstu
% ######################################





% ######################################
\section{Kanoniczne prace}
% Tytuł danego działu

% \vspace{\spaceTwo}
\vspace{\spaceThree}
% ######################################



% ############################
\Work{ % Autor i tytuł dzieła
  Isaac Newton \\
  „Matematyczne zasady filozofii przyrody”,
  \cite{NewtonMatematyczneZasadyFilozofiiPrzyrody2011} }


% ##################
\CenterBoldFont{Błędy}


\begin{center}

  \begin{tabular}{|c|c|c|c|c|}
    \hline
    & \multicolumn{2}{c|}{} & & \\
    Strona & \multicolumn{2}{c|}{Wiersz} & Jest
                              & Powinno być \\ \cline{2-3}
    & Od góry & Od dołu & & \\
    \hline
    16  & &  2 & Dodajęy & Dodaję \\
    % & & & & \\
    % & & & & \\
    % & & & & \\
    % & & & & \\
    \hline
  \end{tabular}

\end{center}


\vspace{\spaceTwo}
% ############################










% ######################################
\newpage
\section{Matematyczne ujęcie mechaniki Newtona}

\vspace{\spaceTwo}
% \vspace{\spaceThree}
% ######################################



% ############################
\Work{ % Autor i tytuł dzieła
  Władimir Arnold \\
  „Metody matematyczne mechaniki klasycznej”,
  \cite{ArnoldMetodyMatematyczneMechanikiKlasycznej1981} }


% ##################
\CenterBoldFont{Uwagi}


\start \textbf{Rozdział 7.} W~tym rozdziale nie znalazłem dowodu, ani
żadnej wskazówki, że~należy samemu pokazać, iż w~lokalnym układzie
współrzędnych zachodzi dobrze znany wzór:
\begin{equation*}
  \label{eq:ArnoldMetodyMatematyczne-01}
  d f = \partial_{ i } f\, d x^{ i }.
\end{equation*}
Zastosowanie tego wzoru znacznie ułatwia rozwiązywanie dalszych zadań
w~tym rozdziale, a~niektóre nie wiem nawet jak zrobić bez niego.

\vspace{\spaceFour}



\start \textbf{Str. 71.} W~twierdzeniu Poincar\'{e}go o~powracaniu
założenie o~ciągłości $g$ wydaje się bardzo nienaturalne. Wydaje się,
że~najlepiej jest je zamienić na żądanie mierzalności tej funkcji.

\vspace{\spaceFour}





% ##################
\CenterBoldFont{Błędy}


\begin{center}

  \begin{tabular}{|c|c|c|c|c|}
    \hline
    & \multicolumn{2}{c|}{} & & \\
    Strona & \multicolumn{2}{c|}{Wiersz} & Jest
                              & Powinno być \\ \cline{2-3}
    & od góry & od dołu & & \\
    \hline
    12  & &  2 & matematycz netak & matematyczne tak \\
    18  &  3 & & $\Phi( \vecxbold, \dot{ \vecxbold } )$
           & % $\mathbf{F}( \mathbf{ x }, \dot{ \mathbf{ x } } )$
    \\
    19  & &  1 & $\vecgbold\, \vecxbold$ & $-\vecgbold\, \vecxbold$ \\
    23  &  4 & & $f( \dot{ x } )$ & $f( x )$ \\
    24  &  1 & & Narysujem y & Narysujemy \\
    34  & &  2 & zorientowane & zorientowanej \\
    37  & & 10 & $\ddot{ \vecrbold } - r \dot{ \varphi }^{ 2 }$
           & $\ddot{ r } - r \dot{ \varphi }^{ 2 }$ \\
    37  & &  6 & $\dot{ r } -$ & $\ddot{ r } -$ \\
    60  & 10 & & napsać & napisać \\
    61  & &  8 & $\sqrt{ ( \dot{ q }_{ 1 }^{ 2 } + \dot{ q }_{ 2 }^{ 2 }
                 + \dot{ q }_{ 3 }^{ 2 } ) }$
           & $\frac{ 1 }{ 2 } m ( \dot{ q }_{ 1 }^{ 2 } + \dot{ q }_{ 2 }^{ 2 }
             + \dot{ q }_{ 3 }^{ 2 } )$ \\
    62  &  2 & & $m\, \dot{ \vecrbold }$ & $m\, \dot{ r }$ \\
    64  & &  8 & $G( x, p )$ & $G( x_{ 0 }, p )$ \\
    77  & &  9 & $S^{ 2 }$ & $S^{ 1 }$ \\
    % & & & & \\
    % & & & & \\
    % & & & & \\
    % & & & & \\
    81  &  5 & & $TM$ & $TM_{ x }$ \\
    81  &  9 & & $\mathbf{ \eta_{ i } }$ & $\eta_{ i }$ \\
    81  & &  5 & związką & wiązką \\
    81  & &  3 & $\vectbold_{ 0 }$ & $t_{ 0 }$ \\
    82  & 15 & & $m_{ 1 }${  }, & $m_{ 1 }$ \\
    86  &  2 & & \textit{Lagrange’a, to $( M, L )$}
           & \textit{Lagrange’a $( M, L )$, to} \\
    98  & &  3 & $\omega^{ 2 }$ & $\omega_{ 0 }^{ 2 }$ \\
    124 & 13 & & $\mathbf{Q}$ & $Q$ \\
    165 & &  9 & postc & postaci \\
    169 & & 10 & Prykład & Przykład \\
    186 & & 11 & $T^{ * } V$ & $T^{ * } V_{ x }$ \\
    225 & 13 & & $H( \partial L / \partial \dot{ \vecpbold }, \vecqbold )$
           & $H( \partial L / \partial \dot{ \vecqbold }, \vecqbold )$ \\
    242 &  3 & & \textit{Jacobiego} & Jacobiego \\
    268 & &  8 & $g$ & $\vecgbold$ \\
    % 291 & & & & \\ % Jak się pisze w LaTeXu cyrlicą?
    351 &  1 & & $P_{ * }TM_{ X }$ & $P_{ * }TM_{ x }$ \\
    351 &  1 & & $T\gfrak_{ p }^{ * })$ & $T\gfrak_{ p }^{ * }$ \\
    373 & 18 & & A.~Arez & A.~Avez \\
    395 & & 11 & \textit{Poincar\'{e}'s} & \textit{Poincar\'{e}s} \\
    % & & & & \\
    \hline
  \end{tabular}

\end{center}


\noindent
\StrWd{29}{4} \\
\Jest  tworzy sferę dwuwymiarową. \\
\Powin można przekształcić w~sferę dwuwymiarową. \\
\StrWd{42}{10} \\
\Jest  Słońce znajduje~się nie w~centrum \\
\Powin ale~Słońce nie znajduje~się w~centrum \\
\StrWd{71}{15} \\
\Jest  do swego\ldots \\
\Powin dowolnie blisko swego\ldots \\


\vspace{\spaceTwo}
% ############################










% ############################
\Work{ % Autorzy i tytuł dzieła
  Roman Stanisław Ingarden, Andrzej Jamiołkowski \\
  „Mechanika klasyczna”,
  \cite{IngardenJamiolkowskiMechanikaKlasyczna1980} }


% ##################
\CenterBoldFont{Uwagi do konkretnych stron}


\start \Str{9--12}

\vspace{\spaceFour}



\start Str. 19. Bardzo ciężko jest zrozumieć uwagę, że w dwóch układach pochodne po czasie są różne, pomimo iż czas płynie tak samo. Proponuję następujące wyjaśnienie tego problemu:

Zauważmy, że w dwóch różnych układach odniesienia $x$ oraz
$\tilde{ x }$ będą różnymi funkcjami czasu (na razie zostawiamy na
boku głębszą dyskusję ontologicznej natury wykonywanych tu operacji).
Wytłumaczmy to na przykładzie: niech $\tilde{ x }$ będzie niezerowym
wektorem i niech układ $\tilde{ \Ocal }$ wykonuje obrót wokół
$\tilde{ p }_{ 0 }$. Teraz w układzie $\Ocal$ $\tilde{ x }$
jest wektorem o stałych współrzędnych, podczas gdy w układzie
$\tilde{ \Ocal }$ dokonuje on obrotu. Podobnie wektory bazy
układu $\tilde{ \Ocal }$ są postrzegane jako nieruchome w tym
układzie, lecz jako obracające się w
układzie $\Ocal$.

(Dyskusja ta wymaga udoskonalenia). Zauważmy, że każda pochodna ma
człon wynikający z różniczkowania współrzędnych i wektorów bazy.
Jeżeli więc mamy dany jakąś funkcje wektorową jako funkcję czasu, to
od wyboru układu odniesienia zależy nie tylko postać funkcyjna
współrzędnych, ale też czy mamy różniczkować dane wektory. W pewnym
sensie (bo do tej pory wszystko to jest niedoprecyzowane) pochodne
konkretnych funkcji skalarnych są takie same w każdym układzie
odniesienia, bo nie wchodzi do nich pochodna wektorów bazy.



Str. 20.
$\frac{ d\vecebold_{ 1 } }{ dt } = \vecomegabold \times
\vecebold_{ 1 } \, ,$

Str. 20.
$\frac{ \tilde{ \dPL } \tilde{ \vecxbold } }{ \dPL t }
= \frac{ d\tilde{ x }^{ i } }{ dt } \vecebold_{ i } \, ,$

Str. 20. \ldots także z faktu, że
$\dPL \tilde{ x }^{ i } / \dPL t = \tilde{ \dPL } \tilde{ x
}^{ i } / \dPL t$\ldots

Str. 21.
$\vecvbold = \tilde{ \vecvbold } + \vecvbold_{ 0 } +
\vecomegabold \times \tilde{ \vecxbold } \, ,$

Str. 21. ????
$\frac{ d\vecvbold }{ dt } = \frac{ d\tilde{ \vecvbold } }{ dt }
+ \vecomegabold \times \tilde{ \vecvbold } + \frac{
  d\vecvbold_{ 0 } }{ dt } + \frac{ \vecomegabold }{ t }
\times \tilde{ \vecxbold } + \vecomegabold  \times \bigg(
\frac{ \tilde{ \dPL } \tilde{ \vecxbold } }{ \dPL t } +
\vecomegabold \times \tilde{ \vecxbold } \bigg) \, .$
Sprawdzić.

Str. 24. Obraz odwzorowanie
$X : T \rightarrow E^{ 3N }$\ldots

Str. 28. \ldots chwili $t \in T$ funkcje\ldots

Str. 36. \ldots oraz że nie zależy on od wyboru układu
współrzędnych\ldots


\vspace{\spaceTwo}
% ############################










% ############################
\Work{ % Autorzy i tytuł dzieła
  J. I. Nejmark, N. A. Fufajew \\
  „Dynamika układów nieholonomicznych”,
  \cite{NejmarkFufajewDynamikaUkladowNieholonomicznych1971} }


% ##################
\CenterBoldFont{Błędy}


\begin{center}

  \begin{tabular}{|c|c|c|c|c|}
    \hline
    & \multicolumn{2}{c|}{} & & \\
    Strona & \multicolumn{2}{c|}{Wiersz}
                            & Jest & Powinno być \\ \cline{2-3}
    & Od góry & Od dołu & & \\
    \hline
    9   &  6 & & i wielu & wielu \\
    9   &  7 & & nczonych & uczonych \\
    11  & &  1 & $\delta$ & $\theta$ \\
    12  &  2 & & prędkość & przyśpieszenie \\
    12  &  3 & & równa & równe \\
    % & & & & \\
    % & & & & \\
    % & & & & \\
    % & & & & \\
    \hline
  \end{tabular}

\end{center}


\vspace{\spaceTwo}
% ############################










% ######################################
\newpage
\section{Książki powstałe po~1945~r.}

\vspace{\spaceTwo}
% \vspace{\spaceThree}

% ######################################



% ############################
\Work{ % Autorzy i tytuł dzieła
  Lew D. Landau, Jewginij M. Lifszyc \\
  „Mechanika”, \cite{LandauLifszycMechanika2006} }


% ##################
\CenterBoldFont{Uwagi do konkretnych stron}


\start \Str{14} Podana tu grupa Galileusza składa~się tylko z~pchnięć,
co według mnie tylko zaciemnia strukturę symetrii czasoprzestrzeni
Galileusza. Pełniejsze omówienie tej grupy można znaleźć w~książce
W.~Arnolda ,,Metody matematyczne mechaniki klasycznej''
\cite{ArnoldMetodyMatematyczneMechanikiKlasycznej1981}.

\vspace{\spaceFour}



\start \Str{13} Przemyślenie jest głębokie, ale przedstawione
stanowczo zbyt krótko, aby było jasne. Spróbuję przedstawić tu pewne
jego rozwinięcie.

Przed wszystkim należy zauważyć, że należy tu rozróżnić jednorodność
i~izotropowość w sensie geometrii przestrzeni i w sensie dynamiki.
Cechy te traktowane jako cechy geometrii czasoprzestrzeni w sensie
geometrii liniowej i różniczkowej, są niezależne od układu
odniesienia. Przejdźmy teraz do problemu dynamiki. Po pierwsze z
doświadczenia wiemy, że możemy przyjąć, iż przestrzeń jest
euklidesowa, jak również że można znaleźć układ odniesienia w którym
cząstki swobodne umieszczone w przestrzeni spoczywają.

\vspace{\spaceFour}



\start \Str{14} $\frac{ \partial L }{ \partial \vecvbold }$ nie jest funkcją tylko
kwadratu prędkości. Jest to wektor o składowych
$( \frac{ \partial L }{ \partial \vecvbold } )_{ i } = 2 \frac{
  \partial L }{ \partial { v^{ 2 } } } v_{ i }$, czyli zależy on
jawnie od składowych prędkości. Widać jednak, że stałość
$\frac{ \partial L }{ \partial \vecvbold }$ wymaga od nas stałości
$\mathbf{ v }$. Jeżeli bowiem rozpatrzymy składową $x$ wektora
(ściślej pola wektorowego)
$\frac{ \partial L }{ \partial \vecvbold }$, mamy warunek na stałość
tego wyrażenia dla dowolnej wartości składowej $x$:
$\frac{ \partial L }{ \partial { v^{ 2 } } } = \frac{ 1 }{ 2
  \vecvbold_{ x } }$. Wyrażenie to należy zakwestionować na paru
poziomach, choćby dlatego, że jest osobliwe dla zerowych prędkości, co
jest niedopuszczalne dla fizycznej teorii. Oczywiście, jeżeli
sprawdzimy również warunek na $y$ składową otrzymamy sprzeczny układ
równań.

\vspace{\spaceFour}



\start \Str{14} Należałoby podać większą dyskusję prędkości względnej
dwóch układów inercjalnych.

\vspace{\spaceFour}



\start \Str{15} Jak można ściślej uzasadnić, że rzeczywiście
potrzebujemy liniowej zależności od prędkości prawej strony równania
wyrażającego równoważność między dwoma lagrażjanami? \Dok

\vspace{\spaceFour}



\start \Str{22} Dyskusja ważności addytywnych zasad zachowania, ma
swoją głębię i wagę, zaciemnia ona jednak pewne szczegóły. Autorzy gdy
ją pisali musieli mieć na myśli procesy rozpraszania, nie wspomnieli
jednak, że jeśli znana jest postać oddziaływania między dwoma
cząstkami, również mamy możliwość wyciągnięcia z praw zachowania
ważnych wniosków. Np. jeśli rozpatrujemy układ dwóch cząstek i znamy
energię kinetyczną jednej z nich i energię oddziaływania, to możemy
obliczyć pewne parametry ruchu drugiej.





% ##################
\CenterBoldFont{Błędy}


\begin{center}

  \begin{tabular}{|c|c|c|c|c|}
    \hline
    & \multicolumn{2}{c|}{} & & \\
    Strona & \multicolumn{2}{c|}{Wiersz} & Jest
                              & Powinno być \\ \cline{2-3}
    & Od góry & Od dołu & & \\
    \hline
    56  & 12 & & poruszały się z tą samą prędkością & spoczywały \\
    % & & & & \\
    % & & & & \\
    \hline
  \end{tabular}

\end{center}


\noindent
\StrWd{27}{2} \\
\Jest $S = S' + \mu \vecVbold \cdot \vecRbold' + \frac{ 1 }{ 2 } \mu V^{ 2 } t$ \\
\Powin $S = S' + \mu \vecVbold \cdot \vecRbold'( t ) - \mu \vecVbold
\cdot \vecRbold'( 0 ) + \frac{ 1 }{ 2 } \mu V^{ 2 } t$ \\


\vspace{\spaceTwo}
% ############################










% ############################
\Work{ % Autor i tytuł dzieła
  Bogdan Skalmierski \\
  „Mechanika z~wytrzymałością materiałów”,
  \cite{SkalmierskiMechanikaZWytrzymalosciaMaterialow1983} }


% ##################
\CenterBoldFont{Uwagi do konkretnych stron}


\start \Str{21} We~wzorze w~drugiej linii zamiast
\begin{equation}
  \label{eq:SkalmierskiMechanikaZWytrzymaloscia-01}
  \sqrt{ 1 - \left( \tfrac{ x }{ x_{ 0 } } \right)^{ 2 } }
\end{equation}
powinno być
\begin{equation}
  \label{eq:SkalmierskiMechanikaZWytrzymaloscia-02}
  \sgn( \cos \varphi ) \,
  \sqrt{ 1 - \left( \tfrac{ x }{ x_{ 0 } } \right)^{ 2 } },
\end{equation}
bo~wykorzystujemy jedynkę trygonometryczną by~wyrazić $\cos$ przez
$\sin$. Ponieważ w~dalszym ciągu obliczeń podnosimy ten człon do
kwadratu, ta niedokładność nie~wpływa na ostateczny wynik.

\vspace{\spaceFour}



\start \Str{36} Aby wyprowadzenie wzoru (3.21) było poprawne,
potrzebujemy by
$| \dot{ \vecebold }_{ 1 } \cdot \vecebold_{ 2 } | = \dot{ \vecebold }_{ 1 }
\cdot \vecebold_{ 2 }$. Oznacza to, że~układ obraca się od~wektora $\vecebold_{ 1 }$
do~$\vecebold_{ 2 }$.




% ##################
\CenterBoldFont{Błędy}


\begin{center}

  \begin{tabular}{|c|c|c|c|c|}
    \hline
    & \multicolumn{2}{c|}{} & & \\
    Strona & \multicolumn{2}{c|}{Wiersz} & Jest
                              & Powinno być \\ \cline{2-3}
    & Od góry & Od dołu & & \\
    \hline
    12  & 16 & & można określić & będziemy oznaczali \\
    13  & & 10 & $\vecbbold_{ y }$ & $\vecabold_{ y }$ \\
    17  &  3 & & $( \vecabold \cdot \veccbold ) \cdot \vecbbold
                 - ( \veccbold \cdot \vecbbold ) \cdot \vecabold$
           & $( \vecabold \cdot \veccbold ) \vecbbold
             - ( \veccbold \cdot \vecbbold ) \vecabold$ \\
    21  &  4 & & $\frac{ y }{ { }_{ 0 } }$ & $\frac{ y }{ { y }_{ 0 } }$ \\
    21  &  4 & & $\sqrt{ 1 \:\: \left( \frac{ x }{ x_{ 0 } } \right)^{ 2 } }$
           & $\sqrt{ 1 - \left( \frac{ x }{ x_{ 0 } } \right)^{ 2 } } $ \\
    23  & &  9 & $+2\beta \cos( 2\omega t )$ & $-2\beta \cos( 2\omega t )$ \\
    31  & 13 & & $x_{ 2 } \frac{ \partial x_{ 2 } }{ \partial r }$
           & $\dot{ x }_{ 2 } \frac{ \partial x_{ 2 } }{ \partial r }$ \\
    31  & &  5 & $\vecabold \frac{ \partial \vecrbold }{ \partial q_{ j } }$
           & $\vecabold \cdot \frac{ \partial \vecrbold }{ \partial q_{ j } }$ \\
    32  &  8 & & $\frac{ \partial \vecrbold^{ 2 } }{ \partial { q^{ j } } }$
           & $\frac{ \partial \vecrbold }{ \partial { q^{ j } } }$ \\
    34  &  6 & & $\dot{ \vecebold }_{ i } \vecebold_{ j }$
           & $\dot{ \vecebold }_{ i } \cdot \vecebold_{ j }$ \\
    34  &  8 & & $\vecebold_{ i } \vecebold_{ j }$
           & $\vecebold_{ i } \cdot \vecebold_{ j }$ \\
    34  & 10 & & $\dot{ \vecebold }_{ i } \vecebold_{ j }
                 + \vecebold_{ i } \dot{ \vecebold }_{ j }$
           & $\dot{ \vecebold }_{ i } \cdot \vecebold_{ j }
             + \vecebold_{ i } \cdot \dot{ \vecebold }_{ j }$ \\
    34  & 11 & & $\dot{ \vecebold }_{ i } \vecebold_{ j }$
           & $\dot{ \vecebold }_{ i } \cdot \vecebold_{ j }$ \\
    34  & 12 & & $\dot{ \vecebold }_{ i } \vecebold_{ j }$
           & $\dot{ \vecebold }_{ i } \cdot \vecebold_{ j }$ \\
    36  & 16 & & $( \xi_{ 1 } \dot{ \vecebold }_{ 1 }
                 + \xi_{ 2 } \dot{ \vecebold }_{ 2 }  ) \vecebold_{ 1 }$
           & $( \xi_{ 1 } \dot{ \vecebold }_{ 1 }
             + \xi_{ 2 } \dot{ \vecebold }_{ 2 }  ) \cdot \vecebold_{ 1 }$ \\
    36  & 16 & & $( \xi_{ 1 } \dot{ \vecebold }_{ 1 }
                 + \xi_{ 2 } \dot{ \vecebold }_{ 2 }  ) \vecebold_{ 2 }$
           & $( \xi_{ 1 } \dot{ \vecebold }_{ 1 }
             + \xi_{ 2 } \dot{ \vecebold }_{ 2 }  ) \cdot \vecebold_{ 2 }$ \\
    36  & 18 & & $\dot{ \vecebold }_{ 1 } \vecebold_{ 2 }$
           & $\dot{ \vecebold }_{ 1 } \cdot \vecebold_{ 2 }$ \\
    36  & 20 & & $\dot{ \vecebold }_{ 1 } \vecebold_{ 2 }$
           & $\dot{ \vecebold }_{ 1 } \cdot \vecebold_{ 2 }$ \\
           %        % & & & & \\
    \hline
  \end{tabular}

\end{center}


\noindent
\StrWg{31}{13} \\[0.3em]
\Jest
$( \dot{ x }_{ 1 } \vecibold + \dot{ x }_{ 2 } \vecjbold )
\left( \frac{ \partial x_{ 1 } }{ \partial r } \vecibold
  + \frac{ \partial x_{ 2 } }{ \partial r } \vecjbold \right)
\cdot \frac{ 1 }{ \vecOnebold }$ \\[0.5em]
\Powin
$( \dot{ x }_{ 1 } \vecibold + \dot{ x }_{ 2 } \vecjbold )
\cdot \left( \frac{ \partial x_{ 1 } }{ \partial r } \vecibold
  + \frac{ \partial x_{ 2 } }{ \partial r } \vecjbold \right)
\frac{ 1 }{ | \vecOnebold | }$ \\


\vspace{\spaceTwo}
% ############################










% #####################################################################
% #####################################################################
% Bibliografia
\bibliographystyle{plalpha} \bibliography{PhilNaturBooks}{}


% ############################

% Koniec dokumentu
\end{document}
