% ---------------------------------------------------------------------
% Podstawowe ustawienia i pakiety
% ---------------------------------------------------------------------
\RequirePackage[l2tabu, orthodox]{nag}  % Wykrywa przestarzałe i niewłaściwe
% sposoby używania LaTeXa. Więcej jest w l2tabu English version.
\documentclass[a4paper,11pt]{article}
% {rozmiar papieru, rozmiar fontu}[klasa dokumentu]
\usepackage[MeX]{polski}  % Polonizacja LaTeXa, bez niej będzie pracował
% w języku angielskim.
\usepackage[utf8]{inputenc}  % Włączenie kodowania UTF-8, co daje dostęp
% do polskich znaków.
\usepackage{lmodern}  % Wprowadza fonty Latin Modern.
\usepackage[T1]{fontenc}  % Potrzebne do używania fontów Latin Modern.



% ------------------------------
% Podstawowe pakiety (niezwiązane z ustawieniami języka)
% ------------------------------
\usepackage{microtype}  % Twierdzi, że poprawi rozmiar odstępów w tekście.
\usepackage{graphicx}  % Wprowadza bardzo potrzebne komendy do wstawiania
% grafiki.
\usepackage{verbatim}  % Poprawia otoczenie VERBATIME.
\usepackage{textcomp}  % Dodaje takie symbole jak stopnie Celsiusa,
% wprowadzane bezpośrednio w tekście.
\usepackage{vmargin}  % Pozwala na prostą kontrolę rozmiaru marginesów,
% za pomocą komend poniżej. Rozmiar odstępów jest mierzony w calach.
% ------------------------------
% MARGINS
% ------------------------------
\setmarginsrb
{ 0.7in}  % left margin
{ 0.6in}  % top margin
{ 0.7in}  % right margin
{ 0.8in}  % bottom margin
{  20pt}  % head height
{0.25in}  % head sep
{   9pt}  % foot height
{ 0.3in}  % foot sep



% ------------------------------
% Często przydatne pakiety
% ------------------------------
% \usepackage{csquotes}  % Pozwala w prosty sposób wstawiać cytaty do tekstu.
% \usepackage{xcolor}  % Pozwala używać kolorowych czcionek (zapewne dużo
% % więcej, ale ja nie potrafię nic o tym powiedzieć).



% ------------------------------
% Pakiety do tekstów z nauk przyrodniczych
% ------------------------------
\let\lll\undefined  % Amsmath gryzie się z językiem pakietami do języka
% polskiego, bo oba definiują komendę \lll. Aby rozwiązać ten problem
% oddefiniowuję tę komendę, ale może tym samym pozbywam się dużego Ł.
\usepackage[intlimits]{amsmath}  % Podstawowe wsparcie od American
% Mathematical Society (w skrócie AMS)
\usepackage{amsfonts, amssymb, amscd, amsthm}  % Dalsze wsparcie od AMS
% \usepackage{siunitx}  % Do prostszego pisania jednostek fizycznych
\usepackage{upgreek}  % Ładniejsze greckie litery
% Przykładowa składnia: pi = \uppi
\usepackage{slashed}  % Pozwala w prosty sposób pisać slash Feynmana.
\usepackage{calrsfs}  % Zmienia czcionkę kaligraficzną w \mathcal
% na ładniejszą. Może w innych miejscach robi to samo, ale o tym nic
% nie wiem.
\usepackage{siunitx}



% ------------------------------
% Tworzenie środowisk (?) „Twierdzenie”, „Definicja”, „Lemat”, etc.
% ------------------------------
\newtheorem{theorem}{Twierdzenie}  % Komenda wprowadzająca otoczenie
% „theorem” do pisania twierdzeń matematycznych
\newtheorem{definition}{Definicja}  % Analogicznie jak powyżej
\newtheorem{corollary}{Wniosek}



% ------------------------------
% Pakiety napisane przez użytkownika.
% Mają być w tym samym katalogu to ten plik .tex
% ------------------------------
\usepackage{latexgeneralcommands}
\usepackage{mathcommands}

% \usepackage{mechanika}  % Pakiet napisany konkretnie dla tego pliku.

% \usepackage{tensor}



% ---------------------------------------------------------------------
% Dodatkowe ustawienia dla języka polskiego
% ---------------------------------------------------------------------
\renewcommand{\thesection}{\arabic{section}.}
% Kropki po numerach rozdziału (polski zwyczaj topograficzny)
\renewcommand{\thesubsection}{\thesection\arabic{subsection}}
% Brak kropki po numerach podrozdziału



% ------------------------------
% Ustawienia różnych parametrów tekstu
% ------------------------------
\renewcommand{\baselinestretch}{1.1}

\renewcommand{\arraystretch}{1.4} % Ustawienie szerokości odstępów między
% wierszami w tabelach.



% ------------------------------
% Pakiet „hyperref”
% Polecano by umieszczać go na końcu preambuły.
% ------------------------------
\usepackage{hyperref}  % Pozwala tworzyć hiperlinki i zamienia odwołania
% do bibliografii na hiperlinki.










% ---------------------------------------------------------------------
% Tytuł, autor, data
\title{Mechanika Newtona \\
  {\Large Błędy i~uwagi}}

\author{Kamil Ziemian}


% \date{}
% ---------------------------------------------------------------------










% ####################################################################
\begin{document}
% ####################################################################





% ######################################
\maketitle % Tytuł całego tekstu
% ######################################





% ######################################
\section{Rozważania ogólne}
% Tytuł danego działu

\vspace{\spaceTwo}
% ######################################



Omawiają mechanikę Newtona warto przedyskutować następujące zagadnienie.
Występujące w~fizyce wielkości zwykle nie są reprezentowane po prostu
liczbami, lecz liczbami posiadającymi jednostki fizycznie, lub jak to
często mówimy wymiar fizyczny, taki jak sekundy, metry, metry na sekundę,
etc. Pozwala to nam napisać
\begin{equation}
  \label{eq:Mechanika-Rozwazania-ogolne-01}
  60 \, \si{s} = 1 \, \si{min},
\end{equation}
gdzie $\si{s}$ oznacza sekundę, a~$\si{min}$ minutę, mimo że ewidentnie
$60$ nie jest równe $1$. Powstaje więc pytanie, jak można matematycznie
sformalizować pojęcie liczby posiadającej jednostki fizyczne? Dla większości
fizyków zajmowanie~się tym byłoby zapewne stratą czasu, gdyż „jest
oczywiste” jak rozumieć wielkość fizyczną posiadającą odpowiedni wymiar,
uważamy jednak, że~patrząc od strony formalizmu matematycznego problem
ten jest interesując, dlatego warto~się nim zająć.

Przedstawimy teraz pewien sposób sformalizowania tego zagadnienia. Zapewne
nie jest on optymalny, powinien jednak wystarczyć na początek. Ponadto
przedstawiona konstrukcja nie bierze pod uwagę dwóch ważnych zagadnień. Po
pierwsze, czy wielkości obecne którymi operuje~się w~fizyce powinny być
reprezentowane zbiorem liczb rzeczywistych $\Rbb$, czy jakimś innym?
Przykładowo, czy fizyce mógłby wystarczyć zbiór liczb wymiernych $\Qbb$?
Drugi ważne pytanie powiązane jest z~pierwszym. Czy jeśli formalizm fizyczny
ma~się opierać na liczbach rzeczywistych, to czy do reprezentowania każdej
wielkości fizycznej potrzebny jest cały zbiór $\Rbb$, czy dla niektórych
wystarczający jest np. zbiór dodatnich liczb rzeczywistych:
$\Rbb_{ + } = \{ x \in \Rbb \, | \, x > 0 \}$? Można bowiem argumentować,
że~wielkości takie jak masa obiektu fizycznego powinny być zawsze dodatnie,
$m > 0$, a~temperatura zawsze większa od zera bezwzględnego, czyli zbiór
liczb rzeczywistych rozciągający~się od minus do plus nieskończoności jest
zbyt duży w~stosunku do naszych realnych potrzeb.

Choć oba te problemy mają swoją wagę, nie będziemy~się wdawać tu w~trudną
dyskusję na ich temat. Dla prostoty przyjmiemy, że~wszystkie wielkości
fizyczne powinny być modelowane na pełnym zbiorze $\Rbb$ i~bazując
na tym założeniu podamy pewną szczególną formalizację liczb rzeczywistych
posiadających jednostki.

W~naszym podejściu przyjmujemy ponadto, że~wszystkie rozważane jednostki
opierają~się na trzech podstawowych wielkościach: mierze długości,
mierze czasu i~mierze masy. Uogólnienie podanych tu konstrukcji na
przypadek systemu z~większą ilością podstawowych wielkości nie powinno
stanowić wielkiego problemu.
Zaczynami więc od wprowadzenia trzech symboli $L$, $T$ i~$M$, które
oznaczają odpowiednio koncepcje długości, czasu i~masy. Podkreślamy,
że~chodzi tu o~koncepcje, nie konkretne wielkości fizyczne. W~szczególności
skoro $L$ oznacza koncepcję długości, to nie przypisujemy temu symbolowi,
ani wielkości liczbowej, ani jednostek. Powinno być dość jasne, że~można
sensownie mówić o~długości stołu, ale nie ma sensu pytać~się o~to czy
sama koncepcja długości ma długość $30 \, \si{cm}$ czy nie\footnote{Dla
  rozwiania pewnych wątpliwości, można oczywiście pytać~się o~długość
  symbolu „$L$”, ale to co innego niż pytanie~się o~długość tego co symbol
  ten oznacza.}.

Rozważmy teraz zbiór\footnote{Od ang. \textit{units}.}
\begin{equation}
  \label{eq:Mechanika-Rozwazania-ogolne-02}
  U_{ 0 } = \{ I, L, T, M \}.
\end{equation}
i~rozważmy grupę przemienną generowaną przez ten zbiór, przy czym $I$ ma
być jej elementem neutralnym. Grupę tą oznaczmy $U$. Tutaj wchodzimy
w~jeden z~bardziej wątpliwych punktów całej konstrukcji, gdyż nie jestem
pewien, czy dobrze korzystamy z~konstrukcji grupy generowanej przez
dowolny, skończony zbiór. W~tej jednak chwili nie widzę powodu, dla
którego ta przeprowadzona w~trochę naiwny sposób konstrukcja miałaby
być błędna.

Grupa $G$ składa~się ze wszystkich możliwych, skończonych kombinacji
elementów $L$, $T$, $M$ i~elementów do nich odwrotnych $L^{ -1 }$, $T^{ -1 }$
i~$M^{ -1 }$ oraz elementu neutralnego $I$. Zawiera więc ona przykładowo
elementy
\begin{equation}
  \label{eq:Mechanika-Rozwazania-ogolne-03}
  \begin{split}
    &L, L^{ 2 }, L^{ 3 }, L^{ 4 }, L^{ -1 }, L^{ -2 }, L^{ -3 },
    T^{ 1 }, T^{ 2 }, T^{ 3 }, T^{ 4 }, T^{ -1 }, T^{ -2 }, T^{ -3 }, T^{ -3 }, \\
    &M^{ 1 }, M^{ 2 }, M^{ 3 }, M^{ -1 }, M^{ -2 }, M^{ -3 },
    L T, L T^{ -1 }, M L, M L^{ -1 }, M L^{ -2 }, \ldots
  \end{split}
\end{equation}
Wprowadzając notację
\begin{equation}
  \label{eq:Mechanika-Rozwazania-ogolne-04}
  u^{ -1 } = \frac{ 1 }{ u } = 1 / u, \quad u \in U,
\end{equation}
oraz analogiczne jak dla ułamków wymiernych prawa operowania ułamkami
utworzonymi z~elementów grupy $U$ możemy zapisać pewne elementy
występujące w~\eqref{eq:Mechanika-Rozwazania-ogolne-03} w~bardziej znajomej
postaci:
\begin{equation}
  \label{eq:Mechanika-Rozwazania-ogolne-05}
  L T^{ -1 } = \frac{ L }{ T }, \quad
  M L^{ -1 } = \frac{ M }{ L }, \quad
  M L^{ -2 } = \frac{ M }{ L^{ 2 } }, \quad \ldots
\end{equation}
Każdemu elementowi $u \in U$ przyporządkowujemy jednowymiarową przestrzeń
wektorową nad $\Rbb$, którą będziemy oznaczać jako $V_{ u }$. Zbiór
wszystkim tych przestrzeni wektorowych będziemy oznaczać przez $S_{ U }$.
Dla przestrzeni $V_{ I }$ określamy wyróżniony wektor bazy $e_{ I }$, który
pozwala nam zdefiniować kanoniczny izomorfizm $V_{ I }$ z~$\Rbb$:
\begin{equation}
  \label{eq:Mechanika-Rozwazania-ogolne-06}
  \omega : V_{ I } \ni v = \alpha \, e_{ I } \mapsto \alpha \in \Rbb.
\end{equation}
Tym samym możemy uznać, iż~$V_{ I }$ reprezentuje przestrzeń wielkości
bezwymiarowych.

Niech teraz będą dane $u_{ 1 }, u_{ 2 } \in U$. Każdej parze taki przestrzeni
przyporządkowujemy pewien rodzaj iloczynu, który będziemy oznaczać
$\psi_{ u_{ 1 }, \, u_{ 2 } }$.
\begin{equation}
  \label{eq:Mechanika-Rozwazania-ogolne-07}
  \psi_{ u_{ 1 }, \, u_{ 2 } } :
  V_{ u_{ 1 } } \times V_{ u_{ 2 } } \to V_{ u_{ 1 } u_{ 2 } },
\end{equation}
gdzie $V_{ u_{ 1 } u _{ 2 } }$ oznacza przestrzeń przyporządkowaną elementowi
$u_{ 1 } u_{ 2 } \in U$. Od $\psi_{ u_{ 1 }, \, u_{ 2 } }$ żądamy był on
odwzorowaniem dwuliniowym. Aby doprecyzować te odwzorowania w~każdej
przestrzeni $V_{ u }$ wybieramy w~dowolny sposób jeden niezerowy wektor,
oznaczmy go $e_{ u }$, i~kładziemy
\begin{equation}
  \label{eq:Mechanika-Rozwazania-ogolne-08}
  \psi_{ u_{ 1 }, \, u_{ 2 } }( e_{ u_{ 1 } }, e_{ u_{ 2 } } ) = e_{ u_{ 1 } u_{ 2 } }.
\end{equation}

Wprowadzamy również operację dzielenia elementu z~$V_{ u_{ 1 } }$ przez
niezerowy element z~$V_{ u_{ 2 } }$:
\begin{equation}
  \label{eq:Mechanika-Rozwazania-ogolne-09}
  \varphi_{ u_{ 1 },\, u_{ 2 } } :
  V_{ u_{ 1 } } \times ( V_{ u_{ 2 } } \setminus \{ 0 \} ) \to V_{ u_{ 1 } / u_{ 2 } }.
\end{equation}
Od $\varphi_{ u_{ 1 }, \, u_{ 2 } }$ żądamy by posiadało następujące własności
\begin{subequations}
  \begin{align}
    \label{eq:Mechanika-Rozwazania-ogolne-10-A}
    \varphi_{ u_{ 1 }, \, u_{ 2 } }( \alpha \, v_{ 1 } + \beta \, v_{ 2 }, w )
    &= \alpha \, \varphi_{ u_{ 1 }, \, u_{ 2 } }( v_{ 1 }, w )
      + \beta \, \varphi_{ u_{ 1 }, \, u_{ 2 } }( v_{ 2 }, w ), \\
    \label{eq:Mechanika-Rozwazania-ogolne-10-B}
    \varphi_{ u_{ 1 }, \, u_{ 2 } }( v, \lambda \, w )
    &= \frac{ 1 }{ \lambda } \, \varphi_{ u_{ 1 }, \, u_{ 2 } }( v, w ), \\
    \label{eq:Mechanika-Rozwazania-ogolne-10-C}
    \varphi_{ u, \, u }( v_{ u }, v_{ u } ) &= I \in V_{ I },
  \end{align}
\end{subequations}
gdzie $u, u_{ 1 }, u_{ 2 } \in U$, $v_{ u } \in V_{ u }$,
$v, v_{ 1 }, v_{ 2 } \in V_{ u_{ 1 } }$, $w \in V_{ u_{ 2 } }$, $w \neq 0$,
$\alpha, \beta, \lambda \in \Rbb$ i~$\lambda \neq 0$. Podobnie jak poprzednio żądamy by zachodziło
\begin{equation}
  \label{eq:Mechanika-Rozwazania-ogolne-11}
  \varphi_{ u_{ 1 }, \, u_{ 2 } }( e_{ u_{ 1 } }, e_{ u_{ 2 } } ) = e_{ u_{ 1 } / u_{ 2 } }.
\end{equation}

Przystąpimy teraz do udowodnienia własności odwzorowań
$\psi_{ u_{ 1 }, u_{ 2 } }$ i~$\varphi_{ u_{ 1 }, u_{ 2 } }$, które pozwalają stwierdzić,
że~zachowują~się one jak mnożenie i~dzielenie liczb rzeczywistych. Na
początek przeanalizujemy analog łączności mnożenia liczb rzeczywistych.
\begin{equation}
  \label{eq:Mechanika-Rozwazania-ogolne-12}
  \psi_{ u_{ 1 } u_{ 2 }, \, u_{ 3 } }\big(
  \psi_{ u_{ 1 }, u_{ 2 } }( v_{ 1 }, v_{ 2 } ), v_{ 3 } \big)
  =
  \psi_{ u_{ 1 }, \, u_{ 2 } u_{ 3 } }\big(
  v_{ 1 }, \psi_{ u_{ 2 }, u_{ 3 } }( v_{ 2 }, v_{ 3 } ) \big),
\end{equation}
gdzie $u_{ 1 }, u_{ 2 }, u_{ 3 } \in U$ $v_{ 1 } \in V_{ u_{ 1 } }$,
$v_{ 2 } \in V_{ u_{ 2 } }$, $v_{ 3 } \in V_{ u_{ 3 } }$. Główny problem z~tą
tożsamością i~jej dowodem polega na tym, że~notacja staje~się dość
zawiła. Jeśli bowiem zapiszemy $v_{ 1 } = \alpha \, e_{ u_{ 1 } }$,
$v_{ 2 } = \beta \, e_{ u_{ 2 } }$, $v_{ 3 } = \gamma \, e_{ u_{ 3 } }$, to dostajemy
\begin{subequations}
  \begin{equation}
    \label{eq:Mechanika-Rozwazania-ogolne-13-A}
    \begin{split}
      \psi_{ u_{ 1 } u_{ 2 }, \, u_{ 3 } }\big(
      \psi_{ u_{ 1 }, \, u_{ 2 } }( v_{ 1 }, v_{ 2 } ), v_{ 3 } \big)
      &=
        \psi_{ u_{ 1 } u_{ 2 }, \, u_{ 3 } }\big(
        \psi_{ u_{ 1 }, \, u_{ 2 } }( \alpha \, e_{ u_{ 1 } }, \beta \, e_{ u_{ 2 } } ),
        \gamma \, e_{ u_{ 3 } } \big) = \\
      &=
        \alpha \beta \gamma \, \psi_{ u_{ 1 } u_{ 2 }, \, u_{ 3 } }\big(
        \psi_{ u_{ 1 }, \, u_{ 2 } }( e_{ u_{ 1 } }, e_{ u_{ 2 } } ),
        e_{ u_{ 3 } } \big),
    \end{split}
  \end{equation}
  \begin{equation}
    \label{eq:Mechanika-Rozwazania-ogolne-13-B}
    \begin{split}
      \psi_{ u_{ 1 }, \, u_{ 2 } u_{ 3 } }\big(
      e_{ u_{ 1 } },
      \psi_{ u_{ 2 }, \, u_{ 3 } }( e_{ u_{ 2 } }, e_{ u_{ 3 } } ) \big)
      &=
        \psi_{ u_{ 1 }, \, u_{ 2 } u_{ 3 } }\big(
        \alpha \, e_{ u_{ 1 } },
        \psi_{ u_{ 2 }, \, u_{ 3 } }( \beta \, e_{ u_{ 2 } }, \gamma \, e_{ u_{ 3 } } )
        \big) = \\
      &=
        \alpha \beta \gamma \, \psi_{ u_{ 1 }, \, u_{ 2 } u_{ 3 } }\big(
        e_{ u_{ 1 } },
        \psi_{ u_{ 2 }, \, u_{ 3 } }( e_{ u_{ 2 } }, e_{ u_{ 3 } } ) \big).
    \end{split}
  \end{equation}
\end{subequations}
Widzimy więc, że~aby zakończyć dowód tej własności wystarczy wykazać
równości
\begin{equation}
  \label{eq:Mechanika-Rozwazania-ogolne-14}
  \psi_{ u_{ 1 } u_{ 2 }, \, u_{ 3 } }\big(
  \psi_{ u_{ 1 }, \, u_{ 2 } }( e_{ u_{ 1 } }, e_{ u_{ 2 } } ), e_{ u_{ 3 } }
  \big)
  =
  \psi_{ u_{ 1 }, \, u_{ 2 } u_{ 3 } }\big(
  e_{ u_{ 1 } }, \psi_{ u_{ 2 }, \, u_{ 3 } }( e_{ u_{ 2 } }, e_{ u_{ 3 } } )
  \big).
\end{equation}
Lewa strona tej zależności wynosi
\begin{equation}
  \label{eq:Mechanika-Rozwazania-ogolne-15}
  \psi_{ u_{ 1 } u_{ 2 }, \, u_{ 3 } }\big(
  \psi_{ u_{ 1 }, \, u_{ 2 } }( e_{ u_{ 1 } }, e_{ u_{ 2 } } ), e_{ u_{ 3 } }
  \big) =
  \psi_{ u_{ 1 } u_{ 2 }, \, u_{ 3 } }\big( e_{ u_{ 1 } u_{ 2 } }, e_{ u_{ 3 } }
  \big) =
  e_{ ( u_{ 1 } u_{ 2 } ) u_{ 3 } },
\end{equation}
natomiast prawa
\begin{equation}
  \label{eq:Arnold-MetodyMatematyczneETC-01}
  \psi_{ u_{ 1 }, \, u_{ 2 } u_{ 3 } }\big(
  e_{ u_{ 1 } }, \psi_{ u_{ 2 }, \, u_{ 3 } }( e_{ u_{ 2 } }, e_{ u_{ 3 } } )
  \big) =
  \psi_{ u_{ 1 }, \, u_{ 2 } u_{ 3 } }( e_{ u_{ 1 } }, e_{ u_{ 2 } u_{ 3 } } ) =
  e_{ u_{ 1 } ( u_{ 2 } u_{ 3 } ) }.
\end{equation}
Tym samym równość \eqref{???} wynika z~łączności mnożenia w~grupie $U$,
a~zatem prawdziwa jest też równość \eqref{????} wyrażająca analog łączności
mnożenia w~zbiorze $S_{ U }$. Analogiczny rachunek pokazuje, że~zachodzi
analog przemienności mnożenia
\begin{equation}
  \label{eq:Arnold-MetodyMatematyczneETC-01}
  \psi_{ u_{ 1 }, \, u_{ 2 } }( v_{ 1 }, v_{ 2 } ) =
  \psi_{ u_{ 2 }, \, u_{ 1 } }( v_{ 2 }, v_{ 1 } ),
\end{equation}
gdzie $u_{ 1 }, u_{ 2 } \in U$, $v_{ 1 } \in V_{ u_{ 1 } }$
i~$v_{ 2 } \in V_{ u_{ 2 } }$. Używają analogicznej metody dowodzimy analogu
prawa łączności dla operacji „dzielenia”
\begin{equation}
  \label{eq:Arnold-MetodyMatematyczneETC-01}
  \varphi_{ u_{ 1 } / u_{ 2 }, \, u_{ 3 } }\big(
  \varphi_{ u_{ 1 }, \, u_{ 2 } }( v_{ 1 }, v_{ 2 } ), v_{ 3 } \big) =
  \varphi_{ u_{ 1 }, \, u_{ 2 } / u_{ 3 } }\big(
  v_{ 1 }, \varphi_{ u_{ 2 }, \, u_{ 3 } }( v_{ 2 }, v_{ 3 } ) \big)
\end{equation}
oraz analog prawa rozdzielności
\begin{equation}
  \label{eq:Arnold-MetodyMatematyczneETC-01}
  \psi_{ u_{ 1 } / u_{ 2 }, \, q_{ 1 } / q_{ 2 } }\big(
  \varphi_{ u_{ 1 }, \, u_{ 2 } }( v_{ 1 }, v_{ 2 } ),
  \varphi_{ q_{ 1 }, \, q_{ 2 } }( w_{ 1 }, w_{ 2 } ) \big) =
  \varphi_{ u_{ 1 } q_{ 1 } / ( u_{ 2 } q_{ 2 } ) }\big(
  \psi_{ u_{ 1 }, \, q_{ 1 } }( v_{ 1 }, w_{ 1 } ),
  \psi_{ u_{ 2 }, \, q_{ 2 } }( v_{ 2 }, w_{ 2 } ) \big),
\end{equation}
gdzie $u_{ 1 }, u_{ 2 }, q_{ 1 }, q_{ 2 } \in U$, $v_{ 1 } \in V_{ u_{ 1 } }$,
$v_{ 2 } \in V_{ u_{ 2 } }$, $w_{ 1 } \in V_{ q_{ 1 } }$ i~$w_{ 2 } \in v_{ q_{ 2 } }$.

Korzystając udowodnionych przed chwilą zależności oraz własności mnożenia
przez liczby w~przestrzeniach wektorowych, możemy udowodnić wiele innych
zależności, analogicznych dla mnożenia i~dzielenia liczb rzeczywistych.
Ponieważ metoda dowodzenia powinna być w~tym momencie jasna, nie
będziemy~się zagłębiać w~szczegółowe dowody.

Ten schemat powinien nam pozwolić opisać wszystkie podstawowe operacje
wykonywane na~wielkościach fizycznych posiadających jednostki, co
zilustrujemy to kilkoma przykładami. Na początku wybieramy odpowiednie bazy
w~przestrzeniach $V_{ L }$, $V_{ T }$ i~$V_{ M }$. Niech tymi bazami będą
$e_{ \si{m} } \in V_{ L }$, $e_{ \si{s} } \in V_{ T }$, $e_{ \si{kg} }$, gdzie
wektor
$e_{ \si{m} }$ będziemy interpretować jako oznaczający metr, $e_{ s }$ jako
sekundę, a~$e_{ \si{kg} }$ jako oznaczający kilogram. Jeżeli teraz mamy dwa
wektory $v_{ 1 }, v_{ 2 } \in V_{ L }$, $v_{ 1 } = 10 \, e_{ \si{m} }$,
$v_{ 2 } = 5 \, e_{ \si{m} }$, to
\begin{equation}
  \label{eq:Mechanika-Rozwazania-ogolne-10}
  v_{ 1 } + v_{ 2 } = 10 \, e_{ \si{m} } + 5 \, e_{ \si{m} } =
  15 \, e_{ \si{m} } \in V_{ L }.
\end{equation}
W~notacji do której jesteśmy bardziej przyzwyczajeni, choć z~nietypowym
użyciem symbolu $v$, powyższa równość przyjęłaby formę
\begin{equation}
  \label{eq:Mechanika-Rozwazania-ogolne-11}
  v_{ 1 } + v_{ 2 } = 10 \, \si{m} + 5 \, \si{m} = 15 \, \si{m}.
\end{equation}
Weźmy teraz $v = 10 \, e_{ \si{m} } \in V_{ L }$
i~$w = 5 \, e_{ \si{s} }\in V_{ T }$. Możemy teraz obliczyć
\begin{equation}
  \label{eq:Mechanika-Rozwazania-ogolne-11}
  \varphi_{ L, \, T }( v, w ) =
  \varphi_{ L, \, T }( 10 \, e_{ \si{m} }, 5 \, e_{ \si{s} } ) =
  10 \varphi_{ L, \, T }( e_{ \si{m} }, 5 \, e_{ \si{s} } ) =
  10 \cdot \frac{ 1 }{ 5 } \varphi_{ L, \, T }( e_{ \si{m} }, e_{ \si{s} } ) =
  2 \varphi_{ L, \, T }( e_{ \si{m} }, e_{ \si{s} } ).
\end{equation}
Wektor $\varphi_{ L, \, T }( e_{ \si{m} }, e_{ \si{s} } )$ będziemy interpretowali
jako metr nad sekundę.










% ######################################
\newpage

\section{Kanoniczne prace o~mechanice Newtona}
% Tytuł danego działu

\vspace{\spaceTwo}
% ######################################



% ############################
\Work{ % Autor i tytuł dzieła
  Isaac Newton \\
  \textit{Matematyczne zasady filozofii przyrody},
  \cite{NewtonMatematyczneZasadyFilozofiiPrzyrody2011}}


% ##################
\CenterBoldFont{Błędy}


\begin{center}

  \begin{tabular}{|c|c|c|c|c|}
    \hline
    Strona & \multicolumn{2}{c|}{Wiersz} & Jest
                              & Powinno być \\ \cline{2-3}
    & Od góry & Od dołu & & \\
    \hline
    16  & &  2 & Dodajęy & Dodaję \\
    % & & & & \\
    % & & & & \\
    % & & & & \\
    % & & & & \\
    \hline
  \end{tabular}

\end{center}

\vspace{\spaceTwo}








% ############################










% ######################################
\newpage

\section{Matematyczne ujęcie mechaniki Newtona}

\vspace{\spaceTwo}
% ######################################



% ############################
\Work{ % Autor i tytuł dzieła
  Władimir Arnold \\
  \textit{Metody matematyczne mechaniki klasycznej},
  \cite{ArnoldMetodyMatematyczneMechanikiKlasycznej1981}}

\vspace{0em}


% ##################
\CenterBoldFont{Uwagi}

\vspace{0em}


\noindent
\textbf{Rozdział 7.} W~tym rozdziale nie znalazłem dowodu, ani
żadnej wskazówki, że~należy samemu pokazać, iż w~lokalnym układzie
współrzędnych zachodzi dobrze znany wzór:
\begin{equation*}
  \label{eq:Arnold-MetodyMatematyczneETC-01}
  d f = \partial_{ i } f\, d x^{ i }.
\end{equation*}
Zastosowanie tego wzoru znacznie ułatwia rozwiązywanie dalszych zadań
w~tym rozdziale, a~niektóre nie wiem nawet jak zrobić bez niego.

\vspace{\spaceFour}





\noindent
\textbf{Str. 71.} W~twierdzeniu Poincar\'{e}go o~powracaniu
założenie o~ciągłości $g$ wydaje się bardzo nienaturalne. Wydaje się,
że~najlepiej jest je zamienić na żądanie mierzalności tej funkcji.

\vspace{\spaceFour}







% ##################
\newpage

\CenterBoldFont{Błędy}


\begin{center}

  \begin{tabular}{|c|c|c|c|c|}
    \hline
    Strona & \multicolumn{2}{c|}{Wiersz} & Jest
                              & Powinno być \\ \cline{2-3}
    & od góry & od dołu & & \\
    \hline
    12  & &  2 & matematycz netak & matematyczne tak \\
    18  &  3 & & $\Phi( \vecxbold, \dot{ \vecxbold } )$
           & % $\mathbf{F}( \mathbf{ x }, \dot{ \mathbf{ x } } )$
    \\
    19  & &  1 & $\vecgbold\, \vecxbold$ & $-\vecgbold\, \vecxbold$ \\
    23  &  4 & & $f( \dot{ x } )$ & $f( x )$ \\
    24  &  1 & & Narysujem y & Narysujemy \\
    34  & &  2 & zorientowane & zorientowanej \\
    37  & & 10 & $\ddot{ \vecrbold } - r \dot{ \varphi }^{ 2 }$
           & $\ddot{ r } - r \dot{ \varphi }^{ 2 }$ \\
    37  & &  6 & $\dot{ r } -$ & $\ddot{ r } -$ \\
    60  & 10 & & napsać & napisać \\
    61  & &  8 & $\sqrt{ ( \dot{ q }_{ 1 }^{ 2 } + \dot{ q }_{ 2 }^{ 2 }
                 + \dot{ q }_{ 3 }^{ 2 } ) }$
           & $\frac{ 1 }{ 2 } m ( \dot{ q }_{ 1 }^{ 2 } + \dot{ q }_{ 2 }^{ 2 }
             + \dot{ q }_{ 3 }^{ 2 } )$ \\
    62  &  2 & & $m\, \dot{ \vecrbold }$ & $m\, \dot{ r }$ \\
    64  & &  8 & $G( x, p )$ & $G( x_{ 0 }, p )$ \\
    77  & &  9 & $S^{ 2 }$ & $S^{ 1 }$ \\
    % & & & & \\
    % & & & & \\
    % & & & & \\
    % & & & & \\
    81  &  5 & & $TM$ & $TM_{ x }$ \\
    81  &  9 & & $\mathbf{ \eta_{ i } }$ & $\eta_{ i }$ \\
    81  & &  5 & związką & wiązką \\
    81  & &  3 & $\vectbold_{ 0 }$ & $t_{ 0 }$ \\
    82  & 15 & & $m_{ 1 }${  }, & $m_{ 1 }$ \\
    86  &  2 & & \textit{Lagrange’a, to $( M, L )$}
           & \textit{Lagrange’a $( M, L )$, to} \\
    98  & &  3 & $\omega^{ 2 }$ & $\omega_{ 0 }^{ 2 }$ \\
    124 & 13 & & $\mathbf{Q}$ & $Q$ \\
    165 & &  9 & postc & postaci \\
    169 & & 10 & Prykład & Przykład \\
    170 & 8 & & k-wymiaro & k-wymiro- \\
    181 & & 15 & obszru & obszaru \\
    186 & & 11 & $T^{ * } V$ & $T^{ * } V_{ x }$ \\
    188 & 1 & & ednoparametrowa & jednoparametrowa \\
    214 & & 6 & rotacja & rotacją \\
    225 & 13 & & $H( \partial L / \partial \dot{ \vecpbold }, \vecqbold )$
           & $H( \partial L / \partial \dot{ \vecqbold }, \vecqbold )$ \\
    % 242 &  3 & & \textit{Jacobiego} & Jacobiego \\
    % 268 & &  8 & $g$ & $\vecgbold$ \\
    % 291 & & & & \\ % Jak się pisze w LaTeXu cyrlicą?
    % 351 &  1 & & $P_{ * }TM_{ X }$ & $P_{ * }TM_{ x }$ \\
    % 351 &  1 & & $T\gFrak_{ p }^{ * })$ & $T\gFrak_{ p }^{ * }$ \\
    % 373 & 18 & & A.~Arez & A.~Avez \\
    % 395 & & 11 & \textit{Poincar\'{e}'s} & \textit{Poincar\'{e}s} \\
    % & & & & \\
    \hline
  \end{tabular}





  \newpage

  \begin{tabular}{|c|c|c|c|c|}
    \hline
    Strona & \multicolumn{2}{c|}{Wiersz} & Jest
                              & Powinno być \\ \cline{2-3}
    & od góry & od dołu & & \\
    \hline
    % 12  & &  2 & matematycz netak & matematyczne tak \\
    % 18  &  3 & & $\Phi( \vecxbold, \dot{ \vecxbold } )$
    %        & % $\mathbf{F}( \mathbf{ x }, \dot{ \mathbf{ x } } )$
    % \\
    % 19  & &  1 & $\vecgbold\, \vecxbold$ & $-\vecgbold\, \vecxbold$ \\
    % 23  &  4 & & $f( \dot{ x } )$ & $f( x )$ \\
    % 24  &  1 & & Narysujem y & Narysujemy \\
    % 34  & &  2 & zorientowane & zorientowanej \\
    % 37  & & 10 & $\ddot{ \vecrbold } - r \dot{ \varphi }^{ 2 }$
    %        & $\ddot{ r } - r \dot{ \varphi }^{ 2 }$ \\
    % 37  & &  6 & $\dot{ r } -$ & $\ddot{ r } -$ \\
    % 60  & 10 & & napsać & napisać \\
    % 61  & &  8 & $\sqrt{ ( \dot{ q }_{ 1 }^{ 2 } + \dot{ q }_{ 2 }^{ 2 }
    %              + \dot{ q }_{ 3 }^{ 2 } ) }$
    %        & $\frac{ 1 }{ 2 } m ( \dot{ q }_{ 1 }^{ 2 } + \dot{ q }_{ 2 }^{ 2 }
    %          + \dot{ q }_{ 3 }^{ 2 } )$ \\
    % 62  &  2 & & $m\, \dot{ \vecrbold }$ & $m\, \dot{ r }$ \\
    % 64  & &  8 & $G( x, p )$ & $G( x_{ 0 }, p )$ \\
    % 77  & &  9 & $S^{ 2 }$ & $S^{ 1 }$ \\
    % % & & & & \\
    % % & & & & \\
    % % & & & & \\
    % % & & & & \\
    % 81  &  5 & & $TM$ & $TM_{ x }$ \\
    % 81  &  9 & & $\mathbf{ \eta_{ i } }$ & $\eta_{ i }$ \\
    % 81  & &  5 & związką & wiązką \\
    % 81  & &  3 & $\vectbold_{ 0 }$ & $t_{ 0 }$ \\
    % 82  & 15 & & $m_{ 1 }${  }, & $m_{ 1 }$ \\
    % 86  &  2 & & \textit{Lagrange’a, to $( M, L )$}
    %        & \textit{Lagrange’a $( M, L )$, to} \\
    % 98  & &  3 & $\omega^{ 2 }$ & $\omega_{ 0 }^{ 2 }$ \\
    % 124 & 13 & & $\mathbf{Q}$ & $Q$ \\
    % 165 & &  9 & postc & postaci \\
    % 169 & & 10 & Prykład & Przykład \\
    % 170 & 8 & & k-wymiaro & k-wymiro- \\
    % 181 & & 15 & obszru & obszaru \\
    % 186 & & 11 & $T^{ * } V$ & $T^{ * } V_{ x }$ \\
    % 188 & 1 & & ednoparametrowa & jednoparametrowa \\
    % 214 & & 6 & rotacja & rotacją \\
    % 225 & 13 & & $H( \partial L / \partial \dot{ \vecpbold }, \vecqbold )$
    %        & $H( \partial L / \partial \dot{ \vecqbold }, \vecqbold )$ \\
    242 &  3 & & \textit{Jacobiego} & Jacobiego \\
    268 & &  8 & $g$ & $\vecgbold$ \\
    % 291 & & & & \\ % Jak się pisze w LaTeXu cyrlicą?
    351 &  1 & & $P_{ * }TM_{ X }$ & $P_{ * }TM_{ x }$ \\
    351 &  1 & & $T\gFrak_{ p }^{ * })$ & $T\gFrak_{ p }^{ * }$ \\
    373 & 18 & & A.~Arez & A.~Avez \\
    395 & & 11 & \textit{Poincar\'{e}'s} & \textit{Poincar\'{e}s} \\
    % & & & & \\
    \hline
  \end{tabular}

\end{center}

\vspace{\spaceTwo}


\noindent
\StrWd{29}{4} \\
\Jest  tworzy sferę dwuwymiarową. \\
\Powin można przekształcić w~sferę dwuwymiarową. \\
\StrWd{42}{10} \\
\Jest  Słońce znajduje~się nie w~centrum \\
\Powin ale~Słońce nie znajduje~się w~centrum \\
\StrWd{71}{15} \\
\Jest  do swego\ldots \\
\Powin dowolnie blisko swego\ldots \\
\StrWg{215}{4} \\
\Jest  wirowej, a~pole jest bezźródłowe. \\
\Powin wirowej. \\



% ############################










% ############################
\newpage

\Work{ % Autorzy i tytuł dzieła
  Roman Stanisław Ingarden, Andrzej Jamiołkowski \\
  \textit{Mechanika klasyczna},
  \cite{IngardenJamiolkowskiMechanikaKlasyczna1980}}

\vspace{0em}


% ##################
\CenterBoldFont{Uwagi do konkretnych stron}

\vspace{0em}


\Str{9--12}

\vspace{\spaceFour}



\noindent
Str. 19. Bardzo ciężko jest zrozumieć uwagę, że w dwóch układach pochodne po czasie są różne, pomimo iż czas płynie tak samo. Proponuję następujące wyjaśnienie tego problemu:

Zauważmy, że w dwóch różnych układach odniesienia $x$ oraz
$\tilde{ x }$ będą różnymi funkcjami czasu (na razie zostawiamy na
boku głębszą dyskusję ontologicznej natury wykonywanych tu operacji).
Wytłumaczmy to na przykładzie: niech $\tilde{ x }$ będzie niezerowym
wektorem i niech układ $\tilde{ \Ocal }$ wykonuje obrót wokół
$\tilde{ p }_{ 0 }$. Teraz w układzie $\Ocal$ $\tilde{ x }$
jest wektorem o stałych współrzędnych, podczas gdy w układzie
$\tilde{ \Ocal }$ dokonuje on obrotu. Podobnie wektory bazy
układu $\tilde{ \Ocal }$ są postrzegane jako nieruchome w tym
układzie, lecz jako obracające się w
układzie $\Ocal$.

(Dyskusja ta wymaga udoskonalenia). Zauważmy, że każda pochodna ma
człon wynikający z różniczkowania współrzędnych i wektorów bazy.
Jeżeli więc mamy dany jakąś funkcje wektorową jako funkcję czasu, to
od wyboru układu odniesienia zależy nie tylko postać funkcyjna
współrzędnych, ale też czy mamy różniczkować dane wektory. W pewnym
sensie (bo do tej pory wszystko to jest niedoprecyzowane) pochodne
konkretnych funkcji skalarnych są takie same w każdym układzie
odniesienia, bo nie wchodzi do nich pochodna wektorów bazy.



Str. 20.
$\frac{ d\vecebold_{ 1 } }{ dt } = \vecomegabold \times \vecebold_{ 1 } \, ,$

Str. 20.
$\frac{ \tilde{ \dPL } \tilde{ \vecxbold } }{ \dPL t }
= \frac{ d\tilde{ x }^{ i } }{ dt } \vecebold_{ i } \, ,$

Str. 20. \ldots także z faktu, że
$\dPL \tilde{ x }^{ i } / \dPL t = \tilde{ \dPL } \tilde{ x
}^{ i } / \dPL t$\ldots

Str. 21.
$\vecvbold = \tilde{ \vecvbold } + \vecvbold_{ 0 } +
\vecomegabold \times \tilde{ \vecxbold } \, ,$

Str. 21. ????
$\frac{ d\vecvbold }{ dt } = \frac{ d\tilde{ \vecvbold } }{ dt }
+ \vecomegabold \times \tilde{ \vecvbold } + \frac{
  d\vecvbold_{ 0 } }{ dt } + \frac{ \vecomegabold }{ t }
\times \tilde{ \vecxbold } + \vecomegabold  \times \bigg(
\frac{ \tilde{ \dPL } \tilde{ \vecxbold } }{ \dPL t } +
\vecomegabold \times \tilde{ \vecxbold } \bigg) \, .$
Sprawdzić.

Str. 24. Obraz odwzorowanie
$X : T \rightarrow E^{ 3N }$\ldots

Str. 28. \ldots chwili $t \in T$ funkcje\ldots

Str. 36. \ldots oraz że nie zależy on od wyboru układu
współrzędnych\ldots


\vspace{\spaceTwo}
% ############################










% ############################
\newpage

\Work{ % Autorzy i tytuł dzieła
  J. I. Nejmark, N. A. Fufajew \\
  \textit{Dynamika układów nieholonomicznych},
  \cite{NejmarkFufajewDynamikaUkladowNieholonomicznych1971}}

% \vspace{0em}


% ##################
\CenterBoldFont{Błędy}


\begin{center}

  \begin{tabular}{|c|c|c|c|c|}
    \hline
    & \multicolumn{2}{c|}{} & & \\
    Strona & \multicolumn{2}{c|}{Wiersz}
                            & Jest & Powinno być \\ \cline{2-3}
    & Od góry & Od dołu & & \\
    \hline
    9   &  6 & & i wielu & wielu \\
    9   &  7 & & nczonych & uczonych \\
    11  & &  1 & $\delta$ & $\theta$ \\
    12  &  2 & & prędkość & przyśpieszenie \\
    12  &  3 & & równa & równe \\
    % & & & & \\
    % & & & & \\
    % & & & & \\
    % & & & & \\
    \hline
  \end{tabular}

\end{center}

\vspace{\spaceTwo}

% ############################










% ######################################
\newpage

\section{Książki powstałe po~1945~r.}

\vspace{\spaceTwo}
% ######################################



% ############################
\Work{ % Autorzy i tytuł dzieła
  Lew D. Landau, Jewginij M. Lifszyc \\
  \textit{Mechanika}, \cite{LandauLifszycMechanika2006}}

\vspace{0em}


% ##################
\CenterBoldFont{Uwagi do konkretnych stron}

\vspace{0em}


\noindent
\Str{14} Podana tu grupa Galileusza składa~się tylko z~pchnięć,
co według mnie tylko zaciemnia strukturę symetrii czasoprzestrzeni
Galileusza. Pełniejsze omówienie tej grupy można znaleźć w~książce
W.~Arnolda \textit{Metody matematyczne mechaniki klasycznej}
\cite{ArnoldMetodyMatematyczneMechanikiKlasycznej1981}.

\vspace{\spaceFour}





\noindent
\Str{13} Przemyślenie jest głębokie, ale przedstawione
stanowczo zbyt krótko, aby było jasne. Spróbuję przedstawić tu pewne
jego rozwinięcie.

Przed wszystkim należy zauważyć, że należy tu rozróżnić jednorodność
i~izotropowość w sensie geometrii przestrzeni i w sensie dynamiki.
Cechy te traktowane jako cechy geometrii czasoprzestrzeni w sensie
geometrii liniowej i różniczkowej, są niezależne od układu
odniesienia. Przejdźmy teraz do problemu dynamiki. Po pierwsze z
doświadczenia wiemy, że możemy przyjąć, iż przestrzeń jest
euklidesowa, jak również że można znaleźć układ odniesienia w którym
cząstki swobodne umieszczone w przestrzeni spoczywają.

\vspace{\spaceFour}





\noindent
\Str{14} $\frac{ \partial L }{ \partial \vecvbold }$ nie jest
funkcją tylko kwadratu prędkości. Jest to wektor o składowych
$( \frac{ \partial L }{ \partial \vecvbold } )_{ i } = 2 \frac{
  \partial L }{ \partial { v^{ 2 } } } v_{ i }$, czyli zależy on
jawnie od składowych prędkości. Widać jednak, że stałość
$\frac{ \partial L }{ \partial \vecvbold }$ wymaga od nas stałości
$\mathbf{ v }$. Jeżeli bowiem rozpatrzymy składową $x$ wektora
(ściślej pola wektorowego)
$\frac{ \partial L }{ \partial \vecvbold }$, mamy warunek na stałość
tego wyrażenia dla dowolnej wartości składowej $x$:
$\frac{ \partial L }{ \partial { v^{ 2 } } } = \frac{ 1 }{ 2
  \vecvbold_{ x } }$. Wyrażenie to należy zakwestionować na paru
poziomach, choćby dlatego, że jest osobliwe dla zerowych prędkości, co
jest niedopuszczalne dla fizycznej teorii. Oczywiście, jeżeli
sprawdzimy również warunek na $y$ składową otrzymamy sprzeczny układ
równań.

\vspace{\spaceFour}





\noindent
\Str{14} Należałoby podać większą dyskusję prędkości względnej dwóch
układów inercjalnych.

\vspace{\spaceFour}





\noindent
\Str{15} Jak można ściślej uzasadnić, że rzeczywiście
potrzebujemy liniowej zależności od prędkości prawej strony równania
wyrażającego równoważność między dwoma lagrażjanami? \Dok

\vspace{\spaceFour}




\noindent
\Str{22} Dyskusja ważności addytywnych zasad zachowania, ma
swoją głębię i wagę, zaciemnia ona jednak pewne szczegóły. Autorzy gdy
ją pisali musieli mieć na myśli procesy rozpraszania, nie wspomnieli
jednak, że jeśli znana jest postać oddziaływania między dwoma
cząstkami, również mamy możliwość wyciągnięcia z praw zachowania
ważnych wniosków. Np. jeśli rozpatrujemy układ dwóch cząstek i znamy
energię kinetyczną jednej z nich i energię oddziaływania, to możemy
obliczyć pewne parametry ruchu drugiej.

\vspace{\spaceFour}





\noindent
\Str{14} Należałoby podać większą dyskusję prędkości względnej
dwóch układów inercjalnych.

\vspace{\spaceFour}





\noindent
 \Str{15} Jak można ściślej uzasadnić, że rzeczywiście
potrzebujemy liniowej zależności od prędkości prawej strony równania
wyrażającego równoważność między dwoma lagrażjanami? \Dok

\vspace{\spaceFour}




\noindent
\Str{22} Dyskusja ważności addytywnych zasad zachowania, ma
swoją głębię i wagę, zaciemnia ona jednak pewne szczegóły. Autorzy gdy
ją pisali musieli mieć na myśli procesy rozpraszania, nie wspomnieli
jednak, że jeśli znana jest postać oddziaływania między dwoma
cząstkami, również mamy możliwość wyciągnięcia z praw zachowania
ważnych wniosków. Np. jeśli rozpatrujemy układ dwóch cząstek i znamy
energię kinetyczną jednej z nich i energię oddziaływania, to możemy
obliczyć pewne parametry ruchu drugiej.





% ##################
\newpage

\CenterBoldFont{Błędy}


\begin{center}

  \begin{tabular}{|c|c|c|c|c|}
    \hline
    & \multicolumn{2}{c|}{} & & \\
    Strona & \multicolumn{2}{c|}{Wiersz} & Jest
                              & Powinno być \\ \cline{2-3}
    & Od góry & Od dołu & & \\
    \hline
    56  & 12 & & poruszały się z tą samą prędkością & spoczywały \\
    % & & & & \\
    % & & & & \\
    \hline
  \end{tabular}

\end{center}

\vspace{\spaceTwo}


\noindent
\StrWd{27}{2} \\
\Jest  $S = S' + \mu \vecVbold \cdot \vecRbold' + \frac{ 1 }{ 2 } \mu V^{ 2 } t$ \\
\Powin $S = S' + \mu \vecVbold \cdot \vecRbold'( t ) - \mu \vecVbold
\cdot \vecRbold'( 0 ) + \frac{ 1 }{ 2 } \mu V^{ 2 } t$ \\



% ############################










% ############################
\newpage

\Work{ % Autor i tytuł dzieła
  Bogdan Skalmierski \\
  \textit{Mechanika}, \cite{SkalmierskiMechanika1998}}

\vspace{0em}


% ##################
\CenterBoldFont{Uwagi}

\vspace{0em}


We wszystkich rozważaniach przestrzeni wektorowych, będziemy~się ograniczać
do dwóch najważniejszych dla fizyki, w~szczególności też dla mechaniki,
typów przestrzeni wektorowych. Mianowicie będziemy rozważać tylko
przestrzenie wektorowe nad ciałem liczby rzeczywistych lub zespolonych.
W~istocie w~mechanice Newtona prawie zawsze wystarczające będzie rozważanie
przestrzeni nad ciałem liczb rzeczywistych.

Dodatkowo przyjmujemy, że~wszystkie rozważane przestrzenie mają skończony
wymiar, chyba że~jest powiedziane inaczej. Wymiar rozważanej przestrzeni
zawsze będziemy oznaczać symbolem~$N$.

\vspace{\spaceFour}





% ##################
\CenterBoldFont{Uwagi do konkretnych stron}


\noindent
\Str{16} W~tym miejscu należałoby dodać następującą uwagę na temat oznaczeń
dla wektorów zaczepionych. Jeżeli wektory zaczepione w~dwóch różnych
punktach $A$ i~$B$ są równoważne, to będziemy je oznaczać tym samym symbolem
np. $\vecabold$. Inaczej mówiąc, jeśli w~punktach $A$ i~$B$ jest zaczepiony
ten sam wektor swobodny, to te dwa wektory swobodne oznaczamy tym samym
symbolem.

W~literaturze zwykle~się nie pisze jawnie, czy dany symbol $\vecabold$
oznacza wektor swobodny, czy wektor zaczepiony w~danym punkcie. Co gorsza
często ten sam symbol oznacza zarówno wektor swobodny $\vecabold$, jak
i~wektor swobodny $\vecabold$ po zaczepieniu w~punkcie $A$. Może to
prowadzić do niejasności i~utrudniać początkującym naukę.

Bardziej przejrzysta byłaby notacja $\vecabold_{ A }$, jednak szansa by to
oznaczenie~się przyjęło jest zaniedbywalnie mała. W~dalszej części
komentarzy będziemy próbowali precyzować co dany wektor dokładnie oznacza.

\vspace{\spaceFour}



\noindent
\Str{16} By uczynić definicję relacji współosiowości wektorów bardziej
ścisłą, przyjmiemy, że~wektor $\vecZeroBold$ jest współosiowy z~dowolnym
innym wektorem.

\vspace{\spaceFour}





\noindent
\Str{17} Warto zatrzymać się na chwilę nad pojęciem \textbf{składowej
  wektora} i~\textbf{współrzędnej wektora}. Niech $V$ będzie rozważaną
przestrzenią wektorową, a~$\vecvbold$ zawartym w~niej wektorem.
\textbf{Współrzędnymi wektora $\vecvbold$ w bazie $\vecebold_{ i }$}
będziemy nazywać ciąg liczb $\alpha_{ i }$ takich, że
\begin{equation}
  \label{eq:SkalmierskiMechanika-01}
  \vecvbold = \sum_{ i = 1 }^{ N } \alpha_{ i } \vecebold_{ i }.
\end{equation}
Gdy nie będzie możliwości nieporozumień, współrzędne wektora $\vecvbold$
będziemy oznaczać symbolami $v_{ i }$, $v_{ i } \equiv \alpha_{ i }$. Analogicznie
współrzędne wektora $\vecabold$ będziemy oznaczać symbolami $a_{ i }$,
wektora $\vecbbold$, $b_{ i }$, etc.

\textbf{Składowymi wektora $\vecvbold$} będziemy nazywać dowolny zbiór
liniowo niezależnych wektorów \\
$\{ \vecwbold_{ 1 }, \vecwbold_{ 2 }, \ldots, \vecwbold_{ k } \}$, $k \geq 1$, taki że
\begin{equation}
  \label{eq:SkalmierskiMechanika-02}
  \vecvbold = \sum_{ i = 1 }^{ k } \vecwbold_{ i }.
\end{equation}

Kilka uwag odnośnie tej definicji. Moglibyśmy przyjąć, że wektory
$\vecwbold_{ i }$ tworzą nie zbiór tylko ciąg, jednak ponieważ dodawanie
(skończonej) liczby wektorów jest przemienne oraz że~w~praktyce rzadko
kiedy podaje~się konkretną numerację zbioru składowych, przyjęliśmy taką
jej . Moglibyśmy też opuścić warunek liniowej niezależności tych
wektorów i~definicja była wciąż użyteczna, jednak to założenie wydaje
nam~się bardzo naturalne w~tym kontekście i~nie powinno sprawiać żadnych
problemu, gdy to pojęcie jest stosowane w literaturze fizycznej.

Wykluczyliśmy możliwość, że $k = 0$, czyli, że zbiór składowych jest pusty.
Ponieważ w~algebrze liniowej przyjmuje się, że suma po pustym zbiorze
wektorów daje nam wektor zerowy, w~skutek tego definicja ta wyklucza rozkład
wektora zerowego na zerową liczbę składowych. Natomiast założenie o~liniowej
niezależność zbioru wyklucza dowolny inny rozkład wektora zerowego na
składowe. Żadna z~tych konsekwencji podanej definicji, nie powinna stanowić
problemu.

Na koniec zauważmy, że~dopuszczamy sytuację gdy $k < N$, co jest przypadkiem
często spotykanym w~praktyce.

\vspace{\spaceFour}





\noindent
\Str{19} Mała uwaga na temat notacji. W~każdej przestrzeni wektorowej $V$
jest określone działanie mnożenia wektora przez liczbę z~zadanego
ciała $\Fbb$: $\textrm{product} : \Fbb \times V \to V$. W sytuacji gdy $\alpha \in \Fbb$,
$\vecabold \in V$, będziemy uznawać, że~następujące wyrażania są sobie
równoważne.
\begin{equation}
  \label{eq:SkalmierskiMechanika-03}
  \textrm{product}( \alpha, \vecabold ) \equiv \alpha \, \vecabold \equiv \vecabold \, \alpha
\end{equation}
Analogiczne, dla $\alpha \neq 0$ uważamy dwa poniższe wyrażenia za równoważne.
\begin{equation}
  \label{eq:SkalmierskiMechanika-04}
  \frac{ 1 }{ \alpha } \vecabold \equiv \frac{ \vecabold }{ \alpha }
\end{equation}

W~przypadku teorii fizycznych w~większości przypadków interesować nas będą
tylko przypadki $\Fbb = \Rbb$ lub $\Fbb = \Cbb$. W~niniejszej książce
wystarczające powinny przestrzenie nad ciałem liczb rzeczywistych.

\vspace{\spaceFour}





\noindent
\Str{19} W tym miejscu powinno być przytoczone „prawo zachowania
wskaźników”, do jego sformułowania potrzebujemy wprowadzić odrobinę
terminologi.

Każdy wskaźnik po którym sumujemy, np. $i$ w wyrażeniu
\begin{equation}
  \label{eq:SkalmierskiMechanika-05}
  \sum_{ i = 1 }^{ N } a^{ i },
\end{equation}
nazywamy \textbf{wskaźnikiem niemym} lub \textbf{martwym}. Taki wskaźnik
jest tylko nazwą zmiennej sumowania, można zmienić go dowolny inny i~treść
matematyczna pozostanie bez zmian. Każdy wskaźnik który nie jest niemym
nazywamy \textbf{wskaźnikiem żywym}.

„Prawo zachowania wskaźników”: każdy wskaźnik żywy występujący po lewej
stronie równości, musi też wystąpić po jej prawej stronie.

\vspace{\spaceFour}





\noindent
\StrWd{19}{10} Jak stwierdziła Iwona Grabska-Gradzińska, lepszym
oznaczeniem od $a_{ xi }$ byłoby $a_{ i,\, x }$. Taka notacja bardzie uwypukla
to, że mamy do czynienia z~iksową składową wektora o~numerze $i$.

\vspace{\spaceFour}





\noindent
\Str{20, 22} Definicja konta między wektorami, choć opierająca~się na tym,
że~„łatwo widać” co to jest kont między wektorami, gdy się je narysuje,
zostawia pewną lukę dla przypadku, gdy jeden wektor jest równy
$\vecZeroBold$.

Aby ją zapełnić przyjmujemy, że~jeśli co najmniej jeden z~wektorów
$\vecabold$ i~$\vecbbold$ jest wektorem zerowym to kąt między nimi
wynosi~$0$ radianów: $\alpha = 0$.

\vspace{\spaceFour}





\noindent
\Str{20} Może nie jest to zaznaczone wyraźnie, ale aby zdefiniować iloczyn
skalarny dwóch wektorów musimy oba przemieścić w~taki sposób, by były
zaczepione w~jednym punkcie. Albo dokonać operacji, która pozwoliłaby je
zaczepić w~wspólnym punkcie. Zauważmy też, że~te same uwagi odnoszą się do
iloczynu wektorowego i~podobnych operacji.

W~definicji „za pomocą obrazka” musimy je zaczepić we wspólnym punkcie, by
zmierzyć kąt jaki jest między nimi. Przykładem innego sposobu liczenia tego
kąta jest traktowanie wektorów jako odcinków skierowanych i~przeniesienie
wektora $\vecbbold$ do punktu końcowego wektora $\vecabold$, obliczenia kąta
$\beta$ między tak utworzonymi odcinkami i~przyjęcie, że kąt między wektorami
jest równy $\alpha = \pi - \beta$ (w~radianach). Widać jednak, że skoro możemy
przesunąć wektor $\vecbbold$ do punktu końcowego wektora $\vecabold$, to
równie dobrze moglibyśmy przenieść go do punktu zaczepienia tego
wektora\footnote{To zagadnienie można podać bardziej ścisłej analizie,
  jednak uważamy, że~obecnej formie jest ono wystarczająco precyzyjnie
  sformułowane w~stosunku do naszych potrzeb.}.

W~przypadku definicji „za pomocą wersorów/współrzędnych”, musimy założyć, że
możemy porównać wersory zaczepione w~dwóch różnych miejscach i~obliczyć
iloczyn skalarny między nimi. Możemy też uznać, że wersor reprezentuje nam
klasę wektorów równoważnych i~w każdym punkcie przestrzeni jest zaczepiony
jeden z nich. Ale tym samym uznaliśmy pojęcie równości „na odległość”,
dzięki czemu możemy sformułować pojęcie przeniesienia wektora

Jakbyśmy więc nie podeszli do zagadnienia, to albo musimy mieć pojęcie
przesuwania wektorów, które ich nie zmienia, albo pojęcie równości wektorów
„na odległość”. W~obecnym kontekście może wydawać~się to drobnostką, jednak
przy dokładnej analizie pewnych wzorów pominięcie tego aspektu może
prowadzić do niejasności i~zbyt pobieżnego rozumienia danego zagadnienia.
Problem ten staje~się czymś bardzo poważnym, gdy przechodzimy do geometrii
różniczkowej.

W~dalszym ciągu będziemy więc przyjmować, że~podczas obliczania wielkości
takich jaki iloczyn skalarny oba wektory zostały przeniesione do pewnego
wspólnego punktu i~dopiero wtedy obliczamy wynik. Dlatego wykonując tą
operację na wektorach $\vecabold$, $\vecbbold$, nie będziemy dodawać im
indeksu dolnego, jak w $\vecabold_{ A }$, by zaznaczyć w~którym miejscu są
one zaczepione.

Powstaje pytanie, skoro rezultatem np. iloczynu wektorowego, jest wektor to
w~którym miejscu jest on zaczepiony? W~takiej sytuacji będziemy pryzmować,
iż~taki wektor jest wektorem swobodnym, chyba że~jest powiedziane inaczej.

\vspace{\spaceFour}





\noindent
\Str{21} Z~formy podanej tu procedury opuszczania wskaźnika, wynika,
że~jeśli $a^{ i }$ są współrzędnymi odpowiedniego wektora, to $a_{ i }$ są
współrzędnymi nie wektora, lecz formy (1-formy). Dopiero korzystając
z~naturalnego izomorfizmu (Czy to jest na pewno naturalny izomorfizm???)
między wektorami i~formami możemy utożsamić te dwa obiekty, co będziemy
w~dalszym ciągu czynić.

\vspace{\spaceFour}





\noindent
\Str{24} Użycie konwencji sumacyjnej we wzorze (1.26) chyba bardzie
zaciemnia, niż rozjaśnia jego treść. Byłby on bardziej zrozumiały, gdyby
został zapisany jako
\begin{equation}
  \label{eq:SkalmierskiMechanika-06}
  \sum_{ j = 1 }^{ 3 } \varepsilon_{ j i k } \, \varepsilon_{ j m n }
  = \delta_{ i m } \, \delta_{ k n } - \delta_{ i n } \, \delta_{ k m }.
\end{equation}

Sam autor ma chyba z~tym problem bo nazywa wyrażenie
$\varepsilon_{ j i k } \, \varepsilon_{ j m n }$ iloczynem, podczas gdy zgodnie z~przyjętą
konwencją należy je uważać za zapis sumy po $j$. Wedle mojej intuicji
językowej wyrażenia arytmetyczne\footnote{Jeśli chodzi o~wyrażenia bardziej
  złożone, np. zawierające funkcję $\sin$, sposób ich nazywania wymaga
  większej dozy refleksji.} nazywamy po wyrażeniu które stoi najniżej
w~hierarchii działań występujących w~danym wyrażeniu. Mówiąc prościej,
to które wykonujemy na końcu. Z tego więc względu sumą nazywamy wyrażenia
$a + b$, $a + bc$, iloczynem $ab$, $a ( b + c )$, etc. Stąd wzór
\eqref{eq:SkalmierskiMechanika-06} nazwalibyśmy raczej sumą, niż iloczynem.

W~skutek tego zamieszania terminologicznego, przedstawiony tu dowód
tożsamości \eqref{eq:SkalmierskiMechanika-06} nie jest całkowicie jasny.
Można go jednak przeprowadzić w następujący sposób\footnote{Dla zupełności
  podamy tu pełny dowód tego twierdzenia.}. Aby uniknąć nieporozumień,
przyjmiemy, że $j_{ 1 }$ oznacza jedną z liczb $1, 2, 3$. Wyrażenie
$\varepsilon_{ j_{ 1 } i k }$ przyjmuje wartość różną od zera wtedy i tylko wtedy, gdy
wszystkie trzy liczby $j_{ 1 }$, $i$, $k$, są między sobą różne. Inaczej
mówiąc, jeśli ciąg $j_{ 1 }, i, k$ stanowi permutacje ciągu $1, 2, 3$.
Jeżeli więc $i = k$ to lewa strona tożsamości
\eqref{eq:SkalmierskiMechanika-06} jest równa 0.
Prawa zaś stronę możemy przepisać jako
\begin{equation}
  \label{eq:SkalmierskiMechanika-07}
  \delta_{ i m } \, \delta_{ i n } - \delta_{ i n } \, \delta_{ i m }.
\end{equation}
Jeżeli $i \neq m$ lub $i \neq n$ to wyrażenie powyżej jest oczywiście równe 0.
Pozostał nam do rozpatrzenia ostatni przypadek: $i = m$ i~$i = n$, bądź
krócej $i = m = n$. W~tym przypadku też oczywiście otrzymujemy 0. Ta sama
analiza stosuje~się do $\varepsilon_{ j m n }$.

Jak powiedziano wyżej, dla ustalonej liczby $j_{ 1 }$ iloczyn
$\varepsilon_{ j_{ 1 } i k }$ z~$\varepsilon_{ j_{ 1 } m n }$ będzie niezerowy wtedy i tylko wtedy,
gdy oba ciągi $j_{ 1 }, i, k$ oraz $j_{ 1 }, m, n$ są różnowartościowe
(inaczej: są permutacjami ciągu $1, 2, 3$). Zachodzi więc
$j_{ 1 } \neq i \neq k \neq j_{ 1 }$ i~analogicznie dla $j_{ 1 } \neq m \neq n \neq j_{ 1 }$.
W~tym przypadku z~sumy \eqref{eq:SkalmierskiMechanika-06} przeżywa tylko
jeden wyraz. Jeśli bowiem $i = 1$, $k = 2$, to niezerowy będzie tylko wyraz
dla którego $j = 3$, analogicznie dla pozostałych przypadków. W takim razie
możliwe są dwie sytuacje: a) $i = m$, $k = n$; b) $i = n$, $k = m$.

W przepadku a) $\varepsilon_{ j_{ 1 } i k } = \varepsilon_{ j_{ 1 } m n }$, więc ich iloczyn jest
równy $1^{ 2 } = 1$ lub $( -1 )^{ 2 } = 1$. Podstawienie zależności między
$j_{ 1 }, i, k, m, n$ do prawej stron wzoru
\eqref{eq:SkalmierskiMechanika-06} pokazuje, że~również jej wartość wynosi
$1$.

Przypadek b). Możliwe są dwie sytuacje:
$\varepsilon_{ j_{ 1 } i k } = 1$, $\varepsilon_{ j_{ 1 } m n } = -1$
lub~$\varepsilon_{ j_{ 1 } i k } = -1$, $\varepsilon_{ j_{ 1 } m n } = 1$. Wobec tego lewa strona wzoru
\eqref{eq:SkalmierskiMechanika-06} wynosi $-1$. Ponownie korzystając
z~relacji między $j_{ 1 }, i, k, m, n$ dostajemy, że~tyle samo wynosi prawa
strona tego wzoru, co kończy dowód.

\vspace{\spaceFour}





\noindent
\Str{24} W~tym miejscu Skalmierski stwierdził, że~prawdziwe jest twierdzenie
które można by wysłowić w~następujący sposób. „Momenty dwóch wektorów
obliczone względem ustalonego punktu są równe wtedy i~tylko wtedy, gdy te
wektory są równoważne i~współosiowe”. Takie twierdzenie jednak nie zachodzi.

Rozważmy najpierw wektory leżące na prostych przechodzących przez punkt
$O$. Momenty wszystkich tych wektorów względem punktu $O$ są równe
$\vecZeroBold$, widzimy więc, że~istnieje nieskończenie wiele wektorów,
które nie muszą być ani równoważne, ani współosiowe, a~momenty ich obu są
równe $\vecZeroBold$. Widzimy też, że o~równoważności i~współosiowości
wektorów których moment wynosi $\vecZeroBold$ nie możemy nic powiedzieć.

Zauważmy też, że~poza przypadkiem wymienionym wyżej, jest tylko jedna inna
możliwość, że~moment wektory jest równy $\vecZeroBold$. Mianowicie gdy
liczymy moment wektora zerowego zaczepionego w~dowolnym punkcie
przestrzeni.

Rozważmy teraz wektory\footnote{Ze względu na wygodę, nie będę w~notacji
  rozróżniał między wektorem, $\alpha$~jego współrzędnymi.}:
$\vecabold = [ 0, 1, 0 ]$ z~wektorem wodzącym jego punktu zaczepienia
$\vecrbold_{ 1 } = [ 1, 0, 0 ]$ i~$\vecbbold = [ -1, 0,  0 ]$ z~wektorem
wodzącym punktu zaczepienia $\vecrbold_{ 2 } = [ 0, 1, 0 ]$. Jak łatwo
sprawdzić zachodzi
\begin{equation}
  \label{eq:SkalmierskiMechanika-07}
  \vecMbold = \vecrbold_{ 1 } \times \vecabold =
  \vecrbold_{ 2 } \times \vecbbold = [ 0, 0, 1 ].
\end{equation}
Wektory $\vecabold$ i~$\vecbbold$ nie są ani równoważne, ani~współliniowe.

Przykład przedstawiony w~poprzednim paragrafie można uogólnić w~następujący
sposób. Mianowicie zauważając, że~iloczyn wektorowy jest niezmiennicze ze
względu na obroty: jeśli dwa wektory $\vecabold$ i~$\vecbbold$ zaczepione
w~punkcie~$A$ obrócimy o~ten sam kąt wokół punktu $A$, to ich iloczyn
wektorowy nie ulegnie zmianie. W~naszym przypadku sytuacja jest trochę
bardziej skomplikowana.

Niech $O$ będzie punktem względem którego liczymy moment wektora, $A_{ 0 }$
punktem w~którym jest zaczepiony wektor $\vecabold_{ 0 }$. Jeśli teraz
obrócimy wektor wodzący punktu $A$, który będziemy oznaczać
$\vecrbold_{ 0 }$, o~kąt $\varphi$ wokół punktu $O$ to otrzymamy nowy wektor
$\vecrbold_{ 1 }$ pokazujący inny punkt. Punkt wskazywany przez wektor
$\vecrbold_{ 1 }$ oznaczmy przez $A_{ 1 }$. Jeśli teraz obrócimy wektor
$\vecabold_{ 0 }$ o~kąt $\varphi$ wokół punktu $A_{ 0 }$, to otrzymany nowy wektor
który oznaczymy przez $\vecabold_{ 1 }$. Teraz wektor $\vecabold_{ 1 }$
zaczepiamy w~punkcie $A_{ 1 }$.

Po tak przeprowadzonej operacji moment wektora $\vecabold_{ 1 }$
zaczepionego w~punkcie $A_{ 1 }$ jest taki sam, jak wektora $\vecabold_{ 0 }$
zaczepionego w~punkcie $A_{ 0 }$. Wynika to z tego, że~tak przeprowadzona
transformacja zachowuje długość wektorów oraz kąt między wektorem
a~wektorem wodzącym jego punktu zaczepienia. Łatwo zauważyć, że~równość
\eqref{eq:SkalmierskiMechanika-07} zachodzi, bo wektor $\vecbbold$ powstaje
przez obrót wektora $\vecabold$ o~kąt $\pi / 2$ zgodnie z~procedurą opisaną
powyżej.

Rozważmy jeszcze jeden przypadek. Niech będzie dany wektor $\vecabold$
o~wektorze wodzącym $\vecrbold_{ 1 }$ i~niech dana będzie liczba rzeczywista
$\alpha \neq 0$. Rozważmy wektor $\vecbbold = \alpha \vecabold$, zaczepiony w~punkcie
danym przez wektor $\vecrbold_{ 2 } = ( 1 / \alpha ) \vecrbold_{ 1 }$. Łatwo
zauważyć, że
\begin{equation}
  \label{eq:SkalmierskiMechanika-08}
  \vecrbold_{ 2 } \times \vecbbold =
  \left( \frac{ 1 }{ \alpha } \vecrbold_{ 1 } \right) \times ( \alpha \vecabold ) =
  \vecrbold_{ 1 } \times \vecabold.
\end{equation}
Otrzymaliśmy całą klasę wektorów, które nie są równoważne, acz są
współosiowe, których momenty względem punktu $O$ są takie same. Zwróćmy
uwagę, że jak w poprzednim przykładzie z~obrotami, tutaj również zmieniamy
nie tylko wektor, ale też jego punkt zaczepienia.

Prawdopodobnie można podać wyczerpującą klasyfikację wektorów posiadających
ten sam moment względem punktu $O$, jedna powyższe przykłady sugerują,
iż~gra nie jest warta świeczki, dlatego poprzestaniemy na udowodnieniu
poniższego twierdzenia.





% #############
\begin{theorem}

  Jeżeli wektory $\vecabold$ i~$\vecbbold$ są równoważne i~współosiowe, to
  ich momenty są równe.

\end{theorem}



\begin{proof}

  Skoro wektory
  $\vecabold$ i~$\vecbbold$ są równoważne i~współosiowe więc różnią się tyko
  punktem zaczepienia na pewnej prostej. Zgodnie z tym co powiedziany
  poprzednio będziemy je więc oznaczać tym samym symbolem $\vecabold$. Tym
  samym mamy
  \begin{equation}
    \label{eq:SkalmierskiMechanika-07}
    \vecMbold_{ 1 } = \vecrbold_{ 1 } \times \vecabold, \quad
    \vecMbold_{ 2 } = \vecrbold_{ 2 } \times \vecabold.
  \end{equation}
  Skoro te wektory różnią się tylko punktem zaczepienia na pewnej prostej,
  zaś na punkt te wskazują wektory $\vecrbold_{ 1 }$ i~$\vecrbold_{ 2 }$ to
  istnieje taki wektor $\vecRbold$ współosiowy z~$\vecabold$, taki że
  $\vecrbold_{ 2 } = \vecrbold_{ 1 } + \vecRbold$. Tym samym mamy
  \begin{equation}
    \label{eq:SkalmierskiMechanika-08}
    \vecMbold_{ 2 } =
    \vecrbold_{ 2 } \times \vecabold =
    ( \vecrbold_{ 1 } + \vecRbold ) \times \vecabold =
    \vecrbold_{ 1 } \times \vecabold + \vecRbold \times \vecabold =
    \vecrbold_{ 1 } \times \vecabold + \vecZeroBold = \vecMbold_{ 1 }.
  \end{equation}

\end{proof}
% #############









% ##################
\newpage

\CenterBoldFont{Błędy}


\begin{center}

  \begin{tabular}{|c|c|c|c|c|}
    \hline
    Strona & \multicolumn{2}{c|}{Wiersz} & Jest
                              & Powinno być \\ \cline{2-3}
    & Od góry & Od dołu & & \\
    \hline
    22 & 13 & & wektor{ }{ }{ }$\vecdbold$ & wektor $\vecdbold$ \\
    22 & &  9 & $a_{ z } b_{ y } \, \hphantom{k} \times \vecjbold$
           & $a_{ z } b_{ y } \, \veckbold \times \vecjbold$ \\
    24 & 17 & &  $( \vecabold \cdot \veccbold ) \cdot \vecbbold
                - ( \vecbbold \cdot \veccbold ) \cdot \vecabold$
           & $( \vecabold \cdot \veccbold ) \vecbbold
             - ( \vecbbold \cdot \veccbold ) \vecabold$ \\
    % & & & & \\
    % & & & & \\
    % & & & & \\
    % & & & & \\
    % & & & & \\
    \hline
  \end{tabular}

\end{center}

\vspace{\spaceTwo}



% ############################










% ############################
\newpage

\Work{ % Autor i tytuł dzieła
  Bogdan Skalmierski \\
  \textit{Mechanika z~wytrzymałością materiałów},
  \cite{SkalmierskiMechanikaZWytrzymalosciaMaterialow1983}}

\vspace{0em}


% ##################
\CenterBoldFont{Uwagi do konkretnych stron}

\vspace{0em}


\noindent
\Str{21} We~wzorze w~drugiej linii zamiast
\begin{equation}
  \label{eq:Skalmierski-MechanikaZWytrzymalosciaETC-01}
  \sqrt{ 1 - \left( \tfrac{ x }{ x_{ 0 } } \right)^{ 2 } }
\end{equation}
powinno być
\begin{equation}
  \label{eq:SkalmierskiMechanikaZWytrzymaloscia-02}
  \sgn( \cos \varphi ) \,
  \sqrt{ 1 - \left( \tfrac{ x }{ x_{ 0 } } \right)^{ 2 } },
\end{equation}
bo~wykorzystujemy jedynkę trygonometryczną by~wyrazić $\cos$ przez
$\sin$. Ponieważ w~dalszym ciągu obliczeń podnosimy ten człon do
kwadratu, ta niedokładność nie~wpływa na ostateczny wynik.

\vspace{\spaceFour}





\noindent
\Str{36} Aby wyprowadzenie wzoru (3.21) było poprawne,
potrzebujemy by
$| \dot{ \vecebold }_{ 1 } \cdot \vecebold_{ 2 } | = \dot{ \vecebold }_{ 1 }
\cdot \vecebold_{ 2 }$. Oznacza to, że~układ obraca się od~wektora
$\vecebold_{ 1 }$ do~$\vecebold_{ 2 }$.





% ##################
\newpage

\CenterBoldFont{Błędy}


\begin{center}

  \begin{tabular}{|c|c|c|c|c|}
    \hline
    Strona & \multicolumn{2}{c|}{Wiersz} & Jest
                              & Powinno być \\ \cline{2-3}
    & Od góry & Od dołu & & \\
    \hline
    12  & 16 & & można określić & będziemy oznaczali \\
    13  & & 10 & $\vecbbold_{ y }$ & $\vecabold_{ y }$ \\
    17  &  3 & & $( \vecabold \cdot \veccbold ) \cdot \vecbbold
                 - ( \veccbold \cdot \vecbbold ) \cdot \vecabold$
           & $( \vecabold \cdot \veccbold ) \vecbbold
             - ( \veccbold \cdot \vecbbold ) \vecabold$ \\
    21  &  4 & & $\frac{ y }{ { }_{ 0 } }$ & $\frac{ y }{ { y }_{ 0 } }$ \\
    21  &  4 & & $\sqrt{ 1 \:\: \left( \frac{ x }{ x_{ 0 } } \right)^{ 2 } }$
           & $\sqrt{ 1 - \left( \frac{ x }{ x_{ 0 } } \right)^{ 2 } } $ \\
    23  & &  9 & $+2\beta \cos( 2\omega t )$ & $-2\beta \cos( 2\omega t )$ \\
    31  & 13 & & $x_{ 2 } \frac{ \partial x_{ 2 } }{ \partial r }$
           & $\dot{ x }_{ 2 } \frac{ \partial x_{ 2 } }{ \partial r }$ \\
    31  & &  5 & $\vecabold \frac{ \partial \vecrbold }{ \partial q_{ j } }$
           & $\vecabold \cdot \frac{ \partial \vecrbold }{ \partial q_{ j } }$ \\
    32  &  8 & & $\frac{ \partial \vecrbold^{ 2 } }{ \partial { q^{ j } } }$
           & $\frac{ \partial \vecrbold }{ \partial { q^{ j } } }$ \\
    34  &  6 & & $\dot{ \vecebold }_{ i } \vecebold_{ j }$
           & $\dot{ \vecebold }_{ i } \cdot \vecebold_{ j }$ \\
    34  &  8 & & $\vecebold_{ i } \vecebold_{ j }$
           & $\vecebold_{ i } \cdot \vecebold_{ j }$ \\
    34  & 10 & & $\dot{ \vecebold }_{ i } \vecebold_{ j }
                 + \vecebold_{ i } \dot{ \vecebold }_{ j }$
           & $\dot{ \vecebold }_{ i } \cdot \vecebold_{ j }
             + \vecebold_{ i } \cdot \dot{ \vecebold }_{ j }$ \\
    34  & 11 & & $\dot{ \vecebold }_{ i } \vecebold_{ j }$
           & $\dot{ \vecebold }_{ i } \cdot \vecebold_{ j }$ \\
    34  & 12 & & $\dot{ \vecebold }_{ i } \vecebold_{ j }$
           & $\dot{ \vecebold }_{ i } \cdot \vecebold_{ j }$ \\
    36  & 16 & & $( \xi_{ 1 } \dot{ \vecebold }_{ 1 }
                 + \xi_{ 2 } \dot{ \vecebold }_{ 2 }  ) \vecebold_{ 1 }$
           & $( \xi_{ 1 } \dot{ \vecebold }_{ 1 }
             + \xi_{ 2 } \dot{ \vecebold }_{ 2 }  ) \cdot \vecebold_{ 1 }$ \\
    36  & 16 & & $( \xi_{ 1 } \dot{ \vecebold }_{ 1 }
                 + \xi_{ 2 } \dot{ \vecebold }_{ 2 }  ) \vecebold_{ 2 }$
           & $( \xi_{ 1 } \dot{ \vecebold }_{ 1 }
             + \xi_{ 2 } \dot{ \vecebold }_{ 2 }  ) \cdot \vecebold_{ 2 }$ \\
    36  & 18 & & $\dot{ \vecebold }_{ 1 } \vecebold_{ 2 }$
           & $\dot{ \vecebold }_{ 1 } \cdot \vecebold_{ 2 }$ \\
    36  & 20 & & $\dot{ \vecebold }_{ 1 } \vecebold_{ 2 }$
           & $\dot{ \vecebold }_{ 1 } \cdot \vecebold_{ 2 }$ \\
           %        % & & & & \\
    \hline
  \end{tabular}

\end{center}

\vspace{\spaceTwo}


\noindent
\StrWg{31}{13} \\[0.3em]
\Jest
$( \dot{ x }_{ 1 } \vecibold + \dot{ x }_{ 2 } \vecjbold )
\left( \frac{ \partial x_{ 1 } }{ \partial r } \vecibold
  + \frac{ \partial x_{ 2 } }{ \partial r } \vecjbold \right)
\cdot \frac{ 1 }{ \vecOneBold }$ \\[0.5em]
\Powin
$( \dot{ x }_{ 1 } \vecibold + \dot{ x }_{ 2 } \vecjbold )
\cdot \left( \frac{ \partial x_{ 1 } }{ \partial r } \vecibold
  + \frac{ \partial x_{ 2 } }{ \partial r } \vecjbold \right)
\frac{ 1 }{ | \vecOneBold | }$ \\



% ############################










% ############################
\newpage

\Work{ % Autor i tytuł dzieła
  Bogdan Skalmierski \\
  \textit{Mechanika. Tom I: Podstawy mechaniki klasycznej},
  \cite{SkalmierskiMechanikPodstawyETCVolI1998}}


% ##################
\CenterBoldFont{Uwagi do konkretnych stron}


\Str{10} Warunek c) w~definicji przestrzeni topologicznej Hausdorffa jest
źle sformułowany, bowiem $\mathrm{R}_{ a }$ jest bez żadnych założeń
otoczeniem punktu B zawartym w~$\mathrm{R}_{ a }$. To co autor chciał tu
podać jest to definicja przestrzeni topologicznej Hausdorffa bazująca na
pojęciu bazy otoczeń (chyba), należy więc zajrzeć do książki do
topologi~by sprawdzić jak należy to poprawić.

\vspace{\spaceFour}



Definicja homeomorfizmu jest trochę nie jasna, można bowiem odczytać ją
tak, że~choć funkcja $f$ musi być ciągła, to żaden zaś warunek nie jest
nałożony na $f^{ -1 }$.

\vspace{\spaceFour}





\StrWg{11}{8} Brak wcięcia akapitu.

\vspace{\spaceFour}





\Str{18}{10} Brak wcięcia akapitu.

\vspace{\spaceFour}






% ##################
\newpage

\CenterBoldFont{Błędy}


\begin{center}

  \begin{tabular}{|c|c|c|c|c|}
    \hline
    & \multicolumn{2}{c|}{} & & \\
    Strona & \multicolumn{2}{c|}{Wiersz} & Jest
                              & Powinno być \\ \cline{2-3}
    & Od góry & Od dołu &  &  \\ \hline
    % & & & & \\
    17 & & 13 & $( \vecabold \times \vecbbold ) \times \vecebold_{ j }$
           & $( \vecabold \times \vecbbold ) \cdot \vecebold_{ j }$ \\
    17 & & 13 & $( \vecebold_{ i } \times \vecebold_{ k } ) \times \vecebold_{ j }$
           & $( \vecebold_{ i } \times \vecebold_{ k } ) \cdot \vecebold_{ j }$ \\
    % & & & & \\
    % & & & & \\
    \hline
  \end{tabular}

\end{center}

\vspace{\spaceTwo}


\noindent
\StrWd{42}{10} \\
\Jest  Słońce znajduje się nie w centrum \\
\Powin ale Słońce nie znajduje się w centrum \\


% ############################










% #####################################################################
% #####################################################################
% Bibliografia

\bibliographystyle{plalpha}

\bibliography{PhilNaturBooks}{}





% ############################

% Koniec dokumentu
\end{document}
