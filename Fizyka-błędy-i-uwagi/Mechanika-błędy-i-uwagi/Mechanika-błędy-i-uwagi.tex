% Autor: Kamil Ziemian

% --------------------------------------------------------------------
% Podstawowe ustawienia i pakiety
% --------------------------------------------------------------------
\RequirePackage[l2tabu, orthodox]{nag}  % Wykrywa przestarzałe i niewłaściwe
% sposoby używania LaTeXa. Więcej jest w l2tabu English version.
\documentclass[a4paper,11pt]{article}
% {rozmiar papieru, rozmiar fontu}[klasa dokumentu]
\usepackage[MeX]{polski}  % Polonizacja LaTeXa, bez niej będzie pracował
% w języku angielskim.
\usepackage[utf8]{inputenc}  % Włączenie kodowania UTF-8, co daje dostęp
% do polskich znaków.
\usepackage{lmodern}  % Wprowadza fonty Latin Modern.
\usepackage[T1]{fontenc}  % Potrzebne do używania fontów Latin Modern.



% ----------------------------
% Podstawowe pakiety (niezwiązane z ustawieniami języka)
% ----------------------------
\usepackage{microtype}  % Twierdzi, że poprawi rozmiar odstępów w tekście.
% \usepackage{graphicx}  % Wprowadza bardzo potrzebne komendy do wstawiania
% grafiki.
% \usepackage{verbatim}  % Poprawia otoczenie VERBATIME.
% \usepackage{textcomp}  % Dodaje takie symbole jak stopnie Celsiusa,
% wprowadzane bezpośrednio w tekście.
\usepackage{vmargin}  % Pozwala na prostą kontrolę rozmiaru marginesów,
% za pomocą komend poniżej. Rozmiar odstępów jest mierzony w calach.
% ----------------------------
% MARGINS
% ----------------------------
\setmarginsrb
{ 0.7in} % left margin
{ 0.6in} % top margin
{ 0.7in} % right margin
{ 0.8in} % bottom margin
{  20pt} % head height
{0.25in} % head sep
{   9pt} % foot height
{ 0.3in} % foot sep



% ----------------------------
% Często przydatne pakiety
% ----------------------------
% \usepackage{csquotes}  % Pozwala w prosty sposób wstawiać cytaty do tekstu.
\usepackage{xcolor}  % Pozwala używać kolorowych czcionek (zapewne dużo
% więcej, ale ja nie potrafię nic o tym powiedzieć).



% ----------------------------
% Pakiety do tekstów z nauk przyrodniczych
% ----------------------------
\let\lll\undefined  % Amsmath gryzie się z językiem pakietami do języka
% polskiego, bo oba definiują komendę \lll. Aby rozwiązać ten problem
% oddefiniowuję tę komendę, ale może tym samym pozbywam się dużego Ł.
\usepackage[intlimits]{amsmath}  % Podstawowe wsparcie od American
% Mathematical Society (w skrócie AMS)
\usepackage{amsfonts, amssymb, amscd, amsthm}  % Dalsze wsparcie od AMS
% \usepackage{siunitx}  % Do prostszego pisania jednostek fizycznych
\usepackage{upgreek}  % Ładniejsze greckie litery
% Przykładowa składnia: pi = \uppi
% \usepackage{slashed}  % Pozwala w prosty sposób pisać slash Feynmana.
\usepackage{calrsfs}  % Zmienia czcionkę kaligraficzną w \mathcal
% na ładniejszą. Może w innych miejscach robi to samo, ale o tym nic
% nie wiem.



% ##########
% Tworzenie otoczeń "Twierdzenie", "Definicja", "Lemat", etc.
\newtheorem{twr}{Twierdzenie}  % Komenda wprowadzająca otoczenie
% ,,twr'' do pisania twierdzeń matematycznych
\newtheorem{defin}{Definicja}  % Analogicznie jak powyżej
\newtheorem{wni}{Wniosek}



% ----------------------------
% Pakiety napisane przez użytkownika.
% Mają być w tym samym katalogu to ten plik .tex
% ----------------------------
\usepackage{mechanika}  % Pakiet napisany konkretnie dla tego pliku.
\usepackage{latexshortcuts}
\usepackage{mathshortcuts}



% --------------------------------------------------------------------
% Dodatkowe ustawienia dla języka polskiego
% --------------------------------------------------------------------
\renewcommand{\thesection}{\arabic{section}.}
% Kropki po numerach rozdziału (polski zwyczaj topograficzny)
\renewcommand{\thesubsection}{\thesection\arabic{subsection}}
% Brak kropki po numerach podrozdziału



% ----------------------------
% Ustawienia różnych parametrów tekstu
% ----------------------------
\renewcommand{\arraystretch}{1.2}  % Ustawienie szerokości odstępów między
% wierszami w tabelach.



% ----------------------------
% Pakiet "hyperref"
% Polecano by umieszczać go na końcu preambuły.
% ----------------------------
\usepackage{hyperref}  % Pozwala tworzyć hiperlinki i zamienia odwołania
% do bibliografii na hiperlinki.





% --------------------------------------------------------------------
% Tytuł, autor, data
\title{Mechanika klasyczna --~błędy i~uwagi}

% \author{}
% \date{}
% --------------------------------------------------------------------





% ####################################################################
\begin{document}
% ####################################################################



% ######################################
\maketitle % Tytuł całego tekstu
% ######################################



% ######################################
\section{Pozycje klasyczne i~sprzed II~Wojny Światowej}
% Tytuł danego działu

% \vspace{\spaceTwo}
\vspace{\spaceThree}
% ######################################


% ##################
\Work{ % Autor i tytuł dzieła
  Isaac Newton \\
  ,,Matematyczne zasady filozofii przyrody'',
  \cite{NewtonMatematyczneZasadyFilozofiiPrzyrody2011} }

\CenterTB{Błędy}
\begin{center}
  \begin{tabular}{|c|c|c|c|c|}
    \hline
    & \multicolumn{2}{c|}{} & & \\
    Strona & \multicolumn{2}{c|}{Wiersz} & Jest
                              & Powinno być \\ \cline{2-3}
    & Od góry & Od dołu & & \\
    \hline
    16  & &  2 & Dodajęy & Dodaję \\
    % & & & & \\
    % & & & & \\
    % & & & & \\
    % & & & & \\
    \hline
  \end{tabular}
\end{center}

\vspace{\spaceTwo}







% ######################################
\newpage
\section{Matematyczne ujęcie mechaniki klasycznej}

\vspace{\spaceTwo}
% \vspace{\spaceThree}

% ######################################



% ##################
\Work{ % Autor i tytuł dzieła
  Władimir Arnold \\
  ,,Metody matematyczne mechaniki klasycznej'',
  \cite{ArnoldMetodyMatematyczneMechanikiKlasycznej1981} }


\CenterTB{Uwagi}

\start \textbf{Rozdział 7.} W~tym rozdziale nie znalazłem dowodu, ani
żadnej wskazówki, że~należy samemu pokazać, iż w~lokalnym układzie
współrzędnych zachodzi dobrze znany wzór:
\begin{equation*}
  d f = \partial_{ i } f\, d x^{ i }.
\end{equation*}
Zastosowanie tego wzoru znacznie ułatwia rozwiązywanie dalszych zadań
w~tym rozdziale, a~niektóre nie wiem nawet jak zrobić bez niego.

\vspace{\spaceFour}


\start \textbf{Str. 71.} W~twierdzeniu Poincar\'{e}go o~powracaniu
założenie o~ciągłości $g$ wydaje się bardzo nienaturalne. Wydaje się,
że~najlepiej jest je zamienić na żądanie mierzalności tej funkcji.

\vspace{\spaceFour}




\CenterTB{Błędy}
\begin{center}
  \begin{tabular}{|c|c|c|c|c|}
    \hline
    & \multicolumn{2}{c|}{} & & \\
    Strona & \multicolumn{2}{c|}{Wiersz} & Jest
                              & Powinno być \\ \cline{2-3}
    & od góry & od dołu & & \\
    \hline
    12  & &  2 & matematycz netak & matematyczne tak \\
    18  &  3 & & $\Phi( \xbf, \xbfd )$
           & % $\mathbf{F}( \mathbf{ x }, \dot{ \mathbf{ x } } )$
    \\
    19  & &  1 & $\gbf\, \xbf$ & $- \gbf\, \xbf$ \\
    23  &  4 & & $f( \xd )$ & $f( x )$ \\
    24  &  1 & & Narysujem y & Narysujemy \\
    34  & &  2 & zorientowane & zorientowanej \\
    37  & & 10 & $\rbfdd - r \vpd^{ 2 }$ & $\rdd - r \vpd^{ 2 }$ \\
    37  & &  6 & $\rd -$ & $\rdd -$ \\
    60  & 10 & & napsać & napisać \\
    61  & &  8 & $\sqrt{ ( \qd_{ 1 }^{ 2 } + \qd_{ 2 }^{ 2 }
                 + \qd_{ 3 }^{ 2 } ) }$
           & $\frac{ 1 }{ 2 } m ( \qd_{ 1 }^{ 2 } + \qd_{ 2 }^{ 2 }
             + \qd_{ 3 }^{ 2 } )$ \\
    62  &  2 & & $m\, \dot{ \mathbf{r} }$ & $m\, \dot{ r }$ \\
    64  & &  8 & $G( x, p )$ & $G( x_{ 0 }, p )$ \\
    77  & &  9 & $S^{ 2 }$ & $S^{ 1 }$ \\
    % & & & & \\
    % & & & & \\
    % & & & & \\
    % & & & & \\
    81  &  5 & & $TM$ & $\TMx$ \\
    81  &  9 & & $\bsym{ \eta_{ i } }$ & $\eta_{ i }$ \\
    81  & &  5 & związką & wiązką \\
    81  & &  3 & $\bsym{t}_{ 0 }$ & $t_{ 0 }$ \\
    82  & 15 & & $m_{ 1 }${  }, & $m_{ 1 }$ \\
    86  &  2 & & \emph{Lagrange'a, to $( M, L )$}
           & \emph{Lagrange'a $( M, L )$, to} \\
    98  & &  3 & $\omega^{ 2 }$ & $\omega_{ 0 }^{ 2 }$ \\
    124 & 13 & & $\bsym{Q}$ & $Q$ \\
    165 & &  9 & postc & postaci \\
    169 & & 10 & Prykład & Przykład \\
    186 & & 11 & $T^{ * }V$ & $T^{ * }V_{ x }$ \\
    225 & 13 & & $H( \pr L / \pr \pbd, \qb )$
           & $H( \pr L / \pr \qbd, \qb )$ \\
    242 &  3 & & \emph{Jacobiego} & Jacobiego \\
    268 & &  8 & $g$ & $\bsym{g}$ \\
    % 291 & & & & \\ % Jak się pisze w LaTeXu cyrlicą?
    351 &  1 & & $P_{ * }TM_{ X }$ & $P_{ * }TM_{ x }$ \\
    351 &  1 & & $T\mf{g}_{ p }^{ * })$ & $T\mf{g}_{ p }^{ * }$ \\
    373 & 18 & & A.~Arez & A.~Avez \\
    395 & & 11 & \emph{Poincar\'{e}'s} & \emph{Poincar\'{e}s} \\
    % & & & & \\
    \hline
  \end{tabular}
\end{center}
\noi
\StrWd{29}{4} \\
\Jest tworzy sferę dwuwymiarową. \\
\Powin można przekształcić w~sferę dwuwymiarową. \\
\StrWd{42}{10} \\
\Jest Słońce znajduje~się nie w~centrum \\
\Powin ale~Słońce nie znajduje~się w~centrum \\
\StrWd{71}{15} \\
\Jest  do swego\ldots \\
\Powin dowolnie blisko swego\ldots \\

\vspace{\spaceTwo}







% ##################
\Work{ % Autorzy i tytuł dzieła
  Roman Stanisław Ingarden, Andrzej Jamiołkowski \\
  ,,Mechanika klasyczna'',
  \cite{IngardenJamiolkowskiMechanikaKlasyczna1980} }


\CenterTB{Uwagi}

\start \Str{9--12}

\start Str. 19. Bardzo ciężko jest zrozumieć uwagę, że w dwóch układach pochodne po czasie są różne, pomimo iż czas płynie tak samo. Proponuję następujące wyjaśnienie tego problemu: \\
Zauważmy, że w dwóch różnych układach odniesienia $x$ oraz
$\tilde{ x }$ będą różnymi funkcjami czasu (na razie zostawiamy na
boku głębszą dyskusję ontologicznej natury wykonywanych tu operacji).
Wytłumaczmy to na przykładzie: niech $\tilde{ x }$ będzie niezerowym
wektorem i niech układ $\tilde{ \mathcal{ O } }$ wykonuje obrót wokół
$\tilde{ p }_{ 0 }$. Teraz w układzie $\mathcal{ O }$ $\tilde{ x }$
jest wektorem o stałych współrzędnych, podczas gdy w układzie
$\tilde{ \mathcal{ O } }$ dokonuje on obrotu. Podobnie wektory bazy
układu $\tilde{ \mathcal{ O } }$ są postrzegane jako nieruchome w tym
układzie, lecz jako obracające się w
układzie $\mathcal{ O }$. \\
(Dyskusja ta wymaga udoskonalenia). Zauważmy, że każda pochodna ma
człon wynikający z różniczkowania współrzędnych i wektorów bazy.
Jeżeli więc mamy dany jakąś funkcje wektorową jako funkcję czasu, to
od wyboru układu odniesienia zależy nie tylko postać funkcyjna
współrzędnych, ale też czy mamy różniczkować dane wektory. W pewnym
sensie (bo do tej pory wszystko to jest niedoprecyzowane) pochodne
konkretnych funkcji skalarnych są takie same w każdym układzie
odniesienia, bo nie wchodzi do nich pochodna wektorów bazy.



\begin{itemize}
\item[--] Str. 20.
  $$\de{}{ \mathbf{ e }_{ 1 } }{ t } = \boldsymbol{ \omega } \times
  \bold{ e }_{ 1 } \, ,$$
\item[--] Str. 20.
  $$\frac{ \tilde{ \difPL } \tilde{ \mathbf{ x } } }{ \difPL t } = \de{}{
    \tilde{ x }^{ i }}{ t } \mathbf{ e }_{ i } \, ,$$
\item[--] Str. 20. \ldots także z faktu, że
  $\difPL \tilde{ x }^{ i } / \difPL t = \tilde{ \difPL } \tilde{ x
  }^{ i } / \difPL t$\ldots
\item[--] Str. 21.
  $$\mathbf{ v } = \tilde{ \mathbf{ v } } + \mathbf{ v }_{ 0 } +
  \boldsymbol{ \omega } \times \tilde{ \mathbf{ x } } \, ,$$
\item[--] Str. 21.
  $$\de{}{ \mathbf{ v } }{ t } = \de{}{ \tilde{ \mathbf{ v } } }{ t }
  + \boldsymbol{ \omega } \times \tilde{ \mathbf{ v } } + \de{}{
    \mathbf{ v }_{ 0 } }{ t } + \de{}{ \boldsymbol{ \omega } }{ t }
  \times \tilde{ \bold{ x } } + \boldsymbol{ \omega } \times \bigg(
  \frac{ \tilde{ \difPL } \tilde{ \bold{ x } } }{ \difPL t } +
  \boldsymbol{ \omega } \times \tilde{ \bold{ x } } \bigg) \, .$$
  Sprawdzić.
\item[--] Str. 24. Obraz odwzorowanie
  $X : T \rightarrow E^{ 3N }$\ldots
\item[--] Str. 28. \ldots chwili $t \in T$ funkcje\ldots
\item[--] Str. 36. \ldots oraz że nie zależy on od wyboru układu
  współrzędnych\ldots
\end{itemize}

\vspace{\spaceTwo}







% ##################
\Work{ % Autorzy i tytuł dzieła
  J. I. Nejmark, N. A. Fufajew \\
  ,,Dynamika układów nieholonomicznych'',
  \cite{NejmarkFufajewDynamikaUkladowNieholonomicznych1971} }



\CenterTB{Błędy}
\begin{center}
  \begin{tabular}{|c|c|c|c|c|}
    \hline
    & \multicolumn{2}{c|}{} & & \\
    Strona & \multicolumn{2}{c|}{Wiersz}
                            & Jest & Powinno być \\ \cline{2-3}
    & Od góry & Od dołu & & \\
    \hline
    9   &  6 & & i wielu & wielu \\
    9   &  7 & & nczonych & uczonych \\
    11  & &  1 & $\delta$ & $\theta$ \\
    12  &  2 & & prędkość & przyśpieszenie \\
    12  &  3 & & równa & równe \\
    % & & & & \\
    % & & & & \\
    % & & & & \\
    % & & & & \\
    \hline
  \end{tabular}
\end{center}

\vspace{\spaceTwo}





% ######################################
\newpage
\section{Książki powstałe po~1945~r.}

\vspace{\spaceTwo} % \vspace{\spaceThree}
% ######################################



% ##################
\Work{ % Autorzy i tytuł dzieła
  Lew D. Landau, Jewginij M. Lifszyc \\
  ,,Mechanika'', \cite{LandauLifszycMechanika2006} }


\CenterTB{Uwagi}

\start \Str{14} Podana tu grupa Galileusza składa~się tylko z~pchnięć,
co według mnie tylko zaciemnia strukturę symetrii czasoprzestrzeni
Galileusza. Pełniejsze omówienie tej grupy można znaleźć w~książce
W.~Arnolda ,,Metody matematyczne mechaniki klasycznej''
\cite{ArnoldMetodyMatematyczneMechanikiKlasycznej1981}.

\vspace{\spaceFour}


\start \Str{13} Przemyślenie jest głębokie, ale przedstawione
stanowczo zbyt krótko, aby było jasne. Spróbuję przedstawić tu pewne
jego rozwinięcie.

Przed wszystkim należy zauważyć, że należy tu rozróżnić jednorodność
i~izotropowość w sensie geometrii przestrzeni i w sensie dynamiki.
Cechy te traktowane jako cechy geometrii czasoprzestrzeni w sensie
geometrii liniowej i różniczkowej, są niezależne od układu
odniesienia. Przejdźmy teraz do problemu dynamiki. Po pierwsze z
doświadczenia wiemy, że możemy przyjąć, iż przestrzeń jest
euklidesowa, jak również że można znaleźć układ odniesienia w którym
cząstki swobodne umieszczone w przestrzeni spoczywają. \\
\start \Str{14} $\pd{}{ L }{ \mathbf{ v } }$ nie jest funkcją tylko
kwadratu prędkości. Jest to wektor o składowych
$( \pd{}{ L }{ \mathbf{ v }} )_{ i } = 2 \pd{}{ L }{ { v^{ 2 } } } v_{
  i }$, czyli zależy on jawnie od składowych prędkości. Widać jednak,
że stałość $\pd{}{ L }{ \mathbf{ v } }$ wymaga od nas stałości
$\mathbf{ v }$. Jeżeli bowiem rozpatrzymy składową $x$ wektora
(ściślej pola wektorowego) $\pd{}{ L }{ \mathbf{ v } }$, mamy warunek
na stałość tego wyrażenia dla dowolnej wartości składowej $x$:
$\pd{}{ L }{ { v^{ 2 } } } = \frac{ 1 }{ 2 \mathbf{ v }_{ x } }$.
Wyrażenie to należy zakwestionować na paru poziomach, choćby dlatego,
że jest osobliwe dla zerowych prędkości, co jest niedopuszczalne dla
fizycznej teorii. Oczywiście, jeżeli sprawdzimy również warunek na
$ y $ składową otrzymamy sprzeczny układ równań.

\vspace{\spaceFour}


\start \Str{14} Należałoby podać większą dyskusję prędkości względnej
dwóch układów inercjalnych.

\vspace{\spaceFour}


\start \Str{15} Jak można ściślej uzasadnić, że rzeczywiście
potrzebujemy liniowej zależności od prędkości prawej strony równania
wyrażającego równoważność między dwoma lagrażjanami? \Dok

\vspace{\spaceFour}


\start \Str{22} Dyskusja ważności addytywnych zasad zachowania, ma
swoją głębię i wagę, zaciemnia ona jednak pewne szczegóły. Autorzy gdy
ją pisali musieli mieć na myśli procesy rozpraszania, nie wspomnieli
jednak, że jeśli znana jest postać oddziaływania między dwoma
cząstkami, również mamy możliwość wyciągnięcia z praw zachowania
ważnych wniosków. Np. jeśli rozpatrujemy układ dwóch cząstek i znamy
energię kinetyczną jednej z nich i energię oddziaływania, to możemy
obliczyć pewne parametry ruchu drugiej.


\CenterTB{Błędy}
\begin{center}
  \begin{tabular}{|c|c|c|c|c|}
    \hline
    & \multicolumn{2}{c|}{} & & \\
    Strona & \multicolumn{2}{c|}{Wiersz} & Jest
                              & Powinno być \\ \cline{2-3}
    & Od góry & Od dołu & & \\
    \hline
    56  & 12 & & poruszały się z tą samą prędkością & spoczywały \\
    % & & & & \\
    % & & & & \\
    \hline
  \end{tabular}
\end{center}
\noi
\StrWd{27}{2} \\
\Jest $S = S' + \mu \mf{V} \cdot \mf{R}' + \fr{ 1 }{ 2 } \mu V^{ 2 } t$ \\
\Powin $S = S' + \mu \mf{V} \cdot \mf{R}'( t ) - \mu \mf{V} \cdot
\mf{R}'( 0 ) + \fr{ 1 }{ 2 } \mu V^{ 2 } t$ \\

\vspace{\spaceTwo}





% ##################
\Work{ % Autor i tytuł dzieła
  Bogdan Skalmierski \\
  ,,Mechanika z~wytrzymałością materiałów'', \cite{Ska83} }


\CenterTB{Uwagi}

\start \Str{21} We~wzorze w~drugiej linii zamiast
\begin{equation}
  \sqrt{1 - \left( \tfrac{ x }{ x_{ 0 } } \right)^{ 2 } }
\end{equation}
powinno być
\begin{equation}
  \mathrm{sgn}( \cos \varphi ) \, \sqrt{1 - \left( \tfrac{ x }{ x_{ 0 } }
    \right)^{ 2 } },
\end{equation}
bo~wykorzystujemy jedynkę trygonometryczną by~wyrazić $\cos$ przez
$\sin$. Ponieważ w~dalszym ciągu obliczeń podnosimy ten człon do
kwadratu, ta niedokładność nie~wpływa na ostateczny wynik.

\vspace{\spaceFour}


\start \Str{36} Aby wyprowadzenie wzoru (3.21) było poprawne,
potrzebujemy by
$| \dot{ \eb }_{ 1 } \cdot \eb_{ 2 } | = \dot{ \eb }_{ 1 } \cdot \eb_{
  2 }$. Oznacza to, że~układ obraca się od~wektora $\eb_{ 1 }$
do~$\eb_{ 2 }$.


\CenterTB{Błędy}
\begin{center}
  \begin{tabular}{|c|c|c|c|c|}
    \hline
    & \multicolumn{2}{c|}{} & & \\
    Strona & \multicolumn{2}{c|}{Wiersz} & Jest
                              & Powinno być \\ \cline{2-3}
    & Od góry & Od dołu & & \\
    \hline
    12  & 16 & & można określić & będziemy oznaczali \\
    13  & & 10 & $\mathbf{ b }_{ y }$ & $\mathbf{ a }_{ y }$ \\
    17  &  3 & & $(\mathbf{ a } \cdot \mathbf{ c } ) \cdot \mathbf{ b }
                 - ( \mathbf{ c } \cdot \mathbf{ b } ) \cdot \mathbf{ a }$
           & $(\mathbf{ a } \cdot \mathbf{ c } ) \mathbf{ b }
             - ( \mathbf{ c } \cdot \mathbf{ b } ) \mathbf{ a }$ \\
    21  &  4 & & $\frac{ y }{ { }_{ 0 } }$ & $\frac{ y }{ { y }_{ 0 } }$ \\
    21  &  4 & & $\sqrt{ 1 \:\: \left( \frac{ x }{ x_{ 0 } }
                 \right)^{ 2 } } $
           & $\sqrt{ 1 - \left( \frac{ x }{ x_{ 0 } } \right)^{ 2 } } $ \\
    23  & &  9 & $+2\beta \cos( 2\omega t )$ & $-2\beta
                                               \cos( 2\omega t )$ \\
    31  & 13 & & $x_{ 2 } \pd{}{ x_{ 2 } }{ r }$
           & $\dot{ x }_{ 2 } \pd{}{ x_{ 2 } }{ r }$ \\
    31  & &  5 & $\mathbf{ a } \pd{}{ \mathbf{ r } }{ q_{ j } }$
           & $\mathbf{ a } \cdot \pd{}{ \mathbf{ r } }{ q_{ j } }$ \\
    32  &  8 & & $\pd{}{ { \mathbf{ r }^{ 2 } } }{ { q^{ j } } }$
           & $\pd{}{ \mathbf{ r } }{ { q^{ j } } }$ \\
    34  &  6 & & $\dot{ \mathbf{ e } }_{ i } \mathbf{ e }_{ j }$
           & $\dot{ \mathbf{ e } }_{ i } \cdot \mathbf{ e }_{ j }$ \\
    34  &  8 & & $\mathbf{ e }_{ i } \mathbf{ e }_{ j }$
           & $\mathbf{ e }_{ i } \cdot \mathbf{ e }_{ j }$ \\
    34  & 10 & & $\dot{ \mathbf{ e } }_{ i } \mathbf{ e }_{ j }
                 + \mathbf{ e }_{ i } \dot{ \mathbf{ e } }_{ j }$
           & $\dot{ \mathbf{ e } }_{ i } \cdot \mathbf{ e }_{ j }
             + \mathbf{ e }_{ i } \cdot \dot{ \mathbf{ e } }_{ j }$ \\
    34  & 11 & & $\dot{ \mathbf{ e } }_{ i } \mathbf{ e }_{ j }$
           & $\dot{ \mathbf{ e } }_{ i } \cdot \mathbf{ e }_{ j }$ \\
    34  & 12 & & $\dot{ \mathbf{ e } }_{ i } \mathbf{ e }_{ j }$
           & $\dot{ \mathbf{ e } }_{ i } \cdot \mathbf{ e }_{ j }$ \\
    36  & 16 & & $( \xi_{ 1 } \dot{ \mathbf{ e } }_{ 1 } + \xi_{ 2 }
                 \dot{ \mathbf{ e } }_{ 2 }  ) \mathbf{ e }_{ 1 }$
           & $( \xi_{ 1 } \dot{ \mathbf{ e } }_{ 1 } + \xi_{ 2 }
             \dot{ \mathbf{ e } }_{ 2 }  ) \cdot \mathbf{ e }_{ 1 }$ \\
    36  & 16 & & $( \xi_{ 1 } \dot{ \mathbf{ e } }_{ 1 } + \xi_{ 2 }
                 \dot{ \mf{ e } }_{ 2 }  ) \mf{ e }_{ 2 }$
           & $( \xi_{ 1 } \dot{ \mf{ e } }_{ 1 } + \xi_{ 2 }
             \dot{ \mf{ e } }_{ 2 }  ) \cdot \mf{ e }_{ 2 }$ \\
    36  & 18 & & $\dot{ \mf{ e } }_{ 1 } \mf{ e }_{ 2 }$
           & $\dot{ \mf{ e } }_{ 1 } \cdot \mf{ e }_{ 2 }$ \\
    36  & 20 & & $\dot{ \mf{ e } }_{ 1 } \mf{ e }_{ 2 }$
           & $\dot{ \mf{ e } }_{ 1 } \cdot \mf{ e }_{ 2 }$ \\
           % & & & & \\
    \hline
  \end{tabular}
\end{center}
\noi
\StrWg{31}{13} \\
\Jest
$( \dotx_{ 1 } \ib + \dotx_{ 2 } \jb ) \left( \pd{}{ x_{ 1 } }{ r }
  \ib + \pd{}{ x_{ 2 } }{ r } \jb \right)
\cdot \fr{ 1 }{ \bsym{ 1 } }$ \\
\Powin
$( \dotx_{ 1 } \ib + \dotx_{ 2 } \jb ) \cdot \left( \pd{}{ x_{ 1 } }{
    r } \ib + \pd{}{ x_{ 2 } }{ r } \jb \right)
\fr{ 1 }{ | \bsym{ 1 } | }$ \\

\vspace{\spaceTwo}



\begin{center}
  \Large{\textbf{Opis.}}
\end{center}

Książka W. Arnolda ,,Metody matematyczne mechaniki klasycznej''
stanowi dzięki głębi swych przemyśleń, naciskowi na~intuicyjne
uzasadnienie stosowanego formalizmu, matematycznenu wyrafionowaniu
i~niezwykłej erudycji autora jedno z~najlepszych dzieł jakie w~XX
wieku napisano na temat mechaniki klasycznej. Równoczesnie jest to
pozycja bardzo trudna, w~której elementarne zagadnienia sąsiadują
z~bardzo skomplikowanymi problemami, wiele kluczowych rozumowań jest
tylko naszkicowanych, a~pełne jej zrozumienie wymaga pewnego obycie
ze~współczesną matematyką.

Celem tych spotkań jest wspólne przestudiowanie i~zrozumienie tego
dzieła na cotygodniowych dwugodzinnych spotkaniach. W~zależności od
preferencji uczestników forma tych spotkań może przybrać postać
wykładów prowadzonych przez \ldots na podstawie poszczególnych
rozdziałów, bądź referowania przez zainteresowane osoby kolejnych
fragmentów książki. W~obu wypadkach wspólna dyskusja przedstawionego
materiału ma być centraną częścią spotkań.

\begin{center}
  \Large{\textbf{Plan.}} \\
\end{center}
\noindent
Poniżej jest lista części książki, które będą przerabiane na
spotkaniach. \\ \newline
\noindent
\textbf{Uwagi.}
\begin{itemize}
\item[--] Choć pierwsza książki traktuje o~mechanice w~sformułowaniu
  Newtona, co nie pokrywa się z planem kursu ,,Mechanika klasyczna''
  profesora Bizonia, to wprowadzone jest w~niej kilka kluczowych dla
  całej książki koncepcji, dlatego też jej część znalazła~się w~planie
  spotkań.
\item[--] Gwiazdką (*) oznaczone są paragrafy których treść można
  w~całości, albo w~większości, pominąć bez szkody dla dalszej części
  książki, jednak ze względu na ciekawy materiał warto rozważyć ich
  przerobienie w całości.
\item[--] Plan może ulec zmianie w~czasie trwania spotkań.
\item[--] Jeżeli ktoś chce zgłosić zastrzeżenie do tego planu prosze
  pisać na adres \ldots .
\end{itemize}
\begin{itemize}
\item[\textbf{Roz. I.}] \textbf{Fakty doświadczalne.}
\item[--] 1. Zasada względności i przyczynowości.
\item[--] 2. Grupa Galileusz i~równania Newtona.
\item[\textbf{Roz. II.}] \textbf{Badanie równań ruchu.}
\item[--] 4. Układy o~jednym stopniu swobody.
\item[--] 5. Układy o~dwóch stopniach swobody.
\item[--] 11*. Rozumowanie oparte na podobieństwie.
\item[\textbf{Roz. III.}] \textbf{Zasada wariacyjna (całość).}
\item[--] 12. Rachunek wariacyjny.
\item[--] 13. Równanie Lagrange'a.
\item[--] 14. Przekształcenie Legendre'a.
\item[--] 15. Równania Hamiltona.
\item[--] 16. Twierdzenie Liouville'a.
\item[\textbf{Roz. IV.}] \textbf{Mechanika Lagrange'a na
    rozmaitościach.}
\item[--] 17. Więzy holonomiczne.
\item[--] 18. Rozmaitości różniczkowalne.
\item[--] 19. Układy dynamiczne Lagrange'a.
\item[--] 20. Twierdzenie E. Noether.
\item[\textbf{Roz. V.}] \textbf{Drgania.}
\item[--] 22. Linearyzacja.
\item[--] 23. Małe drgania.
\item[--] 24*. O~zachowaniu się częstości własnych.
\item[--] 25*. Rezonans parametryczny.
\item[\textbf{Roz. VII.}] \textbf{Formy różniczkowe (całość).}
\item[--] 32. Formy zewnętrzne.
\item[--] 33. Iloczyn zewnętrzny.
\item[--] 34. Formy różniczkowe.
\item[--] 35. Całkowanie form różniczkowych.
\item[--] 36. Różniczkowanie zewnętrzne.
\item[\textbf{Roz. VIII.}] \textbf{Rozmaitości symplektyczne.}
\item[--] 37. Struktura symplektyczna na rozmaitości.
\item[--] 38. Hamiltonowskie potoki fazowe i~ich niezmienniki całkowe.
\item[--] 39. Algebry Liego pól wektorowych.
\item[--] 40. Algebra Liego pól Hamiltona.
\item[--] 41. Geometria symplektyczna.
\item[--] 42*. Rezonans parametryczny w~układach o~wielu stopniach
  swobody.
\item[--] 43. Atlas symplektyczny.
\item[\textbf{Roz. IX.}] \textbf{Formalizm kanoniczny.}
\item[--] 44*. Niezmiennik całkowy Poincar\'{e}go\dywiz Cartana.
\item[--] 45. Konsekwencje twierdzenia o~niezmienniku całkowym
  Poincar\'{e}go\dywiz Cartana.
\item[--] 46*. Zasada Huygensa.
\item[--] 47*. Metoda Jacobiego\dywiz Hamiltona całkowania równań
  kanonicznych Hamiltona.
\item[--] 48. Funkcje tworzące.
\item[\textbf{Roz. IX.}] \textbf{Wprowadzenie do teorii zaburzeń.}
\item[--] 49*. Układy całkowalne.
\item[--] 50*. Współrzędne działanie\dywiz kąt.
\item[--] 51*. Uśrednianie.
\item[--] 52*. Uśrednianie zaburzeń
\item[] \textbf{Uzupełnienia.}
\item[--] 5*. Układy dynamiczne wykazujące symetrię.
\item[--] 8*. Teoria zaburzeń dla~ruchów prawie okresowych
  i~twierdzenie Kołmogorowa.
\item[--] 12*. Osobliwości Lagrange'a.
\item[--] Na co nam jeszcze starczy sił ;). O~ile w~ogóle tu
  dotrzemy\ldots
\end{itemize}

\begin{center}
  \Large{\textbf{Bibliografia.}}
\end{center}

\noindent
Poniższa lista zawiera pozycje zarówno skierowane zarówno dla osób
które chcą przeczytać szersze opracowanie niektórych zagadnień, jak
i~tych które chcą poznać w~jaki sposób można uogólnić omawiane
zagadnienia. Niestety dla wszystkich omawianych problemów nie
udało~się znaleźć zadowalającej\linebreak literatury, pewnych zaś
wartych uwagii pozycji nie~umieszczono na niej, jako mało adekwentnych
do~treści spotkań. \newline
\noindent
\textbf{BWMiI} -- Biblioteka Wydziału Matematyki i Informatyki. \\
\newline
\noindent
\textbf{Cykl W. Arnolda.} Książki te optymalnie byłoby czytać
w~podanej poniżej kolejności, stąd obecność tu omawianego na
spotkaniach dzieła.
\begin{itemize}
\item[--] \emph{Równania różniczkowe zwyczajne} (RRZ), BWMiI.
\item[--] \emph{Metody matematyczne mechaniki klasycznej} (MMMK),
  BWMiI. %Biblioteka Wydziału Matematyki i Informatyki.
\item[--] \emph{Teoria równań różniczkowych} (TRR), BWMiI.
\end{itemize}

% \textbf{Mechanika klasyczna.}
\begin{itemize}
\item[] \textbf{Podstawy matematyczne.}
\item[--] L. Schwartz, \emph{Kurs analizy matematycznej, tom I} (LSI),
  większość bibliotek np. NKFu. Książka trudna, ale zawiera dowód
  chyba każdego twierdzenia z~analizy jakie będzie potrzebne.
\item[--] A. Herdegen, \emph{Algebra liniowa i~geometria} (AH).
  Głównie twierdzenia o~formach kwadratowych\footnote{Jedyną inną
    pozycją, która o~ile wiem zawiera dowody potrzebnych twierdzeń,
    jest książka ,,Wykłady z~algebry liniowej'' I. M. Gelfanda.}.
\item[] \textbf{Struktura czasoprzestrzeni mechaniki Newtona.}
\item[--] W. Kopczyński, A. Trautman, \emph{Czasoprzestrzeń i
    grawitacja} (KT), biblioteki FAISu, NKFu etc. Dobra, krótka
  i~trudna pozycja, jedna z~niewielu które zajmują~się tym tematem.
\item[] \textbf{Mechanika klasyczna.}
\item[--] E. T. Whittaker, \emph{Dynamika analityczna} (ETW),
  biblioteka NKFu. Wiekowa, lecz wciąż warta uwagii pozycja.
\item[--] R. Abraham, J. E. Marsden, \emph{Foundations of Mechanics,
    Second Edition} (FoM2),
  \url{http://authors.library.caltech.edu/25029/} . Monumentalne
  dzieło starające~się z~pełną ścisłością przedstawić mechanikę za
  pomocą metod współczesnej matematyki.
\item[] \textbf{Równania różniczkowe zwyczajne.}
\item[--] W. Walter, \emph{Ordinary differential equations} (WWODEs),
  Springer Link. Rozsądny, współczesny wykład podstaw teorii ODEs.
\item[--] E. Hairer, S. P. N\o rsett, G. Warner, \emph{Solving Ordinary
    Differential Equations} (SODEs), Springer Link. Monumentlane
  dzieło o~tym jak analitycznie, a~przedewszystkim numerycznie
  rozwiązać dane równanie.
\item[] \textbf{Rachunek wariacyjny.}
\item[--] I. M. Gelfand, S. V. Fomin, \emph{Rachunek wariacyjny} (GF),
  większość bibliotek, np. NKFu i FAISu. Standardowy wykład
  klasycznych osiągnięć rachunku wariacyjnego.
\item[--] J. Jost, X. Li-Jost, \emph{Calculus of Variations} (JLJ),
  BWMiI. Podręcznik zawierający wprowadzenie do~wpółczesnych metod
  w~rachunku wariacyjnym.
\item[--] M. Giaquinta, St. Hildebrandt, \emph{Calculus of Variations}
  (GHCoV), Springer Link. Monografia starająca~się dać możlwie
  wyczerpujący opis współczesnych metod.
\item[] \textbf{Geometria różniczkowa.}
\item[--] J. Gancarzewicz, \emph{Zarys współczesnej geometrii
    różniczkowej} (ZWGR). Abstarakcyjna, długa i~niepozbawiona sporych
  błędów, jednak bardzo dobra pozycja dla średnio zawansowanych.
\item[--] R. Sulanke, P. Wintgen, \emph{Geometria różniczkowa i~teoria
    wiązek} (SW), BWMiI. Pozycja wprowadzająca, zawierająca dobre
  wprowadzenie do teorii tożsamości geometryczno\dywiz całkowych
  na~rozmaitościach.
\item[] \textbf{Teoria form.}
\item[--] L. Schwartz, \emph{Kurs analizy matematycznej, tom II}
  (LSII), BWMiI. Bez znajomości teorii całki\linebreak z~I tomu,
  prawie nie do zrozumienia.
\item[--] S. G. Krantz, H. R. Parks, \emph{Geometric Integration
    Theory} (GIT), Springer Link. Zawiera wprowadzenie do teorii
  prądów, pozwalajacej rozważać formy o~wartościach w~dystrybucjach.
\item[] \textbf{Geometria symplektyczna i mechanika.}
\item[--] P. Libermann, Ch.\dywiz M. Marle, \emph{Symplectic Geometry
    and Analytical Mechanics} (SGAM), BWMiI, Springer Link. Wykład
  geometrii syplektycznej ilustrowany zastosowaniami w mechanice.
  \newpage
\item[] \textbf{Układy nieholonomiczne.}
\item[--] J. I. Nejmark, N. A. Fufajew, \emph{Dynamika układów
    nieholonomicznych} (DUN), Allegro\footnote{Prostszy sposób nie
    jest znany.}. Podstawowe, w~dobry sposób staroświeckie, dzieło
  w~tej dziedzinie.
\item[--] E. Massa, E. Paganim \emph{A new look at classical mechanics
    of constrained systems} (EMEP),\newline
  \url{https://eudml.org/doc/76747} . Nowoczesna próba zmierzenia~się
  z~zagadieniem więzów nieholonomicnzych. Dość trudna pozycja.
\item[--] H. Geiges \emph{Contact geometry} (HGCM),
  arXiv:math/0307242v2 [math.SG]
  \url{http://arxiv.org/abs/math/0307242} . Można tu znaleść dobre,
  jak na ten dział matematyki, wprowadzenie w~teorię rozmaitości
  kontaktowych, podstawowę matematycznego opisu układów
  nieholonomicznych.
\item[] \textbf{Geometria różniczkowa poza mechaniką.}
\item[--] G. Svetlichny, \emph{Preparation for Gauge Theory} (PGT),
  arXiv:math-ph/9902027v3 \url{http://arxiv.org/abs/math-ph/9902027} .
  Czasami trochę zbyt zwięzły, lecz merytorycznie bardzo dobry, wykład
  na temat zastosowania geometrii różniczkowej, i~pochodnych działów
  matematyki, do opisu \textbf{klasycznych} teorii pola z~cechowaniem,
  takich jak elektrodynamika, czy pola Yanga\dywiz Millesa.
\end{itemize}

\textbf{Ważne:}
\begin{quote}
  Dla prawdziwego matematyka, jest dużo ważniejsze by wiedzieć jakie
  problemy nie zostały wciąż rozwiązane i~gdzie znane obecnie metody
  okazały~się niewystarczające, niż pamiętać wszystkie liczby których
  iloczyny udało~się do tej pory uzyskać, czy orientować się w~
  ocenianie literatury stowrzonej na przestrzeni ostatnich 20 tysięcy
  lat.
\end{quote}
W. Arnold w~przedmowie do książki ,,Arnold's Problems'', Springer Link, tłumaczenie swobodne.\\
\begin{quote}
  Zauważyliśmy bowiem, że dla początkujących słuchaczy dużą przeszkodę
  w zdobywaniu tej nauki stanowią dzieła różnych teologów: już to
  dlatego, że są nadmiernie przeładowane bezużytecznymi zagadnieniami,
  artykułami i dowodami, już to dlatego, że zagadnienia, z jakimi owi
  początkujący winni koniecznie się zapoznać, nie są podane
  systematycznie: według uporządkowanej kolejności nauk czy traktatów,
  ale omawiane są albo w związku z komentowaniem dzieł, albo z okazji
  dysputy; już to wreszcie dlatego, że częste wałkowanie tego samego
  budziło w ich umysłach nudę i zamęt.

  Ufni w pomoc Bożą i starając się uniknąć tych i podobnych
  niedociągnięć, będziemy usiłowali krótko i jasno - o ile na to sama
  rzecz pozwoli - wyłożyć wszystko, co zakresem swoim obejmuje nauka
  święta.
\end{quote}
Św. Tomasz z~Akwinu we wstępie do \emph{Sumy Teologicznej},
\url{http://www.katedra.uksw.edu.pl/suma/suma_1.pdf}.





% #####################################################################
% #####################################################################
% Bibliografia
\bibliographystyle{plalpha} \bibliography{LibPhilNatur}{}


% ############################

% Koniec dokumentu
\end{document}
