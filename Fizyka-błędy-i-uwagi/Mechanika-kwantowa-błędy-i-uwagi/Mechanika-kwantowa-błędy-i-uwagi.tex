% Autor: Kamil Ziemian

% ---------------------------------------------------------------------
% Podstawowe ustawienia i pakiety
% ---------------------------------------------------------------------
\RequirePackage[l2tabu, orthodox]{nag}  % Wykrywa przestarzałe i niewłaściwe
% sposoby używania LaTeXa. Więcej jest w l2tabu English version.
\documentclass[a4paper,11pt]{article}
% {rozmiar papieru, rozmiar fontu}[klasa dokumentu]
\usepackage[MeX]{polski}  % Polonizacja LaTeXa, bez niej będzie pracował
% w języku angielskim.
\usepackage[utf8]{inputenc}  % Włączenie kodowania UTF-8, co daje dostęp
% do polskich znaków.
\usepackage{lmodern}  % Wprowadza fonty Latin Modern.
\usepackage[T1]{fontenc}  % Potrzebne do używania fontów Latin Modern.



% ------------------------------
% Podstawowe pakiety (niezwiązane z ustawieniami języka)
% ------------------------------
\usepackage{microtype}  % Twierdzi, że poprawi rozmiar odstępów w tekście.
% \usepackage{graphicx}  % Wprowadza bardzo potrzebne komendy do wstawiania
% grafiki.
% \usepackage{verbatim}  % Poprawia otoczenie VERBATIME.
% \usepackage{textcomp}  % Dodaje takie symbole jak stopnie Celsiusa,
% wprowadzane bezpośrednio w tekście.
\usepackage{vmargin}  % Pozwala na prostą kontrolę rozmiaru marginesów,
% za pomocą komend poniżej. Rozmiar odstępów jest mierzony w calach.
% ------------------------------
% MARGINS
% ------------------------------
\setmarginsrb
{ 0.7in}  % left margin
{ 0.6in}  % top margin
{ 0.7in}  % right margin
{ 0.8in}  % bottom margin
{  20pt}  % head height
{0.25in}  % head sep
{   9pt}  % foot height
{ 0.3in}  % foot sep



% ------------------------------
% Często przydatne pakiety
% ------------------------------
% \usepackage{csquotes}  % Pozwala w prosty sposób wstawiać cytaty do tekstu.
\usepackage{xcolor}  % Pozwala używać kolorowych czcionek (zapewne dużo
% więcej, ale ja nie potrafię nic o tym powiedzieć).



% ------------------------------
% Pakiety do tekstów z nauk przyrodniczych
% ------------------------------
\let\lll\undefined  % Amsmath gryzie się z językiem pakietami do języka
% polskiego, bo oba definiują komendę \lll. Aby rozwiązać ten problem
% oddefiniowuję tę komendę, ale może tym samym pozbywam się dużego Ł.
\usepackage[intlimits]{amsmath}  % Podstawowe wsparcie od American
% Mathematical Society (w skrócie AMS)
\usepackage{amsfonts, amssymb, amscd, amsthm}  % Dalsze wsparcie od AMS
% \usepackage{siunitx}  % Do prostszego pisania jednostek fizycznych
\usepackage{upgreek}  % Ładniejsze greckie litery
% Przykładowa składnia: pi = \uppi
% \usepackage{slashed}  % Pozwala w prosty sposób pisać slash Feynmana.
\usepackage{calrsfs}  % Zmienia czcionkę kaligraficzną w \mathcal
% na ładniejszą. Może w innych miejscach robi to samo, ale o tym nic
% nie wiem.



% ##########
% Tworzenie otoczeń "Twierdzenie", "Definicja", "Lemat", etc.
\newtheorem{twr}{Twierdzenie}  % Komenda wprowadzająca otoczenie
% ,,twr'' do pisania twierdzeń matematycznych
\newtheorem{defin}{Definicja}  % Analogicznie jak powyżej
\newtheorem{wni}{Wniosek}



% ------------------------------
% Pakiety napisane przez użytkownika.
% Mają być w tym samym katalogu to ten plik .tex
% ------------------------------
\usepackage{mechanikakwantowa}  % Pakiet napisany konkretnie dla tego pliku.
\usepackage{latexgeneralcommands}
\usepackage{mathcommands}



% ---------------------------------------------------------------------
% Dodatkowe ustawienia dla języka polskiego
% ---------------------------------------------------------------------
\renewcommand{\thesection}{\arabic{section}.}
% Kropki po numerach rozdziału (polski zwyczaj topograficzny)
\renewcommand{\thesubsection}{\thesection\arabic{subsection}}
% Brak kropki po numerach podrozdziału



% ------------------------------
% Ustawienia różnych parametrów tekstu
% ------------------------------
\renewcommand{\arraystretch}{1.2}  % Ustawienie szerokości odstępów między
% wierszami w tabelach.



% ------------------------------
% Pakiet "hyperref"
% Polecano by umieszczać go na końcu preambuły.
% ------------------------------
\usepackage{hyperref}  % Pozwala tworzyć hiperlinki i zamienia odwołania
% do bibliografii na hiperlinki.










% ---------------------------------------------------------------------
% Tytuł, autor, data
\title{Mechanika kwantowa --~błędy i~uwagi}

% \author{}
% \date{}
% ---------------------------------------------------------------------










% ####################################################################
\begin{document}
% ####################################################################





% ######################################
\maketitle % Tytuł całego tekstu
% ######################################





% ######################################
\section{Mechanika kwantowa}

\vspace{\spaceTwo}
% ######################################



% ############################
\Work{ % Autorzy i tytuł dzieła
  Ramamurti Shankar \\
  „Mechanika kwantowa”, \cite{ShankarMechanikaKwantowa2006} }


% ##################
\CenterBoldFont{Uwagi}


\start \Str{74} Na rysunku 1.8 b) aby otrzymać poprawny wykres
pochodnej funkcji Gaussa należy odbić wykres przedstawiony względem osi
$y = 0$.

\vspace{\spaceFour}





% ##################
\CenterBoldFont{Błędy}


\begin{center}

  \begin{tabular}{|c|c|c|c|c|}
    \hline
    & \multicolumn{2}{c|}{} & & \\
    Strona & \multicolumn{2}{c|}{Wiersz} & Jest
                              & Powinno być \\ \cline{2-3}
    & Od góry & Od dołu & & \\
    \hline
    % & & & & \\
    % & & & & \\
    % & & & & \\
    23 & 19 & & antyrównoległą & równoległą \\
    % & & & & \\
    224 & & 6 & $\left\{ -\frac{ 1 }{ i } [ ( y_{ 2  } - y_{ 1 } )^{ 2 }
                + ( y_{ 1 } - y_{ 0 } )^{ 2 } ] \right\}$
           & $\exp\left\{ -\frac{ 1 }{ i } [ ( y_{ 2  } - y_{ 1 } )^{ 2 }
             + ( y_{ 1 } - y_{ 0 } )^{ 2 } ] \right\}$ \\
    225 & 9 & & $N \varepsilon \to t_{ n } - t_{ 0 }$
           & $N \varepsilon = t_{ n } - t_{ 0 }$ \\
    228 & & 8 & $m$ & $\frac{ 1 }{ 2 } m$ \\
    \hline
  \end{tabular}

\end{center}


\noindent
Str. 20. Zdanie na dole strony jest mętne. Popraw to.?????

Str. 26. \ldots tylko wtedy, gdy $| V \rangle = 0$\ldots

Str. 78. % \ii czy i?
$$\ldots = i \int_{a}^{b} \frac{ dg^{ * } }{ dx } f( x ) \dPL x \, .$$

Str. 81.
  $$\langle k' | X | k \rangle = \frac{ 1 }{ 2 \pi } \int_{ -\infty
  }^{ \infty } \e^{ -i k' x } x \e^{ i k x } \dPL x = -i \frac{ d }{ dk } \bigg( \frac{ 1 }{ 2 \pi } \int_{ -\infty }^{ \infty } \e^{ i ( k - k' ) x } \bigg) = -i \delta'( k - k' ) \, .$$ Wobec
  tego, jeśli $| g( k ) \rangle$ jest wektorem, którym w bazie $K$
  odpowiada funkcja $g( k )$, to
$$X| g( k ) \rangle = \bigg| \frac{ -i \dPL g( k ) }{ \dPL k } \bigg\rangle \, .$$
Podsumujmy: w bazie $X$ operator $X$ działa jak $x$, a operator $K$
jak $-i d / d x$ (na funkcje $f( x )$), a w bazie $K$ działa jak
$k$, a operator $X$ jak $-i d / d k$\ldots


\vspace{\spaceTwo}
% ############################










% ############################
\Work{ % Autorzy i tytuł dzieła
  L. I. Schiff \\
  „Mechanika kwantowa”, \cite{Sch87}.}


% ##################
\CenterBoldFont{Uwagi}


\noindent
Nie rozumiem, i~chyba nie~powinienem rozumieć, eksperymentów
ilustrujących zasadę nieoznaczoności i~kolaps funkcji falowej
(słynny eksperyment z~dwoma szczelinami). We wszystkich tych
zagadnieniach centralną rolę odgrywa foton, który z~natury swojej
jest cząstką relatywistyczną i~nie~można go opisać w~ramach
nierelatywistycznej mechaniki kwantowej której poświęcona jest
większa część książki. Co prawda eksperyment dyfrakcji na dwóch
szczelinach można przeprowadzić dla~cząstek nierelatywistycznych, to
przy pozostałych należy~się chwilę zastanowić. Podejrzewam jednak,
że można znaleźć ich nierelatywistyczne odpowiedniki. Jednak w~samej
nierelatywistycznej mechanice kwantowej foton nie występuje, żadna
też inna cząstka nie~jest~potrzeba, by~relacje
nieoznaczoności~wynikały~z~jej formalizmu matematycznego. Jak
zauważył to Konrad Szymański, wynika to po~prostu z~rozciągłości,
w~szczególnym przypadku, przestrzennej paczki falowej.

\vspace{\spaceFour}



\noindent
\Str{21} Jak zauważył Paweł Duch, nierówność (3.2)
nie~może być prawdziwa, bo istnieją stany własne $J_{ z }$.

\vspace{\spaceFour}



\noindent
\Str{59} Choć odległość między dwoma sąsiednimi
wektorami $\veckbold$ wynosi $\frac{ 2 \pi }{ L }$ i~można ją
uczynić dowolnie małą przez odpowiedni dobór $L$, to analogiczne
stwierdzenie odnośnie energii jest błędne. Przyjmując wektory
sąsiednie jako
$\veckbold_{ 1 } = \frac{ 2 \pi }{ L } [ n_{ x }, n_{ y }, n_{ z } ]$
i~$\veckbold_{ 1 } = \frac{ 2 \pi }{ L } [ n_{ x } + 1, n_{ y }, n_{ z } ]$
różnica ich energii kinetycznej wynosi:
\begin{equation}
  \label{eq:Schiff-01}
  \frac{ \hbar^{ 2 } \veckbold_{ 2 }^{ 2 } }{ 2 m }
  - \frac{ \hbar^{ 2 } \veckbold_{ 1 }^{ 2 } }{ 2 m }
  = \frac{ \hbar^{ 2 } }{ 2 m } \frac{ ( 2 n_{ x } + 1 ) }{ L^{ 2 } }.
\end{equation}
Przy ustalonym $L$ ta wielkość jest dowolnie duża dla odpowiednio
wysokiego $n_{ x }$. Odległości między sąsiednimi stanami można
uważać, za małe tylko jeśli mamy górne ograniczenie na energię,
mówiąc inaczej jeśli w~układzie mamy energię Fermiego.





% ##################
\CenterBoldFont{Błędy}


\begin{center}

  \begin{tabular}{|c|c|c|c|c|}
    \hline
    & \multicolumn{2}{c|}{} & & \\
    Strona & \multicolumn{2}{c|}{Wiersz} & Jest
                              & Powinno być \\ \cline{2-3}
    & Od góry & Od dołu & & \\
    \hline
    % & & & & \\
    17 & 12 & & 1904 & 1905 \\
    31 & & 4 & jakakolwiek & taka \\
    35 & 9 & & $| \psi( \vecrbold, t |^{ 2 }$ & $| \psi( \vecrbold, t ) |^{ 2 }$ \\
    42 & 2 & & ograniczone & zlokalizowane \\
    54 & 16 & & wartości & dyskretne wartości \\
    54 & & 2 & dwom & dwóm \\
    62 & & 8 & $z + z'$ & $z - z'$ \\
    % & & & & \\
    \hline
  \end{tabular}

\end{center}


\noindent
\StrWg{21}{16} \\
\Jest  orbity**.Równanie(3.3)implikuje\ldots \\
\Powin orbity**. Równanie (3.3) implikuje\ldots \\
\StrWg{54}{18} \\
\Jest  w~odpowiadających im punktach\ldots  \\
\Powin w~obszarze przez te ścianki zajętym\ldots \\


\vspace{\spaceTwo}
% ############################










% ############################
\Work{ % Autorzy i tytuł dzieła
  Marian Grabowski, Roman S.~Ingarden \\
  „Mechanika kwantowa. Ujęcie w~przestrzeni Hilberta”,
  \cite{GrabowskiIngardenMechanikaKwantowa1987} }


% ##################
\CenterBoldFont{Uwagi do konkretnych stron}


W książce powinna być jawnie zamieszczona informacja, że każda skończenie wymiarowa
podprzestrzeń przestrzeni Hilberta (ogólniej: przestrzeni unormowanej), jest
domknięta. Wynika to, choćby z tego, że każda podprzestrzeń
skończenie wymiarowa jest lokalnie zwarta.

\vspace{\spaceFour}



Str. Jest tu przykład rozumowania z ogromną dziurą. Nie możemy
korzystać z własności przestrzeni Hilberta dopóki nie udowodnimy, że
jest to przestrzeń Hilberta.

\vspace{\spaceFour}



\Str{27} Jest tu pewne zamieszanie odnośnie jednoznaczności
rozkładu. Rozkład na element najbliższy w danej podprzestrzeni i
część ortogonalną musi być jednoznaczny, jeśli istnieje, ze względu
na jednoznaczność rzutu. Nie mniej, nie rozstrzyga to problemu, czy
istnieje alternatywny rzut na te podprzestrzenie. Negatywną odpowiedź
daje nam fakt iż:
$\Mcal \cup \Mcal^{ \bot } = \{ \emptyset \}$.

\vspace{\spaceFour}



\Str{27} Przedstawione tu pojęcie zupełności jest trochę
mylące. Podana tu definicja zupełności odpowiada pojęciu
\textit{totalności} omówionej w książce Waltera Thirring „Fizyka
matematyczna. Tom III”. Przede wszystkim z podanej definicji zbioru
zupełnego nie wynika, że każdy wektor z $\Hcal$ można
przedstawić jako szereg elementów tego zbioru. Stąd właśnie Thirring
rozróżnia pojęcie totalności i zupełności.

\vspace{\spaceFour}



\Str{27} W dowodach Wniosków I i II, jest dwa razy użyte
twierdzenie, że wektor ortogonalny do danego zbioru jest też
ortogonalny do jego domknięcia, w dowodzie pierwszego wniosku
wyrażone słownie, w drugim za pomocą wzorów. Warto byłoby zrobić to
bardzie elegancko.

\vspace{\spaceFour}



\Str{28} W dowodzie wniosku I.2. jest coś dziwnego. Uwaga, że
należy przyjrzeć się uzyskanym sumom prostym i wywnioskować z nich,
iż $[ \Mcal ]=( \Mcal^{ \bot } )^{ \bot }$, równie
dobrze prowadzi od razu do wniosku
$[ \mathcal{ M } ]^{ \bot } = \mathcal{ M }^{ \bot }$. Zachodzi
bowiem twierdzenie: jeżeli
$A_{ 1 } \oplus A_{ 2 } = B_{ 1 } \oplus B_{ 2 } = X$ i
$B_{ i } \subset A_{ i }$ to $B_{ i } = A_{ i }$. Załóżmy,że tak nie
jest. Wtedy istnieje
$( x_{ 1 }, x_{ 2 } ) \in A_{ 1 } \oplus A_{ 2 }$, taka że
$( x_{ 1 }, x_{ 2 } ) \notin B_{ 1 } \oplus B_{ 2 }$. Teraz istnieje
taki $( y_{ 1 }, y_{ 2 } ) \in B_{ 1 } \oplus B_{ 2 }$, że
$x_{ 1 } + x_{ 2 } = y_{ 1 } + y_{ 2 }$, czyli
$A_{ 1 } \oplus A_{ 2 }$ nie jest sumą prostą.

\vspace{\spaceFour}



\Str{38} Pojawia się tu pojęcie operatora ograniczonego, które
jest wprowadzone dopiero na stronie 39.

\vspace{\spaceFour}



\Str{40} W dowodzie lematu II.1, gdy mowa jest o udowodnieniu
pierwszej równości w punkcie a), w istocie udowodniono równość:
\begin{equation}
  \label{eq:GrabowskiIngarden-01}
  \Vert A \Vert = \sup_{ \Vert \varphi \Vert = 1 } \Vert A \varphi \Vert \, .
\end{equation}

\vspace{\spaceFour}



\Str{46} Punkty twierdzenia są ustawione w dziwnej kolejności,
biorąc pod uwagę logikę dowodu.

\vspace{\spaceFour}



\Str{48} Z tego, że dana liczba nie należy do widma, nie wynika
że nie istnieje dla niej rezolwenta.

\vspace{\spaceFour}



\Str{49} Drugie stwierdzenie z punktu (\romannumeral4), jest już
  zawarte w punkcie (\romannumeral2).

\vspace{\spaceFour}



\Str{52} Użyte tu pojęcie funkcji charakterystycznej, nie jest
chyba nigdzie w książce przedstawione.

\vspace{\spaceFour}



\Str{57} Warto byłoby omówić szerszej pojęcie domkniętego
rozszerzenia operatora, domykalności operatora i jego domknięcia. W~szczególności z twierdzenia o wykresie domkniętym wynika, że
domknięcie operatora jest zawsze operatorem ograniczonym.

\vspace{\spaceFour}



\Str{58} Uwaga o twierdzeniu II (\romannumeral3) jest zupełnie
niezrozumiała.

\vspace{\spaceFour}



\Str{58} Zdefiniowaniu rezolwenty i widma operatora
nieograniczonego powinno zostać poświęcone więcej miejsca.

\vspace{\spaceFour}



\Str{58} Nie dodano, że widmo nieograniczonego operatora
domykalnego, definiujemy jako widmo jego domknięcia.

\vspace{\spaceFour}



\Str{58} Jedyność wektora $\eta$ wynika już z lematu Riesza.

\vspace{\spaceFour}



\Str{80} Ustalenie takiej wartości stałej $C$, ani w ogóle
ustalenie jej wartości, nie jest potrzebne w rozważanym zagadnieniu.

\vspace{\spaceFour}



\Str{313} Nie wspomniano tu w jakim sensie dane ciągi funkcji
mają być zbieżne. Osoba znająca teorię całki Lebesgue'a wie, że
wystarczy założyć zbieżność punktową, a nawet tylko zbieżność
punktową prawie wszędzie.

\vspace{\spaceFour}



\Str{315} W twierdzeniu Lebesgue'a o zbieżności majoryzowanej
brakuje założenia o~zbieżności rozważanego ciągu funkcji.





% ##################
\CenterBoldFont{Błędy}


\begin{center}

  \begin{tabular}{|c|c|c|c|c|}
    \hline
    & \multicolumn{2}{c|}{} & & \\
    Strona & \multicolumn{2}{c|}{Wiersz} & Jest
                              & Powinno być \\ \cline{2-3}
    & Od góry & Od dołu & & \\
    \hline
    % & & & & \\
    21  &  1 & & $\{ x ,\! \absOne{ \psi( x ) - \varphi( x ) } > 0 \}$
           & $\{ x ;\, | \psi( x ) - \varphi( x )| > 0 \}$ \\
    27  & 12 & & $\Hcal_{ 1 } \otimes \Hcal_{ 2 }$ & $\Hcal_{ 1 } \oplus \Hcal_{ 2 }$ \\
    27  & 13 & & $\Rcal \otimes \Rcal^{ \bot }$ & $\Rcal \oplus \Rcal^{ \bot }$ \\
    30  & & 18 & $\xi - \varphi_{ k } \Vert$ & $\Vert \xi - \varphi_{ k } \Vert$ \\
    32  & & 11 & Teraz$f( a ) = g( 0 )$,$f( b )$
           & Teraz $f( a ) = g( 0 )$, $f( b )$ \\
    34  & & 15 & & $[ a, b ]$ \\
    101 & &  4 & $Px )$ & $P( x )$ \\
    147 &  2 & & bogaci & ubogaci \\
    311 & 10 & & $i$.Jeżeli & $i$. Jeżeli \\
    311 & 15 & & $X$ spełniającą & $X$, spełniającą \\
    312 & & 10 & $\mu$ skończona & $\mu$-skończona \\
    313 &  6 & & $\{ x ;\! f( x ) > a \}$ & $\{ x ;\, f( x ) > a \}$ \\
    313 & 15 & & $0 = a_{ 0 }$ & $0 \leq a_{ 0 }$ \\
    313 & 15 & & $x,$ & $x;$ \\
    313 & 16 & & $A_{ 0 } \ldots \cup A_{ n }$
           & $A_{ 0 } \cup \ldots \cup A_{ n }$ \\
    315 & 3 & & $A^{ 0 } \leq f_{ 1 }( x )$ & $A: 0 \leq f_{ 1 }( x )$ \\
    318 & & 7 & Caucy'ego & Cauchy'ego \\
    % & & & & \\
    \hline
  \end{tabular}

\end{center}


\noindent
\Str{27} Niech $\varphi \bot [ \Pcal ]$. Wówczas $\varphi \bot \Pcal$ i $\varphi = 0$. \\
\Str{37} \ldots$\psi =
\frac{ \overline{ l ( \varphi_{ 0 } ) } }{ \Vert \varphi_{ 0 } \Vert^{ 2 } } \varphi_{ 0 } \, .$
\Str{40} \ldots punkcie $D( A )$\ldots \\
\Str{40} \ldots określone na $D( A )$\ldots \\
\Str{41} \ldots dużych $n$ mamy $\Vert A_{ n } - A_{ m } \Vert < 2 \epsilon$. \\
\Str{47} \ldots wynika, że $S_{ \lambda } = ( \lambda I - A )^{ -1 } \in B( \Hcal )$. \\
\Str{????} \ldots być równo zbiorowi $\{ 0 \}$ \ldots
\Str{54} \ldots może być równa zbiorowi $\{ 0 \}$\ldots \\
\Str{71} \ldots$A^{ * } A = S U^{ * } U S = S E S$\ldots \\
\Str{74} \ldots interpretacją. Teraz\ldots \\
\Str{85} $\ldots = -i \frac{ d }{ dx } \frac{ 1 }{ \sqrt{ 2 \pi } }
\int\limits_{ -\infty }^{ +\infty } ( \e^{ i x s } - 1 ) \varphi( s ) \dPL s \textrm{.}$ \\
\Str{86} $F^{ -1 } Q \varphi = -i \frac{ d }{ dx } \frac{ 1 }{ \sqrt{ 2 \pi } }
\int\limits_{ -\infty }^{ +\infty } ( \e^{ i x s } - 1 ) \varphi( s ) \dPL s
= \frac{ 1 }{ \hbar } P F^{ -1 } \varphi$ \\
\Str{311} \ldots dla każdego $i$. Jeżeli\ldots


\vspace{\spaceTwo}
% ############################










% ######################################
\newpage
\section{Informatyka kwantowa}

\vspace{\spaceTwo}
% ######################################



% ############################
\Work{ % Autor i tytuł dzieła
  Richard P. Feynman \\
  „Wykłady o~obliczeniach”, \cite{FeynmanWykladyOObliczeniach2007} }


% ##################
\CenterBoldFont{Błędy}


\begin{center}

  \begin{tabular}{|c|c|c|c|c|}
    \hline
    & \multicolumn{2}{c|}{} & & \\
    Strona & \multicolumn{2}{c|}{Wiersz} & Jest
                              & Powinno być \\ \cline{2-3}
    & Od góry & Od dołu & & \\
    \hline
    % & & & & \\
    % & & & & \\
    % & & & & \\
    % & & & & \\
    % & & & & \\
    % & & & & \\
    % & & & & \\
    % & & & & \\
    % & & & & \\
    % & & & & \\
    % & & & & \\
    % & & & & \\
    % & & & & \\
    % & & & & \\
    % & & & & \\
    279 & & 11 & jednen & jeden \\
    \hline
  \end{tabular}

\end{center}


\vspace{\spaceTwo}
% ############################









% #####################################################################
% #####################################################################
% Bibliografia
\bibliographystyle{plalpha}

\bibliography{PhilNaturBooks}{}





% ############################

% Koniec dokumentu
\end{document}
