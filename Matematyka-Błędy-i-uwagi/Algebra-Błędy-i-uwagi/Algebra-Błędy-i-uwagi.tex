% ---------------------------------------------------------------------
% Podstawowe ustawienia i pakiety
% ---------------------------------------------------------------------
\RequirePackage[l2tabu, orthodox]{nag}  % Wykrywa przestarzałe i niewłaściwe
% sposoby używania LaTeXa. Więcej jest w l2tabu English version.
\documentclass[a4paper,11pt]{article}
% {rozmiar papieru, rozmiar fontu}[klasa dokumentu]
\usepackage[MeX]{polski}  % Polonizacja LaTeXa, bez niej będzie pracował
% w języku angielskim.
\usepackage[utf8]{inputenc}  % Włączenie kodowania UTF-8, co daje dostęp
% do polskich znaków.
\usepackage{lmodern}  % Wprowadza fonty Latin Modern.
\usepackage[T1]{fontenc}  % Potrzebne do używania fontów Latin Modern.



% ------------------------------
% Podstawowe pakiety (niezwiązane z ustawieniami języka)
% ------------------------------
\usepackage{microtype}  % Twierdzi, że poprawi rozmiar odstępów w tekście.
% \usepackage{graphicx}  % Wprowadza bardzo potrzebne komendy do wstawiania
% % grafiki.
% \usepackage{verbatim}  % Poprawia otoczenie VERBATIME.
% \usepackage{textcomp}  % Dodaje takie symbole jak stopnie Celsiusa,
% % wprowadzane bezpośrednio w tekście.
\usepackage{vmargin}  % Pozwala na prostą kontrolę rozmiaru marginesów,
% za pomocą komend poniżej. Rozmiar odstępów jest mierzony w calach.
% ------------------------------
% MARGINS
% ------------------------------
\setmarginsrb
{ 0.7in} % left margin
{ 0.6in} % top margin
{ 0.7in} % right margin
{ 0.8in} % bottom margin
{  20pt} % head height
{0.25in} % head sep
{   9pt} % foot height
{ 0.3in} % foot sep



% ------------------------------
% Często przydatne pakiety
% ------------------------------
\usepackage{csquotes}  % Pozwala w prosty sposób wstawiać cytaty do tekstu.
\usepackage{xcolor}  % Pozwala używać kolorowych czcionek (zapewne dużo
% więcej, ale ja nie potrafię nic o tym powiedzieć).



% ------------------------------
% Pakiety do tekstów z nauk przyrodniczych
% ------------------------------
\let\lll\undefined  % Amsmath gryzie się z pakietami do języka
% polskiego, bo oba definiują komendę \lll. Aby rozwiązać ten problem
% oddefiniowuję tę komendę, ale może tym samym pozbywam się dużego Ł.
\usepackage[intlimits]{amsmath}  % Podstawowe wsparcie od American
% Mathematical Society (w skrócie AMS)
\usepackage{amsfonts, amssymb, amscd, amsthm}  % Dalsze wsparcie od AMS
% \usepackage{siunitx}  % Do prostszego pisania jednostek fizycznych
\usepackage{upgreek}  % Ładniejsze greckie litery
% Przykładowa składnia: pi = \uppi
\usepackage{slashed}  % Pozwala w prosty sposób pisać slash Feynmana.
\usepackage{calrsfs}  % Zmienia czcionkę kaligraficzną w \mathcal
% na ładniejszą. Może w innych miejscach robi to samo, ale o tym nic
% nie wiem.



% ------------------------------
% Tworzenie środowisk (?) „Twierdzenie”, „Definicja”, „Lemat”, etc.
% ------------------------------
% Komenda wprowadzająca otoczenie „theorem” do pisania twierdzeń
% matematycznych.
\newtheorem{theorem}{Twierdzenie}
% Analogicznie jak powyżej
\newtheorem{definition}{Definicja}
\newtheorem{corollary}{Wniosek}



% ------------------------------
% Pakiety napisane przez użytkownika.
% Mają być w tym samym katalogu to ten plik .tex
% ------------------------------
% \usepackage{reedsimon}  % Pakiet napisany konkretnie dla tego pliku.
\usepackage{latexgeneralcommands}
\usepackage{mathcommands}



% ---------------------------------------------------------------------
% Dodatkowe ustawienia dla języka polskiego
% ---------------------------------------------------------------------
\renewcommand{\thesection}{\arabic{section}.}
% Kropki po numerach rozdziału (polski zwyczaj topograficzny)
\renewcommand{\thesubsection}{\thesection\arabic{subsection}}
% Brak kropki po numerach podrozdziału



% ------------------------------
% Ustawienia różnych parametrów tekstu
% ------------------------------
\renewcommand{\baselinestretch}{1.1}

% Ustawienie szerokości odstępów między wierszami w tabelach.
\renewcommand{\arraystretch}{1.4}



% ------------------------------
% Pakiet „hyperref”
% Polecano by umieszczać go na końcu preambuły.
% ------------------------------
\usepackage{hyperref}  % Pozwala tworzyć hiperlinki i zamienia odwołania
% do bibliografii na hiperlinki.










% ---------------------------------------------------------------------
% Tytuł i autor tekstu
\title{Algebra \\
  {\Large Błędy i~uwagi}}

\author{Kamil Ziemian}


% \date{}
% ---------------------------------------------------------------------










% ####################################################################
% Początek dokumentu
\begin{document}
% ####################################################################





% ######################################
\maketitle % Tytuł całego tekstu
% ######################################





% ############################
\Work{ % Autor i tytuł dzieła
  Andrzej Białynicki-Birula \\
  \textit{Zarys algebry}, \cite{BialynickiBirulaZarysAlgebry1987}}

\vspace{0em}


% ##################
\CenterBoldFont{Uwagi}

\vspace{0em}


\noindent
\Str{11} Podany tu system aksjomatów jest arcynieporęczny, przez co
w~dalszych partiach książki nie~jest stosowany konsekwentnie. Problem
polega na tym, że~ponieważ wedle tej definicji 0 nie jest liczbą
naturalną, mając wie dwie liczby naturalne $p$, $q$, $p \geq q$ nie
wiemy, czy ich różnica $p - q$ jest liczbą naturalną. Wszak może
zachodzić $p = q$, wtedy $p - q = 0$ i~odejmowanie wyprowadza nas poza
zakres liczb naturalnych.

W~szczególności wykonane w~dowodzie twierdzenia 3.3 dzielenie
$q_{ 1 } | p_{ 1 } - q_{ 1 }$ okazuje~się być operacją wykraczającą
poza teorię, bo $p_{ 1 } = q_{ 1 }$. Fakt, że~jest to dowód nie~wprost
(lub przez kontrapozycję), nie wydaje~się zmieniać istoty rzeczy.
Również w~twierdzeniu 3.5 jest mowa o~dodawaniu do liczby naturalnej
liczby całkowitej $r$, która może być równa 0, co uwidacznia potrzebę
rozszerzenia teorii, by nie stanowiła dla nas niepotrzebnego
obciążenia.

Wiadomo, że można to zrobić w sposób prosty (Grzegorczyk, zarys
arytmetyki teoretycznej), przyjmując jako jedne z elementów
podstawowych teorii liczb naturalnych, nie liczbę $1$, lecz liczbę
$0$. Liczbę jeden definiujemy wtedy jako $1 := S( 0 )$ i~cała
przedstawiona tu teoria pozostaje w~mocy.





% ##################
\newpage

\CenterBoldFont{Błędy}


\begin{center}

  \begin{tabular}{|c|c|c|c|c|}
    \hline
    Strona & \multicolumn{2}{c|}{Wiersz} & Jest
                              & Powinno być \\ \cline{2-3}
    & Od góry & Od dołu & & \\
    \hline
    11  & & 11 & \textbf{całkowitych} & \textbf{naturalnych} \\
    11  & &  1 & \textbf{całkowitych} & \textbf{naturalnych} \\
    14  & 14 & & $A^{ 2 } \quad A$ & $A^{ 2 } \to A$ \\
    15  &  3 & & $( a^{ 1 } )^{ -1 }$ & $( a^{ -1 } )^{ -1 }$ \\
    16  & 13 & & \textit{lącznym} & \textit{łącznym} \\
    24  &  9 & & $if\big( \varphi( a_{ 1 }, \ldots, a_{ n } ) \big)$
           & $i \circ f\big( \varphi( a_{ 1 }, \ldots, a_{ n } ) \big)$ \\
    24  & 10 & & $if\big( \varphi( a_{ 1 }, \ldots, a_{ n } ) \big)$
           & $i \circ f\big( \varphi( a_{ 1 }, \ldots, a_{ n } ) \big)$ \\
    25  & &  1 & $s( a )$ & $[ s( a ) ]$ \\
    31  & &  4 & $\pi_{ t }( i' \circ i )$
           & $\pi_{ t } \circ ( i' \circ i )$ \\
    33  & 21 & & \S{} 2 & \S{} 4 \\
    % & & & & \\
    % & & & & \\
    % & & & & \\
    291  & 11 & & 1974. & 1974). \\
    % & & & & \\
    % & & & & \\
    % & & & & \\
    \hline
  \end{tabular}

\end{center}

\vspace{\VerSpaceSix}


Str. 37. \ldots$j k = -k j$\ldots \\
Str. 80. \ldots więc $g h_{ 1 } g^{ -1 } \in G_{ y }$\ldots \\
Str. 80. Istotnie, $h \in \varphi_{ x }^{ -1 }( g x )$\ldots \\
Str. 81. \ldots orbit $H x_{ 1 } \cup \ldots \cup H x_{ l } \, .$



% ############################










% ############################
\newpage

\Work{ % Autor i tytuł dzieła
  M. Hamermesh \\
  \textit{Teoria grup w zastosowaniu do zagadnień fizycznych}, \cite{}}



Str. 35. Może właściwszym byłoby zdefiniowanie podgrupy
  właściwej jako podgrupy różnej od $G$ i $e$?

Str. 14.
$$ \mathbf{ x } = \mathbf{ a }^{ -1 } \mathbf{ x }' \, ,$$


\vspace{\spaceTwo}


% ############################










% ######################################
\newpage

\section{Algebra liniowa}

\vspace{\spaceTwo}
% ######################################



% ############################
\Work{ % Autor i tytuł dzieła
  Henryk Arodź, Krzysztof Rościszewski \\
  \textit{Algebra i~geometria analityczna w~zadaniach}, \cite{}}

\vspace{0em}


% ##################
\CenterBoldFont{Uwagi do~konkretnych stron}

\vspace{0em}


Str. 64. Brakuje macierzy $\mathbf{I} \, .$





% ##################
\newpage

\CenterBoldFont{Błędy}


\begin{center}

  \begin{tabular}{|c|c|c|c|c|}
    \hline
    Strona & \multicolumn{2}{c|}{Wiersz} & Jest
                              & Powinno być \\ \cline{2-3}
    & Od góry & Od dołu & & \\
    \hline
    & & & & \\
    & & & & \\
    & & & & \\
    & & & & \\
    \hline
  \end{tabular}

\end{center}

\vspace{\VerSpaceSix}


Str. 15.
$$\sqrt[ n ]{ z } = \sqrt[ n ]{ | z | } \exp( i \frac{ \varphi + 2 k
  \pi }{ n } ) \, .$$ \\
Str. 15. \ldots $Re \; z = a$\ldots \\
Str. 74. (14) $6 \boldsymbol{ \gamma } \, ,$ \\



% ############################










% ############################
\Work{ % Autor i tytuł dzieła
  Jacek Gancarzewicz \\
  \textit{Algebra liniowa i~jej zastosowania},
  \cite{GancarzewiczAlgebraLiniowa2004}}

\vspace{0em}


% ##################
\CenterBoldFont{Uwagi}

\vspace{0em}


\noindent
\Str{11}

\vspace{\spaceFour}





\noindent
\Str{34} Alternatywny, prosty dowód podpunkt (2) twierdzenia
3.7, polega na przyjęciu, że~$x \neq 0$ i~pomnożeniu równość $xy = 0$
z~lewej strony przez $x^{ -1 }$.

\vspace{\spaceFour}





\noindent
W pierwszym rozdziale nie określono znaku permutacji identycznościowej.

\vspace{\spaceFour}





\noindent
Nie określono operacji odejmowania wektorów. Z definicji jest
to:$$x - y := x + ( -y ) \, .$$

\vspace{\spaceFour}





\noindent
Problem sumowania po pustym zbiorze wskaźników nie został omówiony.
Powinno oczywiście być $\sum_{ \substack{ \iota \in \emptyset } } v_{ \iota } = 0 \, .$

\vspace{\spaceFour}





\noindent
Str. 40. Jest pewna luka w dowodzie drugiej wersji zasadniczego
twierdzenia algebry. Procedura dzielenia pokazuje bowiem, że po n
krokach wielomian jest stopnia 0, czyli musi być stałą, nie pokazano
jednak, że stała ta wynosi $a_{ n }$. Aby to pokazać należy
stwierdzić, co można pokazać indukcyjnie, że w iloczynach typu
$( z - z_{ 1 } )^{ \alpha_{ 1 } } \ldots( z - z_{ j } )^{ \alpha_{ j } }$
wyraz przy najniższej potędze wynosi 1 i następnie skorzystać z
twierdzenia, że dwa wielomiany w ciele liczb zespolonych są równe
wtedy i tylko wtedy gdy wszystkie ich współczynniki są równe.

\vspace{\spaceFour}





\noindent
Str. 60. W ostatniej permutacji trzeba jedną 7 zastąpić 4.

\vspace{\spaceFour}





\noindent
Str. 76. Błąd w numeracji podpunktów twierdzenia.

\vspace{\spaceFour}





\noindent
Str. 341. W twierdzeniu 45.1 powinno być $n \geq 1 \, .$ Dla n = 0
istnieje 0 punktów afonicznie niezależnych.





% ##################
\newpage

\CenterBoldFont{Błędy}


\begin{center}

  \begin{tabular}{|c|c|c|c|c|}
    \hline
    Strona & \multicolumn{2}{c|}{Wiersz} & Jest
                              & Powinno być \\ \cline{2-3}
    & Od góry & Od dołu & & \\
    \hline
    28  & & 13 & $x \neq 0$ & $x \neq y$ \\
    34  & & 12 & $x ( y y^{ -1 } x^{ -1 }$ & $x ( y y^{ -1 } ) x^{ -1 }$ \\
    53  &  6 & & $\Real( ( \eta \eta' ) \eta'' ) )$
           & $\Real\big( ( \eta \eta' ) \eta'' \big)$ \\
    59  & 16 & & $x_{ 3 }$ & $x_{ 1 }$ \\
    % & & & & \\
    % & & & & \\
    % & & & & \\
    % & & & & \\
    % & & & & \\
    % & & & & \\
    % & & & & \\
    % & & & & \\
    % & & & & \\
    % & & & & \\
    % & & & & \\
    % & & & & \\
    % & & & & \\
    \hline
  \end{tabular}

\end{center}

\vspace{\spaceTwo}


\noindent
\StrWierszG{420}{15} \\
\Jest
\begin{equation*}
  k_{ f } =
  \begin{cases}
    \dim Z_{ f }, & gdy\:\, Z_{ f } \neq \oldemptyset, \\
    \quad \quad \quad \quad \quad -1, & gdy\:\, Z_{ f } = \oldemptyset,
  \end{cases}
\end{equation*}
\Powin
\begin{equation*}
  k_{ f }
  =
  \begin{cases}
    \dim Z_{ f }, & \textrm{gdy}\:\, Z_{ f } \neq \oldemptyset, \\
    -1, & \textrm{gdy}\:\, Z_{ f } = \oldemptyset,
  \end{cases}
\end{equation*}
\Powin  Wykonując jawnie różniczkowanie w~(10) \\
\StrWierszD{71}{16} \\
\Jest
$R\left( x, \sqrt{ a x^{ 4 } + b x^{ 3 } + c x^{ 2 } + d x + e }
\right.$ \\
\Powin
$R\left( x, \sqrt{ a x^{ 4 } + b x^{ 3 } + c x^{ 2 } + d x + e }
\right)$ \\

\vspace{\spaceTwo}


Str. 41. \ldots to funkcja kwadratowa $z^2+az+b$\ldots \\
Str. 44.
  $$\overline{ \xi } = \overline{ z_{ 1 } } - z_{ 2 } j \, .$$ \\
Str. 59. $( x_{ 1 }, \ldots, x_{ k } ) \leq ( y_{ 1 }, \ldots, y_{ k } ) \, .$ \\
Str. 62. \ldots i skalara $a \in F$\ldots \\

Str. 64.
  $$x = 1 x = ( a^{ -1 } a ) x = a^{ -1 }( a x ) = a^{ -1 } 0 = 0 \,
  .$$ \\
Str. 66. Wymieniając w twierdzeniu 8.8 równoważne warunki,
otrzymujemy\ldots \\
Str. 75. \ldots otrzymujemy
$f( a f^{ -1 }( y ) + b f^{ -1 } ( y' ) ) = a y + b y'$\ldots \\
Str. 76. Na podstawie twierdzenia 1.2 punkt 1. \\
Str. 86. \ldots z zasady kontrapozycji. \\
Str. 110. $( a_{ 0 }, a_{ 1 }, a_{ 2 }, \ldots )$. \\
Str. 121. \ldots otrzymujemy
$\alpha = \sum_{ i = 1 }^{ n } \lambda_{ i } e^{ * }_{ i } .$ \\
Str. 134. \ldots nie występują one w tezie twierdzenia. \\
Str. 143. \ldots bo jeżeli dla $\alpha \in U^{ * }$\ldots \\
Str. 143.
  $$f( e_{ i } ) = \sum_{ j = 1 }^{ m } a_{ j i } \overline{ { e } }_{
    j } \, , \qquad f'( e'_{ i } ) = \sum_{ j = 1 }^{ m } a_{ j i } \overline{
    { e }' }_{ j } \, ,$$ \\
Str. 147.
$$f : M( m, 1; F ) = F^{ m } \ni x \longrightarrow A x \in M( n, 1 ;
F ) = F^{ n } \, ,$$ \\
Str. 195.
\ldots$\mathcal{ J }_{ 1 } : F \otimes X \rightarrow X$ \textit{oraz}
$\mathcal{ J }_{ 2 } : X \otimes F \rightarrow X$\ldots \\
Str. 223. \ldots\textit{gdzie $K \in M( p; F )$ oraz}\ldots \\
Str. 238. Niech $x_{ 1 }, \ldots, x_{ n } \in F ,$ będzie\ldots \\
Str. 238. \ldots przez punkty $x_{ 1 }$,\ldots,$x_{ n }$ nazywamy
wyznacznik
$$\mathcal{ V }_{ x_{ 1 }, \ldots, x_{ n } } = \ldots$$ \\
Str. 238. Jeżeli wśród punktów $x_{ 1 },\ldots, x_{ n }$\ldots \\
Str. 238. \ldots że $\mathcal{ V }_{ x_{ 1 }, \ldots, x_{ n } } \neq 0 .$ \\
Str. 238.
$$\mathcal{ V }_{ x_{ 1 }, \ldots, x_{ n } } = \prod_{ 1 \leq i < j
  \leq n - 1 }( x_{ j } - x_{ i } ) \neq 0 \, .$$ \\
Str. 238. \ldots pierwiastków $x_{ 1 }, \ldots, x_{ n - 1 }$\ldots \\
Str. 253. \ldots \textit{nieujemnych} $p, q, p', q'$\ldots \\
Str. 269. \ldots liniowe
$\rho_{ V }^{ p } : \Lambda^{ p } V^{ * } \rightarrow L^{ p }_{ a } ( V )$\ldots \\
Str. 269. \begin{displaymath}
  \begin{split}
    (\rho_{ V }^{ p } \circ \Lambda^{ p }( f^{ * } ) ) ( \beta_{ 1 } \wedge \ldots \wedge \beta_{ p } ) &= \rho_{ V }^{ p } ( f^{ * }( \beta_{ 1 } ) \wedge \ldots \wedge f^{ * } ( \beta_{ p } ) ) \\
                                                                    &= \rho_{ V }^{ p } (\beta_{ 1 } \circ f \wedge \ldots \wedge \beta_{ p } \circ f )\\
                                                                    &= u_{ V }^{ p } ( \beta_{ 1 } \circ f, \ldots, \beta_{ p } \circ f ) \, .
  \end{split}
\end{displaymath} \\
Str. 272. \ldots że lewa strona równości\ldots \\
Str. 272. Obliczmy prawą stronę równości\ldots \\
Str. 272. $$P = \frac{ 1 }{ ( p + q )! p! q! } \ldots$$ \\
Str. 272. \begin{displaymath}
  \begin{split}
    P \quad =& \quad \frac{ 1 }{ ( p + q )! p! q! } \sum_{ \sigma \in S_{ p + q } } \sum_{ \rho \in  S_{ p } } \sum_{ \tau \in S_{ q } } \textrm{sgn} \, \sigma \, \textrm{sgn} \, \rho \, \textrm{sgn} \, \tau \\
         & \quad \qquad \alpha_{ 1 } ( v_{ \sigma( 1 ) } ) \ldots \alpha_{ p } ( v_{ \sigma ( p ) } ) \alpha_{ p + 1 } ( v_{ \sigma( p + 1 ) } ) \ldots \alpha_{ p + q } ( v_{ \sigma ( p + q ) } ) \\
    = & \quad \frac{ p! q! }{ ( p + q )! p! q! } \sum_{ \sigma \in S_{ p + q } } \textrm{sgn} \, \sigma \\
         & \quad \qquad \alpha_{ 1 } ( v_{ \sigma( 1 ) } ) \ldots \alpha_{ p } ( v_{ \sigma ( p ) } ) \alpha_{ p
           + 1 } ( v_{ \sigma( p + 1 ) } ) \ldots \alpha_{ p + q } ( v_{ \sigma ( p + q ) } )
           \, .
  \end{split}
\end{displaymath} \\
Str. 276. \ldots gdyż macierz $A$\ldots \\
Str. 304. \ldots indukowane
$T^{ p } ( f ) : T^{ p } ( V ) \rightarrow T^{ p } (W)$\ldots \\
Str. 322. \ldots element $\omega \in L( X_{ 1 } ,\ldots, X_{ k }; Y )$ \ldots \\
Str. 324. \ldots w przypadku $X_{ 1 } = \ldots = X_{ k } = X$\ldots \\
Str. 328. Odwzorowanie $A_{ X } : \bigotimes^{ k } X \rightarrow \Lambda^{ k } X$\ldots \\
Str. 328. \ldots dla rodziny $\{ A_{ X } \} \, .$ \\
Str. 329. \ldots $k$-liniowe
$u^{ a }_{ X } : X \times \ldots \times X \rightarrow \bigotimes^{ k } X$ i
$u^{ s }_{ X } : X \times \ldots \times X \rightarrow \bigotimes^{ k } X$\ldots \\
Str. 354. Zadanie \romannumeral6.7a zostało rozwiązane w
twierdzeniu 44.11. \\
Str. 363. \ldots należą do $M( 0; F )$\ldots \\
Str. 364. \ldots oraz $v = p( f |_{ U } ) ( v ) \in U_{ 1 }$\ldots \\


% ############################
















% ############################
\newpage

\Work{ % Autor i tytuł dzieła
  Andrzej Herdegen \\
  \textit{Algebra liniowa i~geometria},
  \cite{HerdegenAlgebraLiniowaIGeometria2010}}

\vspace{0em}


% ##################
\CenterBoldFont{Uwagi}

\vspace{0em}




% ##################
\CenterBoldFont{Błędy}


\begin{center}

  \begin{tabular}{|c|c|c|c|c|}
    \hline
    Strona & \multicolumn{2}{c|}{Wiersz} & Jest
                              & Powinno być \\ \cline{2-3}
    & Od góry & Od dołu & & \\
    \hline
    8   & 6 & & $p$ ; & $p$; \\
    % & & & & \\
    % & & & & \\
    % & & & & \\
    % & & & & \\
    \hline
  \end{tabular}

\end{center}

\vspace{\VerSpaceSix}




% ############################










% ############################
\newpage

\Work{ % Autor i tytuł dzieła
  Andrzej Staruszkiewicz \\
  \textit{Algebra i~geometria. Wykłady dla fizyków, tom~I}, \cite{ASAG}}

\vspace{0em}


% ##################
\CenterBoldFont{Uwagi}

\vspace{0em}


Str. 14. \ldots $\det A \neq 0$, to \\
$\textrm{rz} ( AB ) = \textrm{rz} ( B ) \, .$ \\


% \vspace{\spaceFour}
% ############################










% #####################################################################
% #####################################################################
% Bibliografia

\bibliographystyle{plalpha}

\bibliography{MathematicsBooks}{}





% ############################

% Koniec dokumentu
\end{document}
