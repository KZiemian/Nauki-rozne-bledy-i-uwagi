% ------------------------------------------------------------------------------------------------------------------
% Basic configuration and packages
% ------------------------------------------------------------------------------------------------------------------
% Package for discovering wrong and outdated usage of LaTeX.
% More information to be found in l2tabu English version.
\RequirePackage[l2tabu, orthodox]{nag}
% Class of LaTeX document: {size of paper, size of font}[document class]
\documentclass[a4paper,11pt]{article}



% ------------------------------------------------------
% Packages not tied to particular normal language
% ------------------------------------------------------
% This package should improved spaces in the text
\usepackage{microtype}
% Add few important symbols, like text Celcius degree
\usepackage{textcomp}



% ------------------------------------------------------
% Polonization of LaTeX document
% ------------------------------------------------------
% Basic polonization of the text
\usepackage[MeX]{polski}
% Switching on UTF-8 encoding
\usepackage[utf8]{inputenc}
% Adding font Latin Modern
\usepackage{lmodern}
% Package is need for fonts Latin Modern
\usepackage[T1]{fontenc}



% ------------------------------------------------------
% Setting margins
% ------------------------------------------------------
\usepackage[a4paper, total={14cm, 25cm}]{geometry}



% ------------------------------------------------------
% Setting vertical spaces in the text
% ------------------------------------------------------
% Setting space between lines
\renewcommand{\baselinestretch}{1.1}

% Setting space between lines in tables
\renewcommand{\arraystretch}{1.4}



% ------------------------------------------------------
% Packages for scientific papers
% ------------------------------------------------------
% Switching off \lll symbol, that I guess is representing letter "Ł"
% It collide with `amsmath' package's command with the same name
\let\lll\undefined
% Basic package from American Mathematical Society (AMS)
\usepackage[intlimits]{amsmath}
% Equations are numbered separately in every section
\numberwithin{equation}{section}

% Other very useful packages from AMS
\usepackage{amsfonts}
\usepackage{amssymb}
\usepackage{amscd}
\usepackage{amsthm}

% Package with better looking calligraphy fonts
\usepackage{calrsfs}

% Package with better looking greek letters
% Example of use: pi -> \uppi
\usepackage{upgreek}
% Improving look of lambda letter
\let\oldlambda\Lambda
\renewcommand{\lambda}{\uplambda}




% ------------------------------------------------------
% BibLaTeX
% ------------------------------------------------------
% Package biblatex, with biber as its backend, allow us to handle
% bibliography entries that use Unicode symbols outside ASCII
\usepackage[
language=polish,
backend=biber,
style=alphabetic,
url=false,
eprint=true,
]{biblatex}

\addbibresource{Logika-i-teoria-mnogości-Bibliography.bib}





% ------------------------------------------------------
% Defining new environments (?)
% ------------------------------------------------------
% Defining enviroment "Wniosek"
\newtheorem{corollary}{Wniosek}
\newtheorem{definition}{Definicja}
\newtheorem{theorem}{Twierdzenie}





% ------------------------------------------------------
% Private packages
% You need to put them in the same directory as .tex file
% ------------------------------------------------------
% Package containing various command useful for working with a text
\usepackage{latexgeneralcommands}
% Package containing commands and other code useful for working with
% mathematical text
\usepackage{mathcommands}





% ------------------------------------------------------
% Package "hyperref"
% They advised to put it on the end of preambule
% ------------------------------------------------------
% It allows you to use hyperlinks in the text
\usepackage{hyperref}










% ------------------------------------------------------------------------------------------------------------------
% Title and author of the text
\title{Logika i~teoria mnogości \\
  {\Large Błędy i~uwagi}}

\author{Kamil Ziemian}


% \date{}
% ------------------------------------------------------------------------------------------------------------------










% ####################################################################
% Beginning of the document
\begin{document}
% ####################################################################





% ######################################
% Title of the text
\maketitle
% ######################################





% ######################################
\section{Logika}

% \vspace{\spaceTwo}
% ######################################













% ######################################
\newpage

\section{Andrzej Grzegorczyk \\
  \textit{Zarys logiki matematycznej},
  \parencite{Grzegorczyk-Zarys-logiki-matematycznej-Pub-1975}}

\VerSpaceTwo
% ######################################



% ############################
\subsection{Uwagi ogólne}

\label{ssec:Grzegorczyk-Zarys-ETC-Uwagi-ogolne}
% ############################


\noindent
W~tej książce zbiory liczbowe są zwykle zapisywane za pomocą liter
kaligrafowanych, przykładowo zbiór liczb całkowitych jest oznaczany symbolem
$\Ccal$, a~zbiór liczby wymiernych symbolem $\Wcal$. By pozostać jednak
w~zgodzie ze~współcześnie stosowaną notacją, w~tych notatkach będziemy je
zapisywali w~następujący, dobrze znany sposób. Przykładowo, zbiór liczb
całkowitych będziemy oznaczać przez $\Cbb$, zaś zbiór liczb
wymiernych przez $\Qbb$, etc. Czasem, gdy będzie to wskazane zgodnością
z~tekstem książki, możemy zrobić wyjątek od tej zasady.

\VerSpaceSix





\noindent
W~komentarzu do strony 16 wyjaśniliśmy, czemu stosowane w~tej książce
nazewnictwo dla funkcji nie jest najlepsze. By~uniknąć wynikłych stąd
problemów w~tych notatkach będziemy posługiwali~się standardowymi dziś
terminami „injekcja”, „suriekcja” i~„bijekcja”.

\VerSpaceTwo





% ######################################
\section{Uwagi do~konkretnych stron}

\label{sec:Grzegorczyk-Zarys-ETC-Uwagi-do-konkretnych-stron}
% ######################################


\noindent
\Str{16} Wprowadzone tutaj nazwy dla różnych typów funkcji powodują pewne
problemy w~dalszym ciągu książki. Autor Zdecydowała się nadać funkcji
różnowartościowej $f : X \to Y$ nazwę \textit{funkcja wzajemnie
  jednoznaczną}\footnote{Nie cytuję dosłownie tekstu książki,
  bo wymaga on poprawienia i~ujednoznacznienia.}, a funkcji która
jest różnowartościowa i~spełnia dodatkowo warunek $f( X ) = Y$ nazwę
\textit{funkcji wzajemnie jednoznacznej odwzorowującej zbiór $X$ na cały
  zbiór $Y$}, już na stronie następnej stronie (str. 17) stosuje to
nazewnictwo niekonsekwentnie. W~paragrafie trzecim na tej stronie pisze,
że~„relacja $R$ wyznacza odwzorowanie wzajemnie jednoznaczne zbioru $X$
w~zbiór $Y$”, gdzie chodzi mu o~odwzorowanie $X$ na cały zbiór $Y$, czyli
funkcję $f$ taką, że
$f( X ) = Y$. Aby z tego wybrnąć przyjmujemy umowę, że~sformułowania
„na zbiór $Y$” i~„na cały zbiór $Y$” oba oznaczają to sama, czyli że
dane odwzorowanie pokrywa swoimi wartościami cały zbiór $Y$
(równoważnie $f( X ) = Y$). Jeśli $f( X ) \neq Y$ będziemy mówić, że
\textit{funkcja $f$ odwzorowuje zbiór $X$ w~zbiór $Y$}.

Dalej na stronie 17 pisze, że jeśli zawęzimy potęgowanie oraz
pierwiastkowanie do liczb dodatnich, to wówczas obie funkcje są
„wzajemnie jednoznaczne” i~„jedna jest odwrotnością drugiej”. W~tym
miejscu prawie na pewno przez stwierdzenie, że tak określone
potęgowanie i~pierwiastkowanie są funkcjami „wzajemnie jednoznacznymi”
należy rozumieć nie tylko to, że~są one różnowartościowe, ale też
że~odwzorowują zbiór liczby dodatnich na cały zbiór liczb dodatnich.

Dodatkowo na stronie 16 pojawia się pojęcie „relacji jednoznacznej”,
przez co autor rozumie chyba relację która wyznacza pewną funkcję,
czyli spełnia warunki (2) i~(3) ze str.~10. Na stronie 17 jest zaś
mowa o „relacji wzajemnie jednoznacznej” co z kolei oznacza relację
$R$ taką, że zarówno relacja $R$ jak i~relacja do niej odwrotna
wyznaczają funkcje. Jak widać pojęcie jednoznaczności dla funkcji
i~dla relacji oznacza zupełnie coś innego.

Proponuję następujące wyjście z tej sytuacji. Funkcję $f: X \to Y$
będziemy nazywać \textbf{jednoznaczną} jeśli jest różnowartościowa.
Funkcję która jest różnowartościowa i~dla której zachodzi
$f( X ) = Y$, będziemy nazywali \textbf{wzajemnie jednoznaczną}. Pojęć
„relacja jednoznaczna” i~„relacja wzajemnie jednoznaczna” będziemy
unikać. Zamiast tego będziemy mówić o~\textbf{relacji wyznaczającej funkcję}.

Dalsza lektura pokaże, czy książka pozwala na stosowanie tej
terminologi, czy też okaże~się to zbyt skomplikowane w~praktyce. Do
tego jednak momentu będziemy poprawiać jej fragmenty w~zgodzie z~tym
co zostało powyżej napisane.

% \vspace{\spaceFour}





% ##################
\newpage

\CenterBoldFont{Błędy}


\begin{center}

  \begin{tabular}{|c|c|c|c|c|}
    \hline
    Strona & \multicolumn{2}{c|}{Wiersz} & Jest
    & Powinno być \\ \cline{2-3}
           & Od góry & Od dołu & & \\
    \hline
    % W książce kropka po $Y$ jest bardzo duża w książce.
    11  & & \hphantom{0}3 & $Y$. & $Y$. \\
    11  & & \hphantom{0}2 & $\langle \Cbb, +, \cdot \rangle$
           & $\langle \Cbb, +, \cdot, 0, 1 \rangle$ \\
    12  & \hphantom{0}3 & & \textit{ciałem}
           & \textit{ciałem uporządkowanym} \\
    12  & \hphantom{0}6 & & $\langle \{ 0, 1, 2 \}, \oplus \rangle$
           & $\langle \{ 0, 1, 2 \}, \oplus, 0 \rangle$ \\
    12  & & 17 & $\langle \Cbb, +, \cdot \rangle$
           & $\langle \Cbb, +, \cdot, 0, 1 \rangle$ \\
    12  & & 16 & $\langle \Rbb, +, \cdot, < \rangle$
           & $\langle \Rbb, +, \cdot, 0, 1, < \rangle$ \\
    17  & & 14 & $\Rbb, +, \cdot, <$ & $\Rbb, +, \cdot, 0, 1, <$ \\
    16  & 12 & & str. 12 & str. 10 \\
    18  & 20 & & (dla & \big(dla \\
    20  & 21 & & \big($f \in Y$)\big) & \big($f \in Y$\big)\big) \\
    20  & &  6 & liczby{ } $\neq 0$ & liczby $\neq 0$ \\
    22  & & 11 & zwrotność & samozwrotność \\
    22  & &  3 & wzajemnej jednoznaczności & jednoznaczności \\
    22  & &  2 & przeciwobraz & przeciwobraz obrazu \\
    25  &  4 & & (21); & (21)); \\
    29  & & 13 & $\forall x W$ & $\forall \, x \; \exists \, W$ \\
    34  &  1 & & (23) & (23') \\
    34  & 20 & & $\forall u$ & $x \in \bigcap X \equiv \forall \, u$ \\
    % & & & & \\
    % & & & & \\
    % & & & & \\
    % & & & & \\
    % & & & & \\
    % & & & & \\
    \hline
  \end{tabular}

\end{center}

\VerSpaceSix


\noindent
\StrWierszDol{7}{4} \\
\Jest przedmioty~~$a_{ 1 }, a_{ 2 }, \ldots, a_{ k }$ \\
\PowinnoByc przedmioty $a_{ 1 }, a_{ 2 }, \ldots, a_{ k }$ \\



% ######################################










% ######################################
\newpage

\subsection{Willard Van Orman Quine \textit{Logika matematyczna},
  \cite{Quine-Logika-matematyczna-Pub-1974}}


% ##################
\CenterBoldFont{Błędy}


\begin{center}

  \begin{tabular}{|c|c|c|c|c|}
    \hline
    Strona & \multicolumn{2}{c|}{Wiersz} & Jest
                              & Powinno być \\ \cline{2-3}
    & Od góry & Od dołu & & \\
    \hline
    % & & & & \\
    19  & &  2 & ęzyka & języka \\
    21  & & 10 & o fałszywych poprzednikach i & o \\
    22  & 18 & & „jeżeli --- to --- & „jeżeli --- to ---” \\
    27  &  5 & & „jeżeli --- to” & „jeżeli --- to ---” \\
    % & & & & \\
    % & & & & \\
    \hline
  \end{tabular}

\end{center}

\VerSpaceSix


% ######################################










% ######################################
\newpage

\section{Graham Priest \textit{Logika}}

% \vspace{\spaceTwo}
% ######################################


% ##################
\CenterBoldFont{Błędy}


\begin{center}

  \begin{tabular}{|c|c|c|c|c|}
    \hline
    Strona & \multicolumn{2}{c|}{Wiersz} & Jest
    & Powinno być \\ \cline{2-3}
           & Od góry & Od dołu & & \\
    \hline
    144 & \hphantom{0}8 & & „jest”~~predykacji & „jest”~predykacji \\
    144 & 18 & & \textit{każdy} & każdy \\
    144 & 18 & & \textit{żaden} & żaden \\
    % & & & & \\
    % & & & & \\
    % & & & & \\
    \hline
  \end{tabular}

\end{center}

\VerSpaceSix


% ######################################
















% ######################################
\newpage
\section{Teoria mnogości}

% \vspace{\spaceTwo}
% ######################################



% ############################










% ####################################################################
% ####################################################################
% Bibliography

\printbibliography





% ############################
% End of the document

\end{document}
