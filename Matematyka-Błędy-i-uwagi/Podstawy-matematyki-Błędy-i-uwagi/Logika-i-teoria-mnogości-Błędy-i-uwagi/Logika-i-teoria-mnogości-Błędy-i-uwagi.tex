% ---------------------------------------------------------------------
% Basic configuration and packages
% ---------------------------------------------------------------------
% Package for discovering wrong and outdated usage of LaTeX.
% More information to be found in l2tabu English version.
\RequirePackage[l2tabu, orthodox]{nag}
% Class of LaTeX document: {size of paper, size of font}[document class]
\documentclass[a4paper,11pt]{article}



% ---------------------------------------
% Packages not tied to particular normal language
% ---------------------------------------
% This package should improved spaces in the text.
\usepackage{microtype}
% Add few important symbols, like text Celcius degree
\usepackage{textcomp}



% ---------------------------------------
% Polonization of LaTeX document
% ---------------------------------------
% Basic polonization of the text
\usepackage[MeX]{polski}
% Switching on UTF-8 encoding
\usepackage[utf8]{inputenc}
% Adding font Latin Modern
\usepackage{lmodern}
% Package is need for fonts Latin Modern
\usepackage[T1]{fontenc}



% ---------------------------------------
% Setting margins
% ---------------------------------------
\usepackage[a4paper, total={14cm, 25cm}]{geometry}



% ---------------------------------------
% Setting vertical spaces in the text
% ---------------------------------------
% Setting space between lines
\renewcommand{\baselinestretch}{1.1}

% Setting space between lines in tables
\renewcommand{\arraystretch}{1.4}



% ---------------------------------------
% Packages for scientific papers
% ---------------------------------------
% Switching off \lll symbol, that I guess is representing letter ``Ł''.
% It collide with `amsmath' package's command with the same name
\let\lll\undefined
% Basic package from American Mathematical Society (AMS)
\usepackage[intlimits]{amsmath}
% Equations are numbered separately in every section.
\numberwithin{equation}{section}

% Other very useful packages from AMS
\usepackage{amsfonts}
\usepackage{amssymb}
\usepackage{amscd}
\usepackage{amsthm}

% Package with better looking calligraphy fonts
\usepackage{calrsfs}

% Package with better looking greek letters
% Example of use: pi -> \uppi
\usepackage{upgreek}
% Improving look of lambda letter
\let\oldlambda\Lambda
\renewcommand{\lambda}{\uplambda}





% ---------------------------------------
% Defining new environments (?)
% ---------------------------------------
% Defining enviroment ``Wniosek''
\newtheorem{corollary}{Wniosek}
\newtheorem{definition}{Definicja}
\newtheorem{theorem}{Twierdzenie}





% ------------------------------
% Private packages
% You need to put them in the same directory as .tex file
% ------------------------------
% Contains various command useful for working with a text
\usepackage{latexgeneralcommands}
% Contains definitions useful for working with mathematical text
\usepackage{mathcommands}





% ------------------------------
% Package ``hyperref''
% They advised to put it on the end of preambule
% ------------------------------
% It allows you to use hyperlinks in the text
\usepackage{hyperref}










% ---------------------------------------------------------------------
% Tytuł i autor tekstu
\title{Logika i~teoria mnogości \\
  {\Large Błędy i~uwagi}}

\author{Kamil Ziemian}


% \date{}
% ---------------------------------------------------------------------










% ####################################################################
% Początek dokumentu
\begin{document}
% ####################################################################





% ######################################
\maketitle  % Tytuł całego tekstu
% ######################################





% ######################################
\section{Logika}

% \vspace{\spaceTwo}
% ######################################



% ############################
% \Work{ % Autor i tytuł dzieła
  Józef W.~Bremer \\
  \textit{Wprowadzenie do~logiki},
  \cite{Bremer-Wprowadzenie-do-logiki-Wyd-2004}


% ##################
\CenterBoldFont{Uwagi}


\noindent
\StrWierszGora{132}{1} Zdanie „$p \vee q ( ( p \land q ) \to ( p \land q ) )$” nie ma
żadnego sensu logicznego, musiał zostać zgubiony spójnik logiczny
po~pierwszym~$q$. Niestety nie wiem który z nich należy tam umieścić.





% ##################
\CenterBoldFont{Błędy}


\begin{center}

  \begin{tabular}{|c|c|c|c|c|}
    \hline
    & \multicolumn{2}{c|}{} & & \\
    Strona & \multicolumn{2}{c|}{Wiersz} & Jest
                              & Powinno być \\ \cline{2-3}
    & Od góry & Od dołu & & \\
    \hline
    \hphantom{00}4 & \hphantom{0}8 & & LATEX & \LaTeX \\
    \hphantom{0}13 & & \hphantom{0}8 & \textit{formalnej}
           & \textit{formalnej}; \\
    20  & & 15 & Ockhama”$^{ 7 }$. & Ockhama”$^{ 7 }$, \\
    21  & 14 & & Współczesnych & współczesnych \\
    26  & &  2 & \textit{Lwowsko-\! Warszawska}
           & \textit{Lwowsko-Warszawska} \\
    45  & & 10 & konkretna & konkretną \\
    86  & 10 & & przypadku.. & przypadku. \\
    88  & 11 & & średniego & pośredniego \\  % ???
    98  & &  9 & $P \underline{ e } S$ & $S \underline{ e } P$ \\
    114 &  9 & & zwane\textit{prawo} & zwane \textit{prawo} \\
    117 &  6 & & jedzie & jadą \\
    119 & & 14 & współczesnej~. & współczesnej. \\
    122 & &  4 & wniosek $\equiv P$) & wniosek $\equiv P$)” \\
    125 & & 14 & „nie-analityczne & „nie-analityczne” \\
    131 &  6 & & „$\neg p \vee \neg q''$ & „$\neg p \vee \neg q$” \\
    131 &  6 & & $\equiv$,\hspace{2pt},$\neg ( p \land q )$”
           & $\equiv$ „$\neg ( p \land q )$” \\
    132 &  8 & & $\neg( \neg p\;\;\; \land\; \neg q ) $
           & $\neg( \neg p \land \neg q ) $ \\
    132 & 17 & & $p\quad \to \quad q$ & $p \to q$ \\
    132 & & 16 & $p\quad \to \quad q$ & $p \to q$ \\
    % & & & & \\
    % & & & & \\
    % & & & & \\
    % & & & & \\
    % & & & & \\
    \hline
  \end{tabular}

\end{center}

\VerSpaceSix


% ############################










% ######################################
\newpage
\section{Andrzej Grzegorczyk \\
  \textit{Zarys logiki matematycznej},
  \cite{Grzegorczyk-Zarys-Logiki-Matematycznej-Wyd-1975}}
% ######################################

\VerSpaceTwo


% ##################
\CenterBoldFont{Uwagi do~konkretnych stron}


\noindent
\Str{11}

\VerSpaceFour





\noindent
\Str{16} Na tej stronie Grzegorczyk swoim wyborem nazw dla
własności funkcji stworzył problem, który czyni język jego książki
niejednoznacznym. Zdecydowała się nadać funkcji $f : X \to Y$
różnowartościowej nazwę \textit{funkcji wzajemnie
  jednoznaczną}\footnote{Nie cytuję dosłownie sformułowań z~książki,
  bo wymagają one poprawienia i~ujednoznacznienia.}, a funkcji która
jest różnowartościowa i~spełnia dodatkowo warunek $f( X ) = Y$ nazwę
\textit{funkcji wzajemnie jednoznacznej odwzorowującej zbiór $X$ na cały
  zbiór $Y$}.

Już na stronie następnej stronie (str. 17) stosuje to nazewnictwo
niekonsekwentnie. Pisze, że „relacja $R$ wyznacza odwzorowanie
wzajemnie jednoznaczne zbioru $X$ na zbiór $Y$”, gdzie chodzi mu
o~odwzorowanie $X$ na cały zbiór $Y$, czyli funkcję $f$ taką, że
$f( X ) = Y$. Aby z tego wybrnąć przyjmujemy umowę, że~sformułowania
„na zbiór $Y$” i~„na cały zbiór $Y$” oba oznaczają to sama, czyli że
dane odwzorowanie pokrywa swoimi wartościami cały zbiór $Y$
(równoważnie $f( X ) = Y$). Jeśli $f( X ) \neq Y$ będziemy mówić, że
\textit{funkcja $f$ odwzorowuje zbiór $X$ w~zbiór $Y$}.

Dalej na stronie 17 pisze, że jeśli zawęzimy potęgowanie oraz
pierwiastkowanie do liczb dodatnich, to wówczas obie funkcje są
„wzajemnie jednoznaczne” i~„jedna jest odwrotnością drugiej”. W~tym
miejscu prawie na pewno przez stwierdzenie, że tak określone
potęgowanie i~pierwiastkowanie są funkcjami „wzajemnie jednoznacznymi”
należy rozumieć nie tylko to, że~są one różnowartościowe, ale też
że~odwzorowują zbiór liczby dodatnich na cały zbiór liczb dodatnich.

Dodatkowo na stronie 16 pojawia się pojęcie „relacji jednoznacznej”,
przez co autor rozumie chyba relację która wyznacza pewną funkcję,
czyli spełnia warunki (2) i~(3) ze str.~10. Na stronie 17 jest zaś
mowa o „relacji wzajemnie jednoznacznej” co z kolei oznacza relację
$R$ taką, że zarówno relacja $R$ jak i~relacja do niej odwrotna
wyznaczają funkcje. Jak widać pojęcie jednoznaczności dla funkcji
i~dla relacji oznacza zupełnie coś innego.

Proponuję następujące wyjście z tej sytuacji. Funkcję $f: X \to Y$
będziemy nazywać \textbf{jednoznaczną} jeśli jest różnowartościowa.
Funkcję która jest różnowartościowa i~dla której zachodzi
$f( X ) = Y$, będziemy nazywali \textbf{wzajemnie jednoznaczną}. Pojęć
„relacja jednoznaczna” i~„relacja wzajemnie jednoznaczna” będziemy
unikać. Zamiast tego będziemy mówić o~\textbf{relacji wyznaczającej funkcję}.

Dalsza lektura pokaże, czy książka pozwala na stosowanie tej
terminologi, czy też okaże~się to zbyt skomplikowane w~praktyce. Do
tego jednak momentu będziemy poprawiać jej fragmenty w~zgodzie z~tym
co zostało powyżej napisane.

% \vspace{\spaceFour}





% ##################
\newpage

\CenterBoldFont{Błędy}


\begin{center}

  \begin{tabular}{|c|c|c|c|c|}
    \hline
    Strona & \multicolumn{2}{c|}{Wiersz} & Jest
                              & Powinno być \\ \cline{2-3}
    & Od góry & Od dołu & & \\
    \hline
    % W książce kropka po $Y$ jest bardzo duża w książce.
    11  & &  3 & $Y$. & $Y$. \\
    & & & & \\
    % & & & & \\
    % & & & & \\
    % & & & & \\
    12  &  3 & & \textit{ciałem} & \textit{ciałem uporządkowanym} \\
    16  & 12 & & str. 12 & str. 10 \\
    18  & 20 & & (dla & \big(dla \\
    20  & 21 & & \big($f \in Y$)\big) & \big($f \in Y$\big)\big) \\
    20  & &  6 & liczby{ } $\neq 0$ & liczby $\neq 0$ \\
    22  & & 11 & zwrotność & samozwrotność \\
    22  & &  3 & wzajemnej jednoznaczności & jednoznaczności \\
    22  & &  2 & przeciwobraz & przeciwobraz obrazu \\
    25  &  4 & & (21); & (21)); \\
    29  & & 13 & $\forall x W$ & $\forall \, x \; \exists \, W$ \\
    34  &  1 & & (23) & (23') \\
    34  & 20 & & $\forall u$ & $x \in \bigcap X \equiv \forall \, u$ \\
    % & & & & \\
    % & & & & \\
    % & & & & \\
    % & & & & \\
    % & & & & \\
    % & & & & \\
    \hline
  \end{tabular}

\end{center}

\VerSpaceSix


\noindent
\StrWierszDol{7}{4} \\
\Jest przedmioty~~$a_{ 1 }, a_{ 2 }, \ldots, a_{ k }$ \\
\PowinnoByc przedmioty $a_{ 1 }, a_{ 2 }, \ldots, a_{ k }$ \\



% ############################










% ############################
\newpage

% \Work{ % Autor i tytuł dzieła
  Willard Van Orman Quine \\
  \textit{Logika matematyczna}, \cite{Quine-Logika-matematyczna-Wyd-1974}


% ##################
\CenterBoldFont{Błędy}


\begin{center}

  \begin{tabular}{|c|c|c|c|c|}
    \hline
    Strona & \multicolumn{2}{c|}{Wiersz} & Jest
                              & Powinno być \\ \cline{2-3}
    & Od góry & Od dołu & & \\
    \hline
    % & & & & \\
    19  & &  2 & ęzyka & języka \\
    21  & & 10 & o fałszywych poprzednikach i & o \\
    22  & 18 & & „jeżeli --- to --- & „jeżeli --- to ---” \\
    27  &  5 & & „jeżeli --- to” & „jeżeli --- to ---” \\
    % & & & & \\
    % & & & & \\
    \hline
  \end{tabular}

\end{center}

\VerSpaceSix

% ############################










% ######################################
\newpage
\section{Teoria mnogości}

% \vspace{\spaceTwo}
% ######################################



% ############################










% ####################################################################
% ####################################################################
% Bibliografia

\bibliographystyle{plalpha}

\bibliography{MathematicsBooks}{}





% ############################

% Koniec dokumentu
\end{document}
