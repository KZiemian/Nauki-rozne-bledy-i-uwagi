% ------------------------------------------------------------------------------------------------------------------
% Basic configuration and packages
% ------------------------------------------------------------------------------------------------------------------
% Package for discovering wrong and outdated usage of LaTeX.
% More information to be found in l2tabu English version.
\RequirePackage[l2tabu, orthodox]{nag}
% Class of LaTeX document: {size of paper, size of font}[document class]
\documentclass[a4paper,11pt]{article}



% ------------------------------------------------------
% Packages not tied to particular normal language
% ------------------------------------------------------
% This package should improved spaces in the text
\usepackage{microtype}
% Add few important symbols, like text Celcius degree
\usepackage{textcomp}



% ------------------------------------------------------
% Polonization of LaTeX document
% ------------------------------------------------------
% Basic polonization of the text
\usepackage[MeX]{polski}
% Switching on UTF-8 encoding
\usepackage[utf8]{inputenc}
% Adding font Latin Modern
\usepackage{lmodern}
% Package is need for fonts Latin Modern
\usepackage[T1]{fontenc}



% ------------------------------------------------------
% Setting margins
% ------------------------------------------------------
\usepackage[a4paper, total={14cm, 25cm}]{geometry}



% ------------------------------------------------------
% Setting vertical spaces in the text
% ------------------------------------------------------
% Setting space between lines
\renewcommand{\baselinestretch}{1.1}

% Setting space between lines in tables
\renewcommand{\arraystretch}{1.4}



% ------------------------------------------------------
% Packages for scientific papers
% ------------------------------------------------------
% Switching off \lll symbol, that I guess is representing letter "Ł"
% It collide with `amsmath' package's command with the same name
\let\lll\undefined
% Basic package from American Mathematical Society (AMS)
\usepackage[intlimits]{amsmath}
% Equations are numbered separately in every section
\numberwithin{equation}{section}

% Other very useful packages from AMS
\usepackage{amsfonts}
\usepackage{amssymb}
\usepackage{amscd}
\usepackage{amsthm}

% Package with better looking calligraphy fonts
\usepackage{calrsfs}

% Package with better looking greek letters
% Example of use: pi -> \uppi
\usepackage{upgreek}
% Improving look of lambda letter
\let\oldlambda\Lambda
\renewcommand{\lambda}{\uplambda}




% ------------------------------------------------------
% BibLaTeX
% ------------------------------------------------------
% Package biblatex, with biber as its backend, allow us to handle
% bibliography entries that use Unicode symbols outside ASCII
\usepackage[
language=polish,
backend=biber,
style=alphabetic,
url=false,
eprint=true,
]{biblatex}

\addbibresource{Teoria-układów-dynamicznych-ETC-Bibliography.bib}





% ------------------------------------------------------
% Defining new environments (?)
% ------------------------------------------------------
% Defining enviroment "Wniosek"
\newtheorem{corollary}{Wniosek}
\newtheorem{definition}{Definicja}
\newtheorem{theorem}{Twierdzenie}





% ------------------------------------------------------
% Local packages
% You need to put them in the same directory as .tex file
% ------------------------------------------------------
% Package containing various command useful for working with a text
\usepackage{./Local-packages/general-commands}
% Package containing commands and other code useful for working with
% mathematical text
\usepackage{./Local-packages/math-commands}





% ------------------------------------------------------
% Package "hyperref"
% They advised to put it on the end of preambule
% ------------------------------------------------------
% It allows you to use hyperlinks in the text
\usepackage{hyperref}










% ------------------------------------------------------------------------------------------------------------------
% Title and author of the text
\title{Teoria układów dynamicznych \\
  {\Large Błędy i~uwagi}}

\author{Kamil Ziemian}


% \date{}
% ------------------------------------------------------------------------------------------------------------------










% ####################################################################
\begin{document}
% ####################################################################





% ################################################
\maketitle % Tytuł całego tekstu
% ################################################





% ################################################
\section{Wiesław Szlenk \textit{Wstęp do~teorii gładkich układów dynamicznych},
  \parencite{Szlenk-Wstep-do-teorii-gladkich-ukladow-ETC-Pub-1982}}

\vspace{0em}
% ################################################


% % ##################
% \CenterBoldFont{Uwagi}

% \vspace{0em}


% \noindent


% \VerSpaceFour





% % ##################
% \CenterBoldFont{Uwagi do konkretnych stron}

% \vspace{0em}


\noindent
\StrWierszeDol{7}{5--6} Pojawia~się tu pojęcie „relatywistycznego przyrostu
masy”, jednak z~punktu widzenia dzisiejszej fizyki, lepiej byłoby mówić
o~„relatywistycznym wzorze na~pęd”.

\VerSpaceFour





\noindent
\StrWierszeGora{9}{5--6} Autor rozpoczyna tu praktykę, umieszczania definicji
ważnych pojęć nie w~głównym tekście książki, ale w~przypisach, które
zaczynają~się na stronie~$259$. W~naszej ocenie, jest to zły pomysł
i~definicje takich obiektów matematycznych, jak przywoływana tu ciągła
półgrupa ciągłych odwzorowań, powinny zostać wprowadzone do~głównej części
tej książki. Oszczędziłoby by to czytelnikowi przechodzenia do przypisów,
w~wielu kluczowych punktach wykładu.

\VerSpaceFour









% \noindent
% \Str{23} W~przeprowadzanym tutaj wyprowadzaniu wzoru (2.6) pojawia się
% następujący problem. Czy dopuszczamy by zmienne $a$ i~$b$ w~nim występujące
% mogły przyjmować wartość $0$? Jeśli odpowiedź na to pytanie jest twierdząca,
% to stajemy przed problemem tego, iż~we wzorze (2.6) pojawiają~się
% człony $a^{ 0 }$ i~$b^{ 0 }$. Ponieważ jeśli $b = 0$ to poszukiwana zależność
% redukuje się do wyrażenia $a^{ n }$ (analogicznie dla $a = 0$), proponuję by
% przyjąć, iż rozważamy tylko sytuację, gdy~$a \neq 0$ i~$b \neq 0$.

% \VerSpaceFour





% \Str{25--26} W~opis trójkąta Pascala wkradła się pewna nieścisłość.
% Mianowicie by
% narysować pierwszą linię zawierającą tylko symbol $\binom{ 0 }{ 0 }$ musimy
% wiedzieć
% ile on wynosi, a~nie jest wcale jasne, czy jego wartość wynika
% przeprowadzonych
% do tej pory rozważań. Jeśli nie to należy przyjąć, że~z~definicji
% \begin{equation}
%   \label{Gancarzewicz-Arytmetyka-03}
%   \binom{ 0 }{ 0 } := 1.
% \end{equation}

% Drugi problem dotyczy stwierdzenia, że~gdy znamy $n$-tą linię, to następną
% $( n + 1 )$-szą linię budujemy tak, że~dodajemy po dwa kolejne współczynniki
% z~poprzedniej linii. Bardziej poprawne byłoby stwierdzenie, iż~na początku
% i~końcu $( n + 1 )$-szej linii zapisujemy wartość symboli
% $\binom{ n + 1 }{ 0 } = \binom{ n + 1 }{ n + 1 } = 1$, pozostałe zaś
% współczynniki otrzymujemy dodając dwa kolejne współczynniki z~linii $n$-tej,
% tak jak pokazano to na rysunku. Kwestia tego, gdzie jest początek i~koniec
% linii, gdzie musimy wpisać wartości symboli $\binom{ n + 1 }{ 0 }$
% i~$\binom{ n + 1 }{ n + 1 }$ w~praktyce nie stanowi problemu, dlatego nie
% będziemy~się w~ten problem zagłębiać.

% \VerSpaceFour










% ##################
\newpage

\CenterBoldFont{Błędy}

\VerSpaceFive


\begin{center}

  \begin{tabular}{|c|c|c|c|c|}
    \hline
    Strona & \multicolumn{2}{c|}{Wiersz} & Jest
    & Powinno być \\ \cline{2-3}
    & Od góry & Od dołu & & \\
    \hline
    \hphantom{0}5 & & \hphantom{0}2 & innych & pomocą innych \\
    \hphantom{0}7 & \hphantom{0}7 & & $-m$ & $m$ \\
    \hphantom{0}7 & & \hphantom{0}1 & $-m_{ i }$ & $m_{ i }$ \\
    \hphantom{0}8 & \hphantom{0}5 & & rozwiązalny & rozwiązywalny \\
    % & & & & \\
    % & & & & \\
    % & & & & \\
    \hline
  \end{tabular}

\end{center}

\VerSpaceTwo


% \StrWierszeDol{24}{4--5} \\
% \Jest jeszcze raz zmienić wskaźnik sumacyjny, tym razem $i$ zamieniamy
% na~$j$, \\
% \PowinnoByc zmienić nazwę wskaźnika sumacyjnego z~$i$ na~$j$ \\
% \textbf{Str. 49, trzecia kolumna tabelki, wiersz 7.} \\
% \Jest $= 1632$ \\
% \PowinnoByc $= 1642$ \\
% \StrWierszeDol{54}{14} \\
% \Jest \textit{są jedną z~następującyc par: $00$, $25$. $50$, $75$} \\
% \PowinnoByc \textit{tworzą jedną z~następujących liczb: $00$, $25$, $50$,
%   $75$} \\
% ############################










% ####################################################################
% ####################################################################
% Bibliography

\printbibliography





% ############################
% End of the document

\end{document}
