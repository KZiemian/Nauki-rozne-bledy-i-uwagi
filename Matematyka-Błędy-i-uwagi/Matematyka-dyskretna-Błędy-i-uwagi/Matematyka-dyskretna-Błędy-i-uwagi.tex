% ------------------------------------------------------------------------------------------------------------------
% Basic configuration and packages
% ------------------------------------------------------------------------------------------------------------------
% Package for discovering wrong and outdated usage of LaTeX.
% More information to be found in l2tabu English version.
\RequirePackage[l2tabu, orthodox]{nag}
% Class of LaTeX document: {size of paper, size of font}[document class]
\documentclass[a4paper,11pt]{article}



% ------------------------------------------------------
% Packages not tied to particular normal language
% ------------------------------------------------------
% This package should improved spaces in the text
\usepackage{microtype}
% Add few important symbols, like text Celcius degree
\usepackage{textcomp}



% ------------------------------------------------------
% Polonization of LaTeX document
% ------------------------------------------------------
% Basic polonization of the text
\usepackage[MeX]{polski}
% Switching on UTF-8 encoding
\usepackage[utf8]{inputenc}
% Adding font Latin Modern
\usepackage{lmodern}
% Package is need for fonts Latin Modern
\usepackage[T1]{fontenc}



% ------------------------------------------------------
% Setting margins
% ------------------------------------------------------
\usepackage[a4paper, total={14cm, 25cm}]{geometry}



% ------------------------------------------------------
% Setting vertical spaces in the text
% ------------------------------------------------------
% Setting space between lines
\renewcommand{\baselinestretch}{1.1}

% Setting space between lines in tables
\renewcommand{\arraystretch}{1.4}



% ------------------------------------------------------
% Packages for scientific papers
% ------------------------------------------------------
% Switching off \lll symbol, that I guess is representing letter "Ł"
% It collide with `amsmath' package's command with the same name
\let\lll\undefined
% Basic package from American Mathematical Society (AMS)
\usepackage[intlimits]{amsmath}
% Equations are numbered separately in every section
\numberwithin{equation}{section}

% Other very useful packages from AMS
\usepackage{amsfonts}
\usepackage{amssymb}
\usepackage{amscd}
\usepackage{amsthm}

% Package with better looking calligraphy fonts
\usepackage{calrsfs}

% Package with better looking greek letters
% Example of use: pi -> \uppi
\usepackage{upgreek}
% Improving look of lambda letter
\let\oldlambda\Lambda
\renewcommand{\lambda}{\uplambda}




% ------------------------------------------------------
% BibLaTeX
% ------------------------------------------------------
% Package biblatex, with biber as its backend, allow us to handle
% bibliography entries that use Unicode symbols outside ASCII
\usepackage[
language=polish,
backend=biber,
style=alphabetic,
url=false,
eprint=true,
]{biblatex}

\addbibresource{Systemy-operacyjne-Bibliography.bib}





% ------------------------------------------------------
% Defining new environments (?)
% ------------------------------------------------------
% Defining enviroment "Wniosek"
\newtheorem{corollary}{Wniosek}
\newtheorem{definition}{Definicja}
\newtheorem{theorem}{Twierdzenie}





% ------------------------------------------------------
% Local packages
% You need to put them in the same directory as .tex file
% ------------------------------------------------------
% Package containing various command useful for working with a text
\usepackage{general-commands}
% Package containing commands and other code useful for working with
% mathematical text
\usepackage{math-commands}





% ------------------------------------------------------
% Package "hyperref"
% They advised to put it on the end of preambule
% ------------------------------------------------------
% It allows you to use hyperlinks in the text
\usepackage{hyperref}










% ------------------------------------------------------------------------------------------------------------------
% Title and author of the text
\title{Matematyka dyskretna \\
  {\Large Błędy i~uwagi}}

\author{Kamil Ziemian}


% \date{}
% ------------------------------------------------------------------------------------------------------------------










% ####################################################################
\begin{document}
% ####################################################################





% ######################################
% Title of the text
\maketitle
% ######################################









% ######################################
\section{Oznaczenia i~konwencje}

\label{sec:Oznaczenia-i-konwencje}
% ######################################










% ######################################
\section{Ronald L.~Graham, Donald E.~Knuth, Oren Patashniki
  \textit{Matematyka konkretna},
  \cite{Graham-Knuth-Patashnik-Matematyka-Konkretna-Wyd-2012}}

\label{sec:Graham-Knuth-Patashnik-Matematyka-konkretna}
% ######################################


% ##################
\CenterBoldFont{Uwagi ogólne}

\vspace{0em}


Odstępy w~tekście, zwłaszcza po znaku kropki, wydają mi~się często zbyt duże.
Ten problem pojawia~się jednak tak często, że~nie będę odnotowywał tu
każdego jego wystąpienia.










% ##################
\CenterBoldFont{Uwagi do konkretnych stron}

\vspace{0em}


\noindent
\StrWierszDol{14}{6} Przed tym wierszem powinien znajdować~się pionowy
odstęp. Wtedy układ logiczny tekstu będzie bardziej spójny z~jego układem
graficznym.





% % #############
% \begin{theorem}
%   \label{thm:FichtenholzVolII-01}

%   Każde dwie funkcje pierwotne funkcji $f: I \to \Rbb$ różnią~się
%   o~funkcje stałą.

% \end{theorem}



% \begin{proof}

%   Oznaczmy dowolne dwie funkcje pierwotne $f$ przez $F_{ 1 }$
%   oraz~$F_{ 2 }$ i~oznaczmy $g( x ) := F_{ 1 }( x ) - F_{ 2 }( x )$.
%   Zachodzi wówczas
%   \begin{equation}
%     \label{eq:FichtenholzVolII-06}
%     \frac{ d g( x ) }{ dx } =
%     \frac{ d }{ dx } \big( F_{ 1 }( x ) - F_{ 2 }( x ) \big) = 0, \quad
%     \forall x \in I.
%   \end{equation}
%   Rozważmy dwa dowolne punkty $x_{ 1 }, x_{ 2 } \in I$. Wraz z~nimi
%   w~$I$ ~się cały odcinek $[ x_{ 1 }, x_{ 2 } ]$. Spełnione~są tym
%   samym założenia twierdzenia o~wartości średniej, na~mocy którego
%   \begin{equation}
%     \label{eq:FichtenholzVolII-07}
%     g( x_{ 2 } ) = g( x_{ 1 } ) + g'( c ) ( x_{ 2 } - x_{ 1 } )
%     = g( x_{ 1 } ), \quad c \in ( x_{ 1 }, x_{ 2 } ).
%   \end{equation}

% \end{proof}
% % #############





% Przypomnijmy, że~dowolny zbiór w~$\Rbb$ można rozłożyć na~składowe
% spójne (jest to szczególny przypadek ogólnego twierdzenia dla
% przestrzeni topologicznych, zob.~str.~79,
% \cite{SchwartzKursAnalizyMatematycznejVolI1979}). Wynika stąd
% następujący wniosek.





% % #############
% \begin{corollary}
%   \label{cor:FichtenholzVolII-01}

%   Niech $A = \bigcup_{ \iota \in \Ical } I_{ \iota }$, przy czym każdy
%   z~przedziałów
%   $I_{ \iota }$ jest składową spójną $A$ i~niech $f: A \to \Rbb$
%   będzie różniczkowalna na~każdym $I_{ \iota }$. Wówczas dwie funkcje
%   pierwotne~$f$ różnią~się o~funkcje schodkową $g( x )$, która jest
%   stała na~każdym $I_{ \iota }$.

% \end{corollary}



% \begin{proof}

%   Ponieważ suma dwóch różnych $I_{ \iota }$ nie przedziałem zawartym
%   w~$A$, pochodna jest dobrze określona w~każdym punkcie tego
%   zbioru\footnote{W~przeciwnym razie mogłoby dojść do~sytuacji,
%     że~$I_{ 1 } = (0, 1)$, $I_{ 2 } = [ 1, 2 )$, więc
%     $( 0, 2 ) \subset A$, lecz funkcja nie jest wszędzie
%     różniczkowalna, bo~pochodna w~$1$ nie istnieje.}, więc dla każdego
%   $I_{ \iota }$ można zastosować twierdzenie
%   \ref{thm:FichtenholzVolII-01}, skąd wynika, że~funkcja $G( x )$ jest
%   stała na~tym zbiorze.

% \end{proof}
% % #############





% Podany powyżej kontrprzykład pokazuje, że~silniejszy wynik niż dla
% funkcji określonej na zbiorze $A$ nie może zachodzić. Widzimy teraz,
% że~problem jest bardziej złożony, niż~się powszechnie twierdzi. Błąd
% wynika zapewne stąd, iż~większość osób nie interesuje badanie tak
% „subtelnego” przypadku, jakim jest dziedzina funkcji bardziej
% skomplikowana niż~zbiór~$I$, choć przykład funkcji $1 / x$ pokazuje,
% że~sytuacja ta jest spotykana w~konkretnych, prostych rachunkach.

% Otwartym pozostaje pytanie, czy sam problem można postawić w~bardziej
% ogólnej postaci? Po pierwsze twierdzenie o~wartości średniej wymaga by
% funkcja była ciągła na przedziale $[ a, b ]$, a~różniczkowalna tylko
% w~$( a, b )$ (zob.~punkt~112, str.~196,
% \cite{FichtenholzRachunekRozniczkowyETCVolI2005}), co~potencjalnie
% może prowadzić do~uogólnienia pojęcia funkcji pierwotnej. Po~drugie,
% czy~można zdefiniować w~sposób sensowny pochodną funkcji na~zbiorze,
% który nie jest postaci $\bigcup_{ \iota \in \Ical } I_{ \iota }$ (przy
% zadanym ograniczeniu na~sumę dwóch $I_{ \iota }$)? Jeśli by tak było,
% to pytanie o~postać funkcji pierwotnych jest otwarte, a~wynik ciężki
% do~przewidzenia.

% \vspace{\spaceFour}





% \noindent
% \Str{34} Nie jestem w~stanie zrozumieć, dlaczego największym
% wspólnym dzielnikiem wielomianów $Q$ i~$Q'$ jest $Q_{ 1 }$. Może to
% jakiś znany fakt z~algebry? \Dok

% \vspace{\spaceFour}





% \noindent
% \textbf{Str.~223, punkt [363, 5)].} Problem z~tym przykładem
% jest taki, że
% \begin{equation}
%   \label{eq:FichtenholzVolII-08}
%   \sum_{ n = 1 }^{ \infty } \frac{ x^{ 2n - 1 } }{ 1 - x^{ 2n } }
%   \neq \sum_{ n = 1 }^{ \infty } \left( \frac{ 1 }{ 1 - x^{ 2n - 1 } }
%     - \frac{ 1 }{ 1 - x^{ 2n } } \right).
% \end{equation}
% Gdyby tak równość zachodziła, to po pierwsze
% \begin{equation}
%   \label{eq:FichtenholzVolII-09}
%   \begin{split}
%     \sum_{ i = 1 }^{ n } \frac{ x^{ 2i - 1 } }{ 1 - x^{ 2n } }
%     &=
%       \frac{ 1 }{ 1 - x } - \frac{ 1 }{ 1 - x^{ 2 } }
%       + \frac{ 1 }{ 1 - x^{ 3 } } - \frac{ 1 }{ 1 - x^{ 4 } }
%       + \ldots + \frac{ 1 }{ 1 - x^{ 2n - 1 } }
%       - \frac{ 1 }{ 1 - x^{ 2n } } \\
%     &\neq \frac{ 1 }{ 1 - x } - \frac{ 1 }{ 1 - x^{ 2n } }.
%   \end{split}
% \end{equation}
% Wynika to z~tego, że~z~wyjątkiem $n = 1$
% \begin{equation}
%   \label{eq:FichtenholzVolII-10}
%   \frac{ x^{ 2n - 1 } }{ 1 - x^{ 2n } } \neq
%   \frac{ 1 }{ 1 - x^{ 2n - 1 } } - \frac{ 1 }{ 1 - x^{ 2n } }.
% \end{equation}
% Prawidłowa równość ma~postać
% \begin{equation}
%   \label{eq:FichtenholzVolII-11}
%   \sum_{ n = 1 }^{ \infty } \frac{ x^{ 2n - 1 } ( 1 - x ) }{ ( 1 - x^{ 2n - 1 } )
%     ( 1 - x^{ 2n } ) }
%   \neq
%   \sum_{ n = 1 }^{ \infty } \left( \frac{ 1 }{ 1 - x^{ 2n - 1 } }
%     - \frac{ 1 }{ 1 - x^{ 2n } } \right),
% \end{equation}
% jednak postać wyrazu ogólnego szeregu po~lewej stronie tej równości
% jest znacznie mniej elegancka i~zwięzła, niż~tego
% w~\eqref{eq:FichtenholzVolII-08}, wątpliwe więc, że~Fichtenholz chciał
% podać tu~ten przykład.

% Wydaje mi~się, że~nie jest to kwestia drobnej literówki we~wzorze
% \eqref{eq:FichtenholzVolII-08}, lecz~został tu~popełniony poważniejszy
% błąd, którego nie jestem w~stanie poprawić.

% \vspace{\spaceFour}





% \noindent
% \Str{228} Twierdzenie~2 jest podane w~błędnej formie
% i~udowodnione w~nie najlepszy sposób. Poniżej podana jest poprawna
% wersja twierdzenie i~jaśniejsza, mam nadzieję, wersja dowodu.





% % #############
% \begin{theorem}
%   \label{thm:Fichtenholz-VolII-02}

%   Jeśli istnieje granica\footnote{Zakładamy przy tym, że~$b_{ n }$.}
%   \begin{equation}
%     \label{eq:FichtenholzVolII-12}
%     \lim\limits_{ n \to \infty } \frac{ a_{ n } }{ b_{ n } } = K, \quad
%     ( 0 \leq K \leq +\infty ),
%   \end{equation}
%   to, gdy $K < +\infty$, ze~zbieżności szeregu~(B) wynika zbieżność
%   szeregu~(A), lub równoważnie, z~rozbieżności szeregu (A) wynika
%   rozbieżność szeregu (B). Gdy natomiast $K > 0$, ze zbieżności
%   szeregu (A) wynika zbieżność szeregu (B), lub~równoważnie,
%   z~rozbieżności szeregu~(B) wynika rozbieżność szeregu~(A).

% \end{theorem}



% \begin{proof}
%   Niech $K < +\infty$, wtedy istnieje takie~$N$, że~dla $n > N$
%   \begin{equation}
%     \label{eq:FichtenholzVolII-13}
%     \frac{ a_{ n } }{ b_{ n } } < K + \epsilon, \quad
%     K + \epsilon > 0,
%   \end{equation}
%   skąd
%   \begin{equation}
%     \label{eq:FichtenholzVolII-14}
%     a_{ n } < ( K + \epsilon ) b_{ n }.
%   \end{equation}
%   Z~tej równości wynika, że~zbieżność szeregu~(B) pociąga za~sobą
%   zbieżność~(A), jak również, iż~rozbieżność szeregu~(A) implikuje
%   rozbieżność szeregu~(B). Oba twierdzenia~są jednak równoważne
%   na~mocy zasady kontrapozycji.

%   Analogicznie jeśli~$K > 0$, wówczas istnieje takie~$N$, że~dla
%   $n > N$
%   \begin{equation}
%     \label{eq:FichtenholzVolII-15}
%     a_{ n } > ( K - \epsilon ) b_{ n }, \quad
%     K - \epsilon > 0.
%   \end{equation}
%   Z~tej nierówności wynika, że~zbieżność szeregu~(A) implikuje
%   zbieżność szeregu~(B), zaś rozbieżność szeregu~(B) pociąga za~sobą
%   rozbieżność szeregu~(A). Jak poprzednio, oba te~twierdzenia~są
%   równoważne na~mocy zasady kontrapozycji.

% \end{proof}
% % #############


% \vspace{\spaceFour}





% \noindent
% \StrWg{237}{5} Aby~zachodził wzór $\Dcal_{ n } = \tfrac{ x }{ n + 1 }$,
% należy numerować wyrazy tego szeregu od~0.

% \vspace{\spaceFour}





% ##################
\newpage

\CenterBoldFont{Błędy}


\begin{center}

  \begin{tabular}{|c|c|c|c|c|}
    \hline
    Strona & \multicolumn{2}{c|}{Wiersz} & Jest
                              & Powinno być \\ \cline{2-3}
    & Od góry & Od dołu & & \\
    \hline
    11 & \hphantom{0}2 & & jeszcze~? & jeszcze? \\
           % & & & & \\
           % & & & & \\
           % & & & & \\
           % & & & & \\
           % & & & & \\
    % \hphantom{0}8 & & 11 & $< | \Delta P | <$ & $\leq | \Delta P | \leq$ \\[0.2em]
    % \hphantom{0}8 & & \hphantom{0}7 & $< \frac{ | \Delta P | }{ \Delta x } <$
    %        & $\leq \frac{ | \Delta P | }{ \Delta x } \leq$ \\[0.2em]
    % 12 & & \hphantom{0}5 & $a^{ n }$ & $a_{ n }$ \\
    % 13 & \hphantom{0}7 & & $( x - a )^{ k } \, dx$
    % & $\int ( x - a )^{ -k } \, dx$ \\
    % 18 & & \hphantom{0}4 & $\frac{ 1 }{ 3 }$ & $\frac{ 1 }{ 2 }$ \\
    % 18 & & \hphantom{0}1 & więc$\cdot t$ & więc $t$ \\
    % 23 & 16 & & $n \_ 1$ & $n - 1$ \\
    % 23 & 17 & & $n \_ 2$ & $n - 2$ \\
    % 23 & & 16 & otrzvmujemy & otrzymujemy \\
    % 25 & \hphantom{0}1 & & $e^{ \cdot ( k + 1 ) t}$ & $e^{ ( k + 1 ) t}$ \\
    % 28 & & \hphantom{0}5 & z$\cdot$algebry & z~algebry \\[0.2em]
    % 29 & & 11 & $\frac{ P( x ) }{ ( x - a )^{ k - 1 } Q_{ 1 }( x ) }$
    %        & $\frac{ P_{ 1 }( x ) }{ ( x - a )^{ k - 1 } Q_{ 1 }( x ) }$ \\
    % 34 & \hphantom{0}3 & & [lub (6) & [lub (6)] \\
    % 34 & \hphantom{0}4 & & lub (6)] & [lub (6)] \\
    % 35 & & 17 & $\left[ \frac{ a x^{ 2 } + b x + c }
    %              { x^{ 3 } + x^{ 2 } + x + 1 } \right]$
    %        & $\left[ \frac{ a x^{ 2 } + b x + c }
    %          { x^{ 3 } + x^{ 2 } + x + 1 } \right]'$ \\
    % 35 & & 12 & $x^{ 2 } + x^{ 2 } + x + 1$
    %        & $x^{ 3 } + x^{ 2 } + x + 1$ \\
    % 36 & 14 & & $x^{ \dot{ 2 } }$ & $x^{ 2 }$ \\
    % 39 & \hphantom{0}3 & & $\sqrt[ m ]{
    %              \frac{ \alpha x + \beta }{ \gamma x + \delta } } \, dx$
    %        & $\sqrt[ m ]{
    %          \frac{ \alpha x + \beta }{ \gamma x + \delta } }$ \\
    % 40 & \hphantom{0}2 & & $\frac{ 2t - 1 }{ \sqrt{ 3 } }$
    %        & $\frac{ 2t + 1 }{ \sqrt{ 3 } }$ \\
    \hline
  \end{tabular}





  \newpage

  \begin{tabular}{|c|c|c|c|c|}
    \hline
    Strona & \multicolumn{2}{c|}{Wiersz} & Jest
                              & Powinno być \\ \cline{2-3}
    & Od góry & Od dołu & & \\
    \hline
    % 40  & & 12 & ${ m + 1 \atop n }$ & $\frac{ m + 1 }{ n }$ \\
    % 44  & & \hphantom{0}4 & $\sqrt{ a x }$ & $\sqrt{ a } x$ \\
    % 44  & & \hphantom{0}1 & $\sqrt{ a x }$ & $\sqrt{ a } x$ \\
    % 45  & & 13 & $2 \sqrt{ a } t - b$ & $2 \sqrt{ c } t - b$ \\
    % 46  & & \hphantom{0}8 & $( 2 a x + b^{ 2 } )$ & $( 2 a x + b )^{ 2 }$ \\
    % 222 & & \hphantom{0}5 & [9] jest rzeczywiście & jest oczywiście \\
    % 224 & & 17 & $A'_{ n + m }$ & $A'_{ n - m }$ \\
    % 226 & & \hphantom{0}7 & $H_{ 2k }$ & $H_{ 2^{ k } }$ \\
    % 227 & 11 & & $2^{ k + 1 } + 1$ & $2^{ k - 1 } + 1$ \\
    % 229 & & \hphantom{0}9 & $< \frac{ 1 }{ 2^{ n } }$
    % & $\leq \frac{ 1 }{ 2^{ n } }$ \\
    % 231 & \hphantom{0}8 & & $1\; ;$ & $1;$ \\
    % 233 & & \hphantom{0}5 & $\epsilon = \Ecal - 1$ & $\epsilon < \Ecal - 1$ \\
    % 233 & & \hphantom{0}5 & $\Ecal - \epsilon = 1$ & $\Ecal - \epsilon > 1$ \\
    % 234 & 11 & & $\Dcal_{ n } > 1$ & $\Dcal_{ n } \geq 1$ \\
    % 238 & & \hphantom{0}3 & $( 1 + x )^{ 1 / x } + \ln( 1 + x )$
    %        & $( 1 + x )^{ 1 / x } \ln( 1 + x )$ \\
           & & & & \\
           & & & & \\
           % & & & & \\
           % & & & & \\
           % & & & & \\
           % & & & & \\
           % 479 & \hphantom{0}2 & & $\int\limits_{ 0 }^{ +\infty }$
           % & $\int\limits_{ a }^{ +\infty }$ \\[0.8em]
           % 479 & \hphantom{0}4 & & $\int\limits_{ a }^{ A } \frac{ dx }{ x }$
           % & $\int\limits_{ a }^{ A } \frac{ dx }{ x^{ \lambda } }$ \\[0.5em]
           % 483 & \hphantom{0}4 & & $I^{ 3 }$ & $I^{ 2 }$ \\
           % 526 & & \hphantom{0}7 & $1 + t)$ & $( 1 + t )$ \\
    \hline
  \end{tabular}

\end{center}

\VerSpaceSix


% \StrWd{34}{3} \\
% \Jest  Zróżniczkujmy (10) obustronnie \\
% \Powin Wykonując jawnie różniczkowanie w~(10) \\
% \StrWd{71}{16} \\
% \Jest  $R\left( x, \sqrt{ a x^{ 4 } + b x^{ 3 } + c x^{ 2 } + d x + e }
% \right.$ \\
% \Powin $R\left( x, \sqrt{ a x^{ 4 } + b x^{ 3 } + c x^{ 2 } + d x + e }
% \right)$ \\


% ############################










% ####################################################################
% ####################################################################
% Bibliography

\printbibliography





% ############################
% End of the document

\end{document}
