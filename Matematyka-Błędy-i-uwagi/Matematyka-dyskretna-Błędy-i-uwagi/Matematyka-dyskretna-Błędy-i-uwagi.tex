% ------------------------------------------------------------------------------------------------------------------
% Basic configuration and packages
% ------------------------------------------------------------------------------------------------------------------
% Package for discovering wrong and outdated usage of LaTeX.
% More information to be found in l2tabu English version.
\RequirePackage[l2tabu, orthodox]{nag}
% Class of LaTeX document: {size of paper, size of font}[document class]
\documentclass[a4paper,11pt]{article}



% ------------------------------------------------------
% Packages not tied to particular normal language
% ------------------------------------------------------
% This package should improved spaces in the text
\usepackage{microtype}
% Add few important symbols, like text Celcius degree
\usepackage{textcomp}



% ------------------------------------------------------
% Polonization of LaTeX document
% ------------------------------------------------------
% Basic polonization of the text
\usepackage[MeX]{polski}
% Switching on UTF-8 encoding
\usepackage[utf8]{inputenc}
% Adding font Latin Modern
\usepackage{lmodern}
% Package is need for fonts Latin Modern
\usepackage[T1]{fontenc}



% ------------------------------------------------------
% Setting margins
% ------------------------------------------------------
\usepackage[a4paper, total={14cm, 25cm}]{geometry}



% ------------------------------------------------------
% Setting vertical spaces in the text
% ------------------------------------------------------
% Setting space between lines
\renewcommand{\baselinestretch}{1.1}

% Setting space between lines in tables
\renewcommand{\arraystretch}{1.4}



% ------------------------------------------------------
% Packages for scientific papers
% ------------------------------------------------------
% Switching off \lll symbol, that I guess is representing letter "Ł"
% It collide with `amsmath' package's command with the same name
\let\lll\undefined
% Basic package from American Mathematical Society (AMS)
\usepackage[intlimits]{amsmath}
% Equations are numbered separately in every section
\numberwithin{equation}{section}

% Other very useful packages from AMS
\usepackage{amsfonts}
\usepackage{amssymb}
\usepackage{amscd}
\usepackage{amsthm}

% Package with better looking calligraphy fonts
\usepackage{calrsfs}

% Package with better looking greek letters
% Example of use: pi -> \uppi
\usepackage{upgreek}
% Improving look of lambda letter
\let\oldlambda\Lambda
\renewcommand{\lambda}{\uplambda}




% ------------------------------------------------------
% BibLaTeX
% ------------------------------------------------------
% Package biblatex, with biber as its backend, allow us to handle
% bibliography entries that use Unicode symbols outside ASCII
\usepackage[
language=polish,
backend=biber,
style=alphabetic,
url=false,
eprint=true,
]{biblatex}

\addbibresource{Matematyka-dyskretna-Bibliography.bib}





% ------------------------------------------------------
% Defining new environments (?)
% ------------------------------------------------------
% Defining enviroment "Wniosek"
\newtheorem{corollary}{Wniosek}
\newtheorem{definition}{Definicja}
\newtheorem{theorem}{Twierdzenie}





% ------------------------------------------------------
% Local packages
% You need to put them in the same directory as .tex file
% ------------------------------------------------------
% Package containing various command useful for working with a text
\usepackage{./Local-packages/general-commands}

% Package containing commands and other code useful for working with
% mathematical text
\usepackage{./Local-packages/math-commands}





% ------------------------------------------------------
% Wonderful package PGF/TikZ
% ------------------------------------------------------
\usepackage{tikz}

% Loding TikZ libraries
% \usetikzlibrary{decorations.markings}

% Pics for drawing charts
% \usepackage{PGF-TikZ-Chart-pics}

% Pics for drawing geometric objects
% \usepackage{PGF-TikZ-Geometry-pics}

% Styles for arrows
% \usepackage{PGF-TikZ-Arrows-styles}





% ------------------------------------------------------
% Package "hyperref"
% They advised to put it on the end of preambule
% ------------------------------------------------------
% It allows you to use hyperlinks in the text
\usepackage{hyperref}










% ------------------------------------------------------------------------------------------------------------------
% Title and author of the text
\title{Matematyka dyskretna \\
  {\Large Błędy i~uwagi}}

\author{Kamil Ziemian}


% \date{}
% ------------------------------------------------------------------------------------------------------------------










% ####################################################################
\begin{document}
% ####################################################################





% ######################################
% Title of the text
\maketitle
% ######################################









% % ######################################
% \section{Oznaczenia i~konwencje}

% \label{sec:Oznaczenia-i-konwencje}
% % ######################################










% ######################################
\section{Ronald L.~Graham, Donald E.~Knuth, Oren Patashniki
  \textit{Matematyka konkretna},
  \parencite{Graham-Knuth-Patashnik-Matematyka-Konkretna-Pub-2012}}

\label{sec:Graham-Knuth-Patashnik-Matematyka-konkretna}
% ######################################


% ##################
\CenterBoldFont{Uwagi ogólne}

\vspace{0em}


\noindent
Odstępy w~tekście tej książki, zwłaszcza te występujące po znaku kropki,
wydają mi~się często zbyt duże. Ten problem pojawia~się jednak tak często,
że~nie będę odnotowywał tu każdego jego wystąpienia.

\VerSpaceFour





\noindent
Do tej pory nie udało mi~się znaleźć w~tej książce informacji, czy liczby
naturalne zawierając~$0$, czy też nie. By uniknąć niedomówień, przyjmujemy,
że~zbiór liczb naturalny jest postaci
\begin{equation}
  \label{eq:1111}
  \Nbb = \{ 0, 1, 2, 3, \ldots \}.
\end{equation}

Różnorakie ciągi rozpatrywane w~książce, w~zależności od potrzeby, będą
zaczynały~się od indeksu $0$ lub~$1$, co oczywiście nie przedstawia żadnego
problemu. Jednocześnie jednak warto doprecyzować pojęcie sumy~$S_{ n }$,
oznaczającą sumę początkowych liczb naturalnych. Definiujemy ją jako
\begin{equation}
  \label{eq:1112}
  S_{ n } := \sum_{ i = 0 }^{ n } i.
\end{equation}
Zgodnie więc z~definicją $S_{ 0 } = 0$. Natomiast symbol $S_{ n }$ oznacza
sumę $n + 1$ (!) początkowych liczb naturalnym. Możliwe, że~autorzy
korzystają z~konwencji, wedle której suma w~$S_{ n }$ przebiega od $i = 1$
do~$n$, jednak nie powinno to prowadzić do większych problemów w~tych
notatkach.

\VerSpaceFour





\noindent
Dla zupełności przedstawimy tutaj indukcyjny dowód dobrze znanego wzoru
\begin{equation}
  \label{eq:1113}
  S_{ n } = \frac{ n ( n + 1 ) }{ 2 }.
\end{equation}
Dla bazy indukcyjnej otrzymujemy od razu

\negVerSpaceFour


\begin{subequations}

  \begin{align}
    \label{eq:aa}
    &\hspace{1em} 0 = 0, \\
    &\frac{ 0 ( 0 + 1 ) }{ 2 } = 0.
  \end{align}

\end{subequations}


\noindent
Dla kroku indukcyjnego mamy

\begin{subequations}

  \begin{align}
    \label{eq:aa}
    S_{ n - 1 } &= \frac{ ( n - 1 ) n }{ 2 }, \\[0.2em]
    S_{ n } = S_{ n - 1 } + n = \frac{ ( n - 1 ) n }{ 2 }
                &+ n =
                  \frac{ ( n - 1 ) n  + 2n }{ 2 } =
                  \frac{ n ( n + 1 ) }{ 2 }.
  \end{align}

\end{subequations}








% ##################
\CenterBoldFont{Uwagi do konkretnych stron}

\vspace{0em}


\noindent
\StrWierszDol{14}{6} Przed tym wierszem powinien znajdować~się pionowy
odstęp. Wtedy układ graficzny tekstu będzie bardziej spójny z~jego układem
logicznym.

\VerSpaceFour





\noindent
\StrWierszeDol{14}{10 i~11} Odstęp między tymi wierszami powinien być
mniejszy, tak by Czytelnik od razu zauważył, że~są one częścią objaśnienia
tego samego symbolu.

\VerSpaceFour





\noindent
\textbf{Str. 15, rysunek wieży z~Hanoi.} Na rysunku tym są podpisane
pręty A i~B, pominięto jednak opis pręta~$C$.

\VerSpaceFour





\noindent
\StrWierszDol{18}{12} Czytamy tutaj „Zgadnij postać wyrażenia matematycznego
opisującego szukaną wartość”. Nawet jeśli to jest poprawne tłumaczenie
z~angielskiego, to biorąc pod uwagę, że w~punkcie trzecim czytam „Znajdź
i~udowodnij”, treść omawianego zadania, w~której sporo miejsca poświęcono
wyprowadzeniu konkretnego wyrażenia oraz sens jaki, w~naszej opinii, tekst
powinien przekazywać, proponujemy by zmienić ten tekst został zmieniony
na~„Znajdź postać\ldots”. Postulowaną zmianę umieściliśmy, też w~tabeli w~dziale
„Błędy”.

\VerSpaceFour





\noindent
\Str{21} Indukcyjny dowód wzoru
\begin{equation}
  \label{eq:1111}
  L_{ n } = \frac{ n ( n + 1 ) }{ 2 } + 1,
\end{equation}
można też przeprowadzić w~następujący sposób. Na podstawie „rozwinięcia”
formuły na $L_{ n }$ postulujemy związek
\begin{equation}
  \label{eq:1113}
  L_{ n } = 1 + S_{ n },
\end{equation}
gdzie $S_{ n }$ to suma szeregu kolejnych liczb naturalnych. Baza indukcyjna
jest spełniona, mamy bowiem

\negVerSpaceFive


\begin{subequations}

  \begin{align}
    \label{eq:1114}
    L_{ 0 } &= 1, \\
    1 + S_{ 0 } &= 1 + 0 = 1,
  \end{align}

\end{subequations}

\noindent
zgodnie z~tym co powiedziano wcześniej o~definicji symbolu~$S_{ n }$. Gdy
chodzi o~krok indukcyjny, mamy
\begin{equation}
  \label{eq:1115}
  L_{ n } = L_{ n - 1 } + n = 1 + S_{ n - 1 } + n = 1 + S_{ n }.
\end{equation}

\VerSpaceFour





\noindent
Wkraczając w~sferę przesadnego pedantyzmu, przyjrzyjmy~się podanej tu
przybliżonej definicji zwartego wyrażenia matematycznego.




% ##################
\begin{quote}

  Wyrażenie na wartość $f( n )$ jest w~postaci zwartej, jeśli możemy je
  wyliczyć stosując skończoną, niezależna od~$n$, liczbę „dobrze znanych”
  działań standardowych.

\end{quote}
% ##################





Autorzy stwierdzają, że~wyrażenie $2^{ n }$ jest w~postaci zwartej, gdyż
niezależnie od~wartości $n$ obliczenie go sprowadza~się do wykonania jednej
wartości potęgowania. Tutaj jednak natrafiamy na problem, który autorzy
omawiają na przykładzie funkcji silnia. Stwierdzają bowiem, że~wyrażenie
$1 \cdot 2 \cdot 3 \cdot \ldots \cdot n$ nie jest w~postaci zwartej, zaś $n!$ jest, mimo że
oba są sobie równe. Tak samo, $2^{ n }$ jest przez nich zakwalifikowane jako wyrażenie zwarte, lecz równoważny mu iloczyn
\begin{equation}
  \label{eq:aa}
  \underbrace{2 \cdot 2 \cdot \ldots \cdot 2}_{ n \text{ razy} } = 2^{ n },
\end{equation}
już nie możemy być tak zakwalifikowane, bo zawiera $n - 1$ operacji mnożenia.

Jak sami autorzy zaznaczają, definicja ta jest tylko przybliżona, ma pomóc
nam wyrobić sobie intuicję co to jest wyrażenie w~postaci zwartej, a~nie
podawać dokładne kryterium. Dlatego te uwagi nie podważają tego co napisano
w~książce, kierują nas jednak ku konkretnemu problemowi. Czy na to czy dane
wyrażenie jest zwarte, ma wpływ to, czy istnieje odpowiednio „przyzwoita”
procedura rachunkowa pozwalające je obliczyć, choćby tylko w~przybliżeniu?
W~książce znajdujemy stwierdzenie, że~dla silni istnieje wzór Stirlinga,
który pozwala obliczyć silnię z~dobrą dokładnością, a~który zawiera tylko
„przyzwoite” operacje, dlatego $n!$ można uważać za wyrażenie w~postaci
zwartej. Przemawia to na rzecz twierdzącej odpowiedzi na zadane wcześniej
pytanie.

Zagadnienie to jest niewątpliwie ciekawe, na razie musimy jednak przerwać
nasze rozważania w~tym miejscu. Zaznaczmy tylko następujące pytanie:
czy istnieje procedura rachunkowa, która pozwala uznać $2^{ n }$ za
wyrażenie w~postaci zwartej?








% ##################
\begin{figure}[h]

  \label{fig:Leja-Funkcje-zespolone-01}

  \centering


  \begin{tikzpicture}

    \coordinate (A) at (0,0);
    \coordinate (B) at (4,0);
    \coordinate (C) at (4,1.5);
    \coordinate (D) at (0,1.5);
    \coordinate (center) at (2,0.75);



    \draw[dotted] (A) -- (center);

    \draw[dotted] (D) -- (center);

    \draw (center) -- (B);

    \draw (center) -- (C);


    \node at (3.5,0.75) {1};


    \node at (2,-1) {Wersja~(a)};





    \begin{scope}[xshift=5cm]



    \end{scope}

  \end{tikzpicture}

  \caption{Obecna wersja rysunku na dole strony 22~(a), proponowana
    wersja~(b)}


\end{figure}
% ##################










% % #############
% \begin{theorem}
%   \label{thm:FichtenholzVolII-01}


% \end{theorem}



% \begin{proof}

% \end{proof}
% % #############









% % #############
% \begin{corollary}
%   \label{cor:FichtenholzVolII-01}


% \begin{proof}


% \end{proof}
% % #############
















% ##################
\newpage

\CenterBoldFont{Błędy}


\begin{center}

  \begin{tabular}{|c|c|c|c|c|}
    \hline
    Strona & \multicolumn{2}{c|}{Wiersz} & Jest
                              & Powinno być \\ \cline{2-3}
    & Od góry & Od dołu & & \\
    \hline
    11 & \hphantom{0}2 & & jeszcze~? & jeszcze? \\
    18 & & 12 & Zgadnij & Znajdź \\
    % & & & & \\
    % & & & & \\
    % & & & & \\
    % & & & & \\
    \hline
  \end{tabular}





  % \newpage

  % \begin{tabular}{|c|c|c|c|c|}
  %   \hline
  %   Strona & \multicolumn{2}{c|}{Wiersz} & Jest
                                             %   & Powinno być \\ \cline{2-3}
  %   & Od góry & Od dołu & & \\
  %   \hline
  %              & & & & \\
  %          & & & & \\
  %          % & & & & \\
  %          % & & & & \\
  %          % & & & & \\
  %          % & & & & \\
  %   \hline
  % \end{tabular}

\end{center}

\VerSpaceSix





% ############################










% ####################################################################
% ####################################################################
% Bibliography

\printbibliography





% ############################
% End of the document

\end{document}
