% ---------------------------------------------------------------------
% Basic configuration and packages
% ---------------------------------------------------------------------
% Package for discovering wrong and outdated usage of LaTeX.
% More information to be found in l2tabu English version.
\RequirePackage[l2tabu, orthodox]{nag}
% Class of LaTeX document: {size of paper, size of font}[document class]
\documentclass[a4paper,11pt]{article}



% ---------------------------------------
% Packages not tied to particular normal language
% ---------------------------------------
% This package should improved spaces in the text.
\usepackage{microtype}
% Add few important symbols, like text Celcius degree
\usepackage{textcomp}



% ---------------------------------------
% Polonization of LaTeX document
% ---------------------------------------
% Basic polonization of the text
\usepackage[MeX]{polski}
% Switching on UTF-8 encoding
\usepackage[utf8]{inputenc}
% Adding font Latin Modern
\usepackage{lmodern}
% Package is need for fonts Latin Modern
\usepackage[T1]{fontenc}



% ---------------------------------------
% Setting margines
% ---------------------------------------
% Package for easy settings of margins. Unit of measurement is inch.
\usepackage{vmargin}
\setmarginsrb
{ 0.7in}  % left margin
{ 0.6in}  % top margin
{ 0.7in}  % right margin
{ 0.8in}  % bottom margin
{  20pt}  % head height
{0.25in}  % head sep
{   9pt}  % foot height
{ 0.3in}  % foot sep



% ---------------------------------------
% Setting vertical spaces in the text
% ---------------------------------------
% Setting space between lines
\renewcommand{\baselinestretch}{1.1}

% Setting space between lines in tables
\renewcommand{\arraystretch}{1.4}



% ---------------------------------------
% Packages for scientific papers
% ---------------------------------------
% Switching off \lll symbol, that I guess is representing letter ``Ł''.
% It collide with `amsmath' package's command with the same name
\let\lll\undefined
% Basic package from American Mathematical Society (AMS)
\usepackage[intlimits]{amsmath}
% Equations are numbered separately in every section.
\numberwithin{equation}{section}

% Other very useful packages from AMS
\usepackage{amsfonts, amssymb, amscd, amsthm}
% Better looking calligraphy fonts
\usepackage{calrsfs}

% Better looking greek letters
% Example of use: pi -> \uppi
\usepackage{upgreek}
% Improving look of lambda letter
\let\oldlambda\Lambda
\renewcommand{\lambda}{\uplambda}





% ------------------------------
% Private packages
% You need to put them in the same directory as .tex file
% ------------------------------
% Contains various command useful for working with a text
\usepackage{latexgeneralcommands}
% Contains definitions useful for working with mathematical text
\usepackage{mathcommands}





% ------------------------------
% Package ``hyperref''
% They advised to put it on the end of preambule
% ------------------------------
% It allows you to use hyperlinks in the text
\usepackage{hyperref}









% ---------------------------------------------------------------------
% Tytuł, autor, data
\title{Rachunek prawdopodobieństwa \\
  {\Large Błędy i~uwagi}}

\author{Kamil Ziemian}


% \date{}
% ---------------------------------------------------------------------










% ####################################################################
\begin{document}
% ####################################################################





% ######################################
\maketitle % Tytuł całego tekstu
% ######################################





% ######################################
\section{Patrick Billingsley
  \textit{Prawdopodobieństwo i~miara},
  \cite{BillingsleyPrawdopodobienstwoIMiara2009}}
% ############################

\vspace{0em}


% ##################
\CenterBoldFont{Uwagi}

\vspace{0em}


\noindent
Warto byłoby przepisać tę~książkę w~\LaTeX u. Widać bowiem, że~wzory
matematyczne zostały zapisane w~jakimś, zapewne pochodzącym z~czasów PRLu
systemie i~symbole są przez to~niskiej jakości, takie pikselowate.

\VerSpaceFour





\noindent
Pionowe kreski w~wykresach funkcji schodkowych, takich jak
te~na~stronie 17, powinny być rysowane linią przerywaną, nie zaś jak
obecnie ciągłą. Stosuje~się to oczywiście do wszystkich funkcji, które
posiadają nieciągłość w~formie skoku.

\VerSpaceFour





% ##################
\CenterBoldFont{Uwagi do~konkretnych stron}


\noindent
\Str{17} Przez liczbę dwójkowo-wymierną rozumie chyba Billingsley liczbę
którą da~się zapisać w~postaci $p / 2^{ n }$, $p < 2^{ n }$.

\VerSpaceFour





\noindent
\Str{17} Pojęcie rzędu przedziału może stwarzać problem, bo~wydaje~się,
że~Billingsley przyjął, iż~liczby $a_{ 0 }, a_{ 1 }, \ldots, a_{ n }$ tworzą
podział rzędu $n$, jeśli każda z~nich da~się zapisać w~postaci
$p / 2^{ n }$, niezależnie od~tego, czy jest to forma nieskracalna czy nie.
Wnioskuję to z~tego, że~ciąg $0, 1/4, 2/4, 3/4, 1$ powinien tworzyć podział
rzędu 2, jednak $2/4 = 1/2$.

Należałoby więc rząd przedziału zdefiniować w~ten sposób, że~każdą
z~liczb $a_{ i }$ można zapisać w~postaci $p_{ i } / 2^{ n }$ przy
czym co~najmniej inna z~nich jest w~postaci nieskracalnej. Tym samym
pojęcie to~stanie~się jednoznaczne.

\VerSpaceFour





% ##################
\CenterBoldFont{Błędy}


\begin{center}

  \begin{tabular}{|c|c|c|c|c|}
    \hline
    Strona & \multicolumn{2}{c|}{Wiersz} & Jest
                              & Powinno być \\ \cline{2-3}
    & Od góry & Od dołu & & \\
    \hline
    %     & & & & \\
    %     & & & & \\
    %     & & & & \\
    %     & & & & \\
    %     & & & & \\
    %     & & & & \\
    %     & & & & \\
    %     & & & & \\
    %     & & & & \\
    %     & & & & \\
    %     & & & & \\
    %     & & & & \\
    %     & & & & \\
    %     & & & & \\
    %     & & & & \\
    %     & & & & \\
    \hline
  \end{tabular}

\end{center}

\VerSpaceSix


% ############################









% ######################################
\section{William Feller \textit{Wstęp do rachunku prawdopodobieństwa. Tom~I},
  \cite{FellerWstepDoRachunkuPrawdopodobienstwaVolI2006}}

\label{sec:Feller-Wstep-do-rachunku-ETC}
% ######################################



% ######################################
\subsection{Uwagi do~konkretnych stron}

\label{subsec:Uwagi-do-konkretnych-stron}
% ######################################



\noindent
\StrWierszeG{13}{9--11} Według mnie zdanie „Potrzebujemy ścisłej
matematycznej teorii opartej na zasadach ogólnie przyjętych w~geometrii
i~mechanice.”, powinno być zastąpione przez „Potrzebujemy ścisłej
matematycznej teorii opartej na ogólnych zasadach, jak tych przyjętych
w~geometrii i~mechanice.”.

{\Large Od strony 17 należy zacząć ponowne czytanie książki.}



% ##################
\CenterBoldFont{Błędy}


\begin{center}

  \begin{tabular}{|c|c|c|c|c|}
    \hline
    Strona & \multicolumn{2}{c|}{Wiersz} & Jest
                              & Powinno być \\ \cline{2-3}
    & Od góry & Od dołu & & \\
    \hline
    18 & & 5 & \ldots lub & \ldots{} lub \\
    19 & & & $- |$ & $- \}$ \\
    28 & & 9 & wielu sposobami & na wiele sposobów \\
    % & & & & \\
    % & & & & \\
    % & & & & \\
    \hline
  \end{tabular}

\end{center}

\VerSpaceSix



% ############################










% ####################################################################
% ####################################################################
% Bibliografia

\bibliographystyle{plalpha}

\bibliography{MathematicsBooks}{}





% ############################

% Koniec dokumentu
\end{document}
