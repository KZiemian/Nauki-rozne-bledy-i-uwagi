% ---------------------------------------------------------------------
% Podstawowe ustawienia i pakiety
% ---------------------------------------------------------------------
\RequirePackage[l2tabu, orthodox]{nag} % Wykrywa przestarzałe i niewłaściwe
% sposoby używania LaTeXa. Więcej jest w l2tabu English version.


\documentclass[a4paper,11pt]{article}
% {rozmiar papieru, rozmiar fontu}[klasa dokumentu]
\usepackage[MeX]{polski} % Polonizacja LaTeXa, bez niej będzie pracował
% w języku angielskim.
\usepackage[utf8]{inputenc} % Włączenie kodowania UTF-8, co daje dostęp
% do polskich znaków.
\usepackage[T1]{fontenc} % Potrzebne do używania fontów Latin Modern.
\usepackage{lmodern} % Wprowadza fonty Latin Modern.



% ------------------------------
% Podstawowe pakiety (niezwiązane z ustawieniami języka)
% ------------------------------
\usepackage{microtype} % Twierdzi, że poprawi rozmiar odstępów w tekście.
\usepackage{graphicx} % Wprowadza bardzo potrzebne komendy do wstawiania
% grafiki.
\usepackage{verbatim} % Poprawia otoczenie VERBATIME.
\usepackage{textcomp} % Dodaje takie symbole jak stopnie Celsiusa,
% wprowadzane bezpośrednio w tekście.
\usepackage{vmargin} % Pozwala na prostą kontrolę rozmiaru marginesów,
% za pomocą komend poniżej. Rozmiar odstępów jest mierzony w calach.
% ------------------------------
% MARGINS
% ------------------------------
\setmarginsrb
{ 0.7in}  % left margin
{ 0.6in}  % top margin
{ 0.7in}  % right margin
{ 0.8in}  % bottom margin
{  20pt}  % head height
{0.25in}  % head sep
{   9pt}  % foot height
{ 0.3in}  % foot sep



% ------------------------------
% Często przydatne pakiety
% ------------------------------
% \usepackage{csquotes} % Pozwala w prosty sposób wstawiać cytaty do tekstu.
\usepackage{xcolor} % Pozwala używać kolorowych czcionek (zapewne dużo
% więcej, ale ja nie potrafię nic o tym powiedzieć).



% ------------------------------
% Pakiety do tekstów z nauk przyrodniczych
% ------------------------------
\let\lll\undefined % Amsmath gryzie się z pakietami do języka
% polskiego, bo oba definiują komendę \lll. Aby rozwiązać ten problem
% oddefiniowuję tę komendę, ale może tym samym pozbywam się dużego Ł.
\usepackage[intlimits]{amsmath} % Podstawowe wsparcie od American
% Mathematical Society (w skrócie AMS)
\usepackage{amsfonts, amssymb, amscd, amsthm} % Dalsze wsparcie od AMS
\usepackage{bm}  % Daję komendę \bm do pogrubionej czcionki matematycznej
% \usepackage{siunitx} % Do prostszego pisania jednostek fizycznych
\usepackage{upgreek} % Ładniejsze greckie litery
% Przykładowa składnia: pi = \uppi
\usepackage{slashed} % Pozwala w prosty sposób pisać slash Feynmana.
\usepackage{calrsfs} % Zmienia czcionkę kaligraficzną w \mathcal
% na ładniejszą. Może w innych miejscach robi to samo, ale o tym nic
% nie wiem.



% ------------------------------
% Tworzenie środowisk (?) „Twierdzenie”, „Definicja”, „Lemat”, etc.
% ------------------------------
% Komenda wprowadzająca otoczenie „theorem” do pisania twierdzeń
% matematycznych.
\newtheorem{theorem}{Twierdzenie}
% Analogicznie jak powyżej
\newtheorem{definition}{Definicja}
\newtheorem{corollary}{Wniosek}



% ------------------------------
% Pakiety napisane przez użytkownika.
% Mają być w tym samym katalogu to ten plik .tex
% ------------------------------
\usepackage{latexgeneralcommands}
\usepackage{mathcommands}

 % Pakiet napisany między innymi dla tego pliku.
\usepackage{ODEcommands}





% ---------------------------------------------------------------------
% Dodatkowe ustawienia dla języka polskiego
% ---------------------------------------------------------------------
\renewcommand{\thesection}{\arabic{section}.}
% Kropki po numerach rozdziału (polski zwyczaj topograficzny)
\renewcommand{\thesubsection}{\thesection\arabic{subsection}}
% Brak kropki po numerach podrozdziału



% ------------------------------
% Ustawienia różnych parametrów tekstu
% ------------------------------
\renewcommand{\baselinestretch}{1.1}

% Ustawienie szerokości odstępów między wierszami tabeli.
\renewcommand{\arraystretch}{1.4}



% ------------------------------
% Pakiet "hyperref"
% Polecano by umieszczać go na końcu preambuły.
% ------------------------------
\usepackage{hyperref} % Pozwala tworzyć hiperlinki i zamienia odwołania
% do bibliografii na hiperlinki.










% ------------------------------------------------------------------------------------
% Tytuł, autor, data
\title{Równania różniczkowe zwyczajne \\
  {\Large Błędy i~uwagi}}

\author{Kamil Ziemian}


% \date{}
% ------------------------------------------------------------------------------------










% ####################################################################
\begin{document}
% ####################################################################





% ######################################
\maketitle % Tytuł całego tekstu
% ######################################





% ##############################
\AuthorsAndTitleOfWork{
  Władimir Igoriewicz Arnold \\
  \textit{Równania różniczkowe zwyczajne},
  \cite{ArnoldRownaniaRozniczkoweZwyczajne1975}}


% ##################
\newpage

\CenterBoldFont{Błędy}


\begin{center}

  \begin{tabular}{|c|c|c|c|c|}
    \hline
    Strona & \multicolumn{2}{c|}{Wiersz} & Jest
                              & Powinno być \\ \cline{2-3}
    & Od góry & Od dołu & & \\
    \hline
    5   & &  7 & 1968 - 196 & 1968 - 1969 \\
    11  & 17 & & mechanice klasycznej & mechanice kwantowej \\
    15  & & 16 & rozdziale 6 & rozdziale 5 \\
    28  & 15 & & wzór (8) & wzór \\
    34  &  9 & & $\dot{ x }_{ 1 } = x_{ 2 }$ & $\dot{ x }_{ 1 } = x_{ 1 }$ \\
    47  & & 12 & $x_{ i } = \varphi_{ i }( x_{ 1 }, \ldots, x_{ n } )$
           & $x_{ i } = \varphi_{ i }( y_{ 1 }, \ldots, y_{ n } )$ \\
    53  & & 14 & obrót & obrót krzywych całkowych \\
    56  & 12 & & osobliwym & nieosobliwym \\
    61  &  7 & & $\vecxbold$???, $\vecalphabold_{ 0 }$
           & $\vecxbold$, $\vecalphabold$ \\
    64  & & 11 & ????$\vecgbold( t_{ 2 }, t_{ 1 }, \vecxbold )
                = \vecgbold^{ t_{ 2 } }_{ t_{ 1 } }( \vecxbold, t_{ 1 } )$
           & $\vecxbold^{ t_{ 2 } }_{ t_{ 1 } }( \vecxbold, t_{ 1 } )
             = \vecgbold( t_{ 2 }, t_{ 1 }, \vecxbold )$ \\
    64  & & 10 & $( \vecvarphibold( t ), t )$
           & $( t, \vecvarphibold( t ) )$ \\[0.3em]
    66  & 17 & & $\vecvbold( t, \vecxbold, \dot{ \vecalphabold } )$
           & $\vecvbold( t, \vecxbold, \vecalphabold )$??? \\[0.3em]
    70  & & 13 & $\dot{ p }_{ i } = \frac{ \partial H }{ \partial q_{ i } }$
           & $\dot{ p }_{ i } = -\frac{ \partial H }{ \partial q_{ i } }$ \\[0.3em]
    71  & & 15 & $\frac{ \partial \vecvbold_{ 0 } }{ \vecxbold }$
           & $\frac{ \partial \vecvbold_{ 0 } }{ \partial \vecxbold }$ \\[0.4em]
    72  &  4 & & „niezaburzonego” & „zaburzonego” \\
    73  &  5 & & $\vecxbold( 0$ & $\vecxbold( 0 )$ \\
    90  & &  3 & \textit{wraz z pochodną dla} $x = 0$
           & \textit{dla} $x = 0$ \\
    92  &  3 & & $U( x( O ) )$ & $U( x( 0 ) )$ \\[0.3em]
    123 &  6 & & $^{ \Rbb }A : \Cbb^{ m } \to { }^{ \Rbb } \Cbb^{ n }$
           & ${ }^{ \Rbb }A : { }^{ \Rbb } \Cbb^{ m }
             \to { }^{ \Rbb } \Cbb^{ n }$ \\
    125 &  5 & & $\mathbf{I}$ & $I$ \\
    % & & & & \\
    % & & & & \\
    % & & & & \\
    % & & & & \\
    \hline
  \end{tabular}

\end{center}

\VerSpaceSix


\noindent
\StrWierszD{66}{7} \\
\Jest  \textit{tyłu do~brzegu} \\
\Powin \textit{tyłu nieograniczenie albo~do~brzegu} \\
\StrWierszG{110}{9} \\
\Jest  sumą częściową szeregu --~iloczynu \\
\Powin jest sumą części wyrazów iloczynu \\



% ############################










% ############################
\AuthorsAndTitleOfWork{ % Autor i tytuł dzieła
  N. M. Matwiejew \\
  \textit{Metody całkowania równana różniczkowych zwyczajnych},
  \cite{MatwiejewMetodyCalkowaniaRownanRozniczkowychZwyczajnych1982}}

\vspace{0em}


% ##################
\CenterBoldFont{Błędy}


\begin{center}

  \begin{tabular}{|c|c|c|c|c|}
    \hline
    Strona & \multicolumn{2}{c|}{Wiersz} & Jest
                              & Powinno być \\ \cline{2-3}
    & Od góry & Od dołu & & \\
    \hline
    5   & &  9 & Dodzimy & Dowodzimy \\
    5   & &  8 & potkowych & początkowych \\
    5   & &  7 & poąątkowych & początkowych \\
    10  & & 19 & damy & mamy \\
    % 15  & & & & \\
    % & & & & \\
    % & & & & \\
    \hline
  \end{tabular}

\end{center}

\VerSpaceSix


\noindent
\StrWierszG{15}{13}
\Jest  w~sensie ustępu \\
\Powin w~sensie zdefiniowanym w~ustępie \\
\StrWierszD{20}{2} \\
\Jest  i~nie ma rozwiązania określonego w~tym samym przedziale
nie~identycznego z~rozwiązaniem $y = y( x )$ chociażby w~jednym
punkcie przedziału $\absOne{ x - x_{ 0 } } \leq h$ różnym
od~punktu $x = x_{ 0 }$. \\
\Powin i~nie istnieje inne rozwiązanie określone w~przedziale
$\absOne{ x - x_{ 0 } } \leq h_{ 1 } \leq h$ które nie byłoby równe
rozwiązaniu $y = y( x )$ w~każdym punkcie przedziału
$\absOne{ x - x_{ 0 } } \leq h_{ 1 }$. \\


% ############################










% ####################################################################
% ####################################################################
% Bibliografia

\bibliographystyle{plalpha}

\bibliography{MathematicsBooks}{}





% ############################

% Koniec dokumentu
\end{document}
