% ------------------------------------------------------------------------------------------------------------------
% Basic configuration and packages
% ------------------------------------------------------------------------------------------------------------------
% Package for discovering wrong and outdated usage of LaTeX.
% More information to be found in l2tabu English version.
\RequirePackage[l2tabu, orthodox]{nag}
% Class of LaTeX document: {size of paper, size of font}[document class]
\documentclass[a4paper,11pt]{article}



% ------------------------------------------------------
% Packages not tied to particular normal language
% ------------------------------------------------------
% This package should improved spaces in the text
\usepackage{microtype}
% Add few important symbols, like text Celcius degree
\usepackage{textcomp}



% ------------------------------------------------------
% Polonization of LaTeX document
% ------------------------------------------------------
% Basic polonization of the text
\usepackage[MeX]{polski}
% Switching on UTF-8 encoding
\usepackage[utf8]{inputenc}
% Adding font Latin Modern
\usepackage{lmodern}
% Package is need for fonts Latin Modern
\usepackage[T1]{fontenc}



% ------------------------------------------------------
% Setting margins
% ------------------------------------------------------
\usepackage[a4paper, total={14cm, 25cm}]{geometry}



% ------------------------------------------------------
% Setting vertical spaces in the text
% ------------------------------------------------------
% Setting space between lines
\renewcommand{\baselinestretch}{1.1}

% Setting space between lines in tables
\renewcommand{\arraystretch}{1.4}



% ------------------------------------------------------
% Packages for scientific papers
% ------------------------------------------------------
% Switching off \lll symbol, that I guess is representing letter "Ł"
% It collide with `amsmath' package's command with the same name
\let\lll\undefined
% Basic package from American Mathematical Society (AMS)
\usepackage[intlimits]{amsmath}
% Equations are numbered separately in every section
\numberwithin{equation}{section}

% Other very useful packages from AMS
\usepackage{amsfonts}
\usepackage{amssymb}
\usepackage{amscd}
\usepackage{amsthm}

% Package with better looking calligraphy fonts
\usepackage{calrsfs}

% Package with better looking greek letters
% Example of use: pi -> \uppi
\usepackage{upgreek}
% Improving look of lambda letter
\let\oldlambda\Lambda
\renewcommand{\lambda}{\uplambda}




% ------------------------------------------------------
% BibLaTeX
% ------------------------------------------------------
% Package biblatex, with biber as its backend, allow us to handle
% bibliography entries that use Unicode symbols outside ASCII
\usepackage[
language=polish,
backend=biber,
style=alphabetic,
url=false,
eprint=true,
]{biblatex}

\addbibresource{Logika-i-teoria-mnogości-Bibliography.bib}





% ------------------------------------------------------
% Defining new environments (?)
% ------------------------------------------------------
% Defining enviroment "Wniosek"
\newtheorem{corollary}{Wniosek}
\newtheorem{definition}{Definicja}
\newtheorem{theorem}{Twierdzenie}





% ------------------------------------------------------
% Local packages
% You need to put them in the same directory as .tex file
% ------------------------------------------------------
% Package containing various command useful for working with a text
\usepackage{general-commands}
% Package containing commands and other code useful for working with
% mathematical text
\usepackage{math-commands}





% ------------------------------------------------------
% Package "hyperref"
% They advised to put it on the end of preambule
% ------------------------------------------------------
% It allows you to use hyperlinks in the text
\usepackage{hyperref}










% ------------------------------------------------------------------------------------------------------------------
% Title and author of the text
\title{Równania różniczkowe zwyczajne \\
  {\Large Błędy i~uwagi}}

\author{Kamil Ziemian}


% \date{}
% ------------------------------------------------------------------------------------------------------------------










% ####################################################################
\begin{document}
% ####################################################################





% ######################################
\maketitle % Tytuł całego tekstu
% ######################################





% ##############################
\section{
  Władimir Igoriewicz Arnold \\
  \textit{Równania różniczkowe zwyczajne},
  \cite{ArnoldRownaniaRozniczkoweZwyczajne1975}}


% ##################
\newpage

\CenterBoldFont{Błędy}


\begin{center}

  \begin{tabular}{|c|c|c|c|c|}
    \hline
    Strona & \multicolumn{2}{c|}{Wiersz} & Jest
                              & Powinno być \\ \cline{2-3}
    & Od góry & Od dołu & & \\
    \hline
    5   & &  7 & 1968 - 196 & 1968 - 1969 \\
    11  & 17 & & mechanice klasycznej & mechanice kwantowej \\
    15  & & 16 & rozdziale 6 & rozdziale 5 \\
    28  & 15 & & wzór (8) & wzór \\
    34  &  9 & & $\dot{ x }_{ 1 } = x_{ 2 }$ & $\dot{ x }_{ 1 } = x_{ 1 }$ \\
    47  & & 12 & $x_{ i } = \varphi_{ i }( x_{ 1 }, \ldots, x_{ n } )$
           & $x_{ i } = \varphi_{ i }( y_{ 1 }, \ldots, y_{ n } )$ \\
    53  & & 14 & obrót & obrót krzywych całkowych \\
    56  & 12 & & osobliwym & nieosobliwym \\
    61  &  7 & & $\vecxBold$???, $\vecalphaBold_{ 0 }$
           & $\vecxBold$, $\vecalphaBold$ \\
    64  & & 11 & ????$\vecgBold( t_{ 2 }, t_{ 1 }, \vecxBold )
                = \vecgBold^{ t_{ 2 } }_{ t_{ 1 } }( \vecxBold, t_{ 1 } )$
           & $\vecxBold^{ t_{ 2 } }_{ t_{ 1 } }( \vecxBold, t_{ 1 } )
             = \vecgBold( t_{ 2 }, t_{ 1 }, \vecxBold )$ \\
    64  & & 10 & $( \vecvarphiBold( t ), t )$
           & $( t, \vecvarphiBold( t ) )$ \\[0.3em]
    66  & 17 & & $\vecvBold( t, \vecxBold, \dot{ \vecalphaBold } )$
           & $\vecvBold( t, \vecxBold, \vecalphaBold )$??? \\[0.3em]
    70  & & 13 & $\dot{ p }_{ i } = \frac{ \partial H }{ \partial q_{ i } }$
           & $\dot{ p }_{ i } = -\frac{ \partial H }{ \partial q_{ i } }$ \\[0.3em]
    71  & & 15 & $\frac{ \partial \vecvBold_{ 0 } }{ \vecxBold }$
           & $\frac{ \partial \vecvBold_{ 0 } }{ \partial \vecxBold }$ \\[0.4em]
    72  &  4 & & „niezaburzonego” & „zaburzonego” \\
    73  &  5 & & $\vecxBold( 0$ & $\vecxBold( 0 )$ \\
    90  & &  3 & \textit{wraz z pochodną dla} $x = 0$
           & \textit{dla} $x = 0$ \\
    92  &  3 & & $U( x( O ) )$ & $U( x( 0 ) )$ \\[0.3em]
    123 &  6 & & $^{ \Rbb }A : \Cbb^{ m } \to { }^{ \Rbb } \Cbb^{ n }$
           & ${ }^{ \Rbb }A : { }^{ \Rbb } \Cbb^{ m }
             \to { }^{ \Rbb } \Cbb^{ n }$ \\
    125 &  5 & & $\mathbf{I}$ & $I$ \\
    % & & & & \\
    % & & & & \\
    % & & & & \\
    % & & & & \\
    \hline
  \end{tabular}

\end{center}

\VerSpaceSix


\noindent
\StrWierszDol{66}{7} \\
\Jest \textit{tyłu do~brzegu} \\
\PowinnoByc \textit{tyłu nieograniczenie albo~do~brzegu} \\
\StrWierszGora{110}{9} \\
\Jest sumą częściową szeregu --~iloczynu \\
\PowinnoByc jest sumą części wyrazów iloczynu \\



% ############################










% ############################
\section{ % Autor i tytuł dzieła
  N. M. Matwiejew \\
  \textit{Metody całkowania równana różniczkowych zwyczajnych},
  \cite{MatwiejewMetodyCalkowaniaRownanRozniczkowychZwyczajnych1982}}

\vspace{0em}


% ##################
\CenterBoldFont{Błędy}


\begin{center}

  \begin{tabular}{|c|c|c|c|c|}
    \hline
    Strona & \multicolumn{2}{c|}{Wiersz} & Jest
                              & Powinno być \\ \cline{2-3}
    & Od góry & Od dołu & & \\
    \hline
    5   & &  9 & Dodzimy & Dowodzimy \\
    5   & &  8 & potkowych & początkowych \\
    5   & &  7 & poąątkowych & początkowych \\
    10  & & 19 & damy & mamy \\
    % 15  & & & & \\
    % & & & & \\
    % & & & & \\
    \hline
  \end{tabular}

\end{center}

\VerSpaceSix


\noindent
\StrWierszGora{15}{13}
\Jest w~sensie ustępu \\
\PowinnoByc w~sensie zdefiniowanym w~ustępie \\
\StrWierszDol{20}{2} \\
\Jest i~nie ma rozwiązania określonego w~tym samym przedziale
nie~identycznego z~rozwiązaniem $y = y( x )$ chociażby w~jednym
punkcie przedziału $\absOne{ x - x_{ 0 } } \leq h$ różnym
od~punktu $x = x_{ 0 }$. \\
\PowinnoByc i~nie istnieje inne rozwiązanie określone w~przedziale
$\absOne{ x - x_{ 0 } } \leq h_{ 1 } \leq h$ które nie byłoby równe
rozwiązaniu $y = y( x )$ w~każdym punkcie przedziału
$\absOne{ x - x_{ 0 } } \leq h_{ 1 }$. \\


% ############################






% ####################
\section{ % Autor i tytuł dzieła
  Władimir Igoriewicz Arnold \\
  ,,Równania różniczkowe zwyczajne'', \cite{} }


Błędy
\begin{center}
  \begin{tabular}{|c|c|c|c|c|}
    \hline
    & \multicolumn{2}{c|}{} & & \\
    Strona & \multicolumn{2}{c|}{Wiersz} & Jest
                              & Powinno być \\ \cline{2-3}
    & Od góry & Od dołu & & \\
    \hline
    11  & 17 & & mechanice klasycznej & mechanice kwantowej \\
    15  & & 16 & rozdziale 6 & rozdziale 5 \\
    28  & 15 & & wzór (8) & wzór \\
    34  &  9 & & $\dot{ x }_{ 1 } = x_{ 2 }$ & $\dot{ x }_{ 1 } = x_{ 1 }$ \\
    47  & & 12 & $x_{ i } = \varphi_{ i }( x_{ 1 }, \ldots, x_{ n } )$
           & $x_{ i } = \varphi_{ i }( y_{ 1 }, \ldots, y_{ n } )$ \\
    53  & & 14 & obrót & obrót krzywych całkowych \\
    56  & 12 & & osobliwym & nieosobliwym \\
    61  &  7 & & $\vecxBold$, $\boldsymbol{\alpha}_{ 0 }$
           & $\vecxBold$, $\boldsymbol{\alpha}$ \\
    % 64  & & 11 & $\vecgBold( t_{ 2 }, t_{ 1 }, \vecxBold )
    %             = \vecgBold^{ t_{ 2 } }_{ t_{ 1 } }( \vecxBold, t_{ 1 } )$
    %        & $\gbf^{ t_{ 2 } }_{ t_{ 1 } }( \xbf, t_{ 1 } )
    %          = \gbf( t_{ 2 }, t_{ 1 }, \xbf )$ \\
    % 64  & & 10 & $( \vpb( t ), t )$ & $( t, \vpb( t ) )$ \\
    % 66  & 17 & & $\vbf( t, \xbf, \dot{ \alb } )$
    %        & $\vbf( t, \xbf, \alb )$ \\
    % 70  & & 13 & $\pdot_{ i } = \pd{ }{ H }{ q_{ i } }$
    %        & $\pdot_{ i } = -\pd{ }{ H }{ q_{ i } }$ \\
    % 71  & & 15 & $\frac{ \partial \vbf_{ 0 } }{ \xbf }$
    %        & $\pd{}{ \vbf_{ 0 } }{ \xbf }$ \\
    % 72  &  4 & & ,,niezaburzonego'' & ,,zaburzonego'' \\
    % 73  &  5 & & $\xbf( 0$ & $\xbf( 0 )$ \\
    % 90  & &  3 & \emph{wraz z pochodną dla} $x = 0$ & \emph{dla} $x = 0$ \\
    % 92  &  3 & & $U( x( O ) )$ & $U( x( 0 ) )$ \\
    % 123 &  6 & & $^{ \R }A : \C^{ m } \to { }^{ \R }\C^{ n }$
    %        & ${ }^{ \R }A : { }^{ \R }\C^{ m } \to { }^{ \R }\C^{ n }$ \\
    % 125 &  5 & & $\mathbf{I}$ & $I$ \\
    % & & & & \\
    \hline
  \end{tabular}
\end{center}
\noindent
\StrWierszDol{66}{7} \\
\Jest \textit{tyłu do~brzegu} \\
\PowinnoByc \textit{tyłu nieograniczenie albo~do~brzegu} \\
\StrWierszGora{110}{9} \\
\Jest sumą częściową szeregu --~iloczynu \\
\PowinnoByc  jest sumą części wyrazów iloczynu \\





% ####################
\section{
  N. M. Matwiejew \\
  ,,Metody całkowania równana różniczkowych zwyczajnych'',
  \cite{MatwiejewMetodyCalkowaniaRownanRozniczkowychZwyczajnych82} }


Uwagi

\Str{15} Równanie (10) zostało bardzo elegancko wyprowadzone,
przy założeniu, że~funkcja uwikłana dana równaniem (9) spełnia
równanie różniczkowe (1), nie odpowiada to jednak na pytanie czy jeśli
funkcja $y( x )$ spełnia równanie (10), to spełnia też interesujące
nas równanie wyjściowe. Dowód tego twierdzenia jest następujący.
Załóżmy, że $\Phi_{ y }' \neq 0$, tak by było zapewnione istnienie
funkcji uwikłanej. Jeżeli teraz spełnione jest równanie (10), to można
je przekształcić do postaci:
\begin{equation}
  -\frac{ \Phi_{ x }' }{ \Phi_{ y }' } = f( x, y ).
\end{equation}
Lewa strona tej równości jest równa pochodnej funkcji uwikłanej
$y( x )$, określonej wzorem (9). \\
\Str{16}??? \\
\Str{31} ???


Błędy
\begin{center}
  \begin{tabular}{|c|c|c|c|c|}
    \hline
    & \multicolumn{2}{c|}{} & & \\
    Strona & \multicolumn{2}{c|}{Wiersz} & Jest
                              & Powinno być \\ \cline{2-3}
    & Od góry & Od dołu & & \\
    \hline
    5   & &  9 & Dodzimy & Dowodzimy \\
    5   & &  8 & potkowych & początkowych \\
    5   & &  7 & poąątkowych & początkowych \\
    10  & & 19 & damy & mamy \\
    15  & & & & \\
    % & & & & \\
    % & & & & \\
    \hline
  \end{tabular}
\end{center}

\noindent
\StrWierszGora{15}{13}
\Jest w~sensie ustępu \\
\PowinnoByc w~sensie zdefiniowanym w~ustępie \\
\StrWierszDol{20}{2} \\
\Jest i~nie ma rozwiązania określonego w~tym samym przedziale
nie~identycznego z~rozwiązaniem $y = y( x )$ chociażby w~jednym
punkcie przedziału $\absOne{ x - x_{ 0 } } \leq h$ różnym
od~punktu $x = x_{ 0 }$. \\
\PowinnoByc i~nie istnieje inne rozwiązanie określone w~przedziale
$\absOne{ x - x_{ 0 } } \leq h_{ 1 } \leq h$ które nie byłoby równe
rozwiązaniu $y = y( x )$ w~każdym punkcie przedziału
$\absOne{ x - x_{ 0 } } \leq h_{ 1 }$. \\

\VerSpaceTwo










% ####################################################################
% ####################################################################
% Bibliography

\printbibliography





% ############################
% End of the document

\end{document}
