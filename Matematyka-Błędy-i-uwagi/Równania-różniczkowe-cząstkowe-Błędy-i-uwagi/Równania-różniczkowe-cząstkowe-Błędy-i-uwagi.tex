% ------------------------------------------------------------------------------------------------------------------
% Basic configuration and packages
% ------------------------------------------------------------------------------------------------------------------
% Package for discovering wrong and outdated usage of LaTeX.
% More information to be found in l2tabu English version.
\RequirePackage[l2tabu, orthodox]{nag}
% Class of LaTeX document: {size of paper, size of font}[document class]
\documentclass[a4paper,11pt]{article}



% ------------------------------------------------------
% Packages not tied to particular normal language
% ------------------------------------------------------
% This package should improved spaces in the text
\usepackage{microtype}
% Add few important symbols, like text Celcius degree
\usepackage{textcomp}



% ------------------------------------------------------
% Polonization of LaTeX document
% ------------------------------------------------------
% Basic polonization of the text
\usepackage[MeX]{polski}
% Switching on UTF-8 encoding
\usepackage[utf8]{inputenc}
% Adding font Latin Modern
\usepackage{lmodern}
% Package is need for fonts Latin Modern
\usepackage[T1]{fontenc}



% ------------------------------------------------------
% Setting margins
% ------------------------------------------------------
\usepackage[a4paper, total={14cm, 25cm}]{geometry}



% ------------------------------------------------------
% Setting vertical spaces in the text
% ------------------------------------------------------
% Setting space between lines
\renewcommand{\baselinestretch}{1.1}

% Setting space between lines in tables
\renewcommand{\arraystretch}{1.4}



% ------------------------------------------------------
% Packages for scientific papers
% ------------------------------------------------------
% Switching off \lll symbol, that I guess is representing letter "Ł"
% It collide with `amsmath' package's command with the same name
\let\lll\undefined
% Basic package from American Mathematical Society (AMS)
\usepackage[intlimits]{amsmath}
% Equations are numbered separately in every section
\numberwithin{equation}{section}

% Other very useful packages from AMS
\usepackage{amsfonts}
\usepackage{amssymb}
\usepackage{amscd}
\usepackage{amsthm}

% Package with better looking calligraphy fonts
\usepackage{calrsfs}

% Package with better looking greek letters
% Example of use: pi -> \uppi
\usepackage{upgreek}
% Improving look of lambda letter
\let\oldlambda\Lambda
\renewcommand{\lambda}{\uplambda}




% ------------------------------------------------------
% BibLaTeX
% ------------------------------------------------------
% Package biblatex, with biber as its backend, allow us to handle
% bibliography entries that use Unicode symbols outside ASCII
\usepackage[
language=polish,
backend=biber,
style=alphabetic,
url=false,
eprint=true,
]{biblatex}

\addbibresource{Logika-i-teoria-mnogości-Bibliography.bib}





% ------------------------------------------------------
% Defining new environments (?)
% ------------------------------------------------------
% Defining enviroment "Wniosek"
\newtheorem{corollary}{Wniosek}
\newtheorem{definition}{Definicja}
\newtheorem{theorem}{Twierdzenie}





% ------------------------------------------------------
% Local packages
% You need to put them in the same directory as .tex file
% ------------------------------------------------------
% Package containing various command useful for working with a text
\usepackage{./Local-packages/general-commands}
% Package containing commands and other code useful for working with
% mathematical text
\usepackage{./Local-packages/math-commands}





% ------------------------------------------------------
% Package "hyperref"
% They advised to put it on the end of preambule
% ------------------------------------------------------
% It allows you to use hyperlinks in the text
\usepackage{hyperref}










% ------------------------------------------------------------------------------------------------------------------
% Title and author of the text
\title{Równania różniczkowe cząstkowe \\
  {\Large Błędy i~uwagi}}

\author{Kamil Ziemian}


% \date{}
% ------------------------------------------------------------------------------------------------------------------










% ######################################################################
\begin{document}
% ######################################################################





% ######################################
\maketitle % Tytuł całego tekstu
% ######################################





% ############################
\section{ % Autor i tytuł dzieła
  L.C. Evans \\
  \textit{Równania różniczkowe cząstkowe},
  \cite{EvansRowaniaRozniczoweCzastkowe2008}}

\vspace{0em}


% ##################
\CenterBoldFont{Uwagi do~konkretnych stron}

\vspace{0em}


\noindent
\Str{594}

\VerSpaceFour





\noindent
\Str{597} Nie rozumiem, dlaczego trzeba zakładać we~wzorze całkowania po
włóknach, że~$f$ jest ciągła. Wydaje~się, że~całkowalność i~mierzalność
wystarczają.

\VerSpaceFour






% ##################
\CenterBoldFont{Błędy}


\begin{center}

  \begin{tabular}{|c|c|c|c|c|}
    \hline
    Strona & \multicolumn{2}{c|}{Wiersz} & Jest
                              & Powinno być \\ \cline{2-3}
    & Od góry & Od dołu & & \\
    \hline
    19  &  4 & & równań & tych równań \\
    % & & & & \\
    % & & & & \\
    % & & & & \\
    % & & & & \\
    % & & & & \\
    % & & & & \\
    593 &  5 & & $a, b > 0$ & $a, b \in \Rbb$ \\
    597 &  1 & & \textit{współrzędnych) iloczyn}
           & \textit{współrzędnych)} \\
    597 &  2 & & $x_{ n } >$ & $x_{ n } =$ \\
    600 & & 4 & $x + h_{ i } e_{ i } \in U_{ \epsilon }$
           & $x + h_{ i } e_{ i } \in U$ \\
    %% & & & & \\
    601 & & 10 & $W \: \subset \subset U$ & $W \subset \subset U$ \\
    601 & &  5 & $W \: \subset \subset U$ & $W \subset \subset U$ \\
    614 &  9 & & $u \: \in H$ & $u \in H$\\
    615 & &  9 & miary & miary. \\
    % & & & & \\
    \hline
  \end{tabular}

\end{center}

\VerSpaceTwo









% ############################










% ############################
\newpage

\section{ % Autor i tytuł dzieła
  Lars H\"{o}rmander \\
  \textit{The Analysis of Linear Partial Differential Operators~I} \\
  \textit{Distributon Theory and~Fourier Analysis},
  \cite{HormanderAnalysisPartialDifferentialOperators1983}}


% ##################
\CenterBoldFont{Błędy}


\begin{center}

  \begin{tabular}{|c|c|c|c|c|}
    \hline
    Strona & \multicolumn{2}{c|}{Wiersz} & Jest
                              & Powinno być \\ \cline{2-3}
    & Od góry & Od dołu & & \\
    \hline
    6   &  5 & & $M >$ & $M =$ \\
    % & & & & \\
    \hline
  \end{tabular}

\end{center}

\VerSpaceTwo


\noindent
\StrWierszGora{1}{3} \\
\Jest every function is not differentiable. \\
\PowinnoByc not every function is differentiable. \\





% ############################










% ############################
\newpage

\section{ % Autorka i tytuł dzieła
  Hanna Marcinkowska \\
  \textit{Wstęp do~teorii równań różniczkowych cząstkowych},
  \cite{MarcinkowskaWstepRownanRozniczkowychCzastkowych1986}}

\vspace{0em}


% ##################
\CenterBoldFont{Uwagi do~konkretnych stron}


\noindent
\Str{15} Założenie, że~$\vectBold$ jest wektorem jednostkowym, nie
jest w~ogóle wykorzystane w~dowodzie i~jako takie należy je odrzucić. Wzór
\begin{equation}
  \label{eq:MarcinkowskaWDTRRCz-01}
  t_{ n } =
  \sum_{ j = 1 }^{ n - 1 } ( D_{ j } f\left( g_{ 1 }( s_{ 0 } ), \ldots,
  g_{ n - 1 }( s_{ 0 } ) \right) g_{ j }'( s_{ 0 } ),
\end{equation}
wynika nie z~warunku $\absOne{ \vectBold }$, lecz z~tego,
iż $g_{ j }'( s_{ 0 } ) = t_{ j }$, $j = 1, \ldots, n - 1$ i~z~tego, że krzywa
\begin{equation}
  \label{eq:MarcinkowskaWDTRRCz-02}
  I \ni s \mapsto [ g_{ 1 }( s ), \ldots, g_{ n - 1 }( s ),
  f( g_{ 1 }( s ), \ldots, g_{ n - 1 }( s ) ) ] \in \Rbb^{ n }
\end{equation}
leży na~powierzchni $\Sigma$.

\VerSpaceFour








% ##################
\CenterBoldFont{Błędy}


\begin{center}

  \begin{tabular}{|c|c|c|c|c|}
    \hline
    Strona & \multicolumn{2}{c|}{Wiersz} & Jest
                              & Powinno być \\ \cline{2-3}
    & Od góry & Od dołu & & \\
    \hline
    8   &  9 & & $| \vecaBold | \;\; | \vecbBold |$
           & $| \vecaBold | \, | \vecbBold |$ \\
    % & & & & \\
    % & & & & \\
    % & & & & \\
    % & & & & \\
    % & & & & \\
    \hline
  \end{tabular}

\end{center}

\VerSpaceTwo








% ############################










% ############################
\newpage

\section{P. Strzelecki \\
  \textit{Krótkie wprowadzenie do równań różniczkowych cząstkowych},
  \cite{StrzeleckiKrotkieWprowadzenieETC2006}}


% ##################
\CenterBoldFont{Uwagi}

Str. 117.???





% ##################
\CenterBoldFont{Błędy}


\begin{center}

  \begin{tabular}{|c|c|c|c|c|}
    \hline
    Strona & \multicolumn{2}{c|}{Wiersz} & Jest
                              & Powinno być \\ \cline{2-3}
    & Od góry & Od dołu & & \\
    \hline
    117 & & & $\exp\left( \frac{ 1 }{ 1 - | x |^{ 2 } } \right)$
           & $\exp\left( -\frac{ 1 }{ 1 - | x |^{ 2 } } \right)$ \\
           % & & & & \\
           % & & & & \\
           % & & & & \\
    \hline
  \end{tabular}

\end{center}

\VerSpaceTwo






% ############################










% ############################
\newpage

\section{Zofia Szmydt \\
  \textit{Transformacja Fouriera i~równania różniczkowe liniowe},
  \cite{SzmydtTransformacjaFourieraIRownaniaRozniczkoweLiniowe1972}}


% ##################
\CenterBoldFont{Błędy}


\begin{center}

  \begin{tabular}{|c|c|c|c|c|}
    \hline
    Strona & \multicolumn{2}{c|}{Wiersz} & Jest
                              & Powinno być \\ \cline{2-3}
    & Od góry & Od dołu & & \\
    \hline
    13  & 13 & & ta & \emph{ta} \\
    % & & & & \\
    % & & & & \\
    32  & &  3 & $\boldsymbol{ \{ } q_{ k } \}$ & $\{ q_{ k } \}$ \\
    % & & & & \\
    % & & & & \\
    % & & & & \\
    % & & & & \\
    % & & & & \\
    % & & & & \\
    % & & & & \\
    % & & & & \\
    \hline
  \end{tabular}

\end{center}

\VerSpaceTwo


\noindent
\StrWierszGora{308}{6} \\
\Jest Arytmetyka teoretyczna \\
\PowinnoByc \textit{Arytmetyka teoretyczna} \\




% ############################

































% ####################################################################
% ####################################################################
% Bibliography

\printbibliography





% ############################
% End of the document

\end{document}
