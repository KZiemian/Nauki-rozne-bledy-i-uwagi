% ------------------------------------------------------------------------------------------------------------------
% Basic configuration and packages
% ------------------------------------------------------------------------------------------------------------------
% Package for discovering wrong and outdated usage of LaTeX.
% More information to be found in l2tabu English version.
\RequirePackage[l2tabu, orthodox]{nag}
% Class of LaTeX document: {size of paper, size of font}[document class]
\documentclass[a4paper,11pt]{article}



% ------------------------------------------------------
% Packages not tied to particular normal language
% ------------------------------------------------------
% This package should improved spaces in the text
\usepackage{microtype}
% Add few important symbols, like text Celcius degree
\usepackage{textcomp}



% ------------------------------------------------------
% Polonization of LaTeX document
% ------------------------------------------------------
% Basic polonization of the text
\usepackage[MeX]{polski}
% Switching on UTF-8 encoding
\usepackage[utf8]{inputenc}
% Adding font Latin Modern
\usepackage{lmodern}
% Package is need for fonts Latin Modern
\usepackage[T1]{fontenc}



% ------------------------------------------------------
% Setting margins
% ------------------------------------------------------
\usepackage[a4paper, total={14cm, 25cm}]{geometry}



% ------------------------------------------------------
% Setting vertical spaces in the text
% ------------------------------------------------------
% Setting space between lines
\renewcommand{\baselinestretch}{1.1}

% Setting space between lines in tables
\renewcommand{\arraystretch}{1.4}



% ------------------------------------------------------
% Packages for scientific papers
% ------------------------------------------------------
% Switching off \lll symbol, that I guess is representing letter "Ł"
% It collide with `amsmath' package's command with the same name
\let\lll\undefined
% Basic package from American Mathematical Society (AMS)
\usepackage[intlimits]{amsmath}
% Equations are numbered separately in every section
\numberwithin{equation}{section}

% Other very useful packages from AMS
\usepackage{amsfonts}
\usepackage{amssymb}
\usepackage{amscd}
\usepackage{amsthm}

% Package with better looking calligraphy fonts
\usepackage{calrsfs}

% Package with better looking greek letters
% Example of use: pi -> \uppi
\usepackage{upgreek}
% Improving look of lambda letter
\let\oldlambda\Lambda
\renewcommand{\lambda}{\uplambda}




% ------------------------------------------------------
% BibLaTeX
% ------------------------------------------------------
% Package biblatex, with biber as its backend, allow us to handle
% bibliography entries that use Unicode symbols outside ASCII
\usepackage[
language=polish,
backend=biber,
style=alphabetic,
url=false,
eprint=true,
]{biblatex}

\addbibresource{Funkcje-rzeczywiste-i-zespolone-Bibliography.bib}





% ------------------------------------------------------
% Defining new environments (?)
% ------------------------------------------------------
% Defining enviroment "Wniosek"
\newtheorem{corollary}{Wniosek}
\newtheorem{definition}{Definicja}
\newtheorem{theorem}{Twierdzenie}





% ------------------------------------------------------
% Local packages
% You need to put them in the same directory as .tex file
% ------------------------------------------------------
% Package containing various command useful for working with a text
\usepackage{./Local-packages/general-commands}
% Package containing commands and other code useful for working with
% mathematical text
% \usepackage{math-commands}





% ------------------------------------------------------
% Package "hyperref"
% They advised to put it on the end of preambule
% ------------------------------------------------------
% It allows you to use hyperlinks in the text
\usepackage{hyperref}










% ------------------------------------------------------------------------------------------------------------------
% Title and author of the text
\title{Funkcje rzeczywiste i~zespolone \\
  {\Large Błędy i~uwagi}}

\author{Kamil Ziemian}


% \date{}
% ------------------------------------------------------------------------------------------------------------------










% ####################################################################
\begin{document}
% ####################################################################





% ######################################
\maketitle
% ######################################





% ######################################
\section{Antoni Wawrzyńczyk
  \textit{Współczesna teoria funkcji specjalnych},
  \parencite{Wawrzynczyk-Wspolczesna-teoria-funkcji-specjalnych-Pub-1978}}

% \vspace{0em}
% ######################################


% % ##################
% \CenterBoldFont{Uwagi do~konkretnych stron}

% \vspace{0em}


% \noindent
% \Str{8-9} Jest tutaj mowa o~tym, że zbiór z~powtórzeniami może
% zawierać, przykładowo, $s$ nierozróżnialnych elementów $a$. Stwierdzenie,
% że~mogą istnieć matematyczne obiekty nierozróżnialne, ale jednak stanowiące
% osobne byty, prosi~się o~głębszą filozoficzną analizę.

% By to zilustrować, rozpatrzmy następujący eksperyment myślowy. Rozpatrzymy
% często pojawiający się w~zadaniach problem, dwóch nierozróżnialnych kul
% o~tym samym kolorze, którym w~naszym przypadku jest kolor czerwony. By
% uczynić ten przedmiot naszych rozważań bardziej „namacalnym”, przyjmijmy,
% że~mają one 10 cm średnicy. Takie kule można by ustawić na stole, jedną na
% jego lewej stronie, drugą na jego prawej, i~choć wciąż byłyby
% nierozróżnialne, to jednak widzielibyśmy, że~na stole stoją dwie kule,
% a~nie jedna. Moglibyśmy więc powiedzieć, że~„Na stole stoją dwie
% \textit{różne} kule.”, jest to bowiem normalny sposób używania języka
% polskiego do opisu takiej sytuacji. Widać więc, że~to nazewnictwo łatwo
% może prowadzić do terminologicznego zamieszania, bo teraz możemy powiedzieć
% „Na stole stoją dwie różne nierozróżnialne kule”.

% Termin „nierozróżnialny” należy więc rozumieć w~ten sposób, że~gdy ktoś
% w~świecie naszego eksperymentu myślowego podał nam jedną z~tych kul, to
% patrząc tylko na nią, nie bylibyśmy w~stanie powiedzieć, czy to jest kula
% która stała na prawej, czy na lewej stronie stołu. Wciąż jednak bylibyśmy
% w~stanie postawić ją obok drugiej kuli i~stwierdzić, że~mamy do czynienia
% z~dwoma osobnymi obiektami, nie jednym. To zagadnienie wymaga dalszej analizy
% filozoficznej.

% Przejdźmy teraz ze~świata eksperymentu myślowego, do świata fizycznego. Nie
% należy się spodziewać~się by istniały tu dwie, prawdziwie nierozróżnialne
% czerwone kule o~średnicy 10 cm. Na jednej może znajdować~się
% charakterystyczna plama w~jaśniejszym odcieniu czerwieni, charakterystyczna
% nierówność na powierzchni, czyniąca kulę „mniej kulistą”, etc. Możemy jednak
% mówić o~tym, że~dwie kule w~rzeczywistym świecie są nierozróżnialne,
% w~poniższym sensie. Jeśli dwie kule leżą jak poprzednio po lewej i~prawej
% stronie stołu, to jeśli zostanie nam ona podana, to w~danych warunkach nie
% będziemy w~stanie tylko poprzez oglądanie jej stwierdzić, czy stała ona po
% lewej czy po prawej stronie stołu.

% Przez „w~danych warunkach” rozumiem to, że,~przykładowo, choć jedna kula ma
% charakterystyczną plamę w~jaśniejszym kolorze czerwieni, to nie potrafimy
% jej dostrzec gołym okiem, potrzebowalibyśmy do tego jakiegoś urządzenia,
% choćby była nim lupa powiększająca. Stwierdzenie „w~danych warunkach”
% moglibyśmy uściślić i~omówić szerzej, jednak nie wydaje~się nam by było to
% potrzebne. W~większości przypadków jest bardzo proste do ustalenia, co
% naprawdę oznacza „nierozróżnialność dwóch obiektów w~danych warunkach”.

% Wróćmy teraz do świata z~naszego eksperymentu myślowego. Łatwo zauważyć,
% że~kule które rozważaliśmy nie musiały mieć ani koloru czerwonego, ani
% średnicy 10 cm, dlaczego więc poświęciliśmy czas na~doprecyzowanie tego?
% Zwróćmy uwagę na to, że~w~zadaniach o~losowaniu lub wybieraniu kul
% z~pudełka, często podaje się, iż~kule w~liczbie $n_{ 1 }$ ma dany kolor,
% $n_{ 2 }$ kul ma inny kolor, $n_{ 3 }$ jeszcze inny, etc., za to prawie nigdy
% nie podaje~się jaki rozmiar mają te kule. Wynika to z~tego, że~dla treści
% problemu ma znaczenie to, żeby kule można było podzielić na kilka
% grup wedle jakiejś konkretnej cechy i~tutaj kolor jest bardzo naturalnym
% wyborem. Z~punktów widzenia treści problemu równie dobrze można byłoby
% przyjąć, iż~wszystkie one mają kolor biały, a~na ich powierzchni jest jeden
% z~trzech wybranych symboli z~japońskiego sylabariusza hiragana. Dlaczego
% zamiast tego mówi~się o~kolorowych kulach, jest oczywiste.

% Można by również powiedzieć, że~kule dzielą~się na te o~średnicy 8~cm, 9~cm
% i~10 cm, ale również taka ewentualność przegrywa z~kulami w~różnych kolorach.
% Wobec takiego wyboru, rozmiar kul nie ma żadnego i~można go w~omawianiu
% zagadnienia pominąć.

% W~naszych rozważaniach sprecyzowaliśmy kolor kul, jak i~ich rozmiar, by
% zwrócić uwagę na często pomijane możliwości jakie dają nam światy
% z~eksperymentów myślowych, które mogą pomóc nam zrozumieć pewne filozoficzne
% i~pojęciowe zagadnienia, jakie wiążą~się z~rozważaniem takich rzeczy jak
% dwie nierozróżnialne kule. Jeśli bowiem, jak to często czyni~się w~zadaniach,
% rozważamy pudełko w~którym jest sześć nierozróżnialnych kul czerwonych
% i~pięć nierozróżnialnych kul zielonych, możemy też pomyśleć o~sytuacji,
% gdy dwie czerwone kule wyjmujemy z~pudełka i~kładziemy na stole. To zaś
% ostatnie, jak pokazaliśmy, może prowadzić do~rozważania problemów tego czym
% są dwa nierozróżnialne obiekty, które mimo nierozróżnialności, są wciąż
% dwoma osobnymi obiektami. Jak zaznaczyliśmy wcześniej, problem ten wymaga
% głębszej analizy filozoficznej, niż ta przedstawiona tutaj, jednak ponieważ
% nie wiemy jak można byłoby go rozwiązać, pozostaniemy przy tym co
% zostało do tej pory napisane, a~co powinno wystarczyć do zadowalającego
% zrozumienia tej książki.

% Na koniec zauważmy, że~w~mechanice kwantowej pojawiają się cząstki
% nierozróżnialne i~również to zagadnienie wymaga głębszej analizy
% filozoficznej, którą jednak należy przeprowadzić w~innym miejscu. Jeśli
% w~tej książce pojawi~się problem w~którym będziemy rozważać nierozróżnialne
% cząstki kwantowe, spróbujemy do tego tematu wrócić.

% \VerSpaceFour




% \noindent
% \textbf{Str.~9, wiersze 21--22.} W~tym miejscu użyto dość niezręcznego
% sformułowania, że~zbiór z~powtórzeniami będziemy „reprezentować jako ciąg
% jego elementów”, ujęty w~nawiasy okrągłe. W~matematyce słowo „ciąg” zwykle
% oznacza obiekt, w~którym kolejność elementów ma znaczenie, ciąg
% dwuelementowy $0, 1$, jest różny od ciągu $1, 0$. Jak zaś wynika z~treści
% książki, tak jak kolejność wyliczania elementów zbioru wewnątrz nawiasów
% $\{ \ldots \}$ nie ma znaczenia, tak samo nie ma znaczenia kolejność wyliczania
% elementów zbioru z~powtórzeniami wewnątrz nawiasów $( \ldots )$. W~szczególności, dla elementów $a$, $b$, $a \neq b$ zachodzą przykładowe równości.
% \begin{equation}
%   \label{eq:Lipski-Marek-01}
%   ( a, a, b, b, b ) = ( b, b, b, a, a ) = ( a, b, a, b, b ) =
%   ( a, b, b, b, a ).
% \end{equation}

% Zamiast słowa „ciąg” proponujemy użycie słowa „lista”. Słowo lista rzadko,
% jeśli w~ogóle, pojawia~się w~matematyce, zaś w~języku codziennym, gdy mówimy
% choćby o~„liście zakupów”, to wiemy, że~kolejność elementów na tej liście
% jest bez znaczenia.

% \VerSpaceFour



% ##################
\newpage

\CenterBoldFont{Błędy}


\begin{center}

  \begin{tabular}{|c|c|c|c|c|}
    \hline
    Strona & \multicolumn{2}{c|}{Wiersz} & Jest
                              & Powinno być \\ \cline{2-3}
    & Od góry & Od dołu & & \\
    \hline
    \hphantom{0}6 & & 13 & funkcji & teoria funkcji \\
    12 & & 18 & $S( n )\! : \hphantom{0} = S( X )$ & $S( n ) := S( X )$ \\
    13 & & \hphantom{0}5 & $\beta \, ( \HorSpaceOne \cdot \HorSpaceOne , \cdot )$
    & $\beta ( \HorSpaceOne \cdot \HorSpaceOne , \cdot )$ \\
    13 & & \hphantom{0}3 & $\beta \, ( \HorSpaceOne \cdot \HorSpaceOne , \cdot )$
    & $\beta ( \HorSpaceOne \cdot \HorSpaceOne , \cdot )$ \\
    % & & & & \\
    % & & & & \\
    % & & & & \\
    % & & & & \\
    % & & & & \\
    % & & & & \\
    % & & & & \\
    % & & & & \\
    % & & & & \\
    \hline
  \end{tabular}

\end{center}

\VerSpaceTwo


\noindent
\StrWierszDol{7}{4} \\

% ############################










% ####################################################################
% ####################################################################
% Bibliography

\printbibliography





% ############################
% End of the document

\end{document}
