% ------------------------------------------------------------------------------------------------------------------
% Basic configuration and packages
% ------------------------------------------------------------------------------------------------------------------
% Package for discovering wrong and outdated usage of LaTeX.
% More information to be found in l2tabu English version.
\RequirePackage[l2tabu, orthodox]{nag}
% Class of LaTeX document: {size of paper, size of font}[document class]
\documentclass[a4paper,11pt]{article}



% ------------------------------------------------------
% Packages not tied to particular normal language
% ------------------------------------------------------
% This package should improved spaces in the text
\usepackage{microtype}
% Add few important symbols, like text Celcius degree
\usepackage{textcomp}



% ------------------------------------------------------
% Polonization of LaTeX document
% ------------------------------------------------------
% Basic polonization of the text
\usepackage[MeX]{polski}
% Switching on UTF-8 encoding
\usepackage[utf8]{inputenc}
% Adding font Latin Modern
\usepackage{lmodern}
% Package is need for fonts Latin Modern
\usepackage[T1]{fontenc}



% ------------------------------------------------------
% Setting margins
% ------------------------------------------------------
\usepackage[a4paper, total={14cm, 25cm}]{geometry}



% ------------------------------------------------------
% Setting vertical spaces in the text
% ------------------------------------------------------
% Setting space between lines
\renewcommand{\baselinestretch}{1.1}

% Setting space between lines in tables
\renewcommand{\arraystretch}{1.4}



% ------------------------------------------------------
% Packages for scientific papers
% ------------------------------------------------------
% Switching off \lll symbol, that I guess is representing letter "Ł"
% It collide with `amsmath' package's command with the same name
\let\lll\undefined
% Basic package from American Mathematical Society (AMS)
\usepackage[intlimits]{amsmath}
% Equations are numbered separately in every section
\numberwithin{equation}{section}

% Other very useful packages from AMS
\usepackage{amsfonts}
\usepackage{amssymb}
\usepackage{amscd}
\usepackage{amsthm}

% Package with better looking calligraphy fonts
\usepackage{calrsfs}

% Package with better looking greek letters
% Example of use: pi -> \uppi
\usepackage{upgreek}
% Improving look of lambda letter
\let\oldlambda\Lambda
\renewcommand{\lambda}{\uplambda}




% ------------------------------------------------------
% BibLaTeX
% ------------------------------------------------------
% Package biblatex, with biber as its backend, allow us to handle
% bibliography entries that use Unicode symbols outside ASCII
\usepackage[
language=polish,
backend=biber,
style=alphabetic,
url=false,
eprint=true,
]{biblatex}

\addbibresource{Systemy-operacyjne-Bibliography.bib}





% ------------------------------------------------------
% Defining new environments (?)
% ------------------------------------------------------
% Defining enviroment "Wniosek"
\newtheorem{corollary}{Wniosek}
\newtheorem{definition}{Definicja}
\newtheorem{theorem}{Twierdzenie}





% ------------------------------------------------------
% Local packages
% You need to put them in the same directory as .tex file
% ------------------------------------------------------
% Package containing various command useful for working with a text
\usepackage{general-commands}
% Package containing commands and other code useful for working with
% mathematical text
\usepackage{math-commands}





% ------------------------------------------------------
% Package "hyperref"
% They advised to put it on the end of preambule
% ------------------------------------------------------
% It allows you to use hyperlinks in the text
\usepackage{hyperref}










% ------------------------------------------------------------------------------------------------------------------
% Title and author of the text
\title{Arytmetyka i~teoria liczb \\
  {\Large Błędy i~uwagi}}

\author{Kamil Ziemian}


% \date{}
% ------------------------------------------------------------------------------------------------------------------










% ####################################################################
\begin{document}
% ####################################################################





% ######################################
\maketitle % Tytuł całego tekstu
% ######################################





% ############################
\section{ % Autor i tytuł dzieła
  Jacek Gancarzewicz \\
  \textit{Arytmetyka}, \cite{GancarzewiczArytmetyka2000}}

\vspace{0em}


% ##################
\CenterBoldFont{Uwagi}

\vspace{0em}


\noindent
W~książce do oznaczenia najważniejszych zbiorów używa~się fontów
pogrubionych: $\mathbf{N}$, $\mathbf{Z}$, $\mathbf{Q}$, etc. W~tych
notatkach będziemy używali standardowych fontów „blackboard bold”: $\Nbb$,
$\Zbb$, $\Qbb$,~etc. Wyjątkiem będzie sytuacja, gdy w~danym miejscu
książki został użyty zły symbol, wtedy będziemy dążyć, by wygląd wzoru który
będziemy poprawiać, był maksymalnie podobny do oryginału w~książce.

\VerSpaceFour





% ##################
\CenterBoldFont{Uwagi do konkretnych stron}

\vspace{0em}


\noindent
\Str{6} Na tej stronie numery stron do których odnosi~się spis treści nie
leżą na jednej linii pionowej, a~powinny.

\VerSpaceFour





\noindent
\Str{7} Na tej stronie można zauważyć, że~w~tej książeczce odstępy po kropce
może być bardzo duży. Nie wiem czy jest to wynik użytych ustawień \LaTeX a,
takich jak włączenie lub nie opcji \texttt{\textbackslash frenchspacing},
czy jakiś innych czynników. Niezależnie od tego, tego typu konwekcji
i~błędów edytorskich nie będziemy zapisywać w~tych notatkach, czyniąc
wyjątek dla tych, które są jawnie błędne lub wyglądają bardzo źle.

\VerSpaceFour





\noindent
\Str{10} Na tej stronie możemy zaobserwować, że~w~książeczce użyta jest
konwencja wedle której cudzysłów zapisujemy jako ”cytowany tekst”, wydaje
mi~się, że~jest to francuska konwencja zapisu. W~naszej ocenie lepiej
wygląda stosowana w~Polsce konwencja zapisu: „cytowany tekst”. Z~tego
powodu, gdy będziemy poprawiać jakiś fragment tekstu, w~którym znajduje~się
cudzysłów, w~poprawionej wersji będziemy stosować tę drugą konwencję.

\VerSpaceFour





\noindent
\StrWierszDol{20}{8} Duży odstęp po tą linią to chyba błąd edytorski.

\VerSpaceFour





\noindent
\Str{20} Podane tu wyjaśnienie definicji rekurencyjnej jest niełatwe
w~zrozumieniu i~może nawet nie być poprawne. Warto by ją poprawić.

\VerSpaceFour





\noindent
\Str{23} Na tej stronie podany jest wzór
\begin{equation}
  \label{eq:Gancarzewicz-Arytmetyka-01}
  a_{ n } =
  \frac{ 1 }{ \sqrt{ 5 } }
  \Big\{ \Big( \frac{ 1 + \sqrt{ 5 } }{ 2 } \Big)^{ n }
  - \Big( \frac{ 1 - \sqrt{5} }{ 2 } \Big)^{ n } \Big\}
\end{equation}
nie wygląda najlepiej, ze względu na to jak wysokość nawiasów ma~się
do wysokości zawartych w~nich wyrażeń. Według mnie, znaczniej lepiej
wyglądały on, gdyby został zapisany w~następujący sposób
\begin{equation}
  \label{eq:Gancarzewicz-Arytmetyka-01}
  a_{ n } =
  \frac{ 1 }{ \sqrt{ 5 } }
  \left\{ \left( \frac{ 1 + \sqrt{ 5 } }{ 2 } \right)^{ n }
  - \left( \frac{ 1 - \sqrt{5} }{ 2 } \right)^{ n } \right\},
\end{equation}
ewentualnie
\begin{equation}
  \label{eq:Gancarzewicz-Arytmetyka-02}
  a_{ n } =
  \frac{ 1 }{ \sqrt{ 5 } }
  \left\{ \left( \tfrac{ 1 + \sqrt{ 5 } }{ 2 } \right)^{ n }
  - \left( \tfrac{ 1 - \sqrt{5} }{ 2 } \right)^{ n } \right\}.
\end{equation}

W~dalszych ciągu tych notatek, poza wyjątkowymi sytuacjami nie będziemy
odnosić~się do tego typu problemów typograficznych.

\VerSpaceFour





\noindent
\Str{23} W~przeprowadzanym tutaj wyprowadzaniu wzoru (2.6) pojawia się
następujący problem. Czy dopuszczamy by zmienne $a$ i~$b$ w~nim występujące
mogły przyjmować wartość $0$? Jeśli odpowiedź na to pytanie jest twierdząca,
to stajemy przed problemem tego, iż~we wzorze (2.6) pojawiają~się
człony $a^{ 0 }$ i~$b^{ 0 }$. Ponieważ jeśli $b = 0$ to poszukiwana zależność
redukuje się do wyrażenia $a^{ n }$ (analogicznie dla $a = 0$), proponuję by
przyjąć, iż rozważamy tylko sytuację, gdy~$a \neq 0$ i~$b \neq 0$.

\VerSpaceFour





\Str{25--26} W~opis trójkąta Pascala wkradła się pewna nieścisłość.
Mianowicie by
narysować pierwszą linię zawierającą tylko symbol $\binom{ 0 }{ 0 }$ musimy
wiedzieć
ile on wynosi, a~nie jest wcale jasne, czy jego wartość wynika
przeprowadzonych
do tej pory rozważań. Jeśli nie to należy przyjąć, że~z~definicji
\begin{equation}
  \label{Gancarzewicz-Arytmetyka-03}
  \binom{ 0 }{ 0 } := 1.
\end{equation}

Drugi problem dotyczy stwierdzenia, że~gdy znamy $n$-tą linię, to następną
$( n + 1 )$-szą linię budujemy tak, że~dodajemy po dwa kolejne współczynniki
z~poprzedniej linii. Bardziej poprawne byłoby stwierdzenie, iż~na początku
i~końcu $( n + 1 )$-szej linii zapisujemy wartość symboli
$\binom{ n + 1 }{ 0 } = \binom{ n + 1 }{ n + 1 } = 1$, pozostałe zaś
współczynniki otrzymujemy dodając dwa kolejne współczynniki z~linii $n$-tej,
tak jak pokazano to na rysunku. Kwestia tego, gdzie jest początek i~koniec
linii, gdzie musimy wpisać wartości symboli $\binom{ n + 1 }{ 0 }$
i~$\binom{ n + 1 }{ n + 1 }$ w~praktyce nie stanowi problemu, dlatego nie
będziemy~się w~ten problem zagłębiać.

\VerSpaceFour





\noindent
\Str{26} Przedstawiony tu rysunek z~trójkątem Pascala można by uczynić
ładniejszym na wiele różnych sposobów.

\VerSpaceFour





\noindent
\Str{41} Od tego miejsca rozróżnienie między cyfrą, a~liczbą składającą~się
z~jednej cyfry zaczyna nabierać specjalnego znaczenia, choć ta książeczka
nie poświęca temu wiele miejsca. Oczywiście, często dla wygody nie
rozróżnia~się tych dwóch bytów, warto jednak uprzednio przyjrzeć się temu
zagadnieniu dokładnie.

Cyfry to symbole z~pewnego ustalonego zbioru, które będziemy notować
w~nawiasach. Czyli „$0$” jest jedną spośród cyfr arabskich. Samym cyfrom
nie przypisujemy sensu liczbowego, sens ten posiadają dopiero ciągi cyfr
skonstruowane według odpowiednich reguł. Przykładowo, często za ciągi cyfr
reprezentujące liczby uważa~się tylko te które zaczynają~się, czytając od
lewej do prawej, od cyfry różnej od~„$0$”, z~wyjątkiem ciągu składającego
się tylko z~jednego elementu: $0$. Taki ciąg oznacza liczbę zero. Inny
przykładem liczby skonstruowanej z~cyfr arabskich jest $17$, która
jest dwuelementowym ciągiem cyfr „$1$”, „$7$”.

Wśród ciągów cyfr występują ciągi o~długości jeden, takie jak wspomniany już
ciąg $0$ lub $1$, istnieje jednak różnica między cyfrą „$1$” i~liczbą $1$.
Cyfra „$1$” to symbol który nie oznacza żadnej wartości liczbowej, podczas
gdy $1$ to jednoelementowy ciąg cyfr oznaczający liczbę jeden.

W~tych rozważaniach nie zajęliśmy~się problemem, czy dany poprawny ciąg
cyfr, powiedzmy $17$, \textit{jest} liczbą siedemnaście, czy
\textit{oznacza} liczbę siedemnaście? Jest to ważkie zagadnienie z~zakresu
filozofii matematyki i~zapewne innych działów ludzkiej wiedzy, lecz w~tym
momencie nasza ignorancja nie pozwala nam zabrać na jego temat głosu.

\VerSpaceFour





\noindent
\StrWierszGora{41}{11} Odwołanie do przypisu w~tej linii wygląda naprawdę
brzydko, ale nie wiem jak można byłoby to poprawić.

\VerSpaceFour





\noindent
\Str{42} Według mnie symbol $\mod$ oznaczający \textbf{równość dwóch liczb
  modulo $k$} wygląda znacznie lepiej, niż używany w~tej książce symbol
$mod$. Dlatego też w~tych notatkach zawsze będę~się posługiwał pierwszą
z~podanych wersji.

\VerSpaceFour





\noindent
\Str{43} Kiedy już wiemy, że istnieje jedna i~tylko jedna liczba całkowita
$k$, dla której zachodzi
\begin{equation}
  \label{eq:Gancarzewicz-Arytmetyka-04}
  b = k a + r,
\end{equation}
to możemy określić $r = b - k a$. tylko czy to dowodzi jednoznaczności $r$?
Jeśli nawet nie, to łatwo ten fakt udowodnić. Załóżmy, że~zachodzi
\begin{subequations}
  \begin{align}
    \label{eq:Gancarzewicz-Arytmetyka-05-A}
    b = k a + r_{ 1 }, \\
    \label{eq:Gancarzewicz-Arytmetyka-05-B}
    b = k a + r_{ 2 }.
  \end{align}
\end{subequations}
Od~razu otrzymujemy
\begin{equation}
  \label{eq:Gancarzewicz-Arytmetyka-06}
  k a + r_{ 1 } = k a + r_{ 2 } \quad \iff \quad r_{ 1 } = r_{ 2 }.
\end{equation}

\VerSpaceFour





\Str{43} Analiza problemu tego czy liczba $b$ jest podzielna przez
liczbę~$a$, który jest podana w~dalszej części książeczki, byłby prostsza,
gdyby w~tym miejscu zostało napisane jawnie, że~z~definicji podzielność
liczby $b$ przez liczbę $a$ jest równoważna warunkowi
\begin{equation}
  \label{eq:Gancarzewicz-Arytmetyka-07}
  b = 0 ( \mod a ).
\end{equation}
Ten zaś warunek jest równoważny
\begin{equation}
  \label{eq:Gancarzewicz-Arytmetyka-08}
  0 = b ( \mod a ).
\end{equation}

\VerSpaceFour





\noindent
\Str{47} Na tej stronie podana jest informacja, że~za poprawnie
skonstruowany ciąg cyfr, który reprezentuje liczbę uznaje~się ciąg, którego
pierwsza cyfra, czytając od lewej do prawej jest różna od „$0$”.
Dodaje też, że~niekiedy rezygnuje~się z~tego warunku. Pominięto jednak
uwagę, że~ciąg jednoelementowy $0$ zawsze uważa~się za poprawny ciąg cyfr,
reprezentujący liczbę zero.

\VerSpaceFour





\noindent
\Str{51} Na tej stronie Gancarzewicz pisze o~„alfabecie (staro-)chińskim”.
Wydaje mi~się, że~z~punktu widzenia językoznawstwa, nie jest to alfabet,
ale pewien system hieroglificzny. Ponieważ jest to jednak książka
do~matematyki, nie do językoznawstwa, możemy wybaczyć autorowi ewentualne
nieścisłości.

\VerSpaceFour

















% ##################
\newpage

\CenterBoldFont{Błędy}

\VerSpaceFive


\begin{center}

  \begin{tabular}{|c|c|c|c|c|}
    \hline
    Strona & \multicolumn{2}{c|}{Wiersz} & Jest
    & Powinno być \\ \cline{2-3}
    & Od góry & Od dołu & & \\
    \hline
    10  &  2 & & ” $p$ & „$p$ \\
    10  &  3 & & ” $p$ & „$p$ \\
    10  &  4 & & ” $p$ & „$p$ \\
    10  &  6 & & ” $p$ & „$p$ \\
    10  & 13 & & fałszywe, Są & fałszywe. Są \\
    12  & &  5 & $n_{ n }$ & $n_{ 0 }$ \\
    14  &  5 & & $x \in X \, , y \in Y$ & $x \in X, \, y \in Y$ \\
    15  &  4 & & $f( x' ) \, ,$ & $f( x' ),$ \\
    16  &  2 & & można) & można \\
    20  &  3 & & $a_{ n }$m, & $a_{ n }$, \\
    21  & 17 & & $25, \ldots \ldots$ & $25, \ldots$ \\
    21  & &  7 & $640, \ldots \ldots$ & $640, \ldots$ \\
    24  & 13 & & $b^{ i }$ & $b^{ i }{}'$ \\
    26  & &  6 & $T( 4 ), \ldots \ldots$ & $T( 4 ), \ldots$ \\
    27  &  1 & & $T( n )$ & $T( 2 )$ \\
    31  &  6 & & $( n - i + 1 )! ( n - i )$
    & $( n - i )! \, ( n - i + 1 )$ \\
    31  & &  8 & $T( n )$ & $T( 2 )$ \\
    33  &  7 & & $\left( \frac{ 1 - \sqrt{ 5 } }{ 2 } \right.$
    & $\left( \frac{ 1 - \sqrt{ 5 } }{ 2 } \right)$ \\
    41  &  9 & & $p a$ & $a \, p$ \\
    41  & &  1 & $a p$ & $a \, p$ \\
    41  & &  5 & $ba \, p = a \, ( b p )$ & $a \, p , c = a \, ( p \, c )$ \\
    42  &  7 & & $a \, p q$ & $a \, p \, q$ \\
    42  &  8 & & $p = \pm 1$ & $p = q = \pm 1$ \\
    42  &  7 & & $pq$ & $p \, q$ \\
    42  & 10 & & że & dowodzą,~że \\
    42  & &  3 & $a_{ 2 } ( mod \, k ) \, ,$ & $a_{ 2 } \, ( \mod \, k ),$ \\
    43  &  6 & & $( a_{ 2 } - b_{ 2 } )$ & $( a_{ 2 } + b_{ 2 } )$ \\
    43  &  9 & & $- a_{ 1 } b_{ 2 }$ & $- a_{ 2 } b_{ 2 }$ \\
    47  & &  1 & ( 5.1) & (5.1) \\
    48  &  2 & & $a_{ 0 }$ & $c_{ 0 }$ \\
    53  &  6 & & $\cdots 10^{ k } c_{ k }$ & $\cdots + 10^{ k } \, c_{ k }$ \\
    53  & &  3 & $, 10^{ i } c_{ i }$ & $10^{ i } \, c_{ i }$ \\
    56  & &  8 & $\displaystyle \sum_{ = 0 }^{ s }$
    & $\displaystyle \sum_{ i = 0 }^{ s }$ \\
    % & & & & \\
    % & & & & \\
    \hline
  \end{tabular}

\end{center}

\VerSpaceTwo


\StrWierszeDol{24}{4--5} \\
\Jest jeszcze raz zmienić wskaźnik sumacyjny, tym razem $i$ zamieniamy
na~$j$, \\
\PowinnoByc zmienić nazwę wskaźnika sumacyjnego z~$i$ na~$j$ \\
\textbf{Str. 49, trzecia kolumna tabelki, wiersz 7.} \\
\Jest $= 1632$ \\
\PowinnoByc $= 1642$ \\
\StrWierszeDol{54}{14} \\
\Jest \textit{są jedną z~następującyc par: $00$, $25$. $50$, $75$} \\
\PowinnoByc \textit{tworzą jedną z~następujących liczb: $00$, $25$, $50$,
  $75$} \\
% ############################










% ####################################################################
% ####################################################################
% Bibliography

\printbibliography





% ############################
% End of the document

\end{document}
