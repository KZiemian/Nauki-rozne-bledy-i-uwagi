% ---------------------------------------------------------------------
% Podstawowe ustawienia i pakiety
% ---------------------------------------------------------------------
\RequirePackage[l2tabu, orthodox]{nag} % Wykrywa przestarzałe i niewłaściwe
% sposoby używania LaTeXa. Więcej jest w l2tabu English version.
\documentclass[a4paper,11pt]{article}
% {rozmiar papieru, rozmiar fontu}[klasa dokumentu]
\usepackage[MeX]{polski} % Polonizacja LaTeXa, bez niej będzie pracował
% w języku angielskim.
\usepackage[utf8]{inputenc} % Włączenie kodowania UTF-8, co daje dostęp
% do polskich znaków.
\usepackage{lmodern} % Wprowadza fonty Latin Modern.
\usepackage[T1]{fontenc} % Potrzebne do używania fontów Latin Modern.



% ------------------------------
% Podstawowe pakiety (niezwiązane z ustawieniami języka)
% ------------------------------
\usepackage{microtype} % Twierdzi, że poprawi rozmiar odstępów w tekście.
\usepackage{graphicx} % Wprowadza bardzo potrzebne komendy do wstawiania
% grafiki.
\usepackage{verbatim} % Poprawia otoczenie VERBATIME.
\usepackage{textcomp} % Dodaje takie symbole jak stopnie Celsiusa,
% wprowadzane bezpośrednio w tekście.
\usepackage{vmargin} % Pozwala na prostą kontrolę rozmiaru marginesów,
% za pomocą komend poniżej. Rozmiar odstępów jest mierzony w calach.
% ------------------------------
% MARGINS
% ------------------------------
\setmarginsrb
{ 0.7in}  % left margin
{ 0.6in}  % top margin
{ 0.7in}  % right margin
{ 0.8in}  % bottom margin
{  20pt}  % head height
{0.25in}  % head sep
{   9pt}  % foot height
{ 0.3in}  % foot sep





% ------------------------------
% Często używane pakiety
% ------------------------------
% \usepackage{csquotes} % Pozwala w prosty sposób wstawiać cytaty do tekstu.
% \usepackage{xcolor} % Pozwala używać kolorowych czcionek (zapewne dużo
% % więcej, ale ja nie potrafię nic o tym powiedzieć).



% ------------------------------
% Pakiety do tekstów z nauk przyrodniczych
% ------------------------------
\let\lll\undefined % Amsmath gryzie się z językiem pakietami do języka
% polskiego, bo oba definiują komendę \lll. Aby rozwiązać ten problem
% oddefiniowuję tę komendę, ale może tym samym pozbywam się dużego Ł.
\usepackage[intlimits]{amsmath} % Podstawowe wsparcie od American
% Mathematical Society (w skrócie AMS)
\usepackage{amsfonts, amssymb, amscd, amsthm} % Dalsze wsparcie od AMS
% \usepackage{siunitx} % Do prostszego pisania jednostek fizycznych
\usepackage{upgreek} % Ładniejsze greckie litery
% Przykładowa składnia: pi = \uppi
\usepackage{slashed} % Pozwala w prosty sposób pisać slash Feynmana.
\usepackage{calrsfs} % Zmienia czcionkę kaligraficzną w \mathcal
% na ładniejszą. Może w innych miejscach robi to samo, ale o tym nic
% nie wiem.



% ------------------------------
% Pakiety których pliki *.sty mają być w tym samym katalogu co ten plik
% ------------------------------
\usepackage{latexgeneralcommands}
\usepackage{mathcommands}



% ---------------------------------------------------------------------
% Dodatkowe ustawienia dla języka polskiego
% ---------------------------------------------------------------------
\renewcommand{\thesection}{\arabic{section}.}
% Kropki po numerach rozdziału (polski zwyczaj topograficzny)
\renewcommand{\thesubsection}{\thesection\arabic{subsection}}
% Brak kropki po numerach podrozdziału



% ------------------------------
% Ustawienia różnych parametrów tekstu
% ------------------------------
\renewcommand{\baselinestretch}{1.1}

\renewcommand{\arraystretch}{1.4} % Ustawienie szerokości odstępów między
% wierszami w tabelach.



% ------------------------------
% Pakiet „hyperref”
% Polecano by umieszczać go na końcu preambuły
% ------------------------------
\usepackage{hyperref} % Pozwala tworzyć hiperlinki i zamienia odwołania
% do bibliografii na hiperlinki










% ---------------------------------------------------------------------
% Tytuł, autor, data
\title{Historia nauki \\
  {\Large Błędy i~uwagi}}

\author{Kamil Ziemian}


% \date{}
% ---------------------------------------------------------------------










% ####################################################################
% Początek dokumentu
\begin{document}
% ####################################################################



% ######################################
\maketitle % Tytuł całego tekstu
% ######################################



% ##################
\Work{ % Autor i tytuł dzieła
  Nicolas Bourbaki \\
  \textit{Elementy historii matematyki},
  \cite{BourbakiElementyHistoriiMatematyki1980}}

\vspace{0em}


% ##################
\CenterBoldFont{Uwagi}

\vspace{0em}


\noindent
\Str{63} Dziwne, że~autorzy formułują tu własność systemów
liczbowych pisząc, że~iloraz $b_{ n } / b_{ n - 1 }$ jest równy tej
samej liczbie~$b$, zamiast po~prostu stwierdzić,
iż~$b_{ n } = b \, b_{ n - 1 }$.

\vspace{\spaceFour}





\noindent
\Str{68} Autorzy piszą tu w~taki sposób, jakby Grecy
i~rachmistrzowie byli dwoma zupełnie różnymi grupami ludzi.
Najprawdopodobniej jednak rachmistrzowie sami byli Grekami.

\vspace{\spaceFour}





\noindent
\StrWd{243}{6} Mniejszy nawias okrągły jest za~duży, przez co źle wygląda.
W~tej chwili nie wiem jednak zmodyfikować tekst, by ten problem naprawić.





% ##################
\newpage

\CenterBoldFont{Błędy}


\begin{center}

  \begin{tabular}{|c|c|c|c|c|}
    \hline
    Strona & \multicolumn{2}{c|}{Wiersz} & Jest
                              & Powinno być \\ \cline{2-3}
    & Od góry & Od dołu & & \\
    \hline
    8   & 11 & & metodzie ” & metodzie” \\
    16  & & 17 & poprzednikom~($^{ 4 }$) & poprzednikom($^{ 4 }$) \\
    18  & &  3 & „implikuje”) ($^{ 1 }$) & „implikuje”)($^{ 1 }$) \\
    30  & & & 13) ($^{ 1 }$) & 13)($^{ 1 }$) \\
    37  & 15 & & [193c]),obejmujący & [193c]), obejmujący \\
    39  & &  7 & środka & środka” \\
    41  & 13 & & zdumieniu ($^{ 2 }$) & zdumieniu($^{ 2 }$) \\
    43  & 12 & & stwierdza & stwierdza też \\
    43  & &  9 & myśl & pomysł \\
    44  & & & analizy ($^{ 2 }$) & analizy($^{ 2 }$) \\
    44  & & 17 & chwili ($^{ 2 }$) & chwili($^{ 2 }$) \\
    45  & 17 & & 448) & 448)) \\
    48  & &  9 & potocznym & sformalizowanym \\
    60  & 18 & & est & jest \\
    61  & &  4 & matematyków,a & matematyków, a \\
    65  &  1 & & niepustego & niepustych \\
    70  &  2 & & zespolonych~($^{ 1 }$) & zespolonych($^{ 1 }$) \\
    70  &  8 & & wektorów~($^{ 2 }$) & wektorów($^{ 2 }$) \\
    70  & 15 & & równoważności~($^{ 3 }$) & równoważność($^{ 3 }$) \\
    71  &  3 & & pierwsza ($^{ 1 }$) & pierwsza($^{ 1 }$) \\
    71  & 15 & & skończonego ($^{ 2 }$) & skończonego($^{ 2 }$) \\
    72  &  1 & & jeszcze & wciąż \\
    72  &  3 & & grup”~($^{ 2 }$) & grup”($^{ 2 }$) \\
    72  &  9 & & abstrakcyjnej~($^{ 2 }$) & abstrakcyjnej($^{ 2 }$) \\
    72  & & 16 & 84) ($^{ 3 }$) & 84)($^{ 3 }$) \\
    73  & 10 & & wieku~($^{ 1 }$) & wieku($^{ 1 }$) \\
    73  & & 12 & roli; & roli, \\
    74  & 12 & & Jordana-H\"{o}ldera”~($^{ 1 }$)
    & Jordana-H\"{o}ldera”($^{ 1 }$) \\
    74  & 15 & & ]139b] & [139b] \\
    \hline
  \end{tabular}





  \newpage

  \begin{tabular}{|c|c|c|c|c|}
    \hline
    Strona & \multicolumn{2}{c|}{Wiersz} & Jest
                              & Powinno być \\ \cline{2-3}
    & Od góry & Od dołu & & \\
    \hline
    82  & &  9 & \textit{metafizycznego}”; & \textit{metafizycznego}”); \\
    83  & 12 & & Hamilton ($^{ 1 }$) & Hamilton($^{ 1 }$) \\
    84  & &  6 & a~m.~in. & m.~in. \\
    85  & & 16 & Jacobi ($^{ 1 }$) & Jacobi($^{ 1 }$) \\
    86  & & 12 & używali & używają \\
    89  & &  4 & niezależne, & niezależne; \\
    90  & & 12 & 193)) & 193) \\
    90  & &  4 & niekwadratowych ($^{ 1 }$) & niekwadratowych($^{ 1 }$) \\
    91  &  6 & & niewymierna ($^{ 1 }$) & niewymierna($^{ 1 }$) \\
    91  & & 17 & okręgi ($^{ 2 }$) & okręgi($^{ 2 }$) \\
    92  &  1 & & wiemy ($^{ 1 }$) & wiemy($^{ 1 }$) \\
    92  & 14 & & algebraicznych ($^{ 2 }$) & algebraicznych($^{ 2 }$) \\
    93  & 10 & & niewiele ($^{ 1 }$) & niewiele($^{ 1 }$) \\
    93  & 15 & & rachunkach ($^{ 2 }$) & rachunkach($^{ 2 }$) \\
    93  & & 17 & kopiować analogiczne wzory & szukać analogicznych
                                              wzorów \\
    93  & &  3 & skopiowany & wzorowany \\
    94  & 11 & & mianowicie ($^{ 1 }$) & mianowicie($^{ 1 }$) \\
    94  & &  9 & w~~którym & z~którego \\
    95  &  4 & & Algebrze & \textit{Algebrze} \\
    95  &  8 & & nowoczesnych($^{ 1 }$).Wreszcie
           & nowoczesnych($^{ 1 }$). Wreszcie \\
    95  & 10 & & trzeciego ($^{ 2 }$) & trzeciego($^{ 2 }$) \\
    95  & 16 & & dodatnie ($^{ 3 }$) & dodatnie($^{ 3 }$) \\
    95  & & 13 & ,meno de & „meno de \\
    96  & & 17 & 98) ($^{ 1 }$) & 98)($^{ 1 }$) \\
    99  &  2 & & danego ($^{ 1 }$) & danego($^{ 1 }$) \\
    99  &  3 & & \textit{metafizykę} ($^{ 2 }$)
           & \textit{metafizykę}($^{ 2 }$) \\
    99  &  6 & & wartości ($^{ 3 }$) & wartości($^{ 3 }$) \\
    100 & & 18 & pierwiastków ($^{ 1 }$) & pierwiastków($^{ 1 }$) \\
    101 &  1 & & Gaussa ($^{ 1 }$) & Gaussa($^{ 1 }$) \\
    101 &  7 & & pierwszego ($^{ 2 }$) & pierwszego($^{ 2 }$) \\
    \hline
  \end{tabular}





  \newpage

  \begin{tabular}{|c|c|c|c|c|}
    \hline
    Strona & \multicolumn{2}{c|}{Wiersz} & Jest
                              & Powinno być \\ \cline{2-3}
    & Od góry & Od dołu & & \\
    \hline
    102 &  8 & & Abel ($^{ 2 }$) & Abel($^{ 2 }$) \\
    103 & 10 & & równania) ($^{ 2 }$) & równania)($^{ 2 }$) \\
    106 &  1 & & topologicznej ($^{ 1 }$) & topologicznej($^{ 1 }$) \\
    106 & 12 & & koła ($^{ 2 }$) & koła($^{ 2 }$) \\
    106 & & 12 & zespolonych ($^{ 3 }$) & zespolonych($^{ 3 }$) \\
    106 & 16 & & [148a) & [148a] \\
    107 & & 13 & 23) ($^{ 2 }$) & 23)($^{ 2 }$) \\
    109 & 12 & & Euklidesa” ($^{ 1 }$) & Euklidesa”($^{ 1 }$) \\
    110 &  9 & & ([dane] & [dane] \\
    110 & 15 & & całkowitej ($^{ 1 }$) & całkowitej($^{ 1 }$) \\
    111 & 18 & & kwadratem) ($^{ 2 }$) & kwadratem)($^{ 2 }$) \\
    112 &  7 & & niewiadomych~~($^{ 1 }$) & niewiadomych($^{ 1 }$) \\
    113 &  8 & & 373) -- ; & 373); \\
    113 & 11 & & $p$-grup ($^{ 2 }$) & $p$-grup($^{ 2 }$) \\
    114 & 18 & & 174) ($^{ 1 }$) & 174)($^{ 1 }$) \\
    115 &  9 & & rzędu~~($^{ 1 }$) & rzędu($^{ 1 }$) \\
    115 & 18 & & później ($^{ 2 }$) & później($^{ 2 }$) \\
    116 &  1 & & wieku ($^{ 1 }$) & wieku($^{ 1 }$) \\
    116 & 15 & & jedności ($^{ 2 }$) & jedności($^{ 2 }$) \\
    117 &  2 & & [150b] ($^{ 1 }$) & [150b]($^{ 1 }$) \\
    118 & &  9 & algebrze ($^{ 1 }$) & algebrze($^{ 1 }$) \\
    118 & &  7 & topologia ($^{ 2 }$) & topologia($^{ 2 }$) \\
    119 &  1 & & jeszcze prawdziwe & prawdziwe \\
    120 & & 13 & 488) ($^{ 1 }$) & 488)($^{ 1 }$) \\
    121 &  5 & & algebraiczne ($^{ 1 }$) & algebraiczne($^{ 1 }$) \\
    121 &  3 & & $\Zbb[ i |$ & $\Zbb[ i ]$ \\
    122 &  3 & & Gaussa ($^{ 1 }$) & Gaussa($^{ 1 }$) \\
    123 & & 11 & Galoisa ”czynników  & Galoisa ” czynników \\
    % ???????????
    124 & & 10 & \textit{skończona} ($^{ 1 }$)
           & \textit{skończona}($^{ 1 }$) \\
    125 &  6 & & podstawowe”~~($^{ 1 }$) & podstawowe”($^{ 1 }$) \\
    \hline
  \end{tabular}





  \newpage

  \begin{tabular}{|c|c|c|c|c|}
    \hline
    Strona & \multicolumn{2}{c|}{Wiersz} & Jest
                              & Powinno być \\ \cline{2-3}
    & Od góry & Od dołu & & \\
    \hline
    126 & 18 & & 1880 ($^{ 1 }$) & 1880($^{ 1 }$) \\
    127 & &  6 & $\Zbb[ \sqrt{ -3 } ]$) ($^{ 1 }$)
           & $\Zbb[ \sqrt{ -3 } ]$)($^{ 1 }$) \\
    128 &  1 & & $\mathbb{Z}[ \theta, \theta' ]$ ($^{ 1 }$)
           & $\mathbb{Z}[ \theta, \theta' ]$($^{ 1 }$) \\
    129 &  8 & & postacią ($^{ 1 }$) & postacią($^{ 1 }$) \\
    135 &  2 & & topologicznych ($^{ 1 }$) & topologicznych($^{ 1 }$) \\
    138 &  8 & & \textit{prymarnego} ($^{ 1 }$)
           & \textit{prymarnego}($^{ 1 }$) \\
    138 & 10 & & wykazuje ($^{ 2 }$) & wykazuje($^{ 2 }$) \\
    138 & 11 & & pierścieniach ($^{ 3 }$) & pierścieniach($^{ 3 }$) \\
    142 &  5 & & algebraicznej ($^{ 1 }$) & algebraicznej($^{ 1 }$) \\
    142 & &  7 & wcześniej ($^{ 2 }$) & wcześniej($^{ 2 }$) \\
    143 & & 16 & lokalnemu ($^{ 2 }$) & lokalnemu($^{ 2 }$) \\
    144 & & 16 & bezwzględnych ($^{ 1 }$) & bezwzględnych($^{ 1 }$) \\
    146 &  3 & & najogólniejszych ($^{ 1 }$) & najogólniejszych($^{ 1 }$) \\
    146 &  7 & & normalizacji ($^{ 2 }$) & normalizacji($^{ 2 }$) \\
    149 & & 14 & sformułowań ($^{ 1 }$) & sformułowań($^{ 1 }$) \\
    150 &  6 & & (31)) ($^{ 1 }$) & (31))($^{ 1 }$) \\
    150 & 12 & & algebr ($^{ 2 }$) & algebr($^{ 2 }$) \\
    151 &  8 & & algebry ($^{ 1 }$) & algebry($^{ 1 }$) \\
    151 & & 14 & 274) ($^{ 2 }$) & 274)($^{ 2 }$) \\
    152 & 17 & & nilpotentnego ($^{ 1 }$) & nilpotentnego ($^{ 1 }$) \\
    154 &  3 & & dalej ($^{ 1 }$) & dalej ($^{ 1 }$) \\
    154 & 16 & & podstawowego ($^{ 2 }$) & podstawowego ($^{ 2 }$) \\
    155 &  1 & & 102) ($^{ 1 }$) & 102) ($^{ 1 }$) \\
    155 &  3 & & centrum ($^{ 2 }$) & centrum($^{ 2 }$) \\
    155 & &  7 & teorii ($^{ 3 }$) & teorii($^{ 3 }$) \\
    156 & &  9 & uwagi ($^{ 1 }$) & uwagi($^{ 1 }$) \\
    157 &  1 & & algebra ($^{ 1 }$) & algebra($^{ 1 }$) \\
    157 &  3 & & ówdzie ($^{ 2 }$) & ówdzie($^{ 2 }$) \\
    157 & 13 & & minimalnemu ($^{ 3 }$) & minimalnemu($^{ 3 }$) \\
    157 & 20 & & liniowej ($^{ 4 }$) & liniowej($^{ 4 }$) \\
    % 191 & & 11 & geometrycznej ($^{ 3 }$) & geometrycznej($^{ 3 }$) \\
    % 194 & 13 & & arytmetyczno-geometrycznej~~($^{ 1 }$)
    %        & arytmetyczno-geometrycznej($^{ 1 }$) \\
    % 196 & &  8 & rzeczywiste ($^{ 1 }$) & rzeczywiste($^{ 1 }$) \\
    % 199 &  9 & & ciągłość) ($^{ 1 }$) & ciągłość)($^{ 1 }$) \\
    % 201 &  8 & & wymiaru) ($^{ 1 }$) & wymiaru)($^{ 1 }$) \\
    % 202 &  9 & & płaszczyzny ($^{ 1 }$) & płaszczyzny($^{ 1 }$) \\
    % 202 & &  6 & 1764) ($^{ 2 }$) & 1764)($^{ 2 }$) \\
    % 203 &  8 & & algebry~~($^{ 1 }$) & algebry($^{ 1 }$) \\
    % 208 &  4 & & parazwarta ($^{ 1 }$) & parazwarta($^{ 1 }$) \\
    % 215 &  3 & & wierze ($^{ 1 }$) & wierze($^{ 1 }$) \\
    % % & & & & \\
    % 227 &  6 & & całkowaniu~~($^{ 1 }$) & całkowaniu($^{ 1 }$) \\
    % 242 &  3 & & 599)) & 599) \\
    % % & & & & \\
    % 252 & & 12 & Euler czasem nawet & nawet Euler czasem \\
    % % & & & & \\
    % 279 &  2 & & rozszerzenia~($^{ 1 }$) & rozszerzenia($^{ 1 }$) \\
    % % & & & & \\
    \hline
  \end{tabular}





  \newpage

  \begin{tabular}{|c|c|c|c|c|}
    \hline
    Strona & \multicolumn{2}{c|}{Wiersz} & Jest
                              & Powinno być \\ \cline{2-3}
    & Od góry & Od dołu & & \\
    \hline
    158 & & 17 & homologicznej ($^{ 1 }$) & homologicznej($^{ 1 }$) \\
    158 & & 15 & minimalnego ($^{ 2 }$) & minimalnego($^{ 2 }$) \\
    159 & 13 & & dydaktycznych ($^{ 1 }$) & dydaktycznych($^{ 1 }$) \\
    160 &  6 & & euklidesowego~~($^{ 1 }$) & euklidesowego($^{ 1 }$) \\
    160 & 14 & & odróżnienia ($^{ 2 }$) & odróżnienia($^{ 2 }$) \\
    162 &  2 & & 249)) & 259) \\
    162 &  6 & & wieku ($^{ 1 }$) & wieku($^{ 1 }$) \\
    162 & 12 & & przeciwnego ($^{ 2 }$) & przeciwnego($^{ 2 }$) \\
    164 &  1 & & niezmiennikiem ($^{ 1 }$) & niezmiennikiem($^{ 1 }$) \\
    164 &  5 & & oczywistego ($^{ 2 }$) & oczywistego($^{ 2 }$) \\
    166 &  4 & & 315) ($^{ 1 }$) & 315)($^{ 1 }$) \\
    167 & 14 & & algebraicznej ($^{ 1 }$) & algebraicznej($^{ 1 }$) \\
    167 & & 10 & 371) ($^{ 2 }$) & 371)($^{ 2 }$) \\
    168 & 10 & & U~~Ponceleta & U~Ponceleta \\
    168 & &  9 & rezultatów ($^{ 1 }$) & rezultatów($^{ 1 }$) \\
    169 &  3 & & 515) ($^{ 1 }$) & 515)($^{ 1 }$) \\
    170 &  5 & & jej ($^{ 1 }$) & jej($^{ 1 }$) \\
    171 & & 13 & podobieństw ($^{ 1 }$) & podobieństw($^{ 1 }$) \\
    173 & 14 & & niezmienników ($^{ 2 }$) & niezmienników($^{ 2 }$) \\
    182 & &  4 & stwarzając & dające \\
    187 & 12 & & poprzedników ($^{ 1 }$) & poprzedników($^{ 1 }$) \\
    187 & 14 & & odkrycia ($^{ 2 }$) & odkrycia($^{ 2 }$) \\
    188 & 16 & & jedności ($^{ 1 }$) & jedności($^{ 1 }$) \\
    189 & & 18 & wielkość ($^{ 2 }$) & wielkość($^{ 2 }$) \\
    190 &  6 & & wygodnej ($^{ 1 }$) & wygodnej($^{ 1 }$) \\
    191 &  4 & & niewymierność ($^{ 1 }$) & niewymierność($^{ 1 }$) \\
    191 & & 11 & geometrycznej ($^{ 3 }$) & geometrycznej($^{ 3 }$) \\
    194 & 13 & & arytmetyczno-geometrycznej~~($^{ 1 }$)
           & arytmetyczno-geometrycznej($^{ 1 }$) \\
    196 & &  8 & rzeczywiste ($^{ 1 }$) & rzeczywiste($^{ 1 }$) \\
    199 &  9 & & ciągłość) ($^{ 1 }$) & ciągłość)($^{ 1 }$) \\
    % 201 &  8 & & wymiaru) ($^{ 1 }$) & wymiaru)($^{ 1 }$) \\
    % 202 &  9 & & płaszczyzny ($^{ 1 }$) & płaszczyzny($^{ 1 }$) \\
    % 202 & &  6 & 1764) ($^{ 2 }$) & 1764)($^{ 2 }$) \\
    % 203 &  8 & & algebry~~($^{ 1 }$) & algebry($^{ 1 }$) \\
    % 208 &  4 & & parazwarta ($^{ 1 }$) & parazwarta($^{ 1 }$) \\
    % 215 &  3 & & wierze ($^{ 1 }$) & wierze($^{ 1 }$) \\
    % % & & & & \\
    % 227 &  6 & & całkowaniu~~($^{ 1 }$) & całkowaniu($^{ 1 }$) \\
    % 242 &  3 & & 599)) & 599) \\
    % % & & & & \\
    % 252 & & 12 & Euler czasem nawet & nawet Euler czasem \\
    % % & & & & \\
    % 279 &  2 & & rozszerzenia~($^{ 1 }$) & rozszerzenia($^{ 1 }$) \\
    % % & & & & \\
    \hline
  \end{tabular}





  \newpage

  \begin{tabular}{|c|c|c|c|c|}
    \hline
    Strona & \multicolumn{2}{c|}{Wiersz} & Jest
                              & Powinno być \\ \cline{2-3}
    & Od góry & Od dołu & & \\
    \hline
    % 158 & & 17 & homologicznej ($^{ 1 }$) & homologicznej($^{ 1 }$) \\
    % 158 & & 15 & minimalnego ($^{ 2 }$) & minimalnego($^{ 2 }$) \\
    % 159 & 13 & & dydaktycznych ($^{ 1 }$) & dydaktycznych($^{ 1 }$) \\
    % 160 &  6 & & euklidesowego~~($^{ 1 }$) & euklidesowego($^{ 1 }$) \\
    % 160 & 14 & & odróżnienia ($^{ 2 }$) & odróżnienia($^{ 2 }$) \\
    % 162 &  2 & & 249)) & 259) \\
    % 162 &  6 & & wieku ($^{ 1 }$) & wieku($^{ 1 }$) \\
    % 162 & 12 & & przeciwnego ($^{ 2 }$) & przeciwnego($^{ 2 }$) \\
    % 164 &  1 & & niezmiennikiem ($^{ 1 }$) & niezmiennikiem($^{ 1 }$) \\
    % 164 &  5 & & oczywistego ($^{ 2 }$) & oczywistego($^{ 2 }$) \\
    % 166 &  4 & & 315) ($^{ 1 }$) & 315)($^{ 1 }$) \\
    % 167 & 14 & & algebraicznej ($^{ 1 }$) & algebraicznej($^{ 1 }$) \\
    % 167 & & 10 & 371) ($^{ 2 }$) & 371)($^{ 2 }$) \\
    % 168 & 10 & & U~~Ponceleta & U~Ponceleta \\
    % 168 & &  9 & rezultatów ($^{ 1 }$) & rezultatów($^{ 1 }$) \\
    % 169 &  3 & & 515) ($^{ 1 }$) & 515)($^{ 1 }$) \\
    % 170 &  5 & & jej ($^{ 1 }$) & jej($^{ 1 }$) \\
    % 171 & & 13 & podobieństw ($^{ 1 }$) & podobieństw($^{ 1 }$) \\
    % 173 & 14 & & niezmienników ($^{ 2 }$) & niezmienników($^{ 2 }$) \\
    % 182 & &  4 & stwarzając & dające \\
    % 187 & 12 & & poprzedników ($^{ 1 }$) & poprzedników($^{ 1 }$) \\
    % 187 & 14 & & odkrycia ($^{ 2 }$) & odkrycia($^{ 2 }$) \\
    % 188 & 16 & & jedności ($^{ 1 }$) & jedności($^{ 1 }$) \\
    % 189 & & 18 & wielkość ($^{ 2 }$) & wielkość($^{ 2 }$) \\
    % 190 &  6 & & wygodnej ($^{ 1 }$) & wygodnej($^{ 1 }$) \\
    % 191 &  4 & & niewymierność ($^{ 1 }$) & niewymierność($^{ 1 }$) \\
    % 191 & & 11 & geometrycznej ($^{ 3 }$) & geometrycznej($^{ 3 }$) \\
    % 194 & 13 & & arytmetyczno-geometrycznej~~($^{ 1 }$)
    %        & arytmetyczno-geometrycznej($^{ 1 }$) \\
    % 196 & &  8 & rzeczywiste ($^{ 1 }$) & rzeczywiste($^{ 1 }$) \\
    % 199 &  9 & & ciągłość) ($^{ 1 }$) & ciągłość)($^{ 1 }$) \\
    201 &  8 & & wymiaru) ($^{ 1 }$) & wymiaru)($^{ 1 }$) \\
    202 &  9 & & płaszczyzny ($^{ 1 }$) & płaszczyzny($^{ 1 }$) \\
    202 & &  6 & 1764) ($^{ 2 }$) & 1764)($^{ 2 }$) \\
    203 &  8 & & algebry~~($^{ 1 }$) & algebry($^{ 1 }$) \\
    208 &  4 & & parazwarta ($^{ 1 }$) & parazwarta($^{ 1 }$) \\
    215 &  3 & & wierze ($^{ 1 }$) & wierze($^{ 1 }$) \\
    % & & & & \\
    227 &  6 & & całkowaniu~~($^{ 1 }$) & całkowaniu($^{ 1 }$) \\
    242 &  3 & & 599)) & 599) \\
    % & & & & \\
    252 & & 12 & Euler czasem nawet & nawet Euler czasem \\
    % & & & & \\
    279 &  2 & & rozszerzenia~($^{ 1 }$) & rozszerzenia($^{ 1 }$) \\
    % & & & & \\
    \hline
  \end{tabular}

\end{center}

\vspace{\spaceTwo}


\noindent
\StrWd{42}{17} \\
\Jest  już od~początku nie przestał  \\
\Powin od samego początku nie przestawał \\
\StrWg{81}{2--3} \\
\Jest  stworzył w~XVII wieku Desargues \\
\Powin stworzonej w~XVII wieku przez Desargues'a \\
\StrWg{81}{18} \\
\Jest  a~wkrótce pod~nazwą zasady dualności \\
\Powin wkrótce nazwana zasadą dualności \\
\StrWd{97}{5} \\
\Jest  w~tym czasie jeszcze \\
\Powin jeszcze w~tym czasie \\
\StrWd{121}{14} \\
\Jest  \textit{wielu innych liczb pierwszych} [niż 5] \\
\Powin \textit{wielu innych} [niż 5] \textit{liczb pierwszych} \\
\StrWg{122}{8} \\
\Jest  którą bez przerwy zajmować~się będzie prawie wyłącznie przez 25 lat \\
\Powin którą będzie zajmował~się prawie wyłącznym i~niemal bez przerwy
przez 25 lat \\
\StrWg{240}{10} \\
\Jest  pytanie takie postawić \\
\Powin postawić takie pytanie \\


% ############################










% % ############################
% \Work{ % Autor i tytuł dzieła
%   C. B. Boyer \\
%   „Historia rachunku różniczkowego i~całkowego \\
%   i~rozwój jego pojęć”,
%   \cite{BoyerHistoriaRachunkuRozniczkowegoICalkowego1964} }


% % ##################
% \CenterBoldFont{Uwagi}


% \start W~całej książce angielskie zwarte i~treściwe słowo „calculus”
% jest zastąpione długim polskim terminem „rachunek różniczkowy
% i~całkowy”, co często prowadzi do bardzo niezgrabnych stylistycznie
% zdań. Lepiej byłoby wprowadzi do książki, obok powyższego, termin
% „analiza matematyczna”, który można ładnie skrócić do „analizy”.

% \vspace{\spaceThree}





% % ##################
% \CenterBoldFont{Uwagi do konkretnych stron}


% \start \StrWd{18}{2} Umieszczenie w~tym samym zdaniu stwierdzenia
% o~ścisłym sformułowaniu analizy już u~jej początków oraz faktu,
% że~matematycy byli niewrażliwi na pewne subtelności, jest dość
% karkołomne. Nie wspominając już o~tym, że~te „subtelności” były
% często bardzo poważne.

% \vspace{\spaceFour}



% \start \StrWd{19}{18} Użyte tu określenie „mistycyzm imaginacyjnej
% spekulacji” jest wyraźnie niesprawiedliwe w~stosunku do metafizyki,
% najważniejszego działu filozofii. Nie~zmienia tego fakt, że~Boyer mógł
% mieć na myśli tylko transcendentalną metafizykę ze~szkoły Kanta.

% \vspace{\spaceFour}



% \start \Str{23} Stwierdzenie, że pewne podstawowe idea zostały
% usunięte z~analizy matematycznej, szerzej zaś, z~matematyki, są~mocno
% wątpliwe.

% \vspace{\spaceFour}



% \start \StrWd{26}{6} Nazwanie podanych wyżej pojęć „sztucznymi”,
% ciężko jest mi nazwać czymś innym, niż nieczułością na piękno
% matematyki.

% \vspace{\spaceFour}



% \start \Str{28} Ponieważ drugie wydanie tej książki ukazało~się w~1949~r.,
% autor nie mógł wiedzieć, że~w~latach 60 XX w., głównie za sprawą
% prac Abrahama Robinsona zostanie sformułowana analiza niestandardowa,
% oparta na ścisły pojęciu nieskończenie małych liczb.





% % ##################
% \CenterBoldFont{Błędy}


% \begin{center}

%   \begin{tabular}{|c|c|c|c|c|}
%     \hline
%     & \multicolumn{2}{c|}{} & & \\
%     Strona & \multicolumn{2}{c|}{Wiersz} & Jest
%                               & Powinno być \\ \cline{2-3}
%     & Od góry & Od dołu & & \\
%     \hline
%     29  & &  4 & [(376] & ([376] \\
%     42  & 13 & & [402 & [402] \\
%     % & & & & \\
%     % & & & & \\
%     \hline
%   \end{tabular}

% \end{center}

% \vspace{\spaceTwo}
% % ############################










% % ############################
% \Work{ % Redaktor i tytuł dzieła
%   Red. A.P.~Juszkiewicz \\
%   \textit{Historia matematyki. Tom III: Matematyka XVIII stulecia} \\
%   \cite{}}


% %% % ##################
% %% \CenterBoldFont{Uwagi}


% %% \start W~całej książce angielskie zwarte i~treściwe słowo „calculus”
% %% jest zastąpione długim polskim terminem „rachunek różniczkowy
% %% i~całkowy”, co często prowadzi do bardzo niezgrabnych stylistycznie
% %% zdań. Lepiej byłoby wprowadzi do książki, obok powyższego, termin
% %% „analiza matematyczna”, który można ładnie skrócić do „analizy”.

% %% \vspace{\spaceThree}





% %% % ##################
% %% \CenterBoldFont{Uwagi do konkretnych stron}


% %% \start \StrWd{18}{2} Umieszczenie w~tym samym zdaniu stwierdzenia
% %% o~ścisłym sformułowaniu analizy już u~jej początków oraz faktu,
% %% że~matematycy byli niewrażliwi na pewne subtelności, jest dość
% %% karkołomne. Nie wspominając już o~tym, że~te „subtelności” były
% %% często bardzo poważne.

% %% \vspace{\spaceFour}



% %% \start \StrWd{19}{18} Użyte tu określenie „mistycyzm imaginacyjnej
% %% spekulacji” jest wyraźnie niesprawiedliwe w~stosunku do metafizyki,
% %% najważniejszego działu filozofii. Nie~zmienia tego fakt, że~Boyer mógł
% %% mieć na myśli tylko transcendentalną metafizykę ze~szkoły Kanta.

% %% \vspace{\spaceFour}



% %% \start \Str{23} Stwierdzenie, że pewne podstawowe idea zostały
% %% usunięte z~analizy matematycznej, szerzej zaś, z~matematyki, są~mocno
% %% wątpliwe.

% %% \vspace{\spaceFour}



% %% \start \StrWd{26}{6} Nazwanie podanych wyżej pojęć „sztucznymi”,
% %% ciężko jest mi nazwać czymś innym, niż nieczułością na piękno
% %% matematyki.

% %% \vspace{\spaceFour}



% %% \start \Str{28} Ponieważ drugie wydanie tej książki ukazało~się w~1949~r.,
% %% autor nie mógł wiedzieć, że~w~latach 60 XX w., głównie za sprawą
% %% prac Abrahama Robinsona zostanie sformułowana analiza niestandardowa,
% %% oparta na ścisły pojęciu nieskończenie małych liczb.





% % ##################
% \CenterBoldFont{Błędy}


% \begin{center}

%   \begin{tabular}{|c|c|c|c|c|}
%     \hline
%     Strona & \multicolumn{2}{c|}{Wiersz} & Jest
%                               & Powinno być \\ \cline{2-3}
%     & Od góry & Od dołu & & \\
%     \hline
%     264 & & 10 & jej własnymi środkami & środkami spoza niej samej \\
%     267 & &  6 & znał & znał go \\
%     283 & & 17 & drugiej. Maclaurin & drugiej, Maclaurin \\
%     % & & & & \\
%     % & & & & \\
%     % & & & & \\
%     % & & & & \\
%     % & & & & \\
%     \hline
%   \end{tabular}

% \end{center}

% \vspace{\spaceTwo}






% % ############################











% % ############################
% \Work{ % Autor i tytuł dzieła
%   R. Rhodes \\
%   „Jak powstała bomba atomowa”, \cite{Rhodes00} }


% % ##################
% \CenterBoldFont{Błędy}


% \begin{center}

%   \begin{tabular}{|c|c|c|c|c|}
%     \hline
%     & \multicolumn{2}{c|}{} & & \\
%     Strona & \multicolumn{2}{c|}{Wiersz} & Jest
%                               & Powinno być \\ \cline{2-3}
%     & Od góry & Od dołu & & \\
%     \hline
%     719 & &  6 & 1993 & 1933 \\
%     % & & & & \\
%     % & & & & \\
%     \hline
%   \end{tabular}

% \end{center}

% \vspace{\spaceTwo}
% % ############################










% % ############################
% \Work{ % Autor i tytuł dzieła
%   A. K. Wróblewski \\
%   „Historia fizyki”, \cite{Wroblewski06} }


% % ##################
% \CenterBoldFont{Błędy}


% \begin{center}

%   \begin{tabular}{|c|c|c|c|c|}
%     \hline
%     & \multicolumn{2}{c|}{Wiersz} & & \\ \cline{2-3}
%     Strona & Od góry & Od dołu & Jest & Powinno być \\
%     & (kolumna) & (kolumna) & & \\
%     \hline
%     203 & 3 (2) & & Jacob 'sGravesande'a & Jacob's Gravesande'a \\
%     % & & & & \\
%     \hline
%   \end{tabular}

% \end{center}

% \vspace{\spaceTwo}
% % ############################










% ####################################################################
% ####################################################################
% Bibliografia

\bibliographystyle{plalpha}

\bibliography{HistoryBooks}{}





% ############################

% Koniec dokumentu
\end{document}
