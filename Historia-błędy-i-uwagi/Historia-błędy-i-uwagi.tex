% Autor: Kamil Ziemian

% ---------------------------------------------------------------------
% Podstawowe ustawienia i pakiety
% ---------------------------------------------------------------------
\RequirePackage[l2tabu, orthodox]{nag} % Wykrywa przestarzałe i niewłaściwe
% sposoby używania LaTeXa. Więcej jest w l2tabu English version.
\documentclass[a4paper,11pt]{article}
% {rozmiar papieru, rozmiar fontu}[klasa dokumentu]
\usepackage[MeX]{polski} % Polonizacja LaTeXa, bez niej będzie pracował
% w języku angielskim.
\usepackage[utf8]{inputenc} % Włączenie kodowania UTF-8, co daje dostęp
% do polskich znaków.
\usepackage{lmodern} % Wprowadza fonty Latin Modern.
\usepackage[T1]{fontenc} % Potrzebne do używania fontów Latin Modern.



% ---------------------------------------
% Podstawowe pakiety (niezwiązane z ustawieniami języka)
% ---------------------------------------
\usepackage{microtype} % Twierdzi, że poprawi rozmiar odstępów w tekście.
\usepackage{textcomp} % Dodaje takie symbole jak stopnie Celsiusa,
% wprowadzane bezpośrednio w tekście.



% ---------------------------------------
% MARGINS
% ---------------------------------------
\usepackage{vmargin} % Pozwala na prostą kontrolę rozmiaru marginesów,
% za pomocą komend poniżej. Rozmiar odstępów jest mierzony w calach.
\setmarginsrb
{ 0.7in}  % left margin
{ 0.6in}  % top margin
{ 0.7in}  % right margin
{ 0.8in}  % bottom margin
{  20pt}  % head height
{0.25in}  % head sep
{   9pt}  % foot height
{ 0.3in}  % foot sep



% ---------------------------------------
% Często używane pakiety
% ---------------------------------------
% \usepackage{csquotes} % Pozwala w prosty sposób wstawiać cytaty do tekstu.
\usepackage{xcolor} % Pozwala używać kolorowych czcionek (zapewne dużo
% więcej, ale ja nie potrafię nic o tym powiedzieć).



% ---------------------------------------
% Pakiety których pliki *.sty mają być w tym samym katalogu co ten plik
% ---------------------------------------
\usepackage{latexgeneralcommands}



% ---------------------------------------
% Dodatkowe ustawienia dla języka polskiego
% ---------------------------------------
\renewcommand{\thesection}{\arabic{section}.}
% Kropki po numerach rozdziału (polski zwyczaj topograficzny)
\renewcommand{\thesubsection}{\thesection\arabic{subsection}}
% Brak kropki po numerach podrozdziału



% ---------------------------------------
% Ustawienia różnych parametrów tekstu
% ---------------------------------------
\renewcommand{\baselinestretch}{1.1}

\renewcommand{\arraystretch}{1.4} % Ustawienie szerokości odstępów między
% wierszami w tabelach.



% ---------------------------------------
% Pakiet „hyperref”
% Polecano by umieszczać go na końcu preambuły
% ---------------------------------------
\usepackage{hyperref} % Pozwala tworzyć hiperlinki i zamienia odwołania
% do bibliografii na hiperlinki










% ---------------------------------------------------------------------
% Tytuł, autor, data
\title{Historia \\
  {\Large Błędy i~uwagi}}

\author{Kamil Ziemian}


% \date{}
% ---------------------------------------------------------------------










% ####################################################################
% Początek dokumentu
\begin{document}
% ####################################################################





% ######################################
\maketitle  % Tytuł całego tekstu
% ######################################





% ######################################
\section{Syntezy historyczne}

\vspace{\spaceTwo}
% ######################################



% ############################
\Work{ % Autor i tytuł dzieła
  Paul Johnson \\
  \textit{Narodziny nowoczesności},
  \cite{JohnsonNarodzinyNowoczesnoci1995}}

\vspace{0em}


% ##################
\CenterBoldFont{Błędy}

\vspace{0em}


\begin{center}

  \begin{tabular}{|c|c|c|c|c|}
    \hline
    Strona & \multicolumn{2}{c|}{Wiersz} & Jest
                              & Powinno być \\ \cline{2-3}
    & Od góry & Od dołu & & \\
    \hline
    29  &  2 & & cali, członie & cali, o członie \\
    142 &  1 & & Barbaji & Barbajowi \\
    142 & & 14 & w nową operą & z nową operą \\
    345 & 15 & & XIX & XVIII \\
    345 & 18 & & od & na od \\
    409 &  8 & & sposób & nie sposób \\
    % & & & & \\
    % & & & & \\
    % & & & & \\
    \hline
  \end{tabular}





  % \begin{tabular}{|c|c|c|c|c|}
  %   \hline
  %   & \multicolumn{2}{c|}{} & & \\
  %   Strona & \multicolumn{2}{c|}{Wiersz} & Jest
  %   & Powinno być \\ \cline{2-3}
  %   & Od góry & Od dołu & & \\
  %   \hline
  %   %   & & & & \\
  %   %   & & & & \\
  %   %   & & & & \\
  %   %   & & & & \\
  %   %   & & & & \\
  %   %   & & & & \\
  %   %   & & & & \\
  %   %   & & & & \\
  %   %   & & & & \\
  %   %   & & & & \\
  %   %   & & & & \\
  %   %   & & & & \\
  %   %   & & & & \\
  %   %   & & & & \\
  %   %   & & & & \\
  %   %   & & & & \\
  %   %   & & & & \\
  %   %   & & & & \\
  %   %   & & & & \\
  %   %   & & & & \\
  %   %   & & & & \\
  %   %   & & & & \\
  %   %   & & & & \\
  %   %   & & & & \\
  %   %   & & & & \\
  %   %   & & & & \\
  %   %   & & & & \\
  %   %   & & & & \\
  %   %   & & & & \\
  %   %   & & & & \\
  %   %   & & & & \\
  %   %   & & & & \\
  %   %   & & & & \\
  %   %   & & & & \\
  %   %   & & & & \\
  %   %   & & & & \\
  %   %   & & & & \\
  %   %   & & & & \\
  %   \hline
  % \end{tabular}

\end{center}

\vspace{\spaceTwo}


% ############################










% ############################
\newpage

\Work{ % Autor i tytuł dzieła
  Conrad Totman \\
  \textit{Historia Japonii}, \cite{TotmanHistoriaJaponii2009}}

\vspace{0em}

% ##################
\CenterBoldFont{Uwagi}

\vspace{0em}


\start \Str{XXXII} Ponieważ do~regionu Tokai należą prowincje
o~numerach 27--34, więc należałoby~się spodziewać, że~do regionu Kinki
należą prowincję o~numerach 35--40. Zamiast tego stoją tam liczby
36--40, jednak nie wiem jak to poprawić. Podobny problem jest
z~regionami Kinki i~Shikoku.

\vspace{\spaceFour}





% \start

% \vspace{\spaceFour}


% \start \StrWd{23}{4}

% \vspace{\spaceFour}


% ##################
\CenterBoldFont{Błędy}


\begin{center}

  \begin{tabular}{|c|c|c|c|c|}
    \hline
    Strona & \multicolumn{2}{c|}{Wiersz} & Jest
                              & Powinno być \\ \cline{2-3}
    & Od góry & Od dołu & & \\
    \hline
    XV  & &  5 & trnskrypcji & transkrypcji \\
    % & & & & \\
    % & & & & \\
    % & & & & \\
    % & & & & \\
    % & & & & \\
    % & & & & \\
    % & & & & \\
    % & & & & \\
    % & & & & \\
    % & & & & \\
    % & & & & \\
    % & & & & \\
    % & & & & \\
    \hline
  \end{tabular}





  % \begin{tabular}{|c|c|c|c|c|}
  %   \hline
  %   & \multicolumn{2}{c|}{} & & \\
  %   Strona & \multicolumn{2}{c|}{Wiersz} & Jest
  %   & Powinno być \\ \cline{2-3}
  %   & Od góry & Od dołu & & \\
  %   \hline
  %   %   & & & & \\
  %   %   & & & & \\
  %   %   & & & & \\
  %   %   & & & & \\
  %   %   & & & & \\
  %   %   & & & & \\
  %   %   & & & & \\
  %   %   & & & & \\
  %   %   & & & & \\
  %   %   & & & & \\
  %   %   & & & & \\
  %   %   & & & & \\
  %   %   & & & & \\
  %   %   & & & & \\
  %   %   & & & & \\
  %   %   & & & & \\
  %   %   & & & & \\
  %   %   & & & & \\
  %   %   & & & & \\
  %   %   & & & & \\
  %   %   & & & & \\
  %   %   & & & & \\
  %   %   & & & & \\
  %   %   & & & & \\
  %   %   & & & & \\
  %   %   & & & & \\
  %   %   & & & & \\
  %   %   & & & & \\
  %   %   & & & & \\
  %   %   & & & & \\
  %   %   & & & & \\
  %   %   & & & & \\
  %   %   & & & & \\
  %   %   & & & & \\
  %   %   & & & & \\
  %   %   & & & & \\
  %   %   & & & & \\
  %   %   & & & & \\
  %   \hline
  % \end{tabular}

\end{center}

\vspace{\spaceTwo}


% \noi
% \StrWd{}{} \\
% \Jest \Pow
% \StrWg{}{} \\
% \Jest
% ############################










% ######################################
\newpage

\section{Dzieje świata lat 1789--1914}


\vspace{\spaceTwo}
% ######################################



% ############################
\newpage

\Work{ % Autor i tytuł dzieła
  James M. McPherson \\
  \textit{Battle Cry~of Freedom. Historia Wojny Secesyjnej},
  \cite{McPhersonBattleCryOfFreedom2016}}

\vspace{0em}

% ##################
\CenterBoldFont{Uwagi}

\vspace{0em}


\start W~książce angielskie określenie „Civil War” zostało
przełożone jako Wojna Secesyjna, jednak znacznie lepszym tłumaczeniem
jest „Wojna Domowa” i~jej będę w~razie potrzeby używał.

\vspace{\spaceFour}





\start W~języku częściej mówi~się o~„American Revolution”
niż~„American War~of Independence”, jednak w~Polsce przyjęło~się
błędnie przekładać oba te określenia jako „Wojna o~Niepodległość”.
Należałoby dosłownie tłumaczyć pierwsze określenie jako „Amerykańską
Rewolucję”, bądź „Rewolucję Amerykańską” i~tej wersji będę w razie
potrzeby używał.

\vspace{\spaceFour}





% ##################
\CenterBoldFont{Uwagi do konkretnych stron}


\start \StrWd{23}{4} Podane tu~stwierdzenie, że~Amerykanie w~1850~r.
zapełnili imperium nabyte w~1803 od~Napoleona wydaje~się mocno
przesadzone. Zauważmy, że~zgodnie z~tym co pisze dalej McPherson
populacja Stanów Zjednoczonych dopiero w~kilka lat później
prześcignęła Wielką Brytanię, której powierzchnia jest mniejsza niż
dzisiejszego stanu Michigan, i~tylko dwa i~pół razy większa od~tej
stanu Luizjany. Zakładając, że~mowa jest o~liczebności ludności
mieszkającej w~samej Wielkiej Brytanii, nie zaś liczonej razem
z~koloniami, pokazuje to jak małe było wtedy zaludnienie ogromnych
terenów Ameryki Północnej. Ciężko więc twierdzić, że~ziemie te zostały
zapełnione, trafniej byłoby powiedzieć, iż~zostały zaludnione.

% \vspace{\spaceFour}





% ##################
\CenterBoldFont{Błędy}

\vspace{\spaceFive}


\begin{center}

  \begin{tabular}{|c|c|c|c|c|}
    \hline
    Strona & \multicolumn{2}{c|}{Wiersz} & Jest
                              & Powinno być \\ \cline{2-3}
    & Od góry & Od dołu & & \\
    \hline
    8   & &  2 & rally ,round & rally, round \\
    14  & & 21 & Stowarzyszeń & „Stowarzyszeń \\
    17  & &  6 & sprzeczności.. & sprzeczności. \\
    23  & & 11 & niewolnictwo jednak & jednak niewolnictwo \\
    25  & & 13 & \textit{Encclopedia~od} & \textit{Encyclopedia~of} \\
    26  & 17 & & 1960~r. & 1860~r. \\
    30  &  7 & & 1854~r. & w~1854~r. \\
    33  & & 19 & codzienne & „codzienne \\
    40  & & 15 & społeczeństwie. & społeczeństwie”. \\
    % & & & & \\
    % & & & & \\
    % & & & & \\
    % & & & & \\
    % & & & & \\
    % & & & & \\
    % & & & & \\
    % & & & & \\
    % & & & & \\
    \hline
  \end{tabular}





  % \begin{tabular}{|c|c|c|c|c|}
  %   \hline
  %   & \multicolumn{2}{c|}{} & & \\
  %   Strona & \multicolumn{2}{c|}{Wiersz} & Jest
  %   & Powinno być \\ \cline{2-3}
  %   & Od góry & Od dołu & & \\
  %   \hline
  %   %   & & & & \\
  %   %   & & & & \\
  %   %   & & & & \\
  %   %   & & & & \\
  %   %   & & & & \\
  %   %   & & & & \\
  %   %   & & & & \\
  %   %   & & & & \\
  %   %   & & & & \\
  %   %   & & & & \\
  %   %   & & & & \\
  %   %   & & & & \\
  %   %   & & & & \\
  %   %   & & & & \\
  %   %   & & & & \\
  %   %   & & & & \\
  %   %   & & & & \\
  %   %   & & & & \\
  %   %   & & & & \\
  %   %   & & & & \\
  %   %   & & & & \\
  %   %   & & & & \\
  %   %   & & & & \\
  %   %   & & & & \\
  %   %   & & & & \\
  %   %   & & & & \\
  %   %   & & & & \\
  %   %   & & & & \\
  %   %   & & & & \\
  %   %   & & & & \\
  %   %   & & & & \\
  %   %   & & & & \\
  %   %   & & & & \\
  %   %   & & & & \\
  %   %   & & & & \\
  %   %   & & & & \\
  %   %   & & & & \\
  %   %   & & & & \\
  %   \hline
  % \end{tabular}

\end{center}

\vspace{\spaceTwo}


\noindent
\StrWd{31}{22} \\
\Jest  Podczas gdy nie odrzucali tezy \\
\Powin Nie~odrzucając tezy \\
\StrWg{35}{13} \\
\Jest  i~okresami bezrobocia, spowodowanymi recesją \\
\Powin i~okresów bezrobocia, spowodowanych recesją \\


% ############################










% ######################################
\newpage

\subsection{Dzieje świata po~1914~r.}


\vspace{\spaceThree}
% ######################################



% ############################
\Work{ % Autor i tytuł dzieła
  Martin Gilbert \\
  \textit{Pierwsza wojna światowa}, \cite{GilbertPierwszaWojnaSwiatowa2003}}

\vspace{0em}


% ##################
\CenterBoldFont{Uwagi}

\vspace{0em}


\start \StrWg{21}{1} Ponieważ książka ta został pierwotnie wydania
w~1994 roku, przynajmniej prawa autorskie Martina Gilberta zostały
wtedy zatwierdzone, a~data powstania „Wstępu” to~20~czerwca 1994~roku,
pojawia~się problem. Przytaczana wypowiedź z~wojny na~terenie Bośni,
nie mogła pochodzić z~26 grudnia 1996~roku. Najpewniej chodzi
tu~o~26~grudnia 1993 roku, ale~pewności mieć nie~mogę.

\vspace{\spaceFour}





\start \StrWg{70}{2} Nie wiem kto popełnił błąd, żołnierz, autor
czy~tłumacz, ale to zdanie o~martwym doboszu jest bez sensu.

\vspace{\spaceFour}





\start \StrWd{70}{5} To zdanie jest na~pewno źle przetłumaczone,
ale~nie~wiem jakie je poprawić.






% ##################
\CenterBoldFont{Błędy}

\vspace{\spaceFive}


\begin{center}

  \begin{tabular}{|c|c|c|c|c|}
    \hline
    Strona & \multicolumn{2}{c|}{Wiersz} & Jest
                              & Powinno być \\ \cline{2-3}
    & Od góry & Od dołu & & \\
    \hline
    68  & &  2 & Wielka Brytania & Rosja \\
    69  &  2 & & kulturowo & kulturowo'' \\
    69  & 16 & & 1 sierpnia & 12 sierpnia \\
    70  &  8 & & Pułk Feuchtingera, kiedy & Kiedy pułk Feuchtingera \\
    70  & &  4 & Sir Edward Gray, kiedy & Kiedy sir Edward Gray \\
    122 & & 10 & dal & dał \\
    177 & 15 & & „Walcie, aż~lufy pękną”. & <<Walcie, aż lufy pękną>>”. \\
           % & & & & \\
           % & & & & \\
           % & & & & \\
           % & & & & \\
    \hline
  \end{tabular}

\end{center}

\vspace{\spaceTwo}



% ############################










% ############################
\newpage

\Work{ % Autor i tytuł dzieła
  Paul Johnson \\
  \textit{Historia świata XX wieku, od~Rewolucji Październikowej} \\
  \textit{do~<<Solidarności>>. Tom~I},
  \cite{JohnsonHistoriaSwiataXXWiekuVolI2009}}

\vspace{0em}


% ##################
\CenterBoldFont{Uwagi}

\vspace{0em}


\start \Str{17--18} Terminy id, ego, superego nie zostały wprowadzone
przez Freuda, lecz przez jego tłumacza, bądź tłumaczy, na~język
angielski. Sam Freud używał zwykłych słów z~języka niemieckiego:
das~Es, Ich, Uberich. (Powinno to być omówione w książce Burzyńskiej
i~Markowskiego \cite{BM09}).

\vspace{\spaceFour}





\start \Str{56} Książka Keynesa \textit{Ekonomiczne konsekwencje
  pokoju}, nie mogła ukazać~się pod koniec 1917 roku.
Najprawdopodobniej chodzi tu o~koniec roku 1919. Powinno~się tu też
znaleźć obszerniejsze omówienie treści tej książki.

\vspace{\spaceFour}





\start \Str{63} Głosowanie nad traktatem o~którym tu mowa
nie~odbyło~się w~marcu 1919 roku, lecz w~marcu 1920 r.

\vspace{\spaceFour}





\start \Str{72} Wyrażoną tu opinię, że~Polska skorzystała z~obawy
Wielkiej Brytanii przed zalewem bolszewizmu, warto skonfrontować z~tym
co wielokrotnie mówił
\href{https://www.youtube.com/watch?v=yfQ7rpq_irA}{Andrzej Nowak} i~co
opisał w~„Pierwszej zdradzie zachodu”.

\vspace{\spaceFour}





\start \Str{90} Wydaje~się, że~opisany tu~ciąg przemówień Lenina
i~relacja Krupskiej o~tym jak położył~się spać bez słowa, dotyczą
wydarzeń z~jednego dnia, tego którego Lenin wrócił on do Rosji. Jeśli
to prawda powinno to zostać lepiej zaznaczone w~tekście, w~chwili
obecnej, nie jest to w~pełni jasne.

\vspace{\spaceFour}





\start \Str{189} Opis udziału niemieckich wojskowych w~zawieszeniu
przez Niemcy broni w~I~Wojnie Światowej, powinien być bardziej
wyczerpujący, w~tym momencie jest zbyt zwięzły, aby był jasny.

\vspace{\spaceFour}





\start \Str{189} W~tym miejscu po~raz pierwszy zostaje użyte
określenie „Alianci” na~członków Ententy. Jest to~chyba anachronizm,
którego nie powinno~się stosować jako, że~nazwa „Aliantów” jest
powszechnie przyjęta dla~sojuszu z~II, a~nie z~I, Wojny Światowej.

\vspace{\spaceFour}





\start \Str{223} Drugie zdanie drugiego paragrafu na~tej stronie jest
źle skonstruowane, nie wiem jednak jak je poprawić. Mimo tego, tego
jego sens jest jasny.

\vspace{\spaceFour}





\start \Str{232} Ludendorff spada tu jak z~nieba, aby zostać naczelnym
wodzem w~rządzie Hitlera, powołanym podczas puczu monachijskiego.
Warto byłoby napisać skąd on~się w~ogóle w~tym miejscu wziął.

\vspace{\spaceFour}





\start \Str{237--238} Tekst byłby znacznie bardziej logiczny, gdy
zamiast zdania „Poincar\'{e} manifestował arystokratyczną pogardę
dla~wulgarności klasy średniej i~francuskiego braku równowagi
emocjonalnej”, było „Poincar\'{e} manifestował arystokratyczną
pogardę dla~wulgarności klasy średniej i~francuski brak równowagi
emocjonalnej”.

\vspace{\spaceFour}





\start \Str{241} Stwierdzenie o~kurczącej~się populacji Francji jest
mało udane, bowiem jak zaraz potem Johnson wskazuje, populacja ta
w~rzeczywistości rosła. Problemem jest to, że~rosła ona bardzo słabo
w~porównaniu z~innymi państwami i~tym samym malał stosunek liczby
mieszkańców Francji do mieszkańców innych krajów w~Europie.

\vspace{\spaceFour}





\start \Str{262} Na tej stronie pojawia~się po~raz pierwszy
sformułowanie „teoria spisku”, którą lepiej byłoby zastąpić
przyjętym w~języku polskim terminem „teoria spiskowa”. Użycie
takiego słownictwa wynika zapewne z~tego, że~jeśli dobrze rozumiem,
książkę przetłumaczono jeszcze w~latach 80 XX~w., kiedy nie było
jeszcze w~języku polskim ustalonej nazwy na~to~zjawisko.

\vspace{\spaceFour}





\start \Str{288}

\vspace{\spaceFour}





\start \Str{304} W~2016~r. byłem na wykładzie na~temat chrześcijaństwa
w~Japonii w~XVI~wieku. Choć sam prowadzący przyznawał, że~w~tej
historii pewne kluczowe punkty są do~dziś niezrozumiałe, to
jednocześnie w~świetle wszystkich rzeczy o~jakich mówił, stwierdzenie,
że~chrześcijaństwo zostało odrzucone w~skutek kłótni misjonarzy, jest
gigantycznym uproszczeniem, a~może nawet wypaczeniem, historii.

\vspace{\spaceFour}





\start \Str{303} Czytając ten fragment odniosłem wrażenie, że~cesarz
Meiji był stary człowiekiem, gdy sprawował swą władzę,
w~rzeczywistości jednak objął formalnie panowanie, gdy miał 15 lat,
zmarł zaś w~wieku 60 lat. Jego następca cesarz Yoshihito urodził~się,
gdy miał on 27 lat, więc można wykluczyć wpływ wieku Meiji na stan
zdrowia jego następcy, co ten fragment mógł sugerować.

\vspace{\spaceFour}





\start \Str{305}

\vspace{\spaceFour}





\start \Str{310} Rok 1944 jako data przystąpienia Japonii do II~Wojny
Światowej jest błędny, od 1937~r. prowadziła już drugą wojnę
chińsko-japońską, zaś w~grudniu 1941 roku dokonała ataku na Stany
Zjednoczone. Jest to zapewne kolejna w~tej książce literówka, nie wiem
jednak jak~ją poprawić.

\vspace{\spaceFour}





\start \Str{345} Brak numeru strony w~prawy górnym rogu.

\vspace{\spaceFour}





\start \Str{345} Opisane tu wydarzenia, okupacja Korei przez
Japończyków i~ich reakcja na~sytuację w~Chinach powinny być
przedstawione szerzej. W~obecnej chwili przez swoją zwięzłość jest to
dosyć niejasne i~chaotyczne.

\vspace{\spaceFour}





\start \Str{454} Nie rozumiem dlaczego Stalin chował za~plecy prawą
rękę, skoro jego lewa była uszkodzona. Czy to dlatego, że~nie był
w~stanie schować lewej ręki za~plecami?

\vspace{\spaceFour}





\start \Str{457} W~tekście brak odwołania do~przypisu 11.

\vspace{\spaceFour}





\start \Str{459}

\vspace{\spaceFour}





\start \Str{465}

\vspace{\spaceFour}





\start \Str{477--478}





% ##################
\newpage

\CenterBoldFont{Błędy}

\vspace{\spaceFive}


\begin{center}

  \begin{tabular}{|c|c|c|c|c|}
    \hline
    Strona & \multicolumn{2}{c|}{Wiersz} & Jest
                              & Powinno być \\ \cline{2-3}
    & Od góry & Od dołu & & \\
    \hline
    15  & 17 & & Mendla & prac Mendla \\
    28  &  8 & & cywilizacyjne & cywilizowane \\
    36  & 10 & & zastąpić & zaspokoić \\
    51  & & & Jedyny & Jeden \\ % Dokończ.
    54  & 11 & & [dotyczących planu) & [dotyczących planu] \\
    & & & ] & \\
    55  & &  1 & M. Keynes & J. M. Keynes \\
    64  &  2 & & kształt ów & ów kształt \\
    65  & & 14 & późniejszy doradca & doradca \\
    68  &  5 & & szśćdziesiątych & sześćdziesiątych \\
    82  &  6 & & Ghandi: & Ghandi. \\
    89  &  5 & & odbyć & przebiegać \\
    111 & 12 & & Ogó1norosyjski & Ogólnorosyjski \\
    138 & & 10 & socjalrewolucjonistom & socjalrewolucjoniści \\
    142 & &  8 & poszczegó1nymi & poszczególnymi \\
    154 & 11 & & spadl & spadł \\
    155 & & 15 & przemysł kluczowy & kluczowe gałęzie przemysłu \\
    160 & 14 & & „wolność prac” & <<wolność pracy>>” \\
    161 & &  5 & upijać” & upijać”. \\
    185 & & 16 & można & nie~można \\
    190 & 17 & & więc równe & równe \\
    221 &  3 & & był & nie był \\
    243 &  8 & & nie & nie została \\
    248 & &  3 & 1853 & 1853 -- przy. red.] \\
    252 & &  4 & Enqu\'{e}te sur la~monarchie
           & \textit{Enqu\'{e}te sur la~monarchie} \\
    259 & 14 & & Algierczyków & Algierczykom \\
    281 & &  2 & red. & red.] \\
    292 & 15 & & Jesteśmy & „Jesteśmy \\
    314 & &  6 & Korupcja & korupcja \\
    336 &  6 & & wiec & więc \\
    351 &  5 & & podejrzanego & „podejrzanego \\
    382 & 11 & & wzrosty & wzrosły \\
    392 & 10 & & dały & dawały \\
    393 & &  2 & Steffens,{ }{ }\textit{Individualism}
           & Steffens, \textit{Individualism} \\
    408 & & 13 & problem6w & problemów \\
    423 &  9 & & ulega kwestii & podlega dyskusji \\
    426 & &  7 & zalegle & zaległe \\
    \hline
  \end{tabular}





  \begin{tabular}{|c|c|c|c|c|}
    \hline
    & \multicolumn{2}{c|}{} & & \\
    Strona & \multicolumn{2}{c|}{Wiersz} & Jest
                              & Powinno być \\ \cline{2-3}
    & Od góry & Od dołu & & \\
    \hline
    427 &  4 & & Ministerstwa. Zdrowia & Ministerstwa Zdrowia \\
    436 &  1 & & wielkim & „wielkim \\
    436 & & 13 & to jedynie & jedynie \\
    444 &  5 & & zaspokoić & uspokoić \\
    445 & & 10 & nie żądające & żądające \\
    449 &  9 & & Białym. Domu & Biały Domu \\
    457 &  9 & & Problemy & „Problemy \\
    461 & & 10 & miłosierny!$^{ 20 }$ & miłosierny!”$^{ 20 }$. \\
    % & & & & \\
    % & & & & \\
    % & & & & \\
    \hline
  \end{tabular}





  % \begin{tabular}{|c|c|c|c|c|}
  %   \hline
  %   & \multicolumn{2}{c|}{} & & \\
  %   Strona & \multicolumn{2}{c|}{Wiersz} & Jest
  %   & Powinno być \\ \cline{2-3}
  %   & Od góry & Od dołu & & \\
  %   \hline
  %   %   & & & & \\
  %   %   & & & & \\
  %   %   & & & & \\
  %   %   & & & & \\
  %   %   & & & & \\
  %   %   & & & & \\
  %   %   & & & & \\
  %   %   & & & & \\
  %   %   & & & & \\
  %   %   & & & & \\
  %   %   & & & & \\
  %   %   & & & & \\
  %   %   & & & & \\
  %   %   & & & & \\
  %   %   & & & & \\
  %   %   & & & & \\
  %   %   & & & & \\
  %   %   & & & & \\
  %   %   & & & & \\
  %   %   & & & & \\
  %   %   & & & & \\
  %   %   & & & & \\
  %   %   & & & & \\
  %   %   & & & & \\
  %   %   & & & & \\
  %   %   & & & & \\
  %   %   & & & & \\
  %   %   & & & & \\
  %   %   & & & & \\
  %   %   & & & & \\
  %   %   & & & & \\
  %   %   & & & & \\
  %   %   & & & & \\
  %   %   & & & & \\
  %   %   & & & & \\
  %   %   & & & & \\
  %   %   & & & & \\
  %   %   & & & & \\
  %   \hline
  % \end{tabular}

\end{center}

\vspace{\spaceTwo}


\noindent
\textbf{Okładka} \\
\Jest  ”Solidarności” \\
\Powin „Solidarności” \\
\Str{1} \\
\Jest  \textbf{Historia świata} \\
\Powin \textbf{Historia świata XX wieku} \\
\Str{3} \\
\Jest  \textbf{Historia świata} \\
\Powin \textbf{Historia świata XX wieku} \\


% ############################










% ######################################
\newpage

\subsection{Świat po~1945~r.}


\vspace{\spaceTwo}
% ######################################


% ##################
\Work{ % Autor i tytuł działa
  Tony Judt \\
  \textit{Powojnie. Historia Europy od~roku 1945}, \cite{JudtPowojnie2016}}

\vspace{0em}


% ##################
\CenterBoldFont{Uwagi}

\vspace{0em}


\start \Str{28} Stwierdzenie, że~aż do lat trzydziestych~XIX~w. babcie
hiszpańskie straszyły dzieci Napoleonem, jest trochę niezręczne. Wojny
napoleońskie skończyły~się dopiero w~1815 roku, więc chodzi tu
o~wydarzenia sprzed 25--40 lat, co nie wydaje~się obecnie zbyt długim
czasem, choć możliwe, że~w~XIX wieku taka długa pamięć była czymś
niezwykłym. Jeśli to właśnie Judy chciał przekazać, to można było
to~zdanie sformułować lepiej.

\vspace{\spaceFour}





\start \StrWd{102}{8} Nie potrafię zrozumieć co~w~tym kontekście miało
znaczyć zdanie „co~wyjątkowo miało~się okazać nieskuteczne”.

\vspace{\spaceFour}





\start \Str{111} Warto byłoby podać trochę więcej informacji o~zimie
roku 1947, aby~pozwolić czytelnikom poczuć jej siłę. Kilka zadań
podających dokładnie temperaturę panującą wtedy w~Europie i~czas przez
jaki~się utrzymywała, byłoby zupełnie wystarczające.

\vspace{\spaceFour}





\start \StrWd{125}{7} Należałoby podać, w~jakiej walucie~są wyrażone,
przedstawione tu wydatki.

\vspace{\spaceFour}





\start \Str{131} W~ostatnim akapicie na~tej stronie jest mowa
o~frontach ludowych i~narodowych w~taki sposób, że~nie można zrozumieć
o~co tak naprawdę chodzi.

\vspace{\spaceFour}





\start \StrWg{150}{20} Zdanie „zbliża~się czas wielkich zawirowań~--
a~tym samym konieczność określenia przez Związek Radziecki
wynikających z~tego korzyści” nie jest zbyt dobrze skonstruowane
i~trochę niezrozumiałe.

\vspace{\spaceFour}





\start \StrWg{159}{20} Zdanie „pozwoli Niemcom gnić, dopóki owoce
niemieckiej urazy i~beznadziei nie wpadną mu do~koszyka” nie jest
ani~zbyt jasne, ani~nie brzmi zbyt dobrze w~języku polskim.

\vspace{\spaceFour}





\start \StrWg{174}{18} Słowa Milovana Dżilasa warto byłoby opatrzyć
komentarzem. W~tym momencie ich brzmienie jest trochę dziwne, a~sens
niepewny.

\vspace{\spaceFour}





\start \StrWg{197}{8} W~tym wierszu jest mowa o~przeżyciu przez
Brytyjczyków Pierwszej~Wojny Światowej, ale bardzo możliwe, że~jest
to~błąd i~tak naprawdę chodzi o~Drugą.

\vspace{\spaceFour}





\start \Str{204} Przypis konsultanta wydania polskiego, który obecnie
znajduje~się na~końcu zdania w~wierszu 21 od~góry, powinienem
znajdować~się na końcu zdania w~wierszy drugim od~góry.

\vspace{\spaceFour}





\start \StrWg{258}{17} Z~kontekstu ciężko wywnioskować, kim było
„dwóch lewicowych członków ruchu oporu”. Ten fragment powinien być
poprawiony.

\vspace{\spaceFour}





\start \StrWd{282}{3} Na~końcu książki nie ma odniesienia do przypisu
z~tej linii.

\vspace{\spaceFour}





\start \StrWd{417}{19--18} O~Luckym Luku, czyli na polski Mającym
Szczęście Łukaszu, można stwierdzić wiele, ale~nie to,
że~jest~„nieszczęsny”. Również uznanie tego komiksu za belgijski,
jest dla mnie kontrowersyjne.

\vspace{\spaceFour}





\start \Str{474} Raymond Aron był zapewne całe życie antykomunistą,
ale~ponieważ w~jednym z~wydań \textit{Opium dla~intelektualistów}
napisał, że~książkę można traktować jako marksistowską krytykę pewnych
zjawisk, jego stosunek do~intelektualnego dziedzictwa marksizmu,
pozostaje sprawą otwartą.

\vspace{\spaceFour}





\start \Str{510} Choć „Dziady. Część III” zostały napisane po
Powstaniu Listopadowym, to jednak jego akcja rozgrywa~się kilka lat
przed tym wydarzeniem i~nie dotyczy losów powstańców, lecz losów ludzi
uciskanych przez władzę cara. Co~zresztą w~1968 roku również brzmiało
bardzo współcześnie. Należy zaznaczyć, że~trzeciej części „Dziadów”
nie można sprowadzić tylko do tego wątku, choć jest on jednym
z~najważniejszych.

\vspace{\spaceFour}





\start \StrWg{513}{7} W~1970 roku protesty o~szczególnych
konsekwencjach miały miejsce w~Szczecinie, nie można więc ich
ograniczać tylko do Gdańska.

\vspace{\spaceFour}





\start \StrWg{526}{14} Wydaje mi~się, że rozumiem sens zdania
o~optymistycznym zapatrzeniu w~postindustrialne wyobcowanie
i~bezduszność, ale według mnie nie powinno~się pisać w~taki przewrotny
i~skomplikowany sposób, aby czytelnik nie zgubił~się.

\vspace{\spaceFour}





\start \StrWg{556}{8} Nie rozumiem co miały znaczyć słowa
„dla~obojętnych skrajów ruchu robotniczego”.

\vspace{\spaceFour}





\start \Str{561} Nie wiem czy mogę~się w~pełni zgodzić ze
stwierdzeniem, że~Foucault był w~głębi duszy racjonalistą. Możliwe,
lecz zapewne była to specyficzna odmiana racjonalizmu.

\vspace{\spaceFour}





\start \StrWd{585}{18} Panków to dzielnica Berlina, gdzie
w~początkowym okresie istnienia NRD~znajdowały się rezydencje władz
tego kraju.

\vspace{\spaceFour}





\start \StrWg{602}{13} Na~końcu książki nie ma odniesienia do przypisu
z~tej linii.

\vspace{\spaceFour}





\start \StrWd{655}{16--14} Zdanie „przeczyć fundamentalnemu
powinowactwu demokratycznego państwa opiekuńczego (bez względu na~to,
jak bardzo niewystarczająco) z~kolektywistycznym planem komunizmu”
jest napisane w~niezrozumiały sposób.

\vspace{\spaceFour}





\start \StrWd{681}{4--3} Nie rozumiem co miało dokładnie znaczyć
zdanie „Epoka zastoju Leonida Breżniewa (Michaił Gorbaczow)
żywiła~się wieloma złudzeniami”.

\vspace{\spaceFour}





\start \StrWd{681}{2} W~Polsce przyjęła~się pisonia tego nazwiska
„Potiomkin” nie jak w~krajach anglojęzycznych „Potemkin”.

\vspace{\spaceFour}





\start \StrWd{693}{16} Na~końcu książki nie ma odniesienia do przypisu
z~tej linii.

\vspace{\spaceFour}





\start \Str{694} Bill Clinton miał 46 lat, gdy został prezydentem USA
w~1993 roku, był więc młodszy od Michaił Gorbaczow który miał 54, gdy
został w~1985 roku Sekretarzem Generalnym KPZR. Jednak nie można
twierdzić, że~Gorbaczow był młodszy od każdego prezydenta USA do
Clintona, bowiem najmłodszym prezydentem do roku 2017, jest Theodore
Roosevelt który miał tylko 42 lata, gdy~objął ten urząd w~1901 roku.

\vspace{\spaceFour}





\start \StrWd{703}{18} Słowo „niezrównany” brzmi trochę dziwnie
w~tym kontekście. Może powinno być „nierówny”.

\vspace{\spaceFour}





\start \StrWd{718}{2} Zwrot „bezwarunkowe niezrozumienie” jest
dziwny i~ciężki do zrozumienia. Po~angielsku brzmiał zapewne
„unconditional misunderstanding”, warto się zastanowić nad jego
lepszym tłumaczeniem.

\vspace{\spaceFour}





\start \StrWd{734}{19} NRD pojawia~się w~tym wersie dość
niespodziewanie, może chodziło o~Czechosłowację?

\vspace{\spaceFour}





\start \StrWd{742}{10} Na~końcu książki nie ma odniesienia do przypisu
z~tej linii.

\vspace{\spaceFour}





\start \StrWd{757}{6--5} Zdanie „którego rządzący byli
komunistycznymi satrapami, przejęli kontrolę nad tym obszarem”, brzmi
jakoś niezręcznie. Może dałoby~się je sformułować lepiej?

\vspace{\spaceFour}





\start \StrWd{782}{15} Powinno tu być wyjaśnione czym~są prawa
ciągnięcia.

\vspace{\spaceFour}





\start \StrWd{805}{9} Na~końcu książki nie ma odniesienia do przypisu
z~tej linii.

\vspace{\spaceFour}





\start \StrWd{823}{12-10} Zdanie „W~kraju było teraz więcej osób
mówiących po~holendersku niż~francusku (w~stosunku trzy do~dwóch),
które produkowały na~głowę mieszkańca i~zarabiał więcej.” źle brzmi
i~nie od razu zrozumiałe.

\vspace{\spaceFour}





\start \textbf{Wkładka~1, str.~6, u~góry.} Dla ułatwienia czytelnikom
orientacji, warto byłoby napisać, o~jaki akt stworzenia tu chodzi.

\vspace{\spaceFour}





\start \textbf{Wkładka~1, str.~6, u~dołu.} Cytowane są tu te same słowa
Clementa Attlee co na stronie~127, jednak te dwie wersje nie są
identyczne.





% ##################
\newpage

\CenterBoldFont{Błędy}

\vspace{\spaceFive}


\begin{center}

  \begin{tabular}{|c|c|c|c|c|}
    \hline
    Strona & \multicolumn{2}{c|}{Wiersz} & Jest
                              & Powinno być \\ \cline{2-3}
    & Od góry & Od dołu & & \\
    \hline
    22  & 16 & & w~nich żyli & żyli w~nich \\
    34  & 10 & & --supermani & --~supermani \\
    35  & & 17 & prądu & braku prądu \\
    35  & & 16 & dość & lecz dość \\
    58  & &  7 & postępującą & postępującej \\
    58  & &  6 & degeneracją & degeneracji \\
    99  & 10 & & to & za~to \\
    127 & 11 & & \textit{metody} & \textit{metody dozwolone} \\
    131 & 13 & & radziecką & bolszewicką \\
    185 & 19 & & być może & może \\
    199 & 11 & & przeregulowanej prawnie & prawnie przeregulowanej \\
    199 & 18 & & miastem & miasto \\
    203 & &  4 & Europy Środkowej & na~Europę Środkową \\
    264 & 13 & & Odstawiliśmy & „Odstawiliśmy \\
    356 & 16 & & męża & ojca \\
    380 & 12 & & k~o~n~t~r~rewolucji & k~o~n~t~r~r~e~w~o~l~u~c~j~i \\
    393 & 23 & & zwyczajowo tradycyjnie & zwyczajowo i~tradycyjnie \\
    409 & & 12 & manipulować. & manipulować''. \\
    409 & & 11 & dziesięcioleciu''. & dziesięcioleciu. \\
    430 & &  3 & specjalnością.: & specjalnością: \\
    457 & 15 & & (czy & czy \\
    457 & &  2 & wino. & wino''. \\
    478 & &  9 & dziewięćdziesiątych & sześćdziesiątych \\
    490 & & 14 & je! & je!'' \\
    509 & & 18 & uczeni & uczelni \\
    526 & &  5 & polityka & że~polityka \\
    555 & &  8 & się wydaje & wydaje~się \\
    564 &  9 & & and & i \\
    565 &  8 & & przeglądzie & w~przeglądzie \\
    582 & & 11 & Nemiec & Niemiec \\
    \hline
  \end{tabular}





  \begin{tabular}{|c|c|c|c|c|}
    \hline
    Strona & \multicolumn{2}{c|}{Wiersz} & Jest
                              & Powinno być \\ \cline{2-3}
    & Od góry & Od dołu & & \\
    \hline
    596 & &  4 & Andreas) & Andreas \\
    611 &  5 & & 1976 & 1975 \\
    676 &  1 & & zresztą nie była & nie była zresztą \\
    682 & 12 & & zawłaszczenie odśrodkowego & odśrodkowe zawłaszczenie \\
    691 & &  3 & a~nawet & nawet \\
    707 &  2 & & na nowo & nowe \\
    710 & 12 & & pierwszy hotel & hotel \\
    711 & 19 & & był & nie~był \\
    718 &  2 & & działania & pracy \\
    793 &  7 & & przez & wobec \\
    793 & &  8 & międzynarodową... & międzynarodową. \\
    793 & &  3 & nawet & on nawet \\
    798 & &  6 & walczyć & wlec~się \\
    815 & 15 & & \textit{Beatrice Webb (1925)} & Beatrice Webb (1925) \\
    838 &  3 & & w~związku z~tym & w~tym czasie \\
    937 & & 13 & Tabu & Różne tabu \\
    967 & & 11 & góry(Schuman & góry (Schuman \\
    976 &  6 & & praktycznie & w~praktyce \\
    977 & &  7 & wpław & wpłat \\
    \hline
  \end{tabular}

\end{center}

\vspace{\spaceTwo}


\noindent
\StrWd{110}{6} \\
\Jest  na~samą perspektywę pokoju \\
\Powin na~samą myśl o~perspektywie pokoju \\
\StrWg{417}{15} \\
\Jest  którymi~się w~nich chwalono \\
\Powin w~których~się nimi chwalono \\
\StrWg{555}{4} \\
\Jest  „nowy patriotyzm” za~granicą \\
\Powin „nowy patriotyzm” \\
\StrWd{617}{2} \\
\Jest  w~hiszpańskich przedsiębiorstwach \\
\Powin hiszpańskich przedsiębiorstw \\
\StrWd{698}{4} \\
\Jest  który rozpada~się w~wartości 37~miliardów rozpadów na~sekundę \\
\Powin w~którym dochodzi do~37~miliardów rozpadów na~sekundę \\



% ############################










% ######################################
\newpage

\section{Świat po~1989~r.}


\vspace{\spaceThree}
% ######################################



% ######################################
\subsection{Były Blok Sowiecki po~1989~r.}

\vspace{\spaceThree}
% ######################################



% ############################
\Work{ % Autor i tytuł dzieła
  Philipp Ther \\
  \textit{Nowy ład na starym kontynencie. Historia neoliberalnej} \\
  \textit{Europy}, \cite{TherNowyLad2015}}

\vspace{0em}


% ##################
\CenterBoldFont{Uwagi do~konkretnych stron}

\vspace{0em}


\start \Str{120} Jest tu mowa, że~Lech Wałęsa był pierwszym
prezydentem demokratycznej polski. Trzeba sprawdzić, czy ten tytuł nie
powinien przypaść gen.~Wojciechowi Jaruzelskiemu.

% \vspace{\spaceFour}





% ##################
\CenterBoldFont{Błędy}

\vspace{\spaceFive}


\begin{center}

  \begin{tabular}{|c|c|c|c|c|}
    \hline
    Strona & \multicolumn{2}{c|}{Wiersz} & Jest
                              & Powinno być \\ \cline{2-3}
    & Od góry & Od dołu & & \\
    \hline
    13  & &  5 & Federalnej, & Federalnej. \\
    14  & 16 & & „ruskich” & o~„ruskich” \\
    16  &  6 & & o & jako \\
    30  & 12 & & wschodniego & zachodniego \\
    33  &  2 & & \textit{prospects}” & \textit{prospects} \\
    41  & 13 & & po1917 & po~1917 \\
    43  &  3 & & po1918 & po~1918 \\
    49  & &  2 & \textit{1989}„ & \textit{1989}, \\
    50  & & 13 & doprowadziło & nie~doprowadziło \\
    51  & &  1 & 61-65,71 & 61--65, 71 \\
    57  & &  2 & 1985--1988 & \textit{1985--1988} \\
    60  & &  7 & NRD.nCi & NRD. Ci \\
    60  & &  1 & \textit{War},: & \textit{War}, \\
    % & & & & \\
    62  & &  9 & przypieczętowa & przypieczętowano \\
    71  & &  8 & Niemczami~i & Niemcami~niż \\
    80  & & 17 & z~Lewobrzeżną & Lewobrzeżną \\
    85  & & 11 & w & o \\
    88  & &  6 & \textbf{\textit{Anders}} & Anders \\
    109 &  1 & & & Tilly’emu \\
    124 & 11 & & 5) & 5). \\
    146 &  1 & & ( ang. & (ang. \\
    % & & & & \\
    % & & & & \\
    % & & & & \\
    % & & & & \\
    \hline
  \end{tabular}

\end{center}

\vspace{\spaceTwo}


\noindent
\StrWd{30}{22} \\
\Jest  metropolii”$^{ 14 }$. Natomiast w~regionalnej \\
\Powin metropolii”$^{ 14 }$, w~regionalnej \\


% ############################










% ######################################
\subsection{Różne dzieła historyczne}

\vspace{\spaceThree}
% ######################################



% ############################
\Work{ % Autor i tytuł dzieła
  Christopher A. Ferrara \\
  \textit{Liberty: The God That Failed}, \cite{}}


% ##################
\CenterBoldFont{Błędy}


\begin{center}

  \begin{tabular}{|c|c|c|c|c|}
    \hline
    Strona & \multicolumn{2}{c|}{Wiersz} & Jest
                              & Powinno być \\ \cline{2-3}
    & Od góry & Od dołu & & \\
    \hline
    23  &  3 & & in & in an other \\
    40  & &  1 & \textit{St. Saint} & \textit{Saint} \\
    207 & &  2 & Government & \textit{Government} \\
    227 & 11 & & doing-not & doing--not \\
    % & & & & \\
    % & & & & \\
    % & & & & \\
    \hline
  \end{tabular}

\end{center}

\vspace{\spaceTwo}



% ############################










% ######################################
\newpage

\section{Dzieje wschodniej Eurazji}


\vspace{\spaceTwo}
% ######################################



% ############################
\Work{ % Autor i tytuł dzieła
  Andrew Gordon \\
  \textit{Nowożytna historia Japonii},
  \cite{GordonNowozytnaHistoriaJaponii2010}}

\vspace{0em}

% ##################
\CenterBoldFont{Uwagi}

\vspace{0em}


\start \StrWg{46}{3} Jest tu podana ilość mieszkańców Tokio w~1720~r.,
ale~do 1868~r. to miasto nosiło nazwę~Edo.

\vspace{\spaceFour}





\start \Str{83} Nie rozumiem na czym polegała dewaluacja złotych monet
przeprowadzona przez \textit{bakufu}, ani czemu wywołało to zwiększenie
podaży pieniądza i~inflację.

\vspace{\spaceFour}





\start \StrWg{85}{10} Nie jest napisane kim jest ten potężny
reformator i~wróg cudzoziemców.

\vspace{\spaceFour}





\start \Str{88} Powinno zostać tu dokładniej opisany zamach na Iiego
oraz jawnie napisane, czy zginął on w~tym zamachu, bądź w~skutek
niego. Z~kontekstu wynika, że~Iiego zginął.

\vspace{\spaceFour}





\start \Str{91}{15} Sens tego zdania miał być chyba następujący. Gdyby
rząd \textit{bakufu} przetrwał, to w~skutek jego reform powstałby system
zbliżony do tego, który stworzyła restauracja Meiji.

\vspace{\spaceFour}





\start \StrWd{134}{13--11} Nie jest wyjaśnione która grupa
z~wymienionych grup odziedziczyła z~okresu Tokugawów zachowania
ksenofobiczne.

\vspace{\spaceFour}





\start \StrWd{424}{11} Zgodnie z~tym co było napisane na~stronie~417
kobieta ta raczej nie fałszowała banknotów, lecz~potwierdzenia
depozytu z~lokalnej agencji kredytowej.

\vspace{\spaceFour}





\start \StrWg{425}{5} W~tym miejscu jest mowa o~sytuacji gospodarczej
Japonii na przełomie lat osiemdziesiątych i~dziewięćdziesiątych XX
wieku, należy więc zwrócić uwagę, że~Unia Europejska istnieje w~sensie
formalnym od~1~listopada 1993~roku. Może~się więc zdarzyć, że~podana
tu nadwyżka handlowa odnosi~się do~czasu, gdy Unia Europejska jeszcze
formalnie nie istniał i~użycie jej nazwy jest formą skrótu myślowego.

\vspace{\spaceFour}





\start \Str{432} Podany tu średni czas czas trwania kadencji premierów
Japonii są błędne bądź problematyczne. Według nich w~latach 1955--1989
było 12 premierów, przyjmując więc, że~pierwszym z~tej dwunastki jest
Hatoyama Ichir\^{o}\footnote{Hatoyama został premierem jeszcze w~1954
  roku, ale~ponieważ urzędował cały 1955 rok, uznałem, że~należy go
  wliczyć do tej listy. Do~obliczeń włączyłem pełen czas jego
  kadencji, co wydaje~się rozsądne, bo~powinno to dać mniej powodów
  do~zamieszania.}, a~ostatnim Takeshita Noboru, średni czas ich
kadencji to 2.9 roku, podczas, gdy mediana to tylko 2.2 roku. Jeśli
zaś rozważymy kwartyl górny $3 / 4$ to wynosi\footnote{Kwartyl górny
  wybrałem w~ten sposób, że~powyżej niego jest 25\% populacji, jego
  samego zaś umieszczam pośród pozostałych 75\%.} on~3.4. Wszystkie te
liczby są niższe od~podanej tu~średniej~3.7 roku.

Zauważmy jednak, że~gdyby przyjąć tak jak w~książce pisze, że~między
1955 a~1989 rokiem upłynęły 44 lata, jawny błąd, to średnia czasu
urzędowania rzeczywiście wychodzi 3.7 roku. Ja~przyjąłem, że~skoro
$1989 - 1955 = 34$ to taką długość należy przypisać temu okresowi,
oznacza to bowiem tylko kilkumiesięczny błąd w~sumie długości
kadencji.

Co do czasu urzędowania premierów w~latach 1989--2000, to aby uzyskać
liczbę 10 premierów, należy liczyć od~Takeshita Noboru, który zaczął
kadencję jeszcze w~1987~r., do~Mori Yoshir\^{o}, który zakończył ją
w~2001~r. Przyjęcie więc, że~dziesięciu premierów, sprawowało urząd
dwanaście lat, jest po~prostu bardzo niedokładnym postawieniem sprawy,
by~nie powiedzieć niechlujnym.




% ##################
\newpage

\CenterBoldFont{Błędy}

\vspace{\spaceFive}


\begin{center}

  \begin{tabular}{|c|c|c|c|c|}
    \hline
    Strona & \multicolumn{2}{c|}{Wiersz} & Jest
                              & Powinno być \\ \cline{2-3}
    & Od góry & Od dołu & & \\
    \hline
    7   & 17 & & 398 & 298 \\
    29  &  5 & & nstąpiły & nastąpiły \\
    50  & 11 & & przepychały & przepychało \\
    50  & 19 & & sześć & sześćdziesiąt \\
    67  &  3 & & Kaitokud\^{o} & nad Kaitokud\^{o} \\
    77  &  4 & & prądów & tych prądów \\
    95  &  6 & & zamach & zamachem \\
    116 & & 12 & 1887 & 1867 \\
    123 & 17 & & ich właśnie & ich \\
    123 & & 10 & to & o~tym to \\
    146 &  4 & & 23~6475 & 236~475 \\
    156 &  1 & & osiemnastowieczni & dziewiętnastowieczni \\
    160 & 12 & & temat & łamach \\
    166 & 10 & & legacji & delegacji \\
    168 & & 5 & 265 & 264 \\
    190 & 11 & & kich & ich \\
    199 & & 7 & „kastowości & „kastowości” \\
    220 & &  4 & dawały & dodawały \\
    323 & 22 & & ojczyzny & Japonii \\
    326 & & 21 & główną & jednak główną \\
    368 & &  7 & wstrzymywania & utrzymywania \\
    380 & 25 & & pracy kluby & pracy, kluby \\
    432 & 16 & & czterdzieści cztery & trzydzieści cztery \\
    465 &  8 & & 56 & 256 \\
    465 & & 18 & Tokyo1963 & Tokyo 1963 \\
    \hline
  \end{tabular}

\end{center}

\vspace{\spaceTwo}


\noindent
\StrWd{240}{7} \\
\Jest  gospodarczej i~militarnej przewagi \\
\Powin o~gospodarczą i~militarną przewagę \\


% ############################










% ############################
\newpage

\Work{ % Autor i tytuł dzieła
  Jakub Polit \\
  \textit{Smutny kontynent. Z~dziejów Azji wschodniej w~XX~wieku},
  \cite{PolitSmutnyKontynent2002}}

\vspace{0em}


% ##################
\CenterBoldFont{Uwagi}

\vspace{0em}


\start \Str{28} Wedle tej pracy wojska japońskie wkroczyły na Syberię
w~1918~r., zaś wycofały~się w~1922, interwencja wojskowa trwała więc
cztery lata nie siedem. Siedem lat upłynęło między wkroczeniem wojsk,
a~podpisaniem pekińskiego traktatu japońsko-rosyjskiego w~1925~r.

\vspace{\spaceFour}





\start \Str{34} Według znakomitej audycji
\href{https://www.bbc.co.uk/programmes/b00pcm9f}{\textit{The~Samurai}}
(około 39~minuty) z~cyklu In~Our Time, Japończycy nie brali
do~samolotów mieczy, bowiem rozmiary kokpitu na~to nie~pozwalały.
Zaś~zdjęcia przedstawiające pilotów wsiadających do~swoich maszyn
z~mieczami były upozowane do~celów propagandowych.

\vspace{\spaceFour}





\start \StrWd{48}{11} Nie rozumiem, czemu zatopienie przez japońskie
samoloty dwóch statków na Yangzi, zostało określone jako atak piracki.

\vspace{\spaceFour}





\start \StrWd{100}{21} „Jan Baptysta” to po polsku „Jan Chrzciciel”.

\vspace{\spaceFour}





\start \Str{151} Nagłówek występujący na~nieparzystych stronach tego
artykułu to~„Douglas MacArthur,Dalekowschodni Generał\ldots”, podczas gdy
powinien mieć postaci „Douglas MacArthur, Dalekowschodni Generał\ldots”.

\vspace{\spaceFour}





% ##################
\CenterBoldFont{Błędy}


\begin{center}

  \begin{tabular}{|c|c|c|c|c|}
    \hline
    Strona & \multicolumn{2}{c|}{Wiersz} & Jest
                              & Powinno być \\ \cline{2-3}
    & Od góry & Od dołu & & \\
    \hline
    12  & & 12 & \textit{others} & \textit{Others} \\
    16  &  6 & & współwinny mi & współwinnymi \\
    17  & &  4 & \textit{Presidents. The} & \textit{Presidents: The} \\
    21  &  4 & & pro- si & prosi \\
    21  &  6 & & wobec Balfoura & Balfourowi \\
    25  & & 10 & 1939~, & 1939, \\
    28  &  1 & & polowy & połowy \\
    28  & &  6 & \textit{intervention} & \textit{Intervention} \\
    29  & &  2 & \textit{Asia. Tradition} & \textit{Asia: Tradition} \\
    34  & &  4 & \textit{Nanking. The} & \textit{Nanking: The} \\
    38  & &  6 & E.~Russell~of Liverpool & E.~Russell \\
    40  & & 18 & \textit{Criminal. The} & \textit{Criminal: The} \\
    40  & & 15 & \textit{smuggling} & \textit{Smuggling} \\
    40  & & 15 & \textit{puppets} & \textit{Puppets} \\
    40  & & 14 & \textit{atrocities} & \textit{Atrocities} \\
    43  & &  2 & Cyt & Cyt. \\
    44  & &  5 & \textit{Sun. The} & \textit{Sun:~The} \\
    46  & &  8 & \textit{Territory:The} & \textit{Territory: The} \\
    47  & &  7 & s.194 & s.~194 \\
    47  & &  6 & E.~Russell~of Liverpool & E.~Russell \\
    49  & &  4 & historia) , & historia), \\
    \hline
  \end{tabular}





  \begin{tabular}{|c|c|c|c|c|}
    \hline
    & \multicolumn{2}{c|}{} & & \\
    Strona & \multicolumn{2}{c|}{Wiersz} & Jest
                              & Powinno być \\ \cline{2-3}
    & Od góry & Od dołu & & \\
    \hline
    49  & &  8 & \textit{Japan. History} & \textit{Japan: History} \\
    52  & &  6 & \textit{Hopkins. An} & \textit{Hopkins: An} \\
    52  & &  2 & \textit{Mao. A} & \textit{Mao: A} \\
    53  & &  4 & \textit{China. The} & \textit{China: The} \\
    54  & &  4 & \textit{Tangle. The} & \textit{Tangle: The} \\
    55  & &  6 & \textit{Sun. The} & \textit{Sun: The} \\
    57  & &  5 & zob.: & zob. \\
    57  & &  4 & 1964,s. & 1964, s. \\
    59  &  6 & & min & mln \\
    59  & &  2 & \textit{China. The} & \textit{China: The} \\
    60  & & 12 & \textit{Dream. A} & \textit{Dream: A} \\
    61  &  3 & & jako by & jakoby \\
    61  & &  2 & \textit{Coexistence. The} & \textit{Coexistence: The} \\
    68  & &  4 & \textit{World. The} & \textit{World: The} \\
    69  & 13 & & a\textit{mba}- & \textit{amba}- \\
    71  & &  7 & \textit{Center. 300} & \textit{Center: 300} \\
    74  & & 14 & pretensje. Odebrałby & pretensje, odebrałby \\
    79  & &  5 & \textit{War. America} & \textit{War: America} \\
    80  & &  5 & \textit{Tibet. A} & \textit{Tibet: A} \\
    81  & &  9 & \textit{its history} & \textit{Its History} \\
    84  & &  4 & \textit{Sen, Dzieło} & \textit{Sen. Dzieło} \\
    100 & &  9 & de & ode \\
    101 & &  5 & \textit{Vietnam. A} & \textit{Vietnam: A} \\
    102 & 14 & & z~biurze & w~biurze \\
    105 & & 17 & \textit{Lake. The} & \textit{Lake: The} \\
    109 & &  3 & \textit{April. The} & \textit{April: The} \\
    110 & &  1 & 292) & 292 \\
    117 & &  6 & \textit{Wietnam: the} & \textit{Wietnam: The} \\
    118 & & 15 & -- czy & czy \\
    119 & &  5 & \textit{Minh;} & \textit{Minh:} \\
    121 & &  9 & \textit{Lake. The} & \textit{Lake: The} \\
    125 & &  8 & 407). & 407), \\
    125 & &  6 & za:~.~J. & za:~J. \\
    125 & &  5 & \textit{Retrospect. The} & \textit{Retrospect: The} \\
    132 & & 18 & kłamliwie$^{ 8 }$ , & kłamliwie$^{ 8 }$, \\
    132 & & 16 & Xianlianga$^{ 9 }$ ) & Xianlianga$^{ 9 }$) \\
    133 & &  7 & \textit{Marshall. The} & \textit{Marshall: The} \\
    133 & &  6 & \textit{autobiographical notes}
           & \textit{Autobiographical Notes} \\
    135 & & 14 & \textit{Combat. Korea} & \textit{Combat: Korea} \\
    \hline
  \end{tabular}





  \begin{tabular}{|c|c|c|c|c|}
    \hline
    & \multicolumn{2}{c|}{} & & \\
    Strona & \multicolumn{2}{c|}{Wiersz} & Jest
                              & Powinno być \\ \cline{2-3}
    & Od góry & Od dołu & & \\
    \hline
    135 & & 11 & \textit{Warfare. Secrets} & \textit{Warfare: Secrets} \\
    138 & &  6 & \textit{Logai. The} & \textit{Logai: The} \\
    138 & &  4 & (ur.1938) & (ur.~1938) \\
    141 & & 18 & przekroczony''$^{ 27 }$~. & przekroczony''$^{ 27 }$. \\
    141 & & 14 & szacunki$^{ 28 }$~. & szacunki$^{ 28 }$. \\
    142 &  1 & & oszkalowanych''$^{ 29 }$~. & oszkalowanych''$^{ 29 }$. \\
    142 &  5 & & mln$^{ 30 }$~. & mln$^{ 30 }$. \\
    142 & 13 & & chodziło$^{ 31 }$~. & chodziło$^{ 31 }$. \\
    142 & & 18 & obozie$^{ 32 }$~. & obozie$^{ 32 }$. \\
    142 & &  5 & \textit{China. Reflections}
           & \textit{China: Reflections} \\
    142 & & 12 & „~czymś & „czymś \\
    143 & 19 & & 10\%~! & 10\%! \\
    143 & 15 & & trzydziestki$^{ 33 }$~. & trzydziestki$^{ 33 }$. \\
    143 & &  8 & jedzeniem$^{ 34 }$~. & jedzeniem$^{ 34 }$. \\
    143 & &  5 & obozie$^{ 35 }$~.  & obozie$^{ 35 }$. \\
    146 &  6 & & 10\%~! & 10\%! \\
    148 &  3 & & kwiatów$^{ 44 }$~. & kwiatów$^{ 44 }$. \\
    148 & & & ile?”$^{ 45 }$~. & ile?”$^{ 45 }$ \\
    151 & 19 & & bardziej niewielu & mniej \\
    153 & &  3 & \textit{MacArthur. An} & \textit{MacArthur: An} \\
    154 & &  8 & \textit{Dream. A} & \textit{Dream: A} \\
    155 & &  8 & \textit{March: an} & \textit{March: An} \\
    159 & &  6 & \textit{War. The} & \textit{War: The} \\
    160 & &  8 & \textit{Sun. The} & \textit{Sun: The} \\
    160 & &  4 & \textit{Downfall. The} & \textit{Downfall: The} \\
    161 & &  9 & \textit{MacArthur. The} & \textit{MacArthur: The} \\
    161 & &  2 & \textit{Japan, The} & \textit{Japan: The} \\
    162 & &  7 & \textit{Peace. MacArthur} & \textit{Peace: MacArthur} \\
    162 & &  3 & \textit{Japan. A} & \textit{Japan: A} \\
    162 & &  2 & \textit{MacArthur. His} & \textit{MacArthur: His} \\
    166 & &  6 & \textit{War. The} & \textit{War: The} \\
    166 & &  3 & \textit{Ho}pe & \textit{Hope} \\
    168 & &  2 & \textit{Combat. Korea} & \textit{Combat: Korea} \\
    170 & &  4 & \textit{MacArthur, 1941-1951}
           & \textit{MacArthur: 1941--1951} \\
    170 & &  3 & obsesją , & obsesją, \\
    171 &  5 & & żołnierzem.$^{ 52 }$ & żołnierzem”$^{ 52 }$. \\
    178 & &  3 & \textit{Japan. A} & \textit{Japan: A} \\
    182 & &  4 & \textit{Hirohito. Emperor} & \textit{Hirohito: Emperor} \\
    182 & &  8 & \textit{procesowi}.$^{ 22 }$
           & \textit{procesowi}”.$^{ 22 }$ \\
    \hline
  \end{tabular}





  \begin{tabular}{|c|c|c|c|c|}
    \hline
    & \multicolumn{2}{c|}{} & & \\
    Strona & \multicolumn{2}{c|}{Wiersz} & Jest
                              & Powinno być \\ \cline{2-3}
    & Od góry & Od dołu & & \\
    \hline
    183 & & 12 & lnukai & Inukai \\
    184 & &  5 & \textit{Japan. The} & \textit{Japan: The} \\
    185 & & 11 & \textit{Nanking. The} & \textit{Nanking: The} \\
    185 & & 10 & \textit{l997,passim} & 1997, \textit{passim} \\
    185 & &  5 & \textit{Hirahito. Behind} & \textit{Hirahito: Behind} \\
    185 & &  2 & \textit{Hirahito. Behind} & \textit{Hirahito: Behind} \\
    187 & 13 & & \textit{terno} & \textit{tenno} \\
    188 & & 11 & \textit{Peace. MacArthur} & \textit{Peace: MacArthur} \\
    193 & &  7 & \textit{Prince. Akihito} & \textit{Prince: Akihito} \\
    193 & &  5 & \textit{Mishima. A} & \textit{Mishima: A} \\
    193 & &  4 & \textit{Yukio. The} & \textit{Yukio: The} \\
    194 & & 17 & \textit{eks}-króla & eks-króla \\
    194 & & 15 & jako przywódca & przywódca \\
    194 & &  9 & tenno & \textit{tenno} \\
    197 & 13 & & smród”$^{ 2 }$~. & smród”$^{ 2 }$. \\
    197 & &  2 & $^{ 2 }$H.~Lory & $^{ 2 }$ H.~Lory \\
    198 & &  1 & \textit{Tojo. The} & \textit{Tojo: The} \\
    200 & &  7 & \textit{Patriots: a} & \textit{Patriots: A} \\
    201 & &  2 & \textit{Nomonhan. Japan} & \textit{Nomonhan: Japan} \\
    203 & &  2 & \textit{1939-1946. The} & \textit{1939--1946: The} \\
    204 & &  8 & \textit{Admiral. Yamamoto} & \textit{Admiral: Yamamoto} \\
    204 & &  4 & por.~: & por.: \\
    207 & &  4 & \textit{Sun. The} & \textit{Sun: The} \\
    210 & 11 & & Mości”$^{ 32 }$~. & Mości”$^{ 32 }$. \\
    210 & 18 & & decyzji”.$^{ 33 }$ & decyzji”$^{ 33 }$. \\
    210 & &  6 & prewencyjnemu”$^{ 35 }$~. & prewencyjnemu”$^{ 35 }$. \\
    213 & & 10 & \textit{II. Selected} & \textit{II: Selected} \\
    216 & & 10 & \textit{Justice. The} & \textit{Justice: The} \\
    220 & &  4 & \textit{Admiral. Yamamoto} & \textit{Admiral: Yamamoto} \\
    221 & &  7 & \textit{Alliance. The} & \textit{Alliance: The} \\
    221 & &  4 & \textit{Decline. A} & \textit{Decline: A} \\
    223 & &  2 & \textit{Pacific. A} & \textit{Pacific: A} \\
    224 & &  2 & \textit{History. Japanes} & \textit{History: Japanes} \\
    225 & &  9 & \textit{Polityczno- dyplomatyczne}
           & \textit{Polityczno-dyplomatyczne} \\
    225 & &  7 & \textit{between} & \textit{Between} \\
    225 & &  3 & \textit{Conference 1921-22. Naval}
           & \textit{Conference 1921--22: Naval} \\
    227 & &  3 & \textit{Choice. Japan's} & \textit{Choice: Japan's} \\
    230 & &  5 & \textit{Enemies. The} & \textit{Enemies: The} \\
    232 & &  9 & \textit{Sun. The} & \textit{Sun: The} \\
    \hline
  \end{tabular}





  \begin{tabular}{|c|c|c|c|c|}
    \hline
    & \multicolumn{2}{c|}{} & & \\
    Strona & \multicolumn{2}{c|}{Wiersz} & Jest
                              & Powinno być \\ \cline{2-3}
    & Od góry & Od dołu & & \\
    \hline
    232 & &  3 & \textit{Slept. The} & \textit{Slept: The} \\
    233 & & 18 & \textit{Orange. The} & \textit{Orange: The} \\
    233 & & 16 & dzieło, o~którym & działania, o~których \\
    235 & &  3 & \textit{Sea, Midway} & \textit{Sea: Midway} \\
    235 & 12 & & w~USA & z~USA \\
    238 &  9 & & Wysokością!. & Wysokością! \\
    243 & &  3 & \textit{Kai-shek: soldier and stantesman}
           & \textit{Kai-shek: Soldier and Stantesman} \\
    244 & &  2 & \textit{China. The~first full biography}
           & \textit{China: The~First Full Biography} \\
    247 & 13 & & czaerwcu & czerwcu \\
    248 & & 19 & i~najcięższe & o~najcięższe \\
    248 & &  5 & \textit{Expedition. China's}
           & \textit{Expedition: China's} \\
    249 & & 20 & „panów” wojny” & „panów wojny” \\
    249 & &  4 & S.I.Hsiung & S.I. Hsiung \\
    249 & &  2 & \textit{coup} & \textit{Coup} \\
    249 & &  1 & \textit{nationalism} & \textit{Nationalism} \\
    249 & &  1 & \textit{revolution} & \textit{Revolution} \\
    250 & &  4 & \textit{under} & \textit{Under} \\
    251 & &  3 & \textit{Counter --~revolutionaries}
           & \textit{Cunterrevolutionaries} \\
    252 & & 11 & \textit{Manchuria. The} & \textit{Manchuria: The} \\
    252 & & 10 & \textit{the} & \textit{The} \\
    252 & &  3 & \textit{Change. Essays} & \textit{Change: Essays} \\
    253 & &  2 & \textit{the epic} & \textit{The Epic} \\
    253 & &  2 & \textit{communism's} & \textit{Communism's} \\
    253 & &  1 & \textit{survival} & \\
    255 & & 10 & żony~, & żony \\
    256 & &  4 & \textit{Fight. The} & \textit{Fight: The} \\
    256 & &  2 & \textit{War. From} & \textit{War: From} \\
    261 & &  9 & \textit{East. China} & \textit{East: China} \\
    261 & &  2 & \textit{Creation. My} & \textit{Creation: My} \\
    265 & 18 & & z~absurdalny & w~absurdalny \\
    % & & & & \\
    % & & & & \\
    % & & & & \\
    \hline
  \end{tabular}

\end{center}

\vspace{\spaceTwo}


% ############################










% ############################
\newpage

\Work{ % Autor i tytuł dzieła
  Jakub Polit \\
  \textit{Smutny kontynent. Epizody i~sylwetki z~dziejów Azji Wschodniej
  w~XX~w.}, \cite{PolitSmutnyKontynent2021}}


% % ##################
% \CenterBoldFont{Uwagi}





% ##################
\CenterBoldFont{Błędy}

\vspace{\spaceFive}


\begin{center}

  \begin{tabular}{|c|c|c|c|c|}
    \hline
    Strona & \multicolumn{2}{c|}{Wiersz} & Jest
                              & Powinno być \\ \cline{2-3}
    & Od góry & Od dołu & & \\
    \hline
    7 &  4 & & \textit{spoglądać{ } na} & \textit{spoglądać na} \\
    % & & & & \\
    % & & & & \\
    % & & & & \\
    % & & & & \\
    % & & & & \\
    % & & & & \\
    \hline
  \end{tabular}





  \begin{tabular}{|c|c|c|c|c|}
    \hline
    & \multicolumn{2}{c|}{} & & \\
    Strona & \multicolumn{2}{c|}{Wiersz} & Jest
    & Powinno być \\ \cline{2-3}
    & Od góry & Od dołu & & \\
    \hline
    403 & &  6 & \textit{Ally.} & \textit{Ally:} \\
    408 & &  8 & wespół & wespół~z \\
    412 & &  3 & \textit{Snows. A~History} & \textit{Snows: A~History} \\
    412 & &  3 & \textit{since} & \textit{Since} \\
    416 &  1 & & w przyczyn & z~przyczyn \\
    417 & &  2 & \textit{Generalissimo. Chiang}
           & \textit{The Generalissimo: Chiang} \\
    423 & &  2 & \textit{Enlai. The} & \textit{Enlai: The} \\
    425 &  7 & & Chin nowoczesnych & nowoczesnych Chin \\
    425 & &  4 & \textit{History} $^{ 2 }$ & \textit{History}$^{ 2 }$ \\
    425 & &  2 & \textit{History. The} & \textit{History: The} \\
    428 &  7 & & krepowano & krępowano \\
    428 & & 17 & `W~przyszłości & W~przyszłości \\
    429 & 12 & & Sukcesom i~aktywnościom & W~sukcesach i~aktywnościach \\
    431 & 15 & & poślubiła w~1914~r. & poślubiła \\
    431 & & 15 & „H.H” & „H.H.” \\
    433 & &  5 & sunie & Sunie \\
    %   & & & & \\
    %   & & & & \\
    %   & & & & \\
    %   & & & & \\
    %   & & & & \\
    %   & & & & \\
    %   & & & & \\
    %   & & & & \\
    %   & & & & \\
    %   & & & & \\
    %   & & & & \\
    %   & & & & \\
    %   & & & & \\
    %   & & & & \\
    %   & & & & \\
    %   & & & & \\
    %   & & & & \\
    %   & & & & \\
    %   & & & & \\
    %   & & & & \\
    %   & & & & \\
    %   & & & & \\
    %   & & & & \\
    %   & & & & \\
    \hline
  \end{tabular}

\end{center}

\vspace{\spaceTwo}


% ############################











% ######################################
\newpage

\section{Historia Chin}


\vspace{\spaceTwo}
% ######################################



% ############################
\Work{ % Autor i tytuł dzieła
  Jakub Polit \\
  \textit{Chiny 1946--1949}, \cite{PolitChiny1946Do1949Wyd2010}}


% \CenterTB{Uwagi}





% ##################
\CenterBoldFont{Błędy}

\vspace{\spaceFive}


\begin{center}

  \begin{tabular}{|c|c|c|c|c|}
    \hline
    Strona & \multicolumn{2}{c|}{Wiersz} & Jest
                              & Powinno być \\ \cline{2-3}
    & Od góry & Od dołu & & \\
    \hline
    8   & & 16 & jednej & jednego \\
    % & & & & \\
    % & & & & \\
    % & & & & \\
    % & & & & \\
    % & & & & \\
    % & & & & \\
    % & & & & \\
    % & & & & \\
    \hline
  \end{tabular}

\end{center}

\vspace{\spaceTwo}


% ############################










% ####################################################################
\newpage

\section{Historia gospodarki, ekonomii i~pieniądza}


\vspace{\spaceTwo}
% ####################################################################



% ############################
\Work{ % Autor i tytuł dzieła
  Niall Ferguson \\
  \textit{Niebezpieczne związki. Pieniądze i~władza w~świecie nowożytnym
  1700--2000}, \cite{FergusonNiebezpieczneZwiazki2015}}


% ##################
\CenterBoldFont{Błędy}


\begin{center}

  \begin{tabular}{|c|c|c|c|c|}
    \hline
    Strona & \multicolumn{2}{c|}{Wiersz} & Jest
                              & Powinno być \\ \cline{2-3}
    & Od góry & Od dołu & & \\
    \hline
    14  & & 11 & Stechlinowie & \textit{Stechlinowie} \\
    16  & & 10 & 18 & \textit{18} \\
    17  & & 11 & Nowoczesny system-świat
           & \textit{Nowoczesny system świat} \\
    19  &  3 & & Więcej niż skrajni & \textit{Więcej niż skrajni} \\
    % & & & & \\
    % & & & & \\
    % & & & & \\
    % & & & & \\
    % & & & & \\
    627 & 12 & & \textit{Stechlin} & \textit{Der Stechlin} \\
    % & & & & \\
    \hline
  \end{tabular}

\end{center}

\vspace{\spaceTwo}


\noindent
\StrWd{19}{1} \\
\Jest  Zarys historii gospodarczej XX~wieku \\
\Powin \textit{Zarys historii gospodarczej XX~wieku}


% ############################










% ######################################
\newpage

\section{Biografie}


\vspace{\spaceTwo}
% ######################################



% ############################
\Work{ % Autor i tytuł dzieła
  Sławomir Cenckiewicz \\
  \textit{Anna Solidarność}, \cite{CenckiewiczAnnaSolidarnosc2010}}


% ##################
\CenterBoldFont{Błędy}

\vspace{\spaceFive}


\begin{center}

  \begin{tabular}{|c|c|c|c|c|}
    \hline
    Strona & \multicolumn{2}{c|}{Wiersz} & Jest
                              & Powinno być \\ \cline{2-3}
    & Od góry & Od dołu & & \\
    \hline
    21 & 10 & & zajęłam & zajęła \\
    % & & & & \\
    \hline
  \end{tabular}

\end{center}

\vspace{\spaceTwo}


% ############################










% ############################
\newpage

\Work{ % Autorka i tytuł dzieła
  Masha Gessen \\
  \textit{Putin. Człowiek bez twarzy},
  \cite{GessenPutinCzlowiekBezTwarzy2012}}

\vspace{0em}

% ##################
\CenterBoldFont{Uwagi}

\vspace{0em}


\start \StrWd{9}{6} Zamieszczony tu komentarz odnośnie słowa
„lustracja”, które ma wedle niego pochodzić z~greki, jest zapewne
wynikiem niedbałości tłumacza. Polskie słowo „lustracja”, pochodzi
najpewniej od słowa „lustro”, które wydaje~się w~ogóle nie związane
z~greką. Prawdopodobnie ten fragment został przetłumaczony
mechanicznie, bez~refleksji, że~w~języku polskim, w~przeciwieństwie
do~oryginału, ten związek etymologiczny nie zachodzi.

\vspace{\spaceFour}





\start \Str{75--76} W~przedstawionej tu opowieści jest pewna
niekonsekwencja. Na~75 stronie pisze, że~Putin był w~tłumie
drezdeńczyków nacierających na budynek Stasi, czyli musiał
znajdować~się na zewnątrz. Jednak na~następnej stronie pisze,
że~wyszedł do owego tłumu na zewnątrz, więc musiał znajdować~się
w~środku budynku. Nigdzie nie jest napisane, jak i~dlaczego opuścił
tłum i~wszedł do siedziby Stasi.

\vspace{\spaceFour}





\start \Str{93} Sacharow urodził~się w~1921 r., Gorbaczow zaś w~1931,
w~1989 mięli więc odpowiednio 68 i~58 lat. Nazwanie Gorbaczowa młody
to pewne nadużycie, wynikające zapewne z~kontrastu między schorowany,
bliski śmierci Sacharowem, a~pełnym energii Michaiłem.

\vspace{\spaceFour}





\start \textbf{Str. 101, akapit trzeci.} Powinno tu być jawniej napisane,
że~wracamy do historii Putina.

\vspace{\spaceFour}





\start \textbf{Tylna okładka, wiersz 5 (od dołu).} Masha Gessen
urodziła~się w~1967~r. więc w~latach 1981--1991 miła od~14 do~24 lat,
jest więc wysoce nieprawdopodobne, by~w~tym okresie pracowała
w~Stanach Zjednoczonych. Zapewne chodziło o~to, że~wówczas tam
mieszkała.





% ##################
\CenterBoldFont{Błędy}

\vspace{\spaceFive}


\begin{center}

  \begin{tabular}{|c|c|c|c|c|}
    \hline
    Strona & \multicolumn{2}{c|}{Wiersz} & Jest
                              & Powinno być \\ \cline{2-3}
    & Od góry & Od dołu & & \\
    \hline
    32 &  2 & & zdawali się nie & nie zdawali się \\
    48 &  7 & & przed wyznaczeniem & po wyznaczeniu \\
    59 & & 11 & założyciela & twórcy \\
    63 &  4 & & karierze$^{ 34 }$ & karierze \\
    63 &  6 & & n~i~c~h”. & n~i~c~h”$^{ 34 }$. \\
    63 &  9 & & twarze$^{ 35 }$ & twarze \\
    63 & 10 & & znaczenie”. & znaczenie”$^{ 35 }$. \\
    63 & & 17 & ważnego$^{ 36 }$ & ważnego \\
    63 & & 13 & KGB”. & KGB”$^{ 36 }$. \\
    64 &  2 & & międzyludzkich<<$^{ 37 }$ & międzyludzkich<< \\
    64 &  5 & & międzyludzkich” & międzyludzkich”$^{ 37 }$ \\
    65 &  4 & & przyjaciółko$^{ 40 }$ & przyjaciółko \\
    78 &  5 & & robić$^{ 68 }$ & \\
    78 &  7 & & błędy?” & błędy?”$^{ 68 }$ \\
    % & & & & \\
    % & & & & \\
    \hline
  \end{tabular}

\end{center}

\vspace{\spaceTwo}


% ############################










% ############################
\Work{ % Autor i tytuł dzieła
  ???? \\
  , \cite{}}


% ##################
\CenterBoldFont{Błędy}


\begin{center}

  \begin{tabular}{|c|c|c|c|c|}
    \hline
    Strona & \multicolumn{2}{c|}{Wiersz} & Jest
                              & Powinno być \\ \cline{2-3}
    & Od góry & Od dołu & & \\
    \hline
    % & & & & \\
        % & & & & \\
            % & & & & \\
                % & & & & \\
                    % & & & & \\
                        % & & & & \\
                            % & & & & \\
                                % & & & & \\
                                    % & & & & \\
                                        % & & & & \\
                                            % & & & & \\
                                                % & & & & \\
                                                    % & & & & \\
    \hline
  \end{tabular}

\end{center}

\vspace{\spaceTwo}


% ############################










% ############################
\newpage

\Work{ % Autor i tytuł dzieła
  Charles Moore \\
  \textit{Margaret Thatcher. Autoryzowana biografia~2}, 
  \cite{MooreMargaretThatcherPLVolII2019}}

\vspace{0em}


% ##################
\CenterBoldFont{Uwagi do konkretnych stron}

\vspace{0em}


\StrWg{989}{14--15} Według mnie tekst ,,zapobiec degeneracji wolności z~anarchiczną 
i~destrukcyjną samowolą'' jest błędny. Zgaduję, że~poprawna jego wersja powinna 
brzmieć mniej więcej ,,zapobiec degeneracji wolności przez anarchiczną 
i~destrukcyjną samowolę''.

\vspace{\spaceFour}





\Str{990} Wydaje~się, że~przypis oznaczony czterema gwiazdkami powinien odnosić~się 
do~pewnych konkretnych słów pani Thatcher, jednak w~głównych tekście brak jest 
cudzysłowów, które pozwalałyby stwierdzić, które konkretne słowa wypowiedziała ona,
a~które~są autorstwa Tomasza Tulejskiego. 





% ##################
\CenterBoldFont{Błędy}

\vspace{\spaceFive}


\begin{center}

  \begin{tabular}{|c|c|c|c|c|}
    \hline
    Strona & \multicolumn{2}{c|}{Wiersz} & Jest
                              & Powinno być \\ \cline{2-3}
    & Od góry & Od dołu & & \\
    \hline
    % & & & & \\
    % & & & & \\
    % & & & & \\
                % & & & & \\
                    % & & & & \\
                        % & & & & \\
                            % & & & & \\
                                % & & & & \\
                                    % & & & & \\
                                        % & & & & \\
    989 & & 17 & mniej & niej \\
    992 &  2 & & on , & on, \\
    996 & &  2 & A.F. & F.A. \\
    \hline
  \end{tabular}

\end{center}

\vspace{\spaceTwo}


% ############################










% ####################################################################
% ####################################################################
% Bibliografia

\bibliographystyle{plalpha}

\bibliography{HistoryBooks}{}





% ############################

% Koniec dokumentu
\end{document}