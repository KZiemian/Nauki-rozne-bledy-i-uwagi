% ---------------------------------------------------------------------
% Podstawowe ustawienia i pakiety
% ---------------------------------------------------------------------
\RequirePackage[l2tabu, orthodox]{nag} % Wykrywa przestarzałe i niewłaściwe
% sposoby używania LaTeXa. Więcej jest w l2tabu English version.
\documentclass[a4paper,11pt]{article}
% {rozmiar papieru, rozmiar fontu}[klasa dokumentu]
\usepackage[MeX]{polski} % Polonizacja LaTeXa, bez niej będzie pracował
% w języku angielskim.
\usepackage[utf8]{inputenc} % Włączenie kodowania UTF-8, co daje dostęp
% do polskich znaków.
\usepackage{lmodern} % Wprowadza fonty Latin Modern.
\usepackage[T1]{fontenc} % Potrzebne do używania fontów Latin Modern.



% ------------------------------
% Podstawowe pakiety (niezwiązane z ustawieniami języka)
% ------------------------------
\usepackage{microtype} % Twierdzi, że poprawi rozmiar odstępów w tekście.
\usepackage{graphicx} % Wprowadza bardzo potrzebne komendy do wstawiania
% grafiki.
\usepackage{verbatim} % Poprawia otoczenie VERBATIME.
\usepackage{textcomp} % Dodaje takie symbole jak stopnie Celsiusa,
% wprowadzane bezpośrednio w tekście.
\usepackage{vmargin} % Pozwala na prostą kontrolę rozmiaru marginesów,
% za pomocą komend poniżej. Rozmiar odstępów jest mierzony w calach.
% ------------------------------
% MARGINS
% ------------------------------
\setmarginsrb
{ 0.7in}  % left margin
{ 0.6in}  % top margin
{ 0.7in}  % right margin
{ 0.8in}  % bottom margin
{  20pt}  % head height
{0.25in}  % head sep
{   9pt}  % foot height
{ 0.3in}  % foot sep



% ------------------------------
% Często przydatne pakiety
% ------------------------------
% \usepackage{csquotes} % Pozwala w prosty sposób wstawiać cytaty do tekstu.
% \usepackage{xcolor} % Pozwala używać kolorowych czcionek (zapewne dużo
% % więcej, ale ja nie potrafię nic o tym powiedzieć).



% ------------------------------
% Pakiety do tekstów z nauk przyrodniczych
% ------------------------------
\let\lll\undefined % Amsmath gryzie się z językiem pakietami do języka
% polskiego, bo oba definiują komendę \lll. Aby rozwiązać ten problem
% oddefiniowuję tę komendę, ale może tym samym pozbywam się dużego Ł.
\usepackage[intlimits]{amsmath} % Podstawowe wsparcie od American
% Mathematical Society (w skrócie AMS)
\usepackage{amsfonts, amssymb, amscd, amsthm} % Dalsze wsparcie od AMS
% \usepackage{siunitx} % Do prostszego pisania jednostek fizycznych
\usepackage{upgreek} % Ładniejsze greckie litery
% Przykładowa składnia: pi = \uppi
\usepackage{slashed} % Pozwala w prosty sposób pisać slash Feynmana.
\usepackage{calrsfs} % Zmienia czcionkę kaligraficzną w \mathcal
% na ładniejszą. Może w innych miejscach robi to samo, ale o tym nic
% nie wiem.



% ------------------------------
% Tworzenie środowisk (?) „Twierdzenie”, „Definicja”, „Lemat”, etc.
% ------------------------------
\newtheorem{theorem}{Twierdzenie}  % Komenda wprowadzająca otoczenie
% „theorem” do pisania twierdzeń matematycznych
\newtheorem{definition}{Definicja}  % Analogicznie jak powyżej
\newtheorem{corollary}{Wniosek}



% ------------------------------
% Pakiety napisane przez użytkownika.
% Mają być w tym samym katalogu to ten plik .tex
% ------------------------------
\usepackage{latexgeneralcommands}
\usepackage{mathcommands}





% --------------------------------------------------------------------
% Dodatkowe ustawienia dla języka polskiego
% --------------------------------------------------------------------
\renewcommand{\thesection}{\arabic{section}.}
% Kropki po numerach rozdziału (polski zwyczaj topograficzny)
\renewcommand{\thesubsection}{\thesection\arabic{subsection}}
% Brak kropki po numerach podrozdziału



% ------------------------------
% Ustawienia różnych parametrów tekstu
% ------------------------------
\renewcommand{\baselinestretch}{1.1}

\renewcommand{\arraystretch}{1.4}  % Ustawienie szerokości odstępów między
% wierszami w tabelach.



% ------------------------------
% Pakiet „hyperref”
% Polecano by umieszczać go na końcu preambuły.
% ------------------------------
\usepackage{hyperref} % Pozwala tworzyć hiperlinki i zamienia odwołania
% do bibliografii na hiperlinki.










% ---------------------------------------------------------------------
% Tytuł i autor tekstu
\title{Algorytmy \\
  {\Large Błędy i~uwagi}}

\author{Kamil Ziemian}


% \date{}
% ---------------------------------------------------------------------










% ####################################################################
\begin{document}
% ####################################################################





% ######################################
\maketitle  % Tytuł całego tekstu
% ######################################





% ############################
\Work{ % Autor i tytuł dzieła
  Thomas H.~Cormen, Charles E.~Leiserson, Ronald L.~Rivest, Clifford Stein \\
  \textit{Wprowadzenie do~algorytmów},
  \cite{CormenAtAlWprowadzenieDoAlgorytmow2022}}

\vspace{0em}


% ##################
\CenterBoldFont{Uwagi}

\vspace{0em}


\noindent W~nauce o~algorytmach w~jej dzisiejszym stanie ważne miejsce
zajmuje badanie następujące pytanie. Jeśli algorytm pobiera $n \in \Nbb$
danych wejściowych, jak można oszacować od góry i~od dołu ilość operacji
jaką algorytm wykona by uzyskać pożądanych rezultat? W~innym sformułowaniu
zamiast pytać~się ilość operacji, pytamy~się o~oszacowanie czasu jaki zajmie
wykonanie algorytmu, jednak to zagadnienie zwykle jest sprowadzane do tego
samego problemu matematycznego, który różni~się od poprzedniego tylko
wartością pewnych stałych liczbowych.

Prowadzi to w~naturalny sposób do badania funkcji typu
\begin{equation}
  \label{eq:CormenAtAl-WprowadzenieDoAlgorytmow-01}
  f : \Nbb \to \Rbb.
\end{equation}
Ponieważ jednak nie ma sensu mówić, przynajmniej w~normalnych warunkach, ani
o~ujemnej liczbie kroków jakie wykona algorytm, ani o~ujemny czasie jego
trwania, więc możemy~się ograniczyć do badania funkcji o~wartościach
nieujemnych.
\begin{equation}
  \label{eq:CormenAtAl-WprowadzenieDoAlgorytmow-02}
  f : \Nbb \to \Rbb_{ + }, \quad
  \Rbb_{ + } := \{ x \in \Rbb \, | \, x \geq 0 \}.
\end{equation}
Przypadek gdy $f( n_{ 0 } ) = 0$, dla pewnego $n_{ 0 }$, czyli algorytm
w~ogóle nie będzie musiał niczego robić, wydaje~się dziwny, ale~dla
uproszczenia rozważań warto dopuścić występowanie takich funkcji. Opisują
one przypadek, gdy rozwiązanie jest od razu dane, więc nie musimy nic robić
by je uzyskać.

Ze względu na to, że~ilość operacji wykonywanych przez algorytm przy
dowolnej ilości danych wejściowych, też jest liczbą naturalną,
a~przynajmniej piszącemu te słowa nie jest znany
przypadek, gdzie byłoby inaczej, to możemy ograniczyć rozważaną klasę
funkcji. Mianowicie do funkcji, które są postaci
\begin{equation}
  \label{eq:CormenAtAl-WprowadzenieDoAlgorytmow-03}
  f( n ) = C g( n ), \qquad
  C > 0, \quad
  g : \Nbb \to \Nbb.
\end{equation}
W~szczególności sama $f$ może przyjmować tylko wartości naturalne.

Wydaje~się, że~dla autorów tej książki było tak oczywiste, że~rozważane
funkcje są nieujemne, iż~zapomnieli oni napisać jawnie, że~należy przyjmować,
iż~wszystkie rozważane funkcje są nieujemne, chyba że wskazano inaczej. Tym
samym podają wyniki, które są prawdziwe tylko gdy operujemy funkcjami
nieujemnymi, ale~nigdzie nie jest to napisane jawnie.

Przyjrzyjmy~się choćby dodatkowi~A w~tej książce. Na jego początku jest mowa
o~sumach i~szeregach które posiadają wyrazy ujemne, lecz potem prawie bez
wyjątku analizowane są tylko szeregi o~wyrazach dodatnich. Należałoby
sprawdzić, czy~któreś z~podanych w~tym rozdziale twierdzeń nie jest
prawdziwe, tylko dla takich sum oraz szeregów, bo inaczej czytelnik może
zostać wprowadzony w~błąd.

Warte uwagi jest to, że~w~tym dodatku liczby ujemne prawie~się nie
pojawiają, a~nawet gdy~się pojawiają, to może to być rezultatem z~rozkładu
pewnej liczby \textit{dodatniej} na różnicę dwóch liczb. Tak jest na
stronie~1173, gdzie analizowana jest suma teleskopowa o~wyrazie ogólnym
\begin{equation}
  \label{eq:CormenAtAl-WprowadzenieDoAlgorytmow-04}
  a_{ k } = \frac{ 1 }{ k ( k + 1 ) } =
  \frac{ 1 }{ k } - \frac{ 1 }{ k + 1 }.
\end{equation}

Innym założeniem które mogło ujść autorom niezauważenie, jest przyjęcie,
że~dla każdej funkcji $f : \Nbb \to \Rbb_{ + }$ zachodzi
\begin{equation}
  \label{eq:CormenAtAl-WprowadzenieDoAlgorytmow-05}
  \forall \, n \in \Nbb, \quad
  f( n ) \geq 1,
\end{equation}
tym samym za pomocą notacji asymptotycznej wprowadzonej na stronie ??? tej
książki możemy zapisać
\begin{equation}
  \label{eq:CormenAtAl-WprowadzenieDoAlgorytmow-06}
  1 = O\left( f( n ) \right).
\end{equation}
Założenie to jest naturalne, bowiem należy~się spodziewać, iż dla dowolnej
ilości danych wyjściowych, może z~wyjątkiem $n = 1$, każdy algorytm wykona
co najmniej jedną operację. Problem ile zajmie czas wykonanie danego
algorytmu jest bardziej subtelny, bowiem zależy od tego w~jakich jednostkach
mierzymy czas wykonania operacji, lecz można przypuszczać, iż~autorzy
książki i~tutaj przyjęli warunek analogiczny
do~\eqref{eq:CormenAtAl-WprowadzenieDoAlgorytmow-05}. Po prostu dlatego,
że~argument „każdy algorytm wykonuje co najmniej jedną operację”, ma tak
ogromną siłę perswazji.

Łatwo jednak podać funkcję, która nie spełnia warunku
\eqref{eq:CormenAtAl-WprowadzenieDoAlgorytmow-06}. Weźmy
\begin{equation}
  \label{eq:CormenAtAl-WprowadzenieDoAlgorytmow-07}
  f( n ) =
  \begin{cases}
    1, & n = 0, \\
    \frac{ 1 }{ n }, & n \geq 1,
  \end{cases}
\end{equation}
dla której mamy
\begin{equation}
  \label{eq:CormenAtAl-WprowadzenieDoAlgorytmow-08}
  1 \neq O\!\left( \tfrac{ 1 }{ n } \right), \quad
  \frac{ 1 }{ n } = O( 1 ).
\end{equation}
Dla ścisłości rozważań należy więc warunek
\eqref{eq:CormenAtAl-WprowadzenieDoAlgorytmow-05} przyjąć jawnie.

\vspace{\spaceFour}










% ##################
\CenterBoldFont{Uwagi do~konkretnych stron}


\noindent
\StrWierszeG{4}{7--8} Zbigniew Rudy uświadomił mi, że jest pewien problem
z~podanym tu stwierdzeniem „algorytm jest ciągiem kroków obliczeniowych
przekształcających dane wejściowe w~wyjściowe”. Chodzi o~to, że~słowo
„ciąg” sugeruje, że~kroki obliczeniowe wykonuje~się jeden po drugim
w~określonej kolejności, co nie jest prawdą dla tak ważnych
dziś\footnote{Piszę to w~2023 roku.} algorytmów równoległych. W~ich przypadku
przynajmniej część obliczeń wykonuje~się w~tym samym czasie w~różnych
jednostkach obliczeniowych. W~szczególności nie jest ustalone, który krok
obliczeń równoległych wykona~się przed którym. Dodatkowo jeśli przy
pierwszym wykonywaniu algorytmu równoległego najpierw wykona~się krok~$A$,
a~potem~$B$, to przy następnym jego wykonywaniu może~się zdarzyć, iż~krok~$B$
wykona~się przed krokiem $A$.

\vspace{\spaceFour}




















\noindent
\Str{1171} Trzeba poważnie przemyśleć co to znaczy, że~w~zależności
\begin{equation}
  \label{eq:CormenAtAl-WprowadzenieDoAlgorytmow-09}
  \sum_{ k = 1 }^{ n } \Theta\big( f( k ) \big) =
  \Theta\left( \sum_{ k = 1 }^{ n } f( k ) \right),
\end{equation}
symbol $\Theta$ po lewej stronie oznacza oszacowanie asymptotyczne
w~zmiennej~$k$, zaś po prawej w~zmiennej $n$. Wydaje mi~się, że~coś jest
z~tym nie w~porządku.

\vspace{\spaceFour}





\noindent
\Str{1175--1176} Przeanalizujmy jeszcze raz dokładnie, błędny dowód fałszywej
zależności $\sum_{ k = 1 }^{ n } = O( n )$, zaczynając od ścisłej rachunków.

Dla $n = 1$ nam
\begin{equation}
  \label{eq:CormenAtAl-WprowadzenieDoAlgorytmow-10}
  \sum_{ k = 1 }^{ 1 } = 1,
\end{equation}
dla $n = 2$ dostajemy
\begin{equation}
  \label{eq:CormenAtAl-WprowadzenieDoAlgorytmow-11}
  \sum_{ k = 1 }^{ 2 } k = \sum_{ k = 1 }^{ 1 } k + 2 = 1 + 2 = 1.5 \cdot 2,
\end{equation}
w~trzecim kroku mamy
\begin{equation}
  \label{eq:CormenAtAl-WprowadzenieDoAlgorytmow-12}
  \sum_{ k = 1 }^{ 3 } k = \sum_{ k = 1 }^{ 2 } k + 3 = 1.5 \cdot 2 + 3 = 2 \cdot 3,
\end{equation}
zaś w~czwartym
\begin{equation}
  \label{eq:CormenAtAl-WprowadzenieDoAlgorytmow-13}
  \sum_{ k = 1 }^{ 4 } k = 2 \cdot 3 + 4 = 2.5 \cdot 4.
\end{equation}

Przyjmijmy teraz, że~mamy stałą $c_{ n }$, taką~że zachodzi
\begin{equation}
  \label{eq:CormenAtAl-WprowadzenieDoAlgorytmow-14}
  \sum_{ k = 1 }^{ n } k = c_{ n } n.
\end{equation}
Stąd dostajemy
\begin{equation}
  \label{eq:CormenAtAl-WprowadzenieDoAlgorytmow-15}
  \sum_{ k = 1 }^{ n + 1 } k = \sum_{ k = 1 }^{ n } k + n + 1 = c_{ n } n + n + 1 =
  \frac{ ( c_{ n } + 1 ) n + 1 }{ n + 1 } ( n + 1 ),
\end{equation}
czyli
\begin{equation}
  \label{eq:CormenAtAl-WprowadzenieDoAlgorytmow-16}
  c_{ n + 1 } = \frac{ ( c_{ n } + 1 ) n + 1 }{ n + 1 },
\end{equation}
przy czym $c_{ 1 } = 1$. Można sprawdzić, że~rozwiązaniem tego problemu,
dobrze znanym ze wzoru na sumę postępu arytmetycznego, jest
$c_{ n } = 0.5 ( n + 1 )$, czyli inaczej mówiąc
\begin{equation}
  \label{eq:CormenAtAl-WprowadzenieDoAlgorytmow-17}
  c_{ n + 1 } = c_{ n } + 0.5,
\end{equation}
co jest zgodne z~wzorami
\eqref{eq:CormenAtAl-WprowadzenieDoAlgorytmow-10}-\eqref{eq:CormenAtAl-WprowadzenieDoAlgorytmow-13}.
Widzimy więc, że~nie istnieje taka liczba $C$ by dla każdego $n$ zachodziło
\begin{equation}
  \label{eq:CormenAtAl-WprowadzenieDoAlgorytmow-18}
  \sum_{ k = 1 }^{ n } k \leq C n.
\end{equation}

Co więc było źródłem błędu w~podanym tu niepoprawnym dowodzie? Otóż w~kroku
indukcyjnym zawsze~dowodzimy naszego twierdzenia $T( n + 1 )$ przy
założeniu, że~prawdziwe jest twierdzenie $T( n )$, dla pewnego
\textit{ustalonego}~$n$. Zaś dla ustalonego $n$ zawsze istnieje liczba
$C \geq 0$, taka~że zachodzi
\begin{equation}
  \label{eq:CormenAtAl-WprowadzenieDoAlgorytmow-19}
  \sum_{ k = 1 }^{ n } k \leq C n.
\end{equation}
Natomiast notacja asymptotyczna, jak sama nazwa wskazuje, mówi o~zachowaniu
asymptotycznym funkcji, czyli, że~funkcja spełnia pewną własność dla
wszystkich $n \geq n_{ 0 }$, gdzie $n_{ 0 }$ jest pewną ustaloną liczbą. Zatem
gdy dowodzimy kroku indukcyjnego możemy z~twierdzenia $T( n )$ przy
ustalonym $n$ wywnioskować, że~nierówność typu
\eqref{eq:CormenAtAl-WprowadzenieDoAlgorytmow-19} jest prawdziwa dla
ustalonego $n$, ewentualnie dla liczb $0, 1, 2, \ldots, n$, gdzie $n$ jest
cały czas ustalone. Widzimy więc, że~na podstawie założenie $T( n )$
w~kroku indukcyjnym nie jesteśmy w~stanie nic powiedzieć o~zachowaniu
asymptotycznym danej funkcji, a~właśnie takiego rozumowania użyliśmy
w~przytoczonym tu dowodzie.

Płynie stąd wniosek, że~w~dowodach indukcyjnych najlepiej całkowicie unikać
notacji asymptotycznej. W~ostateczność, z~dużą dozą ostrożności można
korzystać w~nich z~relacji asymptotycznych, które zostały udowodnione innymi
metodami.

To co zostało powiedziane powyżej, nie przeczy oczywiście temu, że~stosując
w~sposób niedozwolony symbole asymptotyczne w~dowodach indukcyjnych otrzymano
wiele poprawnych twierdzeń. Takie sytuacje zapewne zdarzają~się dość
często.





















% ##################
\newpage

\CenterBoldFont{Błędy}


\begin{center}

  \begin{tabular}{|c|c|c|c|c|}
    \hline
    Strona & \multicolumn{2}{c|}{Wiersz} & Jest
                              & Powinno być \\ \cline{2-3}
    & Od góry & Od dołu & & \\
    \hline
    % & & & & \\
    % & & & & \\
    % & & & & \\
    % & & & & \\
    % & & & & \\
    % & & & & \\
    % & & & & \\
    % & & & & \\
    % & & & & \\
    % & & & & \\
    % & & & & \\
    % & & & & \\
    % & & & & \\
    \hline
  \end{tabular}

\end{center}

\vspace{\spaceTwo}



% ############################










% ####################################################################
% ####################################################################
% Bibliografia

\bibliographystyle{plalpha}

\bibliography{MathComScienceBooks}{}





% ############################

% Koniec dokumentu
\end{document}
