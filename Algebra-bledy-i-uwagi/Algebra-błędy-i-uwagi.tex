% Autor: Kamil Ziemian

% --------------------------------------------------------------------
% Podstawowe ustawienia i pakiety
% --------------------------------------------------------------------
\RequirePackage[l2tabu, orthodox]{nag}  % Wykrywa przestarzałe i niewłaściwe
% sposoby używania LaTeXa. Więcej jest w l2tabu English version.
\documentclass[a4paper,11pt]{article}
% {rozmiar papieru, rozmiar fontu}[klasa dokumentu]
\usepackage[MeX]{polski}  % Polonizacja LaTeXa, bez niej będzie pracował
% w języku angielskim.
\usepackage[utf8]{inputenc}  % Włączenie kodowania UTF-8, co daje dostęp
% do polskich znaków.
\usepackage{lmodern}  % Wprowadza fonty Latin Modern.
\usepackage[T1]{fontenc}  % Potrzebne do używania fontów Latin Modern.



% ----------------------------
% Podstawowe pakiety (niezwiązane z ustawieniami języka)
% ----------------------------
\usepackage{microtype}  % Twierdzi, że poprawi rozmiar odstępów w tekście.
\usepackage{graphicx}  % Wprowadza bardzo potrzebne komendy do wstawiania
% grafiki.
\usepackage{verbatim}  % Poprawia otoczenie VERBATIME.
\usepackage{textcomp}  % Dodaje takie symbole jak stopnie Celsiusa,
% wprowadzane bezpośrednio w tekście.
\usepackage{vmargin}  % Pozwala na prostą kontrolę rozmiaru marginesów,
% za pomocą komend poniżej. Rozmiar odstępów jest mierzony w calach.
% ----------------------------
% MARGINS
% ----------------------------
\setmarginsrb
{ 0.7in} % left margin
{ 0.6in} % top margin
{ 0.7in} % right margin
{ 0.8in} % bottom margin
{  20pt} % head height
{0.25in} % head sep
{   9pt} % foot height
{ 0.3in} % foot sep



% ----------------------------
% Często przydatne pakiety
% ----------------------------
\usepackage{csquotes}  % Pozwala w prosty sposób wstawiać cytaty do tekstu.
\usepackage{xcolor}  % Pozwala używać kolorowych czcionek (zapewne dużo
% więcej, ale ja nie potrafię nic o tym powiedzieć).



% ----------------------------
% Pakiety do tekstów z nauk przyrodniczych
% ----------------------------
\let\lll\undefined  % Amsmath gryzie się z językiem pakietami do języka
% polskiego, bo oba definiują komendę \lll. Aby rozwiązać ten problem
% oddefiniowuję tę komendę, ale może tym samym pozbywam się dużego Ł.
\usepackage[intlimits]{amsmath}  % Podstawowe wsparcie od American
% Mathematical Society (w skrócie AMS)
\usepackage{amsfonts, amssymb, amscd, amsthm}  % Dalsze wsparcie od AMS
% \usepackage{siunitx}  % Do prostszego pisania jednostek fizycznych
\usepackage{upgreek}  % Ładniejsze greckie litery
% Przykładowa składnia: pi = \uppi
\usepackage{slashed}  % Pozwala w prosty sposób pisać slash Feynmana.
\usepackage{calrsfs}  % Zmienia czcionkę kaligraficzną w \mathcal
% na ładniejszą. Może w innych miejscach robi to samo, ale o tym nic
% nie wiem.



% ##########
% Tworzenie otoczeń "Twierdzenie", "Definicja", "Lemat", etc.
\newtheorem{twr}{Twierdzenie}  % Komenda wprowadzająca otoczenie
% ,,twr'' do pisania twierdzeń matematycznych
\newtheorem{defin}{Definicja}  % Analogicznie jak powyżej
\newtheorem{wni}{Wniosek}



% ----------------------------
% Pakiety napisane przez użytkownika.
% Mają być w tym samym katalogu to ten plik .tex
% ----------------------------
\usepackage{reedsimon}  % Pakiet napisany konkretnie dla tego pliku.
\usepackage{latexshortcuts}
\usepackage{mathshortcuts}



% --------------------------------------------------------------------
% Dodatkowe ustawienia dla języka polskiego
% --------------------------------------------------------------------
\renewcommand{\thesection}{\arabic{section}.}
% Kropki po numerach rozdziału (polski zwyczaj topograficzny)
\renewcommand{\thesubsection}{\thesection\arabic{subsection}}
% Brak kropki po numerach podrozdziału



% ----------------------------
% Ustawienia różnych parametrów tekstu
% ----------------------------
\renewcommand{\arraystretch}{1.2}  % Ustawienie szerokości odstępów między
% wierszami w tabelach.



% ----------------------------
% Pakiet "hyperref"
% Polecano by umieszczać go na końcu preambuły.
% ----------------------------
\usepackage{hyperref}  % Pozwala tworzyć hiperlinki i zamienia odwołania
% do bibliografii na hiperlinki.





% ####################################################################
% Początek dokumentu
\begin{document}
% ####################################################################



% ######################################
\Main{Algebra, błędy i~uwagi}  % Tytuł całego tekstu

\vspace{\spaceTwo} \vspace{\spaceThree}
% ######################################

% ,,P\ldots'' oznacza, że w wydaniu ,,\ldots'' błąd został poprawiony.\\





% ##################
\Work{ % Autor i tytuł dzieła
  Andrzej Białynicki\dywiz Birula \\
  ,,Zarys algebry'', \cite{BialynickiBirulaZarysAlgebry87} }


\CenterTB{Uwagi}

\start \Str{11} Podany tu system aksjomatów jest arcynieporęczny,
przez co w~dalszych partiach książki nie~jest stosowany konsekwentnie.
Problem polega na tym, że~ponieważ wedle tej definicji 0 nie jest
liczbą naturalną, mając dwie liczby naturalne $p$, $q$, $p \geq q$ nie
wiemy, czy ich różnica $p - q$ jest liczbą naturalną. W~szczególności
wykonane w~dowodzie twierdzenia 3.3 dzielenie
$q_{ 1 } | p_{ 1 } - q_{ 1 }$ okazuje~się być operacją wykraczającą
poza teorię, bo $p_{ 1 } = q_{ 1 }$. Fakt, że~jest to dowód nie~wprost
(lub przez kontrapozycję), nie wydaje~się zmieniać istoty rzeczy.
Również w~twierdzeniu 3.5 jest mowa o~dodawaniu do liczby naturalnej
liczby całkowitej $r$, która może być równa 0, co uwidacznia potrzebę
rozszerzenia teorii. Jeżeli to jest możliwe, należałoby zmodyfikować
system Peano tak, by liczby naturalne uwzględniały 0.


\CenterTB{Błędy}
\begin{center}
  \begin{tabular}{|c|c|c|c|c|}
    \hline
    & \multicolumn{2}{c|}{} & & \\
    Strona & \multicolumn{2}{c|}{Wiersz} & Jest
                              & Powinno być \\ \cline{2-3}
    & Od góry & Od dołu & & \\
    \hline
    % & & & & \\
    14 & 14 & & $A^{ 2 } \quad A$ & $A^{ 2 } \ra A$ \\
    15 & 3 & & $( a^{ 1 } )^{ -1 }$ & $( a^{ -1 } )^{ -1 }$ \\
    16 & 13 & & \emph{lącznym} & \emph{łącznym} \\
    24 & 9 & & $if\big( \vp( a_{ 1 }, \ld, a_{ n } ) \big)$
           & $i \ci f\big( \vp( a_{ 1 }, \ld, a_{ n } ) \big)$ \\
    24 & 10 & & $if\big( \vp( a_{ 1 }, \ld, a_{ n } ) \big)$
           & $i \ci f\big( \vp( a_{ 1 }, \ld, a_{ n } ) \big)$ \\
    25 & & 1 & $s( a )$ & $[ s( a ) ]$ \\
    31 & & 4 & $\pi_{ t }( i' \ci i )$ & $\pi_{ t } \ci ( i' \ci i )$ \\
    33 & 21 & & \S{} 2 & \S{} 4 \\
    % & & & & \\
    % & & & & \\
    % & & & & \\
    291 & 11 & & 1974. & 1974). \\
    % & & & & \\
    % & & & & \\
    % & & & & \\
    \hline
  \end{tabular}
\end{center}

Powinno być:
\begin{itemize}
\item[--] Str. 37. \ldots$j k = -k j$\ldots
\item[--] Str. 80. \ldots więc $g h_{ 1 } g^{ -1 } \in G_{ y }$\ldots
\item[--] Str. 80. Istotnie, $h \in \varphi_{ x }^{ -1 }( g x )$\ldots
\item[--] Str. 81. \ldots orbit
  $H x_{ 1 } \cup \ldots \cup H x_{ l } \, .$
  % \item[--] Str.
  % \item[--] Str.
  % \item[--] Str.
  % \item[--] Str.
  % \item[--] Str.
  % \item[--] Str.
  % \item[--] Str.
\end{itemize}

\vspace{\spaceTwo}





% ####################
\Work{ % Autor i tytuł dzieła
  Andrzej Herdegen \\
  ,,Algebra liniowa i~geometria'',
  \cite{HerdegenAlgebraLiniowaIGeometria10} }


\CenterTB{Uwagi}


\CenterTB{Błędy}
\begin{center}
  \begin{tabular}{|c|c|c|c|c|}
    \hline
    & \multicolumn{2}{c|}{} & & \\
    Strona & \multicolumn{2}{c|}{Wiersz} & Jest
                              & Powinno być \\ \cline{2-3}
    & Od góry & Od dołu & & \\
    \hline
    8 & 6 & & $p$ ; & $p$; \\
    % & & & & \\
    % & & & & \\
    % & & & & \\
    % & & & & \\
    \hline
  \end{tabular}
\end{center}





% #####################################################################
% #####################################################################
% Bibliografia
\bibliographystyle{alpha} \bibliography{Bibliography}{}


% ############################

% Koniec dokumentu
\end{document}
