% Autor: Kamil Ziemian

% --------------------------------------------------------------------
% Podstawowe ustawienia i pakiety
% --------------------------------------------------------------------
\RequirePackage[l2tabu, orthodox]{nag}  % Wykrywa przestarzałe i niewłaściwe
% sposoby używania LaTeXa. Więcej jest w l2tabu English version.
\documentclass[a4paper,11pt]{article}
% {rozmiar papieru, rozmiar fontu}[klasa dokumentu]
\usepackage[MeX]{polski}  % Polonizacja LaTeXa, bez niej będzie pracował
% w języku angielskim.
\usepackage[utf8]{inputenc}  % Włączenie kodowania UTF-8, co daje dostęp
% do polskich znaków.
\usepackage{lmodern}  % Wprowadza fonty Latin Modern.
\usepackage[T1]{fontenc}  % Potrzebne do używania fontów Latin Modern.



% ----------------------------
% Podstawowe pakiety (niezwiązane z ustawieniami języka)
% ----------------------------
\usepackage{microtype}  % Twierdzi, że poprawi rozmiar odstępów w tekście.
% \usepackage{graphicx}  % Wprowadza bardzo potrzebne komendy do wstawiania
% % grafiki.
% \usepackage{verbatim}  % Poprawia otoczenie VERBATIME.
% \usepackage{textcomp}  % Dodaje takie symbole jak stopnie Celsiusa,
% % wprowadzane bezpośrednio w tekście.
\usepackage{vmargin}  % Pozwala na prostą kontrolę rozmiaru marginesów,
% za pomocą komend poniżej. Rozmiar odstępów jest mierzony w calach.
% ----------------------------
% MARGINS
% ----------------------------
\setmarginsrb
{ 0.7in} % left margin
{ 0.6in} % top margin
{ 0.7in} % right margin
{ 0.8in} % bottom margin
{  20pt} % head height
{0.25in} % head sep
{   9pt} % foot height
{ 0.3in} % foot sep



% ----------------------------
% Często przydatne pakiety
% ----------------------------
\usepackage{csquotes}  % Pozwala w prosty sposób wstawiać cytaty do tekstu.
\usepackage{xcolor}  % Pozwala używać kolorowych czcionek (zapewne dużo
% więcej, ale ja nie potrafię nic o tym powiedzieć).



% ----------------------------
% Pakiety do tekstów z nauk przyrodniczych
% ----------------------------
\let\lll\undefined  % Amsmath gryzie się z językiem pakietami do języka
% polskiego, bo oba definiują komendę \lll. Aby rozwiązać ten problem
% oddefiniowuję tę komendę, ale może tym samym pozbywam się dużego Ł.
\usepackage[intlimits]{amsmath}  % Podstawowe wsparcie od American
% Mathematical Society (w skrócie AMS)
\usepackage{amsfonts, amssymb, amscd, amsthm}  % Dalsze wsparcie od AMS
% \usepackage{siunitx}  % Do prostszego pisania jednostek fizycznych
\usepackage{upgreek}  % Ładniejsze greckie litery
% Przykładowa składnia: pi = \uppi
\usepackage{slashed}  % Pozwala w prosty sposób pisać slash Feynmana.
\usepackage{calrsfs}  % Zmienia czcionkę kaligraficzną w \mathcal
% na ładniejszą. Może w innych miejscach robi to samo, ale o tym nic
% nie wiem.



% ##########
% Tworzenie otoczeń "Twierdzenie", "Definicja", "Lemat", etc.
\newtheorem{twr}{Twierdzenie}  % Komenda wprowadzająca otoczenie
% ,,twr'' do pisania twierdzeń matematycznych
\newtheorem{defin}{Definicja}  % Analogicznie jak powyżej
\newtheorem{wni}{Wniosek}



% ----------------------------
% Pakiety napisane przez użytkownika.
% Mają być w tym samym katalogu to ten plik .tex
% ----------------------------
% \usepackage{reedsimon}  % Pakiet napisany konkretnie dla tego pliku.
\usepackage{latexshortcuts}
\usepackage{mathshortcuts}



% --------------------------------------------------------------------
% Dodatkowe ustawienia dla języka polskiego
% --------------------------------------------------------------------
\renewcommand{\thesection}{\arabic{section}.}
% Kropki po numerach rozdziału (polski zwyczaj topograficzny)
\renewcommand{\thesubsection}{\thesection\arabic{subsection}}
% Brak kropki po numerach podrozdziału



% ----------------------------
% Ustawienia różnych parametrów tekstu
% ----------------------------
\renewcommand{\arraystretch}{1.2}  % Ustawienie szerokości odstępów między
% wierszami w tabelach.



% ----------------------------
% Pakiet "hyperref"
% Polecano by umieszczać go na końcu preambuły.
% ----------------------------
\usepackage{hyperref}  % Pozwala tworzyć hiperlinki i zamienia odwołania
% do bibliografii na hiperlinki.





% --------------------------------------------------------------------
% Tytuł, autor, data
\title{Algebra --~błędy i~uwagi}

% \author{}
% \date{}
% --------------------------------------------------------------------





% ####################################################################
% Początek dokumentu
\begin{document}
% ####################################################################



% ######################################
\maketitle  % Tytuł całego tekstu
% ######################################

% ,,P\ldots'' oznacza, że w wydaniu ,,\ldots'' błąd został poprawiony.\\





% ##################
\Work{ % Autor i tytuł dzieła
  Andrzej Białynicki\dywiz Birula \\
  ,,Zarys algebry'', \cite{BialynickiBirulaZarysAlgebry87} }


\CenterTB{Uwagi}

\start \Str{11} Podany tu system aksjomatów jest arcynieporęczny,
przez co w~dalszych partiach książki nie~jest stosowany konsekwentnie.
Problem polega na tym, że~ponieważ wedle tej definicji 0 nie jest
liczbą naturalną, mając dwie liczby naturalne $p$, $q$, $p \geq q$ nie
wiemy, czy ich różnica $p - q$ jest liczbą naturalną. W~szczególności
wykonane w~dowodzie twierdzenia 3.3 dzielenie
$q_{ 1 } | p_{ 1 } - q_{ 1 }$ okazuje~się być operacją wykraczającą
poza teorię, bo $p_{ 1 } = q_{ 1 }$. Fakt, że~jest to dowód nie~wprost
(lub przez kontrapozycję), nie wydaje~się zmieniać istoty rzeczy.
Również w~twierdzeniu 3.5 jest mowa o~dodawaniu do liczby naturalnej
liczby całkowitej $r$, która może być równa 0, co uwidacznia potrzebę
rozszerzenia teorii. Jeżeli to jest możliwe, należałoby zmodyfikować
system Peano tak, by liczby naturalne uwzględniały 0.


\CenterTB{Błędy}
\begin{center}
  \begin{tabular}{|c|c|c|c|c|}
    \hline
    & \multicolumn{2}{c|}{} & & \\
    Strona & \multicolumn{2}{c|}{Wiersz} & Jest
                              & Powinno być \\ \cline{2-3}
    & Od góry & Od dołu & & \\
    \hline
    % & & & & \\
    14 & 14 & & $A^{ 2 } \quad A$ & $A^{ 2 } \to A$ \\
    15 & 3 & & $( a^{ 1 } )^{ -1 }$ & $( a^{ -1 } )^{ -1 }$ \\
    16 & 13 & & \emph{lącznym} & \emph{łącznym} \\
    24 & 9 & & $if\big( \vp( a_{ 1 }, \ld, a_{ n } ) \big)$
           & $i \circ f\big( \vp( a_{ 1 }, \ld, a_{ n } ) \big)$ \\
    24 & 10 & & $if\big( \vp( a_{ 1 }, \ld, a_{ n } ) \big)$
           & $i \circ f\big( \vp( a_{ 1 }, \ld, a_{ n } ) \big)$ \\
    25 & & 1 & $s( a )$ & $[ s( a ) ]$ \\
    31 & & 4 & $\pi_{ t }( i' \circ i )$ & $\pi_{ t } \circ ( i' \circ i )$ \\
    33 & 21 & & \S{} 2 & \S{} 4 \\
    % & & & & \\
    % & & & & \\
    % & & & & \\
    291 & 11 & & 1974. & 1974). \\
    % & & & & \\
    % & & & & \\
    % & & & & \\
    \hline
  \end{tabular}
\end{center}

Powinno być:
\begin{itemize}
\item[--] Str. 37. \ldots$j k = -k j$\ldots
\item[--] Str. 80. \ldots więc $g h_{ 1 } g^{ -1 } \in G_{ y }$\ldots
\item[--] Str. 80. Istotnie, $h \in \varphi_{ x }^{ -1 }( g x )$\ldots
\item[--] Str. 81. \ldots orbit
  $H x_{ 1 } \cup \ldots \cup H x_{ l } \, .$
  % \item[--] Str.
  % \item[--] Str.
  % \item[--] Str.
  % \item[--] Str.
  % \item[--] Str.
  % \item[--] Str.
  % \item[--] Str.
\end{itemize}

\vspace{\spaceTwo}





% ######################################
\newpage
\section{Algebra liniowa}

\vspace{\spaceTwo}
% ######################################



% ####################
\Work{ % Autor i tytuł dzieła
  Jacek Gancarzewicz \\
  ,,Algebra liniowa i jej zastosowania'', \cite{Gan04} }


Uwagi:
\begin{itemize}
\item[--] \Str{34} Aby udowodnić punkt (2) twierdzenia 3.7, wystarczy
  przyjąć, że~$x \neq 0$ i~pomnożyć równość $xy = 0$ z~lewej strony
  przez $x^{ -1 }$.
\item W pierwszym rozdziale nie określono znaku permutacji
  \\identycznościowej.
\item Nie określono operacji odejmowania wektorów. Z definicji jest
  to:$$x - y := x + ( -y ) \, .$$
\item Problem sumowania po pustym zbiorze wskaźników nie został
  omówiony. Powinno oczywiście być
  $\sum_{ \substack{ \iota \in \emptyset } } v_{ \iota } = 0 \, .$
\item Str. 40. Jest pewna luka w dowodzie drugiej wersji zasadniczego
  twierdzenia algebry. Procedura dzielenia pokazuje bowiem, że po n
  krokach wielomian jest stopnia 0, czyli musi być stałą, nie pokazano
  jednak, że stała ta wynosi $a_{ n }$. Aby to pokazać należy
  stwierdzić, co można pokazać indukcyjnie, że w iloczynach typu
  $( z - z_{ 1 } )^{ \alpha_{ 1 } } \ldots( z - z_{ j } )^{ \alpha_{ j
    } }$ wyraz przy najniższej potędze wynosi 1 i następnie skorzystać
  z twierdzenia, że dwa wielomiany w ciele liczb zespolonych są równe
  wtedy i tylko wtedy gdy wszystkie ich współczynniki są równe.
\item Str. 60. W ostatniej permutacji trzeba jedną 7 zastąpić 4.
\item Str. 76. Błąd w numeracji podpunktów twierdzenia.
  % \item Str.
\item Str. 341. W twierdzeniu 45.1 powinno być $n \geq 1 \, .$ Dla n =
  0 istnieje 0 punktów afinicznie niezależnych.
\end{itemize}

Błędy:\\
\begin{center}
  \begin{tabular}{|c|c|c|c|c|}
    \hline
    & \multicolumn{2}{c|}{} & & \\
    Strona & \multicolumn{2}{c|}{Wiersz} & Jest
                              & Powinno być \\ \cline{2-3}
    & Od góry & Od dołu & & \\
    \hline
    % & & & & \\
    28  & & 13 & $x \neq 0$ & $x \neq y$ \\
    34  & & 12 & $x( y y^{ -1 } x^{ -1 }$ & $x( y y^{ -1 } ) x^{ -1 }$ \\
    59  & 16 & & $x_{ 3 }$ & $x_{ 1 }$ \\
    % & & & & \\
    % & & & & \\
    \hline
  \end{tabular}
\end{center}

Powinno być:
\begin{itemize}
\item[--] Str. 41. \ldots to funkcja kwadratowa $z^2+az+b$\ldots
\item[--] Str. 44.
  $$\overline{ \xi } = \overline{ z_{ 1 } } - z_{ 2 } j \, .$$
\item[--] Str. 59.
  $( x_{ 1 }, \ldots, x_{ k } ) \leq (y_{ 1 }, \ldots, y_{ k } ) \, .$
\item[--] Str. 62. \ldots i skalara $a \in F$\ldots
\item[--] Str. 64.
  $$x = 1 x = ( a^{ -1 } a ) x = a^{ -1 }( a x ) = a^{ -1 } 0 = 0 \,
  .$$
\item[--] Str. 66. Wymieniając w twierdzeniu 8.8 równoważne warunki,
  otrzymujemy\ldots
\item[--] Str. 75. \ldots otrzymujemy
  $f( a f^{ -1 }( y ) + b f^{ -1 } ( y' ) ) = a y + b y'$\ldots
\item[--] Str. 76. Na podstawie twierdzenia 1.2 punkt 1.
\item[--] Str. 86. \ldots z zasady kontrapozycji.
\item[--] Str. 110. $( a_{ 0 }, a_{ 1 }, a_{ 2 }, \ldots )$.
\item[--] Str. 121. \ldots otrzymujemy
  $\alpha = \sum_{ i = 1 }^{ n } \lambda_{ i } e^{ * }_{ i } .$
\item[--] Str. 134. \ldots nie występują one w tezie twierdzenia.
\item[--] Str. 143. \ldots bo jeżeli dla $\alpha \in U^{ * }$\ldots
\item[--] Str. 143.
  $$f( e_{ i } ) = \sum_{ j = 1 }^{ m } a_{ j i } \overline{ { e } }_{
    j } \, , \qquad f'( e'_{ i } ) = \sum_{ j = 1 }^{ m } a_{ j i }
  \overline{ { e }' }_{ j } \, ,$$
\item[--] Str. 147.
  $$f : M( m, 1; F ) = F^{ m } \ni x \longrightarrow A x \in M( n, 1 ;
  F ) = F^{ n } \, ,$$
\item[--] Str. 195.
  \ldots$\mathcal{ J }_{ 1 } : F \otimes X \rightarrow X$ \emph{oraz}
  $\mathcal{ J }_{ 2 } : X \otimes F \rightarrow X$\ldots
\item[--] Str. 223. \ldots\emph{gdzie $K \in M( p; F )$ oraz}\ldots
\item[--] Str. 238. Niech $x_{ 1 }, \ldots, x_{ n } \in F ,$
  będzie\ldots
\item[--] Str. 238. \ldots przez punkty $x_{ 1 }$,\ldots,$x_{ n }$
  nazywamy wyznacznik
$$\mathcal{ V }_{ x_{ 1 }, \ldots, x_{ n } } = \ldots$$
\item[--] Str. 238. Jeżeli wśród punktów
  $x_{ 1 },\ldots, x_{ n }$\ldots
\item[--] Str. 238. \ldots że
  $\mathcal{ V }_{ x_{ 1 }, \ldots, x_{ n } } \neq 0 .$
\item[--] Str. 238.
  $$\mathcal{ V }_{ x_{ 1 }, \ldots, x_{ n } } = \prod_{ 1 \leq i < j
    \leq n - 1 }( x_{ j } - x_{ i } ) \neq 0 \, .$$
\item[--] Str. 238. \ldots pierwiastków
  $x_{ 1 }, \ldots, x_{ n - 1 }$\ldots
\item[--] Str. 253. \ldots \emph{nieujemnych} $p, q, p', q'$\ldots
\item[--] Str. 269. \ldots liniowe
  $\rho_{ V }^{ p } : \Lambda^{ p } V^{ * } \rightarrow L^{ p }_{ a }
  ( V )$\ldots
\item[--] Str. 269. \begin{displaymath}
    \begin{split}
      (\rho_{ V }^{ p } \circ \Lambda^{ p }( f^{ * } ) ) ( \beta_{ 1 } \wedge \ldots \wedge \beta_{ p } ) &= \rho_{ V }^{ p } ( f^{ * }( \beta_{ 1 } ) \wedge \ldots \wedge f^{ * } ( \beta_{ p } ) ) \\
      &= \rho_{ V }^{ p } (\beta_{ 1 } \circ f \wedge \ldots \wedge \beta_{ p } \circ f )\\
      &= u_{ V }^{ p } ( \beta_{ 1 } \circ f, \ldots, \beta_{ p }
      \circ f ) \, .
    \end{split}
  \end{displaymath}
\item[--] Str. 272. \ldots że lewa strona równości\ldots
\item[--] Str. 272. Obliczmy prawą stronę równości\ldots
\item[--] Str. 272. $$P = \frac{ 1 }{ ( p + q )! p! q! } \ldots$$
\item[--] Str. 272. \begin{displaymath}
    \begin{split}
      P \quad =& \quad \frac{ 1 }{ ( p + q )! p! q! } \sum_{ \sigma \in S_{ p + q } } \sum_{ \rho \in  S_{ p } } \sum_{ \tau \in S_{ q } } \textrm{sgn} \, \sigma \, \textrm{sgn} \, \rho \, \textrm{sgn} \, \tau \\
      & \quad \qquad \alpha_{ 1 } ( v_{ \sigma( 1 ) } ) \ldots \alpha_{ p } ( v_{ \sigma ( p ) } ) \alpha_{ p + 1 } ( v_{ \sigma( p + 1 ) } ) \ldots \alpha_{ p + q } ( v_{ \sigma ( p + q ) } ) \\
      = & \quad \frac{ p! q! }{ ( p + q )! p! q! } \sum_{ \sigma \in S_{ p + q } } \textrm{sgn} \, \sigma \\
      & \quad \qquad \alpha_{ 1 } ( v_{ \sigma( 1 ) } ) \ldots
      \alpha_{ p } ( v_{ \sigma ( p ) } ) \alpha_{ p + 1 } ( v_{
        \sigma( p + 1 ) } ) \ldots \alpha_{ p + q } ( v_{ \sigma ( p +
        q ) } ) \, .
    \end{split}
  \end{displaymath}
\item[--] Str. 276. \ldots gdyż macierz $A$\ldots
\item[--] Str. 304. \ldots indukowane
  $T^{ p } ( f ) : T^{ p } ( V ) \rightarrow T^{ p } (W)$\ldots
\item[--] Str. 322. \ldots element
  $\omega \in L( X_{ 1 } ,\ldots, X_{ k }; Y )$ \ldots
\item[--] Str. 324. \ldots w przypadku
  $X_{ 1 } = \ldots = X_{ k } = X$\ldots
\item[--] Str. 328. Odwzorowanie
  $A_{ X } : \bigotimes^{ k } X \rightarrow \Lambda^{ k } X$\ldots
\item[--] Str. 328. \ldots dla rodziny $\{ A_{ X } \} \, .$
\item[--] Str. 329. \ldots $k$-liniowe
  $u^{ a }_{ X } : X \times \ldots \times X \rightarrow \bigotimes^{ k
  } X$ i
  $u^{ s }_{ X } : X \times \ldots \times X \rightarrow \bigotimes^{ k
  } X$\ldots
  % \item[--] Str. 354. Zadanie \romannumeral6.7a zostało rozwiązane w
  %   twierdzeniu 44.11.
\item[--] Str. 363. \ldots należą do $M( 0; F )$\ldots
\item[--] Str. 364. \ldots oraz
  $v = p( f |_{ U } ) ( v ) \in U_{ 1 }$\ldots
\end{itemize}





% ####################
\Work{ % Autor i tytuł dzieła
  Andrzej Herdegen \\
  ,,Algebra liniowa i~geometria'',
  \cite{HerdegenAlgebraLiniowaIGeometria10} }


\CenterTB{Uwagi}


\CenterTB{Błędy}
\begin{center}
  \begin{tabular}{|c|c|c|c|c|}
    \hline
    & \multicolumn{2}{c|}{} & & \\
    Strona & \multicolumn{2}{c|}{Wiersz} & Jest
                              & Powinno być \\ \cline{2-3}
    & Od góry & Od dołu & & \\
    \hline
    8   & 6 & & $p$ ; & $p$; \\
    % & & & & \\
    % & & & & \\
    % & & & & \\
    % & & & & \\
    \hline
  \end{tabular}
\end{center}





\begin{center}
  Andrzej Staruszkiewicz \\
  ,,Algebra i geometria. Wykłady dla fizyków. Tom I'', \cite{ASAG}.
\end{center}


Powinno być:
\begin{itemize}
\item[--] Str. 14. \ldots $\det A \neq 0$, to
  $\textrm{rz} ( AB ) = \textrm{rz} ( B ) \, .$
  % \item[--] Str.
  % \item[--] Str.
  % \item[--] Str.
  % \item[--] Str.
  % \item[--] Str.
  % \item[--] Str.
  % \item[--] Str.
  % \item[--] Str.
\end{itemize}


\begin{center}
  Henryk Arodź, Krzysztof Rościszewski\\
  ,,Algebra i geometria analityczna w zadaniach'', \cite{AR}.
\end{center}


Uwagi:
\begin{itemize}
\item Str. 64. Brakuje macierzy $\bd{I} \, .$
  % \item
  % \item
  % \item
  % \item
  % \item
  % \item
\end{itemize}

Błędy:\\
\begin{tabular}{|c|c|c|c|c|}
  \hline
  & \multicolumn{2}{c|}{} & & \\
  Strona & \multicolumn{2}{c|}{Wiersz} & Jest
                            & Powinno być \\ \cline{2-3}
  & Od góry & Od dołu & & \\ 
  \hline
  & & & & \\
  & & & & \\
  & & & & \\
  & & & & \\
  \hline
\end{tabular}

Powinno być:
\begin{itemize}
\item[--] Str. 15.
  $$\sqrt[ n ]{ z } = \sqrt[ n ]{ | z | } \exp( i \frac{ \varphi + 2 k
    \pi }{ n } ) \, .$$
\item[--] Str. 15. \ldots $Re \; z = a$\ldots
  % \item[--] Str.
  % \item[--] Str.
\item[--] Str. 74. (14) $6 \boldsymbol{ \gamma } \, ,$
\end{itemize}






% #####################################################################
% #####################################################################
% Bibliografia
\bibliographystyle{plalpha} \bibliography{LibMathInfo}{}


% ############################

% Koniec dokumentu
\end{document}
