% Autor: Kamil Ziemian

% --------------------------------------------------------------------
% Podstawowe ustawienia i pakiety
% --------------------------------------------------------------------
\RequirePackage[l2tabu, orthodox]{nag}  % Wykrywa przestarzałe i niewłaściwe
% sposoby używania LaTeXa. Więcej jest w l2tabu English version.
\documentclass[a4paper,11pt]{article}
% {rozmiar papieru, rozmiar fontu}[klasa dokumentu]
\usepackage[MeX]{polski}  % Polonizacja LaTeXa, bez niej będzie pracował
% w języku angielskim.
\usepackage[utf8]{inputenc}  % Włączenie kodowania UTF-8, co daje dostęp
% do polskich znaków.
\usepackage{lmodern}  % Wprowadza fonty Latin Modern.
\usepackage[T1]{fontenc}  % Potrzebne do używania fontów Latin Modern.



% ----------------------------
% Podstawowe pakiety (niezwiązane z ustawieniami języka)
% ----------------------------
\usepackage{microtype}  % Twierdzi, że poprawi rozmiar odstępów w tekście.
\usepackage{graphicx}  % Wprowadza bardzo potrzebne komendy do wstawiania
% grafiki.
\usepackage{verbatim}  % Poprawia otoczenie VERBATIME.
\usepackage{textcomp}  % Dodaje takie symbole jak stopnie Celsiusa,
% wprowadzane bezpośrednio w tekście.
\usepackage{vmargin}  % Pozwala na prostą kontrolę rozmiaru marginesów,
% za pomocą komend poniżej. Rozmiar odstępów jest mierzony w calach.
% ----------------------------
% MARGINS
% ----------------------------
\setmarginsrb
{ 0.7in} % left margin
{ 0.6in} % top margin
{ 0.7in} % right margin
{ 0.8in} % bottom margin
{  20pt} % head height
{0.25in} % head sep
{   9pt} % foot height
{ 0.3in} % foot sep



% ----------------------------
% Często używane pakiety
% ----------------------------
\usepackage{csquotes}  % Pozwala w prosty sposób wstawiać cytaty do tekstu.
\usepackage{xcolor}  % Pozwala używać kolorowych czcionek (zapewne dużo
% więcej, ale ja nie potrafię nic o tym powiedzieć).





% --------------------------------------------------------------------
% Dodatkowe ustawienia dla języka polskiego
% --------------------------------------------------------------------
\renewcommand{\thesection}{\arabic{section}.}
% Kropki po numerach rozdziału (polski zwyczaj topograficzny)
\renewcommand{\thesubsection}{\thesection\arabic{subsection}}
% Brak kropki po numerach podrozdziału



% ----------------------------
% Pakiety napisane przez użytkownika.
% Mają być w tym samym katalogu to ten plik .tex
% ----------------------------
\usepackage{latexshortcuts}



% ----------------------------
% Ustawienia różnych parametrów tekstu
% ----------------------------
\renewcommand{\arraystretch}{1.2}  % Ustawienie szerokości odstępów między
% wierszami w tabelach.



% ----------------------------
% Pakiet "hyperref"
% Polecano by umieszczać go na końcu preambuły.
% ----------------------------
\usepackage{hyperref}  % Pozwala tworzyć hiperlinki i zamienia odwołania
% do bibliografii na hiperlinki.






% ####################################################################
% Początek dokumentu
\begin{document}
% ####################################################################



% ######################################
\Main{Kultura ogólna --~błędy i~uwagi}

\vspace{\spaceTwo} \vspace{\spaceThree}
% ######################################



% ##################
\Work{ % Autor i tytuł dzieła
  E. Michael Jones \\
  ,,Zdeprawowani moderniści'',
  \cite{EMichaelJonesZdeprawowaniModernisci14} }


\CenterTB{Błędy}
\begin{center}
  \begin{tabular}{|c|c|c|c|c|}
    \hline
    & \multicolumn{2}{c|}{} & & \\
    Strona & \multicolumn{2}{c|}{Wiersz} & Jest
                              & Powinno być \\ \cline{2-3}
    & Od góry & Od dołu & & \\
    \hline
    16  & & 19 & Claya & Gaya \\
    24  &  4 & & człowieczeństwa & człowieczeństwa'' \\
    25  & 11 & & Samoa & ,,Samoa  % ''
    \\
    25  & & 14 & wyłączności & Wyłączności \\
    27  & &  1 & beztroskich>> & beztroskich>>'' \\
    29  & 12 & & roku & roku. \\
    30  & &  9 & <<młodej studentki''  % >>
           & <<młodej studentki>> \\
    30  & &  8 & bawełniane sukienki>> & <<bawełniane sukienki>> \\
    37  & 11 & & wówczas'' & wówczas \\
    37  & & 16 & niego & z~niego \\
    38  & &  2 & zbagatelizowałaś, & zbagatelizowałaś. \\
    49  & 12 & & za & z \\
    56  & 17 & & którym & których \\
    56  &  3 & & Wilberforce'a** & Wilberforce'a* \\
    67  & & 13 & \emph{Whay} & \emph{What} \\
    71  & 14 & & ,, Wartości  % ''
           & ,,Wartości  % ''
    \\
    85  & & 13 & \emph{lalek} & \emph{lalek}'' \\
    91  & &  4 & a oni & ,,a oni  % ''
    \\
    99  &  9 & & maja & mają \\
    101 & & 11 & naukowym & z naukowym \\
    103 & & 13 & ATA & ATS \\
    105 & 19 & & mniej silna & silniejsza \\
    107 & 19 & & Voris & Vorisem \\
    108 &  2 & & rzeczywistości & o rzeczywistości \\
    111 & 21 & & Indiana, & Indiana. \\
    111 & & 20 & w na & na \\
    119 & & 11 & oszukiwałam''$^{ 2 }$.W & oszukiwałam''$^{ 2 }$. W \\
    122 & 20 & & od & do \\
    161 & &  2 & wyraźn0e & wyraźne \\
    165 & &  3 & ktrego & którego \\
    \hline
  \end{tabular}

  \begin{tabular}{|c|c|c|c|c|}
    \hline
    & \multicolumn{2}{c|}{} & & \\
    Strona & \multicolumn{2}{c|}{Wiersz} & Jest
                              & Powinno być \\ \cline{2-3}
    & Od góry & Od dołu & & \\
    \hline
    171 & 12 & & zajmującysię & zajmujący~się \\
    186 & &  5 & umożliwiła & uniemożliwiła \\
    199 & & 15 & wiary & utraty wiary \\
    200 & &  1 & Mogło & Mogła \\
    201 & & 14 & roku & roku. \\
    205 & & 19 & Laetesa & Klaudiusza \\
    207 &  7 & & taka & taką \\
    211 & & 18 & ,,Szczyt  % ''
           & Szczyt \\
    214 &  5 & & [ w~nagrodę] & [w~nagrodę] \\
    214 &  6 & & najwyraźniej .. & najwyraźniej\ld \\
    217 & 22 & & Fread & Freuda \\
    218 & &  2 & od~z & z \\
    226 & 10 & & Mannung & Manning \\
    228 & & 17 & pytanie:, & pytanie: \\
    230 & 12 & & 1963 roku & 1963 roku. \\
    230 & 13 & & 1425 & 1525 \\
    237 & &  4 & 560 & 1560 \\ \hline
  \end{tabular}
\end{center}
\noi \\
\StrWd{17}{10} \\
\Jest nieczystości pierworodnej córki jest ślepotą ducha. \\
\Pow pierworodną córką nieczystości jest ślepota ducha. \\
\StrWg{201}{8} \\
\Jest otorbił~się w~kokonie psychoanalizy samego siebie\ldots \\
\Pow otorbił samego siebie w~kokonie psychoanalizy\ldots \\
\StrWg{209}{20} \\
\Jest \emph{Gdyby potrafił być perwersyjny, byłby zdrowy, podobnie jak
  ojciec$^{ 202 }$}. \\
\Pow ,,Gdyby potrafił być perwersyjny, byłby zdrowy,
podobnie jak ojciec''$^{ 202 }$. \\

\vspace{\spaceTwo}





% ######################################
\newpage
\Field{Kultura japońska}

\vspace{\spaceThree}
% ######################################


% ##################
\Work{ % Autor i tytuł dzieła
  Renata Iwicka \\
  ,,Źródła klasycznej demonologii japońskiej'',
  \cite{IwickaZrodlaKlasycznejDemonologiJaponskiej17} }


\CenterTB{Uwagi}

\start \StrWg{128}{7} To~zdanie jest sformułowane w~taki sposób,
że~nie rozumiem ani~jak położona jest oś~o~której mowa, ani jak
nazywają~się leżące na~niej bramy.

\vspace{\spaceFour}


\start \StrWg{129}{12--16} Fragment ten jest napisany w~taki sposób,
że~nie byłem w~stanie zrozumieć z~niego, która pisownia nazwy bramy
jest pierwotna, a~która powstała później. Z~reszty książki wynika,
że~pierwotna nazwa to ,,Rajomon'',  % Raj\={o}mon
późniejsza zaś~to ,,Rashomon''.  % Rash\={o}mon

\CenterTB{Błędy}
\begin{center}
  \begin{tabular}{|c|c|c|c|c|}
    \hline
    & \multicolumn{2}{c|}{} & & \\
    Strona & \multicolumn{2}{c|}{Wiersz} & Jest
                              & Powinno być \\ \cline{2-3}
    & Od góry & Od dołu & & \\
    \hline
    10  & &  2 & Kioto, 2012 & Kioto 2012 \\
    13  & &  6 & \emph{Yokai} Database: & \emph{Yokai Database}: \\
    % Nad ,,o'' w ,,Yokai'' powinna być kreska akcentu.
    % & & & & \\
    % & & & & \\
    % & & & & \\
    125 & &  6 & spostrzegania & postrzegania \\
    % & & & & \\
    % & & & & \\
    \hline
  \end{tabular}
\end{center}

\vspace{\spaceTwo}





% ####################
\Work{
  Zdzisław Krasnodębski \\
  ,,Zwycięzca po~przejściach. Zebrane eseje i~szkice~V'',
  \cite{KrasnodebskiZwyciezkaPoPrzejsciach12} }


\CenterTB{Uwagi}

\start \Str{317} Pominięto miejsce i~datę pierwszej publikacji
artykułu \emph{Nie udawaj Greka, Polsko!}

\vspace{\spaceTwo}





% ##################
\Work{ % Autor i tytuł dzieła
  Jan Sowa \\
  ,,Fantomowe ciało króla. Peryferyjne zmagania z~nowoczesną formą'',
  \cite{SowaFantomoweCialoKrola11} }


\CenterTB{Uwagi}

\start \StrWg{36}{15} Zdanie ,,jego aktywność ściśle wiąże~się
związana ze społeczną\ldots'', powinno brzmieć ,,jego aktywność ściśle
wiąże~się ze społeczną\ldots'' lub ,,jego aktywność jest ściśle
związana ze społeczną\ldots''. Choć na podstawie tekstu, nie można
wybrać wersji zamierzonej przez autora, nie~jest to jednak problemem,
bo obie przekazują tą samą treść.

\vspace{\spaceFour}


\start \StrWg{63}{5--9} Zawarty tu tekst, jest źle skonstruowany
gramatycznie.

\CenterTB{Błędy}
\begin{center}
  \begin{tabular}{|c|c|c|c|c|}
    \hline
    & \multicolumn{2}{c|}{} & & \\
    Strona & \multicolumn{2}{c|}{Wiersz} & Jest
                              & Powinno być \\ \cline{2-3}
    & Od góry & Od dołu & & \\
    \hline
    92  & &  6 & 1140 & 1440 \\
    102 & 11 & & a bo & bo \\
    118 &  8 & & dawały & nie dawały \\
    300 & 17 & & mogły & mogło \\
    362 & 12 & & torrusa & torusa \\
    367 & 15 & & c\tb{oś} & \tb{coś} \\
    % & & & & \\
    \hline
  \end{tabular}
\end{center}

\vspace{\spaceTwo}





% \Work{
% Paul Johnson\\
% ,,Narodziny nowoczesności'', \cite{Joh95}.}


% \CenterTB{Błędy}

% \begin{center}
%   \begin{tabular}{|c|c|c|c|c|}
%     \hline
%     & \multicolumn{2}{c|}{} & & \\
%           %     Strona & \multicolumn{2}{c|}{Wiersz} & Jest
%     %     & Powinno być \\ \cline{2-3}
%           %     & od góry & od dołu & & \\
%     %     \hline
%     & & & & \\
%     29 & 2 & & cali, członie & cali, o członie \\
%     142 & 1 & & Barbaji & Barbajowi \\
%     142 & & 14 & w nową operą & z nową operą \\
%     345 & 15 & & XIX & XVIII \\
%     345 & 18 & & od & na od \\
%     409 & 8 & & sposób & nie sposób \\ \hline
%   \end{tabular}
% \end{center}






% \Work{
% J. Kofman, W. Roszkowski \\
% ,,Transformacja i postkomunizm'', \cite{KR99}.}


% \CenterTB{Błędy}

% \begin{center}
%   \begin{tabular}{|c|c|c|c|c|}
%     \hline
%     & \multicolumn{2}{c|}{} & & \\
%           %     Strona & \multicolumn{2}{c|}{Wiersz} & Jest
%     %     & Powinno być \\ \cline{2-3}
%           %     & od góry & od dołu & & \\
%     %     \hline
%     & & & & \\
%     10 & & 17 & jedynie & jedynej \\
%     13 & & 4 & o ekspansji & do ekspansji \\
%     16 & 9 & & marntrawstwem & marnotrawstwem \\
%     & & & & \\
%     & & & &  \\ \hline
%   \end{tabular}
% \end{center}





% \Work{
% Red. A. Nowak \\
% ,,Historie Polski w~XIX wieku. Tom I: Kominy, ludzie i~obłoki:
% modernizacja i~kultura.'', \cite{HPXIX1}.}

% % Uwagi:
% % \begin{itemize}
% % \item[--] \Str{45} T.~S.~ Eliot jest na tej stronie nazwany
% %   ,,wielkim poetą katolickim'', acz z~tego co wiem do Kościoła
% %   nigdy
% %   nie przyszedł, zamiast tego dołączył do jakiegoś wyznania
% %   anglokatolickiego. Zaś przymiotnik ,,wielki'' w~odniesieniu to
% %   tego poety, którego twórczości nie da~się czytać, jest już
% %   na~pewno błędny.
% % \end{itemize}

% \CenterTB{Błędy}

% \begin{center}
%   \begin{tabular}{|c|c|c|c|c|}
%     \hline
%     & \multicolumn{2}{c|}{} & & \\
%           %     Strona & \multicolumn{2}{c|}{Wiersz} & Jest
%     %     & Powinno być \\ \cline{2-3}
%           %     & od góry & od dołu & & \\
%     %     \hline
%     & & & & \\
%     16 & & 15 & równości równość & równości \\ \hline
%   \end{tabular}
% \end{center}






% \Work{
% A. K. Wróblewski \\
% ,,Historia fizyki'', \cite{Wro06}.}


% Błędy:\\
% \begin{center}
%   \begin{tabular}{|c|c|c|c|c|}
%     \hline
%     %     & \multicolumn{2}{c|}{} & & \\
%       & \multicolumn{2}{c|}{Wiersz} & & \\ \cline{2-3}
%       %       Strona & od góry & od dołu & Jest & Powinno być \\
%       & (kolumna) & (kolumna) & & \\ \hline
%       & & & & \\
%       %       203 & 3 (2) & & Jacob 'sGravesande'a & Jacob's
%       %       Gravesande'a \\
%       & & & & \\ \hline
%   \end{tabular}
% \end{center}




% ########################################
\newpage
\Field{Kino, film -- błędy i~uwagi}

\vspace{\spaceThree}
% ########################################


% ##################
\Work{ % Redaktor i tytuł dzieła
  Red. Piotr Kletowski \\
  ,,Europejskie kino gatunków'',
  \cite{RedKletowskiEuropejskieKinoGatunkow16} }


\CenterTB{Uwagi}

\start \StrWd{129}{1} W~filmie templariuszom nie wyłupiono oczu, lecz
stracono i~powieszono na drzewach. Tam dzikie ptaki wyjadły ich oczy.

\CenterTB{Błędy}
\begin{center}
  \begin{tabular}{|c|c|c|c|c|}
    \hline
    & \multicolumn{2}{c|}{} & & \\
    Strona & \multicolumn{2}{c|}{Wiersz} & Jest
                              & Powinno być \\ \cline{2-3}
    & Od góry & Od dołu & & \\
    \hline
    87 & 11 & & wybraną & jedną wybraną \\
    % & & & & \\
    % & & & & \\
    % & & & & \\
    % & & & & \\
    \hline
  \end{tabular}
\end{center}
\noi
\StrWg{160}{12} \\
\Jest \emph{Za kilka dolarów więcej} Monco \\
\Pow filmu \emph{Dobry, zły i~brzydki} Blondie \\

\vspace{\spaceTwo}





% ####################
\Work{ % Autor i tytuł dzieła
  Tadeusz Szczepański \\
  ,,Zwierciadło Bergmana'', \cite{SzczepanskiZwierciadloBergmana07} }


\CenterTB{Uwagi}

\start \StrWg{131}{14} Postać pastora zagrał Gunnar Olsson.

\vspace{\spaceFour}


\start \StrWd{132}{17} Protestancki biskup Edvard Verg\'{e}rus
to~jeden z~bohaterów filmu Bergmana \emph{Fanny i~Aleksander}.

\vspace{\spaceFour}


\start \Str{146} Ponieważ Monika nie tylko udało~się uciec
z~mieszczańskiej willi, bez żadnych konsekwencji dla jej dalszego
życia, ale także zdobyć pieczeń po~którą tam poszła,
uważam~że~nazwanie jej działań ,,sromotnie nieudanymi'', jest
co~najmniej nietrafne.

\vspace{\spaceFour}

\CenterTB{Błędy}
\begin{center}
  \begin{tabular}{|c|c|c|c|c|}
    \hline
    & \multicolumn{2}{c|}{} & & \\
    Strona & \multicolumn{2}{c|}{Wiersz} & Jest
                              & Powinno być \\ \cline{2-3}
    & od góry & od dołu & & \\
    \hline
    31  & & 15 & zamiłowania ch & zamiłowaniach \\
    101 & 19 & & przypowieść & opowieść \\
    139 & 12 & & pozamał & pozamał\dywiz \\
    214 &  2 & & (\r{A}ke Fridell) Tubal & \r{A}ke Fridell (Tubal) \\
    % & & & & \\
    % & & & & \\
    % & & & & \\
    % & & & & \\
    \hline
  \end{tabular}
\end{center}

\vspace{\spaceTwo}





% #####################################################################
% #####################################################################
% Bibliografia
\bibliographystyle{alpha} \bibliography{Bibliography}{}


% ############################

% Koniec dokumentu
\end{document}