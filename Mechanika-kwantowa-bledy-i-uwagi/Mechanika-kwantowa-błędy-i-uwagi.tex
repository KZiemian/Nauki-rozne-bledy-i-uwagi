% Autor: Kamil Ziemian

% --------------------------------------------------------------------
% Podstawowe ustawienia i pakiety
% --------------------------------------------------------------------
\RequirePackage[l2tabu, orthodox]{nag}  % Wykrywa przestarzałe i niewłaściwe
% sposoby używania LaTeXa. Więcej jest w l2tabu English version.
\documentclass[a4paper,11pt]{article}
% {rozmiar papieru, rozmiar fontu}[klasa dokumentu]
\usepackage[MeX]{polski}  % Polonizacja LaTeXa, bez niej będzie pracował
% w języku angielskim.
\usepackage[utf8]{inputenc}  % Włączenie kodowania UTF-8, co daje dostęp
% do polskich znaków.
\usepackage{lmodern}  % Wprowadza fonty Latin Modern.
\usepackage[T1]{fontenc}  % Potrzebne do używania fontów Latin Modern.



% ----------------------------
% Podstawowe pakiety (niezwiązane z ustawieniami języka)
% ----------------------------
\usepackage{microtype}  % Twierdzi, że poprawi rozmiar odstępów w tekście.
% \usepackage{graphicx}  % Wprowadza bardzo potrzebne komendy do wstawiania
% grafiki.
% \usepackage{verbatim}  % Poprawia otoczenie VERBATIME.
% \usepackage{textcomp}  % Dodaje takie symbole jak stopnie Celsiusa,
% wprowadzane bezpośrednio w tekście.
\usepackage{vmargin}  % Pozwala na prostą kontrolę rozmiaru marginesów,
% za pomocą komend poniżej. Rozmiar odstępów jest mierzony w calach.
% ----------------------------
% MARGINS
% ----------------------------
\setmarginsrb
{ 0.7in} % left margin
{ 0.6in} % top margin
{ 0.7in} % right margin
{ 0.8in} % bottom margin
{  20pt} % head height
{0.25in} % head sep
{   9pt} % foot height
{ 0.3in} % foot sep



% ----------------------------
% Często przydatne pakiety
% ----------------------------
% \usepackage{csquotes}  % Pozwala w prosty sposób wstawiać cytaty do tekstu.
\usepackage{xcolor}  % Pozwala używać kolorowych czcionek (zapewne dużo
% więcej, ale ja nie potrafię nic o tym powiedzieć).



% ----------------------------
% Pakiety do tekstów z nauk przyrodniczych
% ----------------------------
\let\lll\undefined  % Amsmath gryzie się z językiem pakietami do języka
% polskiego, bo oba definiują komendę \lll. Aby rozwiązać ten problem
% oddefiniowuję tę komendę, ale może tym samym pozbywam się dużego Ł.
\usepackage[intlimits]{amsmath}  % Podstawowe wsparcie od American
% Mathematical Society (w skrócie AMS)
\usepackage{amsfonts, amssymb, amscd, amsthm}  % Dalsze wsparcie od AMS
% \usepackage{siunitx}  % Do prostszego pisania jednostek fizycznych
% \usepackage{upgreek}  % Ładniejsze greckie litery
% Przykładowa składnia: pi = \uppi
% \usepackage{slashed}  % Pozwala w prosty sposób pisać slash Feynmana.
\usepackage{calrsfs}  % Zmienia czcionkę kaligraficzną w \mathcal
% na ładniejszą. Może w innych miejscach robi to samo, ale o tym nic
% nie wiem.



% ##########
% Tworzenie otoczeń "Twierdzenie", "Definicja", "Lemat", etc.
\newtheorem{twr}{Twierdzenie}  % Komenda wprowadzająca otoczenie
% ,,twr'' do pisania twierdzeń matematycznych
\newtheorem{defin}{Definicja}  % Analogicznie jak powyżej
\newtheorem{wni}{Wniosek}



% ----------------------------
% Pakiety napisane przez użytkownika.
% Mają być w tym samym katalogu to ten plik .tex
% ----------------------------
\usepackage{mechanikakwantowa}  % Pakiet napisany konkretnie dla tego pliku.
\usepackage{latexshortcuts}
\usepackage{mathshortcuts}



% --------------------------------------------------------------------
% Dodatkowe ustawienia dla języka polskiego
% --------------------------------------------------------------------
\renewcommand{\thesection}{\arabic{section}.}
% Kropki po numerach rozdziału (polski zwyczaj topograficzny)
\renewcommand{\thesubsection}{\thesection\arabic{subsection}}
% Brak kropki po numerach podrozdziału



% ----------------------------
% Ustawienia różnych parametrów tekstu
% ----------------------------
\renewcommand{\arraystretch}{1.2}  % Ustawienie szerokości odstępów między
% wierszami w tabelach.



% ----------------------------
% Pakiet "hyperref"
% Polecano by umieszczać go na końcu preambuły.
% ----------------------------
\usepackage{hyperref}  % Pozwala tworzyć hiperlinki i zamienia odwołania
% do bibliografii na hiperlinki.





% --------------------------------------------------------------------
% Tytuł, autor, data
\title{Mechanika kwantowa --~błędy i~uwagi}

% \author{}
% \date{}
% --------------------------------------------------------------------


% ####################################################################
\begin{document}
% ####################################################################



% ######################################
\maketitle % Tytuł całego tekstu
% ######################################



% ######################################
\section{Mechanika kwantowa}

\vspace{\spaceTwo}
% ######################################



% ##################
\Work{ % Autorzy i tytuł dzieła
  Ramamurti Shankar \\
  ,,Mechanika kwantowa'', \cite{ShankarMechanikaKwantowa2006} }


\CenterTB{Uwagi}

\start \Str{74} Na rysunku 1.8 b) aby otrzymać poprawny wykres
pochodnej funkci Gaussa należy odbić wykres przedstawiony względem osi
$y = 0$.

\CenterTB{Błędy}
\begin{center}
  \begin{tabular}{|c|c|c|c|c|}
    \hline
    & \multicolumn{2}{c|}{} & & \\
    Strona & \multicolumn{2}{c|}{Wiersz} & Jest
                              & Powinno być \\ \cline{2-3}
    & Od góry & Od dołu & & \\
    \hline
    % & & & & \\
    23  & 19 & & antyrównoległą & równoległą \\
    % & & & & \\
    % & & & & \\
    \hline
  \end{tabular}
\end{center}

Błędy:\\
\begin{itemize}
\item[--] Str. 20. Zdanie na dole strony jest mętne. Popraw to.?????
\item[--] Str. 26. \ldots tylko wtedy, gdy $| V \rangle = 0$\ldots
\item[--] Str. 78. % \ii czy i?
  $$\ldots = i \int_{a}^{b} \dd{}{ g^{ * } }{ x } f( x ) \de x \,
  .$$
\item[--] Str. 81.
  $$\langle k' | X | k \rangle = \frac{ 1 }{ 2 \uppi } \int_{ -\infty
  }^{ \infty } \e^{ -\ii k' x } x \e^{ \ii k x } \de x = -\ii \dd{}{}{
    k } \bigg( \frac{ 1 }{ 2 \uppi } \int_{ -\infty }^{ \infty } \e^{
    \ii ( k - k' ) x } \bigg) = -\ii \delta'( k - k' ) \, .$$ Wobec
  tego, jeśli $| g( k ) \rangle$ jest wektorem, którym w bazie $K$
  odpowiada funkcja $g( k )$, to
$$X| g( k ) \rangle = \bigg| \frac{ -\ii \de g( k ) }{ \de k } \bigg\rangle \, .$$
Podsumujmy: w bazie $X$ operator $X$ działa jak $x$, a operator $K$
jak $-\ii \de / \de x$ (na funkcje $f( x )$), a w bazie $K$ działa jak
$k$, a operator $X$ jak $-\ii \de / \de k$\ldots
\end{itemize}
\newpage





% ##################
\Work{ % Autorzy i tytuł dzieła
  Marian Grabowski, Roman S.~Ingarden \\
  ,,Mechanika kwantowa. Ujęcie w~przestrzeni Hilberta'',
  \cite{GrabowskiIngardenMechanikaKwantowa1987} }

Uwagi:
\begin{itemize}
\item Powinna być zamieszczona uwaga, że każda skończenie wymiarowa
  podprzestrzeń przestrzeni Hilberta (ogólniej: unormowanej), jest
  domknięta. Wynika to, choćby z tego, że każda podprzestrzeń
  skończenie wymiarowa jest lokalnie zwarta.
\item Str. Jest tu przykład rozumowania z ogromną dziurą. Nie mozemy
  kożytać z własności przetrzeni Hilberta dopóki nie udowodnimy, że
  jest to przestrzeń Hilberta.
\item Str. 27. Jest tu pewne zamieszanie odnośnie jednoznaczności
  rozkładu. Rozkład na element najbliższy w danej podprzestrzeni i
  część ortogonalą musi być jednoznaczny, jeśli istnieje, ze względu
  na jednoznaczność rzutu. Nie mniej, nie rozstrzyga to problmu, czy
  istnieje alternatywny rzut na te poprzestrzenie. Negatwyną odpowiedź
  daje nam fakt iż:
  $\mathcal{ M } \cup \mathcal{ M }^{ \bot } = \{ \emptyset \}$.
\item Str. 27. Przedstawione tu pojęcie zupełności jest trochę
  myslące. Podana tu definicja zupełności odpowiada pojęciu
  \emph{totalności} omówionej w książce Waltera Thirring ,,Fizyka
  matematyczna. Tom 3''. Przedewszystkim z podanej definicji zbioru
  zupełnego nie wynika, że każdy wektor z $\mathcal{ H }$ można
  przedstawić jako szereg elementów tego zbioru. Stąd właśnie Thirring
  rozróżnia pojęcie totalności i zupełności.
\item Str. 27. W dowodach Wniosków I i II, jest dwa razy użyte
  twierdzenie, że wektor ortogonalny do danego zbioru jest też
  ortogonalny do jego domknięcia, w dowodzie pierwszego wniosku
  wyrażone słownie, w drugim za pomocą wzorów. Warto byłoby zrobić to
  bardzie elegancko.
\item Str. 28. W dowodzie wniosku I.2. jest coś dziwnego. Uwaga, że
  należy przyjrzeć się uzyskanym sumom prostym i wywnioskować z nich,
  iż $[ \mathcal{ M } ]=( \mathcal{ M }^{ \bot } )^{ \bot }$,równe
  dobrze prowadzi od razu do wniosku
  $[ \mathcal{ M } ]^{ \bot } = \mathcal{ M }^{ \bot }$. Zachodzi
  bowiem twierdzenie: jeżeli
  $A_{ 1 } \oplus A_{ 2 } = B_{ 1 } \oplus B_{ 2 } = X$ i
  $B_{ i } \subset A_{ i }$ to $B_{ i } = A_{ i }$. Załóżmy,że tak nie
  jest. Wtedy istnieje
  $( x_{ 1 }, x_{ 2 } ) \in A_{ 1 } \oplus A_{ 2 }$, taka że
  $( x_{ 1 }, x_{ 2 } ) \notin B_{ 1 } \oplus B_{ 2 }$. Teraz istnieje
  taki $( y_{ 1 }, y_{ 2 } ) \in B_{ 1 } \oplus B_{ 2 }$, że
  $x_{ 1 } + x_{ 2 } = y_{ 1 } + y_{ 2 }$, czyli
  $A_{ 1 } \oplus A_{ 2 }$ niejest sumą prostą.
\item Str. 38. Pojawia się tu pojęcie operatora ograniczonego, które
  jest wprowadzone dopiero na stronie 39.
\item Str. 40. W dowodzie lematu II.1, gdy mowa jest o udowodnieniu
  pierwszej równości w punkcie a), w istocie udowodniono równość:
  $$|| A || = \sup_{ || \varphi || = 1 } || A \varphi || \, .$$
\item Str. 46. Punkty twierdzenia są ustawione w dziwnej kolejności,
  biorąc pod uwagę logikę dowodu.
\item Str. 48. Z tego, że dana liczba nie należy do widma, nie wynika
  że nie istnieje dla niej rezolwenta.
\item Str. 49. Drugie stwierdzenie z punktu (\romannumeral4), jest już
  zawarte w punkcie (\romannumeral2).
\item Str. 52. Użyte tu pojęcie funkcji charakterystycznej, nie jest
  chyba nigdzie w książce przedstawione.
\item Str. 57. Warto byłoby omówić szerszej pojęcie domkniętego
  rozszerzenia operatora, domykalności operatora i jego domknięcia. W
  szczególności z twierdzenia o wykresie domkniętym wynika, że
  domknięcie opratora jest zawsze operatorem ograniczonym.
\item Str. 58. Uwaga o twierdzeniu II (\romannumeral3) jest zupełnie
  niezrozumiała.
\item Str. 58. Zdefiniowaniu rezolwenty i widma operatora
  nieograniczonego powinno zostać poświęcone więcej miejsca.
\item Str. 58. Nie dodano, że widmo nieograniczonego operatora
  domykalnego, definiujemy jako widmo jego domknięcia.
\item Str. 58. Jedyność wektora $\eta$ wynika już z lematu Riesza.
\item Str. 80. Ustalenie takiej wartości stałej $C$, ani w ogóle
  ustalenie jej wartości, nie jest potrzebne w rozważanym zagadnieniu.
\item Str. 313. Nie wspomniano tu w jakim sensie dane ciągi funkcji
  mają być zbieżne. Osoba znająca teorię całki Lebesgue'a wie, że
  wystarczy założyć zbieżność punktową, a nawet tylko zbieżność
  punktową prawie wszędzie.
\item Str. 315. W twierdzeniu Lebesgue'a o zbieżności majoryzowanej
  brakuje założenia o zbieżności rozważanego ciągu funkcji.
\end{itemize}


\CenterTB{Błędy}
\begin{center}
  \begin{tabular}{|c|c|c|c|c|}
    \hline
    & \multicolumn{2}{c|}{} & & \\
    Strona & \multicolumn{2}{c|}{Wiersz} & Jest
                              & Powinno być \\ \cline{2-3}
    & Od góry & Od dołu & & \\
    \hline
    % & & & & \\
    21 & 1 & & $\{ x ,\! \abso{ \psi( x ) - \varphi( x ) } > 0 \}$
           & $\{ x ;\, | \psi( x ) - \varphi( x )| > 0 \}$ \\
    27 & 12 & & $\Hc_{ 1 } \ot \Hc_{ 2 }$ & $\Hc_{ 1 } \oplus \Hc_{ 2 }$ \\
    27 & 13 & & $\Rc \otimes \mathcal{R}^{ \bot }$ & $\mathcal{R} \oplus \mathcal{R}^{ \bot }$ \\
    30 & & 18 & $\xi - \varphi_{ k } ||$ & $|| \xi - \varphi_{ k } ||$ \\
    32 & & 11 & Teraz$f( a ) = g( 0 )$,$f( b )$ & Teraz $f( a ) = g( 0 )$, $f( b )$ \\
    34 & & 15 & & $[ a, b ]$ \\
    101 & & 4 & $Px )$ & $P( x )$ \\
    311 & 10 & & $i$.Jeżeli & $i$. Jeżeli \\
    311 & 15 & & $X$ spełniającą & $X$, spełniającą \\
    312 & & 10 & $\mu$ skończona & $\mu$\dywiz skończona \\
    313 & 6 & & $\{ x ;\! f( x ) > a \}$ & $\{ x ; f( x ) > a \}$ \\
    313 & 15 & & $0 = a_{ 0 }$ & $0 \leq a_{ 0 }$ \\
    313 & 15 & & $x,$ & $x;$ \\
    313 & 16 & & $A_{ 0 } \ldots \cup A_{ n }$ & $A_{ 0 } \cup \ldots \cup A_{ n }$ \\
    315 & 3 & & $A^{ 0 } \leq f_{ 1 }( x )$ & $A$: $0 \leq f_{ 1 }( x )$ \\
    318 & & 7 & Caucy'ego & Cauchy'ego \\
    % & & & & \\
    \hline
  \end{tabular}
\end{center}


\begin{itemize}
\item[--] Str. 27. Niech $\varphi \bot [ \mathcal{ P } ]$. Wówczas
  $\varphi \bot \mathcal{ P }$ i $\varphi = 0$.
\item[--] Str. 37.
  \ldots$\psi = \frac{ \overline{ l ( \varphi_{ 0 } ) } }{ ||
    \varphi_{ 0 } ||^{ 2 } } \varphi_{ 0 } \, .$
\item[--] Str. 40. \ldots punkcie $D( A )$\ldots
\item[--] Str. 40. \ldots określone na $D( A )$\ldots
\item[--] Str. 41. \ldots dużych $n$ mamy
  $|| A_{ n } - A_{ m } || < 2 \epsilon$.
\item[--] Str. 47. \ldots wynika, że
  $S_{ \lambda } = ( \lambda I - A )^{ -1 } \in B( \mathcal{ H } )$.
\item[--] \ldots być równo zbiorowi $\{ 0 \}$ \ldots
\item[--] Str. 54 \ldots może być równa zbiorowi $\{ 0 \}$\ldots
\item[--] Str. 71. \ldots$A^{ * } A = S U^{ * } U S = S E S$\ldots
\item[--] Str. 74. \ldots interpretacją. Teraz\ldots
\item[--] Str. 85.
  $$\ldots = -i \dd{}{}{ x } \frac{ 1 }{ \sqrt{ 2 \pi } }
  \int\limits_{ -\infty }^{ +\infty } ( \e^{ \ii x s } - 1 ) \varphi(
  s ) \de s \textrm{.}$$
\item[--] Str. 86.
  $$F^{ -1 } Q \varphi = -\ii \dd{}{}{ x } \frac{ 1 }{ \sqrt{ 2 \pi }
  } \int\limits_{ -\infty }^{ +\infty } ( \e^{ \ii x s } - 1 )
  \varphi( s ) \de s = \frac{ 1 }{ \hbar } P F^{ -1 } \varphi$$
\item[--] Str. 311. \ldots dla każdego $i$. Jeżeli\ldots
\end{itemize}





% ######################################
\section{Informatyka kwantowa}

\vspace{\spaceTwo}
% ######################################



% ##################
\Work{ % Autor i tytuł dzieła
  Richard P. Feynman \\
  ,,Wykłady o~obliczeniach'', \cite{FeynmanWykladyOObliczeniach2007} }

\CenterTB{Błędy}
\begin{center}
  \begin{tabular}{|c|c|c|c|c|}
    \hline
    & \multicolumn{2}{c|}{} & & \\
    Strona & \multicolumn{2}{c|}{Wiersz} & Jest
                              & Powinno być \\ \cline{2-3}
    & Od góry & Od dołu & & \\
    \hline
    % & & & & \\
    % & & & & \\
    % & & & & \\
    % & & & & \\
    % & & & & \\
    % & & & & \\
    % & & & & \\
    % & & & & \\
    % & & & & \\
    % & & & & \\
    % & & & & \\
    % & & & & \\
    % & & & & \\
    % & & & & \\
    % & & & & \\
    279 & & 11 & jednen & jeden \\
    \hline
  \end{tabular}
\end{center}

\vspace{\spaceTwo}










% #####################################################################
% #####################################################################
% Bibliografia
\bibliographystyle{plalpha} \bibliography{LibPhilNatur}{}


% ############################

% Koniec dokumentu
\end{document}
