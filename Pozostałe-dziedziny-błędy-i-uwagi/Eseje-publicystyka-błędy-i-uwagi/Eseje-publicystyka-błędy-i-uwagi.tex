% Autor: Kamil Ziemian

% ---------------------------------------------------------------------
% Podstawowe ustawienia i pakiety
% ---------------------------------------------------------------------
\RequirePackage[l2tabu, orthodox]{nag}  % Wykrywa przestarzałe i niewłaściwe
% sposoby używania LaTeXa. Więcej jest w l2tabu English version.
\documentclass[a4paper,11pt]{article}
% {rozmiar papieru, rozmiar fontu}[klasa dokumentu]
\usepackage[MeX]{polski}  % Polonizacja LaTeXa, bez niej będzie pracował
% w języku angielskim.
\usepackage[utf8]{inputenc}  % Włączenie kodowania UTF-8, co daje dostęp
% do polskich znaków.
\usepackage{lmodern}  % Wprowadza fonty Latin Modern.
\usepackage[T1]{fontenc}  % Potrzebne do używania fontów Latin Modern.



% ---------------------------------------
% Podstawowe pakiety (niezwiązane z ustawieniami języka)
% ---------------------------------------
\usepackage{microtype}  % Twierdzi, że poprawi rozmiar odstępów w tekście.
% \usepackage{graphicx}  % Wprowadza bardzo potrzebne komendy do wstawiania
% % grafiki.
% \usepackage{verbatim}  % Poprawia otoczenie VERBATIME.
% \usepackage{textcomp}  % Dodaje takie symbole jak stopnie Celsiusa,
% % wprowadzane bezpośrednio w tekście.
\usepackage{vmargin}  % Pozwala na prostą kontrolę rozmiaru marginesów,
% za pomocą komend poniżej. Rozmiar odstępów jest mierzony w calach.
% ---------------------------------------
% MARGINS
% ---------------------------------------
\setmarginsrb
{ 0.7in} % left margin
{ 0.6in} % top margin
{ 0.7in} % right margin
{ 0.8in} % bottom margin
{  20pt} % head height
{0.25in} % head sep
{   9pt} % foot height
{ 0.3in} % foot sep



% ---------------------------------------
% Często używane pakiety
% ---------------------------------------
% \usepackage{csquotes}  % Pozwala w prosty sposób wstawiać cytaty do tekstu.
\usepackage{xcolor}  % Pozwala używać kolorowych czcionek (zapewne dużo
% więcej, ale ja nie potrafię nic o tym powiedzieć).



% ---------------------------------------
% Pakiety napisane przez użytkownika.
% Mają być w tym samym katalogu to ten plik .tex
% ---------------------------------------
\usepackage{latexgeneralcommands}



% ---------------------------------------------------------------------
% Dodatkowe ustawienia dla języka polskiego
% ---------------------------------------------------------------------
\renewcommand{\thesection}{\arabic{section}.}
% Kropki po numerach rozdziału (polski zwyczaj topograficzny)
\renewcommand{\thesubsection}{\thesection\arabic{subsection}}
% Brak kropki po numerach podrozdziału



% ---------------------------------------
% Ustawienia różnych parametrów tekstu
% ---------------------------------------
\renewcommand{\arraystretch}{1.2}  % Ustawienie szerokości odstępów między
% wierszami w tabelach.





% ---------------------------------------
% Pakiet „hyperref”
% Polecano by umieszczać go na końcu preambuły.
% ---------------------------------------
\usepackage{hyperref}  % Pozwala tworzyć hiperlinki i zamienia odwołania
% do bibliografii na hiperlinki.










% ---------------------------------------------------------------------
% Tytuł, autor, data
\title{Eseje, publicystyka~-- błędy i~uwagi}

% \author{}
% \date{}
% ---------------------------------------------------------------------










% ####################################################################
% Początek dokumentu
\begin{document}
% ####################################################################





% ######################################
\maketitle % Tytuł całego tekstu
% ######################################





% ############################
\Work{ % Autor i tytuł dzieła
  Marek Jan Chodakiewicz \\
  „O~cywilizacji śmierci. \\
  Jak zatrzymać antykulturę totalitarnych mniejszości”,
  \cite{ChodakiewiczOCywilizacjiSmierci2019} }


% ##################
\CenterBoldFont{Uwagi ogólne}


\start \textbf{Str. 11, wiersze 6, 10.} Po tych wiersza w~tekście
powinien znajdować~się odstęp.

% \vspace{\spaceFour}





% ##################
\CenterTB{Błędy}

\begin{center}

  \begin{tabular}{|c|c|c|c|c|}
    \hline
    & \multicolumn{2}{c|}{} & & \\
    Strona & \multicolumn{2}{c|}{Wiersz} & Jest
                              & Powinno być \\ \cline{2-3}
    & Od góry & Od dołu & & \\
    \hline
    14  & 14 & & śmierci śmierci & śmierci \\
    16  & & 16 & Ldweicy & Lewicy \\
    43  & & 20 & \emph{Bogomil :} & \emph{Bogomil:} \\
    43  & & 14 & \emph{Inqistion} & \emph{Inquisition} \\
    53  & & 11 & Company. & Company, \\
    54  & &  4 & 1\textbf{8} & 18 \\
    54  & &  3 & \emph{Bueaty} & \emph{Beauty} \\
    62  & &  5 & Use & \emph{Use} \\
    62  & &  4 & The~History~of Sexuality
           & \emph{The~History~of Sexuality} \\
    63  &  3 & & wyjątków & z~wyjątków \\
    64  & &  2 & The~Gay Metropolis & \emph{The~Gay Metropolis} \\
    67  & 17 & & Action League & \emph{Action League} \\
    68  & & 12 & ludzkiej & ludzkiej'' \\
    69  & & 14 & religi świata & Religi Świata \\
    % & & & & \\
    % & & & & \\
    % & & & & \\
    % & & & & \\
    \hline
  \end{tabular}



  % \begin{tabular}{|c|c|c|c|c|}
  %   \hline
  %   & \multicolumn{2}{c|}{} & & \\
  %   Strona & \multicolumn{2}{c|}{Wiersz} & Jest
  %   & Powinno być \\ \cline{2-3}
  %   & Od góry & Od dołu & & \\
  %   \hline
  %   %   & & & & \\
  %   %   & & & & \\
  %   %   & & & & \\
  %   %   & & & & \\
  %   %   & & & & \\
  %   %   & & & & \\
  %   %   & & & & \\
  %   %   & & & & \\
  %   %   & & & & \\
  %   %   & & & & \\
  %   %   & & & & \\
  %   %   & & & & \\
  %   %   & & & & \\
  %   %   & & & & \\
  %   %   & & & & \\
  %   %   & & & & \\
  %   %   & & & & \\
  %   %   & & & & \\
  %   %   & & & & \\
  %   %   & & & & \\
  %   %   & & & & \\
  %   %   & & & & \\
  %   %   & & & & \\
  %   %   & & & & \\
  %   %   & & & & \\
  %   %   & & & & \\
  %   %   & & & & \\
  %   %   & & & & \\
  %   %   & & & & \\
  %   %   & & & & \\
  %   %   & & & & \\
  %   %   & & & & \\
  %   %   & & & & \\
  %   %   & & & & \\
  %   %   & & & & \\
  %   %   & & & & \\
  %   %   & & & & \\
  %   %   & & & & \\
  %   \hline
  % \end{tabular}


  % \begin{tabular}{|c|c|c|c|c|}
  %   \hline
  %   & \multicolumn{2}{c|}{} & & \\
  %   Strona & \multicolumn{2}{c|}{Wiersz} & Jest
  %   & Powinno być \\ \cline{2-3}
  %   & Od góry & Od dołu & & \\
  %   \hline
  %   %   & & & & \\
  %   %   & & & & \\
  %   %   & & & & \\
  %   %   & & & & \\
  %   %   & & & & \\
  %   %   & & & & \\
  %   %   & & & & \\
  %   %   & & & & \\
  %   %   & & & & \\
  %   %   & & & & \\
  %   %   & & & & \\
  %   %   & & & & \\
  %   %   & & & & \\
  %   %   & & & & \\
  %   %   & & & & \\
  %   %   & & & & \\
  %   %   & & & & \\
  %   %   & & & & \\
  %   %   & & & & \\
  %   %   & & & & \\
  %   %   & & & & \\
  %   %   & & & & \\
  %   %   & & & & \\
  %   %   & & & & \\
  %   %   & & & & \\
  %   %   & & & & \\
  %   %   & & & & \\
  %   %   & & & & \\
  %   %   & & & & \\
  %   %   & & & & \\
  %   %   & & & & \\
  %   %   & & & & \\
  %   %   & & & & \\
  %   %   & & & & \\
  %   %   & & & & \\
  %   %   & & & & \\
  %   %   & & & & \\
  %   %   & & & & \\
  %   \hline
  % \end{tabular}

\end{center}


\noindent
\Jest  Selected Writings~of Alexandra Kollontai \\
\Powin \emph{Selected Writings~of Alexandra Kollontai} \\

% \StrWd{11}{16} \\
% \Jest  Amerian Seafarers Union \\
% \Powin \emph{Amerian Seafarers Union} \\



% \StrWd{11}{5} \\
% \Jest  Council~of Economic Advisers \\
% \Powin \emph{Council~of Economic Advisers} \\
% \StrWd{11}{1} \\
% \Jest  Council on~Foregin Relations \\
% \Powin \emph{Council on~Foregin Relations} \\
% \StrWg{12}{11} \\
% \Jest \emph{Congress}) --~Kanadyjski Kongres Związków Zawodowych \\
% \Powin  \emph{Congress} --~Kanadyjski Kongres Związków Zawodowych) \\
% \StrWd{12}{18} \\
% \Jest Emergency Committe for~Aid to~Poland \\
% \Powin  \emph{Emergency Committe for~Aid to~Poland} \\
% \StrWd{12}{8} \\
% \Jest (\emph{Froce Ouvri\'{e}re} ) \\
% \Powin (\emph{Froce Ouvri\'{e}re} --~Główna Konfederacja Pracy
% --~Siły
% Pracy) \\
% \StrWg{13}{10} \\
% \Jest Generalized System~of Preferences \\
% \Powin  \emph{Generalized System~of Preferences} \\
% \StrWd{13}{4} \\
% \Jest \emph{Leuven}) \\
% \Powin \emph{Leuven} --~Kotlicki Ośrodek Dokumentacyjny i~Badań
% Katolickich Uniwersytetu Leuven \\
% \StrWg{15}{1} \\
% \Jest Polish American Congress Charitable Foundation \\
% \Powin  \emph{Polish American Congress Charitable Foundation} \\
% \StrWg{15}{3} \\
% \Jest Polish-American Enterprise Fund \\
% \Powin  \emph{Polish-American Enterprise Fund} \\
% \StrWg{15}{10} \\
% \Jest Postal, Telegraph and~Telephone International \\
% \Powin  \emph{Postal, Telegraph and~Telephone International} \\
% \StrWg{43}{2} \\
% \Jest \emph{Univeristy~of Illinois at~Urbana Champaign, 1998} \\
% \Powin  Univeristy~of Illinois at~Urbana-Champaign, 1998 \\

\vspace{\spaceTwo}
% ############################





% % ##################
% \Work{ % Tytuł i autor dzieła
% Norman Davies \\
% ,,Serce Europy'', \cite{DaviesSerceEuropy2014} }


% \CenterTB{Uwagi}

% \start \Str{16} Davies pisze tu o~tym, że~rzut trójwymiarowej Ziemi
% na~dwuwymiarową mapę musi pociągać za sobą zniekształcenia
% odwzorowywanego terenu, jednak mnie~się wydaje, że~źle zrozumiał,
% gdzie leży problem. Powierzchnię Ziemi można uważać za dwuwymiarową,
% tak jak mapę i~to raczej nie jest problemem w~kartografii.

% Jednocześnie wiadomo na mapie albo kąty albo kształt lądów nie mogą
% być wiernie oddane. Jest to jednak związane z~czymś innym niż
% wymiar, mianowicie z~Theorema Egregium Gaussa. Jako szczególny
% przypadek wynika z~niego, że~takie zniekształcenia przy
% przekształcaniu dwuwymiarowej powierzchni w~inną dwuwymiarową
% powierzchnię muszą się pojawić, jeśli mają one różne krzywizny
% Gaussa, dla~płaszczyzny zaś jest ona równa 0, a~dla sfery
% $\frac{ 1 }{ r^{ 2 } }$.

% \vspace{\spaceFour}


% \start \StrWg{17}{12} W~tym miejscu Davies stosuje często spotykany
% anachronizm pisząc o~Watykanie jako metonimie Ojca Świętego
% i~najwyższych władz Kościoła. Do~czasu zajęcia przez wojska Victora
% Emanuela~II Państwa Kościelnego w~1870~roku, władza doczesna papieża
% nie ograniczała~się tylko do~tej dzielnicy Rzymu, lecz przez wieki
% dotyczyła ogromnych połaci Półwyspu Apenińskiego. Dlatego trafniej
% w~kontekście 1765~r. byłoby pisać o~decyzji Rzymu.

% \vspace{\spaceFour}


% \start \Str{28} Wydaje mi~się, że~nazwanie wszystkich wymienionych
% tu~przez Davies państw dyktaturami, jest błędne. Państwo Watykańskie
% i~Tybet Dalajlamy nie są demokracjami, ale~to za~mało, aby uznać je
% za~dyktatury, bo~dyktatura jest pojęciem czysto negatywnym
% i~piętnującym.

% \vspace{\spaceFour}


% \start \StrWg{52}{10} Zostawienie w~tym wersie słowa ,,Batman''
% pisanego z~dużej litery, jest chyba błędem tłumacza. Sugeruje to,
% że~chodzi o~Człowieka\dywiz Nietoperza, jednanego z~najbardziej
% znanych superbohaterów amerykańskich, jednak jest bardziej
% prawdopodobne, że~angielskie słowo ,,batman'' występuje
% to~w~znaczeniu ,,ordynans''.

% % \CenterTB{Błędy}
% % \begin{center}
% %   \begin{tabular}{|c|c|c|c|c|}
% %     \hline
% %     & \multicolumn{2}{c|}{} & & \\
% %     Strona & \multicolumn{2}{c|}{Wiersz} & Jest
% %     %     & Powinno być \\ \cline{2-3}
% %     %     & Od góry & Od dołu & & \\
% %     %     \hline
% %     & & & & \\
% %     & & & & \\ \hline
% %     % %   \end{tabular}
% %     % % \end{center}

% \vspace{\spaceTwo}





% % ##################
% \Work{
% Andrzej Nowak \\
% ,,Dzieje Polski. Tom~I do~1202: Skąd nasz ród'',
% \cite{NowakDziejePolskiTomI2014} }


% \CenterTB{Uwagi}

% \start \textbf{Strona tytułowa.} W informacjach o~autorze jest
% podane, że~był redaktorem naczelnym ,,ARCANA'' w~latach 1994--2012,
% lecz prawidłowy okresem są chyba lata 1995--2012.

% \vspace{\spaceFour}


% \start \Str{62} W~ostatnim paragrafie jest mowa o~czterech królów
% Słowian, ale~wymienionych jest tylko trzech.

% \vspace{\spaceFour}


% % POPRAW
% \start \StrWg{85}{16} ,,Nie było innej drogi do~Europy w~końcu X~w.
% jak poprzez chrzest, nie~było dalej w~Europie innej drogi do
% humanizmu jak poprzez chrześcijaństwo.'' To zdanie chyba najlepiej
% oddaje problem z~postrzeganiem chrześcijaństwa nie tylko przez
% Nowaka, ale~i~przez przytłaczającą większość, jeśli nie~całość,
% polskiej myśli patriotycznej. Chrześcijaństwo, nawet nie
% rzymski\dywiz katolicyzm, jest tylko środkiem do osiągnięcia
% doczesnych, świeckich, ziemskich celów, takich jak dostanie~się
% do~,,Europy'', albo kultywowanie humanizmu, nie~zaś wiarą pochodzącą
% od Boga i~jedyną drogą do życia wiecznego. To~zaś zredukowanie Boga
% i~wiary, do~doczesnych korzyści, to~straszliwe zło.

% \vspace{\spaceFour}


% \start \StrWd{92}{21--20} Nie potrafię zrozumieć, czy chodziło o~to,
% że~Otton~II przejściowo zdobył Akwizgran, czy że~miejsce to było
% przejściowo stolicą cesarską.

% \vspace{\spaceFour}


% \start \Str{96} W~swoim wykładzie z~cyklu
% \href{https://www.youtube.com/watch?v=QovVLT2fitc}{,,Filary
% Polskości: Mieszko i~Bolesław''} Nowak znacznie wyraźniej niż w~tej
% książce, pokazał cały cynizm polityczny Mieszka~I. O~możliwości
% takiego spojrzenia na tego władcę Nowak mówi tam otwarcie
% jednocześnie, zarówno na wykładzie jaki i~książce, próbuje przykryć
% ten cynizm, nazywając działania Mieszka ,,majstersztykiem polityki
% polskiej''.

% \vspace{\spaceFour}


% \start \StrWg{144}{24} Jest tu napisane, że~Miecław był cześnikiem,
% jednak jaka była nadworna funkcja cześnika jest wyjaśnione dopiero
% na~stronie~150.

% \vspace{\spaceFour}


% \start \StrWd{163}{17--15} Zdanie ,,Rozpoczyna od~modlitwy
% do~św.~Piotra, patrona Stolicy Apostolskiej w~Rzymie i~zarazem
% Gertrudowego syna, jadącego właśnie do~Rzymu, do~papieża.'' można
% zrozumieć w~ten sposób, że~Gertruda modli~się do swojego syna, co
% jest nonsensowne. Prawdopodobnie miało być ,,zarazem patrona
% Gerturdowego syna'', co czyni całe zdanie zrozumiałym i~sensownym.


% \CenterTB{Błędy}
% \begin{center}
%   \begin{tabular}{|c|c|c|c|c|}
%     \hline
%     & \multicolumn{2}{c|}{} & & \\
%     Strona & \multicolumn{2}{c|}{Wiersz} & Jest
%                               & Powinno być \\ \cline{2-3}
%     & Od góry & Od dołu & & \\
%     \hline
%     41  &  3 & & W X w. jeszcze & Jeszcze w X w. \\
%     53  & &  1 & Księga Wyjścia & Księga Rodzaju \\
%     61  & 10 & & do dziejów & dla dziejów \\
%     100 & 14 & & Kto & Kto to \\
%     169 & 17 & & mnie & do mnie \\
%     202 & &  1 & posiąść & przesiąść \\
%     236 & 14 & & ,,Gall''). & ,,Gall''. \\
%     %     & & & & \\
%     %     & & & & \\
%     %     & & & & \\
%     \hline
%   \end{tabular}
% \end{center}
% \noi
% \StrWg{61}{8} \\
% \Jest  z~Rocznika kapituły krakowskiej dawnego \\
% \Powin z~dawnego Rocznika kapituły krakowskiej \\

% \vspace{\spaceTwo}





% % ######################################
% \newpage
% \section{Dzieje Polski z lat 1795--1914}

% \vspace{\spaceTwo}
% % ######################################



% % ##################
% \Work{ % Autor i tytuł dzieła
% Red. Andrzej Nowak \\
% ,,Historie Polski w~XIX wieku. Kominy, ludzie i~obłoki: \\
% modernizacja i~kultura. Tom~I'', \cite{HistoriaPolskiXIXTomI2013} }


% \CenterTB{Błędy}
% \begin{center}
%   \begin{tabular}{|c|c|c|c|c|}
%     \hline
%     & \multicolumn{2}{c|}{} & & \\
%     Strona & \multicolumn{2}{c|}{Wiersz} & Jest
%                               & Powinno być \\ \cline{2-3}
%     & Od góry & Od dołu & & \\
%     \hline
%     16 & & 15 & równości równość & równości \\
%     %     & & & & \\
%     %     & & & & \\
%     %     & & & & \\
%     %     & & & & \\
%     \hline
%   \end{tabular}
% \end{center}

% \vspace{\spaceTwo}





% % ######################################
% \newpage
% \section{Dzieje Polski po~roku 1914}

% % \vspace{\spaceTwo}
% % ######################################



% % ##################
% \Work{ % Autor i tytuł dzieła
% Paweł Zyzak \\
% ,,Efekt domina. Czy Ameryka obaliła komunizm w~Polsce? \\
% Tom~I'', \cite{ZyzakEfektDominaTomI2016} }


% % \CenterTB{Uwagi}

% \CenterTB{Błędy}
% \begin{center}
%   \begin{tabular}{|c|c|c|c|c|}
%     \hline
%     & \multicolumn{2}{c|}{} & & \\
%     Strona & \multicolumn{2}{c|}{Wiersz} & Jest
%                               & Powinno być \\ \cline{2-3}
%     & Od góry & Od dołu & & \\
%     \hline
%     11  &  4 & & Labor & \emph{Labor} \\
%     11  & & 15 & in & \emph{in} \\
%     12  &  3 & & \emph{Konfederacja} & Konfederacja \\
%     12  & 23 & & CSS & CSSA \\
%     13  & 21 & & Development & \emph{Development} \\
%     13  & 22 & & The & \emph{The} \\
%     13  & & 11 & The~International Rescue & \emph{The~International
%                                             Rescue} \\
%     14  & &  3 & Office & \emph{Office} \\
%     16  & 15 & & I~Rozwoju & i~Rozwoju \\
%     16  & & 14 & \emph{Labour} & \emph{Trade Union} \\
%     17  &  5 & & miała obejmować & obejmować \\
%     31  & &  2 & \emph{WCFL} & \emph{WCL} \\
%     32  & &  7 & podziałało & podziałało~to \\
%     37  & 16 & & \emph{Konfekcji Damskiej} & Konfekcji Damskiej \\
%     41  &  3 & & polskiego.~ZRK & polskiego~ZRK \\
%     47  & &  1 & Katyń & \emph{Katyń} \\
%     48  & & 12 & za-oceanicznych & zaoceanicznych \\
%     %     & & & & \\
%     %     & & & & \\
%     %     & & & & \\
%     %     & & & & \\
%     %     & & & & \\
%     %     & & & & \\
%     %     & & & & \\
%     %     & & & & \\
%     %     & & & & \\
%     %     & & & & \\
%     %     & & & & \\
%     %     & & & & \\
%     %     & & & & \\
%     %     & & & & \\
%     %     & & & & \\
%     %     & & & & \\
%     %     & & & & \\
%     %     & & & & \\
%     %     & & & & \\
%     %     & & & & \\
%     %     & & & & \\
%     %     & & & & \\
%     %     & & & & \\
%     \hline
%   \end{tabular}
% \end{center}
% \noi
% \textbf{Tyla okładka, wiersz 17.} \\
% \Jest Sorosa ,Williama \\
% \Powin Sorosa, Williama \\
% \StrWd{11}{16} \\
% \Jest  Amerian Seafarers Union \\
% \Powin \emph{Amerian Seafarers Union} \\
% \StrWd{11}{5} \\
% \Jest  Council~of Economic Advisers \\
% \Powin \emph{Council~of Economic Advisers} \\
% \StrWd{11}{1} \\
% \Jest  Council on~Foregin Relations \\
% \Powin \emph{Council on~Foregin Relations} \\
% \StrWg{12}{11} \\
% \Jest \emph{Congress}) --~Kanadyjski Kongres Związków Zawodowych \\
% \Powin  \emph{Congress} --~Kanadyjski Kongres Związków Zawodowych) \\
% \StrWd{12}{18} \\
% \Jest Emergency Committe for~Aid to~Poland \\
% \Powin  \emph{Emergency Committe for~Aid to~Poland} \\
% \StrWd{12}{8} \\
% \Jest (\emph{Froce Ouvri\'{e}re} ) \\
% \Powin (\emph{Froce Ouvri\'{e}re} --~Główna Konfederacja Pracy
% --~Siły
% Pracy) \\
% \StrWg{13}{10} \\
% \Jest Generalized System~of Preferences \\
% \Powin  \emph{Generalized System~of Preferences} \\
% \StrWd{13}{4} \\
% \Jest \emph{Leuven}) \\
% \Powin \emph{Leuven} --~Kotlicki Ośrodek Dokumentacyjny i~Badań
% Katolickich Uniwersytetu Leuven \\
% \StrWg{15}{1} \\
% \Jest Polish American Congress Charitable Foundation \\
% \Powin  \emph{Polish American Congress Charitable Foundation} \\
% \StrWg{15}{3} \\
% \Jest Polish-American Enterprise Fund \\
% \Powin  \emph{Polish-American Enterprise Fund} \\
% \StrWg{15}{10} \\
% \Jest Postal, Telegraph and~Telephone International \\
% \Powin  \emph{Postal, Telegraph and~Telephone International} \\
% \StrWg{43}{2} \\
% \Jest \emph{Univeristy~of Illinois at~Urbana Champaign, 1998} \\
% \Powin  Univeristy~of Illinois at~Urbana-Champaign, 1998 \\

% \vspace{\spaceTwo}





% % ##################
% \Work{ % Autor i tytuł dzieła
% Paweł Zyzak \\
% ,,Efekt domina. Czy Ameryka obaliła komunizm w~Polsce? \\
% Tom~II'', \cite{ZyzakEfektDominaTomII2016} }


% \CenterTB{Uwagi}

% \start \StrWg{653}{3} Nie jest podana data wydania książki
% Dubinsky'ego i~Raskina. W~skutek tego daty tej brakuje również
% w~pierwszym tomie, str.~39, wiersz ósmy od~dołu.

% \CenterTB{Błędy}
% \begin{center}
%   \begin{tabular}{|c|c|c|c|c|}
%     \hline
%     & \multicolumn{2}{c|}{} & & \\
%     Strona & \multicolumn{2}{c|}{Wiersz} & Jest
%                               & Powinno być \\ \cline{2-3}
%     & Od góry & Od dołu & & \\
%     \hline
%     %     & & & & \\
%     %     & & & & \\
%     %     & & & & \\
%     %     & & & & \\
%     %     & & & & \\
%     %     & & & & \\
%     %     & & & & \\
%     %     & & & & \\
%     %     & & & & \\
%     %     & & & & \\
%     %     & & & & \\
%     %     & & & & \\
%     %     & & & & \\
%     %     & & & & \\
%     %     & & & & \\
%     %     & & & & \\
%     %     & & & & \\
%     %     & & & & \\
%     %     & & & & \\
%     %     & & & & \\
%     %     & & & & \\
%     %     & & & & \\
%     664 & & 22 & Katyń & \emph{Katyń} \\
%     %     & & & & \\
%     %     & & & & \\
%     \hline
%   \end{tabular}
% \end{center}







% % ############################
% \subsection{Lata 1914--1939}

% \vspace{\spaceThree}
% % ############################



% % ##################
% \Work{ % Tytuł i autor dzieła
% Wojciech Roszkowski \\
% ,,Najnowsza historia Polski: 1914--1939'',
% \cite{RoszkowskiNajnowszaHistoriaPolski14-39Wyd2011} }


% \CenterTB{Uwagi}

% \start Karygodną, i~to~niezależnie od uznawanej metodologi pisania
% -prac historycznych, cechą całego tego wydania ,,Najnowszej historii
% -Polski'', jest nieumieszczenie w~każdej tomie listy używanych
% skrótów. -Należy dodać, że~jeśli skrót został wprowadzony w~jednym
% tomie, to~nie -jest już wyjaśniany w~następnych, co~dodatkowo
% komplikuje sprawę.

% \vspace{\spaceFour}


% \start Ciekawym wydaje~się zauważanie, że~w~tej książce Roszkowski
% zrealizował chyba idealnie, jedno z~założeń zaprojektowanego przez
% piłsudczyków programu edukacji, przyjętego po reformie
% jędrzejewiczowskiej (1932): sprowadzenia lat 1918--1920 wyłączenie
% do~tematu walki o~granice. Więcej na ten temat \\
% w~Andrzej Chojnowski ,,Kwestia patriotyzmu w~poszukiwaniach
% programowych obozu piłsudczykowskiego'', str.~136
% \cite{PatriotyzmPolakow2006}.

% \vspace{\spaceFour}


% \start \Str{26} Podany tu opis przyczyn wybuchu I~Wojny Światowej,
% zwłaszcza bardzo silne stwierdzenie, że~Austro\dywiz Węgry
% wypowiedziały wojnę Serbii pod naciskiem Niemiec, warto
% skonfrontować z~tym co pisze M. Gilbert w~swojej książce na temat
% tego przedziwnego wydarzenia
% \cite{GilbertPierwszaWojnaSwiatowa2003}.

% \vspace{\spaceFour}


% \start \StrWd{78}{4} Cudzysłów otwarty w~tym wierszu nigdy nie
% został zamknięty, przez co~nie wiadomo, gdzie~się kończy cytat.

% \vspace{\spaceFour}


% \start \StrWd{425}{6} Przy nazwisku ,,Unrag Józef'' nie ma podanej
% strony na~której~się ta postać pojawia.

% \vspace{\spaceFour}


% \CenterTB{Błędy}
% \begin{center}
%   \begin{tabular}{|c|c|c|c|c|}
%     \hline
%     & \multicolumn{2}{c|}{} & & \\
%     Strona & \multicolumn{2}{c|}{Wiersz} & Jest
%                               & Powinno być \\ \cline{2-3}
%     & Od góry & Od dołu & & \\
%     \hline
%     20  & &  9 & sita & siła \\
%     27  & & 16 & z agrozić & zagrozić \\
%     29  & 21 & & Bąjończyków & Bajończyków \\
%     30  & 21 & & uczestniczyło w~niej & wśród jej członków było \\
%     31  & &  8 & Hans Beseler & Hans von Beseler \\
%     36  &  3 & & POW & POW. \\
%     37  & 12 & & LLOYDA & LOYDA \\
%     50  &  7 & & H.Wereszycki & H. Wereszycki \\
%     50  &  8 & & R.Dmowski & R.~Dmowski \\
%     50  &  8 & & J.Molenda & J.~Molenda \\
%     50  &  8 & & \emph{Pibudnczcy} & \emph{Piłsudczycy} \\
%     50  & 21 & & \emph{Pobki} & \emph{Polski} \\
%     68  & 14 & & j~ednolitego & jednolitego \\
%     73  & & 13 & 1920 R & 1920 R. \\
%     & & 17 & W braku & Z braku \\
%     %     & & & & \\
%     %     & & & & \\
%     %     & & & & \\
%     %     & & & & \\
%     %     & & & & \\
%     %     & & & & \\
%     %     & & & & \\
%     %     & & & & \\
%     %     & & & & \\
%     %     & & & & \\
%     411 &  4 & & \emph{50} & 50 \\
%     411 &  7 & & \emph{210, 211} & 210, 211 \\
%     411 &  9 & & \emph{113, 114, 170, 259} & 113, 114, 170, 259 \\
%     411 & 14 & & \emph{114} & 14 \\
%     411 & 18 & & \emph{389} & 389 \\
%     411 & &  6 & \emph{390} & 390 \\
%     411 & &  5 & \emph{391} & 391 \\
%     411 & &  3 & \emph{14}  & 14 \\
%     411 & &  3 & \emph{260} & 260 \\
%     412 &  2 & & \emph{390, 391} & 390, 391 \\
%     412 &  3 & & \emph{390, 391} & 390, 391 \\
%     412 &  5 & & \emph{50}  & 50 \\
%     412 &  7 & & \emph{210} & 210 \\
%     412 & 13 & & \emph{308} & 308 \\
%     412 & 13 & & \emph{171} & 171 \\
%     412 & 15 & & \emph{171} & 171 \\
%     412 & 19 & & \emph{211} & 211 \\
%     412 & 20 & & \emph{308} & 308 \\
%     412 & & 15 & \emph{113} & 113 \\
%     412 & &  9 & \emph{391} & 391 \\
%     412 & &  1 & \emph{14}  & 14 \\
%     412 & &  1 & \emph{113} & 113 \\
%     413 &  2 & & \emph{260} & 260 \\
%     \hline
%   \end{tabular}

%   \begin{tabular}{|c|c|c|c|c|}
%     \hline
%     & \multicolumn{2}{c|}{} & & \\
%     Strona & \multicolumn{2}{c|}{Wiersz} & Jest
%                               & Powinno być \\ \cline{2-3}
%     & Od góry & Od dołu & & \\
%     \hline
%     413 &  4 & & \emph{307} & 307 \\
%     413 &  8 & & \emph{50}  &  50\\
%     413 & 12 & & \emph{171, 210, 308, 309, 390}
%            & 171, 210, 308, 309, 390 \\
%     413 & 13 & & \emph{391} & 391 \\
%     413 & 15 & & \emph{390} & 390 \\
%     413 & 15 & & \emph{51}  & 51 \\
%     413 & 17 & & \emph{171} & 171 \\
%     413 & 19 & & \emph{390} & 390 \\
%     413 & & 20 & \emph{390} & 390 \\
%     413 & & 18 & \emph{307} & 307 \\
%     413 & & 17 & \emph{114} & 114 \\
%     413 & & 10 & \emph{113} & 113 \\
%     413 & &  7 & \emph{390} & 390 \\
%     413 & &  4 & \emph{50}  & 50 \\
%     413 & &  3 & \emph{260} & 260 \\
%     414 &  2 & & \emph{260} & 260 \\
%     414 &  3 & & \emph{170} & 170 \\
%     414 &  6 & & \emph{114} & 114 \\
%     414 &  7 & & \emph{170} & 170 \\
%     414 & 11 & & \emph{210} & 210 \\
%     414 & 12 & & \emph{211} & 211 \\
%     414 & & 19 & \emph{51}  & 51 \\
%     414 & & 16 & \emph{210} & 210 \\
%     414 & & 15 & \emph{113} & 113 \\
%     414 & & 10 & \emph{390} & 390 \\
%     414 & &  6 & \emph{307} & 307 \\
%     414 & &  2 & \emph{260} & 260 \\
%     %     & & & & \\
%     %     & & & & \\
%     %     & & & & \\
%     %     & & & & \\
%     %     & & & & \\
%     %     & & & & \\
%     %     & & & & \\
%     %     & & & & \\
%     %     & & & & \\
%     %     & & & & \\
%     %     & & & & \\ % & & & & \\
%     %     & & & & \\
%     %     & & & & \\
%     %     & & & & \\
%     %     & & & & \\
%     %     & & & & \\
%     %     & & & & \\
%     %     & & & & \\ % & & & & \\
%     %     & & & & \\
%     %     & & & & \\
%     %     & & & & \\
%     %     & & & & \\
%     %     & & & & \\
%     %     & & & & \\
%     %     & & & & \\
%     %     & & & & \\
%     %     & & & & \\
%     %     & & & & \\
%     %     & & & & \\
%     %     & & & & \\
%     %     & & & & \\
%     %     & & & & \\
%     %     & & & & \\
%     \hline
%   \end{tabular}

%   %   \begin{tabular}{|c|c|c|c|c|}
%   %   \hline
%   %   & \multicolumn{2}{c|}{} & & \\
%   %         %   Strona & \multicolumn{2}{c|}{Wiersz} & Jest
%   %         %   & Powinno być \\ \cline{2-3}
%   %         %   & Od góry & Od dołu & & \\
%   %         %   \hline
%   %   %   & & & & \\
%   %   %   & & & & \\
%   %   %   & & & & \\
%   %   %   & & & & \\
%   %   %   & & & & \\
%   %   %   & & & & \\
%   %   %   & & & & \\
%   %   %   & & & & \\
%   %   %   & & & & \\
%   %   %   & & & & \\
%   %   %   & & & & \\
%   %   %   & & & & \\
%   %   %   & & & & \\
%   %   %   & & & & \\
%   %   %   & & & & \\
%   %   %   & & & & \\
%   %   %   & & & & \\
%   %   %   & & & & \\
%   %   %   & & & & \\
%   %   %   & & & & \\
%   %   %   & & & & \\
%   %   %   & & & & \\
%   %   %   & & & & \\
%   %   %   & & & & \\
%   %   %   & & & & \\
%   %   %   & & & & \\
%   %   %   & & & & \\
%   %   %   & & & & \\
%   %   %   & & & & \\
%   %   %   & & & & \\
%   %   %   & & & & \\
%   %   %   & & & & \\
%   %   %   & & & & \\
%   %   %   & & & & \\
%   %   %   & & & & \\
%   %   %   & & & & \\
%   %   %   & & & & \\
%   %   %   & & & & \\
%       %   \hline
%       % \end{tabular}

% \begin{tabular}{|c|c|c|c|c|}
%     \hline
%     & \multicolumn{2}{c|}{} & & \\
%     Strona & \multicolumn{2}{c|}{Wiersz} & Jest
%                               & Powinno być \\ \cline{2-3}
%     & Od góry & Od dołu & & \\
%     \hline
%     425 &  8 & & \emph{170} & 170 \\
%     425 & 13 & & \emph{50}  & 50 \\
%     425 & 20 & & Adolf25 & Adolf 25 \\
%     425 & 23 & & \emph{170} & 170 \\
%     425 & & 16 & \emph{114, 170, 211, 260} & 114, 170, 211, 260 \\
%     425 & & 15 & \emph{308, 391} & 308, 391 \\
%     425 & & 14 & \emph{389} & 389 \\
%     425 & & 13 & \emph{113, 170, 171} & 113, 170, 171 \\
%     425 & & 12 & \emph{308} & 308 \\
%     425 & &  8 & \emph{50}  &  50 \\
%     425 & &  3 & \emph{389} & 389 \\
%     426 &  2 & & \emph{114} & 114 \\
%     426 &  9 & & \emph{50}  &  50 \\
%     426 & 14 & & \emph{211} & 211 \\
%     426 & 17 & & \emph{50}  &  50 \\
%     426 & 18 & & \emph{114} & 114 \\
%     426 & 21 & & \emph{259} & 259 \\
%     426 & 22 & & \emph{260} & 260 \\
%     426 & 23 & & \emph{308} & 308 \\
%     426 & &  6 & \emph{50}  &  50 \\
%     426 & &  5 & \emph{170} & 170 \\
%     426 & &  3 & \emph{170} & 170 \\
%     426 & &  2 & \emph{210, 211} & 210, 211 \\
%     427 &  4 & & \emph{389} & 389 \\
%     427 &  7 & & \emph{170, 391} & 170, 391 \\
%     427 &  8 & & \emph{114} & 114 \\
%     \hline
%   \end{tabular}
% \end{center}
% \noi
% \StrWd{47}{6} \\
% \Jest  nadchodzącą zimą \\
% \Powin nadchodzącej zimy \\

% \vspace{\spaceTwo}





% % ############################
% \subsection{Lata 1939--1989}

% \vspace{\spaceThree}
% % ############################



% % ####################
% \Work{ % Redaktorzy i tytuł dzieła
% Redakcja i~opracowanie Adam Dziurok, Filip Musiał \\
% ,,Instrukcje, wytyczne, pisma Departamentu IV~Ministerstwa Spraw
% Wewnętrznych z~lat 1962--1989. Wybór dokumentów'',
% \cite{RedDziurokMusialInstrukcjeWytycznePisma2017} }

% \CenterTB{Uwagi}

% \start \Str{905} Nazwiska pisane czcionką prostą (antykwą?) należą
% do~bohaterów historii omawianej w~tym tomie, zaś~te pisane kursywą
% do~badaczy i~historyków.

% \vspace{\spaceTwo}





% % ####################
% \Work{ % Redaktor i tytuł dzieła
% Red. Piotr Franaszka \\
% ,,Granice kompromisu. Naukowcy wobec aparatu władzy ludowej'',
% \cite{RedFranaszekGraniceKompromisu2015} }


% % \CenterTB{Uwagi}

% \CenterTB{Błędy}
% \begin{center}
%   \begin{tabular}{|c|c|c|c|c|}
%     \hline
%     & \multicolumn{2}{c|}{} & & \\
%     Strona & \multicolumn{2}{c|}{Wiersz} & Jest
%                               & Powinno być \\ \cline{2-3}
%     & Od góry & Od dołu & & \\
%     \hline
%     9  & &  2 & A.Dziuba & A.~Dziuba \\
%     17 & 14 & & ZMP$^{ 33 }$\ldots & ZMP$^{ 33 }$. \\
%     %     & & & & \\
%     %     & & & & \\
%     %     & & & & \\
%     \hline
%   \end{tabular}
% \end{center}

% \vspace{\spaceTwo}





% % ####################
% \Work{ % Autor i tytuł dzieła
% Wojciech Roszkowski \\
% ,,Najnowsza historia Polski: 1939--1945'',
% \cite{RoszkowskiNajnowszaHistoriaPolski39-45Wyd2011} }


% \CenterTB{Uwagi}

% \CenterTB{Błędy}
% \begin{center}
%   \begin{tabular}{|c|c|c|c|c|}
%     \hline
%     & \multicolumn{2}{c|}{} & & \\
%     Strona & \multicolumn{2}{c|}{Wiersz} & Jest
%                               & Powinno być \\ \cline{2-3}
%     & Od góry & Od dołu & & \\
%     \hline
%     10 & & 5 & Wisy & Wisły \\
%     %     & & & & \\
%     %     & & & & \\
%     %     & & & & \\
%     \hline
%   \end{tabular}
% \end{center}

% \vspace{\spaceTwo}

% % \Work{
% % W. Roszkowski \\
% % ,,Najnowsza historia Polski: 1914--1939'', \cite{Ros11a}.}


% % \CenterTB{Uwagi:} \start Karygodną i to~niezależnie od uznawanej
% % metodologi pisania prac historycznych, cechą całego tego wydania
% % ,,Najnowszej historii Polski'', jest nieumieszczenie w~każdej tomie
% % listy używanych skrótów,
% % niezależnie od tego, czy zostały one wprowadzone w~tym, czy~w~którymś z~poprzednich. \\
% % \start Ciekawym wydaje~się zauważanie, że~w~tej książce Roszkowski
% % zrealizował chyba idealnie, jedno z~założeń zaprojektowanego przez
% % piłsudczyków programu edukacji, przyjętego po reformie
% % jędrzejewiczowskiej (1932): sprowadzenia lat 1918--1920 wyłączenie
% % do~tematu walki o~granice. Więcej na ten temat \\ w~Andrzej Chojnowski ,,Kwestia patriotyzmu w~poszukiwaniach programowych obozu piłsudczykowskiego'', str. 136 \cite{PP06}. \\
% % \start \Str{26} Podany tu opis przyczyn wybuchu I~Wojny Światowej,
% % zwłaszcza bardzo silne stwierdzenie, że~Austro\dywiz Węgry
% % wypowiedziały wojnę Serbii pod naciskiem Niemiec, warto
% % skonfrontować z~tym co pisze M. Gilbert w~swojej książce na temat
% % tego przedziwnego wydarzenia \cite{Gil03}. \\
% % \start \StrWd{78}{4} Cudzysłów otwarty w~tym wierszu nigdy nie został zamknięty, przez co~nie wiadomo, gdzie~się kończy cytat. \\

% % Błędy:\\
% % \begin{center}
% %   \begin{tabular}{|c|c|c|c|c|}
% %     \hline
% %     & \multicolumn{2}{c|}{} & & \\
% %     Strona & \multicolumn{2}{c|}{Wiersz}& Jest & Powinno być \\ \cline{2-3}
% %     & Od góry & Od dołu &  &  \\ \hline
% %     & & & & \\
% %     20 & & 9 & sita & siła \\
% %     27 & & 16 & z agrozić & zagrozić \\
% %     31 & & 8 & Hans Beseler & Hans von Beseler \\
% %     36 & 3 & & POW & POW. \\
% %     37 & 12 & & LLOYDA & LOYDA \\
% %     47 & & 6 & przed nadchodzącą zimą & nadchodzącej zimy \\
% %     50 & 21 & & \emph{Pobki} & \emph{Polski} \\
% %     73 & & 13 & 1920 R & 1920 R. \\
% %     & & 17 & W braku & Z braku \\
% %     & & & & \\ \hline
% %   \end{tabular}
% % \end{center}





% % ####################
% \Work{ % Autor i tytuł dzieła
% Wojciech Roszkowski \\
% ,,Najnowsza historia Polski: 1980--1989'',
% \cite{RoszkowskiNajnowszaHistoriaPolski80-89Wyd2011} }


% \CenterTB{Uwagi}

% \start \Str{8} Jest to jeden z~największych przykładów straszliwie
% suchej, pozbawiającej wydarzenia z~przeszłości realności, a~także
% nie pozwalającej~się zorientować w~symbolach o~ogromnej wadze,
% historiografii pozytywistycznej, chyba szkoły niemieckiej, jaką
% reprezentuje Roszkowski. Jeśli dobrze wywnioskowałem z~tego
% wystąpienia
% \href{https://www.youtube.com/watch?v=6B93_3CCMac}{Sławomira
% Cenckiewicza}, to słynny skok Wałęsy przez płot, wedle niego był to
% w~istocie mur, miał miejsce w~opisywanym tu dniu 22~sierpnia 1980~r.
% Wałęsa musiał przeskoczyć ten mur, właśnie dlatego, że~spóźnił~się
% na główne otwarcie stoczni. Tylko pomarzyć jak~by to opisał Paul
% Johnson.

% \vspace{\spaceFour}


% \start \Str{16} Choć nie~czytałem poprzednich tomów, i~być może jest
% tam informacja o~tym kim jest Karol Modzelewski, to w~tym tomie jest
% wymieniony tylko dwa razy, i~z tego powodu powinna być podana jakaś
% informacja o~nim, by~czytelnik wiedział kto jest autorem tak wiele
% znaczącej nazwy jak ,,Solidarność''.

% \vspace{\spaceFour}


% \start \StrWd{23}{6} Cudzysłów otwarty w~tym wierszu nigdy nie
% został zamknięty, przez co~nie wiadomo, gdzie~się kończy cytat.

% \vspace{\spaceFour}


% \start \StrWd{26}{6} Nie jest podane co było tematem omawianej
% tu~narady sztabowej.


% % Błędy:\\
% % \begin{center}
% %   \begin{tabular}{|c|c|c|c|c|}
% %     \hline
% %     & \multicolumn{2}{c|}{} & & \\
% %           %     Strona & \multicolumn{2}{c|}{Wiersz} & Jest
% %     %     & Powinno być \\ \cline{2-3}
% %           %     & Od góry & Od dołu & & \\
% %     %     \hline
% %     & & & & \\
% %     & & & & \\ \hline
% %     % %   \end{tabular}
% %     % % \end{center}

% \vspace{\spaceTwo}





% % ##################
% \Work{ % Autor i tytuł dzieła
% Wojciech Roszkowski \\
% ,,Najnowsza historia Polski: 1980--1989'',
% \cite{RoszkowskiNajnowszaHistoriaPolski80-89Wyd2011} }


% \CenterTB{Uwagi}

% \start \Str{8} Jest to jeden z~najgorszych przykładów straszliwie
% suchej, pozbawiającej przeszłości realności, a~także nie
% pozwalającej~się zorientować w~symbolach o~ogromnej wadze,
% historiografii pozytywistycznej, chyba szkoły niemieckiej, jaką
% reprezentuje Roszkowski. Jeśli dobrze wywnioskowałem z~tego
% wystąpienia
% \href{https://www.youtube.com/watch?v=6B93_3CCMac}{Sławomira
% Cenckiewicza}, to słynny skok Wałęsy przez płot, wedle Cenckiewicza
% niego był to w~istocie mur, miał miejsce w~opisywanym tu dniu
% 22~sierpnia 1980~r. Wałęsa musiał przeskoczyć ten mur, właśnie
% dlatego, że~spóźnił~się na główne otwarcie stoczni, kiedy wszedł by
% do niej chowając~się w~tłumie tysięcy pracowników. Tylko pomarzyć
% jak~by to opisał Paul Johnson.

% \vspace{\spaceFour}


% \start \Str{16} Choć nie~czytałem poprzednich tomów, i~być może jest
% tam informacja o~tym kim jest Karol Modzelewski, to w~tym tomie jest
% wymieniony tylko dwa razy i~nie da~się zrozumieć kim on jest.
% Powinna być tu podana jakaś krótka informacja, kim on był, tak aby
% osoba czytająca tylko ten tom, wiedział kto jest autorem tak wiele
% znaczącej nazwy jak ,,Solidarność''.

% \vspace{\spaceFour}


% \start \StrWd{23}{6} Cudzysłów otwarty w~tym wierszu nigdy nie
% został zamknięty, przez co~nie wiadomo, gdzie~się kończy cytat.

% \vspace{\spaceFour}


% \start \StrWd{26}{6} Nie jest podane co było tematem omawianej
% tu~narady sztabowej.


% % Błędy:\\
% % \begin{center}
% %   \begin{tabular}{|c|c|c|c|c|}
% %     \hline
% %     & \multicolumn{2}{c|}{} & & \\
% %           %     Strona & \multicolumn{2}{c|}{Wiersz} & Jest
% %     %     & Powinno być \\ \cline{2-3}
% %           %     & od góry & Od dołu & & \\
% %     %     \hline
% %     & & & & \\
% %     & & & & \\ \hline
% %     % %   \end{tabular}
% %     % % \end{center}

% \vspace{\spaceTwo}





% % ##################
% \Work{ % Redaktorzy i tytuł dzieła
% Redakcja naukowa Mirosław Sikora; współpraca Piotr Fuglewicz \\
% ,,High\dywiz tech za~żelazną kurtyną: elektronika, komputery
% i~systemy sterowania w~PRL'', \cite{SikoraFuglewiczHighTech2017} }


% % \CenterTB{Uwagi}

% % \start \StrWd{26}{6} Nie jest podane co było tematem omawianej
% % tu~narady sztabowej.


% Błędy:\\
% \begin{center}
%   \begin{tabular}{|c|c|c|c|c|}
%     \hline
%     & \multicolumn{2}{c|}{} & & \\
%     Strona & \multicolumn{2}{c|}{Wiersz} & Jest
%                               & Powinno być \\ \cline{2-3}
%     & od góry & Od dołu & & \\
%     \hline
%     4   & &  7 & przeciwko & Przeciwko \\
%     22  & 13 & & o\ld{} fantastyce & o~fantastyce \\ % Czy to aby na pewno
%     %     błąd?
%     50  & & 15 & tej uchwały & tej \\
%     %     & & & & \\
%     %     & & & & \\
%     %     & & & & \\
%     \hline
%   \end{tabular}
% \end{center}

% \vspace{\spaceTwo}





% % % ############################
% % \newpage
% % \Field{Polska po~1939~r.}
% % % \Subfield{}

% % \vspace{\spaceTwo} \vspace{\spaceThree}
% % % ############################



% % ############################
% \section{Dzieje Polski po~1989~r.}

% \vspace{\spaceTwo}
% % ############################



% % ##################
% \Work{ % Autorzy i tytuł dzieła
% Jan Kofman, Wojciech Roszkowski \\
% ,,Transformacja i postkomunizm'',
% \cite{KofmanRoszkowskiTransformacjaIPostkomunizm1999} }


% \CenterTB{Uwagi}

% \start \Str{42} Przeliczyłem za~pomocą komputera przytoczone tu~dane
% i~otrzymany wynik nie zawsze pokrywał~się z~tym co zostało podane
% w~ostatniej kolumnie. Dokładniej, wyniki różniły~się zawsze, jednak
% w~wielu wypadkach był to zapewne wynik przyjętego sposobu
% zaokrąglania, dlatego w~błędach odnotowałem tylko te przypadki,
% gdy~różnica była rzędu procenta lub~większa.

% \vspace{\spaceFour}


% \CenterTB{Błędy}

% \begin{center}
%   \begin{tabular}{|c|c|c|c|c|}
%     \hline
%     & \multicolumn{2}{c|}{} & & \\
%     Strona & \multicolumn{2}{c|}{Wiersz} & Jest
%                               & Powinno być \\ \cline{2-3}
%     & od góry & od dołu & & \\
%     \hline
%     13  & &  4 & o~ekspansji & do~ekspansji \\
%     16  &  9 & & marntrawstwem & marnotrawstwem \\
%     18  &  2 & & jego do & do~jego \\
%     26  & & 19 & dziewięćdziesięciokrotne & czterdziestokrotnie \\
%     26  & &  5 & 1998 & 1988 \\
%     42  & 11 & & 80,2 & 78,5 \\
%     42  & 14 & & (107,4) & (108,3) \\
%     42  & & 11 & 77,8 &  76,6 \\
%     42  & &  8 & (35,7) & (37,4) \\
%     %     & & & & \\
%     %     & & & & \\
%     %     & & & & \\
%     %     & & & & \\
%     %     & & & & \\
%     %     & & & & \\
%     \hline
%   \end{tabular}
% \end{center}

% \vspace{\spaceTwo}





% % ##################
% \Work{ % Autor i tytuł dzieła
% Wojciech Roszkowski \\
% ,,Najnowsza historia Polski: 1989--2011'',
% \cite{RoszkowskiNajnowszaHistoriaPolski89-11Wyd2011} }


% % \CenterTB{Uwagi}

% \CenterTB{Błędy}
% \begin{center}
%   \begin{tabular}{|c|c|c|c|c|}
%     \hline
%     & \multicolumn{2}{c|}{} & & \\
%     Strona & \multicolumn{2}{c|}{Wiersz} & Jest
%                               & Powinno być \\ \cline{2-3}
%     & Od góry & Od dołu & & \\
%     \hline
%     23 & &  2 & na~naturalnym & za~naturalnym \\
%     38 & & 11 & IX & XI \\
%     45 & &  2 & przez & przed \\
%     46 & & 15 & 2,5\%\% & 2,5\% \\
%     65 & & 16 & wygrali & wyciągnęli \\
%     %     & & & & \\
%     %     & & & & \\
%     %     & & & & \\
%     %     & & & & \\
%     %     & & & & \\
%     %     & & & & \\
%     %     & & & & \\
%     \hline
%   \end{tabular}
% \end{center}

% \vspace{\spaceTwo}







% % ######################################
% \section{Historia Polski --~zbiory artykułów}

% \vspace{\spaceTwo}
% % ######################################


% % ##################
% \Work{ % Autor i tytuł dzieła
% Janusz Cisek \\
% ,,Oskar Halecki. Historyk --~Szermierz Wolności'',
% \cite{CisekOskarHalecki2009} }


% \CenterTB{Uwagi}

% \start \Str{5} Tytuł części ,,Historyk Kościoła'' jest trochę
% nieadekwatna, bowiem dwa z~trzech zamieszczonych tu artykułów
% dotyczą historii Kościoła w~Polsce, co jest bardzo małym wycinkiem
% z~tego zagadnienia.

% \vspace{\spaceFour}


% \start \StrWg{48}{4} Artykuł ten mógł~się rzeczywiście ukazać
% w~wymienionym tu roku 1963, jednak spodziewałbym~się raczej tego,
% że~został opublikowany w~1966 roku.

% \vspace{\spaceFour}


% \start \StrWg{70}{7} Słowa ,,i~jego pielgrzymki'' mogą być wynikiem
% błędu, w~poprawnej wersji powinny brzmieć ,,i~jego pielgrzymi''.
% Mogą też mieć następujący sens. Poeci romantyczni byli
% ,,wieszczami'' narodu, jak i~w~szczególny sposób, czasów jego
% pielgrzymki.

% \CenterTB{Błędy}
% \begin{center}
%   \begin{tabular}{|c|c|c|c|c|}
%     \hline
%     & \multicolumn{2}{c|}{} & & \\
%     Strona & \multicolumn{2}{c|}{Wiersz} & Jest
%                               & Powinno być \\ \cline{2-3}
%     & Od góry & Od dołu & & \\
%     \hline
%     44  & 14 & & Honorowej: & Honorowej, \\
%     90  & 10 & & W~braku & Z~braku \\
%     93  & & 12 & Rzym. & Rzym, \\
%     %     & & & & \\
%     %     & & & & \\
%     %     & & & & \\
%     %     & & & & \\
%     %     & & & & \\
%     %     & & & & \\
%     %     & & & & \\
%     %     & & & & \\
%     %     & & & & \\
%     %     & & & & \\
%     %     & & & & \\
%     %     & & & & \\
%     %     & & & & \\
%     %     & & & & \\
%     %     & & & & \\
%     %     & & & & \\
%     %     & & & & \\
%     %     & & & & \\
%     %     & & & & \\
%     %     & & & & \\
%     %     & & & & \\
%     \hline
%   \end{tabular}
% \end{center}

% ######################################
\section{Eseje i~publicystyka}

\vspace{\spaceTwo}
% ######################################



% ############################
\Work{ % Autor i tytuł dzieła
  Andrzej Nowak \\
  „Strachy i lachy. Przemiany polskiej pamięci 1982--2012”,
  \cite{Nowak12} }


\CenterTB{Uwagi}

\start \Str{47} T.~S.~Eliot jest na tej stronie nazwany „wielkim
poetą katolickim”, acz z~tego co wiem do Kościoła nigdy nie
przyszedł, zamiast tego dołączył do jakiegoś wyznania
anglokatolickiego. Zaś użycie przymiotnika „wielki” w~odniesieniu to
tego poety, którego twórczości nie da~się czytać, jest już na~pewno
błędem.

\vspace{\spaceTwo}
% ############################










% ############################
\Work{ % Autor i tytuł dzieła
  Andrzej Nowak \\
  „Intelektualna historia III~RP. Rozmowy z~lat 1990--2012”,
  \cite{NowakIntelektualnaHistoriaIIIRP2013} }


% ##################
\CenterTB{Błędy}

\begin{center}

  \begin{tabular}{|c|c|c|c|c|}
    \hline
    & \multicolumn{2}{c|}{} & & \\
    Strona & \multicolumn{2}{c|}{Wiersz} & Jest
                              & Powinno być \\ \cline{2-3}
    & Od góry & Od dołu & & \\
    \hline
    195 &  6 & & pomocą & pomocy \\
    579 & &  3 & osób. & osób, \\
    580 & & 10 & od & do \\
    665 &  7 & & The~National Interest” & „The~National Interest” \\
    % & & & & \\
    % & & & & \\
    \hline
  \end{tabular}

\end{center}

\noindent
\tb{Przednia okładka, wiersz 14.} \\
\Jest \textit{ImperologicalStudies.APolishPerspective}(2011);
\textit{Czaswalki} \\
\Powin \textit{Imperological Studies. A Polish Perspective} (2011);
\textit{Czas walki} \\
\tb{Przednia okładka, wiersz 10 (od dołu).} \\
\Jest  \ldots w~Brnie \\
\Powin w~Brnie \\

\vspace{\spaceTwo}
% ############################







% ##################
\Work{ % Autor i tytuł dzieła
  Andrzej Nowak \\
  „Historia i~polityka”, \cite{NowakHistoriaIPolityka2016} }


\CenterTB{Uwagi}

\start \Str{} Nowak popełni tu pewien błąd pisząc o~grze komputerowej
„Dzikie Pola”, jest to standardowa stołowa gra RPG i~nie ma nic
wspólnego z~komputerem.


\CenterTB{Błędy}
\begin{center}

  \begin{tabular}{|c|c|c|c|c|}
    \hline
    & \multicolumn{2}{c|}{} & & \\
    Strona & \multicolumn{2}{c|}{Wiersz} & Jest
                              & Powinno być \\ \cline{2-3}
    & Od góry & Od dołu & & \\
    \hline
    6   &  2 & & 448 & 484 \\
    18  & 18 & & „ kult & „kult \\
    % & & & & \\
    % & & & & \\
    % & & & & \\
    % & & & & \\
    99  & &  2 & (1918--1920) & (1918--2008) \\
    133 & & 23 & 1981 & 1918 \\
    135 & &  5 & 361 (przytoczony & 361. Przytoczony \\
    % & & & & \\
    % & & & & \\
    \hline
  \end{tabular}

\end{center}

\vspace{\spaceTwo}








% ####################################################################
% ####################################################################
% Bibliografia
\bibliographystyle{plalpha}

\bibliography{LibAliaDoc}{}





% ############################

% Koniec dokumentu
\end{document}