% Autor: Kamil Ziemian

% ---------------------------------------------------------------------
% Podstawowe ustawienia i pakiety
% ---------------------------------------------------------------------
\RequirePackage[l2tabu, orthodox]{nag}  % Wykrywa przestarzałe i niewłaściwe
% sposoby używania LaTeXa. Więcej jest w l2tabu English version.
\documentclass[a4paper,11pt]{article}
% {rozmiar papieru, rozmiar fontu}[klasa dokumentu]
\usepackage[MeX]{polski}  % Polonizacja LaTeXa, bez niej będzie pracował
% w języku angielskim.
\usepackage[utf8]{inputenc}  % Włączenie kodowania UTF-8, co daje dostęp
% do polskich znaków.
\usepackage{lmodern}  % Wprowadza fonty Latin Modern.
\usepackage[T1]{fontenc}  % Potrzebne do używania fontów Latin Modern.



% ------------------------------
% Podstawowe pakiety (niezwiązane z ustawieniami języka)
% ------------------------------
\usepackage{microtype}  % Twierdzi, że poprawi rozmiar odstępów w tekście.
% \usepackage{graphicx}  % Wprowadza bardzo potrzebne komendy do wstawiania
% % grafiki.
% \usepackage{verbatim}  % Poprawia otoczenie VERBATIME.
% \usepackage{textcomp}  % Dodaje takie symbole jak stopnie Celsiusa,
% wprowadzane bezpośrednio w tekście.
\usepackage{vmargin}  % Pozwala na prostą kontrolę rozmiaru marginesów,
% za pomocą komend poniżej. Rozmiar odstępów jest mierzony w calach.
% ------------------------------
% MARGINS
% ------------------------------
\setmarginsrb
{ 0.7in} % left margin
{ 0.6in} % top margin
{ 0.7in} % right margin
{ 0.8in} % bottom margin
{  20pt} % head height
{0.25in} % head sep
{   9pt} % foot height
{ 0.3in} % foot sep



% ------------------------------
% Często używane pakiety
% ------------------------------
% \usepackage{csquotes}  % Pozwala w prosty sposób wstawiać cytaty do tekstu.
\usepackage{xcolor}  % Pozwala używać kolorowych czcionek (zapewne dużo
% więcej, ale ja nie potrafię nic o tym powiedzieć).





% ---------------------------------------------------------------------
% Dodatkowe ustawienia dla języka polskiego
% ---------------------------------------------------------------------
\renewcommand{\thesection}{\arabic{section}.}
% Kropki po numerach rozdziału (polski zwyczaj topograficzny)
\renewcommand{\thesubsection}{\thesection\arabic{subsection}}
% Brak kropki po numerach podrozdziału



% ------------------------------
% Pakiety napisane przez użytkownika.
% Mają być w tym samym katalogu to ten plik .tex
% ------------------------------
\usepackage{latexgeneralcommands}



% ------------------------------
% Ustawienia różnych parametrów tekstu
% ------------------------------
\renewcommand{\arraystretch}{1.2}  % Ustawienie szerokości odstępów między
% wierszami w tabelach.



% ------------------------------
% Pakiet "hyperref"
% Polecano by umieszczać go na końcu preambuły.
% ------------------------------
\usepackage{hyperref}  % Pozwala tworzyć hiperlinki i zamienia odwołania
% do bibliografii na hiperlinki.










% ---------------------------------------------------------------------
% Tytuł tekstu
\title{Kultura~-- błędy i~uwagi}

% \author{}
% \date{}
% ---------------------------------------------------------------------










% ####################################################################
% Początek dokumentu
\begin{document}
% ####################################################################





% ######################################
\maketitle  % Tytuł całego tekstu
% ######################################





% ######################################
\section{Kultura od~XVII wieku do~dziś}

\vspace{\spaceTwo}
% ######################################










% ############################
\subsection{Kultura XIX i~XX wieku}

\vspace{\spaceThree}
% ############################



% ############################
\Work{ % Autor i tytuł dzieła
  E. Michael Jones \\
  „Zdeprawowani moderniści”,
  \cite{EMichaelJonesZdeprawowaniModernisci14} }


% ##################
\CenterBoldFont{Błędy}


\begin{center}

  \begin{tabular}{|c|c|c|c|c|}
    \hline
    & \multicolumn{2}{c|}{} & & \\
    Strona & \multicolumn{2}{c|}{Wiersz} & Jest
                              & Powinno być \\ \cline{2-3}
    & Od góry & Od dołu & & \\
    \hline
    16  & & 19 & Claya & Gaya \\
    24  &  4 & & człowieczeństwa & człowieczeństwa'' \\
    25  & 11 & & Samoa & „Samoa \\
    25  & & 14 & wyłączności & Wyłączności \\
    27  & &  1 & beztroskich>> & beztroskich>>” \\
    29  & 12 & & roku & roku. \\
    30  & &  9 & <<młodej studentki” % >>
           & <<młodej studentki>> \\
    30  & &  8 & bawełniane sukienki>> & <<bawełniane sukienki>> \\
    37  & 11 & & wówczas” & wówczas \\
    37  & & 16 & niego & z~niego \\
    38  & &  2 & zbagatelizowałaś, & zbagatelizowałaś. \\
    49  & 12 & & za & z \\
    56  & 17 & & którym & których \\
    56  &  3 & & Wilberforce'a** & Wilberforce'a* \\
    67  & & 13 & \emph{Whay} & \emph{What} \\
    71  & 14 & & „ Wartości & „Wartości \\
    85  & & 13 & \emph{lalek} & \emph{lalek}” \\
    91  & &  4 & a oni & „a oni \\
    99  &  9 & & maja & mają \\
    101 & & 11 & naukowym & z naukowym \\
    103 & & 13 & ATA & ATS \\
    105 & 19 & & mniej silna & silniejsza \\
    107 & 19 & & Voris & Vorisem \\
    108 &  2 & & rzeczywistości & o rzeczywistości \\
    111 & 21 & & Indiana, & Indiana. \\
    111 & & 20 & w na & na \\
    119 & & 11 & oszukiwałam”$^{ 2 }$.W & oszukiwałam”$^{ 2 }$. W \\
    122 & 20 & & od & do \\
    161 & &  2 & wyraźn0e & wyraźne \\
    165 & &  3 & ktrego & którego \\
    \hline
  \end{tabular}





  \begin{tabular}{|c|c|c|c|c|}
    \hline
    & \multicolumn{2}{c|}{} & & \\
    Strona & \multicolumn{2}{c|}{Wiersz} & Jest
                              & Powinno być \\ \cline{2-3}
    & Od góry & Od dołu & & \\
    \hline
    171 & 12 & & zajmującysię & zajmujący~się \\
    186 & &  5 & umożliwiła & uniemożliwiła \\
    199 & & 15 & wiary & utraty wiary \\
    200 & &  1 & Mogło & Mogła \\
    201 & & 14 & roku & roku. \\
    205 & & 19 & Laetesa & Klaudiusza \\
    207 &  7 & & taka & taką \\
    211 & & 18 & „Szczyt & Szczyt \\
    214 &  5 & & [ w~nagrodę] & [w~nagrodę] \\
    214 &  6 & & najwyraźniej .. & najwyraźniej\ldots \\
    217 & 22 & & Fread & Freuda \\
    218 & &  2 & od~z & z \\
    226 & 10 & & Mannung & Manning \\
    228 & & 17 & pytanie:, & pytanie: \\
    230 & 12 & & 1963 roku & 1963 roku. \\
    230 & 13 & & 1425 & 1525 \\
    237 & &  4 & 560 & 1560 \\
    \hline
  \end{tabular}

\end{center}


\noindent
\StrWd{17}{10} \\
\Jest nieczystości pierworodnej córki jest ślepotą ducha. \\
\Powin pierworodną córką nieczystości jest ślepota ducha. \\
\StrWg{201}{8} \\
\Jest otorbił~się w~kokonie psychoanalizy samego siebie\ldots \\
\Powin otorbił samego siebie w~kokonie psychoanalizy\ldots \\
\StrWg{209}{20} \\
\Jest \emph{Gdyby potrafił być perwersyjny, byłby zdrowy, podobnie jak
  ojciec$^{ 202 }$}. \\
\Powin „Gdyby potrafił być perwersyjny, byłby zdrowy,
podobnie jak ojciec”$^{ 202 }$. \\


\vspace{\spaceTwo}
% ############################










% ######################################
\section{Zachodni półwysep Eurazji}

\vspace{\spaceTwo}
% ######################################



% ############################
\subsection{Zagadnienia ogólne}

\vspace{\spaceThree}
% ############################



% ############################
\Work{ % Autor i tytuł dzieła
  Zdzisław Krasnodębski \\
  „Drzemka rozsądnych. Zebrane eseje i~szkice”,
  \cite{KrasnodebskiDrzemkaRozsadnych06} }


% ##################
\CenterBoldFont{Uwagi}


\begin{center}

  \begin{tabular}{|c|c|c|c|c|}
    \hline
    & \multicolumn{2}{c|}{} & & \\
    Strona & \multicolumn{2}{c|}{Wiersz} & Jest
                              & Powinno być \\ \cline{2-3}
    & Od góry & Od dołu & & \\
    \hline
    11  & &  5 & z OMP & OMP \\
    145 & &  3 & 1980 & 1989 \\
    168 &  1 & & których~się & której~się \\
    168 &  1 & & których żyją & której żyją \\
    223 & &  2 & \emph{Rzeczpospoliej} & \emph{Rzeczpospolitej} \\
    254 & & 15 & partią & partii \\
    310 & & 14 & zachęcić & zachęcić~go \\
    325 & 14 & & spierający & spierający~się \\
    % & & & & \\
    % & & & & \\
    % & & & & \\
    % & & & & \\
    \hline
  \end{tabular}

\end{center}

\noindent
\StrWd{72}{2} \\
\Jest Frankreich 1871, Deutschland 1918 \\
\Powin  \emph{Frankreich 1871, Deutschland 1918} \\

\vspace{\spaceTwo}
% ############################










% ############################
\Work{ % Autor i tytuł dzieła
  Zdzisław Krasnodębski \\
  „Zwycięzca po~przejściach. Zebrane eseje i~szkice~V”,
  \cite{KrasnodebskiZwyciezcaPoPrzejsciach2012} }


% ##################
\CenterBoldFont{Uwagi do~konkretnych stron}


\start \Str{317} Pominięto miejsce i~datę pierwszej publikacji
artykułu \emph{Nie udawaj Greka, Polsko!}


\vspace{\spaceTwo}
% ############################










% ############################
\subsection{Kultura polska i~analizy tego kraju}

\vspace{\spaceThree}
% ############################



% ############################
\Work{ % Autor i tytuł dzieła
  Maria Janion \\
  „Niesamowita Słowiańszczyzna. Fantazmaty literatury”,
   \cite{JanionNiesamowitaSlowianszczyzna2006} }


% ##################
\CenterBoldFont{Uwagi}


\begin{center}

  \begin{tabular}{|c|c|c|c|c|}
    \hline
    & \multicolumn{2}{c|}{} & & \\
    Strona & \multicolumn{2}{c|}{Wiersz} & Jest
                              & Powinno być \\ \cline{2-3}
    & Od góry & Od dołu & & \\
    \hline
    10  & & 12 & postkolonialnej?$^{ 10 }$. & postkolonialnej$^{ 10 }$? \\
    60  & &  6 & puścizny & spuścizny \\
    113 & 18 & & i~\emph{Masław} & \emph{i~Masław} \\
    175 & 16 & & ciemięzców & ciemiężców \\
    175 & 19 & & ciemięzców & ciemiężców \\
    % & & & & \\
    % & & & & \\
    % & & & & \\
    % & & & & \\
    % & & & & \\
    % & & & & \\
    % & & & & \\
    \hline
  \end{tabular}

\end{center}


\noindent
\StrWd{72}{2} \\
\Jest  Frankreich 1871, Deutschland 1918 \\
\Powin \textit{Frankreich 1871}, \textit{Deutschland 1918} \\


\vspace{\spaceTwo}
% ############################










% ############################
\Work{ % Autor i tytuł dzieła
  Jan Sowa \\
  „Fantomowe ciało króla. Peryferyjne zmagania z~nowoczesną formą”,
  \cite{SowaFantomoweCialoKrola11} }


% ##################
\CenterBoldFont{Uwagi}


\start \StrWg{36}{15} Zdanie „jego aktywność ściśle wiąże~się
związana ze społeczną\ldots”, powinno brzmieć „jego aktywność ściśle
wiąże~się ze społeczną\ldots” lub „jego aktywność jest ściśle
związana ze społeczną\ldots”. Choć na podstawie tekstu, nie można
wybrać wersji zamierzonej przez autora, nie~jest to jednak problemem,
bo obie przekazują tą samą treść.

\vspace{\spaceFour}



\start \StrWg{63}{5--9} Zawarty tu tekst, jest źle skonstruowany
gramatycznie.





% ##################
\CenterBoldFont{Błędy}


\begin{center}

  \begin{tabular}{|c|c|c|c|c|}
    \hline
    & \multicolumn{2}{c|}{} & & \\
    Strona & \multicolumn{2}{c|}{Wiersz} & Jest
                              & Powinno być \\ \cline{2-3}
    & Od góry & Od dołu & & \\
    \hline
    92  & &  6 & 1140 & 1440 \\
    102 & 11 & & a bo & bo \\
    118 &  8 & & dawały & nie dawały \\
    300 & 17 & & mogły & mogło \\
    362 & 12 & & torrusa & torusa \\
    367 & 15 & & c\textbf{oś} & \textbf{coś} \\
    % & & & & \\
    \hline
  \end{tabular}

\end{center}


\vspace{\spaceTwo}
% ############################










% ######################################
\section{Kultura japońska}

\vspace{\spaceTwo}
% ######################################



% ############################
\Work{ % Autor i tytuł dzieła
  Renata Iwicka \\
  „Źródła klasycznej demonologii japońskiej”,
  \cite{IwickaZrodlaKlasycznejDemonologiJaponskiej2017} }


% ##################
\CenterBoldFont{Uwagi}


\start \Str{26} Zdjęcie na tej stronie powinno być lepiej opisane.

\vspace{\spaceFour}



\start \Str{49} Użyto tu przydomka „Raiko” % Raik\={o}
by odnieść~się do~postaci Minamoto no~Yorumitsu jednak~to, że~ten
przydomek odnosi~się do~niego stanie~się jasne dopiero na~stronie~52.

\vspace{\spaceFour}



\start \StrWg{94}{14--15} Według tego co~znalazłem, shamisen
to~podobny do~gitary trójstrunowy instrument\footnote{Jednak nie mogę
  gwarantować, że~jest to poprane wyjaśnienie tej nazwy.}. Myślę,
że~czym on~jest powinni być tu~wyjaśnione, wszak nie każdy czytelnik
tej książki, jest takim znawcą kultury japońskiej, by~to wiedzieć.
A~jak znaleźć wiarygodne źródło w~takiej specjalistycznej sprawie, też
nie jest całkiem proste.

\vspace{\spaceFour}



\start \StrWd{95}{3} W~tej książce normalnie zapisuje~się nazwę tej
wyspy jako „Kyushu”. % Ky\={u}sh\={u}

\vspace{\spaceFour}



\start \StrWg{128}{7} To~zdanie jest sformułowane w~taki sposób,
że~nie rozumiem ani~jak położona jest oś~o~której mowa, ani jak
nazywają~się leżące na~niej bramy.

\vspace{\spaceFour}



\start \StrWg{129}{12--16} Fragment ten jest napisany w~taki sposób,
że~nie byłem w~stanie zrozumieć z~niego, która pisownia nazwy bramy
jest pierwotna, a~która powstała później. Z~reszty książki wynika,
że~pierwotna nazwa to „Rajomon”, % Raj\={o}mon
późniejsza zaś~to „Rashomon”. % Rash\={o}mon

\vspace{\spaceFour}



\start \StrWd{139}{15} Aby~zachować płynność przytoczonej
tu~opowieści, należałoby wcześniej wspomnieć, że~dwa bohaterowie
dotarli do jakiejś rezydencji.

\vspace{\spaceFour}



\start \textbf{Str. 148, przypis 441.} Jestem przekonany, że~ten przypis
zamiast mówić jak obecnie, iż~\emph{Rituale Romanum} było używane prze
niektórych egzorcystów od~1999, miał stwierdzać, że~był używany
do~tego roku.

\vspace{\spaceFour}



\start \StrWd{169}{10} Całkiem możliwe, że~zamiast „wykryto dużo
wcześniej” powinno pisać „wykryto bardzo wcześnie”.

\vspace{\spaceFour}



\start \StrWg{170}{} Nie jest dla mnie o~jaki z~wymienionych systemów
jest mowa w~tym zdaniu. Czy chodzi o~\emph{shugendo}, % shugend\={o}
czy też o~\emph{on'yodo}? % on'y\={o}d\={o}

\vspace{\spaceFour}



\start \Str{181} Podtytuł rozdziału jest zapisany tu jako „\emph{Tabu
  w}~shinto”, % shint\={o}
podczas gdy~w~spisie treści na~stronie~6 jako „Tabu
w~\emph{shinto}”. % shint\={o}
Warto byłoby ten zapis ujednolicić, choć może to~być sprzeczne
z~konwencją zapisu tytułów odpowiednich części tekstu.

\vspace{\spaceFour}



\start \Str{187} Drugi paragraf jest nie najlepiej skonstruowany
pod~względem językowym. Jego sens miał być zapewne taki. Gdyby
przyporządkowanie stron świata i~zwierząt różnym wojownikom bazowało
na~ich osiągnięciach i~roli odgrywanej w~historii, to podział ten
wyglądałby inaczej niż ten~który~się przyjął.

\vspace{\spaceFour}



\start \Str{215} Podpis pod tym rysunkiem powinien być obszerniejszy.

\vspace{\spaceFour}



\start \StrWd{222}{9} Może zamiast „użył znaku oznaczającego
kobietę” powinno być „stworzył znak oznaczający kobietę”?

\vspace{\spaceFour}



\start \StrWd{222}{2} Według opisu na angielskiej Wikipedii, główny
bohater mangi
\href{https://en.wikipedia.org/wiki/XxxHolic}{\emph{xxxHOLiC}} jest
nawiedzany przez istoty określane jako \emph{yokai}
% y\={o}kai
i~\emph{ayakashi}.

\vspace{\spaceFour}



\start \Str{223} W~książce tej zwykle zapisuje~się nazwę przywoływanej
tu~wyspy jako „Kyushu”, % Ky\={u}sh\={u}
nie „Kiusiu”.





% ##################
\CenterBoldFont{Błędy}


\begin{center}

  \begin{tabular}{|c|c|c|c|c|}
    \hline
    & \multicolumn{2}{c|}{} & & \\
    Strona & \multicolumn{2}{c|}{Wiersz} & Jest
                              & Powinno być \\ \cline{2-3}
    & Od góry & Od dołu & & \\
    \hline
    10  & &  2 & Kioto, 2012 & Kioto 2012 \\
    12  & & 12 & J.~Tubielewicz~J. & J.Tubielewicz \\
    12  & & 11 & J.~Tubielewicz~J. & J.Tubielewicz \\
    13  & &  6 & \emph{Yokai} Database: & \emph{Yokai Database}: \\
    % Nad ,,o'' w ,,Yokai'' powinna być kreska akcentu.
    17  & 17 & & Samą przestrzeń & O~samej przestrzeni \\
    63  & 13 & & „późniejszym światem & „późniejszym światem” \\
    64  & 18 & & rezydencja obok & rezydencja \\
    72  & &  4 & narodzin & narodzonej \\
    125 & &  6 & spostrzegania & postrzegania \\
    141 & 20 & & tych samych & tych \\
    154 & & 12 & )] & ] \\
    168 & & 15 & kamienia strzałek & kamieni strzałowych \\
    172 & 19 & & obecne & obecny \\
    179 & & 11 & schowku & schowku. \\
    186 & &  2 & pozbawione cudzysłowu & bez cudzysłowów \\
    197 & & 20 & schemat zachowań ludzkich & ludzki schemat zachowań \\
    198 &  7 & & & \emph{yokai$\,^{647}$} \\  % y\={o}kai
    214 & &  2 & księżycu & \emph{księżycu} \\
    230 & &  2 & bogów ludzi & bogów, ludzi \\
    231 & & 17 & pewną & pewnymi \\
    \hline
  \end{tabular}

\end{center}


\noindent
\StrWg{169}{19} \\
\Jest  a~zwłaszcza \\
\Powin których szczególnym uosobieniem w~tym kontekście jest \\
\StrWg{189}{6} \\
\Jest  zarysowaną postać \\
\Powin zarysowanie tej postaci \\
\StrWg{198}{10} \\
\Jest  obejmują dokładnie tę~samą kategorię \\
\Powin są objęte tą~samą kategorią \\


\vspace{\spaceTwo}
% ############################





























% ######################################
\section{Kino}

\vspace{\spaceThree}
% ######################################



% ############################
\Work{ % Redaktor i tytuł dzieła
  Red. Piotr Kletowski \\
  „Europejskie kino gatunków”,
  \cite{RedKletowskiEuropejskieKinoGatunkow16} }


% ##################
\CenterBoldFont{Uwagi}


\start \StrWd{129}{1} W~filmie templariuszom nie wyłupiono oczu, lecz
stracono i~powieszono na drzewach. Tam dzikie ptaki wyjadły ich oczy.





% ##################
\CenterBoldFont{Błędy}


\begin{center}

  \begin{tabular}{|c|c|c|c|c|}
    \hline
    & \multicolumn{2}{c|}{} & & \\
    Strona & \multicolumn{2}{c|}{Wiersz} & Jest
                              & Powinno być \\ \cline{2-3}
    & Od góry & Od dołu & & \\
    \hline
    87 & 11 & & wybraną & jedną wybraną \\
    % & & & & \\
    % & & & & \\
    % & & & & \\
    % & & & & \\
    \hline
  \end{tabular}

\end{center}


\noindent
\StrWg{160}{12} \\
\Jest  \emph{Za kilka dolarów więcej} Monco \\
\Powin filmu \emph{Dobry, zły i~brzydki} Blondie \\


\vspace{\spaceTwo}
% ############################










% ############################
\Work{ % Autor i tytuł dzieła
  Tadeusz Szczepański \\
  „Zwierciadło Bergmana”, \cite{SzczepanskiZwierciadloBergmana2007} }


% ##################
\CenterBoldFont{Uwagi}


\start \StrWg{131}{14} Postać pastora zagrał Gunnar Olsson.

\vspace{\spaceFour}



\start \StrWd{132}{17} Protestancki biskup Edvard Verg\'{e}rus
to~jeden z~bohaterów filmu Bergmana \emph{Fanny i~Aleksander}.

\vspace{\spaceFour}



\start \Str{146} Ponieważ Monika nie tylko udało~się uciec
z~mieszczańskiej willi, bez żadnych konsekwencji dla jej dalszego
życia, ale także zdobyć pieczeń po~którą tam poszła,
uważam~że~nazwanie jej działań „sromotnie nieudanymi”, jest
co~najmniej nietrafne.

\vspace{\spaceFour}



\start \Str{220} W~samym filmie \emph{Siódma pieczęć} nie znalazłem
niczego co~podtrzymywałoby przedstawioną tu interpretację, że~końcowa
scena z~rodziną kuglarzy rozgrywa~się w~czyjejś wyobraźni. Wręcz
przeciwnie, cała narracja filmu przekonuje mnie, że~wydarzyło~się
to~naprawdę.

\vspace{\spaceFour}



\start \Str{247} Na~tym zdjęciu bardzo ciężko dojrzeć, że~malowidło
na~ścianie przedstawie Śmierć z~szachami. Na~szczęście na~następnej
stronie jest wyjaśnione, iż~przedstawia ono Śmierć z~szachami
pod~ręką, która prowadzi koński zaprzęg.

\vspace{\spaceFour}





% ##################
\CenterBoldFont{Błędy}


\begin{center}

  \begin{tabular}{|c|c|c|c|c|}
    \hline
    & \multicolumn{2}{c|}{} & & \\
    Strona & \multicolumn{2}{c|}{Wiersz} & Jest
                              & Powinno być \\ \cline{2-3}
    & od góry & od dołu & & \\
    \hline
    31  & & 15 & zamiłowania ch & zamiłowaniach \\
    101 & 19 & & przypowieść & opowieść \\
    139 & 12 & & pozamał & pozamał\dywiz \\
    214 &  2 & & (\r{A}ke Fridell) Tubal & \r{A}ke Fridell (Tubal) \\
    % & & & & \\
    % & & & & \\
    457 & 18 & & Łódź . & Łódź. \\
    % & & & & \\
    \hline
  \end{tabular}

\end{center}


\vspace{\spaceTwo}
% ############################









% ######################################
\newpage
\section{Literatura}

\vspace{\spaceTwo}
% ######################################



% ############################
\Work{ % Autor i tytuł dzieła
  Teodor Parnicki \\
  „Szkice literackie”, \cite{ParnickiSzkiceLiterackie1979} }


% ##################
\CenterBoldFont{Uwagi do konkretnych stron}


\start \Str{88, 220} Na stronie 88 tytuł tej części to „O~humanizmie
katolicyzmu”, zaś na 220 to „O~humanizmie katolickim”. Sens tych dwóch
tytułów jest oczywiście diametralnie różny. Wydaje się, że~poprawny to
„O~humanizmie katolickim” bo takie słowa są użyte w~tytule szkicu
o~Zofii Kossak.

\vspace{\spaceFour}



\start \Str{69} W~zaczynającym się tu szkicu „Aleksander Puszkin
(w~stulecie śmierci)” wyróżniona jest część pierwsza i~trzecia, ale
nigdzie nie ma drugiej.

\vspace{\spaceFour}





% ##################
\CenterBoldFont{Błędy}


\begin{center}

  \begin{tabular}{|c|c|c|c|c|}
    \hline
    & \multicolumn{2}{c|}{} & & \\
    Strona & \multicolumn{2}{c|}{Wiersz} & Jest
                              & Powinno być \\ \cline{2-3}
    & od góry & od dołu & & \\
    \hline
    XXI   &  6 & & \emph{Juliana}\ldots$^{ 26 }$
           & \emph{Juliana}\ldots”$^{ 26 }$ \\
    XXIII & 15 & Lenowa. & Lenowa). \\
    91  &  9 & & upadku; & upadku. \\
    92  & 23 & & XX & XIX \\
    119 & & 7 & stałe & całe \\
    174 & 13 & & słynnych --- encyklikach & słynnych encyklikach \\
    174 & 15 & & XX. & XX.) \\
    % & & & & \\
    % & & & & \\
    % & & & & \\
    % & & & & \\
    % & & & & \\
    \hline
  \end{tabular}

\end{center}


\vspace{\spaceTwo}
% ############################










% ######################################
\newpage
\section{Malarstwo}

\vspace{\spaceTwo}
% ######################################



% ############################
\Work{ % Autor i tytuł dzieła
  Władysław Strzemiński \\
  „Teoria widzenia”, \cite{StrzeminskiTeoriaWidzenia2016} }


% ##################
\CenterBoldFont{Uwagi}


\start \StrWg{33}{13} Cudzysłów wystający na~lewy margines wygląd mało
estetycznie. W~przypadku dłuższych cytatów, których lewy margines jest
większy, nie~jest to problem.

\vspace{\spaceFour}



\start \StrWd{39}{19} Cudzysłów wystający na~lewy margines wygląd mało
estetycznie.

\vspace{\spaceFour}



\start \Str{146} Ponieważ Monika nie tylko udało~się uciec
z~mieszczańskiej willi, bez żadnych konsekwencji dla jej dalszego
życia, ale także zdobyć pieczeń po~którą tam poszła,
uważam~że~nazwanie jej działań ,,sromotnie nieudanymi'', jest
co~najmniej nietrafne.

\vspace{\spaceFour}



\start \Str{220} W~samym filmie \emph{Siódma pieczęć} nie znalazłem
niczego co~podtrzymywałoby przedstawioną tu interpretację,
że~końcowa scena z~rodziną kuglarzy rozgrywa~się w~czyjejś
wyobraźni. Wręcz przeciwnie, cała narracja filmu przekonuje mnie,
że~wydarzyło~się to~naprawdę.

\vspace{\spaceFour}



\start \Str{247} Na~tym zdjęciu bardzo ciężko dojrzeć, że~malowidło
na~ścianie przedstawie Śmierć z~szachami. Na~szczęście na~następnej
stronie jest wyjaśnione, iż~przedstawia ono Śmierć z~szachami
pod~ręką, która prowadzi koński zaprzęg.

\vspace{\spaceFour}





% ##################
\CenterBoldFont{Błędy}


\begin{center}

  \begin{tabular}{|c|c|c|c|c|}
    \hline
    & \multicolumn{2}{c|}{} & & \\
    Strona & \multicolumn{2}{c|}{Wiersz} & Jest
                              & Powinno być \\ \cline{2-3}
    & od góry & od dołu & & \\
    \hline
    15  & & 12 & 1709 & 1709) \\
    15  & &  3 & na & nad \\
    16  & & 16 & \emph{Teoria widzenia:} & \emph{Teoria widzenia}: \\
    44  &  3 & & rewolucyjny radykalizm & rewolucyjnego radykalizmu \\
    % & & & & \\
    % & & & & \\
    % & & & & \\
    % & & & & \\
    % & & & & \\
    \hline
  \end{tabular}

\end{center}

\vspace{\spaceTwo}
% ############################










% ######################################
\newpage
\section{Muzyka}

\vspace{\spaceTwo}
% ######################################



% ############################
\Work{ % Autor i tytuł dzieła
   Franciszek Wesołowski \\
  „Zasady muzyki”,
  \cite{WesolowskiZasadyMuzyki2017} }


% \CenterTB{Uwagi}

% \start \StrWd{}{}


% ##################
\CenterBoldFont{Błędy}


\begin{center}

  \begin{tabular}{|c|c|c|c|c|}
    \hline
    & \multicolumn{2}{c|}{} & & \\
    Strona & \multicolumn{2}{c|}{Wiersz} & Jest
                              & Powinno być \\ \cline{2-3}
    & Od góry & Od dołu & & \\
    \hline
    12  & &  4 & $440,\! 2$ & $440 \times 2$ \\
    12  & &  3 & $440,\! 3$ & $440 \times 3$ \\
    12  & &  3 & $440,\! 4$ & $440 \times 4$ \\
    % & & & & \\
    % & & & & \\
    % & & & & \\
    % & & & & \\
    \hline
  \end{tabular}

\end{center}


\noindent
\StrWg{160}{12} \\
\Jest  \emph{Za kilka dolarów więcej} Monco \\
\Powin filmu \emph{Dobry, zły i~brzydki} Blondie \\

\vspace{\spaceTwo}
% ############################










% ####################################################################
% ####################################################################
% Bibliografia
\bibliographystyle{plalpha}

\bibliography{}{}





% ############################

% Koniec dokumentu
\end{document}