% ---------------------------------------------------------------------
% Podstawowe ustawienia i pakiety
% ---------------------------------------------------------------------
\RequirePackage[l2tabu, orthodox]{nag}  % Wykrywa przestarzałe i niewłaściwe
% sposoby używania LaTeXa. Więcej jest w l2tabu English version.
\documentclass[a4paper,11pt]{article}
% {rozmiar papieru, rozmiar fontu}[klasa dokumentu]
\usepackage[MeX]{polski}  % Polonizacja LaTeXa, bez niej będzie pracował
% w języku angielskim.
\usepackage[utf8]{inputenc}  % Włączenie kodowania UTF-8, co daje dostęp
% do polskich znaków.
\usepackage{lmodern}  % Wprowadza fonty Latin Modern.
\usepackage[T1]{fontenc}  % Potrzebne do używania fontów Latin Modern.



% ---------------------------------------
% Podstawowe pakiety (niezwiązane z ustawieniami języka)
% ---------------------------------------
\usepackage{microtype}  % Twierdzi, że poprawi rozmiar odstępów w tekście.
% \usepackage{graphicx}  % Wprowadza bardzo potrzebne komendy do wstawiania
% % grafiki.
% \usepackage{verbatim}  % Poprawia otoczenie VERBATIME.
% \usepackage{textcomp}  % Dodaje takie symbole jak stopnie Celsiusa,
% % wprowadzane bezpośrednio w tekście.
\usepackage{vmargin}  % Pozwala na prostą kontrolę rozmiaru marginesów,
% za pomocą komend poniżej. Rozmiar odstępów jest mierzony w calach.
% ---------------------------------------
% MARGINS
% ---------------------------------------
\setmarginsrb
{ 0.7in}  % left margin
{ 0.6in}  % top margin
{ 0.7in}  % right margin
{ 0.8in}  % bottom margin
{  20pt}  % head height
{0.25in}  % head sep
{   9pt}  % foot height
{ 0.3in}  % foot sep



% ---------------------------------------
% Często używane pakiety
% ---------------------------------------
% \usepackage{csquotes}  % Pozwala w prosty sposób wstawiać cytaty do tekstu.
\usepackage{xcolor}  % Packages enabling use of many colors model and many
% additional colors



% ---------------------------------------
% Pakiety napisane przez użytkownika.
% Mają być w tym samym katalogu to ten plik .tex
% ---------------------------------------
\usepackage{latexgeneralcommands}



% ---------------------------------------------------------------------
% Dodatkowe ustawienia dla języka polskiego
% ---------------------------------------------------------------------
\renewcommand{\thesection}{\arabic{section}.}
% Kropki po numerach rozdziału (polski zwyczaj topograficzny)
\renewcommand{\thesubsection}{\thesection\arabic{subsection}}
% Brak kropki po numerach podrozdziału



% ---------------------------------------
% Ustawienia różnych parametrów tekstu
% ---------------------------------------
\renewcommand{\arraystretch}{1.2}  % Ustawienie szerokości odstępów między
% wierszami w tabelach.





% ---------------------------------------
% Pakiet „hyperref”
% Polecano by umieszczać go na końcu preambuły.
% ---------------------------------------
\usepackage{hyperref}  % Pozwala tworzyć hiperlinki i zamienia odwołania
% do bibliografii na hiperlinki.










% ---------------------------------------------------------------------
% Tytuł i autor tekstu
\title{Ekonomia \\
  Błędy i~uwagi}

\author{Kamil Ziemian}
% \date{}
% ---------------------------------------------------------------------










% ####################################################################
% Początek dokumentu
\begin{document}
% ####################################################################





% ######################################
\maketitle % Tytuł całego tekstu
% ######################################





% ######################################
\section{Bardzo wpływowe dzieła}

\vspace{\spaceTwo}
% ######################################



% ############################
\Work{ % Autor i tytuł dzieła
  Adam Smith \\
  \textit{Badania nad naturą i~przyczynami bogactw narodów. Tom~I},
  \cite{} }


% ##################
\CenterBoldFont{Błędy}


\begin{center}

  \begin{tabular}{|c|c|c|c|c|}
    \hline
    & \multicolumn{2}{c|}{} & & \\
    Strona & \multicolumn{2}{c|}{Wiersz} & Jest
                              & Powinno być \\ \cline{2-3}
    & Od góry & Od dołu & & \\
    \hline
    VII & & 1 & (1859) & (1759) \\
    % & & & & \\
    \hline
  \end{tabular}

\end{center}


\vspace{\spaceOne}
% ############################











% ######################################
\section{Austriacka szkoła ekonomi}

\vspace{\spaceTwo}
% ######################################



% ############################
\Work{ % Autor i tytuł dzieła
  J\"{o}rg Guido H\"{u}lsmann \\
  \textit{Etyka produkcji pieniądza},
  \cite{HulsmannEtykaProdukcjiPieniadza2014} }


% ##################
\CenterBoldFont{Uwagi do konkretnych stron}


\Str{II} Logo firmy NowePrzetargi.pl zostało wydrukowane w słabej jakości.

\vspace{\spaceFour}



\StrWd{XXII}{9--10} Według tego co jest tu napisane, dzieło Habigera na
temat okresu 1891--1991 ukazało~się w~roku 1990. To zapewne wynik błędu
drukarskiego.

\vspace{\spaceFour}



\StrWg{4}{9--10} Może wydawać~się dziwne, że w tym kontekście wymienione jest
ósme przykazanie\footnote{Używany jest tu katolicki podział Dekalogu.}. Może
to być błąd, możliwe też, że~H\"{u}lsmann miał tu na myśli związek jak ma to
przykazanie z~kwestią produkcji pieniądza, co omówił na stronie 23
niniejszej książki

\vspace{\spaceFour}



\StrWg{57}{6} W~tej linii pojawia~się anglicyzm „run na bank”. Jak jest
wyjaśnione dalej w~tym samym paragrafie pochodzi to od angielskiego „bank
run”. Wydaje mi~się, że~znacznie lepszym terminem byłby „nalot na bank” albo
„najazd na bank”.

\vspace{\spaceFour}



\textbf{Tylna okładka, wiersz 3 od dołu.} Odstępy w tym wierszu są
zbyt duże.








% ##################
\CenterBoldFont{Błędy}


\begin{center}

  \begin{tabular}{|c|c|c|c|c|}
    \hline
    & \multicolumn{2}{c|}{} & & \\
    Strona & \multicolumn{2}{c|}{Wiersz} & Jest
                              & Powinno być \\ \cline{2-3}
    & Od góry & Od dołu & & \\
    \hline
    XXIV  & & 20 & pieniędzy & pieniędzy” \\
    XXIV  & & 20 & s.~160)”. & s.~160). \\
    XXVII & & 10 & Juan de Mariana: & \textit{Juan de Mariana:} \\
    % & & & & \\
    \hline
  \end{tabular}





  % \begin{tabular}{|c|c|c|c|c|}
  %   \hline
  %   & \multicolumn{2}{c|}{} & & \\
  %   Strona & \multicolumn{2}{c|}{Wiersz} & Jest
  %   & Powinno być \\ \cline{2-3}
  %   & Od góry & Od dołu & & \\
  %   \hline
  %   %   & & & & \\
  %   %   & & & & \\
  %   %   & & & & \\
  %   %   & & & & \\
  %   %   & & & & \\
  %   %   & & & & \\
  %   %   & & & & \\
  %   %   & & & & \\
  %   %   & & & & \\
  %   %   & & & & \\
  %   %   & & & & \\
  %   %   & & & & \\
  %   %   & & & & \\
  %   %   & & & & \\
  %   %   & & & & \\
  %   %   & & & & \\
  %   %   & & & & \\
  %   %   & & & & \\
  %   %   & & & & \\
  %   %   & & & & \\
  %   %   & & & & \\
  %   %   & & & & \\
  %   %   & & & & \\
  %   %   & & & & \\
  %   %   & & & & \\
  %   %   & & & & \\
  %   %   & & & & \\
  %   %   & & & & \\
  %   %   & & & & \\
  %   %   & & & & \\
  %   %   & & & & \\
  %   %   & & & & \\
  %   %   & & & & \\
  %   %   & & & & \\
  %   %   & & & & \\
  %   %   & & & & \\
  %   %   & & & & \\
  %   %   & & & & \\
  %   \hline
  % \end{tabular}





  % \begin{tabular}{|c|c|c|c|c|}
  %   \hline
  %   & \multicolumn{2}{c|}{} & & \\
  %   Strona & \multicolumn{2}{c|}{Wiersz} & Jest
  %   & Powinno być \\ \cline{2-3}
  %   & Od góry & Od dołu & & \\
  %   \hline
  %   %   & & & & \\
  %   %   & & & & \\
  %   %   & & & & \\
  %   %   & & & & \\
  %   %   & & & & \\
  %   %   & & & & \\
  %   %   & & & & \\
  %   %   & & & & \\
  %   %   & & & & \\
  %   %   & & & & \\
  %   %   & & & & \\
  %   %   & & & & \\
  %   %   & & & & \\
  %   %   & & & & \\
  %   %   & & & & \\
  %   %   & & & & \\
  %   %   & & & & \\
  %   %   & & & & \\
  %   %   & & & & \\
  %   %   & & & & \\
  %   %   & & & & \\
  %   %   & & & & \\
  %   %   & & & & \\
  %   %   & & & & \\
  %   %   & & & & \\
  %   %   & & & & \\
  %   %   & & & & \\
  %   %   & & & & \\
  %   %   & & & & \\
  %   %   & & & & \\
  %   %   & & & & \\
  %   %   & & & & \\
  %   %   & & & & \\
  %   %   & & & & \\
  %   %   & & & & \\
  %   %   & & & & \\
  %   %   & & & & \\
  %   %   & & & & \\
  %   \hline
  % \end{tabular}

\end{center}


\noindent
\StrWd{13}{17} \\
\Jest  jako naturalni przywódcy \\
\Powin jako będący naturalnymi przywódcami \\


% \Jest  Amerian Seafarers Union \\
% \Powin \textit{Amerian Seafarers Union} \\



% \StrWd{11}{5} \\
% \Jest  Council~of Economic Advisers \\
% \Powin \textit{Council~of Economic Advisers} \\
% \StrWd{11}{1} \\
% \Jest  Council on~Foregin Relations \\
% \Powin \textit{Council on~Foregin Relations} \\
% \StrWg{12}{11} \\
% \Jest  \textit{Congress}) --~Kanadyjski Kongres Związków Zawodowych \\
% \Powin \textit{Congress} --~Kanadyjski Kongres Związków Zawodowych) \\
% \StrWd{12}{18} \\
% \Jest  Emergency Committe for~Aid to~Poland \\
% \Powin \textit{Emergency Committe for~Aid to~Poland} \\
% \StrWd{12}{8} \\
% \Jest  (\textit{Froce Ouvri\'{e}re} ) \\
% \Powin (\texit{Froce Ouvri\'{e}re} --~Główna Konfederacja Pracy
% --~Siły
% Pracy) \\
% \StrWg{13}{10} \\
% \Jest  Generalized System~of Preferences \\
% \Powin \textit{Generalized System~of Preferences} \\
% \StrWd{13}{4} \\
% \Jest  \textit{Leuven}) \\
% \Powin \textit{Leuven} --~Kotlicki Ośrodek Dokumentacyjny i~Badań
% Katolickich Uniwersytetu Leuven \\
% \StrWg{15}{1} \\
% \Jest  Polish American Congress Charitable Foundation \\
% \Powin \textit{Polish American Congress Charitable Foundation} \\
% \StrWg{15}{3} \\
% \Jest  Polish-American Enterprise Fund \\
% \Powin \textit{Polish-American Enterprise Fund} \\
% \StrWg{15}{10} \\
% \Jest  Postal, Telegraph and~Telephone International \\
% \Powin \textit{Postal, Telegraph and~Telephone International} \\
% \StrWg{43}{2} \\
% \Jest  \textit{Univeristy~of Illinois at~Urbana Champaign, 1998} \\
% \Powin Univeristy~of Illinois at~Urbana-Champaign, 1998 \\


\vspace{\spaceTwo}
% ############################










% ############################
\Work{ % Autor i tytuł dzieła
  Murray N. Rothbard \\
  \textit{Złoto, banki, ludzie. Krótka historia pieniądza},
  \cite{RothbardZlotoBankiLudzie2016} }


% ##################
\CenterBoldFont{Uwagi}





% ##################
\CenterBoldFont{Uwagi do konkretnych stron}


% \vspace{\spaceFour}





% ##################
\CenterBoldFont{Błędy}

\begin{center}

  \begin{tabular}{|c|c|c|c|c|}
    \hline
    & \multicolumn{2}{c|}{} & & \\
    Strona & \multicolumn{2}{c|}{Wiersz} & Jest
                              & Powinno być \\ \cline{2-3}
    & Od góry & Od dołu & & \\
    \hline
    157 & 11 & & poprawianym[...] & poprawianym [...] \\
    159 & & 16 & Busha & H.W. Busha \\
    159 & &  5 & www.cato.org & www.cato.org. \\
    % & & & & \\
    165 & &  6 & 19-29 & 19--29 \\
    165 & &  1 & (1723-1790) & \textit{(1723--1790)} \\
    166 & &  1 & 74-80 & 74--80 \\
    169 & &  3 & Bohm-Bawerka & B\"{o}hm-Bawerka \\
    170 & &  2 & 1929-1933 & 1929--1933 \\
    % & & & & \\
    \hline
  \end{tabular}





  % \begin{tabular}{|c|c|c|c|c|}
  %   \hline
  %   & \multicolumn{2}{c|}{} & & \\
  %   Strona & \multicolumn{2}{c|}{Wiersz} & Jest
  %   & Powinno być \\ \cline{2-3}
  %   & Od góry & Od dołu & & \\
  %   \hline
  %   %   & & & & \\
  %   %   & & & & \\
  %   %   & & & & \\
  %   %   & & & & \\
  %   %   & & & & \\
  %   %   & & & & \\
  %   %   & & & & \\
  %   %   & & & & \\
  %   %   & & & & \\
  %   %   & & & & \\
  %   %   & & & & \\
  %   %   & & & & \\
  %   %   & & & & \\
  %   %   & & & & \\
  %   %   & & & & \\
  %   %   & & & & \\
  %   %   & & & & \\
  %   %   & & & & \\
  %   %   & & & & \\
  %   %   & & & & \\
  %   %   & & & & \\
  %   %   & & & & \\
  %   %   & & & & \\
  %   %   & & & & \\
  %   %   & & & & \\
  %   %   & & & & \\
  %   %   & & & & \\
  %   %   & & & & \\
  %   %   & & & & \\
  %   %   & & & & \\
  %   %   & & & & \\
  %   %   & & & & \\
  %   %   & & & & \\
  %   %   & & & & \\
  %   %   & & & & \\
  %   %   & & & & \\
  %   %   & & & & \\
  %   %   & & & & \\
  %   \hline
  % \end{tabular}





  % \begin{tabular}{|c|c|c|c|c|}
  %   \hline
  %   & \multicolumn{2}{c|}{} & & \\
  %   Strona & \multicolumn{2}{c|}{Wiersz} & Jest
  %   & Powinno być \\ \cline{2-3}
  %   & Od góry & Od dołu & & \\
  %   \hline
  %   %   & & & & \\
  %   %   & & & & \\
  %   %   & & & & \\
  %   %   & & & & \\
  %   %   & & & & \\
  %   %   & & & & \\
  %   %   & & & & \\
  %   %   & & & & \\
  %   %   & & & & \\
  %   %   & & & & \\
  %   %   & & & & \\
  %   %   & & & & \\
  %   %   & & & & \\
  %   %   & & & & \\
  %   %   & & & & \\
  %   %   & & & & \\
  %   %   & & & & \\
  %   %   & & & & \\
  %   %   & & & & \\
  %   %   & & & & \\
  %   %   & & & & \\
  %   %   & & & & \\
  %   %   & & & & \\
  %   %   & & & & \\
  %   %   & & & & \\
  %   %   & & & & \\
  %   %   & & & & \\
  %   %   & & & & \\
  %   %   & & & & \\
  %   %   & & & & \\
  %   %   & & & & \\
  %   %   & & & & \\
  %   %   & & & & \\
  %   %   & & & & \\
  %   %   & & & & \\
  %   %   & & & & \\
  %   %   & & & & \\
  %   %   & & & & \\
  %   \hline
  % \end{tabular}

\end{center}


\noindent
\StrWd{156}{2} \\
\Jest  Prekursorzy Nowej Lewicy. Studia z~myśli społecznej XIX i~XX wieku \\
\Powin \textit{Prekursorzy Nowej Lewicy. Studia z~myśli społecznej XIX
  i~XX wieku}


% \Jest  Amerian Seafarers Union \\
% \Powin \textit{Amerian Seafarers Union} \\



% \StrWd{11}{5} \\
% \Jest  Council~of Economic Advisers \\
% \Powin \textit{Council~of Economic Advisers} \\
% \StrWd{11}{1} \\
% \Jest  Council on~Foregin Relations \\
% \Powin \textit{Council on~Foregin Relations} \\
% \StrWg{12}{11} \\
% \Jest  \textit{Congress}) --~Kanadyjski Kongres Związków Zawodowych \\
% \Powin \textit{Congress} --~Kanadyjski Kongres Związków Zawodowych) \\
% \StrWd{12}{18} \\
% \Jest  Emergency Committe for~Aid to~Poland \\
% \Powin \textit{Emergency Committe for~Aid to~Poland} \\
% \StrWd{12}{8} \\
% \Jest  (\textit{Froce Ouvri\'{e}re} ) \\
% \Powin (\textit{Froce Ouvri\'{e}re} --~Główna Konfederacja Pracy
% --~Siły Pracy) \\
% \StrWg{13}{10} \\
% \Jest Generalized System~of Preferences \\
% \Powin  \textit{Generalized System~of Preferences} \\
% \StrWd{13}{4} \\
% \Jest
% \Powin
% Katolickich Uniwersytetu Leuven \\
% \StrWg{15}{1} \\
% \Jest
% \Powin
% \StrWg{15}{3} \\
% \Jest
% \Powin
% \StrWg{15}{10} \\
% \Jest
% \Powin
% \StrWg{43}{2} \\
% \Jest
% \Powin

\vspace{\spaceTwo}
% ############################



















% ####################################################################
% ####################################################################
% Bibliografia
\bibliographystyle{plalpha}

\bibliography{VariousFieldsBooks}{}





% ############################

% Koniec dokumentu
\end{document}
