% Autor: Kamil Ziemian

% ---------------------------------------------------------------------
% Podstawowe ustawienia i pakiety
% ---------------------------------------------------------------------
\RequirePackage[l2tabu, orthodox]{nag}  % Wykrywa przestarzałe i niewłaściwe
% sposoby używania LaTeXa. Więcej jest w l2tabu English version.
\documentclass[a4paper,11pt]{article}
% {rozmiar papieru, rozmiar fontu}[klasa dokumentu]
\usepackage[MeX]{polski}  % Polonizacja LaTeXa, bez niej będzie pracował
% w języku angielskim.
\usepackage[utf8]{inputenc}  % Włączenie kodowania UTF-8, co daje dostęp
% do polskich znaków.
\usepackage{lmodern}  % Wprowadza fonty Latin Modern.
\usepackage[T1]{fontenc}  % Potrzebne do używania fontów Latin Modern.



% ---------------------------------------
% Podstawowe pakiety (niezwiązane z ustawieniami języka)
% ---------------------------------------
\usepackage{microtype}  % Twierdzi, że poprawi rozmiar odstępów w tekście.
% \usepackage{graphicx}  % Wprowadza bardzo potrzebne komendy do wstawiania
% % grafiki.
% \usepackage{verbatim}  % Poprawia otoczenie VERBATIME.
% \usepackage{textcomp}  % Dodaje takie symbole jak stopnie Celsiusa,
% % wprowadzane bezpośrednio w tekście.
\usepackage{vmargin}  % Pozwala na prostą kontrolę rozmiaru marginesów,
% za pomocą komend poniżej. Rozmiar odstępów jest mierzony w calach.
% ---------------------------------------
% MARGINS
% ---------------------------------------
\setmarginsrb
{ 0.7in} % left margin
{ 0.6in} % top margin
{ 0.7in} % right margin
{ 0.8in} % bottom margin
{  20pt} % head height
{0.25in} % head sep
{   9pt} % foot height
{ 0.3in} % foot sep



% ---------------------------------------
% Często używane pakiety
% ---------------------------------------
% \usepackage{csquotes}  % Pozwala w prosty sposób wstawiać cytaty do tekstu.
\usepackage{xcolor}  % Pozwala używać kolorowych czcionek (zapewne dużo
% więcej, ale ja nie potrafię nic o tym powiedzieć).



% ---------------------------------------
% Pakiety napisane przez użytkownika.
% Mają być w tym samym katalogu to ten plik .tex
% ---------------------------------------
\usepackage{latexshortcuts}



% ---------------------------------------------------------------------
% Dodatkowe ustawienia dla języka polskiego
% ---------------------------------------------------------------------
\renewcommand{\thesection}{\arabic{section}.}
% Kropki po numerach rozdziału (polski zwyczaj topograficzny)
\renewcommand{\thesubsection}{\thesection\arabic{subsection}}
% Brak kropki po numerach podrozdziału



% ---------------------------------------
% Ustawienia różnych parametrów tekstu
% ---------------------------------------
\renewcommand{\arraystretch}{1.2}  % Ustawienie szerokości odstępów między
% wierszami w tabelach.





% ---------------------------------------
% Pakiet „hyperref”
% Polecano by umieszczać go na końcu preambuły.
% ---------------------------------------
\usepackage{hyperref}  % Pozwala tworzyć hiperlinki i zamienia odwołania
% do bibliografii na hiperlinki.










% ---------------------------------------------------------------------
% Tytuł, autor, data
\title{Eseje, felietony, publicystyka~-- błędy i~uwagi}

% \author{}
% \date{}
% ---------------------------------------------------------------------










% ####################################################################
% Początek dokumentu
\begin{document}
% ####################################################################





% ######################################
\maketitle % Tytuł całego tekstu
% ######################################





% ######################################
\section{Austriacka szkoła ekonomi}

\vspace{\spaceTwo}
% ######################################



% ############################
\Work{ % Autor i tytuł dzieła
  Murray N. Rothbard \\
  „Złoto, banki, ludzie. \\
  Krótka historia pieniądza”
  \cite{RothbardZlotoBankiLudzie2016} }


% \CenterTB{Uwagi}

% \start \textbf{Str. 11, wiersze 6, 10.} Po tych wiersza w~tekście
% powinien znajdować~się odstęp.

% \vspace{\spaceFour}





% ##################
\CenterTB{Błędy}

\begin{center}

  \begin{tabular}{|c|c|c|c|c|}
    \hline
    & \multicolumn{2}{c|}{} & & \\
    Strona & \multicolumn{2}{c|}{Wiersz} & Jest
                              & Powinno być \\ \cline{2-3}
    & Od góry & Od dołu & & \\
    \hline
    % & & & & \\
    % & & & & \\
    % & & & & \\
    % & & & & \\
    \hline
  \end{tabular}


  % \begin{tabular}{|c|c|c|c|c|}
  %   \hline
  %   & \multicolumn{2}{c|}{} & & \\
  %   Strona & \multicolumn{2}{c|}{Wiersz} & Jest
  %   & Powinno być \\ \cline{2-3}
  %   & Od góry & Od dołu & & \\
  %   \hline
  %   %   & & & & \\
  %   %   & & & & \\
  %   %   & & & & \\
  %   %   & & & & \\
  %   %   & & & & \\
  %   %   & & & & \\
  %   %   & & & & \\
  %   %   & & & & \\
  %   %   & & & & \\
  %   %   & & & & \\
  %   %   & & & & \\
  %   %   & & & & \\
  %   %   & & & & \\
  %   %   & & & & \\
  %   %   & & & & \\
  %   %   & & & & \\
  %   %   & & & & \\
  %   %   & & & & \\
  %   %   & & & & \\
  %   %   & & & & \\
  %   %   & & & & \\
  %   %   & & & & \\
  %   %   & & & & \\
  %   %   & & & & \\
  %   %   & & & & \\
  %   %   & & & & \\
  %   %   & & & & \\
  %   %   & & & & \\
  %   %   & & & & \\
  %   %   & & & & \\
  %   %   & & & & \\
  %   %   & & & & \\
  %   %   & & & & \\
  %   %   & & & & \\
  %   %   & & & & \\
  %   %   & & & & \\
  %   %   & & & & \\
  %   %   & & & & \\
  %   \hline
  % \end{tabular}


  % \begin{tabular}{|c|c|c|c|c|}
  %   \hline
  %   & \multicolumn{2}{c|}{} & & \\
  %   Strona & \multicolumn{2}{c|}{Wiersz} & Jest
  %   & Powinno być \\ \cline{2-3}
  %   & Od góry & Od dołu & & \\
  %   \hline
  %   %   & & & & \\
  %   %   & & & & \\
  %   %   & & & & \\
  %   %   & & & & \\
  %   %   & & & & \\
  %   %   & & & & \\
  %   %   & & & & \\
  %   %   & & & & \\
  %   %   & & & & \\
  %   %   & & & & \\
  %   %   & & & & \\
  %   %   & & & & \\
  %   %   & & & & \\
  %   %   & & & & \\
  %   %   & & & & \\
  %   %   & & & & \\
  %   %   & & & & \\
  %   %   & & & & \\
  %   %   & & & & \\
  %   %   & & & & \\
  %   %   & & & & \\
  %   %   & & & & \\
  %   %   & & & & \\
  %   %   & & & & \\
  %   %   & & & & \\
  %   %   & & & & \\
  %   %   & & & & \\
  %   %   & & & & \\
  %   %   & & & & \\
  %   %   & & & & \\
  %   %   & & & & \\
  %   %   & & & & \\
  %   %   & & & & \\
  %   %   & & & & \\
  %   %   & & & & \\
  %   %   & & & & \\
  %   %   & & & & \\
  %   %   & & & & \\
  %   \hline
  % \end{tabular}

\end{center}


\noindent
\StrWd{156}{2} \\
\Jest  Prekursorzy Nowej Lewicy. Studia z~myśli społecznej XIX i~XX wieku \\
\Powin \emph{Prekursorzy Nowej Lewicy. Studia z~myśli społecznej XIX i~XX wieku}


% \Jest  Amerian Seafarers Union \\
% \Powin \emph{Amerian Seafarers Union} \\



% \StrWd{11}{5} \\
% \Jest  Council~of Economic Advisers \\
% \Powin \emph{Council~of Economic Advisers} \\
% \StrWd{11}{1} \\
% \Jest  Council on~Foregin Relations \\
% \Powin \emph{Council on~Foregin Relations} \\
% \StrWg{12}{11} \\
% \Jest \emph{Congress}) --~Kanadyjski Kongres Związków Zawodowych \\
% \Powin  \emph{Congress} --~Kanadyjski Kongres Związków Zawodowych) \\
% \StrWd{12}{18} \\
% \Jest Emergency Committe for~Aid to~Poland \\
% \Powin  \emph{Emergency Committe for~Aid to~Poland} \\
% \StrWd{12}{8} \\
% \Jest (\emph{Froce Ouvri\'{e}re} ) \\
% \Powin (\emph{Froce Ouvri\'{e}re} --~Główna Konfederacja Pracy
% --~Siły
% Pracy) \\
% \StrWg{13}{10} \\
% \Jest Generalized System~of Preferences \\
% \Powin  \emph{Generalized System~of Preferences} \\
% \StrWd{13}{4} \\
% \Jest \emph{Leuven}) \\
% \Powin \emph{Leuven} --~Kotlicki Ośrodek Dokumentacyjny i~Badań
% Katolickich Uniwersytetu Leuven \\
% \StrWg{15}{1} \\
% \Jest Polish American Congress Charitable Foundation \\
% \Powin  \emph{Polish American Congress Charitable Foundation} \\
% \StrWg{15}{3} \\
% \Jest Polish-American Enterprise Fund \\
% \Powin  \emph{Polish-American Enterprise Fund} \\
% \StrWg{15}{10} \\
% \Jest Postal, Telegraph and~Telephone International \\
% \Powin  \emph{Postal, Telegraph and~Telephone International} \\
% \StrWg{43}{2} \\
% \Jest \emph{Univeristy~of Illinois at~Urbana Champaign, 1998} \\
% \Powin  Univeristy~of Illinois at~Urbana-Champaign, 1998 \\

\vspace{\spaceTwo}
% ############################



















% ####################################################################
% ####################################################################
% Bibliografia
\bibliographystyle{plalpha}

\bibliography{VariousFieldsBooks}{}





% ############################

% Koniec dokumentu
\end{document}
