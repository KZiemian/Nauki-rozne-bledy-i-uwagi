% ---------------------------------------------------------------------
% Podstawowe ustawienia i pakiety
% ---------------------------------------------------------------------
\RequirePackage[l2tabu, orthodox]{nag} % Wykrywa przestarzałe i niewłaściwe
% sposoby używania LaTeXa. Więcej jest w l2tabu English version.
\documentclass[a4paper,11pt]{article}
% {rozmiar papieru, rozmiar fontu}[klasa dokumentu]
\usepackage[MeX]{polski} % Polonizacja LaTeXa, bez niej będzie pracował
% w języku angielskim.
\usepackage[utf8]{inputenc} % Włączenie kodowania UTF-8, co daje dostęp
% do polskich znaków.
\usepackage{lmodern} % Wprowadza fonty Latin Modern.
\usepackage[T1]{fontenc} % Potrzebne do używania fontów Latin Modern.



% ------------------------------
% Podstawowe pakiety (niezwiązane z ustawieniami języka)
% ------------------------------
\usepackage{microtype} % Twierdzi, że poprawi rozmiar odstępów w tekście.
% \usepackage{graphicx} % Wprowadza bardzo potrzebne komendy do wstawiania
% grafiki.
% \usepackage{verbatim} % Poprawia otoczenie VERBATIME.
\usepackage{textcomp} % Dodaje takie symbole jak stopnie Celsiusa,
% wprowadzane bezpośrednio w tekście.
\usepackage{vmargin} % Pozwala na prostą kontrolę rozmiaru marginesów,
% za pomocą komend poniżej. Rozmiar odstępów jest mierzony w calach.
% ------------------------------
% MARGINS
% ------------------------------
\setmarginsrb
{ 0.7in} % left margin
{ 0.6in} % top margin
{ 0.7in} % right margin
{ 0.8in} % bottom margin
{  20pt} % head height
{0.25in} % head sep
{   9pt} % foot height
{ 0.3in} % foot sep



% ------------------------------
% Często używane pakiety
% ------------------------------
% \usepackage{csquotes} % Pozwala w prosty sposób wstawiać cytaty do tekstu.
\usepackage{xcolor} % Pozwala używać kolorowych czcionek (zapewne dużo
% więcej, ale ja nie potrafię nic o tym powiedzieć).



% ------------------------------
% Pakiety do tekstów z nauk przyrodniczych
% ------------------------------
\let\lll\undefined % Amsmath gryzie się z językiem pakietami do języka
% polskiego, bo oba definiują komendę \lll. Aby rozwiązać ten problem
% oddefiniowuję tę komendę, ale może tym samym pozbywam się dużego Ł.
\usepackage[intlimits]{amsmath} % Podstawowe wsparcie od American
% Mathematical Society (w skrócie AMS)
\usepackage{amsfonts, amssymb, amscd, amsthm} % Dalsze wsparcie od AMS
% \usepackage{siunitx} % Dla prostszego pisania jednostek fizycznych
\usepackage{upgreek} % Ładniejsze greckie litery
% Przykładowa składnia: pi = \uppi
\usepackage{slashed} % Pozwala w prosty sposób pisać slash Feynmana.
\usepackage{calrsfs} % Zmienia czcionkę kaligraficzną w \mathcal
% na ładniejszą. Może w innych miejscach robi to samo, ale o tym nic
% nie wiem.





% ---------------------------------------------------------------------
% Dodatkowe ustawienia dla języka polskiego
% ---------------------------------------------------------------------
\renewcommand{\thesection}{\arabic{section}.}
% Kropki po numerach rozdziału (polski zwyczaj topograficzny)
\renewcommand{\thesubsection}{\thesection\arabic{subsection}}
% Brak kropki po numerach podrozdziału



% ------------------------------
% Pakiety napisane przez użytkownika.
% Mają być w tym samym katalogu to ten plik .tex
% ------------------------------
\usepackage{latexgeneralcommands}
% \usepackage{mathshortcuts}



% ------------------------------
% Ustawienia różnych parametrów tekstu
% ------------------------------
\renewcommand{\arraystretch}{1.2} % Ustawienie szerokości odstępów między
% wierszami w tabelach.



% ------------------------------
% Pakiet "hyperref"
% Polecano by umieszczać go na końcu preambuły.
% ------------------------------
\usepackage{hyperref} % Pozwala tworzyć hiperlinki i zamienia odwołania
% do bibliografii na hiperlinki.










% ---------------------------------------------------------------------
% Tytuł i autor tekstu
\title{Językoznawstwo \\
  Błędy i~uwagi}

\author{Kamil Ziemian}


% \date{}
% ---------------------------------------------------------------------










% ####################################################################
% Początek dokumentu
\begin{document}
% ####################################################################





% ######################################
\maketitle % Tytuł całego tekstu
% ######################################





% ##################
\Work{ % Autor i tytuł dzieła
  John Lyons \\
  \textit{Wstęp do językoznawstwa},
  \cite{LyonsWstepDoJezykoznawstwa1975}}


\start \Str{110} Wprowadzony na tej stronie zapis $p_{ x }( y )$, oznacza
prawdopodobieństwo warunkowe wystąpienia $x$, przy założeniu, że~zaszło~$y$.
Jest on bardzo niewygodny w~użyciu i~warto byłoby zastąpić go jakąś lepszą
notacją.

\vspace{\spaceFour}





\start \Str{124} Nie rozumiem czemu na tej stronie, słowa takie jak „ustną”,
„dźwięczną”, etc., raz zapisuje~się w~nawiasach, a~raz bez nawiasów. Nie
podano żadnej wyraźniej reguły, na podstawie której wybierano by jeden z~tych
sposobów zapisu.

\vspace{\spaceFour}





\start \Str{201} Na tej stronie pierwszy raz zauważyłem następujący fenomen.
Na stronie 200 zdanie angielskie „How are you?” tłumaczone jest na polskie
zdanie „Jak się masz?”, więc w~tekście oba te zdania zaczynają się od dużej
litery. Podczas gdy tutaj zdanie „How do you do?” jest tłumaczone na
„jak~się masz?”, czyli po angielsku jest pisane z~dużej, a~po polsku z~małej
litery. Widzimy tu jawną niekonsekwencję w~zapisie.

Nie potrafię stwierdzić, czy ta strona to pierwsze pierwszy miejsce gdzie ta
niekonsekwencja~się pojawia, a~znalezienie i~poprawienie wszystkich tego
typu przypadków, nie jest warte zachodu.

\vspace{\spaceFour}





\start \Str{212} Zajmiemy~się tutaj tym, jak należy rozumieć „stosunek
liczby morfemów do liczby wyrazów”. Zacznijmy od tego, że~fragment na temat
roli morfemów i~morfów w~teorii języka jest w~mojej ocenie dość niejasny
i~problematyczny. Jakby autor sam nie był pewien, jak należy do~tego
zagadnienia podejść. Podrozdział 5.3.10 na temat rozbieżności między teorią
i~praktyką językoznawstwa stanowi mocny argument na rzecz tego, iż~Lyons był
wewnętrznie skonfliktowany pisząc tą część książki. Stąd dalsze rozważania
problemu muszą bazować na moim odczytaniu wyłożonej tu teorii.

Według niego, przyjęta jest hierarchiczna struktura języka, wedle której
najniższą jednostką jest morfem, a~wyrazy są jednostkami wyższego rzędu
zbudowanymi z~morfemów. Stąd każdy wyraz musi zawierać co najmniej jeden
morfem. W~tym miejscu przyjmiemy założenie, że~każdy pojedynczy morfem jest
wyrazem, choć nie potrafię teraz ustalić, czy nie jest to przypadkiem
wniosek z~podanej tu teorii języka. Jeśli więc oznaczmy przez $m$ liczbę
morfemów, zaś przez $n$ liczbę wyrazów, to~musi zachodzi $n \geq m$, bowiem
połączenie dwóch lub więcej morfemów może dać nam nowy wyraz. Tym samym
\begin{equation}
  \label{eq:Lyons-Wstep-do-jezykoznawstwa-01}
  \frac{ n }{ m } \geq 1.
\end{equation}

Jeśli więc powyższy iloraz jest równy $1$, to każdy wyraz składa~się
z~dokładnie jednego morfemu. W~takiej sytuacji gdyby jeden morfem
pojawiałby~się w~dwóch wyrazach, to wyrazy te musiałyby być identyczne, więc
nie istnieje możliwość „powtórnego wykorzystania” danego morfemu.
Przypadek $\frac{ n }{ m } = 1$ opisuje więc przypadek języka
absolutnie izolującego w~następującym sensie.

Wedle tego co czytamy na stronie 208, dany wyraz $A$ możemy rozumieć jako
zbiór morfemów. Przez odmianę wyrazu $A$ rozumiem wyraz $B$ należący do tego
samego języka, który zawiera wszystkie morfemy wyrazu $A$ i~co najmniej
jeden nowy. By być bardziej precyzyjnymi oznaczymy zbiór morfemów z~których
zbudowany jest wyraz $A$ za~pomocą symbolu $S_{ A }$. W~terminach teorii
mnogości moglibyśmy więc zapisać fakt, że~$B$ jest odmianą wyrazu $A$ jako
\begin{equation}
  \label{eq:Lyons-Wstep-do-jezykoznawstwa-02}
  S_{ A } \subsetneq S_{ B }.
\end{equation}
Przy takim spojrzeniu na kwestię odmiany wyrazów, w~języku dla którego
zachodzi $\frac{ n }{ m } = 1$ żaden wyraz nie może zostać odmieniony.

Na zakończenie zauważmy, że~jeśli mamy dwa różne wyrazy $A$ i~$B$ należące
do wspólnego języka, to teoria mnogości pozwala nam zawsze utworzyć
\begin{equation}
  \label{eq:Lyons-Wstep-do-jezykoznawstwa-03}
  S_{ A } \cup S_{ B }.
\end{equation}
Nie możemy jednak zagwarantować, że~taki zbiór będzie reprezentował
jakikolwiek wyraz rozważanego języka, więc istnienie sumy dwóch dowolnych
zbiorów nie wyklucza istnienia języka absolutnie izolującego.

\vspace{\spaceFour}





\start \Str{241} Na tej stronie znajdujemy dość trudne do zrozumienia
zdanie: „Jeżeli reguły te zastosować kolejno w~taki sposób, żeby każda (poza
pierwszą) służyła do zastąpienia (czyli przepisania) na <<wyjściu>>
poprzedzającej ją reguły symbolu po lewej stronie za~pomocą symbolu
w~nawiasach na prawo od strzałki”. Spróbujemy tu wyjaśnić, jakie jest jej
prawdopodobny sens.

Ponieważ w~przytoczonym fragmencie z~rozważań została wykluczona reguła
oznaczona numerem (1), również tutaj wykluczymy ją na razie z~analizowanych
przypadków. Przyjmijmy, że dane są trzy symbole $A$, $B$ i~$C$, ich sens nie
będzie miał znaczenia dla wyniku, oraz reguła
\begin{equation}
  \label{eq:Lyons-Wstep-do-jezykoznawstwa-04}
  A \to B + C.
\end{equation}
Z~treści podanych na tej stronie wzorów wynika, że~reguła przepisywania dana
przez wzór \eqref{eq:Lyons-Wstep-do-jezykoznawstwa-04} oznacza, iż
w~dowolnej formule zawierające $A$, mamy prawo ten symbol zastąpić przez
$A( B + C )$. Przykładowo, formuła postaci
\begin{equation}
  \label{eq:Lyons-Wstep-do-jezykoznawstwa-05}
  A + B,
\end{equation}
może zostać przepisana jako
\begin{equation}
  \label{eq:Lyons-Wstep-do-jezykoznawstwa-06}
  A( B + C ) + B.
\end{equation}
Regułę przepisywania zapiszemy teraz słownie, aby można było ją łatwiej
porównać omawianym tu problematycznym fragmentem tekstu, odejdziemy przy tym
trochę od stylu książki Lyonsa.

Słowna postać procedury tej procedury jest następująca. Jeżeli w~formule
występuje symbol który w~którejś z~danych reguł występuje po lewej stronie
strzałki, to formuła ta może zostać przepisana w~ten sposób, że~ten symbol
zostanie zastąpiony przez ten sam symbol, do którego dołączono symbol
stojący po prawej stronie strzałki ujęty w nawiasy.

Możemy teraz powrócić do problemu reguły~(1). Ma ona postać
\begin{equation}
  \label{eq:Lyons-Wstep-do-jezykoznawstwa-07}
  \Sigma \to S,
\end{equation}
gdzie $S$ jest odpowiednim symbolem. Dlaczego ta reguła jest traktowana
w~sposób specjalny i~Lyons wyłączył ją z~poprzednich rozważań? Najbardziej
prawdopodobne wyjaśnienie jak wiedzę jest takie, że~symbol $\Sigma$ ma
w~przedstawianej tu teorii specjalny status teorii specjalny status.
Nie jest to symbol oznaczający jakąś część danego języka, lecz stan
początkowy procedury przepisywania, który służy tylko do tego by
wygenerować formułę $\Sigma( S )$, gdzie $S$ jest symbolem oznaczający jakąś
część rozważanego języka. I~ten specjalny status symbolu $\Sigma$ wymusza osobne
traktowanie reguły wedle której jest on przepisywany.

Inaczej mówiąc, aby teoria była ścisła, musimy zostać podany osobny zestaw
reguł dla przepisywania symboli oznaczających elementy danego języka,
a~osobny dla symboli które nie oznaczają elementów tego języka. Wydaje~się
rozsądne przyjęcie, że~Lyons założył, choć nie napisałem tego jawnie,
iż~jedynym symbolem systemu przepisywania, który nie oznacza części języka
jest symbol $\Sigma$, zaś jedyną regułą, która pozwala ten symbol przepisać jest
reguła \eqref{eq:Lyons-Wstep-do-jezykoznawstwa-07}.

\vspace{\spaceFour}





% ##################
\newpage

\CenterBoldFont{Błędy}


\begin{center}

  \begin{tabular}{|c|c|c|c|c|}
    \hline
    & \multicolumn{2}{c|}{} & & \\
    Strona & \multicolumn{2}{c|}{Wiersz} & Jest
                              & Powinno być \\ \cline{2-3}
    & Od góry & Od dołu & & \\
    \hline
    32  & 17 & & inne, & inne; \\
    32  & 20 & & odosobnionych); & odosobnionych), \\
    96  & 15 & & minimalna & maksymalna \\
    96  & 18 & & $p_{ 1 } . p_{ 2 } . p_{ 3 } \ldots p_{ m }$
           & $p_{ 1 } \cdot p_{ 2 } \cdot p_{ 3 } \cdot \ldots \cdot p_{ m }$ \\
    100 & &  4 & 1. Wartość & 1, wartość \\
    100 & &  5 & $\frac{ 1 }{ n }$ (należy & $\frac{ 1 }{ n }$. Należy \\
    127 &  3 & & nieciągłej & ciągłej \\
    163 & &  8 & do & od \\
    228 &  6 & & 7{ }{ }{ } 1 & 7{ }{ } 1 \\
    228 &  7 & & 9{ }{ } 10 & 9 10 \\
    % & & & & \\
    % & & & & \\
    % & & & & \\
    % & & & & \\
    \hline
  \end{tabular}

\end{center}

\vspace{\spaceOne}
% ############################










% ############################
\newpage

\Work{ % Autorzy i tytuł dzieła
  Marie-Anne Paveau, Georges-\'{E}lia Sarfati \\
  \textit{Wielkie teorie językoznawcze \\
    Od~językoznawstwa historyczno-porównawczego do~pragmatyki},
  \cite{PaveauSarfatiWielkieTeorieJezykoznawcze2009}}


% ##################
\CenterBoldFont{Błędy}


\begin{center}

  \begin{tabular}{|c|c|c|c|c|}
    \hline
    & \multicolumn{2}{c|}{} & & \\
    Strona & \multicolumn{2}{c|}{Wiersz} & Jest
                              & Powinno być \\ \cline{2-3}
    & Od góry & Od dołu & & \\
    \hline
    % & & & & \\
    24 & 19 & & 1871'] & 1871) \\
    % & & & & \\
    \hline
  \end{tabular}

\end{center}


\vspace{\spaceOne}
% ############################










% ####################################################################
% ####################################################################
% Bibliografia

\bibliographystyle{plalpha}

\bibliography{VariousFieldsBooks}{}





% ############################

% Koniec dokumentu
\end{document}
