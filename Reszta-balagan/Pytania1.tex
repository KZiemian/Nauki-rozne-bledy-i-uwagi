\documentclass[a4paper,11pt]{article}
\usepackage[polish]{babel}% Tłumaczy na polski teksty automatyczne LaTeXa i pomaga z typografią.
\usepackage[plmath,OT4,MeX]{polski}% Polska notacja we wzorach matematycznych. Ładne polskie
\usepackage[T1]{fontenc}% Pozwala pisać znaki diakrytyczne z języków innych niż polski.
\usepackage[utf8]{inputenc}% Pozwala pisać polskie znaki bezpośrednio.
%\usepackage{indentfirst}% Sprawia, że jest wcięcie w pierwszym akapicie.
\frenchspacing% Wyłącza duże odstępy na końcu zdania. Pakiet babel polski robi to samo, ale to jest %zabezpieczenie jakibym chciał przestać go używać.
\usepackage{fullpage}% Mniejszse marginesy.
\usepackage{amsfonts}% Czcionki matematyczne od American Mathematic Society.
\usepackage{amsmath}% Dalsze wsparcie od AMS. Więc tego, co najlepsze w LaTeX, czyli trybu
%matematycznego.
\usepackage{upgreek}%Lepsze greckie czcionki. Przyklad skladni: pi = \uppi
%\usepackage{txfonts}%Inne ulepszenie greckich liter. Przyklad skladni: pi = \piup
\usepackage{amscd}% Jeszcze wsparcie od AMS.
\usepackage{latexsym}% Więcej symboli.
\usepackage{textcomp}% Pakiet z dziwnymi symbolami.
\usepackage{slashed}% Pozwala pisać slash Feynmana.
\usepackage{xy}% Pozwala rysować grafy.
\usepackage{tensor}% Pozwala prosto używać notacji tensorowej. Albo nawet pięknej notacji
%tensorowej:).
\usepackage{vmargin}
%----------------------------------------------------------------------------------------
%	MARGINS
%----------------------------------------------------------------------------------------
\setmarginsrb           { 0.7in}  % left margin
                        { 0.6in}  % top margin
                        { 0.7in}  % right margin
                        { 0.8in}  % bottom margin
                        {  20pt}  % head height
                        {0.25in}  % head sep
                        {   9pt}  % foot height
                        { 0.3in}  % foot sep
\usepackage{graphicx}% Pozwala wstawiać grafikę.
%\usepackage{url}% Pozwala pisać ładnie znak ~.
\def\mathbi#1{\textbf{\em #1}}
\newcommand{\vt}{$(x_1,x_{2}, \ldots, x_n)$}
\newcommand{\vet}{(x_1,x_{2}, \ldots, x_n)}
\newcommand{\e}{\mathrm{e}}
\newcommand{\ii}{\mathrm{i}}
\newcommand{\de}{\mathrm{d}}
\newcommand{\T}{\mathcal{T}_{ l }}
\newcommand{\Tl}{\mathcal{T}_{ l \lambda }}
\newcommand{\dd}[3]{\frac{ \de^{ #1 } #2 }{ \de #3^{ #1 } }}
\newcommand{\pd}[3]{\frac{ \partial^{ #1 } #2 }{ \partial #3^{ #1 } }}
\title{Obliczenia i~pytania}
\author{}

\begin{document}
%Pytania podzieliłem na~te ogólne i~bardziej szczegółowe odnoszące się do paragrafu 8 pracy
\noindent
Funkcja:
\begin{displaymath}
\theta( x' - x ) j( x ) h( x' ) + \theta( x - x' ) j( x' ) h( x ),
\end{displaymath}
przez podstawienie $j( x ) = \tfrac{ 1 }{ 2 }( h( x ) + \overline{ h( x ) } )$ można sprowadzić do postaci:
\begin{displaymath}
\frac{ 1 }{ 2 } [ h( x ) h( x' ) + \theta( x' - x ) \overline{ h( x ) } h( x' ) + \theta( x - x' ) h( x ) h( x' ) ].
\end{displaymath}
Korzystając z~przybliżenia
\begin{displaymath}
h^{ ( 1 ) }_{ l }( k a + kx ) \approx h^{ ( 1 ) }_{ l }( k a ) e^{ i k x }, ka > \eta > 0.
\end{displaymath}
otrzymujemy, że:
\begin{equation}
\begin{split}
& \int \de x \de x' g( x ) g( x' ) [\theta( x' - x ) j( x ) h( x' ) + \theta( x - x' ) j( x' ) h( x )] = \\
& \pi h( k a )^{ 2 } \widetilde{ g }( k )^{ 2 } + | h( k a ) |^{ 2 } \left( \int \de y g( y ) \right) \int \de x e^{ ikx } \theta( x ) g( x ).
\end{split}
\end{equation}

Jeżeli to ma sens to po wstawieniu otrzymujemy:
\begin{equation}
\begin{split}
a^{ 4 } ( 4 \pi )^{ 2 } & \int\limits_{ \mathbb{R}_{ + }^{ 2 } \setminus \Omega } \de p \de k \; p^{ -\tau + 2 } \frac{ k }{ ( p + k )^{ 2 } } | \widetilde{ g }_{ l }( p ) |^{ 2 } | \widetilde{ g }_{ l }( k ) |^{ 2 } |^{ 2 } | j_{ l }( pa ) |^{ 2 } | j_{ l }( ka ) |^{ 2 } \\
&\times 1/\Big| 1 - 2 \sigma i a^{ 4 } \left( \pi k h_{ l }( k a )^{ 2 } \widetilde{ g }_{ l }( k )^{ 2 } + k | h_{ l }( k a ) |^{ 2 } \left( \int \de y\, g_{ l }( y ) \right) \int \de x\, e^{ ikx } \theta( x ) g_{ l }( x ) \right) \Big|^{ 2 }
\end{split} 
\end{equation}
%\noindent
%\begin{itemize}
%\item[--] Rozważania rozdziału 8.1 w~[GvL] wydają~się potrzebne tylko~by pokazać słabą zbieżność rezolwent części laplasjanu i~jej zaburzenia. To w~obecnym problemie chyba nie jest potrzebne, bo pokazaliśmy już dla całego laplasjanu i~jego zaburzenia, że~zachodzi zbierzność silna.
%\item[--] Do punktu 8.2 w~[GvL]. Aby otrzymać lokalną gęstość energii musimy rozważyć dystrybucję:
%\begin{equation}
%T_{ a }( \varphi, \psi ) = \frac{ 1 }{ 4 } ( \varphi, ( h_{ a } - h ) \psi ) + \frac{ 1 }{ 4 } ( \nabla \varphi, ( h_{ a }^{ -1 } - h^{ - 1 } ) \nabla \psi ).
%\end{equation}
%Analiza tego problemu znacznie się uprości jeśli zachodzi twierdzenie mówiące, że~wystarczy to sprawdzić dla funkcji postaci $f( r ) Y^{ m }_{ l }$. Jakkolwiek istnienie takiego twierdzenia wydaje się prawdopodobne, to nie wiem czy topologia zbieżności w przestrzeni funkcji Schwartza dopuszcza coś takiego. Jeśli tak jest to pierwszy człon można potraktować jak w~[GvL], do drugiego zaś najprościej jest użyć rachunku spektralnego. Wtedy badanie tego wyrażenia powinno się zredukować do zbadania (na razie pomijam kwestie gradientów):
%\begin{equation}
%-\int\limits_{ \mathbb{R}^{ 2 } } \de k \de p \, \frac{ p k }{ p + k } \varphi( p ) \psi( k ) \hat{ g }_{ l }( p ) \overline{ \hat{ g }_{ l }( k ) } \T( k^{ 2 } + i 0 ).
%\end{equation}
%\item[--] W~drugiej formule po wzorze (77) wychodzi mi cały czas przeciwny znak.
%\end{itemize}
%\noindent
%\textbf{Pytania szczególne:}
%\begin{itemize}
%\item[--] Na samym początku jest mowa, że~będą rozpatrywane tylko algebry przypisane do obszaru którego nośnik nie przecina się z~położeniem płytek. Czy dobrze rozumuje, że~służy to zbadaniu tego co w~pracy [CEI] było nazwane ,,local quasi-equvalence''?
%\item[--] Czy $T_{ a }( x, y )$ należy rozumieć jako biliniową dystrybucję daną wzorem:
%\begin{equation}
%T_{ a }( \psi , \varphi ) = \int \de x \de y \; T_{ a }( x, y ) \psi( x ) \varphi( y )?
%\end{equation}
%\end{itemize}


\bibliographystyle{ieeetr}
\bibliography{BibliographyPL}{}



\end{document}