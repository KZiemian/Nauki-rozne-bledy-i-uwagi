\documentclass[a4paper,11pt]{article}
\usepackage[polish]{babel}% Tłumaczy na polski teksty automatyczne LaTeXa i pomaga z typografią.
\usepackage[plmath,OT4,MeX]{polski}% Polska notacja we wzorach matematycznych. Ładne polskie
\usepackage[T1]{fontenc}% Pozwala pisać znaki diakrytyczne z języków innych niż polski.
\usepackage[utf8]{inputenc}% Pozwala pisać polskie znaki bezpośrednio.
\usepackage{indentfirst}% Sprawia, że jest wcięcie w pierwszym akapicie.
\frenchspacing% Wyłącza duże odstępy na końcu zdania. Pakiet babel polski robi to samo, ale to jest %zabezpieczenie jakibym chciał przestać go używać.
\usepackage{amsfonts}% Czcionki matematyczne od American Mathematic Society.
\usepackage{amsmath}% Dalsze wsparcie od AMS. Więc tego, co najlepsze w LaTeX, czyli trybu matematycznego.
\let\lll\undefined
\usepackage{amssymb}
\usepackage{amscd}% Jeszcze wsparcie od AMS.
\usepackage{latexsym}% Więcej symboli.
\usepackage{textcomp}% Pakiet z dziwnymi symbolami.
%\usepackage{slashed}% Pozwala pisać slash Feynmana.
\usepackage{xy}% Pozwala rysować grafy.
\usepackage{tensor}% Pozwala prosto używać notacji tensorowej. Albo nawet pięknej notacji
%tensorowej:).
\usepackage{graphicx}% Pozwala wstawiać grafikę.
\usepackage{vmargin}
%----------------------------------------------------------------------------------------
%	MARGINS
%----------------------------------------------------------------------------------------
\setmarginsrb           { 0.7in}  % left margin
                        { 0.6in}  % top margin
                        { 0.7in}  % right margin
                        { 0.8in}  % bottom margin
                        {  20pt}  % head height
                        {0.25in}  % head sep
                        {   9pt}  % foot height
                        { 0.3in}  % foot sep
%\usepackage{url}% Pozwala pisać ładnie znak ~.
\newcommand{\vt}{$(x_1,x_{2}, \ldots, x_n)$}
\newcommand{\vet}{(x_1,x_{2}, \ldots, x_n)}
\newcommand{\e}{\mathrm{e}}
\newcommand{\ii}{\mathrm{i}}
\newcommand{\id}{\mathrm{id}}
\newcommand{\de}{\mathrm{d}}
\newcommand{\sgn}{\mathrm{sgn}}
\newcommand{\supp}{\mathrm{supp}}
\newcommand{\dd}[3]{\frac{ d^{ #1 } #2 }{ d #3^{ #1 } }}
\newcommand{\pd}[3]{\frac{ \partial^{ #1 } #2 }{ \partial #3^{ #1 } }}
\newcommand{\Prz}{\textbf{Przemyśl.}}
\newcommand{\Dok}{\textbf{Dokończ.}}


\begin{document}


%% \begin{center}
%%   \begin{Large}
%%     \textbf{Równania różniczkowe zwyczajne, błędy i~uwagi.}
%%   \end{Large}
%% \end{center}


\begin{center}
  W. Feller \\
  ,,Wstęp do rachunku prawdopodobieństwa. Tom I.'', \cite{Fel06}.
\end{center}

Błędy:\\
\begin{tabular}{|c|c|c|c|c|}
  \hline
  & \multicolumn{2}{c|}{} & & \\
  Strona & \multicolumn{2}{c|}{Wiersz} & Jest
                            & Powinno być \\ \cline{2-3}
  & Od góry & Od dołu & & \\
  \hline
  % & & & & \\
  28 & & 9 & wielu sposobami & na wiele sposobów \\
  % & & & & \\
  % & & & & \\
  % & & & & \\
  \hline
\end{tabular}

%% \noindent\\
%% \textbf{Str. 15, wiersz 13.}\\
%% \textbf{Jest:} w~sensie ustępu\ldots \\
%% \textbf{Powinno być:} w~sensie zdefiniowanym w~ustępie\ldots \\



% ####################################################################
% ####################################################################
% Bibliografia
\bibliographystyle{alpha} \bibliography{Bibliography}{}



% ############################

% Koniec dokumentu
\end{document}
