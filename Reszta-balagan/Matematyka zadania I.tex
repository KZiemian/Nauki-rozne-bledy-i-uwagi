\documentclass[a4paper,11pt]{article}
\usepackage[polish]{babel}%Tłumaczy na polski teksty automatyczne LaTeXa i pomaga z typografią.
\usepackage{amsfonts}%Czcionki matematyczne od American Mathematic Society.
\usepackage{fullpage}%Mniejszse marginesy.
\usepackage{amsmath}%Dalsze wsparcie od AMS. Więc tego, co najlepsze w LaTeX, czyli trybu matematycznego.
\usepackage[plmath,OT4,MeX]{polski}%Polska notacja we wzorach matematycznych. Ładne polskie czcionki i więcej cudzysłowów. Pełna polonizacja.
\usepackage[utf8]{inputenc}%Pozwala pisać polskie znaki bezpośrednio.
\usepackage{latexsym}%Więcej symboli.
\usepackage{indentfirst}%Sprawia, że jest wcięcie w pierwszym akapicie.
\frenchspacing%Wyłącza duże odstępy na końcu zdania. Pakiet babel polski robi to samo, ale to jest zabezpieczenie jakibym chciał przestać go używać.
\usepackage{graphicx}% Pozwala wstawiać grafikę.
%\usepackage{url}%Pozwala pisać ładnie znak ~.
\usepackage{textcomp}%Pakiet z dziwnymi symbolami.
\newcommand{\cm}{\checkmark}
\begin{document}
\begin{center}
MATEMATYKA. Zadania.
\end{center}



\begin{center}
Andrzej Białynicki-Birula\\
,,Zarys algebry'',\\wydanie \romannumeral1, ABB.
\end{center}
\begin{itemize}
\item[--] \romannumeral1. 
3: 1, 2, 3, 4, 5.\\
4: 1\cm ,2\cm . \\
5: 1, 2, 3, 4. 6: 1, 2, 3, 4, 5, 6, 7, 8, 9, 10, 11, 12, 13, 14, 15.\\
7: 1, 2, 3, 4.\\
8: 1, 2, 3, 4, 5.\\
9: 1, 2, 3, 4, 5, 6, 7, 8.\\
10: 1, 2, 3, 4, 5, 6, 7.\\
11: 1, 2, 3, 4, 5. 
12: 1, 2, 3, 4, 5, 6, 7, 8, 9.\\ 
13: 1, 2.\\
14: 1, 2, 3, 4, 5, 6, 7, 8.\\
15: 1, 2, 3.\\
16: 1, 2, 3, 4. 17: 1, 2, 3, 4, 5.
%\item[--] \romannumeral2. 1: 1\cm, 2, 3, 4, 5, 6, 7, 8, 9. \\
%2: 1, 2, 3, 4, 5. 3: 1, 2, 3, 4, 5, 6, 7, 8, 9, 10. \\
%3: 1, 2, 3, 4, 5. 4: 1, 2, 3, 4, 5, 6, 7, 8. \\
%5: 1, 2, 3, 4. \\
%6: 1, 2, 3, 4, 5, 6, 7, 8. \\
%7: 1, 2, 3, 4. \\
%8: 1, 2, 3, 4, 5, 6, 7, 8. \\
%9: 1, 2, 3, 4, 5, 6, 7, 8, 9. \\
%10: 1, 2, 3, 4. \\
%11: 1, 2, 3, 4. \\
%12: 1, 2, 3, 4, 5. \\
%13: 1, 2, 3, 4, 5, 6, 7, 8, 9, 10, 11.
%\item[--] \romannumeral3. 1: 1, 2, 3.\\
%2: 1, 2, 3, 4, 5.\\
%3: 1, 2, 3, 4, 5, 6, 7, 8.\\
%4: 1, 2, 3, 4, 5, 6.\\
%5: 1, 2, 3, 4.\\
%6: 1, 2, 3, 4, 5.\\
%7: 1, 2, 3, 4, 5.\\
%8: 1, 2, 3, 4.\\
%9: 1, 2, 3.
%\item[--] \romannumeral4. 1: 1, 2, 3, 4, 5.\\
%2: 1, 2.\\
%4: 1, 2, 3, 4, 5, 6.\\
%5: 1.\\
%6: 1, 2, 3, 4, 5.
%\item[--] \romannumeral5. 1: 1, 2, 3.\\
%2: 1, 2, 3, 4, 5, 6, 7.\\
%3: 1, 2. 3: 1, 2, 3, 4, 5, 6.\\
%4: 1, 2.\\
%5: 1, 2, 3, 4, 5, 6.\\
%6: 1, 2, 3, 4, 5, 6, 7, 8.\\
%7: 1, 2, 3, 4, 5, 6.\\
%8: 1, 2, 3.
%\item[--] \romannumeral6. 1: 1, 2, 3, 4.\\
%2: 1, 2.\\
%3: 1, 2.\\
%3: 1, 2.\\
%4: 1, 2, 3, 4, 5, 6, 7, 8.\\ 
%5: 1.\\
%6: 1, 2, 3, 4, 5.\\
%9: 1, 2, 3, 4, 5, 6, 7.\\
%10: 1, 2, 3, 4, 5, 6, 7.\\
%\item[--] \romannumeral7. 1: 1, 2, 3, 4, 5, 6, 7, 8, 9, 10, 11, 12.\\
%2: 1.\\
%3: 1, 2, 3, 4.\\
%4: 1,2.
%\item[--] \romannumeral8. 1: 1, 2, 3, 4, 5.\\
%2: 1, 2, 3, 4, 5, 6, 7, 8.\\
%3: 1.\\
%4: 1, 2, 3, 4, 5, 6, 7, 8, 9, 10, 11, 12.\\
%5: 1, 2, 3.\\
%6: 1,2.\\
%7: 1, 2, 3, 4, 5, 6, 7, 8.
\end{itemize}



\begin{center}
Maciej Bryński, Jerzy Jurkiewicz\\
,,Zarys zadań z algebry'',\\ wydanie \romannumeral2, BJ.
\end{center}
\begin{itemize}
\item[--] \romannumeral1. 1: 1\cm ,2\cm , 3, 4\cm, 5\cm, 6\cm, 7, 8\cm, 9\cm, 10\cm, 11, 12\cm, 13\cm, 14, 15, 16, 17, 18\cm, 19\cm, 20, 21, 22, 23, 24, 25, 26, 27, 28, 29, 30, 31, 32, 33, 34, 35, 36, 37, 38.
%\\2: 1, 2, 3, 4, 5, 6, 7, 8, 9, 10, 11, 12, 13, 14, 15, 16, 17, 18, 19, 20, 21, 22, 23, 24. \\3:1, 2, 3, 4, 5, 6, 7, 8, 9, 10, 11, 12, 13, 14, 15, 16, 17, 18, 19, 20, 21, 22, 23.
%\\4: 1, 2, 3\cm, 4, 5, 6, 7, 8, 9, 10, 11, 12, 13, 14, 15, 16, 17, 18, 19, 20, 21, 22.\\
%5: 1, 2, 3, 4, 5, 6, 7, 8, 9, 10, 11, 12, 13, 14, 15, 16, 17, 18, 19, 20, 21, 22, 23, 24.\\
%3:1, 2, 3, 4, 5, 6, 7, 8, 9, 10, 11, 12, 13, 14, 15, 16, 17, 18, 19, 20, 21, 22, 23, 24, 25, 26, 27.\\
%6: 1, 2, 3, 4, 5, 6, 7, 8, 9, 10, 11, 12, 13, 14, 15, 16, 17, 18, 19, 20, 21, 22, 23, 24, 25, 26, 27, 28, 29.\\
%7: 1, 2, 3, 4, 5, 6, 7, 8, 9, 10, 11, 12, 13, 14, 15, 16, 17, 18, 19, 20, 21, 22, 23, 24, 25, 26, 27, 28, 29, 30, 31, 32, 33, 34.
%\item[--] \romannumeral2. 1: 1, 2, 3, 4, 5, 6, 7, 8, 9, 10, 11, 12, 13, 14, 15, 16, 17, 18, 19, 20, 21, 22, 23, 24, 25, 26, 27, 28, 29.\\
%2: 1, 2, 3, 4, 5, 6, 7, 8, 9, 10, 11, 12, 13, 14, 15, 16, 17, 18, 19, 20, 21, 22.\\
%3: 1, 2, 3, 4, 5, 6, 7, 8, 9, 10, 11, 12, 13, 14, 15, 16, 17, 18, 19, 20, 21, 22, 23, 24, 25, 26, 27, 28, 29, 30, 31, 32, 33, 34, 35, 36, 37, 38, 39, 40 ,41, 42.\\
%4: 1, 2, 3, 4, 5, 6, 7, 8, 9, 10, 11, 12, 13, 14, 15, 16, 17, 18, 19, 20, 21, 22, 23, 24, 25.\\
%5: 1, 2, 3, 4, 5, 6, 7, 8, 9, 10, 11, 12, 13, 14, 15, 16, 17, 18, 19, 20, 21, 22, 23, 24, 25, 26, 27, 28, 29, 30.\\
%6: 1, 2, 3, 4, 5, 6, 7, 8, 9, 10, 11, 12, 13, 14, 15, 16, 17, 18, 19, 20.\\
%7: 1, 2, 3, 4, 5, 6, 7, 8, 9, 10.\\
%8: 1, 2, 3, 4, 5, 6, 7, 8, 9, 10, 11, 12, 13, 14, 15, 16, 17, 18, 19, 20, 21, 22, 23, 24, 25.
%9: 1, 2, 3, 4, 5, 6, 7, 8, 9, 10, 11, 12, 13, 14, 15, 16, 17, 18, 19, 20, 21, 22, 23, 24, 25, 26, 27, 28, 29, 30.
%\item[--] \romannumeral3. 1: 1, 2, 3, 4, 5, 6, 7, 8, 9, 10, 11, 12, 13, 14, 15.\\
%2: 1, 2, 3, 4, 5, 6, 7, 8, 9, 10, 11, 12, 13, 14, 15, 16, 17, 18, 19, 20, 21, 22, 23, 24, 25, 26, 27, 28, 29, 30, 31, 32, 33, 34, 35, 36, 37, 38, 39, 40 ,41, 42, 43, 44, 45, 46.\\
%3: 1, 2, 3, 4, 5, 6, 7, 8, 9, 10, 11, 12, 13, 14, 15, 16, 17, 18, 19, 20, 21, 22, 23.\\
%4: 1, 2, 3, 4, 5, 6, 7, 8, 9, 10, 11, 12, 13, 14.\\
%5: 1, 2, 3, 4, 5, 6, 7, 8, 9, 10, 11, 12, 13, 14, 15, 16, 17, 18, 19, 20, 21, 22, 23, 24, 25, 26, 27, 28, 29, 30, 31, 32, 33, 34, 35, 36, 37, 38, 39, 40 ,41, 42, 43.\\
%6: 1, 2, 3, 4, 5, 6, 7, 8, 9, 10, 11, 12, 13, 14, 15, 16, 17, 18, 19, 20, 21, 22, 23, 24, 25, 26, 27, 28, 29, 30, 31, 32, 33, 34.
%\item[--] \romannumeral4. 1: 1, 2, 3, 4, 5, 6, 7, 8, 9, 10, 11, 12, 13, 14, 15, 16, 17, 18, 19, 20, 21, 22, 23, 24, 25, 26.\\
%2: 1, 2, 3, 4, 5, 6, 7, 8, 9, 10, 11, 12, 13, 14, 15, 16, 17, 18, 19, 20, 21, 22, 23, 24, 25, 26.\\
%3: 1, 2, 3, 4, 5, 6, 7, 8, 9, 10, 11, 12, 13, 14, 15, 16, 17, 18.\\
%4: 1, 2, 3, 4, 5, 6, 7, 8, 9, 10, 11, 12, 13.
%\item[--] \romannumeral5. 1: 1, 2, 3, 4, 5, 6, 7, 8, 9, 10, 11, 12, 13, 14, 15, 16, 17, 18, 19, 20, 21, 22, 23, 24, 25, 26, 27, 28, 29, 30, 31, 32, 33, 34, 35, 36.\\
%2: 1, 2, 3, 4, 5, 6, 7, 8, 9, 10, 11, 12, 13, 14, 15, 16, 17, 18, 19, 20, 21, 22, 23, 24, 25, 26.
\end{itemize}



\begin{center}
Jacek Gancarzewicz\\
,,Algebra liniowa i jej zastosowania'', \cite{JGALJZ}.
\end{center}
\begin{itemize}
\item[--] \romannumeral1 . 1, 2, 3cm, 4cm, 5cm, 6cm, 7cm, 8cm, 9cm, 10cm, 11cm, 12cm, 13cm, 14cm, 15cm, 16, 17cm, 18, 19cm, 20cm, 21cm, 22, 23, 24cm.
%\item[--] \romannumeral2. 1\cm, 2\cm, 3\cm, 4\cm, 5\cm, 6\cm, 7\cm, 8\cm, 9\cm, 10\cm, 11\cm, 12\cm, 13\cm, 14\cm, 15, 16, 17, 18, 19\cm, 20\cm, 21\cm, 22\cm, 23\cm, 24\cm, 25\cm, 26\cm.
%\item[--] \romannumeral3. 1\cm, 2\cm, 3\cm, 4\cm, 5\cm, 6, 7\cm, 8, 9, 10\cm, 11, 12, 13, 14, 15, 16, 17\cm, 18\cm, 19\cm, 20, 21\cm, 22, 23, 24, 25, 26, 27, 28, 29, 30, 31, 32, 33, 34, 35, 36, 37, 38.
%\item[--] \romannumeral4. 1\cm, 2\cm, 3\cm, 4\cm, 5, 6, 7, 8, 9, 10, 11, 12, 13, 14, 15, 16, 17, 18, 19, 20, 21, 22.
%\item[--] \romannumeral5. 1\cm, 2\cm, 3, 4, 5, 6, 7, 8.
%\item[--] \romannumeral6. 1\cm, 2\cm, 3\cm, 4\cm, 5\cm, 6\cm, 7, 8\cm, 9, 10, 11, 12, 13, 14, 15, 16, 17, 18, 19, 20, 21, 22, 23, 24.
%\item[--] \romannumeral7. 1, 2, 3, 4, 5, 6, 7, 8, 9, 10, 11, 12, 13, 14, 15, 16, 17, 18, 19, 20, 21, 22, 23, 24, 25, 26, 27, 28, 29, 30, 31.
%\item[--] \romannumeral8. 1, 2, 3, 4, 5, 6, 7, 8, 9, 10, 11, 12, 13, 14, 15, 16, 17, 18, 19, 20, 21, 22.
%\item[--] \romannumeral9. 1, 2, 3, 4, 5, 6, 7, 8, 9.
\end{itemize}



\begin{center}
Henryk Arodź, Krzysztof Rościszewski\\
,,Algebra i geometria analityczna w zadaniach'', \cite{HAKRAGAZ}.
\end{center}
\begin{itemize}
\item[--] \romannumeral1. 1, 2, 3, 4, 5, 6, 7, 8, 9, 10, 11, 12, 13, 14, 15, 16, 17, 18, 19, 20, 21, 22, 23, 24, 25, 26, 27, 28, 29, 30, 31, 32, 33, 34, 35, 36, 37, 38, 39, 40, 41, 42, 43, 44.
\item[--] \romannumeral2. 1\cm, 2, 3\cm, 4, 5, 6\cm, 7\cm, 8, 9, 10, 11, 12, 13, 14, 15, 16, 17, 18, 19, 20, 21, 22, 23, 24, 25, 26, 27, 28, 29 , 30, 31, 32, 33, 34, 35, 36, 37, 38, 39, 40, 41, 42, 43, 44, 45, 46, 47, 48, 49.
%\item[--] \romannumeral3. 1, 2, 3, 4, 5, 6, 7, 8, 9, 10, 11, 12, 13, 14, 15, 16, 17, 18, 19, 20, 21.
%\item[--] \romannumeral4. 1, 2, 3, 4, 5, 6, 7, 8, 9, 10, 11\cm, 12\cm, 13\cm, 14, 15, 16, 17, 18, 19, 20, 21, 22, 23, 24, 25, 26, 27, 28, 29 , 30, 31, 32, 33, 34, 35, 36, 37, 38, 39, 40, 41, 42, 43, 44, 45, 46.
%\item[--] \romannumeral5. 1, 2, 3, 4, 5, 6, 7, 8, 9, 10, 11, 12, 13, 14, 15, 16, 17, 18, 19, 20, 21, 22, 23, 24, 25, 26, 27, 28, 29 , 30, 31, 32, 33.
%\item[--] \romannumeral6. 1, 2, 3, 4, 5, 6, 7, 8, 9, 10, 11, 12, 13, 14, 15, 16, 17, 18, 19, 20, 21, 22, 23, 24, 25, 26, 27, 28, 29 , 30, 31, 32, 33, 34, 35, 36, 37, 38, 39, 40, 41, 42, 43. 
%\item[--] \romannumeral7. 1, 2, 3, 4, 5, 6, 7, 8, 9, 10, 11, 12, 13, 14, 15, 16, 17, 18, 19, 20, 21, 22, 23, 24, 25.
%\item[--] \romannumeral8. 1, 2, 3, 4, 5, 6, 7, 8, 9, 10, 11, 12, 13, 14, 15, 16, 17, 18, 19, 20, 21.
%\item[--] \romannumeral9. 1, 2, 3, 4, 5, 6, 7, 8, 9, 10, 11, 12, 13, 14, 15, 16, 17, 18, 19, 20, 21, 22, 23, 24, 25, 26, 27, 28, 29, 30, 31, 32, 33, 34, 35, 36, 37, 38, 39, 40, 41, 42, 43, 44.
%\item[--] \romannumeral10. 1, 2, 3, 4, 5, 6, 7, 8, 9, 10, 11, 12, 13, 14, 15, 16, 17, 18, 19, 20, 21, 22, 23, 24, 25, 26.
%\item[--] \romannumeral11. 1, 2, 3, 4, 5, 6, 7, 8, 9, 10, 11, 12, 13, 14, 15, 16.
%\item[--] \romannumeral12. 1, 2, 3, 4, 5, 6, 7, 8, 9, 10, 11, 12, 13, 14, 15, 16, 17.
\end{itemize}



\begin{center}
Jacek Gancarzewicz\\
,,Arytmetyka'' A.
\end{center}
\begin{itemize}
\item[--]\romannumeral1. 1, 2, 3, 4, 5, 6, 7, 8, 9, 10, 11, 12, 13, 14, 15, 16, 17, 18, 19, 20, 21, 22, 23, 24, 25, 26, 27, 28, 29, 30, 31, 32, 33, 34, 35, 36, 37, 38, 39, 40 ,41, 42, 43, 44, 45, 46, 47.
%\item[--]\romannumeral2. 1, 2, 3, 4, 5, 6, 7, 8, 9, 10, 11, 12, 13, 14, 15, 16, 17, 18, 19, 20, 21, 22, 23, 24, 25, 26, 27, 28, 29, 30, 31, 32, 33, 34.
%\item[--]\romannumeral3. 1, 2, 3, 4, 5, 6, 7, 8, 9, 10, 11, 12, 13, 14, 15, 16, 17, 18, 19, 20, 21, 22, 23, 24, 25, 26.
%\item[--]\romannumeral4. 1, 2, 3, 4, 5, 6, 7, 8.
\end{itemize}



\begin{center}
N. Dróbka, K. Szymański\\
,,Zbiór zadań z matematyki'', \cite{NDKSzZZM}.
\end{center}
\begin{itemize}
\item[--]\romannumeral1. 1\cm , 2, 3, 4, 5, 6, 7, 8, 9, 10, 11, 12, 13, 14, 15, 16, 17, 18, 19, 20, 21, 22, 23, 24, 25, 26, 27, 28, 29, 30, 31, 32, 33, 34, 35, 36, 37, 38, 39, 40 ,41, 42, 43, 44, 45, 46, 47.
%\item[--]\romannumeral2. 1, 2, 3, 4, 5, 6, 7, 8, 9, 10, 11, 12, 13, 14, 15, 16, 17, 18, 19, 20, 21, 22, 23, 24, 25, 26, 27, 28, 29, 30, 31, 32, 33, 34.
%\item[--]\romannumeral3. 1, 2, 3, 4, 5, 6, 7, 8, 9, 10, 11, 12, 13, 14, 15, 16, 17, 18, 19, 20, 21, 22, 23, 24, 25, 26.
%\item[--]\romannumeral4. 1, 2, 3, 4, 5, 6, 7, 8.
\end{itemize}



\begin{center}
Kazimierz Kuratowski\\
,,Wstęp do teorii mnogości i topologii'',\\ wydanie \romannumeral9, KK.
\end{center}
\begin{itemize}
\item[--] \romannumeral1. 1\cm, 2\cm, 3\cm, 4\cm, 5, 6\cm, 7\cm, 7a\cm, 8\cm, 9\cm, 10\cm.
%\item[--] \romannumeral2. 1\cm, 2, 3, 4, 5, 6, 7, 8, 9, 10, 11, 12, 13, 14.
%\item[--] \romannumeral3. 1, 2, 3, 4, 5, 6.
%\item[--] \romannumeral4. 1, 2, 3, 4, 5, 6, 7, 8, 9, 10, 11, 12, 13, 14, 15, 16, 17, 18, 19, 20, 21, 22, 23, 24, 25, 26.
%\item[--] \romannumeral5. 1, 2, 3, 4, 5, 6, 7, 8, 9, 10.
%\item[--] \romannumeral6. 1, 2, 3, 4.
%\item[--] \romannumeral7. 1, 2, 3, 4, 5, 6, 7, 8.
%\item[--] \romannumeral8. 1, 2, 3, 4, 5, 6, 7, 8, 9, 10, 11, 12.
%\item[--] \romannumeral9. 1, 2, 3, 4, 5.
%\item[--] \romannumeral10. 1, 2, 3, 4, 5, 6, 7, 8, 9, 10, 11, 12, 13, 14, 15, 16, 17.
%\item[--] \romannumeral11. 1, 2, 3, 4, 5, 6, 7, 8, 9, 10, 11, 12, 13, 14, 15, 16, 17, 18, 19, 20, 21, 22.
%\item[--] \romannumeral12. 1, 2, 3, 4, 5, 6, 7, 8, 9, 10, 11, 12, 13, 14, 15.
%\item[--] \romannumeral13. 1, 2, 3, 4, 5, 6, 7, 8, 9, 10.
%\item[--] \romannumeral14. 1, 2, 3, 4, 5, 6, 7, 8, 9, 10, 11, 12, 13, 14, 15.
%\item[--] \romannumeral15. 1, 2, 3, 4, 5, 6, 7, 8, 9, 10.
%\item[--] \romannumeral16. 1, 2, 3, 4, 5, 6, 7, 8, 9, 10, 11, 12, 13, 14, 15, 16, 17, 18, 19, 20, 21, 22, 23, 24, 25, 26, 27, 28, 29, 30, 31, 32, 33, 34, 35, 36, 37, 38, 39, 40 ,41, 42.
%\item[--] \romannumeral17. 1, 2, 3, 4, 5, 6, 7, 8, 9, 10, 11, 12, 13, 14, 15, 16, 17, 18.
%\item[--] \romannumeral18. 1, 2, 3, 4, 5, 6, 7, 8, 9, 10, 11, 12, 13.
%\item[--] \romannumeral19. 1, 2, 3, 4, 5.
%\item[--] \romannumeral20. 1, 2, 3, 4, 5, 6, 7, 8, 9, 10, 11, 12.
%\item[--] \romannumeral20. 1, 2, 3, 4, 5, 6, 7, 8, 9, 10, 11, 12, 13, 14, 15, 16, 17.
\end{itemize}



\begin{center}
Jacek Gancarzewicz\\
,,Zarys współczesnej geometrii różniczkowej'', \cite{JGZWGR}.
\end{center}
\begin{itemize}
\item[--] \romannumeral1. 1, 2, 3, 4, 5, 6, 7, 8, 9, 10, 11, 12, 13, 14, 15, 16, 17, 18.
%\item[--] \romannumeral2. 1, 2, 3, 4, 5, 6, 7, 8, 9, 10, 11, 12, 13, 14, 15, 16, 17, 18, 19, 20.
%\item[--] \romannumeral3. 1, 2, 3, 4, 5, 6, 7, 8, 9, 10, 11, 12, 13, 14, 15, 16
%\item[--] \romannumeral4. 1, 2, 3, 4, 5, 6, 7, 8.
%\item[--] \romannumeral5. 1, 2, 3, 4, 5.
%\item[--] \romannumeral6. 1, 2, 3, 4, 5, 6, 7, 8.
%\item[--] \romannumeral7. 1, 2.
%\item[--] \romannumeral8. 1, 2, 3, 4, 5.
%\item[--] \romannumeral9. 1, 2, 3, 4, 5, 6.
%\item[--] \romannumeral10. 1, 2, 3, 4, 5, 6, 7, 8, 9, 10, 11, 12, 13, 14.
\end{itemize}



\begin{center}
Wojciech Wojtyński\\
,,Grupy i algebry Liego'',\\ wydanie \romannumeral1, GL.
\end{center}
\begin{itemize}
\item[--] \romannumeral2. 1, 2, 3, 4, 5, 6, 7, 8, 9, 10, 11, 12, 13.\\
%\item[--] \romannumeral3. 1, 2, 3, 4, 5, 6, 7, 8.\\
%\item[--] \romannumeral4. 1, 2, 3, 4, 5.\\
%\item[--] \romannumeral5. 1, 2, 3, 4, 5, 6.\\
%\item[--] \romannumeral3. 1, 2, 3, 4, 5, 6, 7, 8.\\
\end{itemize}



\bibliographystyle{ieeetr}
\bibliography{BibliographySBPL}{}


\end{document}