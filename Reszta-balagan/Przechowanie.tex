%% \begin{center}
%%   Włodzimierz Stankiewicz\\
%%   ,,Zadania z matematyki dla wyższych uczelni technicznych. Tom
%%   I'', \cite{WSZMWUTI}.
%% \end{center}
%%
%% Uwagi:
%% \begin{itemize}
%% \item[--] 21.4. Sposób rozwiązania tego zadania można w wielu
%%   wypadkach uprościć. W książce Schwartza znajdujemy twierdzenie
%%   które w prosty sposób uogólnia się w następujący sposób: niech
%%   funkcja $f$ będzie określona na jakimś podzbiorze $\mathbb{R}$
%%   zawierającym odcinek $[ a, b )$. Jeśli jest różniczkowalna na $(
%%   a, b )$ i jej pochodna posiada granicę $l$ w punkcie $a$ to
%%   istnieje pochodna prawostronna funkcji $f$ w $a$ i zachodzi $f^{
%%   + }( a ) = l$. Zastosowanie tego twierdzenia zwalnia nas w wielu
%%   wypadkach z bezpośredniego obliczania $f^{ + }( a )$.\newline
%%   Analiza przypadku $( a, b ]$ jest analogiczna. Można zadać sobie
%%   pytanie czy można wziąć ogólniejszy zbiór niż $[ a, b )$, jednak
%%   powyższy przypadek jest najważniejszy z punktu zastosowań.
%% \item
%% \item
%% \item
%% \end{itemize}
%%
%% Błędy:\\
%% \begin{tabular}{|c|c|c|c|c|}
%%\hline
%%& \multicolumn{2}{c|}{} & & \\
%%Strona & \multicolumn{2}{c|}{Wiersz}& Jest & Powinno być \\ \cline{2-3}
%%& Od góry & Od dołu &  &  \\ \hline
%%& & & & \\
%%& & & & \\ \hline
%%\end{tabular}
%%
%%


%\begin{center}
%Bogusław Gdowski\\
%,,Elementy geometrii różniczkowej z zadaniami'',\\ wydanie \romannumeral4, BG.
%\end{center}
%\begin{itemize}
%\item[--] \romannumeral1. 1\cm, 2\cm, 3, 4, 5\cm, 6, 7\cm ?, 8, 9, 10, 11, 12, 13, 14, 15, 16, 17.
%\item[--] \romannumeral2. 1: 1, 2, 3, 4, 5, 6, 7, 8, 9, 10, 11, 12, 13, 14, 15.\\
%2: 1, 2, 3, 4, 5, 6, 7, 8, 9, 10, 11.\\
%3: 1, 2, 3, 4, 5, 6, 7, 8, 9, 10, 11, 12, 13, 14, 15, 16, 17, 18, 19, 20, 21, 22, 23, 24, 25, 26, 27, 28, 29, 30, 31, 32, 33, 34, 35, 36, 37, 38, 39, 40 , 41, 42, 43.\\
%4: 1, 2, 3, 4, 5, 6, 7, 8, 9, 10, 11, 12, 13, 14.\\
%5: 1, 2, 3, 4, 5, 6, 7, 8, 9, 10, 11, 12, 13, 14, 15, 16, 17, 18, 19.\\
%6: 1, 2, 3, 4, 5, 6, 7, 8, 9, 10, 11, 12, 13, 14.\\
%7: 1, 2, 3, 4, 5.\\
%8: 1, 2, 3, 4, 5, 6.\\
%9: 1, 2, 3, 4, 5, 6, 7, 8, 9, 10, 11, 12, 13, 14, 15, 16, 17, 18, 19, 20, 21, 22, 23, 24, 25, 26, 27, 28, 29.\\
%10: 1, 2, 3, 4, 5, 6, 7, 8, 9, 10, 11, 12, 13, 14, 15, 16, 17, 18, 19, 20, 21, 22, 23, 24, 25, 26, 27, 28, 29, 30, 31, 32, 33, 34, 35, 36, 37, 38.
%\item[--] \romannumeral3. 1: 1, 2, 3, 4, 5, 6, 7, 8, 9, 10, 11, 12, 13, 14, 15, 16, 17, 18, 19, 20, 21, 22.\\
%2: 1, 2, 3, 4, 5, 6, 7, 8, 9, 10, 11, 12, 13, 14, 15, 16, 17, 18, 19, 20, 21, 22, 23, 24, 25, 26, 27, 28, 29, 30, 31, 32, 33, 34, 35.\\
%3: 1, 2, 3, 4, 5, 6, 7, 8, 9, 10, 11, 12, 13, 14, 15, 16, 17, 18, 19, 20, 21, 22, 23, 24, 25, 26, 27, 28, 29, 30, 31, 32, 33, 34, 35, 36, 37, 38.\\
%4: 1, 2, 3, 4, 5, 6, 7, 8, 9, 10, 11, 12, 13, 14, 15, 16, 17, 18, 19, 20.\\
%5: 1, 2, 3, 4, 5, 6, 7, 8, 9, 10, 11, 12, 13, 14, 15, 16, 17, 18, 19, 20, 21, 22, 23.\\
%6: 1, 2, 3, 4, 5, 6, 7, 8, 9, 10, 11, 12, 13, 14, 15, 16, 17, 18, 19, 20, 21, 22, 23, 24, 25, 26, 27, 28, 29, 30, 31.\\
%7: 1, 2, 3, 4, 5, 6, 7, 8, 9, 10, 11, 12, 13, 14, 15, 16, 17, 18, 19, 20, 21, 22.\\
%8: 1, 2, 3, 4, 5, 6, 7, 8, 9, 10, 11, 12, 13, 14, 15, 16, 17, 18, 19, 20, 21, 22, 23, 24, 25, 26, 27, 28, 29, 30, 31, 32, 33, 34, 35, 36, 37, 38, 39, 40 ,41, 42, 43, 44, 45, 46, 47, 48, 49, 50, 51, 52, 53, 54, 55, 56, 57, 58, 59, 60, 61, 62.\\
%9: 1, 2, 3, 4, 5, 6, 7, 8, 9, 10, 11, 12, 13, 14, 15, 16, 17, 18, 19, 20, 21, 22, 23, 24, 25, 26.\\
%10: 1, 2, 3, 4, 5, 6, 7, 8, 9, 10, 11, 12, 13, 14, 15, 16, 17, 18.
%\end{itemize}
%
%
%
%\begin{center}
%Walter Rudin\\
%,,Podstawy analizy matematycznej'',\\ wydanie \romannumeral6, WR.
%\end{center}
%\begin{itemize}
%\item[--] \romannumeral1. 1, 2, 3, 4, 5, 6, 7, 8, 9, 10, 11, 12, 13, 14, 15, 16, 17, 18, 19, 20.
%\item[--] \romannumeral2. 1, 2, 3, 4, 5, 6, 7, 8, 9, 10, 11, 12, 13, 14, 15, 16, 17, 18, 19, 20, 21, 22, 23, 24, 25, 26, 27, 28, 29 ,30.
%\item[--] \romannumeral3. 1, 2, 3, 4, 5, 6, 7, 8, 9, 10, 11, 12, 13, 14, 15, 16, 17, 18, 19, 20, 21, 22, 23, 24, 25.
%\item[--] \romannumeral4. 1, 2, 3, 4, 5, 6, 7, 8, 9, 10, 11, 12, 13, 14, 15, 16, 17, 18, 19, 20, 21, 22, 23, 24, 25, 26.
%\item[--] \romannumeral5. 1, 2, 3, 4, 5, 6, 7, 8, 9, 10, 11, 12, 13, 14, 15, 16, 17, 18, 19, 20, 21, 22, 23, 24, 25, 26, 27, 28, 29.
%\item[--] \romannumeral6. 1, 2, 3, 4, 5, 6, 7, 8, 9, 10, 11, 12, 13, 14, 15, 16, 17, 18, 19.
%\item[--] \romannumeral7. 1, 2, 3, 4, 5, 6, 7, 8, 9, 10, 11, 12, 13, 14, 15, 16, 17, 18, 19, 20, 21, 22, 23, 24, 25, 26.
%\item[--] \romannumeral8. 1, 2, 3, 4, 5, 6, 7, 8, 9, 10, 11, 12, 13, 14, 15, 16, 17, 18, 19, 20, 21, 22, 23, 24, 25, 26, 27, 28, 29.
%\item[--] \romannumeral9. 1, 2, 3, 4, 5, 6, 7, 8, 9, 10, 11, 12, 13, 14, 15, 16, 17, 18, 19, 20, 21, 22, 23, 24, 25, 26, 27, 28, 29, 30, 31.
%\item[--] \romannumeral10. 1, 2, 3, 4, 5, 6, 7, 8, 9, 10, 11, 12, 13, 14, 15, 16, 17, 18, 19, 20, 21, 22, 23, 24, 25, 26, 27, 28, 29, 30, 31, 32.
%\item[--] \romannumeral11. 1, 2, 3, 4, 5, 6, 7, 8, 9, 10, 11, 12, 13, 14, 15, 16, 17, 18.
%\end{itemize}
%

% \begin{center}
% Walter Rudin\\
% ,,Analiza rzeczywista i zespolona'', \cite{WRARzZ}.
% \end{center}
% \begin{itemize}
% \item[--] \romannumeral1. 1\cm, 2, 3\cm, 4, 5, 6, 7, 8, 9, 10, 11, 12.
%\item[--] \romannumeral2. 1, 2, 3, 4, 5, 6, 7, 8, 9, 10, 11, 12, 13, 14, 15, 16, 17, 18, 19, 20.
%\item[--] \romannumeral3. 1, 2, 3, 4, 5, 6, 7, 8, 9, 10, 11, 12, 13, 14, 15, 16, 17, 18, 19, 20, 21, 22, 23, 24.
%\item[--] \romannumeral4. 1, 2, 3, 4, 5, 6, 7, 8, 9, 10, 11, 12, 13, 14, 15, 16, 17, 18, 19.
%\item[--] \romannumeral5. 1, 2, 3, 4, 5, 6, 7, 8, 9, 10, 11, 12, 13, 14, 15, 16, 17, 18, 19, 20, 21.
%\item[--] \romannumeral6. 1, 2, 3, 4, 5, 6, 7, 8, 9, 10, 11.
%\item[--] \romannumeral7. 1, 2, 3, 4, 5, 6, 7, 8, 9, 10, 11, 12.
%\item[--] \romannumeral8. 1, 2, 3, 4, 5, 6, 7, 8, 9, 10, 11, 12, 13, 14, 15, 16, 17, 18, 19, 20, 21, 22, 23, 24.
%\item[--] \romannumeral9. 1, 2, 3, 4, 5, 6, 7, 8, 9, 10, 11, 12, 13, 14, 15, 16, 17.
%\item[--] \romannumeral10. 1, 2, 3, 4, 5, 6, 7, 8, 9, 10, 11, 12, 13, 14, 15, 16, 17, 18, 19, 20, 21, 22, 23, 24, 25, 26, 27, 28.
%\item[--] \romannumeral11. 1, 2, 3, 4, 5, 6, 7, 8, 9, 10, 11, 12, 13, 14, 15, 16, 17, 18, 19, 20, 21, 22.
% \end{itemize}


%\begin{center}
%Walter Rudin\\
%,,Analiza funkcjonalna'',\\ wydanie \romannumeral2, WRAF.
%\end{center}
%\begin{itemize}
%\item[--] \romannumeral1. 1, 2, 3, 4, 5, 6, 7, 8, 9, 10, 11, 12, 13, 14, 15, 16, 17, 18, 19, 20, 21, 22, 23, 24.
%%\item[--] \romannumeral2. 1, 2, 3, 4, 5, 6, 7, 8, 9, 10, 11, 12, 13, 14, 15, 16.
%%\item[--] \romannumeral3. 1, 2, 3, 4, 5, 6, 7, 8, 9, 10, 11, 12, 13, 14, 15, 16, 17, 18, 19, 20, 21, 22, 23, 24, 25, 26, 27, 28, 29, 30, 31, 32, 33.
%%\item[--] \romannumeral4. 1, 2, 3, 4, 5, 6, 7, 8, 9, 10, 11, 12, 13, 14, 15, 16, 17, 18, 19, 20, 21, 22, 23, 24, 25, 26, 27.
%%\item[--] \romannumeral5. 1, 2, 3, 4, 5, 6, 7, 8, 9, 10, 11, 12, 13, 14, 15, 16, 17, 18, 19.
%%\item[--] \romannumeral6. 1, 2, 3, 4, 5, 6, 7, 8, 9, 10, 11, 12, 13, 14, 15, 16, 17, 18, 19, 20, 21, 22, 23, 24, 25, 26, 27.
%%\item[--] \romannumeral7. 1, 2, 3, 4, 5, 6, 7, 8, 9, 10, 11, 12, 13, 14, 15, 16, 17, 18, 19, 20, 21, 22, 23, 24.
%%\item[--] \romannumeral8. 1, 2, 3, 4, 5, 6, 7, 8, 9, 10, 11, 12, 13, 14, 15, 16.
%%\item[--] \romannumeral9. 1, 2, 3, 4, 5, 6, 7, 8, 9, 10, 11, 12, 13, 14, 15.
%%\item[--] \romannumeral10. 1, 2, 3, 4, 5, 6, 7, 8, 9, 10, 11, 12, 13, 14, 15, 16, 17, 18, 19, 20, 21, 22, 23, 24, 25, 26.
%%\item[--] \romannumeral11. 1, 2, 3, 4, 5, 6, 7, 8, 9, 10, 11, 12, 13, 14, 15, 16, 17, 18, 19, 20, 21, 22.
%%\item[--] \romannumeral12. 1, 2, 3, 4, 5, 6, 7, 8, 9, 10, 11, 12, 13, 14, 15, 16, 17, 18, 19, 20, 21, 22, 23, 24, 25, 26, 27, 28, 29, 30, 31, 32, 33, 34, 35, 36, 37, 38, 39, 40.
%%\item[--] \romannumeral13. 1, 2, 3, 4, 5, 6, 7, 8, 9, 10, 11, 12, 13, 14, 15, 16, 17, 18, 19, 20, 21, 22, 23, 24, 25.
%\end{itemize}

% \begin{center}
%   Włodzimierz Stankiewicz, J. Wojtowicz\\
%   ,,Zadania z matematyki dla wyższych uczelni technicznych, tom
%   II'',\\ wydanie \romannumeral3, WW.
% \end{center}
% \begin{itemize}
% \item[--] \romannumeral1. 5, 6, 7, 8.
%% \item[--] \romannumeral2. 12, 13, 14.
%% \item[--] \romannumeral3. 19, 20, 21, 22.
%% \item[--] \romannumeral4. 24, 25, 26, 27, 28.
%% \item[--] \romannumeral5. 31, 32, 33.
%% \item[--] \romannumeral6. 38, 39, 40, 41, 42, 43 ,44.
%% \item[--] \romannumeral7. 47, 48, 49, 50, 51, 52.
%% \item[--] \romannumeral8. 64, 65, 66, 67, 68, 69, 70, 71, 72, 73,
%%   74, 75, 76, 77, 78, 79, 80, 81, 82, 83.
%% \item[--] \romannumeral9. 86, 87, 88, 89, 90, 91, 92.
%% \item[--] \romannumeral10. 97, 98, 99, 100, 101, 102, 103, 104,
%%   105, 106.
%% \item[--] \romannumeral11. 110, 111, 112, 113, 114, 115.
%% \item[--] \romannumeral12. 120, 121, 122, 123, 124.
%% \item[--] \romannumeral13.
%% \item[--] \romannumeral14.
%% \item[--] \romannumeral15.
%% \item[--] \romannumeral16.
%% \item[--] \romannumeral17.
%% \item[--] \romannumeral18.
%% \item[--] \romannumeral19.
%% \item[--] \romannumeral20.
%% \item[--] \romannumeral21.
%% \item[--] \romannumeral9. 11, 12, 13, 14, 15, 16, 17, 18, 19, 20,
%%   21, 22, 23, 24, 25, 26, 27, 28, 29, 30, 31, 32, 33, 34, 35, 36,
%%   37, 38, 39, 40, 41, 42, 43, 44, 45, 46, 47, 48, 49, 50, 51, 52,
%%   53, 54, 55, 56, 57, 58, 59, 60, 61, 62, 63, 64, 65, 66, 67, 68,
%%   69, 70, 71, 72, 73, 74, 75, 76, 77, 78, 79, 80, 81, 82, 83, 84,
%%   85, 86, 87, 88, 89, 90, 91, 92, 93, 94, 95, 96, 97, 98, 99, 100.
%% \item[--] \romannumeral10. 5, 6, 7, 8, 9, 10, 11, 12, 13, 14, 15,
%%   16, 17, 18, 19, 20, 21, 22, 23, 24, 25, 26, 27, 28, 29, 30, 31,
%%   32.
%% \item[--] \romannumeral11. 4, 5, 6, 7, 8, 9, 10, 11, 12, 13, 14,
%%   15, 16, 17, 18, 19, 20, 21, 22, 23, 24, 25, 26, 27, 28, 29, 30,
%%   31, 32, 33, 34, 35, 36, 37, 38.
%% \item[--] \romannumeral12. 12, 13, 14, 15, 16, 17, 18, 19, 20, 21,
%%   22, 23, 24, 25, 26, 27, 28, 29, 30, 31, 32, 33, 34, 35, 36, 37,
%%   38, 39, 40, 41, 42, 43, 44, 45, 46, 47, 48, 49, 50, 51, 52, 53,
%%   54, 55, 56, 57, 58, 59, 60, 61, 62, 63, 64, 65, 66.
%% \item[--] \romannumeral13. 5, 6, 7, 8, 9, 10, 11, 12, 13, 14, 15,
%%   16, 17, 18, 19, 20, 21, 22, 23, 24, 25, 26, 27, 28, 29, 30, 31,
%%   32, 33, 34, 35, 36, 37, 38, 39, 40, 41, 42, 43, 44, 45, 46, 47.
%% \item[--] \romannumeral14. 11, 12, 13, 14, 15, 16, 17, 18, 19, 20,
%%   21, 22, 23, 24, 25, 26, 27, 28, 29, 30, 31, 32, 33, 34, 35, 36,
%%   37, 38, 39, 40, 41, 42, 43, 44, 45, 46, 47, 48, 49, 50, 51, 52,
%%   53, 54, 55, 56, 57, 58, 59, 60, 61, 62, 63, 64, 65, 66, 67, 68,
%%   69, 70, 71, 72, 73, 74, 75, 76, 77, 78, 79, 80, 81, 82, 83, 84,
%%   85.
%% \item[--] \romannumeral15. 11, 12, 13, 14, 15, 16, 17, 18, 19, 20,
%%   21, 22, 23, 24, 25, 26, 27, 28, 29, 30, 31, 32, 33, 34, 35, 36,
%%   37, 38, 39, 40, 41, 42, 43, 44, 45, 46.
%% \item[--] \romannumeral16. 11, 12, 13, 14, 15, 16, 17, 18, 19, 20,
%%   21, 22, 23, 24, 25, 26, 27, 28, 29, 30, 31, 32, 33, 34, 35, 36,
%%   37, 38, 39, 40, 41, 42, 43, 44, 45, 46, 47, 48, 49, 50, 51, 52,
%%   53, 54, 55, 56, 57, 58, 59, 60.
%% \item[--] \romannumeral17. 12, 13, 14, 15, 16, 17, 18, 19, 20, 21,
%%   22, 23, 24, 25, 26, 27, 28, 29, 30, 31, 32, 33, 34, 35, 36, 37,
%%   38, 39, 40, 41, 42, 43, 44, 45, 46, 47, 48, 49, 50, 51, 52, 53,
%%   54, 55, 56, 57, 58, 59, 60, 61, 62, 63, 64, 65, 66, 67, 68, 69,
%%   70, 71, 72, 73, 74, 75, 76, 77, 78, 79, 80, 81, 82, 83, 84.
%% \item[--] \romannumeral18. 6, 7, 8, 9, 10, 11, 12, 13, 14, 15, 16,
%%   17, 18, 19, 20, 21, 22, 23, 24, 25, 26, 27, 28, 29, 30, 31, 32,
%%   33
%% \item[--] \romannumeral19. 15, 16, 17, 18, 19, 20, 21, 22, 23, 24,
%%   25, 26, 27, 28, 29, 30, 31, 32, 33, 34, 35, 36, 37, 38, 39, 40,
%%   41, 42, 43, 44, 45, 46, 47, 48, 49, 50, 51, 52, 53, 54, 55, 56,
%%   57.
%% \item[--] \romannumeral20. 7, 8, 9, 10, 11, 12, 13, 14, 15, 16, 17,
%%   18, 19, 20, 21, 22, 23, 24, 25, 26, 27, 28, 29, 30, 31, 32, 33.
%% \item[--] \romannumeral21. 8, 9, 10, 11, 12, 13, 14, 15, 16, 17,
%%   18, 19, 20, 21, 22, 23, 24, 25, 26, 27, 28, 29, 30, 31, 32, 33,
%%   34, 35, 36, 37, 38, 39.
%% \item[--] \romannumeral22. 10, 11, 12, 13, 14, 15, 16, 17, 18, 19,
%%   20, 21, 22, 23, 24, 25, 26, 27, 28, 29, 30, 31, 32, 33, 34, 35,
%%   36, 37, 38, 39, 40, 41, 42, 43, 44, 45, 46, 47, 48, 49, 50, 51,
%%   52, 53, 54, 55.
%% \item[--] \romannumeral23. 10, 11, 12, 13, 14, 15, 16, 17, 18, 19,
%%   20, 21, 22, 23, 24, 25, 26, 27, 28, 29, 30, 31, 32, 33, 34, 35.
%% \item[--] \romannumeral24. 7, 8, 9, 10, 11, 12, 13, 14, 15, 16, 17,
%%   18, 19, 20, 21, 22, 23, 24, 25, 26, 27, 28, 29, 30, 31, 32, 33,
%%   34, 35, 36, 37, 38, 39.
%% \item[--] \romannumeral25. 11, 12, 13, 14, 15, 16, 17, 18, 19, 20,
%%   21, 22, 23, 24, 25, 26, 27, 28, 29.
%% \item[--] \romannumeral26. 5, 6, 7, 8, 9, 10, 11, 12, 13, 14, 15,
%%   16, 17, 18, 19, 20, 21, 22, 23, 24, 25, 26, 27, 28, 29, 30, 31,
%%   32, 33, 34, 35, 36, 37, 38, 39, 40.
%% \item[--] \romannumeral30. 8, 9, 10, 11, 12, 13, 14, 15, 16, 17,
%%   18, 19, 20, 21, 22, 23, 24, 25, 26, 27, 28, 29, 30, 31, 32, 33,
%%   34, 35, 36, 37, 38.
%% \item[--] \romannumeral31. 10, 11, 12, 13, 14, 15, 16, 17, 18, 19,
%%   20, 21, 22, 23, 24, 25, 26, 27, 28, 29, 30, 31, 32, 33, 34, 35,
%%   36, 37, 38, 39, 40, 41, 42, 43, 44, 45, 46, 47, 48, 49, 50, 51,
%%   52, 53, 54.
%% \item[--] \romannumeral32. 18, 19, 20, 21, 22, 23, 24, 25, 26, 27,
%%   28, 29, 30, 31, 32, 33, 34, 35.
%% \item[--] \romannumeral33. 10, 11, 12, 13, 14, 15, 16, 17, 18, 19,
%%   20, 21, 22, 23, 24, 25.
%% \item[--] \romannumeral34. 12, 13, 14, 15, 16, 17, 18, 19, 20, 21,
%%   22, 23, 24, 25, 26, 27, 28, 29, 30, 31, 32.
%% \item[--] \romannumeral35. 9, 10, 11, 12, 13, 14, 15, 16, 17, 18,
%%   19, 20.
%% \item[--] \romannumeral36. 9, 10, 11, 12, 13, 14, 15, 16, 17, 18,
%%   19, 20, 21, 22, 23, 24, 25, 26, 27, 28, 29, 30, 31, 32, 33, 34,
%%   35, 36, 37, 38, 39, 40, 41, 42, 43, 44, 45, 46, 47, 48, 49, 50,
%%   51, 52, 53, 54, 55, 56, 57, 58, 59, 60, 61, 62, 63, 64
%% \item[--] \romannumeral37. 5, 6, 7, 8, 9, 10, 11, 12, 13, 14, 15,
%%   16, 17, 18, 19, 20, 21, 22, 23, 24, 25, 26, 27, 28, 29, 30, 31,
%%   32, 33, 34, 35, 36, 37, 38, 39, 40.
%% \item[--] \romannumeral38. 6, 7, 8, 9, 10, 11, 12, 13, 14, 15, 16,
%%   17, 18, 19, 20, 21, 22, 23, 24, 25, 26, 27, 28, 29, 30, 31.
%% \item[--] \romannumeral39. 6, 7, 8, 9, 10, 11, 12, 13, 14, 15, 16,
%%   17, 18, 19, 20, 21, 22, 23, 24, 25, 26, 27, 28, 29, 30.
%% \item[--] \romannumeral40. 6, 7, 8, 9, 10, 11, 12, 13.
%% \item[--] \romannumeral41. 12, 13, 14, 15, 16, 17, 18, 19, 20, 21,
%%   22, 23, 24, 25, 26, 27, 28, 29, 30.
%% \item[--] \romannumeral42. 12, 13, 14, 15, 16, 17, 18, 19, 20, 21,
%%   22, 23, 24, 25, 26, 27, 28, 29, 30, 31, 32, 33, 34, 35, 36, 37,
%%   38, 39, 40, 41, 42, 43, 44, 45.
%% \item[--] \romannumeral43. 7, 8, 9, 10, 11, 12, 13, 14, 15, 16, 17,
%%   18, 19, 20, 21, 22, 23, 24, 25, 26, 27, 28, 29, 30, 31, 32, 33,
%%   34.
% \end{itemize}
%
%
%
%
%
%
%

%
%
% \begin{center}
%   William Feller\\
%   ,,Wstęp do rachunku prawdopodobieństwa, tom I'', WFI.
% \end{center}
% \begin{itemize}
% \item[--] \romannumeral1. 8: 1, 2, 3, 4, 5, 6, 7, 8, 9, 10, 11, 12,
%   13, 14, 15, 16, 17, 18, 19.
%% \item[--] \romannumeral2. 10: 1\cm, 2\cm, 3\cm, 4, 5\cm, 6, 7, 8, 9, 10, 11, 12, 13, 14, 15, 16, 17, 18, 19, 20, 21, 22, 23, 24, 25, 26, 27, 28, 29 , 30, 31, 32, 33, 34, 35, 36, 37, 38, 39, 40, 41, 42, 43, 44, 45.\\
%%   11: 1, 2, 3, 4, 5, 6, 7, 8, 9, 10, 11, 12, 13, 14, 15, 16, 17, 18, 19, 20, 21, 22, 23, 24, 25.\\
%%   12: 1, 2, 3, 4, 5, 6, 7, 8, 9, 10, 11, 12, 13, 14, 15, 16, 17,
%%   18, 19, 20, 21, 22, 23, 24, 25, 26, 27.
%% \item[--]
%%   \romannumeral3. %1, 2, 3, 4, 5, 6, 7, 8, 9, 10, 11, 12, 13, 14, 15, 16, 17, 18, 19, 20, 21, 22, 23, 24, 25, 26, 27.
%% \item[--]
%%   \romannumeral4. %1, 2, 3, 4, 5, 6, 7, 8, 9, 10, 11, 12, 13, 14, 15, 16, 17, 18, 19, 20, 21, 22, 23, 24, 25, 26.
%% \item[--] \romannumeral5. 1, 2, 3, 4, 5, 6, 7, 8, 9, 10, 11, 12,
%%   13, 14, 15, 16, 17, 18, 19, 20, 21, 22, 23, 24, 25, 26, 27, 28,
%%   29 , 30, 31, 32, 33, 34, 35, 36, 37, 38, 39, 40.
%% \item[--] \romannumeral6. 1, 2, 3, 4, 5, 6, 7, 8, 9, 10, 11, 12,
%%   13, 14, 15, 16, 17, 18, 19, 20, 21, 22, 23, 24, 25, 26, 27, 28,
%%   29 , 30, 31, 32, 33, 34, 35, 36, 37, 38, 39, 40, 41, 42, 43, 44,
%%   45, 46, 47, 48.
%% \item[--] \romannumeral7. 1, 2, 3, 4, 5, 6, 7, 8, 9, 10, 11, 12,
%%   13, 14, 15, 16, 17, 18, 19, 20, 21.
%% \item[--] \romannumeral8. 1, 2, 3, 4, 5, 6, 7, 8, 9, 10, 11, 12,
%%   13, 14, 15, 16, 17, 18, 19, 20, 21, 22, 23, 24, 25, 26, 27, 28,
%%   29 , 30, 31, 32, 33, 34, 35, 36, 37, 38, 39, 40, 41.
%% \item[--] \romannumeral10. 1, 2, 3, 4, 5, 6, 7, 8, 9, 10, 11, 12,
%%   13, 14, 15, 16, 17, 18, 19.
%% \item[--] \romannumeral11. 1, 2, 3, 4, 5, 6, 7, 8, 9, 10, 11, 12,
%%   13, 14, 15, 16, 17, 18, 19, 20, 21, 22, 23, 24, 25, 26.
%% \item[--] \romannumeral12. 1, 2, 3, 4, 5, 6, 7, 8, 9.
%% \item[--] \romannumeral13. 1, 2, 3, 4, 5, 6, 7, 8, 9, 10, 11, 12,
%%   13, 14, 15, 16, 17, 18, 19, 20, 21, 22, 23, 24, 25, 26.
%% \item[--] \romannumeral14. 1, 2, 3, 4, 5, 6, 7, 8, 9, 10, 11, 12,
%%   13, 14, 15, 16, 17, 18, 19, 20, 21, 22, 23.
%% \item[--] \romannumeral15. 1, 2, 3, 4, 5, 6, 7, 8, 9, 10, 11, 12,
%%   13, 14, 15, 16, 17, 18, 19, 20, 21, 22, 23, 24, 25, 26, 27, 28.
%% \item[--] \romannumeral17. 1, 2, 3, 4, 5, 6, 7, 8, 9, 10, 11, 12,
%%   13, 14, 15, 16, 17, 18, 19.
% \end{itemize}

%
%
%
%\begin{center}
%  Maciej Bryński, Jerzy Jurkiewicz\\
%  ,,Zarys zadań z algebry'',\\ wydanie \romannumeral2, BJ.
%\end{center}
%\begin{itemize}
%\item[--] \romannumeral1. 1: 1\cm ,2\cm , 3, 4\cm, 5\cm, 6\cm, 7,
%  8\cm, 9\cm, 10\cm, 11, 12\cm, 13\cm, 14, 15, 16, 17, 18\cm, 19\cm,
%  20, 21, 22, 23, 24, 25, 26, 27, 28, 29, 30, 31, 32, 33, 34, 35, 36,
%  37, 38.
%%  \\2: 1, 2, 3, 4, 5, 6, 7, 8, 9, 10, 11, 12, 13, 14, 15, 16, 17,
%%  18, 19, 20, 21, 22, 23, 24. \\3:1, 2, 3, 4, 5, 6, 7, 8, 9, 10, 11,
%%  12, 13, 14, 15, 16, 17, 18, 19, 20, 21, 22, 23.
%%  \\4: 1, 2, 3\cm, 4, 5, 6, 7, 8, 9, 10, 11, 12, 13, 14, 15, 16, 17, 18, 19, 20, 21, 22.\\
%%  5: 1, 2, 3, 4, 5, 6, 7, 8, 9, 10, 11, 12, 13, 14, 15, 16, 17, 18, 19, 20, 21, 22, 23, 24.\\
%%  3:1, 2, 3, 4, 5, 6, 7, 8, 9, 10, 11, 12, 13, 14, 15, 16, 17, 18, 19, 20, 21, 22, 23, 24, 25, 26, 27.\\
%%  6: 1, 2, 3, 4, 5, 6, 7, 8, 9, 10, 11, 12, 13, 14, 15, 16, 17, 18, 19, 20, 21, 22, 23, 24, 25, 26, 27, 28, 29.\\
%%  7: 1, 2, 3, 4, 5, 6, 7, 8, 9, 10, 11, 12, 13, 14, 15, 16, 17, 18,
%%  19, 20, 21, 22, 23, 24, 25, 26, 27, 28, 29, 30, 31, 32, 33, 34.
%%\item[--] \romannumeral2. 1: 1, 2, 3, 4, 5, 6, 7, 8, 9, 10, 11, 12, 13, 14, 15, 16, 17, 18, 19, 20, 21, 22, 23, 24, 25, 26, 27, 28, 29.\\
%%  2: 1, 2, 3, 4, 5, 6, 7, 8, 9, 10, 11, 12, 13, 14, 15, 16, 17, 18, 19, 20, 21, 22.\\
%%  3: 1, 2, 3, 4, 5, 6, 7, 8, 9, 10, 11, 12, 13, 14, 15, 16, 17, 18, 19, 20, 21, 22, 23, 24, 25, 26, 27, 28, 29, 30, 31, 32, 33, 34, 35, 36, 37, 38, 39, 40 ,41, 42.\\
%%  4: 1, 2, 3, 4, 5, 6, 7, 8, 9, 10, 11, 12, 13, 14, 15, 16, 17, 18, 19, 20, 21, 22, 23, 24, 25.\\
%%  5: 1, 2, 3, 4, 5, 6, 7, 8, 9, 10, 11, 12, 13, 14, 15, 16, 17, 18, 19, 20, 21, 22, 23, 24, 25, 26, 27, 28, 29, 30.\\
%%  6: 1, 2, 3, 4, 5, 6, 7, 8, 9, 10, 11, 12, 13, 14, 15, 16, 17, 18, 19, 20.\\
%%  7: 1, 2, 3, 4, 5, 6, 7, 8, 9, 10.\\
%%  8: 1, 2, 3, 4, 5, 6, 7, 8, 9, 10, 11, 12, 13, 14, 15, 16, 17, 18,
%%  19, 20, 21, 22, 23, 24, 25.  9: 1, 2, 3, 4, 5, 6, 7, 8, 9, 10, 11,
%%  12, 13, 14, 15, 16, 17, 18, 19, 20, 21, 22, 23, 24, 25, 26, 27,
%%  28, 29, 30.
%%\item[--] \romannumeral3. 1: 1, 2, 3, 4, 5, 6, 7, 8, 9, 10, 11, 12, 13, 14, 15.\\
%%  2: 1, 2, 3, 4, 5, 6, 7, 8, 9, 10, 11, 12, 13, 14, 15, 16, 17, 18, 19, 20, 21, 22, 23, 24, 25, 26, 27, 28, 29, 30, 31, 32, 33, 34, 35, 36, 37, 38, 39, 40 ,41, 42, 43, 44, 45, 46.\\
%%  3: 1, 2, 3, 4, 5, 6, 7, 8, 9, 10, 11, 12, 13, 14, 15, 16, 17, 18, 19, 20, 21, 22, 23.\\
%%  4: 1, 2, 3, 4, 5, 6, 7, 8, 9, 10, 11, 12, 13, 14.\\
%%  5: 1, 2, 3, 4, 5, 6, 7, 8, 9, 10, 11, 12, 13, 14, 15, 16, 17, 18, 19, 20, 21, 22, 23, 24, 25, 26, 27, 28, 29, 30, 31, 32, 33, 34, 35, 36, 37, 38, 39, 40 ,41, 42, 43.\\
%%  6: 1, 2, 3, 4, 5, 6, 7, 8, 9, 10, 11, 12, 13, 14, 15, 16, 17, 18,
%%  19, 20, 21, 22, 23, 24, 25, 26, 27, 28, 29, 30, 31, 32, 33, 34.
%%\item[--] \romannumeral4. 1: 1, 2, 3, 4, 5, 6, 7, 8, 9, 10, 11, 12, 13, 14, 15, 16, 17, 18, 19, 20, 21, 22, 23, 24, 25, 26.\\
%%  2: 1, 2, 3, 4, 5, 6, 7, 8, 9, 10, 11, 12, 13, 14, 15, 16, 17, 18, 19, 20, 21, 22, 23, 24, 25, 26.\\
%%  3: 1, 2, 3, 4, 5, 6, 7, 8, 9, 10, 11, 12, 13, 14, 15, 16, 17, 18.\\
%%  4: 1, 2, 3, 4, 5, 6, 7, 8, 9, 10, 11, 12, 13.
%%\item[--] \romannumeral5. 1: 1, 2, 3, 4, 5, 6, 7, 8, 9, 10, 11, 12, 13, 14, 15, 16, 17, 18, 19, 20, 21, 22, 23, 24, 25, 26, 27, 28, 29, 30, 31, 32, 33, 34, 35, 36.\\
%%  2: 1, 2, 3, 4, 5, 6, 7, 8, 9, 10, 11, 12, 13, 14, 15, 16, 17, 18,
%%  19, 20, 21, 22, 23, 24, 25, 26.
%\end{itemize}







%\begin{center}
%  Henryk Arodź, Krzysztof Rościszewski\\
%  ,,Algebra i geometria analityczna w zadaniach'', \cite{HAKRAGAZ}.
%\end{center}
%\begin{itemize}
%\item[--] \romannumeral1. 1, 2, 3, 4, 5, 6, 7, 8, 9, 10, 11, 12, 13,
%  14, 15, 16, 17, 18, 19, 20, 21, 22, 23, 24, 25, 26, 27, 28, 29, 30,
%  31, 32, 33, 34, 35, 36, 37, 38, 39, 40, 41, 42, 43, 44.
%\item[--] \romannumeral2. 1\cm, 2, 3\cm, 4, 5, 6\cm, 7\cm, 8, 9, 10,
%  11, 12, 13, 14, 15, 16, 17, 18, 19, 20, 21, 22, 23, 24, 25, 26, 27,
%  28, 29 , 30, 31, 32, 33, 34, 35, 36, 37, 38, 39, 40, 41, 42, 43,
%  44, 45, 46, 47, 48, 49.
%%\item[--] \romannumeral3. 1, 2, 3, 4, 5, 6, 7, 8, 9, 10, 11, 12, 13,
%%  14, 15, 16, 17, 18, 19, 20, 21.
%%\item[--] \romannumeral4. 1, 2, 3, 4, 5, 6, 7, 8, 9, 10, 11\cm,
%%  12\cm, 13\cm, 14, 15, 16, 17, 18, 19, 20, 21, 22, 23, 24, 25, 26,
%%  27, 28, 29 , 30, 31, 32, 33, 34, 35, 36, 37, 38, 39, 40, 41, 42,
%%  43, 44, 45, 46.
%%\item[--] \romannumeral5. 1, 2, 3, 4, 5, 6, 7, 8, 9, 10, 11, 12, 13,
%%  14, 15, 16, 17, 18, 19, 20, 21, 22, 23, 24, 25, 26, 27, 28, 29 ,
%%  30, 31, 32, 33.
%%\item[--] \romannumeral6. 1, 2, 3, 4, 5, 6, 7, 8, 9, 10, 11, 12, 13,
%%  14, 15, 16, 17, 18, 19, 20, 21, 22, 23, 24, 25, 26, 27, 28, 29 ,
%%  30, 31, 32, 33, 34, 35, 36, 37, 38, 39, 40, 41, 42, 43.
%%\item[--] \romannumeral7. 1, 2, 3, 4, 5, 6, 7, 8, 9, 10, 11, 12, 13,
%%  14, 15, 16, 17, 18, 19, 20, 21, 22, 23, 24, 25.
%%\item[--] \romannumeral8. 1, 2, 3, 4, 5, 6, 7, 8, 9, 10, 11, 12, 13,
%%  14, 15, 16, 17, 18, 19, 20, 21.
%%\item[--] \romannumeral9. 1, 2, 3, 4, 5, 6, 7, 8, 9, 10, 11, 12, 13,
%%  14, 15, 16, 17, 18, 19, 20, 21, 22, 23, 24, 25, 26, 27, 28, 29,
%%  30, 31, 32, 33, 34, 35, 36, 37, 38, 39, 40, 41, 42, 43, 44.
%%\item[--] \romannumeral10. 1, 2, 3, 4, 5, 6, 7, 8, 9, 10, 11, 12,
%%  13, 14, 15, 16, 17, 18, 19, 20, 21, 22, 23, 24, 25, 26.
%%\item[--] \romannumeral11. 1, 2, 3, 4, 5, 6, 7, 8, 9, 10, 11, 12,
%%  13, 14, 15, 16.
%%\item[--] \romannumeral12. 1, 2, 3, 4, 5, 6, 7, 8, 9, 10, 11, 12,
%%  13, 14, 15, 16, 17.
%\end{itemize}
%
%
%
%\begin{center}
%  Jacek Gancarzewicz\\
%  ,,Arytmetyka'' A.
%\end{center}
%\begin{itemize}
%\item[--]\romannumeral1. 1, 2, 3, 4, 5, 6, 7, 8, 9, 10, 11, 12, 13,
%  14, 15, 16, 17, 18, 19, 20, 21, 22, 23, 24, 25, 26, 27, 28, 29, 30,
%  31, 32, 33, 34, 35, 36, 37, 38, 39, 40 ,41, 42, 43, 44, 45, 46, 47.
%%\item[--]\romannumeral2. 1, 2, 3, 4, 5, 6, 7, 8, 9, 10, 11, 12, 13,
%%  14, 15, 16, 17, 18, 19, 20, 21, 22, 23, 24, 25, 26, 27, 28, 29,
%%  30, 31, 32, 33, 34.
%%\item[--]\romannumeral3. 1, 2, 3, 4, 5, 6, 7, 8, 9, 10, 11, 12, 13,
%%  14, 15, 16, 17, 18, 19, 20, 21, 22, 23, 24, 25, 26.
%%\item[--]\romannumeral4. 1, 2, 3, 4, 5, 6, 7, 8.
%\end{itemize}
%
%
%
%\begin{center}
%  Jacek Gancarzewicz\\
%  ,,Zarys współczesnej geometrii różniczkowej'', \cite{JGZWGR}.
%\end{center}
%\begin{itemize}
%\item[--] \romannumeral1. 1, 2, 3, 4, 5, 6, 7, 8, 9, 10, 11, 12, 13,
%  14, 15, 16, 17, 18.
%%\item[--] \romannumeral2. 1, 2, 3, 4, 5, 6, 7, 8, 9, 10, 11, 12, 13,
%%  14, 15, 16, 17, 18, 19, 20.
%%\item[--] \romannumeral3. 1, 2, 3, 4, 5, 6, 7, 8, 9, 10, 11, 12, 13,
%%  14, 15, 16
%%\item[--] \romannumeral4. 1, 2, 3, 4, 5, 6, 7, 8.
%%\item[--] \romannumeral5. 1, 2, 3, 4, 5.
%%\item[--] \romannumeral6. 1, 2, 3, 4, 5, 6, 7, 8.
%%\item[--] \romannumeral7. 1, 2.
%%\item[--] \romannumeral8. 1, 2, 3, 4, 5.
%%\item[--] \romannumeral9. 1, 2, 3, 4, 5, 6.
%%\item[--] \romannumeral10. 1, 2, 3, 4, 5, 6, 7, 8, 9, 10, 11, 12,
%%  13, 14.
%\end{itemize}
%
%
%
%\begin{center}
%  Wojciech Wojtyński\\
%  ,,Grupy i algebry Liego'',\\ wydanie \romannumeral1, GL.
%\end{center}
%\begin{itemize}
%\item[--] \romannumeral2. 1, 2, 3, 4, 5, 6, 7, 8, 9, 10, 11, 12, 13.\\
%%\item[--] \romannumeral3. 1, 2, 3, 4, 5, 6, 7, 8.\\
%%\item[--] \romannumeral4. 1, 2, 3, 4, 5.\\
%%\item[--] \romannumeral5. 1, 2, 3, 4, 5, 6.\\
%%\item[--] \romannumeral3. 1, 2, 3, 4, 5, 6, 7, 8.\\
%\end{itemize}









% \Work{
% Steven Weinberg \\
% ,,Teoria pól kwantowych. Podstawy.'', \cite{Wei12}.}


% Uwagi: \\
% \start \Str{27} \\
% Str. 79. Z faktu, że istnieje macierz odwrotna do
% $\eta_{ \mu \nu } \Lambda^{ \mu }_{ \rho }$????!!!!! \\


% Błędy:\\
% \begin{center}
%   \begin{tabular}{|c|c|c|c|c|}
%     \hline
%     & \multicolumn{2}{c|}{} & & \\
%     Strona & \multicolumn{2}{c|}{Wiersz} & Jest & Powinno być \\ \cline{2-3}
%     & Od góry & Od dołu &  &  \\ \hline
%     & & & & \\
%     31 & 4 & & $-c^{ 2 } \hbar^{ 2 } \nabla^{ 2 }$ & $+c^{ 2 } \hbar^{ 2 } \nabla^{ 2 }$ \\
%     52 & 14 & & $b^{ \dagger }( \bold{ k } ) \exp( i \omega_{ \bold{ k } } t )$ & $b^{ \dagger }( \bold{ k } ) \exp( +i \omega_{ \bold{ k } } t )$ \\
%     59 & & 4 & Heinsenberg [72[ & Heinsenberg [72] \\
%     60 & 5 & & przyspieszone & przyśpieszone \\
%     70 & 2 & & 1961. & 1961). \\
%     79 & 3 & & $\eta_{ \mu \nu } dx^{ ' \mu } dx^{ ' \nu }$ & $\eta_{ \mu \nu } dx'^{ \mu } dx'^{ \nu }$ \\
%     79 & 5 & & $\eta_{ \mu \nu } \frac{ \partial x^{ ' \mu } }{ \partial x^{ \rho } } \frac{ \partial x^{ ' \nu } }{ \partial x^{ \sigma } }$ & $\eta_{ \mu \nu } \frac{ \partial x'^{ \mu } }{ \partial x^{ \rho } } \frac{ \partial x'^{ \nu } }{ \partial x^{ \sigma } }$ \\
%     86 & & 2 & $\Lambda_{ \rho }^{ -1 \mu } P^{ \rho }$ & $\tensor[]{ ( \Lambda^{ -1 } ) }{ ^\mu_\rho } P^{ \rho }$ \\
%     282 & 1 & & $S_{ \bold{ p }_{ 1 }', \sigma_{ 1 }', n_{ 1 }'; \, \bold{ p }_{ 2 }', \sigma_{ 2 }', n_{ 2 }';\; \cdots, \;  \bold{ p }_{ 1 }, \sigma_{ 1 }, n_{ 1 } ; \, \bold{ p }_{ 2 }, \sigma_{ 2 }, n_{ 2 }; \, \cdots }$ & $S_{ \bold{ p }_{ 1 }', \sigma_{ 1 }', n_{ 1 }'; \, \bold{ p }_{ 2 }', \sigma_{ 2 }', n_{ 2 }';\; \cdots; \;  \bold{ p }_{ 1 }, \sigma_{ 1 }, n_{ 1 } ; \, \bold{ p }_{ 2 }, \sigma_{ 2 }, n_{ 2 }; \, \cdots }$ \\
%     & & & & \\ \hline
%   \end{tabular}
% \end{center}

% \begin{itemize}
% \item[--] Str. 85. \ldots
% \item[--] Str. 86. Równania (2.5.1) i\ldots
% \item[--] Str. 86. równanie na dole
%   % \item[--]
%   % \item[--]
%   % \item[--]
%   % \item[--]
%   % \item[--]
%   % \item[--]
%   % \item[--]
%   % \item[--]
%   % \item[--]
%   % \item[--]
%   % \item[--]
%   % \item[--]
%   % \item[--]
%   % \item[--]
% \end{itemize}









% \begin{center}
%   Hagen Kleinert \\
%   ,,Path Integrals in Quantum Mechanics, Statistics, Polymer
%   Physics, and Financial Markets'', \cite{Kle06}.
% \end{center}

% Uwagi:\\
% \begin{itemize}
% \item[--] \textbf{Str. 3.}
%   %%   \begin{displaymath}
%   %%   \frac{ \hbar^{ 2 } \mathbf{k}_{ 2 }^{ 2 } }{ 2 m } - \frac{
%   %%   \hbar^{ 2 } \mathbf{k}_{ 1 }^{ 2 } }{ 2 m } = \frac{ \hbar^{ 2
%   %%     %% } }{ 2 m } \frac{ ( 2 n_{ x } + 1 ) }{ L^{ 2 } }.
%   %% \end{displaymath}
%   %% Przy ustalonym $L$ ta wielkość jest dowolnie duża dla odpowiednio
%   %% wysokiego $n_{ x }$. Odległości między sąsiednimi stanami można
%   %% uważać, za małe tylko jeśli mamy górne ograniczenie na energię,
%   %% mówiąc inaczej jeśli w~układzie mamy energię Fermiego.
% \end{itemize}

% \begin{large}
%   \textbf{Błędy:}\\
% \end{large}
% \begin{tabular}{|c|c|c|c|c|}
%   \hline
%   & \multicolumn{2}{c|}{} & & \\
%   Strona & \multicolumn{2}{c|}{Wiersz} & Jest & Powinno być \\ \cline{2-3}
%   & Od góry & Od dołu &  &  \\ \hline
%   & & & & \\
%   %%   17 & 12 & & 1904 & 1905 \\
%   %%   31 & & 4 & jakakolwiek & taka \\
%   %%   35 & 9 & & $| \psi( \mathbf{r}, t |^{ 2 }$ &
%   %%   $| \psi( \mathbf{r},
%   %%   t ) |^{ 2 }$ \\
%   %%   42 & 2 & & ograniczone & zlokalizowane \\
%   %%   54 & 16 & & wartości & dyskretne wartości \\
%   %%   54 & & 2 & dwom & dwum \\
%   %%   62 & & 8 & $z + z'$ & $z - z'$ \\
%   & & & & \\ \hline
% \end{tabular}
% %% \newline
% %% \noindent\\
% %% \textbf{Str. 21, wiersz 16.}\\
% %% \textbf{Jest:} orbity**.Równanie(3.3)implikuje\ldots \\
% %% \textbf{Powinno być:} orbity**. Równanie (3.3) implikuje\ldots \\
% %% \textbf{Str. 54, wiersz 18.}\\
% %% \textbf{Jest:} w~odpowiadających im punktach\ldots \\
% %% \textbf{Powinno być:} w~obszarze przez te ścianki zajętym\ldots
% %% \\




% \begin{center}
%   Walter Thirring\\
%   ,,Fizyka matematyczna. Tom I: Klasyczne układy dynamiczne'',
%   \cite{WTFMI}.
% \end{center}
%
%
%% Uwagi:
%% \begin{itemize}
%% \item Str. 18. W Definicji (2.1,1) powinno być założenie o
%%   istnieniu wektora zerowego.
%% \item
%% \end{itemize}
%
%
% Błędy:\\
% \begin{tabular}{|c|c|c|c|c|}
%\hline
%& \multicolumn{2}{c|}{} & & \\
%Strona & \multicolumn{2}{c|}{Wiersz} & Jest & Powinno być \\ \cline{2-3}
%& Od góry & Od dołu &  &  \\ \hline
%& & & & \\
%23 & 17 & & $x_{ 1 }, x_{ 2 } )$ & $( x_{ 1 }, x_{ 2 } )$ \\
%29 & & 2 & $\partial M = \{ a \} \cup \{ b \}, U_{ 2 } = ( a, b ],$ & $U_{ 2 } = ( a, b ], \Phi_{ 2 } : x \rightarrow b - a,$ \\
%& & &  $\Phi_{ 2 } : x \rightarrow b - a.$ & $\partial M = \{ a \} \cup \{ b \}.$ \\
%30 & 3 & & $\Phi_{ 1 } = :$ & $\Phi_{ 1 } :$ \\
%37 & 10 & & $\frac{ ( x_{ 1 } v_{ 1 } - x_{ 2 } v_{ 2 }}{ x_{ 1 }^{ 2 } + x_{ 2 }^{ 2 } }$ & $\frac{ ( x_{ 1 } v_{ 1 } - x_{ 2 } v_{ 2 } ) }{ x_{ 1 }^{ 2 } + x_{ 2 }^{ 2 } }$ \\
%37 & 14 & & $`T( M ) \times T( M )$ & $T( M ) \times T( M )$ \\
%37 & & 3 & $( q, v ) \rightarrow v$ & $( q, v )$ \\
%38 & 14 & & Wiązka ta jest trywializowalna, & Jeśli wiązka ta jest trywializowalna, \\
%& & & jeśli rozmaitość $X$ jest paralelyzowalna. & to rozmaitość jest pralelyzowalna. \\
%43 & & 9 & $\mathbf{ 1 \, \times \, jednostkowy\; wektor } $ & $\mathbf{ 1 \, \equiv \, jednostkowy\; wektor }$ \\
%& & & $\mathbf{ styczny }$ & $\mathbf{ styczny }$ \\
%43 & & 5 & $\mathbf{ 1 } \times$ wektor jednostkowy & $\mathbf{ 1 } \equiv$ wektor jednostkowy \\
%43 & & 4 & $( q_{ i }, x_{ i }( q ) )$ & $( q_{ i }, X_{ i }( q ) )$ \\
%45 & 1 & & $\mathbf{R}^{ n } \setminus \mathbf{R} \times \{ 0, 0, \ldots, 0 \}$ & $\mathbf{R}^{ n } \setminus \mathbf{R}$ \\
%48 & & 9 & $T( T( M )$ & $T( T( M ) )$ \\
%52 & & 1 & $T_{ q }( M )$ & $T_{ q }( M )$) \\
%53 & 11 & & $L_{ \Theta^{ -1 }_{ C } }( q ) e_{ j } dq^{ i }$ & $L_{ \Theta^{ -1 }_{ C }( q ) e_{ j } } q^{ i }$ \\
%74 & & 16 & & rozmaitości orientowalnych \\
%75 & 9 & & $\mathrm{sup} \; f_{ i }$ & $\mathrm{supp} \; f_{ i }$ \\
%75 & & 10 & do $\mathrm{sup} \; f$ & od $\mathrm{supp} \; f$ \\
%77 & & 8 & $x^{ 2 } + x^{ 2 }$ & $x^{ 2 } + y^{ 2 }$ \\
%& & & & \\ \hline
%\end{tabular}
%
%
%
%\begin{center}
%  Walter Thirring\\
%  ,,Fizyka matematyczna. Tom III: Mechanika kwantowa atomów i
%  cząsteczek.'', \cite{WTFMIII}.
%\end{center}
%
%Uwagi:
%\begin{itemize}
%\item Str. 18. W Definicji (2.1,1) powinno być założenie o istnieniu
%  wektora zerowego.
%\item
%\end{itemize}
%
%Błędy:\\
%\begin{tabular}{|c|c|c|c|c|}
%\hline
%& \multicolumn{2}{c|}{} & & \\
%Strona & \multicolumn{2}{c|}{Wiersz} & Jest & Powinno być \\ \cline{2-3}
%& Od góry & Od dołu &  &  \\ \hline
%& & & & \\
%19 & 15 & & praktyczne.Moc & praktyczne. Moc \\
%%& & & & \\
%%& & & & \\
%%& & & & \\
%& & & & \\ \hline
%\end{tabular}
%
%
%

