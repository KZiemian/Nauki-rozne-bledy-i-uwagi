% Autor: Kamil Ziemian

% ---------------------------------------------------------------------
% Podstawowe ustawienia i pakiety
% ---------------------------------------------------------------------
\RequirePackage[l2tabu, orthodox]{nag}  % Wykrywa przestarzałe i niewłaściwe
% sposoby używania LaTeXa. Więcej jest w l2tabu English version.
\documentclass[a4paper,11pt]{article}
% {rozmiar papieru, rozmiar fontu}[klasa dokumentu]
\usepackage[MeX]{polski}  % Polonizacja LaTeXa, bez niej będzie pracował
% w języku angielskim.
\usepackage[utf8]{inputenc}  % Włączenie kodowania UTF-8, co daje dostęp
% do polskich znaków.
\usepackage{lmodern}  % Wprowadza fonty Latin Modern.
\usepackage[T1]{fontenc}  % Potrzebne do używania fontów Latin Modern.



% ------------------------------
% Podstawowe pakiety (niezwiązane z ustawieniami języka)
% ------------------------------
\usepackage{microtype}  % Twierdzi, że poprawi rozmiar odstępów w tekście.
% \usepackage{graphicx}  % Wprowadza bardzo potrzebne komendy do wstawiania
% % grafiki.
% \usepackage{verbatim}  % Poprawia otoczenie VERBATIME.
% \usepackage{textcomp}  % Dodaje takie symbole jak stopnie Celsiusa,
% % wprowadzane bezpośrednio w tekście.
\usepackage{vmargin}  % Pozwala na prostą kontrolę rozmiaru marginesów,
% za pomocą komend poniżej. Rozmiar odstępów jest mierzony w calach.
% ------------------------------
% MARGINS
% ------------------------------
\setmarginsrb
{ 0.7in}  % left margin
{ 0.6in}  % top margin
{ 0.7in}  % right margin
{ 0.8in}  % bottom margin
{  20pt}  % head height
{0.25in}  % head sep
{   9pt}  % foot height
{ 0.3in}  % foot sep



% ------------------------------
% Często używane pakiety
% ------------------------------
% \usepackage{csquotes}  % Pozwala w prosty sposób wstawiać cytaty do tekstu.
\usepackage{xcolor}  % Pozwala używać kolorowych czcionek (zapewne dużo
% więcej, ale ja nie potrafię nic o tym powiedzieć).


% ------------------------------
% Pakiety do tekstów z nauk przyrodniczych
% ------------------------------
\let\lll\undefined  % Amsmath gryzie się z językiem pakietami do języka
% polskiego, bo oba definiują komendę \lll. Aby rozwiązać ten problem
% oddefiniowuję tę komendę, ale może tym samym pozbywam się dużego Ł.
\usepackage[intlimits]{amsmath}  % Podstawowe wsparcie od American
% Mathematical Society (w skrócie AMS)
\usepackage{amsfonts, amssymb, amscd, amsthm}  % Dalsze wsparcie od AMS
% \usepackage{siunitx}  % Do prostszego pisania jednostek fizycznych
\usepackage{upgreek}  % Ładniejsze greckie litery
% Przykładowa składnia: pi = \uppi
% \usepackage{slashed}  % Pozwala w prosty sposób pisać slash Feynmana.
\usepackage{calrsfs}  % Zmienia czcionkę kaligraficzną w \mathcal
% na ładniejszą. Może w innych miejscach robi to samo, ale o tym nic
% nie wiem.



% ------------------------------
% Pakiety napisane przez użytkownika.
% Mają być w tym samym katalogu to ten plik .tex
% ------------------------------
\usepackage{latexgeneralcommands}
\usepackage{mathcommands}

\usepackage{tensor}



% ---------------------------------------------------------------------
% Dodatkowe ustawienia dla języka polskiego
% ---------------------------------------------------------------------
\renewcommand{\thesection}{\arabic{section}.}
% Kropki po numerach rozdziału (polski zwyczaj topograficzny)
\renewcommand{\thesubsection}{\thesection\arabic{subsection}}
% Brak kropki po numerach podrozdziału



% ------------------------------
% Ustawienia różnych parametrów tekstu
% ------------------------------
\renewcommand{\arraystretch}{1.2}  % Ustawienie szerokości odstępów między
% wierszami w tabelach.





% ------------------------------
% Pakiet „hyperref”
% Polecano by umieszczać go na końcu preambuły.
% ------------------------------
\usepackage{hyperref}  % Pozwala tworzyć hiperlinki i zamienia odwołania
% do bibliografii na hiperlinki.










% ---------------------------------------------------------------------
% Tytuł, autor, data
\title{QFT~-- błędy i~uwagi}

% \author{}
% \date{}
% ---------------------------------------------------------------------










% ####################################################################
% Początek dokumentu
\begin{document}
% ####################################################################





\begin{center}
Jacek Gancarzewicz\\
,,Algebra liniowa i jej zastosowania'',\\ wydanie \romannumeral1, JG.
\end{center}
Jest źle:
\begin{itemize}
\item W pierwszym rozdziale niezdefiniowano znaku permutacji \\identycznościowej.
\item Nie zdefiniowano operacji odejmowania wektorów.
\item Problem sumowania po pustym zbiorze wskaźników nie został omówiony.
\item
\item Str. 60. W ostatniej permutacji trzeba jedną 7 zastąpić 4.
\item Str. 76. Błąd w numeracji punktów twierdzenia.
\item Str.
\item Str. 341. W twierdzeniu 45.1 powinno być $n \geq 1$. Błąd jest poważny.
\end{itemize}
Powinno być:
\begin{itemize}
\item[--] Str. 28. $x\neq y$.
\item[--] Str. 41. $z^2+az+b$.
\item[--] Str. 59. $(x_{1},\ldots,x_{k})\leq (y_{1},\ldots,y_{k})$.
\item[--] Str. 76. Na podstawie twierdzenia 1.2 punkt 1.
\item[--] Str. . $a^{-1} 0=0$.
\item[--] Str. 86. \ldots z zasady kontrapozycji.
\item[--] Str. 110. $(a_{0},a_{1},a_{2},\ldots)$.
\item[--] Str. 143. $\alpha\in U^{*}$. $f'(e'_{j})=\sum_{j=1}^{m}a_{ji}\overset{\_}{e}'_{j}$.
\item[--] Str. 166. Twierdzenie 12.14.
\item[--] Str. 167. $Y^{*}$. $(a_{i,1},...,a_{i,n})\in F^{n}$. Dla każdego ciągu $(x_{1},...,x_{n})\in F^{n}$.
\item[--] Str. 195. $\mathcal{J}_{1}: F\otimes X\rightarrow X$ i $\mathcal{J}_{2}: X\otimes F\rightarrow X$.
\item[--] Str.
\item[--] Str.
\item[--] Str.
\item[--] Str. 354. Zadanie \romannumeral6.7a  zostało rowiązane w twierdzeniu 44.11.
\item[--] Str.
\item[--] Str.
\end{itemize}



\begin{center}
Jacek Gancarzewicz\\
,,Zarys współczesnej geometrii różniczkowej'',\\ wydanie \romannumeral1, GR.
\end{center}
Jest źle:
\begin{itemize}
\item
%\item
%\item
%\item
%\item
%\item
%\item
%\item
%\item
%\item
\end{itemize}
Powinno być:
\begin{itemize}
\item[--] Str. 11. \ldots$ax+by\in V$.
\item[--] Str. 12. \ldots jest kombinacją liniową pozostałych wektorów\ldots
\item[--] Str. 15. \ldots wektorów $y_{1},\ldots,y_{n}\in Y$ istnieje\ldots
\item[--] Str. 18. $f_{1},\ldots,f_{k}$
\item[--] Str. 19. \ldots$f:X\times\ldots\times X \rightarrow Y$\ldots
\item[--] Str. 19.
\begin{align*}
\mathrm{dim}\, L(X_{1},\ldots,X_{k};Y)&=\mathrm{dim}\, L(X_{1},\ldots,X_{k})\; \mathrm{dim}\, Y,\\
\mathrm{dim}\, L^{k}_{s}(X;Y)&=\mathrm{dim}\,L^{k}_{s}(X;Y)\; \mathrm{dim}\, Y,\\
\mathrm{dim}\, L^{k}_{a}(X;Y)&=\mathrm{dim}\,L^{k}_{a}(X;Y)\; \mathrm{dim}\, Y.\\
\end{align*}
\item[--] Str. 43. \ldots wszystkich $\alpha,\beta,\gamma,\rho=1,\ldots,n$ zachodzą \ldots
\item[--] Str. 46. $\pi_{s}^{k}=\pi_{s}^{r}\circ \pi^{k}_{r}$.
\item[--] Str. 48. $\pi^{r}_{s}$.
\item[--] Str. 48. \ldots na def\mbox{}inicję\ldots
\item[--] Str. 49. \ldots$(a_{1},a_{2},a_{3},\ldots)+(b_{1},b_{2},b_{3},\ldots)=(a_{1}+b_{1},a_{2}+b_{2},a_{3}+b_{3},\ldots)$\dots
\item[--] Str. 49. \ldots$\rho(b)=(p^{1}(b),p^{2}(b),p^{3}(b),\ldots)$\dots
\item[--] Str. 51. \ldots oraz odwzorowań ciągłych\ldots
\item[--] Str. 62. $\varphi_{U}=\varphi\circ\mathrm{id}_{U}$.
\item[--]
\item[--]
\item[--]
\end{itemize}


\begin{center}
Laurent Schwartz\\
,,Kurs analizy matematycznej'',\\ wydanie \romannumeral1.
\end{center}
Jest źle:
\begin{itemize}
\item Str. 10. Zbyt restrykcyjne zdef\mbox{}iniowanie różnicy zbiorów.
\item Str. 11. W definicji funkcji nie został poruszony problem jednoznaczego przyporządkowania wartości.
\item Str. 17. Jeżeli $f$ nie jest suriekcją to rodzina zbiorów postaci $f^{-1}(\{z\})$ dalej stanowi rozwarstwienie przestrzeni $E$, z tą różnicą, że zawiera teraz zbiór $\emptyset$. Ponadto dla suriekcji, jeżeli $x \neq y$ to $f^{-1}(\{x\})\neq f^{-1}(\{y\})$, teraz oczywiście może zajść $f^{-1}(\{x\})= f^{-1}(\{y\})=\emptyset$. W szczególności $f^{-1}$ nie musi być odwzorowaniem na zbiór klas abstrakcji. W tej sytuacji nie można uważać $F$ za ,,model'' dla $E/R$.
\item Funkcja $f(x)=\frac{\log(x)}{1-x}$ nie jest bijekcją odcinka ]0,1[ na $\mathbb{R}$, bowiem w całym tym przedziale $f(x)<-1$.
\item Str. 53. Ostatnie zdanie przypis 5 jest napisany zbyt lekomyślnie, sugeruje bowiem, że dla zbierzności danej funkcji nie ma znaczenia czy $a\in A$, czy też $a\notin A$. W rzeczywistości jest wręcz przeciwnego.
\item
\item
\end{itemize}
Powinno być:
\begin{itemize}
\item[--] Str. 13. \ldots można mówić o zbiorze $f^{-1}(\{z\})$ \ldots
\item[--] Str. 14. Ponadto zachodzi zawsze\ldots
\item[--] Str. 16. \ldots $f_{3}$ jest obrazem funkcji $f$\ldots
\item[--] Str. 25. Nie oczywiste są tylko wzory (1.5.1).
\item[--] Str. 31. \ldots bo $\mathbb{Q}$ jest przeliczalne.
\item[--] Str. 37. $||\vec{0}||=0$.
\item[--] Str. 41. W rzeczywistości własność (a$^{\prime\prime}$)\ldots
\item[--] Str. 43. \ldots jest to dopełnienie zewnętrza tego zbioru $A$ lub\ldots
\item[--] Str. 44. \ldots przeliczalną część gęstą.
\item[--] Str. 46. \ldots dopełnienia początku współrzędnych do prostej rzeczywistej\ldots
\item[--] Str. 50. \ldots określona przez $d^{\prime} (x,y)=\inf(d(x,y),1)$\ldots
\item[--] Str. 53. \ldots jest przestrzenią metryczną, zbieżność ciągu\ldots
\item[--] Str. 56. \ldots zbiorów z $E_{1}$ i $E_{2}$. ?
\item[--]
\item[--]
\item[--]
\item[--]
\end{itemize}



\begin{center}
Kazimierz Kuratowski\\
,,Wstęp do teorii mnogości i topologii'',\\ wydanie \romannumeral9, KK.
\end{center}
Jest źle:
\begin{itemize}
\item Brak ,,gramatyki'' pisania zdań złożonych logicznych. Jest to jednak usprawiedliwione względami dydaktycznymi.
\item Nie wspomniano o rozdzielności dodawania zbiorów względem mnożenia. Jest to w zadaniach.
\item Str. 10. Wszystkie dowody ,,nie wprost'', są przypisane regule \\ kontrapozycji.
\end{itemize}



































































































































































































































































































































































































































































































































































































































































































































































































































































































































































































































































% ####################################################################
% ####################################################################
% Bibliografia
\bibliographystyle{plalpha}

\bibliography{PhilNaturBooks}{}





% ############################

% Koniec dokumentu
\end{document}