\documentclass[a4paper,11pt]{article}
\usepackage[polish]{babel}% Tłumaczy na polski teksty automatyczne LaTeXa i pomaga z typografią.
\usepackage[plmath,OT4,MeX]{polski}% Polska notacja we wzorach matematycznych. Ładne polskie
\usepackage[T1]{fontenc}% Pozwala pisać znaki diakrytyczne z języków innych niż polski.
\usepackage[utf8]{inputenc}% Pozwala pisać polskie znaki bezpośrednio.
\usepackage{indentfirst}% Sprawia, że jest wcięcie w pierwszym akapicie.
\frenchspacing% Wyłącza duże odstępy na końcu zdania. Pakiet babel polski robi to samo, ale to jest %zabezpieczenie jakibym chciał przestać go używać.
\usepackage{hyperref}
\usepackage{fullpage}% Mniejszse marginesy.
\usepackage{vmargin}
%----------------------------------------------------------------------------------------
%	MARGINS
%----------------------------------------------------------------------------------------
\setmarginsrb           { 0.7in}  % left margin
                        { 0.6in}  % top margin
                        { 0.7in}  % right margin
                        { 0.8in}  % bottom margin
                        {  20pt}  % head height
                        {0.25in}  % head sep
                        {   9pt}  % foot height
                        { 0.3in}  % foot sep
%\usepackage{amsfonts}% Czcionki matematyczne od American Mathematic Society.
%\usepackage{amsmath}% Dalsze wsparcie od AMS. Więc tego, co najlepsze w LaTeX, czyli trybu
%matematycznego.
%\usepackage{amscd}% Jeszcze wsparcie od AMS.
%\usepackage{latexsym}% Więcej symboli.
%\usepackage{textcomp}% Pakiet z dziwnymi symbolami.
%\usepackage{xy}% Pozwala rysować grafy.
%\usepackage{tensor}% Pozwala prosto używać notacji tensorowej. Albo nawet pięknej notacji
%tensorowej:).
\usepackage{graphicx}% Pozwala wstawiać grafikę.
%\usepackage{url}% Pozwala pisać ładnie znak ~.
\title{Tytuł: jak wam się podoba (wy~ustalcie).}
\author{}
\date{}

\begin{document}

\maketitle

\begin{center}
\Large{\textbf{Opis.}}
\end{center}

Książka W. Arnolda ,,Metody matematyczne mechaniki klasycznej'' stanowi dzięki głębi swych przemyśleń, naciskowi na~intuicyjne uzasadnienie stosowanego formalizmu, matematycznenu wyrafionowaniu i~niezwykłej erudycji autora jedno z~najlepszych dzieł jakie w~XX wieku napisano na temat mechaniki klasycznej. Równoczesnie jest to pozycja bardzo trudna, w~której elementarne zagadnienia sąsiadują z~bardzo skomplikowanymi problemami, wiele kluczowych rozumowań jest tylko naszkicowanych, a~pełne jej zrozumienie wymaga pewnego obycie ze~współczesną matematyką.

Celem tych spotkań jest wspólne przestudiowanie i~zrozumienie tego dzieła na cotygodniowych dwugodzinnych spotkaniach. W~zależności od preferencji uczestników forma tych spotkań może przybrać postać wykładów prowadzonych przez \ldots na podstawie poszczególnych rozdziałów, bądź referowania przez zainteresowane osoby kolejnych fragmentów książki. W~obu wypadkach wspólna dyskusja przedstawionego materiału ma być centraną częścią spotkań.

\begin{center}
\Large{\textbf{Plan.}} \\
\end{center}
\noindent
Poniżej jest lista części książki, które będą przerabiane na spotkaniach. \\ \newline
\noindent
\textbf{Uwagi.}
\begin{itemize}
\item[--] Choć pierwsza książki traktuje o~mechanice w~sformułowaniu Newtona, co nie pokrywa się z planem kursu ,,Mechanika klasyczna'' profesora Bizonia, to wprowadzone jest w~niej kilka kluczowych dla całej książki koncepcji, dlatego też jej część znalazła~się w~planie spotkań.
\item[--] Gwiazdką (*) oznaczone są paragrafy których treść można w~całości, albo w~większości, pominąć bez szkody dla dalszej części książki, jednak ze względu na ciekawy materiał warto rozważyć ich przerobienie w całości.
\item[--] Plan może ulec zmianie w~czasie trwania spotkań.
\item[--] Jeżeli ktoś chce zgłosić zastrzeżenie do tego planu prosze pisać na adres \ldots .
\end{itemize}
\begin{itemize}
\item[\textbf{Roz. I.}] \textbf{Fakty doświadczalne.}
\item[--] 1. Zasada względności i przyczynowości.
\item[--] 2. Grupa Galileusz i~równania Newtona.
\item[\textbf{Roz. II.}] \textbf{Badanie równań ruchu.}
\item[--] 4. Układy o~jednym stopniu swobody.
\item[--] 5. Układy o~dwóch stopniach swobody.
\item[--] 11*. Rozumowanie oparte na podobieństwie.
\item[\textbf{Roz. III.}] \textbf{Zasada wariacyjna (całość).}
\item[--] 12. Rachunek wariacyjny.
\item[--] 13. Równanie Lagrange'a.
\item[--] 14. Przekształcenie Legendre'a.
\item[--] 15. Równania Hamiltona.
\item[--] 16. Twierdzenie Liouville'a.
\item[\textbf{Roz. IV.}] \textbf{Mechanika Lagrange'a na rozmaitościach.}
\item[--] 17. Więzy holonomiczne.
\item[--] 18. Rozmaitości różniczkowalne.
\item[--] 19. Układy dynamiczne Lagrange'a.
\item[--] 20. Twierdzenie E. Noether.
\item[\textbf{Roz. V.}] \textbf{Drgania.}
\item[--] 22. Linearyzacja.
\item[--] 23. Małe drgania.
\item[--] 24*. O~zachowaniu się częstości własnych.
\item[--] 25*. Rezonans parametryczny.
\item[\textbf{Roz. VII.}] \textbf{Formy różniczkowe (całość).}
\item[--] 32. Formy zewnętrzne.
\item[--] 33. Iloczyn zewnętrzny.
\item[--] 34. Formy różniczkowe.
\item[--] 35. Całkowanie form różniczkowych.
\item[--] 36. Różniczkowanie zewnętrzne.
\item[\textbf{Roz. VIII.}] \textbf{Rozmaitości symplektyczne.}
\item[--] 37. Struktura symplektyczna na rozmaitości.
\item[--] 38. Hamiltonowskie potoki fazowe i~ich niezmienniki całkowe.
\item[--] 39. Algebry Liego pól wektorowych.
\item[--] 40. Algebra Liego pól Hamiltona.
\item[--] 41. Geometria symplektyczna.
\item[--] 42*. Rezonans parametryczny w~układach o~wielu stopniach swobody.
\item[--] 43. Atlas symplektyczny.
\item[\textbf{Roz. IX.}] \textbf{Formalizm kanoniczny.}
\item[--] 44*. Niezmiennik całkowy Poincar\'{e}go\dywiz Cartana.
\item[--] 45. Konsekwencje twierdzenia o~niezmienniku całkowym Poincar\'{e}go\dywiz Cartana.
\item[--] 46*. Zasada Huygensa.
\item[--] 47*. Metoda Jacobiego\dywiz Hamiltona całkowania równań kanonicznych Hamiltona.
\item[--] 48. Funkcje tworzące.
\item[\textbf{Roz. IX.}] \textbf{Wprowadzenie do teorii zaburzeń.}
\item[--] 49*. Układy całkowalne.
\item[--] 50*. Współrzędne działanie\dywiz kąt.
\item[--] 51*. Uśrednianie.
\item[--] 52*. Uśrednianie zaburzeń
\item[] \textbf{Uzupełnienia.}
\item[--] 5*. Układy dynamiczne wykazujące symetrię.
\item[--] 8*. Teoria zaburzeń dla~ruchów prawie okresowych i~twierdzenie Kołmogorowa.
\item[--] 12*. Osobliwości Lagrange'a.
\item[--] Na co nam jeszcze starczy sił ;). O~ile w~ogóle tu dotrzemy\ldots
\end{itemize}

\begin{center}
\Large{\textbf{Bibliografia.}}
\end{center}

\noindent
Poniższa lista zawiera pozycje zarówno skierowane zarówno dla osób które chcą przeczytać szersze opracowanie niektórych zagadnień, jak i~tych które chcą poznać w~jaki sposób można uogólnić omawiane zagadnienia. Niestety dla wszystkich omawianych problemów nie udało~się znaleźć zadowalającej\linebreak literatury, pewnych zaś wartych uwagii pozycji nie~umieszczono na niej, jako mało adekwentnych do~treści spotkań.
\newline
\noindent
\textbf{BWMiI} -- Biblioteka Wydziału Matematyki i Informatyki. \\
\newline
\noindent
\textbf{Cykl W. Arnolda.} Książki te optymalnie byłoby czytać w~podanej poniżej kolejności, stąd obecność tu omawianego na spotkaniach dzieła.
\begin{itemize}
\item[--] \emph{Równania różniczkowe zwyczajne} (RRZ), BWMiI.
\item[--] \emph{Metody matematyczne mechaniki klasycznej} (MMMK), BWMiI. %Biblioteka Wydziału Matematyki i Informatyki.
\item[--] \emph{Teoria równań różniczkowych} (TRR), BWMiI.
\end{itemize}

%\textbf{Mechanika klasyczna.}
\begin{itemize}
\item[] \textbf{Podstawy matematyczne.}
\item[--] L. Schwartz, \emph{Kurs analizy matematycznej, tom I} (LSI), większość bibliotek np. NKFu. Książka trudna, ale zawiera dowód chyba każdego twierdzenia z~analizy jakie będzie potrzebne.
\item[--] A. Herdegen, \emph{Algebra liniowa i~geometria} (AH). Głównie twierdzenia o~formach kwadratowych\footnote{Jedyną inną pozycją, która o~ile wiem zawiera dowody potrzebnych twierdzeń, jest książka ,,Wykłady z~algebry liniowej'' I. M. Gelfanda.}.
\item[] \textbf{Struktura czasoprzestrzeni mechaniki Newtona.}
\item[--] W. Kopczyński, A. Trautman, \emph{Czasoprzestrzeń i grawitacja} (KT), biblioteki FAISu, NKFu etc. Dobra, krótka i~trudna pozycja, jedna z~niewielu które zajmują~się tym tematem.
\item[] \textbf{Mechanika klasyczna.}
\item[--] E. T. Whittaker, \emph{Dynamika analityczna} (ETW), biblioteka NKFu. Wiekowa, lecz wciąż warta uwagii pozycja.
\item[--] R. Abraham, J. E. Marsden, \emph{Foundations of Mechanics, Second Edition} (FoM2), \url{http://authors.library.caltech.edu/25029/} . Monumentalne dzieło starające~się z~pełną ścisłością przedstawić mechanikę za pomocą metod współczesnej matematyki.
\item[] \textbf{Równania różniczkowe zwyczajne.}
\item[--] W. Walter, \emph{Ordinary differential equations} (WWODEs), Springer Link. Rozsądny, współczesny wykład podstaw teorii ODEs.
\item[--] E. Hairer, S. P. N\o rsett, G. Warner, \emph{Solving Ordinary Differential Equations} (SODEs), Springer Link. Monumentlane dzieło o~tym jak analitycznie, a~przedewszystkim numerycznie rozwiązać dane równanie. 
\item[] \textbf{Rachunek wariacyjny.}
\item[--] I. M. Gelfand, S. V. Fomin, \emph{Rachunek wariacyjny} (GF), większość bibliotek, np. NKFu i FAISu. Standardowy wykład klasycznych osiągnięć rachunku wariacyjnego.
\item[--] J. Jost, X. Li-Jost, \emph{Calculus of Variations} (JLJ), BWMiI. Podręcznik zawierający wprowadzenie do~wpółczesnych metod w~rachunku wariacyjnym.
\item[--]  M. Giaquinta, St. Hildebrandt, \emph{Calculus of Variations} (GHCoV), Springer Link. Monografia starająca~się dać możlwie wyczerpujący opis współczesnych metod.
\item[] \textbf{Geometria różniczkowa.}
\item[--] J. Gancarzewicz, \emph{Zarys współczesnej geometrii różniczkowej} (ZWGR). Abstarakcyjna, długa i~niepozbawiona sporych błędów, jednak bardzo dobra pozycja dla średnio zawansowanych.
\item[--] R. Sulanke, P. Wintgen, \emph{Geometria różniczkowa i~teoria wiązek} (SW), BWMiI. Pozycja wprowadzająca, zawierająca dobre wprowadzenie do teorii tożsamości geometryczno\dywiz całkowych na~rozmaitościach.
\item[] \textbf{Teoria form.}
\item[--] L. Schwartz, \emph{Kurs analizy matematycznej, tom II} (LSII), BWMiI. Bez znajomości teorii całki\linebreak z~I tomu, prawie nie do zrozumienia.
\item[--] S. G. Krantz, H. R. Parks, \emph{Geometric Integration Theory} (GIT), Springer Link. Zawiera wprowadzenie do teorii prądów, pozwalajacej rozważać formy o~wartościach w~dystrybucjach.
\item[] \textbf{Geometria symplektyczna i mechanika.}
\item[--] P. Libermann, Ch.\dywiz M. Marle, \emph{Symplectic Geometry and Analytical Mechanics} (SGAM), BWMiI, Springer Link. Wykład geometrii syplektycznej ilustrowany zastosowaniami w mechanice.
\newpage
\item[] \textbf{Układy nieholonomiczne.}
\item[--] J. I. Nejmark, N. A. Fufajew, \emph{Dynamika układów nieholonomicznych} (DUN), Allegro\footnote{Prostszy sposób nie jest znany.}. Podstawowe, w~dobry sposób staroświeckie, dzieło w~tej dziedzinie.
\item[--] E. Massa, E. Paganim \emph{A new look at classical mechanics of constrained systems} (EMEP),\newline \url{https://eudml.org/doc/76747} . Nowoczesna próba zmierzenia~się z~zagadieniem więzów nieholonomicnzych. Dość trudna pozycja.
\item[--] H. Geiges \emph{Contact geometry} (HGCM), arXiv:math/0307242v2 [math.SG] \url{http://arxiv.org/abs/math/0307242} . Można tu znaleść dobre, jak na ten dział matematyki, wprowadzenie w~teorię rozmaitości kontaktowych, podstawowę matematycznego opisu układów nieholonomicznych.
\item[] \textbf{Geometria różniczkowa poza mechaniką.}
\item[--] G. Svetlichny, \emph{Preparation for Gauge Theory} (PGT), arXiv:math-ph/9902027v3 \url{http://arxiv.org/abs/math-ph/9902027} . Czasami trochę zbyt zwięzły, lecz merytorycznie bardzo dobry, wykład na temat zastosowania geometrii różniczkowej, i~pochodnych działów matematyki, do opisu \textbf{klasycznych} teorii pola z~cechowaniem, takich jak elektrodynamika, czy pola Yanga\dywiz Millesa.
\end{itemize}

\textbf{Ważne:}
\begin{quote}
Dla prawdziwego matematyka, jest dużo ważniejsze by wiedzieć jakie problemy nie zostały wciąż rozwiązane i~gdzie znane obecnie metody okazały~się niewystarczające, niż pamiętać wszystkie liczby których iloczyny udało~się do tej pory uzyskać, czy orientować się w~ ocenianie literatury stowrzonej na przestrzeni ostatnich 20 tysięcy lat.
\end{quote}
W. Arnold w~przedmowie do książki ,,Arnold's Problems'', Springer Link, tłumaczenie swobodne.\\
\begin{quote}
Zauważyliśmy bowiem, że dla początkujących słuchaczy dużą przeszkodę w
zdobywaniu tej nauki stanowią dzieła różnych teologów: już to dlatego, że są nadmiernie
przeładowane bezużytecznymi zagadnieniami, artykułami i dowodami, już to dlatego, że
zagadnienia, z jakimi owi początkujący winni koniecznie się zapoznać, nie są podane
systematycznie: według uporządkowanej kolejności nauk czy traktatów, ale omawiane są
albo w związku z komentowaniem dzieł, albo z okazji dysputy; już to wreszcie
dlatego, że częste wałkowanie tego samego budziło w ich umysłach nudę i zamęt.

Ufni w pomoc Bożą i starając się uniknąć tych i podobnych niedociągnięć,
będziemy usiłowali krótko i jasno - o ile na to sama rzecz pozwoli - wyłożyć wszystko, co zakresem swoim obejmuje nauka święta.
\end{quote}
Św. Tomasz z~Akwinu we wstępie do \emph{Sumy Teologicznej}, \url{http://www.katedra.uksw.edu.pl/suma/suma_1.pdf}.

\bibliographystyle{ieeetr}
\bibliography{BibliographyBPL}{}



\end{document}