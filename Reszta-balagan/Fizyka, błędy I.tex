\documentclass[a4paper,11pt]{article}
\usepackage[polish]{babel}% Tłumaczy na polski teksty automatyczne LaTeXa i pomaga z typografią.
\usepackage[plmath,OT4,MeX]{polski}% Polska notacja we wzorach matematycznych. Ładne polskie
\usepackage[T1]{fontenc}% Pozwala pisać znaki diakrytyczne z języków innych niż polski.
\usepackage[utf8]{inputenc}% Pozwala pisać polskie znaki bezpośrednio.
\usepackage{indentfirst}% Sprawia, że jest wcięcie w pierwszym akapicie.
\frenchspacing% Wyłącza duże odstępy na końcu zdania. Pakiet babel polski robi to samo, ale to jest %zabezpieczenie jakibym chciał przestać go używać.
\usepackage{fullpage}% Mniejszse marginesy.
\usepackage{amsfonts}% Czcionki matematyczne od American Mathematic Society.
\usepackage{amsmath}% Dalsze wsparcie od AMS. Więc tego, co najlepsze w LaTeX, czyli trybu
%matematycznego.
\usepackage{upgreek}%Lepsze greckie czcionki. Przyklad skladni: pi = \uppi
%\usepackage{txfonts}%Inne ulepszenie greckich liter. Przyklad skladni: pi = \piup
\usepackage{amscd}% Jeszcze wsparcie od AMS.
\usepackage{latexsym}% Więcej symboli.
\usepackage{textcomp}% Pakiet z dziwnymi symbolami.
\usepackage{slashed}% Pozwala pisać slash Feynmana.
%\usepackage{undertilde}
\usepackage{xy}% Pozwala rysować grafy.
\usepackage{tensor}% Pozwala prosto używać notacji tensorowej. Albo nawet pięknej notacji
%tensorowej:).
\usepackage{vmargin}
%----------------------------------------------------------------------------------------
%	MARGINS
%----------------------------------------------------------------------------------------
\setmarginsrb           { 0.7in}  % left margin
                        { 0.6in}  % top margin
                        { 0.7in}  % right margin
                        { 0.8in}  % bottom margin
                        {  20pt}  % head height
                        {0.25in}  % head sep
                        {   9pt}  % foot height
                        { 0.3in}  % foot sep
\usepackage{graphicx}% Pozwala wstawiać grafikę.
%\usepackage{url}% Pozwala pisać ładnie znak ~.
\def\mathbi#1{\textbf{\em #1}}
\newcommand{\vt}{$(x_1,x_{2}, \ldots, x_n)$}
\newcommand{\vet}{(x_1,x_{2}, \ldots, x_n)}


\newcommand{\de}{\mathrm{d}}
\newcommand{\dd}[3]{\frac{ \de^{ #1 } #2 }{ \de #3^{ #1 } }}
\newcommand{\pd}[3]{\frac{ \partial^{ #1 } #2 }{ \partial #3^{ #1 } }}
\newcommand{\Str}[1]{\textbf{Str. #1.}}
\newcommand{\StrWg}[2]{\textbf{Str. #1, wiersz #2.}}
\newcommand{\StrWd}[2]{\textbf{Str. #1, wiersz #2 (od dołu).}}
\newcommand{\Jest}{\textbf{Jest: }}
\newcommand{\Pow}{\textbf{Powinno być: }}
\newcommand{\Prze}{\textbf{Przemyśl.}}
\newcommand{\Dok}{\textbf{Dokończ.}}



\begin{document}



\begin{center}
\LARGE{Fizyka.}\\
\large{Błędy.}
\end{center}





%\begin{center}
%R. P. Feynman, R. B. Leighton, M. Sands\\
%,,Feynmana wykłady z fizyki. Tom 2.1.'', .
%\end{center}
%
%Uwagi:\\
%\begin{itemize}
%\item[--] Str. 65. Stwierdzenie, że pole elektryczne i magnetyczne są niezwiązane dopóki ładunki i prądy są statyczne, jest pewnym uproszczeniem. Jakkolwiek może być ono pedagogicznie uzasadnione to jednak warto byłoby dodać jakiś komentarz dla bardziej dociekliwych czytelników.
%\end{itemize}
%
%Błędy:\\
%\begin{tabular}{|c|c|c|c|c|}
%\hline
%& \multicolumn{2}{c|}{} & & \\
%Strona & \multicolumn{2}{c|}{Wiersz} & Jest & Powinno być \\ \cline{2-3}
%& Od góry & Od dołu &  &  \\ \hline
%& & & & \\
%59 & & 18 & zmieniał & nie zmieniał \\
%& & & & \\
%& & & & \\ \hline
%\end{tabular}

%\begin{center}
%B. Skalmierski \\
%,,Mechanika. Tom I: Podstawy mechaniki klasycznej.'', \cite{BSMI}.
%\end{center}
%
%Uwagi:\\
%\begin{itemize}
%\item[--] Str. 10. Warunek c) w~definicji przestrzeni topologicznej Hausdorffa jest źle sformułowany, bowiem $\mathrm{R}_{ a }$ jest bez żadnych założeń otoczeniem punktu B zawartym w~$\mathrm{R}_{ a }$. To co autor chciał tu podać jest to definicja przestrzeni topologicznej Hausdorffa bazująca na pojęciu bazy otoczeń (chyba), należy więc zajrzeć do książki do topologi~by sprawidzić jak należy to poprawić.
%\item[--] Definicja homeomorfizmu jest trochę nie jasna, można bowiem odczytać ją tak, że~choć funkcja $f$ musi być ciągła, to żaden zaś warunek nie jest nałożyny na $f^{ -1 }$.
%\item[--] Str. 11, linia 8. Brak wcięcia akapitu.
%\item[--] Str. 18, linia 10. Brak wcięcia akapitu.
%\item[--]
%\end{itemize}
%
%Błędy:\\
%\begin{tabular}{|c|c|c|c|c|}
%\hline
%& \multicolumn{2}{c|}{} & & \\
%Strona & \multicolumn{2}{c|}{Wiersz} & Jest & Powinno być \\ \cline{2-3}
%& Od góry & Od dołu &  &  \\ \hline
%& & & & \\
%17 & & 13 & $( \mathbf{a} \times \mathbf{b} ) \times \mathbf{e}_{ j }$ & $( \mathbf{a} \times \mathbf{b} ) \cdot \mathbf{e}_{ j }$ \\
%17 & & 13 & $( \mathbf{e}_{ i } \times \mathbf{e}_{ k } ) \times \mathbf{e}_{ j }$ & $( \mathbf{e}_{ i } \times \mathbf{e}_{ k } ) \cdot \mathbf{e}_{ j }$ \\
%& & & & \\
%& & & & \\ \hline
%\end{tabular}
%\\ \noindent
%\textbf{Str. 42, wiersz 10 (od dołu).} \\
%\textbf{Jest:} Słońce znajduje się nie w centrum\ldots \\
%\textbf{Powinno być:} ale Słońce nie znajduje się w centrum\ldots \\








%
%
%
\begin{center}
J. D. Jackson\\
,,Elektrodynamik klasyczna'', \cite{Jac87}.
\end{center}

Uwagi:\\
\begin{itemize}
\item Dyskusja elektrostatyki powinna się zacząć od dyskusji problemu układu odniesienia.
\item Aby zapewnić fizyczną konsytencje teorii na początku rozdziału I powinno zostać przyjęte, że w rozważanych przypadkach nie ma obecnych pól magnetycznych. Nie jest to minimalny warunek konsytencji teorii, ale najbardziej naturalny.
\item Str. 47. Powinna tu być zamieszczona dyskusja problemu określenia pola elektrycznego w punkcie w którym znajduje się ładunek punktowy.
\end{itemize}


Błędy:\\
\begin{tabular}{|c|c|c|c|c|}
\hline
& \multicolumn{2}{c|}{} & & \\
Strona & \multicolumn{2}{c|}{Wiersz} & Jest & Powinno być \\ \cline{2-3}
& Od góry & Od dołu &  &  \\ \hline
& & & & \\
%58 & 5 & & $\ldots = \rho( \bold{ x }' ) \nabla^{ 2 } \bigg( \frac{ 1 }{ \sqrt{ r^{ 2 } + a^{ 2 } } } \bigg) \de^{ 3 } x' = \ldots$ & $\ldots = \int \rho( \bold{ x }' ) \nabla^{ 2 } \bigg( \frac{ 1 }{ \sqrt{ r^{ 2 } + a^{ 2 } } } \bigg) \de^{ 3 } x' = \ldots$ \\
\end{tabular}

\begin{itemize}
\item[--] Str. 51. \ldots po wewnętrznej stronie, Pristley w analogii\ldots
\item[--] Str. 65. $$w = \frac{ q^{ 2 }_{ 1 } }{ 8 \pi | \bold{ x } - \bold{ x }_{ 1 } |^{ 4 } } + \frac{ q^{ 2 }_{ 1 } }{ 8 \pi | \bold{ x } - \bold{ x }_{ 2 } |^{ 4 } } + \frac{ q_{ 1 } q_{ 2 } \, ( \bold{ x } - \bold{ x }_{ 1 } ) \cdot ( \bold{ x } - \bold{ x }_{ 2 } ) }{ 4 \pi | \bold{ x } - \bold{ x }_{ 1 } |^{ 3 } | \bold{ x } - \bold{ x }_{ 2 } |^{ 3 } } \, .$$
%\item[--]
\item[--] Str. 114. $$Y_{ l m }( \theta, \varphi ) = \sqrt{ \frac{ ( 2l + 1 )( l - m )! }{ 4 \pi ( l + m )! } } P^{ m }_{ l }( \cos \theta ) \e^{ i m \varphi} \, .$$
%\item[--]
%\item[--]
\end{itemize}
\noindent\\
\textbf{Str. 25, wiersz 7.}\\
\textbf{Jest:} w~równaniach Maxwella niesymetrycznie jedynie w~pierwszych dwóch równaniach. \\
\textbf{Powinno być:} nie występują symetrycznie w~równaniach Maxwella, są obecne jedynie w~dwóch pierwszych równaniach. \\


%
%\begin{center}
%F. Rohrlich\\
%,,Klasyczna teoria cząstek naładowanych'', \cite{FRKTCzN}.
%\end{center}
%
%%Uwagi:\\
%%\begin{itemize}
%%\item Dyskusja elektrostatyki powinna się zacząć od dyskusji problemu układu odniesienia.
%%\item Aby zapewnić fizyczną konsytencje teorii na początku rozdziału I powinno zostać przyjęte, że w rozważanych przypadkach nie ma obecnych pól magnetycznych. Nie jest to minimalny warunek konsytencji teorii, ale najbardziej naturalny.
%%\item Str. 47. Powinna tu być zamieszczona dyskusja problemu określenia pola elektrycznego w punkcie w którym znajduje się ładunek punktowy.
%%\end{itemize}
%
%
%Błędy:\\
%\begin{tabular}{|c|c|c|c|c|}
%\hline
%& \multicolumn{2}{c|}{} & & \\
%Strona & \multicolumn{2}{c|}{Wiersz} & Jest & Powinno być \\ \cline{2-3}
%& Od góry & Od dołu &  &  \\ \hline
%& & & & \\
%24 & 7 & & $k \bold{ r }$ & $-k \bold{ r }$ \\
%25 & & 14 & $\mathbf{E} \times \frac{ \mathbf{v} }{ c } \times \mathbf{ B }$ & $\mathbf{E} + \frac{ \mathbf{v} }{ c } \times \mathbf{ B }$ \\
%25 & 4 & & \emph{Electrodynamics},John & \emph{Electrodynamics}, John \\
%27 & 10 & & $v / c^{ 2 }$ & $v^{ 2 } / c^{ 2 }$ \\
%& & & & \\
%& & & & \\ \hline
%\end{tabular}
%
%%\begin{itemize}
%%\item[--] Str. 51. \ldots po wewnętrznej stronie, Pristley w analogii\ldots
%%\item[--] Str. 65. $$w = \frac{ q^{ 2 }_{ 1 } }{ 8 \pi | \bold{ x } - \bold{ x }_{ 1 } |^{ 4 } } + \frac{ q^{ 2 }_{ 1 } }{ 8 \pi | \bold{ x } - \bold{ x }_{ 2 } |^{ 4 } } + \frac{ q_{ 1 } q_{ 2 } \, ( \bold{ x } - \bold{ x }_{ 1 } ) \cdot ( \bold{ x } - \bold{ x }_{ 2 } ) }{ 4 \pi | \bold{ x } - \bold{ x }_{ 1 } |^{ 3 } | \bold{ x } - \bold{ x }_{ 2 } |^{ 3 } } \, .$$
%%%\item[--]
%%\item[--] Str. 114. $$Y_{ l m }( \theta, \varphi ) = \sqrt{ \frac{ ( 2l + 1 )( l - m )! }{ 4 \pi ( l + m )! } } P^{ m }_{ l }( \cos \theta ) \e^{ i m \varphi} \, .$$
%%%\item[--]
%%%\item[--]
%%\end{itemize}
%
%
%
%\begin{center}
%Kerson Huang\\
%,,Mechanika statystyczna'', \cite{KHMS}.
%\end{center}
%
%Uwagi:
%\begin{itemize}
%\item Brak dyskusji zależności wykładanej teorii od układu odniesienia.
%\item Jaka jest struktura matematyczna przestrzeni stanów termodynamicznych (krótko: przestrzeni
%termodynamicznej)?
%\item Przedstawiona tu dyskusja termodynamiki zostawia ogromną ilość pytań, zrówno fizycznych
%jak i matematycznych, bez odpowiedzi.
%\item Prawdopodobieństwo zaistnienia danego stanu w części poświęconej fizyce statystycznej
%nie zostało wyczerpująco omówione.
%\end{itemize}
%
%Powinno być:
%\begin{itemize}
%\item[--] Str. 135-136. \ldots dla których gęstość zależy od $( p, q )$ tylko przez hamiltonian \ldots
%%\item[--]
%%\item[--]
%%\item[--]
%%\item[--]
%%\item[--]
%%\item[--]
%\end{itemize}
%
%
%
%\begin{center}
%Bernard F. Schutz\\
%,,Wstęp do ogólnej teorii względności'', \cite{BSWOTW}.
%\end{center}
%
%Uwagi:
%\begin{itemize}
%\item[--] Str. 35. Wyprowadzenie transformacji Lorentza nie wydaje się poprawne. Brakuje argumentu który by implikował relację $\sigma = \alpha$.
%\item[--] Str. 62. Dowód, że foton ma zerową masę powinien być opatrzony większym komentarzem, opiera się on bowiem na interpretacji składowych czterowektora podanej dla cząstki z niezerową masą. Dlatego nie można jej tak po prostu przenieść dla czterowektora świetlnego. Z drugiej strony sama interpretacja składowych czterowektora o niezerowej masie, nie została potraktowana jako postulat lub uzasadniona fizycznie, lecz po prostu podana. Może tu też warto byłoby dodać jakiś komentarz.
%\item[--] Str. 106. Warto byłoby dodać komentarz wyjaśniający dlaczego wielkości termodynamiczne (określane tu jako skalarne) definiuje się zawsze w układzie CW danego fragmentu płynu.
%\item[--] Str. 108. Zdanie ,,gdy przewodzone jest ciepło, energia będzie niosła pęd'' wymaga głębszego zastanowienia.
%\end{itemize}
%
%Błędy:\\
%\begin{tabular}{|c|c|c|c|c|}
%\hline
%& \multicolumn{2}{c|}{} & & \\
%Strona & \multicolumn{2}{c|}{Wiersz} & Jest & Powinno być \\ \cline{2-3}
%& Od góry & Od dołu &  &  \\ \hline
%& & & & \\
%24 & 2 & & w $\mathcal{O}$ & w $\overline{\mathcal{O}}$ \\
%24 & 9 & & widzenia $\mathcal{O}$ & widzenia $\overline{\mathcal{O}}$ \\
%41 & & 10 & zagara & zegara \\
%82 & & 1 & $( a \quad b \quad \ldots )$ & $( a \: b \: \ldots )$ \\
%83 & 2 & & $( a \quad b \quad \ldots )$ & $( a \: b \: \ldots )$ \\
%143 & & 4 & $B\indices{^\mu_\nu_{;\, \beta}}$ & $B\indices{^\mu_\nu_{,\, \beta}}$ \\
%& & & & \\ \hline
%\end{tabular}
%
%\newpage
%
%\begin{center}
%S. W. Hawking, G. F. R. Ellis\\
%,,The Large Scale Structure of Space-Time'', \cite{SHGETLSSST}.
%\end{center}
%
%%Uwagi:\\
%%\begin{itemize}
%%\item
%%\item
%%\end{itemize}
%
%Błędy:\\
%\begin{tabular}{|c|c|c|c|c|}
%\hline
%& \multicolumn{2}{c|}{} & & \\
%Strona & \multicolumn{2}{c|}{Wiersz} & Jest & Powinno być \\ \cline{2-3}
%& Od góry & Od dołu &  &  \\ \hline
%& & & & \\
%24 & 9 & & $( y, 0 )$ & $( 0, y )$ \\
%31 & & 13 & $\tensor[]{\Phi}{ _{c'}^{c'} }$ & $\tensor[]{\Phi}{ _{c'}^{c} }$ \\
%31 & & 12 & $E_{ b }$ & $\bold{E}_{ b }$ \\
%31 & & 10 & $E_{ b' }$ & $\bold{E}_{ b' }$ \\
%31 & & 6 & $E_{ b' }$ & $\bold{E}_{ b' }$ \\
%& & & & \\ \hline
%\end{tabular}
%
%
%
%\begin{center}
%Richard P. Feynman\\
%,,Wykłady z grawitacji'',\\wydanie \romannumeral1, FWG.
%\end{center}
%
%Uwagi:\\
%
%Powinno być:
%\begin{itemize}
%\item[--] Str. 7. \ldots wynoszący, $\sqrt{ ( 4 \pi \e^{ 2 }/ \hbar c) } = 0.31$\ldots
%%\item[--]
%%\item[--]
%%\item[--]
%%\item[--]
%%\item[--]
%%\item[--]
%\end{itemize}
%
%
%
%\begin{center}
%Henryk Arodź, Leszek Hadasz\\
%,,Lectures on Classical and Quantum Theory of Fields'',\\wydanie \romannumeral1, QFT.
%\end{center}
%
%
%%Uwagi:
%
%
%%Powinno być:
%%\begin{itemize}
%%\item[--]
%%\item[--]
%%\item[--]
%%\item[--]
%%\item[--]
%%\item[--]
%%\end{itemize}
%
%
%
%\newpage
%
%
%
%\begin{center}
%N. N. Bogoliubov, D. V. Shirkov \\
%,,Introduction to the theory of quantized fields'', \cite{NBDSITQF}.
%\end{center}
%
%
%%Uwagii:
%%\begin{itemize}
%%\item Str.
%%%\item
%%%\item[--]
%%%\item[--]
%%%\item[--]
%%%\item[--]
%%%\item[--]
%%\end{itemize}
%
%Uwagi\footnote{W~tych uwagach będę stosował bardziej współczeną notację dla czterowektorów.}:
%\begin{itemize}
%\item[--] Str. 38. Wyjaśnienie w~tekście czemu składowa $U_{ 0 }$ generuje ujemną energię nie jest zbyt jasne. W~istocie sprawa jest prosta: wyrażenie na energie zawiera wyrażenia typu $-U^{ ( - ) }_{ \mu } U^{ ( + )\, \mu }$ przy czym każdy wszystkie iloczyny $U^{ ( - ) }_{ \nu } U^{ ( + )\, \nu }$ (brak sumowania!) są dodatnie. Stąd wyraz czasowy wchodzi do energii jako ujemny człon $-U^{ ( - ) }_{ 0 } U^{ ( + )\, 0 }$ podczas gdy człony przestrzenne dają dodatni wkład. Należy więc wyeliminować człon czasowy.
%\item[--] Str. 45. Warto byłoby jawnie zaznaczyć, że~wektory $\mathbf{e}_{ 1 }$ i~$\mathbf{e}_{ 2 }$ zależą od $k$.
%%\item[--] Str. 72.
%%\item[--] Str. 106. Warto byłoby dodać komentarz wyjaśniający dlaczego wielkości termodynamiczne (określane tu jako skalarne) definiuje się zawsze w układzie CW danego fragmentu płynu.
%%\item[--] Str. 108.
%\end{itemize}
%
%Błędy:\\
%\begin{tabular}{|c|c|c|c|c|}
%\hline
%& \multicolumn{2}{c|}{} & & \\
%Strona & \multicolumn{2}{c|}{Wiersz} & Jest & Powinno być \\ \cline{2-3}
%& Od góry & Od dołu &  &  \\ \hline
%& & & & \\
%6 & 8 & & begun & began \\
%14 & & 3 & $U$ & $u_{ i }$ \\
%19 & 1 & & $u'^{ k }( x' )$ & $u'_{ k }( x' )$ \\
%21 & 7 & & $\delta u_{ i }$ & $\overline{ \delta } u_{ i }$ \\
%21 & 7 & & $\delta$ & $\overline{ \delta }$ \\
%24 & & 8 & $\sum\limits_{ j,\, k < l }$ & $\sum\limits_{ j } \sum\limits_{ k < l }$ \\
%24 & & 8 & $( u_{ j }( x )$ & $u_{ j }( x )$ \\
%25 & 2 & & $\sum\limits_{ i,\, i }$ & $\sum\limits_{ i,\, j }$ \\
%26 & & 4 & $-i \alpha u_{ j }$ & $-i \alpha u_{ j }^{ * }$ \\
%27 & 4 & & $\partial u_{ i } / \partial x^{ k }$ & $\partial u_{ j } / \partial x^{ k }$ \\
%29 & 9 & & (2.19) & (2.20) \\
%30 & & 12 & $\tilde{ \varphi }( x )$ & $\tilde{ \varphi }( k )$ \\
%31 & & 6 & $k_{ 4 } x_{ 4 }$ & $-k_{ 4 } x_{ 4 }$ \\
%33 & 3 & & (3.19) & (3.14) \\
%33 & 3 & & (3.4) & (3.5) \\
%35 & 1 & & $n_{ 3 }^{ 3 }$ & $n_{ 3 }^{ 2 }$ \\
%43 & 13 & & (43) & (4.3) \\
%44 & & 10 & $U^{ *( \mp ) }$ & $U_{ n }^{ *( \mp ) }$ \\
%47 & & 16 & (4.24) & (4.26) \\
%49 & & 14 & $\Gamma^{ m }$ by & by \\
%70 & 1 & & $x^{ 0 }$ & $-x^{ 0 }$ \\
%72 & 8 & & while & but \\
%72 & 8 & & in addition & do not \\
%73 & & 8 & $\theta( k^{ 0 } )$ & $\theta( -k^{ 0 } )$ \\
%74 & & 18 & $\psi^{ ( + ) }( \mathbf{k} )$ & $\psi^{ ( + ) }( -\mathbf{k} )$ \\
%74 & & 17 & $\psi^{ ( - ) }( -\mathbf{k} )$ & $\psi^{ ( - ) }( \mathbf{k} )$ \\
%& & & & \\
%& & & & \\
%& & & & \\
%& & & & \\
%& & & & \\ \hline
%\end{tabular}
%
%
%
%
%
%
%
%
%
%
%


%
%
%
%\begin{center}
%Walter Thirring\\
%,,Fizyka matematyczna. Tom I: Klasyczne układy dynamiczne'', \cite{WTFMI}.
%\end{center}
%
%
%%Uwagi:
%%\begin{itemize}
%%\item Str. 18. W Definicji (2.1,1) powinno być założenie o istnieniu wektora zerowego.
%%\item
%%\end{itemize}
%
%
%Błędy:\\
%\begin{tabular}{|c|c|c|c|c|}
%\hline
%& \multicolumn{2}{c|}{} & & \\
%Strona & \multicolumn{2}{c|}{Wiersz} & Jest & Powinno być \\ \cline{2-3}
%& Od góry & Od dołu &  &  \\ \hline
%& & & & \\
%23 & 17 & & $x_{ 1 }, x_{ 2 } )$ & $( x_{ 1 }, x_{ 2 } )$ \\
%29 & & 2 & $\partial M = \{ a \} \cup \{ b \}, U_{ 2 } = ( a, b ],$ & $U_{ 2 } = ( a, b ], \Phi_{ 2 } : x \rightarrow b - a,$ \\
%& & &  $\Phi_{ 2 } : x \rightarrow b - a.$ & $\partial M = \{ a \} \cup \{ b \}.$ \\
%30 & 3 & & $\Phi_{ 1 } = :$ & $\Phi_{ 1 } :$ \\
%37 & 10 & & $\frac{ ( x_{ 1 } v_{ 1 } - x_{ 2 } v_{ 2 }}{ x_{ 1 }^{ 2 } + x_{ 2 }^{ 2 } }$ & $\frac{ ( x_{ 1 } v_{ 1 } - x_{ 2 } v_{ 2 } ) }{ x_{ 1 }^{ 2 } + x_{ 2 }^{ 2 } }$ \\
%37 & 14 & & $`T( M ) \times T( M )$ & $T( M ) \times T( M )$ \\
%37 & & 3 & $( q, v ) \rightarrow v$ & $( q, v )$ \\
%38 & 14 & & Wiązka ta jest trywializowalna, & Jeśli wiązka ta jest trywializowalna, \\
%& & & jeśli rozmaitość $X$ jest paralelyzowalna. & to rozmaitość jest pralelyzowalna. \\
%43 & & 9 & $\mathbf{ 1 \, \times \, jednostkowy\; wektor } $ & $\mathbf{ 1 \, \equiv \, jednostkowy\; wektor }$ \\
%& & & $\mathbf{ styczny }$ & $\mathbf{ styczny }$ \\
%43 & & 5 & $\mathbf{ 1 } \times$ wektor jednostkowy & $\mathbf{ 1 } \equiv$ wektor jednostkowy \\
%43 & & 4 & $( q_{ i }, x_{ i }( q ) )$ & $( q_{ i }, X_{ i }( q ) )$ \\
%45 & 1 & & $\mathbf{R}^{ n } \setminus \mathbf{R} \times \{ 0, 0, \ldots, 0 \}$ & $\mathbf{R}^{ n } \setminus \mathbf{R}$ \\
%48 & & 9 & $T( T( M )$ & $T( T( M ) )$ \\
%52 & & 1 & $T_{ q }( M )$ & $T_{ q }( M )$) \\
%53 & 11 & & $L_{ \Theta^{ -1 }_{ C } }( q ) e_{ j } dq^{ i }$ & $L_{ \Theta^{ -1 }_{ C }( q ) e_{ j } } q^{ i }$ \\
%74 & & 16 & & rozmaitości orientowalnych \\
%75 & 9 & & $\mathrm{sup} \; f_{ i }$ & $\mathrm{supp} \; f_{ i }$ \\
%75 & & 10 & do $\mathrm{sup} \; f$ & od $\mathrm{supp} \; f$ \\
%77 & & 8 & $x^{ 2 } + x^{ 2 }$ & $x^{ 2 } + y^{ 2 }$ \\
%& & & & \\ \hline
%\end{tabular}
%
%
%
%\begin{center}
%Walter Thirring\\
%,,Fizyka matematyczna. Tom III: Mechanika kwantowa atomów i cząsteczek.'', \cite{WTFMIII}.
%\end{center}
%
%Uwagi:
%\begin{itemize}
%\item Str. 18. W Definicji (2.1,1) powinno być założenie o istnieniu wektora zerowego.
%\item
%\end{itemize}
%
%Błędy:\\
%\begin{tabular}{|c|c|c|c|c|}
%\hline
%& \multicolumn{2}{c|}{} & & \\
%Strona & \multicolumn{2}{c|}{Wiersz} & Jest & Powinno być \\ \cline{2-3}
%& Od góry & Od dołu &  &  \\ \hline
%& & & & \\
%19 & 15 & & praktyczne.Moc & praktyczne. Moc \\
%%& & & & \\
%%& & & & \\
%%& & & & \\
%& & & & \\ \hline
%\end{tabular}
%
%
%
%\begin{center}
%R. F. Streater, A. S. Wightman\\
%,,PCT, spin and statistics, and all that'', \cite{RSAWPCT}.
%\end{center}
%
%%Uwagi:
%%\begin{itemize}
%%\item Str. 18. W Definicji (2.1,1) powinno być założenie o istnieniu wektora zerowego.
%%\item
%%\end{itemize}
%
%
%Błędy:\\
%\begin{tabular}{|c|c|c|c|c|}
%\hline
%& \multicolumn{2}{c|}{} & & \\
%Strona & \multicolumn{2}{c|}{Wiersz} & Jest & Powinno być \\ \cline{2-3}
%& Od góry & Od dołu &  &  \\ \hline
%& & & & \\
%12 & & 3 & $( \tau^{ 2 } )( \tau^{ \mu } )^{ T }( \tau^{ 2 } )^{ -1 }$ & $\tau^{ 2 } ( \tau^{ \mu } )^{ T } \tau^{ 2 }$ \\
%12 & & 1 & $\tau^{ 2 } ( x )^{ T }( \tau^{ 2 } )^{ -1 }$ & $\tau^{ 2 } ( x )^{ T } \tau^{ 2 }$ \\ %Popraw
%13 & 2 & & $\tau^{ 2 } A^{ T }( \tau^{ 2 } )^{ -1 }$ & $\tau^{ 2 } A^{ T } \tau^{ 2 }$ \\
%%& & & & \\
%%& & & & \\
%%& & & & \\
%& & & & \\ \hline
%\end{tabular}
%
%%\begin{center}
%%R. Haag\\
%%,,Local Quantum Physics.\\
%%Filds, Particles, Algebras.'',\\wydanie \romannumeral2, RH.
%%\end{center}
%%Uwagi:
%%\begin{itemize}
%%\item
%%\item
%%\end{itemize}\usepackage{amsmath}
%%Powinno być:
%%\begin{itemize}
%%\item[--]
%%\item[--]
%%\item[--]
%%\item[--]
%%\item[--]
%%\item[--]
%%\item[--]
%%\item[--]
%%\item[--]
%%\item[--]
%%\item[--]
%%\end{itemize}



\bibliographystyle{ieeetr}
\bibliography{Bibliography}{}



\end{document}