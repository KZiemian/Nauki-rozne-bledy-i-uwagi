\documentclass[a4paper]{article}
\usepackage[polish]{babel}
\usepackage[utf8]{inputenc}
\usepackage[MeX]{polski}
\usepackage{fullpage}
\usepackage{verse}

\newcommand{\tb}{\textbf}
\newcommand{\noi}{\noindent}
\newcommand{\attribA}[1]{%
  \nopagebreak{\vspace{3mm}\raggedleft\footnotesize #1\par\vspace{2em}}}
\newcommand{\attribB}[1]{%
  \nopagebreak{\raggedleft\footnotesize #1\par} \vspace{2em}}
\verselinenumbersleft

\begin{document}



Po trzecie, przez świątynię można rozumieć duszę duchową, jak
powiedziano w 1 Kor 3, 17: \emph{Świątynia Boga jest święta, a wy nią
  jesteście.} Wtedy człowiek sprzedaje w~świątyni owce, woły
i~gołębie, gdy w~duszy zachowa zwierzęce poruszenia, dla których
zaprzedaje~się diabłu. Albowiem woły, które służą uprawie roli,
symbolizują ziemskie pragnienia, owce, które są zwierzętami głupimi,
symbolizują ludzką tępotę, gołębie zaś ludzi niestałych. I~to wszystko
Bóg wypędził z~serc ludzi.

\attribA{Św. Tomasz z Akwinu, ,,Komentarz do Ewangelii Jana'',
  rozdział drugi, wykład drugi, 191.}

\noi I~teraz bowiem wiadomo, że~ludzie tej samej profesji odnoszą~się
do~siebie w~sposób podstępny i~zawistny. Garncarz zazdrości
garncarzowi, a~nie stolarzowi. W~ten sposób także nauczyciele szukając
własnej czci, boleją, jeśli ktoś naucza prawdy. Przeciw nim powiada
Grzegorz: ,,Umysł pobożnego pasterza chce, aby~inni nauczali prawdy,
której on sam nie jest w~stanie uczyć''.

\attribA{Św. Tomasz z Akwinu, ,,Komentarz do Ewangelii Jana'',
  rozdział trzeci, wykład piąty, 511.}

\noi Albowiem stałe obcowanie z ludźmi i~zbytnia zażyłość, pomniejsza
szacunek i~rodzi pogardę. Dlatego mamy zwyczaj mniej szanować tych,
z~którymi łączy nas bardziej zażyła więź, a~bardziej szanujemy ludzi,
z~którymi nie~możemy połączyć~się więzią zażyłości. Natomiast
w~odniesieniu do Boga dzieje~się coś przeciwnego. Albowiem im bardziej
ktoś przez miłość i~kontemplację zwiąże~się z~Bogiem więzią zażyłości,
tym bardziej uznaje Go za wyniesionego i~bardziej szanuje, siebie
samego uznając za mniejszego, Hi~42, 5--6: \emph{Słuchałem ucha
  słyszałem cię, a~teraz oko moje cię widzi: przeto sam siebie winię
  i~czynię pokutę w~prochu i~w~popiele.}  Uzasadnieniem takiego stanu
rzeczy jest fakt, iż~człowiek jest istotą o~słabej i~kruchej naturze
i~gdy ktoś z~nim długo obcuje, poznaje jakąś jego słabość i~stąd
zmniejsza się w nim szacunek do niego. A~w~odniesieniu do Boga, który
jest istotą o~niezmierzonej doskonałości, człowiek o~ile postąpi
w~poznaniu Go, o~tyle bardziej podziwia wyniesienie jego doskonałości
i~bardziej Go szanuje.

\attribA{Św. Tomasz z Akwinu, ,,Komentarz do Ewangelii Jana'',
  rozdział czwarty, wykład szósty, 667.}

\noi Trzeba nam iść do przodu, ponieważ ten, kto stoi, wystawia się na
niebezpieczeństwo, że~nie będzie mógł zachować życia łaski. Na drodze
ku Bogu bowiem nie~iść do przodu oznacza cofać się.

\attribA{Św. Tomasz z Akwinu ,,Komentarz do Ewangelii Jana'', rozdział
  czwarty, wykład siódmy, 690.}


%%%%%%%%%%%%%%%%%%%%%%%%%%%%%%%%%%%%%%%%

\noi Grzeszyłem w~latach chłopięcych, gdy niedorzeczności ceniłem
wyżej od~rzeczy pożytecznych. Co mówię! jedne kochałem, drugich
nienawidziłem. Jeden i~jeden -- dwa, dwa i~dwa -- cztery. Jakaż
nienawistna była mi ta śpiewka. A jak błogim widowiskiem przy całej
swej wewnętrznej pustce był ów drewniany, pełen wojowników koń
trojański! I~łuna Troi! I~,,samej Kreuzy cień...''.

\attribA{Św. Augustyn ,,Wyznania'', księga I, 13. \\
  O~swojej miłości do Wergiliusza i~literatury.}

\noi Straszliwa rzeko społecznego obyczaju! Kto się tobie oprzeć
zdoła? Czy nigdy twoje wody nie~opadną, nie~wyschną? Jak długo
będziesz jeszcze gnać nieszczęsnych ludzi ku morzu wielkiemu
i~groźnemu, trudnemu do przebycia, nawet dla tych którzy do drzewa
Krzyża przywarli?

\attribA{Św. Augustyn ,,Wyznania'', księga I, 16.}

\noi Jak to się dzieje, że~jeden w belferskich szatach człowiek może
spokojnie słuchać, jak inny z~takiego samego miotu biedak woła:
,,Zmyślił to Homer; ludzkie sprawy do świata bogów przenosił;
o,~czemuż nie boskie do nas...''

\attribA{Św. Augustyn ,,Wyznania'', księga I, 16.}

\noi Trudno się dziwić, że~tak się pogrążyłem w marnościach
i~odchodziłem daleko od Ciebie, skoro jako wzory do naśladowania
przedstawiano mi ludzi, którzy wstydzili się jak hańby tego,
że~opowiadając o dobrych swoich czynach, popełnili błąd gramatyczny
albo~użyli wyrażeń prowincjonalnych, a~dumnie kroczyli w obłoku
pochwały, jeśli o~swoich niegodziwych pasjach mówili zdaniami
zaokrąglonymi, błyszczącymi obfitą ornamentyką.

\attribA{Św. Augustyn ,,Wyznania'', księga I, 16.}


\noi Stary Wiktoryn był człowiekiem niepospolicie uczonym. Znał do
głębi wszystkie sztuki wyzwolone. Ileż dzieł filozofów przestudiował,
ocenił i~skomentował. Był nauczycielem wielu znakomitych
senatorów. Z~wdzięczności za~pracę profesorską, którą synowie tego
świata uważają za~ogromnie ważną, uczczono go wystawieniem posągu na
Forum w~Rzymie.

\attribA{Św. Augustyn ,,Wyznania'', księga VIII, 2.}

\noi Na co my czekamy? Czy pojąłeś sens tej opowieści? Powstają
prostaczkowie i~zdobywają niebo, a~my, z~całą naszą bezduszną
uczonością tarzamy~się w~ciele i~w~krwi.

\attribA{Św. Augustyn ,,Wyznania'', księga VIII, 8. \\
  Słowa św. Augustyna do swego przyjaciela Alipiusza.}

\noi Jak schlebiający przyjaciele nieraz czynią nas gorszymi, tak
obrażający nas wrogowie nieraz~się przyczyniają do naszej poprawy.

\attribA{Św. Augustyn ,,Wyznania'', księga VIII, 8.}

\noi Potem~się dowiedziałem, że~pewnego dnia podczas pobytu w~Ostii,
gdy ja nie byłem obecny, do kilku moich przyjaciół z~macierzyńską
bezpośredniością mówiła, jak godne pogardy jest to życie i~jakim
dobrem jest śmierć.

\attribA{Św. Augustyn ,,Wyznania'', księga IX, 11. \\ O~swej matce
  św. Monice.}


\noi I~tą nadzieją~się weselę, ilekroć~się weselę zbawiennie. A~inne
tego życia sprawy? Tym mniej zasługują na łzy, im częściej się na nimi
płacze. Tym bardziej się powinno nad nimi płakać, im mniej~się nad
nimi łez leje.

\attribA{Św. Augustyn ,,Wyznania'', księga X, 1.}

\noi Jakże skwapliwie ludzie badają cudze życie, a~jak~się opieszale
zabierają do naprawienia swojego. Czemu chcą~się ode mnie dowiedzieć,
jakim jestem człowiekiem, skoro nie pragną usłyszeć od Ciebie, jacy
oni sami naprawdę~są?

\attribA{Św. Augustyn ,,Wyznania'', księga X, 3.}

\end{document}