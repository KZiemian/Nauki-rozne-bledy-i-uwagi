% ------------------------------------------------------------------------------------------------------------------
% Basic configuration and packages
% ------------------------------------------------------------------------------------------------------------------
% Package for discovering wrong and outdated usage of LaTeX.
% More information to be found in l2tabu English version.
\RequirePackage[l2tabu, orthodox]{nag}
% Class of LaTeX document: {size of paper, size of font}[document class]
\documentclass[a4paper,11pt]{article}



% ------------------------------------------------------
% Packages not tied to particular normal language
% ------------------------------------------------------
% This package should improved spaces in the text.
\usepackage{microtype}
% Add few important symbols, like text Celcius degree
\usepackage{textcomp}



% ------------------------------------------------------
% Polonization of LaTeX document
% ------------------------------------------------------
% Basic polonization of the text
\usepackage[MeX]{polski}
% Switching on UTF-8 encoding
\usepackage[utf8]{inputenc}
% Adding font Latin Modern
\usepackage{lmodern}
% Package is need for fonts Latin Modern
\usepackage[T1]{fontenc}



% ------------------------------------------------------
% Setting margins
% ------------------------------------------------------
\usepackage[a4paper, total={14cm, 25cm}]{geometry}



% ------------------------------------------------------
% Setting vertical spaces in the text
% ------------------------------------------------------
% Setting space between lines
\renewcommand{\baselinestretch}{1.1}

% Setting space between lines in tables
\renewcommand{\arraystretch}{1.4}



% ------------------------------------------------------
% Packages for scientific papers
% ------------------------------------------------------
% Switching off \lll symbol, that I guess is representing letter "Ł"
% It collide with `amsmath' package's command with the same name
\let\lll\undefined
% Basic package from American Mathematical Society (AMS)
\usepackage[intlimits]{amsmath}
% Equations are numbered separately in every section.
\numberwithin{equation}{section}

% Other very useful packages from AMS
\usepackage{amsfonts}
\usepackage{amssymb}
\usepackage{amscd}
\usepackage{amsthm}

% Package with better looking calligraphy fonts
\usepackage{calrsfs}

% Package with better looking greek letters
% Example of use: pi -> \uppi
\usepackage{upgreek}
% Improving look of lambda letter
\let\oldlambda\Lambda
\renewcommand{\lambda}{\uplambda}




% ------------------------------------------------------
% BibLaTeX
% ------------------------------------------------------
% Package biblatex, with biber as its backend, allow us to handle
% bibliography entries that use Unicode symbols outside ASCII.
\usepackage[
language=polish,
backend=biber,
style=alphabetic,
url=false,
eprint=true,
]{biblatex}

\addbibresource{Logika-Bibliography.bib}





% ------------------------------------------------------
% Defining new environments (?)
% ------------------------------------------------------
% Defining enviroment "Wniosek"
\newtheorem{corollary}{Wniosek}
\newtheorem{definition}{Definicja}
\newtheorem{theorem}{Twierdzenie}





% ------------------------------------------------------
% Local packages
% You need to put them in the same directory as .tex file
% ------------------------------------------------------
% Package containing various command useful for working with a text
\usepackage{general-commands}
% Package containing commands and other code useful for working with
% mathematical text
% \usepackage{math-commands}





% ------------------------------------------------------
% Package "hyperref"
% They advised to put it on the end of preambule
% ------------------------------------------------------
% It allows you to use hyperlinks in the text
\usepackage{hyperref}










% ------------------------------------------------------------------------------------------------------------------
% Title and author of the text
\title{Logika \\
  {\Large Błędy i~uwagi}}

\author{Kamil Ziemian}


% \date{}
% ------------------------------------------------------------------------------------------------------------------










% ####################################################################
% Beginning of the document
\begin{document}
% ####################################################################





% ######################################
\maketitle
% ######################################





% ######################################
\section{Józef W.~Bremer \textit{Wprowadzenie do~logiki},
  \cite{Bremer-Wprowadzenie-do-logiki-Pub-2004}}
% ######################################


% ##################
\CenterBoldFont{Uwagi ogólne}




\noindent
\StrWierszGora{132}{1} Zdanie „$p \vee q ( ( p \land q ) \to ( p \land q ) )$” nie ma
żadnego sensu logicznego, musiał zostać zgubiony spójnik logiczny
po~pierwszym~$q$. Niestety nie wiem który z nich należy tam umieścić.





% ##################
\CenterBoldFont{Błędy}


\begin{center}

  \begin{tabular}{|c|c|c|c|c|}
    \hline
    & \multicolumn{2}{c|}{} & & \\
    Strona & \multicolumn{2}{c|}{Wiersz} & Jest
                              & Powinno być \\ \cline{2-3}
    & Od góry & Od dołu & & \\
    \hline
    \hphantom{00}4 & \hphantom{0}8 & & LATEX & \LaTeX \\
    \hphantom{0}13 & & \hphantom{0}8 & \textit{formalnej}
           & \textit{formalnej}; \\
    \hphantom{0}20 & & 15 & Ockhama”$^{ 7 }$. & Ockhama”$^{ 7 }$, \\
    \hphantom{0}21 & 14 & & Współczesnych & współczesnych \\
    \hphantom{0}26 & & \hphantom{0}2 & \textit{Lwowsko-\! Warszawska}
           & \textit{Lwowsko-Warszawska} \\
    \hphantom{0}45 & & 10 & konkretna & konkretną \\
    \hphantom{0}86 & 10 & & przypadku.. & przypadku. \\
    \hphantom{0}88 & 11 & & średniego & pośredniego \\  % ???
    \hphantom{0}98 & & \hphantom{0}9 & $P \underline{ e } S$
           & $S \underline{ e } P$ \\
    114 & \hphantom{0}9 & & zwane\textit{prawo} & zwane \textit{prawo} \\
    117 & \hphantom{0}6 & & jedzie & jadą \\
    119 & & 14 & współczesnej~. & współczesnej. \\
    122 & & \hphantom{0}4 & wniosek $\equiv P$) & wniosek $\equiv P$)” \\
    125 & & 14 & „nie-analityczne & „nie-analityczne” \\
    131 & \hphantom{0}6 & & „$\neg p \vee \neg q''$ & „$\neg p \vee \neg q$” \\
    131 & \hphantom{0}6 & & $\equiv$,\hspace{2pt},$\neg ( p \land q )$”
           & $\equiv$ „$\neg ( p \land q )$” \\
    132 & \hphantom{0}8 & & $\neg( \neg p\;\;\; \land\; \neg q ) $
           & $\neg( \neg p \land \neg q ) $ \\
    132 & 17 & & $p\quad \to \quad q$ & $p \to q$ \\
    132 & & 16 & $p\quad \to \quad q$ & $p \to q$ \\
    % & & & & \\
    % & & & & \\
    % & & & & \\
    % & & & & \\
    % & & & & \\
    \hline
  \end{tabular}

\end{center}

\VerSpaceSix


% ######################################










% ######################################
\section{Graham Priest \textit{Logika},
  \parencite{Priest-Logika-Pub-2023}}
% ######################################



% ##################
\CenterBoldFont{Uwagi do~konkretnych stron}

\vspace{0em}


\noindent
\StrWierszGora{78}{14} Sens występującego tutaj ciąg symboli $jR$ jest
mniej oczywisty w~języku polskim niż w~angielskim, dlatego tutaj podanym
jawnie jego sens. Symbol $j$ oznacza tutaj „Jan”, zaś $R$ oznacza „jest
czerwony”. Warto nadmienić, że~po angielsku zapewne zamiast „Jana” był
„John”, a~symbol $R$ oznaczał „is red”.

\VerSpaceFour





\noindent
\Str{86} Na tej stronie pojawia~się pierwszy raz symbol funkcji minimum
z~dwóch liczb
$min( \HorSpaceOne \cdot \HorSpaceOne , \cdot \HorSpaceOne )$, który jednak
skonfrontowany z~przyjętymi dziś standardami nie wygląda najlepiej.
W~mojej ocenie należałoby go zastąpić symbolem
$\min( \HorSpaceOne \cdot \HorSpaceOne , \cdot \HorSpaceOne )$. Ta sama uwaga dotyczy
symbolu stosowanego do~funkcji maksimum z~dwóch liczb, który powinien być
zastąpiony przez $\max( \HorSpaceOne \cdot \HorSpaceOne , \cdot \HorSpaceOne )$.

\VerSpaceFour





\noindent
\Str{93} Na~stronie tej pojawia~się prawdopodobieństwo tego, że~zdanie~$a$
jest prawdziwe, jest zapisywane symbolem $pr( a )$. Z~tych samych powodów
co w~uwagach do strony~86, proponuję zastąpienie go~przez
symbol~$\Pr( a )$.

\VerSpaceFour





% \noindent
% \StrWierszDol{}{}

% \VerSpaceFour





% \noindent
% \Str{}

% \VerSpaceFour





% \noindent
% \StrWierszGora{}{}

% \VerSpaceFour





% \noindent
% \StrWierszGora{}{}

% \VerSpaceFour





% \noindent
% \StrWierszGora{}{}

% \VerSpaceFour





% \noindent
% \StrWierszGora{}{}

% \VerSpaceFour





% \noindent
% \Str{}

% \VerSpaceFour





% \noindent
% \Str{}

% \VerSpaceFour





% ##################
% \newpage

\CenterBoldFont{Błędy}


\begin{center}

  \begin{tabular}{|c|c|c|c|c|}
    \hline
    Strona & \multicolumn{2}{c|}{Wiersz} & Jest
    & Powinno być \\ \cline{2-3}
    & Od góry & Od dołu & & \\
    \hline
    \hphantom{0}47 & 20 & & to jest & to nie jest \\
    \hphantom{0}87 & \hphantom{0}8 &
    & $| a | \!\! \leq | b |$\textit{:} & $| a | \leq | b |:$ \\
    \hphantom{0}87 & \hphantom{0}9 &
    & $| b | \!\! \leq | a |$\textit{:} & $| b | \leq | a |:$ \\
    149 & & \hphantom{0}4 & \textit{wygrała nagrodę}
    & \textit{wygrała wyścig} \\
    155 & & 10 & \textit{wygrała nagrodę} & \textit{wygrała wyścig} \\
    156 & & \hphantom{0}1 & $\neg \Diamond b)$ & $\neg \Diamond b$ \\
    % & & & & \\
    \hline
  \end{tabular}





  % \newpage

  % \begin{tabular}{|c|c|c|c|c|}
  %   \hline
  %   Strona & \multicolumn{2}{c|}{Wiersz} & Jest
  %                             & Powinno być \\ \cline{2-3}
  %   & Od góry & Od dołu & & \\
  %   \hline
  %   & & & & \\
  %   & & & & \\
  %   \hline
  % \end{tabular}

\end{center}

\VerSpaceTwo


\noindent
\StrWierszDol{149}{3} \\
\Jest \textbf{Rozdział~5.} \\
\PowinnoByc \textbf{Rozdział~5.} Na potrzeby tego zadania przyjmujemy,
że~zdania mogą przyjmować takie wartości logiczne, jak zdania analizowane
w~rozdziale do~którego~się odnosi. Mogą więc być jednocześnie prawdziwe
i~fałszywe, prawdziwe lecz nie fałszywe,~etc.

% ############################








































% ####################################################################
% ####################################################################
% Bibliography

\printbibliography





% ############################
% End of the document

\end{document}
