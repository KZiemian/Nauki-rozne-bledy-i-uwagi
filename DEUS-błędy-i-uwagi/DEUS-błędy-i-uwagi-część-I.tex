% ---------------------------------------------------------------------
% Podstawowe ustawienia i pakiety
% ---------------------------------------------------------------------
\RequirePackage[l2tabu, orthodox]{nag} % Wykrywa przestarzałe i niewłaściwe
% sposoby używania LaTeXa. Więcej jest w l2tabu English version.
\documentclass[a4paper,11pt]{article}
% {rozmiar papieru, rozmiar fontu}[klasa dokumentu]
\usepackage[MeX]{polski} % Polonizacja LaTeXa, bez niej będzie pracował
% w języku angielskim.
\usepackage[utf8]{inputenc} % Włączenie kodowania UTF-8, co daje dostęp
% do polskich znaków.
\usepackage{lmodern} % Wprowadza fonty Latin Modern.
\usepackage[T1]{fontenc} % Potrzebne do używania fontów Latin Modern.



% ------------------------------
% Podstawowe pakiety (niezwiązane z ustawieniami języka)
% ------------------------------
\usepackage{microtype} % Twierdzi, że poprawi rozmiar odstępów w tekście.
% \usepackage{graphicx} % Wprowadza bardzo potrzebne komendy do wstawiania
% % grafiki.
% \usepackage{verbatim} % Poprawia otoczenie VERBATIME.
% \usepackage{textcomp} % Dodaje takie symbole jak stopnie Celsiusa,
% wprowadzane bezpośrednio w tekście.
\usepackage{vmargin} % Pozwala na prostą kontrolę rozmiaru marginesów,
% za pomocą komend poniżej. Rozmiar odstępów jest mierzony w calach.
% ------------------------------
% MARGINS
% ------------------------------
\setmarginsrb
{ 0.7in} % left margin
{ 0.6in} % top margin
{ 0.7in} % right margin
{ 0.8in} % bottom margin
{  20pt} % head height
{0.25in} % head sep
{   9pt} % foot height
{ 0.3in} % foot sep



% ------------------------------
% Często używane pakiety
% ------------------------------
% \usepackage{csquotes} % Pozwala w prosty sposób wstawiać cytaty do tekstu.
\usepackage{xcolor} % Pozwala używać kolorowych czcionek (zapewne dużo
% więcej, ale ja nie potrafię nic o tym powiedzieć).





% ---------------------------------------------------------------------
% Dodatkowe ustawienia dla języka polskiego
% ---------------------------------------------------------------------
\renewcommand{\thesection}{\arabic{section}.}
% Kropki po numerach rozdziału (polski zwyczaj topograficzny)
\renewcommand{\thesubsection}{\thesection\arabic{subsection}}
% Brak kropki po numerach podrozdziału



% ------------------------------
% Pakiety których pliki *.sty mają być w tym samym katalogu co ten plik
% ------------------------------
\usepackage{latexgeneralcommands}



% ------------------------------
% Ustawienia różnych parametrów tekstu
% ------------------------------
\renewcommand{\baselinestretch}{1.1}

\renewcommand{\arraystretch}{1.4} % Ustawienie szerokości odstępów między
% wierszami w tabelach.



% ------------------------------
% Pakiet „hyperref”
% Polecano by umieszczać go na końcu preambuły
% ------------------------------
\usepackage{hyperref} % Pozwala tworzyć hiperlinki i zamienia odwołania
% do bibliografii na hiperlinki










% ---------------------------------------------------------------------
% Tytuł, autor, data
\title{DEUS \\
  {\Large Błędy i~uwagi, część~I}}

\author{Kamil Ziemian}


% \date{}
% ---------------------------------------------------------------------










% ####################################################################
% Początek dokumentu
\begin{document}
% ####################################################################





% ######################################
\maketitle  % Tytuł całego tekstu
% ######################################





% ######################################
\section{Dzieła świętych}

\vspace{\spaceTwo}
% ######################################



% ############################
\Work{ % Autor i tytuł dzieła
  Sanctae Jan Chryzostom \\
  \textit{Homilie na~Księgę Rodzaju (seria pierwsza: Rdz 1--3)},
  \cite{SancteJanChryzostomHomKsiegaRodzaju2008}}

\vspace{0em}


% ##################
\CenterBoldFont{Uwagi}

\vspace{0em}


\noindent
\Str{78} Na~końcu pierwszego akapitu jest jeden nieotwarty
cudzysłów, nie wiadomo więc, gdzie~się powinien zaczynać, a~gdzie
kończyć.

\vspace{\spaceFour}





\noindent
\Str{106} Od końca pierwszego akapitu do~końca tej homilii,
tekst staję~się chaotyczny i~w~pewnym stopniu nielogiczny.
Należałoby~by sprawdzić, czy~został poprawnie przetłumaczony.





% ##################
\CenterBoldFont{Błędy}

\vspace{\spaceFive}


\begin{center}

  \begin{tabular}{|c|c|c|c|c|}
    \hline
    Strona & \multicolumn{2}{c|}{Wiersz} & Jest
                              & Powinno być \\ \cline{2-3}
    & Od góry & Od dołu & & \\
    \hline
    73 & & 11 & w pokoju & pokoju \\
    % & & & & \\
    % & & & & \\
    \hline
  \end{tabular}

\end{center}

\vspace{\spaceTwo}


% ############################










% ############################
\Work{ % Autor i tytuł dzieła
  Sanctae Jak Klimak \\
  \textit{Drabina raju}, \cite{SancteJanKlimakDrabinaRaju2011}}

\vspace{0em}


% ##################
\CenterBoldFont{Uwagi ogólne}

\vspace{0em}


\noindent
W~tym wydaniu przyjęto dziwną regułę, że~liczba odsyłająca do~przypisu nie
jest zaraz przy wyrazie, lecz znajduje~się między nimi odstęp. Na~przykład
na stronie~101 jest „głowo $^{ 1 }$”, a~nie „głowo$^{ 1 }$”. Zdecydowanie
wolę tę drugą konwencję.



\vspace{\spaceTwo}


% ############################










% ############################
\newpage

\Work{ % Autor i tytuł dzieła
  Sanctae Augustyn z~Hippony \\
  \textit{Państw Boże}, \cite{} }


% % ##################
% \CenterBoldFont{Uwagi}


% ##################
\CenterBoldFont{Błędy}

% \vspace{\spaceFive}


\begin{center}

  \begin{tabular}{|c|c|c|c|c|}
    \hline
    Strona & \multicolumn{2}{c|}{Wiersz} & Jest
                              & Powinno być \\ \cline{2-3}
    & Od góry & Od dołu & & \\
    \hline
    51  &  8 & & zostawia” & zostawia \\
    92  & 16 & & Od & „Od \\
    122 &  9 & & punicka. & punicka.” \\
    % & & & & \\
    % & & & & \\
    % & & & & \\
    \hline
  \end{tabular}

\end{center}

\vspace{\spaceTwo}
% ############################










% ############################
\newpage

\Work{ % Autor i tytuł dzieła
  Sanctae Tomasz z Akwinu \\
  \textit{Suma Teologiczna. Tom~I},
  \cite{SancteTomaszZAkwinuSumaTeologicznaTomI1960}}


% % ##################
% \CenterBoldFont{Uwagi ogólne}





% ##################
\CenterBoldFont{Błędy}

% \vspace{\spaceFive}


\begin{center}

  \begin{tabular}{|c|c|c|c|c|}
    \hline
    & \multicolumn{2}{c|}{} & & \\
    Strona & \multicolumn{2}{c|}{Wiersz} & Jest
                              & Powinno być \\ \cline{2-3}
    & Od góry & Od dołu & & \\
    \hline
    28  &  3 & & ( 1 & (1 \\
    31  & &  7 & \textit{WIEDZĄ~?} & \textit{WIEDZĄ?} \\
    32  & & 13 & \textit{WIEDZĄ~?} & \textit{WIEDZĄ?} \\
    32  & &  3 & bo-wiem & bowiem \\
    32  & &  2 & [Mądrość]dała & [Mądrość] dała \\
    33  & 12 & & n~i~ż~s~z~e~~władze & n~i~ż~s~z~e\, władze \\
    33  & 14 & & n~a~d~r~z~ę~d~n~e~j~~władzy
           & n~a~d~r~z~ę~d~n~e~j\, władzy \\
    33  & & 11 & \textit{PRAKTYCZNĄ~?} & \textit{PRAKTYCZNĄ?} \\
    34  & &  1 & za & ta \\
    35  & 13 & & ludzkiego”. & ludzkiego. \\
    35  & &  9 & \textit{MĄDROŚCIĄ~?} & \textit{MĄDROŚCIĄ?} \\
    36  &  3 & & jest & nie jest \\
    37  & 19 & & \textit{BÓG~?} & \textit{BÓG?} \\
    37  & & 14 & jest$^{ 2 }$ & jest''$^{ 2 }$ \\
    37  &  9 & & n & na \\
    38  & & 15 & \textit{UZASADNIAĆ~?} & \textit{UZASADNIAĆ~?} \\
    38  & & 13 & Gdzie & „Gdzie \\
    38  & &  3 & biskupie & o~biskupie \\
    39  & &  1 & teologii~! & teologii! \\
    40  & & 16 & \textit{PRZENOŚNI~?} & \textit{PRZENOŚNI?} \\
    41  & 15 & & niewykształconych”$^{ 5 }$
           & niewykształconych”$^{ 5 }$) \\
    41  & 20 & & i~~d~l~a & i{}\, d~l~a \\
    42  &  5 & & \textit{ZNACZEŃ~?} & \textit{ZNACZEŃ?} \\
    44  &  1 & & \textit{JEST~?} & \textit{JEST~?} \\
    44  & 17 & & \textit{SIEBIE~?} & \textit{SIEBIE?} \\
    45  & &  6 & pond & ponad \\
    46  & 10 & & \textit{BOGA~?} & \textit{BOGA?} \\
    47  & & 13 & \textit{ISTNIEJE~?} & \textit{ISTNIEJ?} \\
    48  & & 16 & druga & Druga \\
    49  & & 14 & poznania,{ }, & poznania, \\
    60  & & 16 & zak & tak \\
    % & & & & \\
    % & & & & \\
    % & & & & \\
    % & & & & \\
    % & & & & \\
    \hline
  \end{tabular}

\end{center}

\vspace{\spaceTwo}



% ############################










% ############################
\newpage

\Work{ % Autor i tytuł dzieła
  Sanctae Tomasz z Akwinu \\
  \textit{O~królowaniu~-- królowi Cypru},
  \cite{SancteTomaszZAkwinuOKrolowaniu2006}}

\vspace{0em}


% ##################
\CenterBoldFont{Uwagi ogólne}

\vspace{0em}


\noindent
Marginesy od~strony brzegu~są zbyt małe, co często utrudnia czytanie.

% \vspace{\spaceFour}





% ##################
\CenterBoldFont{Uwagi do~konkretnych stron}

\vspace{0em}


\noindent
\StrWg{20}{16} Słowa św.~Grzegorza~są tu~przytaczane jako
\textit{Czymże jest władza na~szczycie, jeśli nie burzą umysłu?},
podczas gdy~w~paragrafie~10.5 na~stronie 91 brzmią one \textit{Czymże
  jest burza na~morzu, jeśli nie burzą umysłu?}. Odnośnie tego
problemu warto zobaczyć komentarze tłumacza na stronach 224--225.

\vspace{\spaceFour}





\noindent
\StrWg{32}{2} Obecność wcięcia w~tym wierszu, jest zapewne błędem składu.

\vspace{\spaceFour}





\StrWg{49}{1} Zdanie „Dalej: rzeczy zgodne z~naturą mają
w~sobie doskonałość, bowiem natura posługuje~się jednostkami --~i~tak
jest najlepiej” powinno według mnie brzmieć raczej „Rzeczy zgodne
z~naturą mają w~sobie doskonałość: natura posługuje~się jednostkami
i~tak jest najlepiej”. Jednak bez~znajomości łaciny, nie da~się tego
problemu rozstrzyganą w~sposób merytoryczny. To~samo odnosi~się
do~wszystkich następnych uwaga o~sposobie tłumaczenie tekstu, chyba
że~powiedziano inaczej.

\vspace{\spaceFour}





\noindent
\Str{63} Treść tej~strony sprawia dużo problemów. Po~pierwsze
w~punkcie~6.2 jest mowa o~„dobre tyranii”, a~zgodnie z prowadzoną
klasyfikacją, nie może być czegoś takiego jak tyrania, która jest
dobra. Po~drugie, w~tym samym punkcie jest mowa, że~„dobra tyrania”
nie unicestwia pokoju jaki panuje w~społeczności. Jednak w~punkcie
4.9~na~stronie~55, pisze św.~Tomasz, że~tyrani dla zachowania władzy
zasiewają między poddanymi niezgodę, a~tą która już istnieje
podsycają.

Jak wspomina tłumacz we~„Wstępie”, św.~Tomasz nie ukończył, a~tym
bardziej nie poprawił dzieła, stąd błędy te mogą być tego wynikiem.
Inna możliwość jest taka, że~św.~Tomasz wyłożył tu~błędną, wzajemnie
sprzeczną doktrynę. W~tej sprawie zobacz również uwagi tłumacza
na~stronie~180.

\vspace{\spaceFour}





\noindent
\StrWd{89}{3} Wydaj mi~się, że~zamiast „jeśli jest cnotliwe”
powinno być „jeśli jest własnością cnoty”.

\vspace{\spaceFour}





\noindent
\StrWd{91}{7} Popierając~się pobieżną i~niefachową analizą tekstu
łacińskiego, doszedłem do~wniosku, że~zamiast „ten sam porzuca zwyczaj
czynienia dobrze” powinno być bardziej sensowne w~tym kontekście zdanie
„ten sam porzuca zwyczaj czynienia dobrze w~czasie zamętu”.

\vspace{\spaceFour}





\noindent
\StrWd{101}{13} Jestem słaby z~interpunkcji, mimo tego wydaje
mi~się, że~zamiast „a~kiedy już potrzeba więcej, oni dają królom
z~własnej woli” powinno być „a~kiedy już potrzeba, więcej oni dają
królom z~własnej woli”.

\vspace{\spaceFour}





\noindent
\StrWg{103}{5} Zwykle w~tym wydaniu cytaty łacińskie~są albo~przytaczane
po łacinie, albo tłumaczone na~polski. W~tej linii jednak, cytat jest
przytoczony do~połowy po~łacinie, dalej jest tłumaczenie całości na~polski.

\vspace{\spaceFour}





\noindent
\StrWg{107}{6} Sytuacja taka sama jak na stronie~103, z~tym, że~teraz
po~łacinie przytoczona jest~druga część cytatu.

\vspace{\spaceFour}





\noindent
\textbf{Str.~192, wiersze 15, 14 (od~dołu).} Odstęp między tymi
liniami jest za~duży.

\vspace{\spaceFour}





\noindent
\StrWg{260}{15 i~33} Pod~tymi liniami nie powinien znajdować~się odstęp.

\vspace{\spaceFour}





\noindent
\StrWg{267}{10 i~31} Pod~tymi liniami nie powinien znajdować~się odstęp.

\vspace{\spaceFour}





\noindent
\Str{309} Począwszy od~następującej strony, brak jest numeracji kolejnych
stron.

\vspace{\spaceFour}





\noindent
\StrWg{312}{16} Po~tej linii w~tekście powinien znajdować~się odstęp.

\vspace{\spaceFour}





% ##################
\newpage

\CenterBoldFont{Błędy}

\vspace{\spaceFive}


\begin{center}

  \begin{tabular}{|c|c|c|c|c|}
    \hline
    Strona & \multicolumn{2}{c|}{Wiersz} & Jest
                              & Powinno być \\ \cline{2-3}
    & Od góry & Od dołu & & \\
    \hline
    13  &  6 & & panowala & pawnowała \\
    23  & & 11 & sądzić & wierzyć \\
    24  &  8 & & tu & tu to \\
    29  & & 12 & zawartą & zawartej \\
    90  & & 10 & impune & \textit{impune} \\
    91  & &  1 & impune & \textit{impune} \\
    109 & & 11 & boskiego) & boskiego \\
    111 & &  1 & wszyst & wszyst- \\
    129 & & 11 & Policraticus & \textit{Policraticus} \\
    130 & & 19 & \textit{ypocrita} & \textit{hypocrita} \\
    146 &  5 & & \textit{politeia} & politeia \\
    146 &  6 & & \textit{Politeia}& Politeia \\
    146 & &  9 & \textit{politei} & politei \\
    162 & 11 & & \textit{z~zasady} & z~zasady \\
    163 & &  5 & tyranię & w~tyranię \\
    187 &  3 & & rozdział , & rozdział, \\
    189 & & 17 & \textit{Twarda} & Twarda \\
    209 & &  9 & civitates & \textit{civitates} \\
    209 & &  6 & - \textit{Ibidem.} & \textit{Ibidem.} \\
    240 & & 17 & \textit{ypocrita} & \textit{hypocrita} \\
    240 & &  9 & \textit{ypocrita} & \textit{hypocrita} \\
    241 & &  3 & \textit{DeMalo} & \textit{De malo} \\
    241 & &  1 & DeMalo & \textit{De malo} \\
    244 & 12 & & zasłuży. & zasłuży, \\
    268 & &  2 & \ldots InSent & InSent \\
    286 & 12 & & ma~bowiem był~on & jest \\
    % & & & & \\
    % & & & & \\
    315 & &  2 & Dzieje Polski w~zarysie
    & \textit{Dzieje Polski w~zarysie} \\
    317 & 12 & & Własność & własność \\
    317 & 12 & & Podatki & podatki \\
    317 & 13 & & Gospodarcza & gospodarcza \\
    317 & 13 & & Korupcja & korupcja \\
    317 & 14 & & Granice & granice \\
    \hline
  \end{tabular}

\end{center}

\vspace{\spaceTwo}



% ############################










% ############################
\newpage

\Work{ % Autor i tytuł dzieła
  Sancte Franciszek Salezy \\
  \textit{Filotea}, \cite{SancteFranciszekSalezyFilotea}}

\vspace{0em}


% ##################
\CenterBoldFont{Błędy}

% \vspace{\spaceFive}


\begin{center}

  \begin{tabular}{|c|c|c|c|c|}
    \hline
    Strona & \multicolumn{2}{c|}{Wiersz} & Jest
                              & Powinno być \\ \cline{2-3}
    & Od góry & Od dołu & & \\
    \hline
    10 & 8 & & prowadzićna & prowadzić na \\
    39 & 3 & & \textit{we mnie jest} & \textit{we mnie} \\
    39 & & 7 & rzeź\dywiz wić & rzeźwić \\
    40 & 10 & & szystkim & wszystkim \\
    45 & & 7 & twojemu & ich \\
    56 & & 9 & Ojczyznę.O, & Ojczyznę. O, \\
    59 & & 11 & światowych. & światowych.'' \\
    185 & & 12 & chciwy1 & chciwy$^{ 1 }$ \\
    263 & & 4 & ŚwiętyTomasz & Święty Tomasz \\
    % & & & & \\
    % & & & & \\
    % & & & & \\
    \hline
  \end{tabular}

\end{center}

\vspace{\spaceTwo}


% ############################










% ######################################
\newpage

\section{Dzieła błogosławionych}

\vspace{\spaceTwo}
% ######################################





% ############################
\Work{ % Autor i tytuł dzieła
  Bł. John Henry Newman \\
  \textit{Apologia pro vita sua}, \cite{NewmanApologia2009}}

\vspace{0em}


% ##################
\CenterBoldFont{Uwagi do~konkretnych stron}

\vspace{0em}


\noindent
\StrWd{37}{11} Kiedy ten sam fragment jest cytowany na~stronie~33 święty
Alfons de~Liguori jest tam określany jako święty, tutaj zaś~jako
błogosławiony. Ponieważ jego kanonizacja odbyła~się w~1849 roku, zapewne
w~oryginalnej przytaczanego tekstu jest on określany jako święty.

\vspace{\spaceFour}





\noindent
\StrWd{49}{1} Przez „straszydło na~wróble” to~pewnie tłumaczenie
angielskiego „straw man”. Jeśli tak należy pamiętać, że~w~języku
angielskim funkcjonuje „straw man argument”, choć nie wiem, czy był już
wtedy znany pod tą~nazwą.

\vspace{\spaceFour}





\noindent
\Str{127} Nie jestem w~stanie stwierdzić, co miało znaczyć zdanie „I~przez
tak długi czas nie było możliwe złamanie go w~tym stanie rzeczy, gdyby
wielka zmiana nie zaszła w~warunkach ruchu przeciwnego, który już~się
rozpoczął, aby~mu~stawić opór.”

\vspace{\spaceFour}





\noindent
\StrWd{177}{6} Jeśli Churton zmarł w~1792 roku, w~jaki sposób mógł być
zwolennikiem ruchu oksfordzkiego, który powstał w~XIX~wieku?
Tu~musi być jakiś błąd.

\vspace{\spaceFour}





% ##################
\newpage

\CenterBoldFont{Błędy}

\vspace{\spaceFive}


\begin{center}

  \begin{tabular}{|c|c|c|c|c|}
    \hline
    Strona & \multicolumn{2}{c|}{Wiersz} & Jest
                              & Powinno być \\ \cline{2-3}
    & Od góry & Od dołu & & \\
    \hline
    35  & 12 & & \textit{Części} & części \\
    47  &  3 & & skondensowana & skondensowania \\
    57  &  7 & & tej & tych \\
    61  &  7 & & starta & wytarta \\
    88  & &  8 & pomagającymi & wspomagającymi \\
    117 & 13 & & 3) A~dalej & A~dalej \\
    119 & & 12 & 4) & 3) \\
    122 & & 11 & świat?” & świat? \\
    122 & & 11 & „Ten & >>Ten \\
    122 & & 10 & godzien” & godzien<<  % >>
    \\
    122 & & 10 & „nauczyć & >>nauczyć \\
    122 & &  9 & wyrok” & wyrok<<  % >>
    \\
    130 & &  1 & ukazały, dopóty & ukazały, \\
    134 & 15 & & działo~się & było \\
    139 & & 10 & „Mądrości Bożej” & >>Mądrości Bożej<<”  % >>
    \\
    145 &  1 & & fakt; & fakt \\
    160 & & 11 & co~innego & coś~innego \\
    160 & &  9 & znajdziemy & znajdziemy tam \\
    167 &  2 & & ustępowałem; & ustępowałem, \\
    174 & 12 & & uważa & uważam \\
    176 &  3 & & praktycznego”\ldots „Bogate & praktycznego”, „Bogate \\
    176 & &  5 & przeciwnik & czynnik \\
    178 & &  8 & tak & tak~to \\
    182 &  9 & & intuicje & intuicje odnośnie \\
    182 & 13 & & znosić & zaprzeczać \\
    182 & & 11 & przeciwieństwo & przeciwieństwa \\
    182 & &  1 & „tak” & >>tak<<  % >>
    \\
    182 & &  1 & „nie” & >>nie<<”  % >>
    \\
    183 & 10 & & życzący & nie życzący \\
    193 & & 11 & dzieciństwem” & dzieciństwem \\
    194 & &  8 & punkt & argument \\
    194 & &  7 & punktem & argumentem \\
    194 & &  7 & punktem & argumentem \\
    195 &  3 & & pewien punkt & pewne rzeczy \\
    199 & 11 & & 1156 & 1556 \\
    203 & &  2 & stale & stałe \\
    221 & &  5 & puseyizmu”. & puseyizmu. \\
    224 & & 12 & wnioskiem. & wnioskiem”. \\
    227 & & 11 & Liście & liście \\
    % & & & & \\
    % & & & & \\
    % & & & & \\
    % & & & & \\
    % & & & & \\
    % & & & & \\
    \hline
  \end{tabular}





  \newpage

  \begin{tabular}{|c|c|c|c|c|}
    \hline
    & \multicolumn{2}{c|}{} & & \\
    Strona & \multicolumn{2}{c|}{Wiersz} & Jest
                              & Powinno być \\ \cline{2-3}
    & Od góry & Od dołu & & \\
    \hline
    250 &  5 & & nieistniejącego & istniejącego \\
    254 &  9 & & więzieniu. & więzieniu.'' \\
    254 & 10 & & Boże & „Boże \\
    254 & 10 & & „Śniłem & Śniłem \\
    259 & &  3 & mnie & niej \\
    % & & & & \\
    % & & & & \\
    % & & & & \\
    % & & & & \\
    % & & & & \\
    % & & & & \\
    % & & & & \\
    % & & & & \\
    % & & & & \\
    % & & & & \\
    % & & & & \\
    % & & & & \\
    \hline
  \end{tabular}

\end{center}

\vspace{\spaceTwo}


\noindent
\StrWg{39}{7--8} \\
\Jest  zdobyła~się jedynie na~słowa \\
\Powin usłyszała jedynie słowa \\


% ############################










% ######################################
\newpage
\section{Pozostali autorzy}

\vspace{\spaceTwo}
% ######################################



% ############################
\Work{ % Autor i tytuł dzieła
  Dante \\
  \textit{Komedia}, \cite{DAK}}

\vspace{0em}


% ##################
\CenterBoldFont{Błędy}

% \vspace{\spaceFive}


\begin{center}

  \begin{tabular}{|c|c|c|c|c|}
    \hline
    Strona & \multicolumn{2}{c|}{Wiersz} & Jest
                              & Powinno być \\ \cline{2-3}
    & Od góry & Od dołu & & \\
    \hline
    % 347 & 21 & & ,,Jestem & Jestem \\
    350 & & 9 & jed- nak & jednak \\
    % & & & & \\
    % & & & & \\
    % & & & & \\
    \hline
  \end{tabular}

\end{center}

\vspace{\spaceTwo}


Piekło, Pieśń IV, 131: \ldots którzy wiedzą, Czyściec, Pieśń 1, 44: by was
z tej nocy\ldots


% ############################










% ############################
\newpage

\Work{ % Autor i tytuł dzieła
  Elio Guerriero \\
  \textit{Hans Urs von Balthasar},
  \cite{GuerrieroHansUrsVonBalthasarMonografia2004}}

\vspace{0em}


% ##################
\CenterBoldFont{Uwagi do~konkretnych stron}

\vspace{0em}


\noindent
\Str{41} Pierwszy akapit na tej kończy się cudzysłowem zamykającym, lecz
nie ma w nim nigdzie cudzysłowu otwierającego.

% \vspace{\spaceFour}





% ##################
\CenterBoldFont{Błędy}

\vspace{\spaceFive}


\begin{center}

  \begin{tabular}{|c|c|c|c|c|}
    \hline
    Strona & \multicolumn{2}{c|}{Wiersz} & Jest
                              & Powinno być \\ \cline{2-3}
    & Od góry & Od dołu & & \\
    \hline
    17  & 12 & & \textit{Karl. Barth.} & \textit{Karl Barth.} \\
    139 & &  6 & Tage & \textit{Tage} \\
    153 & &  4 & (1972);(pol. & (1972); (pol. \\
    153 & &  2 & ($^{ 2 }$1988) & (1988) \\
    217 & & 14 & eologa & teologa \\
    % & & & & \\
    % & & & & \\
    % & & & & \\
    \hline
  \end{tabular}

\end{center}
% ############################











% ######################################
\newpage

\section{Święta liturgia}

\vspace{\spaceTwo}
% ######################################



% ############################
\subsection{Święta liturgia po 1962~r.}

\vspace{\spaceThree}
% ############################



% ############################
\Work{ % Autor i tytuł dzieła
  Michael Davies \\
  \textit{Zniszczenie Mszy Świętej czyli~Godly Order Cranmera},
  \cite{DaviesZniszczenieMszySwietej2016} }

\vspace{0em}


% ##################
\CenterBoldFont{Uwagi}

\vspace{0em}


\noindent
Jak wiele książek z~tego wydawnictwa, ta~również ta nie przeszła
odpowiedniego procesu redakcji i~jest pod~wieloma względami źle zedytowana.

\vspace{\spaceFour}





\noindent
W~procesie tworzenia wydania polskiego przepadła gdzieś bibliografia
i~indeks skrótów.





% ##################
\newpage

\CenterBoldFont{Błędy}

\vspace{\spaceFive}


\begin{center}

  \begin{tabular}{|c|c|c|c|c|}
    \hline
    Strona & \multicolumn{2}{c|}{Wiersz} & Jest
                              & Powinno być \\ \cline{2-3}
    & Od góry & Od dołu & & \\
    \hline
    15  &  2 & & z~zbiorach & w~zbiorach \\
    24  & &  6 & stad & stąd \\
    24  &  2 & & woli” & woli \\
    33  &  9 & & ja & ją \\
    37  &  5 & & zapomniało waszym & zapomniał o~waszym \\
    % & & & & \\
    % & & & & \\
    % & & & & \\
    \hline
  \end{tabular}

\end{center}

\vspace{\spaceTwo}


\noindent
\textbf{Tylna okładka.} \\
\Jest  \textit{arcydziełem}”. \\
\Powin \textit{arcydziełem} \\


% ############################










% ############################
\Work{ % Autor i tytuł dzieła
  Michael Davies \\
  \textit{Sobór Papieża Jana. Rewolucja liturgiczna}, \cite{}}


% ##################
\CenterBoldFont{Błędy}

\vspace{\spaceFive}


\begin{center}

  \begin{tabular}{|c|c|c|c|c|}
    \hline
    Strona & \multicolumn{2}{c|}{Wiersz} & Jest
                              & Powinno być \\ \cline{2-3}
    & Od góry & Od dołu & & \\
    \hline
    11 & & 2 & \textit{musimocna} & \textit{musi mocno} \\
    41 & 10 & & „Patrząc & Patrząc \\
    46 & & 2 & Westminster & Westminster. \\
    % 81 & & 17 & ,,wyczuwalny & ,,Wyczuwalny \\
    97 & & 3 & wprowadzen- ia & wprowadzenia \\
    176 & & 9 & presji.„Czy & presji. „Czy \\
    178 & 7 & & \textit{Novost}i & \textit{Novosti} \\
    183 & 5 & & protestanckich & prawosławnych \\
    188 & 14 & & wary & wiary \\
    203 & 14 & & podporządkować. & podporządkować.” \\
    204 & & 15 & małżeństwa. & małżeństwa.” \\
    209 & 3 & & zarazić. & zarazić.” \\
    209 & 6 & & pracę.” & pracę. \\
    243 & & 17 & pominiecie & pominięcie \\
    245 & 7 & & dokumentu. & dokumentu.” \\
    250 & 17 & & nie byłaby & byłaby \\
    263 & 8 & & „<<Wzorcowa>>” & ”<<Wzorcowa>> \\
    265 & 13 & & tam, że & tam \\
    265 & 15 & & ewolucji, & ewolucji. \\
    310 & 5 & & obór & Sobór \\
    323 & 3 & & wyobrazili & wyobrażali \\
    323 & 11 & & nieuzasadnionym & nie uzasadnionym \\
    325 & & 8 & Nie & „Nie \\
    326 & 7 & & władzę. & władzę.” \\
    327 & 3 & & 272 & 322 \\
    330 & 12 & & Podam -- nawet & Podam nawet \\
    330 & 13 & & to & jest \\
    359 & 3 & & 59\% & 41\% \\
    \hline
  \end{tabular}

\end{center}

\vspace{\spaceTwo}


\noindent
\StrWg{44}{16} \\
\Jest  uprawnionych do głosowania \\
\Powin które można głosować \\
\StrWg{174}{4} \\
\Jest  umożliwiaporozumieniezchrześcijaństwem.Strukturalnacałość\textit{DasKapital}
\\
\Powin umożliwia porozumienie z chrześcijaństwem. Strukturalna całość
\textit{Das Kapital} \\


% ############################










% ############################
\newpage

\Work{ % Autor i tytuł dzieła
  M. Davies \\
  \textit{Sobór Watykański II a~wolność religijna},
  \cite{DaviesSoborAWolnoscReligina2002}}


% ##################
\CenterBoldFont{Błędy}

% \vspace{\spaceFive}


\begin{center}

  \begin{tabular}{|c|c|c|c|c|}
    \hline
    & \multicolumn{2}{c|}{} & & \\
    Strona & \multicolumn{2}{c|}{Wiersz} & Jest
                              & Powinno być \\ \cline{2-3}
    & Od góry & Od dołu & & \\
    \hline
    84  &  2 & & społeczeństwie” & „społeczeństwie” \\
    109 & & 15 & nacjonalizmu & racjonalizmu \\
    118 & & 10 & był & został \\
    155 &  9 & & nie & się \\
    193 & 17 & & nie są & są \\
    258 & &  5 & \textit{rolą} & \textit{z rolą} \\
    262 & 17 & & \textit{upadku} & \textit{upadku.} \\
    269 & 12 & & \textit{Quanta cura} & \textit{Quas primas} \\
    269 & 17 & & \textit{Quanta cura} & \textit{Quas primas} \\
    333 & & 13 & jest & co jest \\
    % & & & & \\
    \hline
  \end{tabular}

\end{center}
% ############################










% ############################
\newpage

\Work{ % Autor i tytuł dzieła
  Tracey Rowland \\
  \textit{Wiara Ratzingera, teologia Benedykta XVI},
  \cite{RowlandWiaraRatzingera2010}}

\vspace{0em}


% ##################
\CenterBoldFont{Uwagi do~konkretnych stron}

\vspace{0em}


\noindent
\Str{50} Stwierdzenie, że~św. Jan Paweł II doszedł do~teodramatyki poprzez
tomizm i~fenomenologię, jest mocno wątpliwe. O~ile pamiętam w~pierwszym
artykule z~\cite{PoslugaMysleniaTomIX2011}, jest podane, że dopiero na
studiach seminaryjnych poprzez standardowy podręcznik metafizyki~(?), Karol
Wojtyła zetknął~się z~tomizmem. (Możliwe, że~w~tej pracy podane jest też,
kiedy pierwszy raz spotkał~się z~fenomenologią.) Natomiast co najmniej od
lat szkolnych był zafascynowany teatrem, str.~14--15
\cite{NowakJanPawelIIKronikaZyciaIPontyfikatu2015}. Zanim w~październiku
1942 r. podjął decyzję o zostaniu kapłanem był m.in.~od~kilku lat
zaangażowany w~działalność Teatru Rapsodycznego, z~którego twórcą
i~kierownikiem Mieczysławem Kotlarczykiem prowadził dyskusje, na tematy
wiążące, religię, filozofię, patriotyzm i~sztukę, patrz np.~str.~19--34
w~\cite{NowakJanPawelIIKronikaZyciaIPontyfikatu2015}. Wydaje~się dużo
bardziej prawdopodobne, że~właśnie z~jego doświadczeń teatralnych
wyrosła jego pasja do~teodramatyki.

\vspace{\spaceFour}





\noindent
\Str{52} Choć nie znam twórczości Ryszarda Legutko, wciąż
wydaje mi~się bardzo wątpliwym stwierdzeniem określenie go mianem
teologa. Osobną sprawą jest jego związek z~tradycją tomistyczną.





% ##################
\CenterBoldFont{Błędy}

% \vspace{\spaceFive}


\begin{center}

  \begin{tabular}{|c|c|c|c|c|}
    \hline
    Strona & \multicolumn{2}{c|}{Wiersz} & Jest
                              & Powinno być \\ \cline{2-3}
    & Od góry & Od dołu & & \\
    \hline
    21  & 16 & & 1844--1970 & 1844--1900 \\
    27  & &  9 & transcendentalne piękno & transcendentalnym pięknem \\
    46  & & 13 & 1947 & 1847 \\
    71  & & 11 & „przyjścia & >>przyjścia \\
    71  & &  7 & domu” & domu<< \\
    78  & 12 & & ona z~tego & z~niej \\
    \hline
  \end{tabular}

\end{center}
% ############################









% ############################
\newpage

\Work{ % Autor i tytuł dzieła
  R. M. Wiltgen \\
  \textit{Ren wpada do Tybru}, \cite{WiltgenRenWpadaDoTybru2010}}


% ##################
\CenterBoldFont{Błędy}


\begin{center}

  \begin{tabular}{|c|c|c|c|c|}
    \hline
    Strona & \multicolumn{2}{c|}{Wiersz} & Jest
                              & Powinno być \\ \cline{2-3}
    & Od góry & Od dołu & & \\
    \hline
    18  &  5 & & dwunastu do czterech & dwunastu \\
    40  & & 16 & te & że \\
    42  &  3 & & Kantonu(Dahomej) & Kantonu (Dahomej) \\
    44  & &  6 & konferencji,tego & konferencji, tego \\
    % & & & & \\
    % & & & & \\
    % & & & & \\
    % & & & & \\
    \hline
  \end{tabular}

\end{center}
% ############################










% #####################################################################
% #####################################################################
% Bibliografia

\bibliographystyle{plalpha}

\bibliography{DEUSBooks}{}





% ############################

% Koniec dokumentu
\end{document}
