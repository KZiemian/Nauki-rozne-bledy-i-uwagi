% ---------------------------------------------------------------------
% Basic configuration and packages
% ---------------------------------------------------------------------
% Package for discovering wrong and outdated usage of LaTeX.
% More information to be found in l2tabu English version.
\RequirePackage[l2tabu, orthodox]{nag}
% Class of LaTeX document: {size of paper, size of font}[document class]
\documentclass[a4paper,11pt]{article}



% ---------------------------------------
% Packages not tied to particular normal language
% ---------------------------------------
% This package should improved spaces in the text.
\usepackage{microtype}



% ---------------------------------------
% Polonization of LaTeX document
% ---------------------------------------
% Basic polonization of the text
\usepackage[MeX]{polski}
% Switching on UTF-8 encoding
\usepackage[utf8]{inputenc}
% Adding font Latin Modern
\usepackage{lmodern}
% Package is need for fonts Latin Modern
\usepackage[T1]{fontenc}



% ---------------------------------------
% Setting margins
% ---------------------------------------
% Package for easy settings of margins. Unit of measurement is inch.
\usepackage{vmargin}
\setmarginsrb
{ 0.7in}  % left margin
{ 0.6in}  % top margin
{ 0.7in}  % right margin
{ 0.8in}  % bottom margin
{  20pt}  % head height
{0.25in}  % head sep
{   9pt}  % foot height
{ 0.3in}  % foot sep



% ---------------------------------------
% Setting vertical spaces in the text
% ---------------------------------------
% Setting space between lines
\renewcommand{\baselinestretch}{1.1}

% Setting space between lines in tables
\renewcommand{\arraystretch}{1.4}





% ------------------------------
% Private packages
% You need to put them in the same directory as .tex file
% ------------------------------
% Contains various command useful for working with a text
\usepackage{latexgeneralcommands}





% ------------------------------
% Package ``hyperref''
% They advised to put it on the end of preambule
% ------------------------------
% It allows you to use hyperlinks in the text
\usepackage{hyperref}










% ---------------------------------------------------------------------
% Defining title and author of the text
\title{Historia Chrześcijaństwa Carrollów \\
  {\Large Błędy i~uwagi}}

\author{Kamil Ziemian}


% \date{}
% ---------------------------------------------------------------------










% ####################################################################
% Title of the text
\begin{document}
% ####################################################################





% ######################################
\maketitle  % Tytuł całego tekstu
% ######################################









% ######################################
\section{Ks. Bogusław Kumor
  \textit{Historia Kościoła. Tom~I: Starożytność chrześcijańska},
  \cite{KumorHistoriaKosciolaVolI2003}}
% ######################################



% ##################
\CenterBoldFont{Błędy}


\begin{center}

  \begin{tabular}{|c|c|c|c|c|}
    \hline
    Strona & \multicolumn{2}{c|}{Wiersz} & Jest
                              & Powinno być \\ \cline{2-3}
    & Od góry & Od dołu & & \\
    \hline
    14 & 11 & & rzeciwieństwie & przeciwieństwie \\
    14 & 15 & & jednej formy & jedną formę \\
    % & & & & \\
    % & & & & \\
    % & & & & \\
    % & & & & \\
    % & & & & \\
    % & & & & \\
    \hline
  \end{tabular}

\end{center}

% \vspace{\spaceTwo}



% ############################










% ######################################
\newpage

\section{Historia świętej wiary, I~wiek przed i~I po Chrystusie}

% \vspace{\spaceTwo}
% ######################################



% ############################
{ % Autor i tytuł dzieła
  Wojciech Roszkowski \\
  \textit{Świat Chrystusa. Tom~I}, \cite{RoszkowskiSwiatChrystusVolI2016}}

\vspace{0em}


% ##################
\CenterBoldFont{Uwagi do~konkretnych stron}

\vspace{0em}


\noindent
\StrWierszG{63}{24} Wydaje mi się, że~zaraz przed zdaniem
„Oskarżycielem Mariammy\ldots” powinien znajdować~się tekst mówiący
o~tym, że~Herodes postawił ją przed sądem. Prawdopodobnie został on
zgubiony w~trakcie składania książki.

\VerSpaceFour





\noindent
\StrWierszG{81}{4} Zdanie „miłość skupia i~uruchamia wszystkie cząstki\ldots”
brzmi bardzo niezręcznie i~powinno być jakoś poprawione.

\VerSpaceFour





\noindent
\StrWierszG{84}{8} Podane tu twierdzenie jest prostym wnioskiem
z~twierdzenia Pitagorasa, więc nie wydaje~się, że~powinno ono uchodzić
za~jakieś
szczególnie warte uwagi osiągnięcie Platona. Przynajmniej nie uchodziłoby
za takie dzisiaj, jednak ponieważ matematyka przeszła od V i~IV wieku przed
Chrystusem długą drogę, możliwe, iż w~tamtych czasach znalezienie takiej
konsekwencji twierdzenia Pitagorasa było bardzo ważnym odkryciem.

\VerSpaceFour





\noindent
\StrWierszG{85}{10} Zdanie „przez pomiar kierunków, w~których punkt ten jest
widziany z~dwóch punktów o~znanych odległościach” brzmi niezręcznie i~chyba
warto byłoby je poprawione.

\VerSpaceFour





\noindent
\StrWierszD{95}{5} Nie rozumiem co oznacza przytoczone słowo „kpy”. Może
jest to błędne zapisane słowo „kiepy”?

\VerSpaceFour



% \start \Str{}

% \vspace{\spaceFour}

% \start \Str{}

% \vspace{\spaceFour}

% \start \Str{}

% \vspace{\spaceFour}

% \start \Str{}


% ##################
\newpage

\CenterBoldFont{Błędy}


\begin{center}

  \begin{tabular}{|c|c|c|c|c|}
    \hline
    Strona & \multicolumn{2}{c|}{Wiersz} & Jest
                              & Powinno być \\ \cline{2-3}
    & Od góry & Od dołu & & \\
    \hline
    \hphantom{0}6 & & 12 & ona na & na \\
    19 & 21 & & Ciągłą rachubę & Rachubę \\
    19 & & \hphantom{0}5 & \textit{Time.The} & \textit{Time. The} \\
    31 & & \hphantom{0}2 & passim, Kasjusz & passim; Kasjusz \\
    31 & & \hphantom{0}2 & passim\textit{;} & passim; \\
    33 & & \hphantom{0}3 & położona & położona jest \\
    34 & \hphantom{0}1 & & jest na & na \\
    37 & & \hphantom{0}1 & 2001) & 2001 \\
    38 & & \hphantom{0}2 & \textit{Jasna} & \textit{Yasna} \\
    43 & & \hphantom{0}7 & 1985) & 1985 \\
    43 & & \hphantom{0}2 & W.Malandra & W.~Malandra \\
    43 & & \hphantom{0}2 & //garodman.htm & /garodman.htm \\
    44 & & \hphantom{0}7 & part\textbf{1} & part~1 \\
    45 & & \hphantom{0}4 & \textit{H.Dubs} & \textit{H.~Dubs} \\
    % 47 Ware: Hertfordshire: Wordsworth?????
    49 & & \hphantom{0}9 & Obyczajów & Obyczajów” \\
    58 & & \hphantom{0}5 & Wojna żydowska & \textit{Wojna żydowska} \\
    58 & & \hphantom{0}2 & J.Harrington & J. Harrington \\
    62 & & \hphantom{0}2 & Życie Jezusa Chrystusa
    & \textit{Życie Jezusa Chrystusa} \\
    62 & & \hphantom{0}2 & PX & PAX \\
    65 & & \hphantom{0}1 & s.153--155; & s.~153--155. \\
    70 & \hphantom{0}9 & & zdobyć Rzymowi & zdobyć \\
    70 & & 14 & bowiem także & bowiem \\
    72 & & \hphantom{0}1 &  Dzieje od~założenia miasta
    & \textit{Dzieje od~założenia miasta} \\
    76 & & 14 & pierwszy & pierwszy patrycjusz \\
    77 & \hphantom{0}5 & & żoną, Vipsanią, & żoną Vipsanią, \\
    77 & \hphantom{0}5 & & syna, Drususa. & syna Drususa. \\
    78 & \hphantom{0}3 & & Po & W~miejscach po \\
    % 79?????
    84 & & \hphantom{0}2 & 1970) & 1970 \\
    84 & & \hphantom{0}2 & 2000) & 2000 \\
    \hline
  \end{tabular}





  \newpage

  \begin{tabular}{|c|c|c|c|c|}
    \hline
    Strona & \multicolumn{2}{c|}{Wiersz} & Jest
                              & Powinno być \\ \cline{2-3}
    & Od góry & Od dołu & & \\
    \hline
    84 & & \hphantom{0}1 & Słownik f\textit{ilozofów}
    & \textit{Słownik filozofów} \\
    \hphantom{0}87 & & 16 & strumienia & strumieniem \\
    \hphantom{0}99 & & \hphantom{0}2 & s.188--194 & s.~188--194 \\
    \hphantom{0}99 & & \hphantom{0}2 & J.Harrington & J. Harrington \\
    100 & & \hphantom{0}2 & Wydawnictwo~m & Wydawnictwo~M \\
    105 & & \hphantom{0}8 & Diocletianus & \textit{Diocletianus} \\
    108 & & \hphantom{0}1 & Złota gałąź & \textit{Złota gałąź} \\
    110 & & \hphantom{0}6 & PMich & P. Mich \\
    112 & & 12 & Tereny Alpes Maritimae, & Tereny \\
    112 & & \hphantom{0}2 & Mówią wykopaliska
    & \textit{Mówią wykopaliska} \\
    115 & & \hphantom{0}9 & Berenikę & Berenika \\
    126 & & \hphantom{0}2 & \textit{A.D.},Guntur & \textit{A.D.}, Guntur \\
    126 & & \hphantom{0}1 & P.G.Publishers & P.G. Publishers \\
    127 & & \hphantom{0}2 & starożytnego Rzymu
    & \textit{starożytnego Rzymu} \\
    132 & & \hphantom{0}1 & Życie codzienne & \textit{Życie codzienne} \\
    144 & & \hphantom{0}2 & Wydawnic zy & Wydawniczy \\
    166 & & \hphantom{0}1 & Psalmów,114 & Psalmów, 114 \\
    174 & & \hphantom{0}1 & frg & frg. \\
    175 & & \hphantom{0}5 & G.Miller & G. Miller \\
    175 & & \hphantom{0}5 & Starożytni & \textit{Starożytni} \\
    175 & & \hphantom{0}4 & W.Gołębiewski & W. Gołębiewski \\
    175 & & \hphantom{0}3 & J.Stroynowski & J. Strounowski \\
    179 & & \hphantom{0}1 & Starożytni olimpijczycy
    & \textit{Starożytni olimpijczycy} \\
    180 & & \hphantom{0}2 & Wędrówki po~Helladzie
    & \textit{Wędrówki po~Helladzie} \\
    180 & & \hphantom{0}2 & Starożytni olimpijczycy
    & \textit{Starożytni olimpijczycy} \\
    182 & & \hphantom{0}2 & www1.fhw.gr & www.fhw.gr \\
    185 & & \hphantom{0}3 & B.F.Gajdukiewicz & B.F. Gajdukiewicz \\
    185 & & \hphantom{0}2 & M.I.Rostwcew & M.I. Rostwcew \\
    186 & & \hphantom{0}1 & 1994) & 1994 \\
    187 & & \hphantom{0}3 & B.Baranowski & B. Baranowski \\
    \hline
  \end{tabular}





  \newpage

  \begin{tabular}{|c|c|c|c|c|}
    \hline
    Strona & \multicolumn{2}{c|}{Wiersz} & Jest
    & Powinno być \\ \cline{2-3}
    & Od góry & Od dołu & & \\
    \hline
    188 & & \hphantom{0}2 & Historia Azerbejdżanu
    & \textit{Historia Azerbejdżanu} \\
    190 & & \hphantom{0}1 & własne jęz. & własne z~jęz. \\
    191 & & \hphantom{0}4 & H.Ananikian & H. Ananikian \\
    199 & & \hphantom{0}1 & Starożytni Grecy i~Rzymianie
    & \textit{Starożytni Grecy i~Rzymianie} \\
    199 & & \hphantom{0}1 & Dzieje od~założenia Miasta
           & \textit{Dzieje od~założenia Miasta} \\
    205 & & \hphantom{0}5 & B.T.Batsford & B.T. Batsford \\
    226 & & \hphantom{0}1 & 1989) & 1989 \\
    244 & & \hphantom{0}4 & Rozmowy tuskulańskie
    & \textit{Rozmowy tuskulańskie} \\
    245 & & \hphantom{0}6 & Ody & \textit{Ody} \\
    250 & & \hphantom{0}1 & Osły & \textit{Osły} \\
    253 & & \hphantom{0}2 & CAIS//History & CAIS/History \\
    255 & & \hphantom{0}4 & D.P.M.Weerakkody & D.P.M. Weerakkody \\
    299 & & \hphantom{0}1 & konfucjańskie & \textit{konfucjańskie} \\
    303 & & \hphantom{0}6 & Dialogi konfucjańskie
    & \textit{Dialogi konfucjańskie} \\
    330 & & \hphantom{0}2 & L.Shinne & L. Shinne \\
    332 & & \hphantom{0}2 & prośba & prośba'' \\
    333 & & \hphantom{0}4 & W.Malandra & W. Malandra \\
    346 & & \hphantom{0}3 & Pieśni & \textit{Pieśni} \\
    346 & & \hphantom{0}3 & Commentary on~Catullus
    & „Commentary on~Catullus” \\
    347 & & \hphantom{0}3 & Dialogi konfucjańskie
    & \textit{Dialogi konfucjańskie} \\
    354 & & \hphantom{0}2 & 1957) & 1957 \\
    360 & & \hphantom{0}3 & Starożytni Celtowie
    & \textit{Starożytni Celtowie} \\
    367 & & \hphantom{0}2 & A.Barton & A. Barton \\
    383 & & \hphantom{0}6 & \textit{History} (Dublin:
    & \textit{History}, Dublin: \\
    387 & & \hphantom{0}1 & O~wróżbiarstwie & \textit{O~wróżbiarstwie} \\
    387 & & \hphantom{0}1 & Słowo jest cieniem
    & \textit{Słowo jest cieniem} \\
    394 & & \hphantom{0}4 & A.Bard & A. Bard \\
    395 & & \hphantom{0}2 & Poznaj Świat & „Poznaj Świat” \\
    % 396, PMich ????
    398 & & \hphantom{0}6 & A.Bard & A. Bard \\
    408 & & \hphantom{0}2 & 2010 & 2010) \\
    \hline
  \end{tabular}





  \newpage

  \begin{tabular}{|c|c|c|c|c|}
    \hline
    Strona & \multicolumn{2}{c|}{Wiersz} & Jest
                              & Powinno być \\ \cline{2-3}
    & Od góry & Od dołu & & \\
    \hline
    414 & & \hphantom{0}4 & \textit{Boski, August}
    & \textit{Boski August} \\
    421 & & \hphantom{0}3 & Starożytna Polska
    & \textit{Starożytna Polska} \\
    424 & & \hphantom{0}1 & Mitologia germańska
    & \textit{Mitologia germańska} \\
    424 & & \hphantom{0}1 & 1979) & 1979 \\
    429 & & \hphantom{0}4 & Żaby & \textit{Żaby} \\
    429 & & \hphantom{0}4 & \textit{komedie} & \textit{Komedie} \\
    430 & & \hphantom{0}5 & C.G.Jung & C.G.~Jung \\
    430 & & \hphantom{0}5 & C.Kerenyi & C.~Kerenyi \\
    %   & & & & \\
    %   & & & & \\
    %   & & & & \\
    %   & & & & \\
    %   & & & & \\
    %   & & & & \\
    %   & & & & \\
    %   & & & & \\
    %   & & & & \\
    %   & & & & \\
    %   & & & & \\
    %   & & & & \\
    %   & & & & \\
    %   & & & & \\
    %   & & & & \\
    %   & & & & \\
    %   & & & & \\
    %   & & & & \\
    %   & & & & \\
    %   & & & & \\
    %   & & & & \\
    %   & & & & \\
    %   & & & & \\
    %   & & & & \\
    %   & & & & \\
    %   & & & & \\
    %   & & & & \\
    \hline
  \end{tabular}





  % \begin{tabular}{|c|c|c|c|c|}
  %   \hline
  %   & \multicolumn{2}{c|}{} & & \\
  %   Strona & \multicolumn{2}{c|}{Wiersz} & Jest
  %                             & Powinno być \\ \cline{2-3}
  %   & Od góry & Od dołu & & \\
  %   \hline
  %   %   & & & & \\
  %   %   & & & & \\
  %   %   & & & & \\
  %   %   & & & & \\
  %   %   & & & & \\
  %   %   & & & & \\
  %   %   & & & & \\
  %   %   & & & & \\
  %   %   & & & & \\
  %   %   & & & & \\
  %   %   & & & & \\
  %   %   & & & & \\
  %   %   & & & & \\
  %   %   & & & & \\
  %   %   & & & & \\
  %   %   & & & & \\
  %   %   & & & & \\
  %   %   & & & & \\
  %   %   & & & & \\
  %   %   & & & & \\
  %   %   & & & & \\
  %   %   & & & & \\
  %   %   & & & & \\
  %   %   & & & & \\
  %   %   & & & & \\
  %   %   & & & & \\
  %   %   & & & & \\
  %   \hline
  % \end{tabular}

\end{center}

\VerSpaceSix


\noindent
\StrWierszD{18}{3} \\
\Jest  Dicta. Zbiór łacińskich sentencji, przysłów, zwrotów, powiedzeń
z~indeksem osobowym i~tematycznym, zebrał Czesław Michalunio~SJ \\
\Powin \textit{Dicta. Zbiór łacińskich sentencji, przysłów, zwrotów,
  powiedzeń, z~indeksem osobowym i~tematycznym,
  zebrał Czesław Michalunio~SJ} \\
\StrWierszD{19}{7} \\
\Jest  Geneza chrześcijańskiej rachuby lat (Tyniec: Wydawnictwo
Benedyktynów, 2000); \\
\Powin \textit{Geneza chrześcijańskiej rachuby lat}, Tyniec: Wydawnictwo
Benedyktynów, 2000; \\
\StrWierszD{37}{6} \\
\Jest  Religie świata rzymskiego \\
\Powin \textit{Religie świata rzymskiego} \\
\StrWierszD{37}{6} \\
\Jest  Historia wierzeń i~idei religijnych \\
\Powin \textit{Historia wierzeń i~idei religijnych} \\
\StrWierszD{37}{5} \\
\Jest  Historia wierzeń i~idei religijnych \\
\Powin \textit{Historia wierzeń i~idei religijnych} \\
\StrWierszD{37}{4} \\
\Jest  Starożytne cywilizacje. Wierzenia, mitologia, sztuka \\
\Powin \textit{Starożytne cywilizacje. Wierzenia, mitologia, sztuka} \\
\StrWierszD{37}{2} \\
\Jest  Wielki atlas mitów i~legend świata \\
\Powin \textit{Wielki atlas mitów i~legend świata} \\
\StrWierszD{51}{2} \\
\Jest  Starożytne cywilizacje. Wierzenia, mitologia, sztuka \\
\Powin \textit{Starożytne cywilizacje. Wierzenia, mitologia, sztuka} \\
\StrWierszD{51}{1} \\
\Jest  Historia wierzeń i~idei religijnych \\
\Powin \textit{Historia wierzeń i~idei religijnych} \\
\StrWierszD{58}{3} \\
\Jest  Biblia i~starożytny świat \\
\Powin \textit{Biblia i~starożytny świat} \\
\StrWierszD{58}{2} \\
\Jest  Od~Księgi Rodzaju do~Ewangelii \\
\Powin \textit{Od~Księgi Rodzaju do~Ewangelii} \\
\StrWierszD{66}{4} \\
\Jest  Księdze o~narodzinach Świetej Maryi \\
\Powin \textit{Księdze o~narodzinach Świętej Maryi} \\
\StrWierszD{70}{4} \\
\Jest  Mała encyklopedia kultury antycznej \\
\Powin \textit{Mała encyklopedia kultury antycznej} \\
\StrWierszD{73}{4} \\
\Jest  Starożytni Grecy i~Rzymianie w~życiu prywatnym i~państwowym \\
\Powin \textit{Starożytni Grecy i~Rzymianie w~życiu prywatnym
  i~państwowym} \\
\StrWierszD{73}{3} \\
\Jest  Życie codzienne w~Rzymie \\
\Powin \textit{Życie codzienne w~Rzymie} \\
\StrWierszD{73}{2} \\
\Jest  Census populi. Demografia starożytnego Rzymu \\
\Powin \textit{Census populi. Demografia starożytnego Rzymu} \\
\StrWierszD{76}{2} \\
\Jest  Starożytni Grecy i~Rzymianie w~życiu prywatnym i~państwowym \\
\Powin \textit{Starożytni Grecy i~Rzymianie w~życiu prywatnym
  i~państwowym} \\
\StrWierszD{91}{3} \\
\Jest  O~najwyższy dobru i~złu \\
\Powin \textit{O~najwyższym dobru i~złu} \\
\StrWierszD{92}{2} \\
\Jest  Grecy o~miłości, szczęściu i~życiu \\
\Powin \textit{Grecy o~miłości, szczęściu i~życiu} \\
\StrWierszD{96}{3} \\
\Jest  Grecy o~miłości, szczęściu i~życiu \\
\Powin \textit{Grecy o~miłości, szczęściu i~życiu} \\
\StrWierszD{96}{1} \\
\Jest  Grecy o~miłości, szczęściu i~życiu \\
\Powin \textit{Grecy o~miłości, szczęściu i~życiu} \\
\StrWierszD{99}{3} \\
\Jest  Życie Jezusa Chrystusa \\
\Powin \textit{Życie Jezusa Chrystus} \\
\StrWierszD{100}{3} \\
\Jest  Geneza chrześcijańskiej rachuby lat \\
\Powin \textit{Geneza chrześcijańskiej rachuby lat} \\
\StrWierszD{105}{7} \\
\Jest  Prowincje i~społeczeństwa prowincjonalne we~wschodniej części
basenu Morza Śródziemnego w~okresie od~Augusta do~Sewerów
(31 r.p.n.e. --~235 r.n.e) \\
\Powin \textit{Prowincje i~społeczeństwa prowincjonalne we~wschodniej
  części basenu Morza Śródziemnego w~okresie od~Augusta do~Sewerów
  (31 r.~p.n.e. -- 235 r.~n.e)} \\
\StrWierszD{105}{2} \\
\Jest  Bóstwa, kulty i~rytuały starożytnego Egiptu \\
\Powin \textit{Bóstwa, kulty i~rytuały starożytnego Egiptu} \\
\StrWierszD{109}{1} \\
\Jest  Bóstwa, kulty i~rytuały starożytnego Egiptu \\
\Powin \textit{Bóstwa, kulty i~rytuały starożytnego Egiptu} \\
\StrWierszD{110}{3} \\
\Jest  Bóstwa, kulty i~rytuały starożytnego Egiptu \\
\Powin \textit{Bóstwa, kulty i~rytuały starożytnego Egiptu} \\
\StrWierszD{110}{2} \\
\Jest  Starożytne cywilizacje. Wierzenia, mitologie, sztuka \\
\Powin \textit{Starożytne cywilizacje. Wierzenia, mitologie, sztuka} \\
\StrWierszD{110}{2} \\
\Jest  Wielki atlas mitów i~legend świata \\
\Powin \textit{Wielki atlas mitów i~legend świata} \\
\StrWierszD{130}{2} \\
\Jest  Starożytni Grecy i~Rzymianie \\
\Powin \textit{Starożytni Grecy i~Rzymianie} \\
\StrWierszD{134}{3} \\
\Jest  Historia społeczna starożytnego Rzymu \\
\Powin \textit{Historia społeczna starożytnego Rzymu} \\
\StrWierszD{135}{3} \\
\Jest  Miłość w~starożytnym Rzymie \\
\Powin \textit{Miłość w~starożytnym Rzymie} \\
\StrWierszD{150}{2} \\
\Jest  Grecy o~miłości, szczęściu i~życiu \\
\Powin \textit{Grecy o~miłości, szczęściu i~życiu} \\
\StrWierszD{152}{3} \\
\Jest  Encyklopedia archeologiczna Ziemi Świętej \\
\Powin \textit{Encyklopedia archeologiczna Ziemi Świętej} \\
\StrWierszD{174}{3} \\
\Jest  Historia wychowania w~starożytności \\
\Powin \textit{Historia wychowania w~starożytności} \\
\StrWierszD{174}{1} \\
\Jest  Liryka starożytnej Grecji \\
\Powin \textit{Liryka starożytnej Grecji} \\
\StrWierszD{175}{5} \\
\Jest  Sportowe życie antyczne Grecji \\
\Powin \textit{Sportowe życie antycznej Grecji} \\
\StrWierszD{175}{4} \\
\Jest  Dictionary~of Greek and Roman Antiquities \\
\Powin \textit{Dictionary~of Greek and Roman Antiquites} \\
\StrWierszD{199}{2} \\
\Jest  Historia wychowania w~starożytności \\
\Powin \textit{Historia wychowania w~starożytności} \\
\StrWierszD{208}{2} \\
\Jest  Barwny półświatek starożytnego Rzymu \\
\Powin \textit{Barwny półświatek starożytnego Rzymu} \\
\StrWierszD{217}{2} \\
\Jest  Życie codzienne w~Pompejach \\
\Powin \textit{Życie codzienne w~Pompejach} \\
\StrWierszD{226}{3} \\
\Jest  Buddyzm-dżinizm-religie ludów pierwotnych \\
\Powin \textit{Buddyzm-dżinizm-religie ludów pierwotnych} \\
\StrWierszD{230}{3} \\
\Jest  Życie codzienne Etrusków \\
\Powin \textit{Życie codzienne Etrusków} \\
% 236????
\StrWierszD{236}{2} \\
\Jest  Starożytni Grecy i~Rzymianie\ldots \\
\Powin \textit{Starożytni Grecy i~Rzymianie}\ldots \\
\StrWierszD{242}{2} \\
\Jest  Liryka Starożytnej Grecji \\
\Powin \textit{Liryka Starożytnej Grecji} \\
\StrWierszD{245}{2} \\
\Jest  Życie codzienne Palestynie w~czasach Chrystusa \\
\Powin \textit{Życie codzienne w~Palestynie w~czasach Chrystusa} \\
\StrWierszD{246}{4} \\
\Jest  Piekło. Oddalenie od~Domu Ojca \\
\Powin \textit{Piekło. Oddalenie od~Domu Ojca} \\
\StrWierszD{262}{2} \\
\Jest  Życie codzienne w~dawnych Indiach \\
\Powin \textit{Życie codzienne w~dawnych Indiach} \\
\StrWierszD{273}{23} \\
\Jest  Religie świata rzymskiego \\
\Powin \textit{Religie świata rzymskiego} \\
\StrWierszD{273}{23} \\
\Jest  Historia wierzeń i~idei religijnych \\
\Powin \textit{Historia wierzeń i~idei religijnych} \\
\StrWierszD{274}{2} \\
\Jest  Historia wierzeń i~idei religijnych \\
\Powin \textit{Historia wierzeń i~idei religijnych} \\
\StrWierszD{279}{7} \\
\Jest  Niezwykła technika starożytności \\
\Powin \textit{Niezwykła technika starożytności} \\
\StrWierszD{304}{1} \\
\Jest  Grecy o~miłości, szczęściu i~życiu \\
\Powin \textit{Grecy o~miłości, szczęściu i~życiu} \\
\StrWierszD{312}{5} \\
\Jest  Rzymska elegia miłosna, Wybór i~przekład Anna Świderska \\
\Powin \textit{Rzymska elegia miłosna. Wybór i~przekład Anna Świderska} \\
\StrWierszD{315}{1} \\
\Jest  Wędrówki po~Helladzie \\
\Powin \textit{Wędrówki po~Helladzie} \\
\StrWierszD{315}{1} \\
\Jest  Starożytni olimpijczycy \\
\Powin \textit{Starożytni olimpijczycy} \\
\StrWierszD{332}{3} \\
\Jest  Grecy o~miłości, szczęściu i~życiu \\
\Powin \textit{Grecy o~miłości, szczęściu i~życiu} \\
\StrWierszD{332}{2} \\
\Jest  Grecy o~miłości, szczęściu i~życiu \\
\Powin \textit{Grecy o~miłości, szczęściu i~życiu} \\
\StrWierszD{334}{1} \\
\Jest  miłości, szczęściu i~życiu \\
\Powin \textit{miłości, szczęściu i~życiu} \\
\StrWierszD{336}{4} \\
\Jest  Liryka starożytnej Grecji \\
\Powin \textit{Liryka starożytnej Grecji} \\
\StrWierszD{344}{2} \\
\Jest  Starożytni Grecy i~Rzymianie \\
\Powin \textit{Starożytni Grecy i~Rzymianie} \\
\StrWierszD{364}{2} \\
\Jest  O~najwyższy dobru i~złu \\
\Powin \textit{O~najwyższym dobru i~złu} \\
\StrWierszD{364}{2} \\
\Jest  Słowo jest cieniem czynu \\
\Powin \textit{Słowo jest cieniem czynu} \\
\StrWierszD{366}{1} \\
\Jest  Grecy o~miłości, szczęściu i~życiu \\
\Powin \textit{Grecy o~miłości, szczęściu i~życiu} \\
\StrWierszD{367}{6} \\
\Jest  Liryka starożytnej Grecji \\
\Powin \textit{Liryka starożytnej Grecji} \\
\StrWierszD{367}{4} \\
\Jest  \textit{Astronomic}a \\
\Powin \textit{Astronomica} \\
\StrWierszD{383}{1} \\
\Jest  Słowo jest cieniem czynu \\
\Powin \textit{Słowo jest cieniem czynu} \\
\StrWierszD{387}{2} \\
\Jest  Słowo jest cieniem czynu \\
\Powin \textit{Słowo jest cieniem czynu} \\
\StrWierszD{424}{3} \\
\Jest  Historia wierzeń i~idei religijnych \\
\Powin \textit{Historia wierzeń i~idei religijnych} \\
\StrWierszD{424}{2} \\
\Jest  Starożytne cywilizacje. Wierzenia, mitologia, sztuka \\
\Powin \textit{Starożytne cywilizacje. Wierzenia, mitologia, sztuka} \\
\StrWierszD{424}{2} \\
\Jest  Wielki atlas mitów i~legend świata \\
\Powin \textit{Wielki atlas mitów i~legend świata} \\



% ############################










% ######################################
\newpage

\section{Historia świętej wiary, VI i~VII wiek po Chrystusie}

% ######################################



% ############################
{ % Autor i tytuł dzieła
  Łukasz Czarnecki \\
  \textit{Konstantynopol~626}, \cite{} % {CzarneckiKonstantynopol626Wyd2017}
}

\vspace{0em}


% ##################
\CenterBoldFont{Uwagi do~konkretnych stron}


\noindent
\StrWierszG{58}{18} Mam wątpliwość czy~wszystkie słowa wyróżnione
tu~kursywą~są częścią cytowanego fragmentu.

\VerSpaceFour





% ##################
\CenterBoldFont{Błędy}


\begin{center}

  \begin{tabular}{|c|c|c|c|c|}
    \hline
    Strona & \multicolumn{2}{c|}{Wiersz} & Jest
                              & Powinno być \\ \cline{2-3}
    & Od góry & Od dołu & & \\
    \hline
    \hphantom{0}19 & \hphantom{0}6 & & w~legła gruzach & legła w~gruzach \\
    \hphantom{0}19 & & \hphantom{0}5 & \textit{TheChronicle}
    & \textit{The~Chronicle} \\
    \hphantom{0}22 & & 16 & na~bizantyńską & bizantyńską \\
    \hphantom{0}32 & & \hphantom{0}2 & \textit{wiary, islam}
    & \textit{wiary. Islam} \\
    \hphantom{0}68 & & \hphantom{0}1 & Dzieje Bizancjum
    & \textit{Dzieje Bizancjum} \\
    121 & & 13 & przed & przed nim \\
    126 & & \hphantom{0}2 & The~Armenian & \textit{The~Armenian} \\
    143 & & \hphantom{0}4 & wroga$^{ \textrm{\textit{20}} }$.
    & wroga$^{ 20 }$. \\
    178 & \hphantom{0}5 & & tylko & ma~tylko \\
    202 & & 15 & \textit{wiary, islam} & \textit{wiary. Islam} \\
    203 & \hphantom{0}3 & & \textit{history} & \textit{History} \\
    \hline
  \end{tabular}

\end{center}

\VerSpaceSix



% ############################










% ######################################
\newpage

\section{Historia świętej wiary, XVIII i~XIX wiek}
% ######################################



% ############################
{ % Autor i tytuł dzieła
  Richard Butterwick \\
  \textit{Polska Rewolucja a~Kościół Katolicki 1788--1792},
  \cite{ButterwickPolskaRewolucjaAKosciolKatolicki2012}}

\vspace{0em}


% ##################
\CenterBoldFont{Uwagi do~konkretnych stron}

\vspace{0em}


\noindent
\Str{28} Euzebiusz \\



% ##################
\CenterBoldFont{Błędy}

\begin{center}

  \begin{tabular}{|c|c|c|c|c|}
    \hline
    Strona & \multicolumn{2}{c|}{Wiersz} & Jest
                              & Powinno być \\ \cline{2-3}
    & Od góry & Od dołu & & \\
    \hline
     % &   & &  &  \\
    \hline
  \end{tabular}

\end{center}

\VerSpaceSix


% ############################










% ######################################
\newpage

\section{Historia świętej wiary, XX i~XXI wiek}
% ######################################



% ############################
{ % Autor i tytuł książki
  Red. E. Guerriero, M. Impagliazzo \\
  \textit{Najnowsza historia Kościoła. Katolicy i~kościoły
    chrześcijańskie} \\
  \textit{w~czasie pontyfikatu Jana Pawła II (1978--2005)},
  \cite{GuerrieroImpagliazzoNajnowszaHistoriaKosciola2006} }

\vspace{0em}


% ##################
\CenterBoldFont{Błędy}


\begin{center}

  \begin{tabular}{|c|c|c|c|c|}
    \hline
    Strona & \multicolumn{2}{c|}{Wiersz} & Jest
                              & Powinno być \\ \cline{2-3}
    & Od góry & Od dołu & & \\
    \hline
    \hphantom{0}6 & 10 & & religia miały & nauka miały \\
    \hphantom{0}6 & & 10 & do & od \\
    \hphantom{0}7 & & 11 & dużo & duże \\
    14 & & \hphantom{0}3 & zgodne & zgadzające~się \\
    30 & & \hphantom{0}4 & Afryki & Ameryki Południowej \\
    51 & & 17 & śś. & św. \\
    63 & \hphantom{0}9 & & 1987 & 1986 \\
    \hline
  \end{tabular}

\end{center}

% \vspace{\spaceTwo}


% ############################










% % ######################################
% \section{Dzieła świętych}

% \vspace{\spaceTwo}
% % ######################################








% % ######################################
% \newpage
% \section{Pozostali autorzy}

% \vspace{\spaceTwo}
% % ######################################










% #####################################################################
% #####################################################################
% Bibliography

\bibliographystyle{plalpha}

\bibliography{DEUSBooks}{}





% ############################

% End of the document
\end{document}
