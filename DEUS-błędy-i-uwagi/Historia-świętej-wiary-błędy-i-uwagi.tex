% ---------------------------------------------------------------------
% Podstawowe ustawienia i pakiety
% ---------------------------------------------------------------------
\RequirePackage[l2tabu, orthodox]{nag} % Wykrywa przestarzałe i niewłaściwe
% sposoby używania LaTeXa. Więcej jest w l2tabu English version.
\documentclass[a4paper,11pt]{article}
% {rozmiar papieru, rozmiar fontu}[klasa dokumentu]
\usepackage[MeX]{polski} % Polonizacja LaTeXa, bez niej będzie pracował
% w języku angielskim.
\usepackage[utf8]{inputenc} % Włączenie kodowania UTF-8, co daje dostęp
% do polskich znaków.
\usepackage{lmodern} % Wprowadza fonty Latin Modern.
\usepackage[T1]{fontenc} % Potrzebne do używania fontów Latin Modern.



% ------------------------------
% Podstawowe pakiety (niezwiązane z ustawieniami języka)
% ------------------------------
\usepackage{microtype} % Twierdzi, że poprawi rozmiar odstępów w tekście.
% \usepackage{graphicx} % Wprowadza bardzo potrzebne komendy do wstawiania
% % grafiki.
% \usepackage{verbatim} % Poprawia otoczenie VERBATIME.
% \usepackage{textcomp} % Dodaje takie symbole jak stopnie Celsiusa,
% wprowadzane bezpośrednio w tekście.
\usepackage{vmargin} % Pozwala na prostą kontrolę rozmiaru marginesów,
% za pomocą komend poniżej. Rozmiar odstępów jest mierzony w calach.
% ------------------------------
% MARGINS
% ------------------------------
\setmarginsrb
{ 0.7in}  % left margin
{ 0.6in}  % top margin
{ 0.7in}  % right margin
{ 0.8in}  % bottom margin
{  20pt}  % head height
{0.25in}  % head sep
{   9pt}  % foot height
{ 0.3in}  % foot sep



% ------------------------------
% Często używane pakiety
% ------------------------------
% \usepackage{csquotes} % Pozwala w prosty sposób wstawiać cytaty do tekstu.
\usepackage{xcolor} % Pozwala używać kolorowych czcionek (zapewne dużo
% więcej, ale ja nie potrafię nic o tym powiedzieć).





% ---------------------------------------------------------------------
% Dodatkowe ustawienia dla języka polskiego
% ---------------------------------------------------------------------
\renewcommand{\thesection}{\arabic{section}.}
% Kropki po numerach rozdziału (polski zwyczaj topograficzny)
\renewcommand{\thesubsection}{\thesection\arabic{subsection}}
% Brak kropki po numerach podrozdziału



% ------------------------------
% Pakiety napisane przez użytkownika.
% Mają być w tym samym katalogu to ten plik .tex
% ------------------------------
\usepackage{latexgeneralcommands}



% ------------------------------
% Ustawienia różnych parametrów tekstu
% ------------------------------
\renewcommand{\baselinestretch}{1.1}

\renewcommand{\arraystretch}{1.4} % Ustawienie szerokości odstępów między
% wierszami w tabelach.



% ------------------------------
% Pakiet „hyperref”
% Polecano by umieszczać go na końcu preambuły.
% ------------------------------
\usepackage{hyperref} % Pozwala tworzyć hiperlinki i zamienia odwołania
% do bibliografii na hiperlinki.










% ---------------------------------------------------------------------
% Tytuł, autor, data
\title{Historia świętej wiary \\
  {\Large Błędy i~uwagi}}

\author{Kamil Ziemian}


% \date{}
% ---------------------------------------------------------------------










% ####################################################################
% Początek dokumentu
\begin{document}
% ####################################################################





% ######################################
\maketitle  % Tytuł całego tekstu
% ######################################





% ######################################
\section{Historia świętej wiary, syntezy historyczne}

\vspace{\spaceTwo}
% ######################################



% ############################
\Work{ % Autor i tytuł dzieła
  Warren H.~Carroll \\
  \textit{Historia Chrześcijaństwa. Tom~I: Narodziny Chrześcijaństwa},
  \cite{CarrollHistoriaChrzecijanstwaVolI2009}}

\vspace{0em}


% ##################
\CenterBoldFont{Uwagi do~konkretnych stron}

\vspace{0em}


\noindent
\StrWd{17}{4} Gwiazdka w~tej linii jest za~mała.

\vspace{\spaceFour}





\noindent
\StrWd{41}{13} W~tej linii użyto złego rodzaju gwiazdki i~umieszczono ją
trochę krzywo.

\vspace{\spaceFour}





\noindent
\StrWd{61}{20} Gwiazdka w~tej linii jest za~mała.

\vspace{\spaceFour}





\noindent
\StrWd{65}{6} W~tej linii użyto złego rodzaju gwiazdki.

\vspace{\spaceFour}





\noindent
\StrWd{73}{12} Gwiazdka w~tej linii jest trochę za~mała i~odrobinę krzywo
umieszczona.

\vspace{\spaceFour}





\noindent
\StrWd{75}{4} Gwiazdka w~tej linii jest za mała.

\vspace{\spaceFour}





\noindent
\StrWd{112}{2} Gwiazdka w~tej linii jest za mała.

\vspace{\spaceFour}





\noindent
\StrWd{182}{2} Gwiazdka w~tej linii jest za mała.

\vspace{\spaceFour}





\noindent
\textbf{Str.~199, wiersze 9, 8 (od~dołu).} Gwiazdki w~tych liniach~są zbyt
małe.

\vspace{\spaceFour}





\noindent
\textbf{Str.~200, wiersze 2, 1 (od~dołu).} Gwiazdki w~tych liniach~są zbyt
małe.

\vspace{\spaceFour}





\noindent
\StrWd{249}{1} Gwiazdka w~tej linii jest za mała.

\vspace{\spaceFour}





\noindent
\StrWd{265}{2} Gwiazdka w~tej linii jest za mała.

\vspace{\spaceFour}





\noindent
\StrWd{271}{1} Gwiazdka w~tej linii jest za mała.

\vspace{\spaceFour}





\noindent
\StrWd{282}{14} Gwiazdka w~tej linii jest za mała.

\vspace{\spaceFour}





\noindent
\StrWd{286}{4} Gwiazdka w~tej linii jest za mała.

\vspace{\spaceFour}





\noindent
\textbf{Str.~290, wiersze 4, 3 (od~dołu).} Gwiazdki w~tych liniach~są zbyt
małe.

\vspace{\spaceFour}





\noindent
\StrWd{366}{1} Gwiazdka w~tej linii jest za mała.

\vspace{\spaceFour}





\noindent
\StrWg{378}{20} Gwiazdka w~tej linii jest za mała.

\vspace{\spaceFour}





\noindent
\StrWd{495}{1} Gwiazdka w~tej linii jest za mała.

\vspace{\spaceFour}





\noindent
\StrWd{538}{38} Gwiazdka w~tej linii jest za mała.

\vspace{\spaceFour}



\noindent
\StrWd{562}{20} W~tej linii nie podano autorów publikacji. Nie
wiem czy to~błąd, czy~dla tej publikacji nie~trzeba podawać autorów.

\vspace{\spaceFour}





\noindent
\StrWg{575}{4~od~góry i~1 od~dołu} Słowo „męczennik” jest w~bardzo brzydki
sposób podzielone między te dwie linie.

% \vspace{\spaceFour}


% 185???





% ##################
\newpage

\CenterBoldFont{Błędy}


\begin{center}

  \begin{tabular}{|c|c|c|c|c|}
    \hline
    Strona & \multicolumn{2}{c|}{Wiersz} & Jest
                              & Powinno być \\ \cline{2-3}
    & Od góry & Od dołu & & \\
    \hline
    % & & & & \\
    19  & & 14 & \textit{Fossil~~Evidence} & \textit{Fossil Evidence} \\
    19  & &  5 & \textit{Evolution -- the} & \textit{Evolution: The} \\
    20  & & 21 & le & Le \\
    29  & &  2 & \textit{Huyuk, a} & \textit{Huyuk: A} \\
    30  & &  3 & \textit{Sumerians, Their} & \textit{Sumerians: Their} \\
    33  & &  3 & \textit{Babylon, a} & \textit{Babylon: A} \\
    35  & &  5 & \textit{India, a} & \textit{India: A} \\
    42  & &  2 & \textit{Elba, a} & \textit{Elba: A} \\
    44  & &  1 & \textit{Hyksos, a} & \textit{Hyksos: A} \\
    45  & & 18 & \textit{Desert, a} & \textit{Desert: A} \\
    49  & & 20 & \textit{through} & \textit{Through} \\
    % A Path Through Genesis
    49  & &  2 & \textit{through} & \textit{Through} \\
    53  & & 15 & \textit{Path through} & \textit{A Path Through} \\
    56  & &  5 & \textit{Akhenaten},,  % ''
           & \textit{Akhenaten}, \\
    61  & &  2 & \textit{Israel,} & \textit{Israel:} \\
    63  & & 12 & \textit{Rames II, a} & \textit{Rames II: A} \\
    73  & &  6 & \textit{into} & \textit{Into} \\
    73  & &  6 & \textit{Past, the} & \textit{Past: The} \\
    81  & &  3 & \textit{Israel, Its} & \textit{Israel: Its} \\
    83  & & 20 & \textit{Hazor, the} & \textit{Hazor: The} \\
    84  & &  2 & \textit{Bible,} & \textit{Bible:} \\
    84  & &  1 & \textit{a Historical} & \textit{A Historical} \\
    91  & &  5 & \textit{Worlds}. 94-100 & \textit{Worlds}, s.~94-100 \\
    92  & &  6 & \textit{Testament}, & \textit{Testament}, [w:] \\
    102 & & 16 & \textit{In} & \textit{in} \\
    102 & &  4 & \textit{Israel, its} & \textit{Israel: Its} \\
    107 & &  3 & \textit{Mesopotamia, Portrait}
           & \textit{Mesopotamia: Portrait} \\
    126 & 18 & & \textit{through} & \textit{Through} \\
    131 & &  9 & Babylon & \textit{Babylon} \\
    \hline
  \end{tabular}





  % \begin{tabular}{|c|c|c|c|c|}
  %   \hline
  %   Strona & \multicolumn{2}{c|}{Wiersz} & Jest
  %   & Powinno być \\ \cline{2-3}
  %   & Od góry & Od dołu & & \\
  %   \hline
  %   %   & & & & \\
  %   %   & & & & \\
  %   %   & & & & \\
  %   %   & & & & \\
  %   %   & & & & \\
  %   %   & & & & \\
  %   %   & & & & \\
  %   %   & & & & \\
  %   %   & & & & \\
  %   %   & & & & \\
  %   %   & & & & \\
  %   %   & & & & \\
  %   %   & & & & \\
  %   %   & & & & \\
  %   %   & & & & \\
  %   %   & & & & \\
  %   %   & & & & \\
  %   %   & & & & \\
  %   %   & & & & \\
  %   %   & & & & \\
  %   %   & & & & \\
  %   %   & & & & \\
  %   %   & & & & \\
  %   %   & & & & \\
  %   %   & & & & \\
  %   %   & & & & \\
  %   %   & & & & \\
  %   %   & & & & \\
  %   %   & & & & \\
  %   %   & & & & \\
  %   %   & & & & \\
  %   %   & & & & \\
  %   %   & & & & \\
  %   %   & & & & \\
  %   %   & & & & \\
  %   %   & & & & \\
  %   %   & & & & \\
  %   %   & & & & \\
  %   \hline
  % \end{tabular}





  \newpage

  \begin{tabular}{|c|c|c|c|c|}
    \hline
    Strona & \multicolumn{2}{c|}{Wiersz} & Jest
                              & Powinno być \\ \cline{2-3}
    & Od góry & Od dołu & & \\
    \hline
    132 & &  6 & \textit{Israe}l & \textit{Israel} \\
    145 & & 12 & \textit{Greeks; the} & \textit{Greeks: The} \\
    145 & &  3 & Babylon & \textit{Babylon} \\
    160 & & 21 & \textit{Personality, Its} & \textit{Personality: Its} \\
    166 & &  5 & wspanialej & wspaniałej \\
    171 & &  2 & \textit{Wisdom, the} & \textit{Wisdom: The} \\
    172 & &  4 & \textit{Greeks, the} & \textit{Greeks: The} \\
    178 & &  4 & Agamemnon & \textit{Agamemnon} \\
    182 & &  5 & \textit{Greeks, a} & \textit{Greeks: A} \\
    193 & &  1 & \textit{Great, King} & \textit{Great: King} \\
    203 & &  3 & \textit{B.C} & \textit{B.C.} \\
    218 & &  8 & \textit{146B.C.} & \textit{146~B.C.} \\
    220 & &  2 & \textit{Africanus, Soldier}
           & \textit{Africanus: Soldier} \\
    223 & &  5 & \textit{Carthage, a} & \textit{Carthage: A} \\
    228 & & 14 & \textit{B.~C} & \textit{B.C.} \\
    230 & & 15 & \textit{Empire, Rome's} & \textit{Empire: Rome's} \\
    230 & &  4 & \textit{B.~C.} & \textit{B.C.} \\
    230 & &  1 & \textit{146B.C.} & \textit{146~B.C.} \\
    233 & &  9 & \textit{146B.C.} & \textit{146~B.C.} \\
    233 & &  7 & \textit{146B.C.} & \textit{146~B.C.} \\
    234 & &  5 & \textit{Syria, from} & \textit{Syria: From} \\
    238 & &  7 & \textit{Ezra} & \textit{From Ezra} \\
    238 & &  3 & \textit{Ezra} & \textit{From Ezra} \\
    241 & &  5 & \textit{Ezra} & \textit{From Ezra} \\
    241 & &  4 & \textit{Ezra} & \textit{From Ezra} \\
    241 & &  1 & \textit{Ezra} & \textit{From Ezra} \\
    242 & &  2 & \textit{Ezra} & \textit{From Ezra} \\
    242 & &  2 & \textit{Ezra} & \textit{From Ezra} \\
    259 & &  3 & \textit{Pompey, the} & \textit{Pompey: The} \\
    261 & &  3 & \textit{Caesar, Politician}
           & \textit{Caesar: Politician} \\
    \hline
  \end{tabular}





  \newpage

  \begin{tabular}{|c|c|c|c|c|}
    \hline
    Strona & \multicolumn{2}{c|}{Wiersz} & Jest
                              & Powinno być \\ \cline{2-3}
    & Od góry & Od dołu & & \\
    \hline
    262 & &  6 & \textit{Pompey, the} & \textit{Pompey: The} \\
    266 & &  6 & \textit{Pompey, the} & \textit{Pompey: The} \\
    267 & &  2 & \textit{Rule, from} & \textit{Rule: From} \\
    268 & &  3 & \textit{under} & \textit{Under} \\
    268 & &  1 & \textit{under} & \textit{Under} \\
    275 & &  7 & \textit{under} & \textit{Under} \\
    281 & &  3 & \textit{the} & \textit{The} \\
    281 & &  3 & \textit{Joseph, Their} & \textit{Joseph: Their} \\
    293 & &  3 & \textit{Christ, His} & \textit{Christ: His} \\
    294 & &  8 & \textit{St.~Matthew, a} & \textit{St.~Matthew: A} \\
    295 & &  4 & \textit{Birth, an} & \textit{Birth: An} \\
    300 & &  9 & \textit{Josephus, the} & \textit{Josephus: The} \\
    307 & &  8 & \textit{Birth, an} & \textit{Birth: An} \\
    315 & & 20 & \textit{Bethlehem, an} & \textit{Bethlehem: An} \\
    315 & & 10 & \textit{niemal} & niemal \\
    316 & &  8 & \textit{Dead, Studies} & \textit{Dead: Studies} \\
    320 & & 10 & Jesus Christ, \textit{His} & \textit{Jesus Christ: His} \\
    326 & & 12 & \textit{Antipas, a} & \textit{Antipas: A} \\
    365 & &  9 & 1931 ) & 1931) \\
    374 & & 11 & 549;Belser & 549; Belser \\
    387 & & 10 & \textit{Doctor} & \textit{A~Doctor} \\
    387 & &  9 & \textit{Doctor} & \textit{A~Doctor} \\
    387 & &  5 & \textit{Doctor} & \textit{A~Doctor} \\
    388 & & 17 & \textit{Doctor} & \textit{A~Doctor} \\
    388 & & 13 & \textit{Doctor} & \textit{A~Doctor} \\
    388 & &  1 & \textit{Doctor} & \textit{A~Doctor} \\
    390 & &  7 & \textit{Doctor} & \textit{A~Doctor} \\
    391 & & 11 & \textit{Doctor} & \textit{A~Doctor} \\
    391 & & 10 & \textit{Doctor} & \textit{A~Doctor} \\
    392 & &  3 & \textit{Doctor} & \textit{A~Doctor} \\
    \hline
  \end{tabular}





  \newpage

  \begin{tabular}{|c|c|c|c|c|}
    \hline
    Strona & \multicolumn{2}{c|}{Wiersz} & Jest
                              & Powinno być \\ \cline{2-3}
    & Od góry & Od dołu & & \\
    \hline
    393 & &  8 & \textit{Doctor} & \textit{A~Doctor} \\
    402 & &  9 & \textit{Antipas, a} & \textit{Antipas: A }\\
    407 & &  6 & \textit{Greek, a} & \textit{Greek: A} \\
    408 & &  2 & \textit{Claudius, the} & \textit{Claudius: The} \\
    410 & &  3 & \textit{Luke, a} & \textit{Luke: A} \\
    419 & & 24 & \textit{Doctor} & \textit{A~Doctor} \\
    432 & &  2 & \textit{Kerala, a} & \textit{Kerala: A} \\
    436 & &  5 & \textit{Nero, Reality} & \textit{Nero: Reality} \\
    440 & &  4 & \textit{Exile, a} & \textit{Exile: A} \\
    442 & &  9 & \textit{Tertullian, a} & \textit{Tertullian: A} \\
    443 & & 17 & \textit{Jude, Introduction}
           & \textit{Jude: Introduction} \\
    444 & &  9 & \textit{under} & \textit{Under} \\
    444 & &  2 & \textit{under} & \textit{Under} \\
    449 & &  7 & \textit{under} & \textit{Under} \\
    455 & & 25 & np., & np. \\
    458 & &  4 & \textit{among} & \textit{Among} \\
    458 & &  2 & \textit{Paul, Apostole} & \textit{Paul: Apostole} \\
    459 & &  3 & \textit{Religion; the} & \textit{Religin: The} \\
    471 & &  9 & \textit{Smyrna, a} & \textit{Smyrna: A} \\
    477 & & 17 & \textit{ofthe} & \textit{of the} \\
    478 & &  7 & \textit{Severus, the} & \textit{Severus: The} \\
    490 & & 15 & \textit{Tertulian, a} & \textit{Tertulian: A} \\
    490 & &  1 & cześć & część \\
    492 & & 13 & \textit{during} & \textit{During} \\
    492 & &  2 & \textit{Mani, a} & \textit{Mani: A} \\
    508 & &  4 & \textit{Edessa, the} & \textit{Edessa: The} \\
    533 & &  4 & \textit{Constantine, a} & \textit{Constantine: A} \\
    561 & & 19 & \textit{Bible, a~Historical}
           & \textit{Bible: A~Historical} \\
    561 & & 16 & T. & T., \\
    561 & & 15 & \textit{Moses, the} & \textit{Moses: The} \\
    \hline
  \end{tabular}





  \newpage

  \begin{tabular}{|c|c|c|c|c|}
    \hline
    Strona & \multicolumn{2}{c|}{Wiersz} & Jest
                              & Powinno być \\ \cline{2-3}
    & Od góry & Od dołu & & \\
    \hline
    562 &  4 & & \textit{Ezekiel: the} & \textit{Ezekiel: The} \\
    562 & 10 & & \textit{Abraham, Loved} & \textit{Abraham: Loved} \\
    562 & 11 & & \textit{Desert, a~History} & \textit{Desert: A~History} \\
    562 & 13 & & \textit{Abraham, Father} & \textit{Abraham: Father} \\
    562 & 15 & & \textit{Canaan: the~Ras} & \textit{Canaan: The~Ras} \\
    562 & & 14 & \textit{Judaea} & \textit{Judea} \\
    562 & & 13 & \textit{Israel, from} & \textit{Israel: From} \\
    562 & & 11 & \textit{Jerusalem; Excavating}
           & \textit{Jerusalem: Excavating} \\
    562 & &  4 & \textit{Covenant, a~Study} & \textit{Covenant: A~Study} \\
    563 &  4 & & \textit{Law; Studies} & \textit{Law: Studies} \\
    563 &  6 & & \textit{1-39, Introduction}
           & \textit{1-39: Introduction} \\
    563 & & 18 & \textit{Joshua; Biblical} & \textit{Joshua: Biblical} \\
    563 & & 11 & \textit{Qumran, a} & \textit{Qumran: A} \\
    563 & &  8 & \textit{Israel, its} & \textit{Israel: Its} \\
    563 & &  7 & T, & T. \\
    564 &  1 & & \textit{Hazor, the} & \textit{Hazor: The} \\
    564 &  8 & & \textit{Maccabees, with} & \textit{Maccabees: With} \\
    564 & 12 & & \textit{Akhenaten, Pharaoh}
           & \textit{Akhenaten, Pharaoh} \\
    564 & 13 & & \textit{before} & \textit{Before} \\
    564 & 16 & & \textit{Elba, a} & \textit{Elba: A} \\
    564 & 18 & & \textit{Greeks; the} & \textit{Greeks: The} \\
    564 & & 17 & \textit{Buddhism, its} & \textit{Buddhism: Its} \\
    564 & &  7 & \textit{Darkness; a} & \textit{Darkness: A} \\
    564 & &  3 & \textit{Sumerians; Their} & \textit{Sumerians: Their} \\
    565 &  3 & & \textit{Huyuk, a} & \textit{Huyuk: A} \\
    565 & 13 & & \textit{Mesopotamia, Portrait}
           & \textit{Mesopotamia: Portrait} \\
    565 & 16 & & \textit{Babylon; a} & \textit{Babylon: A} \\
    565 & 18 & & \textit{Rameses~II, a} & \textit{Ramses~II: A} \\
    565 & 19 & & \textit{India; a} & \textit{India: A} \\
    565 & 17 & & \textit{Hyksos, a} & \textit{Hyksos: A} \\
    \hline
  \end{tabular}





  \newpage

  \begin{tabular}{|c|c|c|c|c|}
    \hline
    Strona & \multicolumn{2}{c|}{Wiersz} & Jest
                              & Powinno być \\ \cline{2-3}
    & Od góry & Od dołu & & \\
    \hline
    565 & 14 & & \textit{Egypt, an} & \textit{Egypt: An} \\
    565 & 13 & & \textit{Personality, Its} & \textit{Personality: Its} \\
    566 &  3 & & \textit{B. C.} & \textit{B.C.} \\
    566 &  7 & & \textit{Carthage, a} & \textit{Carthage: A} \\
    566 &  9 & & \textit{Dead; Studies} & \textit{Dead: Studies} \\
    566 & 10 & & \textit{Empire; Rome's} & \textit{Empire: Rome's} \\
    566 & 11 & & \textit{Greeks, a} & \textit{Greeks: A} \\
    566 & 16 & & \textit{Caesar, Politician}
           & \textit{Caesar: Politician} \\
    566 & & 20 & \textit{Pompey, the} & \textit{Pompey: The} \\
    566 & & 19 & \textit{Pompey, the} & \textit{Pompey: The} \\
    566 & & 17 & \textit{Great, King} & \textit{Great: King} \\
    566 & & 13 & \textit{lonians} & \textit{Ionians} \\
    566 & &  9 & A.H.M.,\textit{Sparta} & A.H.M., \textit{Sparta} \\
    566 & &  5 & \textit{Past, the} & \textit{Past: The} \\
    566 & &  2 & \textit{B. C.} & \textit{B.C.} \\
    566 & &  1 & \textit{Wisdom; the} & \textit{Wisdom: The} \\
    567 &  5 & & \textit{Library, Glory} & \textit{Library: Glory} \\
    567 & 19 & & \textit{under} & \textit{Under} \\
    567 & 19 & & \textit{Rule, from} & \textit{Rule: From} \\
    567 & 20 & & \textit{Cicero, a} & \textit{Cicero: A} \\
    567 & &  4 & \textit{Passion, Death} & \textit{Passion: Death} \\
    568 &  2 & & \textit{Matthew, a} & \textit{Matthew: A} \\
    568 & 15 & & \textit{Christ, a} & \textit{Christ: A} \\
    568 & 18 & & \textit{Antipas, a} & \textit{Antipas: A} \\
    568 & & 13 & \textit{Birth, an} & \textit{Birth: An} \\
    568 & &  4 & \textit{among} & \textit{Among} \\
    568 & &  4 & \textit{Peo-Pk} & \textit{People} \\
    569 & 12 & & \textit{Tertullian, a} & \textit{Tertullian: A} \\
    569 & & 21 & \textit{Smyrna, a} & \textit{Smyrna: A} \\
    \hline
  \end{tabular}





  \newpage

  \begin{tabular}{|c|c|c|c|c|}
    \hline
    Strona & \multicolumn{2}{c|}{Wiersz} & Jest
    & Powinno być \\ \cline{2-3}
    & Od góry & Od dołu & & \\
    \hline
    569 & & 13 & \textit{Jude, Introduction}
    & \textit{Jude: Introduction} \\
    569 & &  1 & \textit{Greek, a} & \textit{Greek: A} \\
    570 & & 17 & \textit{Cecilia, Virgin} & \textit{Cecilia: Virgin} \\
    570 & & 15 & \textit{Exile, a} & \textit{Exile: A} \\
    570 & &  5 & \textit{during} & \textit{During} \\
    571 &  3 & & \textit{Religion; the} & \textit{Religion: The} \\
    571 & 16 & & \textit{Eusebian; Essay} & \textit{Eusebian: Essay} \\
    571 & 10 & & \textit{Rome; the} & \textit{Rome: The} \\
    571 & &  7 & \textit{Mani; a} & \textit{Mani: A} \\
    571 & &  1 & \textit{before} & \textit{Before} \\
    572 &  2 & & 1959 & 1959. \\
    572 &  6 & & \textit{Edessa, the} & \textit{Edessa: The} \\
    572 & 11 & & \textit{Josephus, the} & \textit{Josephus: The} \\
    572 & 17 & & \textit{Kerala, a} & \textit{Kerala: A} \\
    572 &  7 & & \textit{Severus, the} & \textit{Severus: The} \\
    573 &  1 & & \textit{Aurelius, His} & \textit{Aurelius: His} \\
    573 &  8 & & \textit{Seneca, a} & \textit{Seneca: A} \\
    573 & 12 & & \textit{Constantine, a} & \textit{Constantine: A} \\
    573 & 15 & & 1948 & 1948. \\
    573 & 20 & & \textit{after} & \textit{After} \\
    573 & 18 & & \textit{Claudius, the} & \textit{Claudius: The} \\
    573 & &  9 & \textit{138A.D.} & \textit{138~A.D.} \\
    573 & &  3 & \textit{Meroe, a} & \textit{Meroe: A} \\
    573 & &  2 & \textit{under} & \textit{Under} \\
    573 & &  2 & \textit{Rule, from} & \textit{Rule: From} \\
    574 &  5 & & \textit{Nero, Reality} & \textit{Nero: Reality} \\
    575 &  4 & & \textit{422} & 422 \\
    575 &  6 & & \textit{421} & 421 \\
    575 &  7 & & pne.- 7 & pne.~-- 7 \\
    % Linia 8, sprawdź czy daty życia Agrypiny są poprawne
    575 &  8 & & \textit{421, 422,} & 421, 422, \\
    \hline
  \end{tabular}





  \newpage

  \begin{tabular}{|c|c|c|c|c|}
    \hline
    Strona & \multicolumn{2}{c|}{Wiersz} & Jest
    & Powinno być \\ \cline{2-3}
    & Od góry & Od dołu & & \\
    \hline
    575 &  8 & & \textit{508} & 508 \\
    575 & 14 & & \textit{40} & 40 \\
    575 & 14 & & \textit{68} & 68 \\
    575 & 15 & & \textit{91} & 91 \\
    575 & 16 & & \textit{407} & 407 \\
    575 & &  7 & \textit{22} & 22 \\
    575 & &  7 & \textit{517} & 517 \\
    575 & &  5 & \textit{528} & 528 \\
    575 & &  4 & \textit{518} & 518 \\
    575 & &  1 & ii & i \\
    %   & & & & \\
    %   & & & & \\
    %   & & & & \\
    %   & & & & \\
    %   & & & & \\
    %   & & & & \\
    %   & & & & \\
    %   & & & & \\
    %   & & & & \\
    %   & & & & \\
    %   & & & & \\
    %   & & & & \\
    %   & & & & \\
    %   & & & & \\
    %   & & & & \\
    %   & & & & \\
    %   & & & & \\
    %   & & & & \\
    %   & & & & \\
    %   & & & & \\
    %   & & & & \\
    %   & & & & \\
    %   & & & & \\
    %   & & & & \\
    %   & & & & \\
    %   & & & & \\
    %   & & & & \\
    %   & & & & \\
    %   & & & & \\
    %   & & & & \\
    %   & & & & \\
    %   & & & & \\
    %   & & & & \\
    %   & & & & \\
    %   & & & & \\
    %   & & & & \\
    %   & & & & \\
    %   & & & & \\
    \hline
  \end{tabular}

\end{center}

\vspace{\spaceTwo}


\noindent
\StrWd{575}{4} \\
\Jest \hspace{5pt} nik \\
\Powin -nik \\

% ############################










% ############################
\newpage

\Work{ % Autor i tytuł dzieła
  Warren H.~Carroll \\
  \textit{Historia Chrześcijaństwa. Tom~II: Budowanie Chrześcijaństwa},
  \cite{CarrollHistoriaChrzecijanstwaVolII2010}}

\vspace{0em}


% ##################
\CenterBoldFont{Uwagi}

\vspace{0em}


\noindent
Tłumaczenie podtytułu tego tomu „Budowanie Chrześcijaństwa” jest wyjątkowo
niezręczne. Należy zwrócić uwagę, że~Carroll nadał swojemu cyklowi tytuł
„History~of Christendom” nie „History~of Christianity”. „Chrisitianity”
tłumaczy się prosto jako „chrześcijaństwo”, „chistendom” nie ma chyba
odpowiednika w~języku polski, w~tym przypadku zaś można jego sens chyba
wyjaśnić, jako wspólnotę ludzi, której sposób życia definiuje
chrześcijaństwo. W~szczególności „Christendom” oznacza również sens
polityczny, jako zbioru państw, które~są połączone wspólną wiarą
chrześcijańską i~tym samym powinny działać jak różne członki jednego ciała.

Jakkolwiek więc tłumaczenie tytułu całego cyklu jako „Historia
Chrześcijaństwa” ma~sens, to tego podtytułu jako „Budowanie
Chrześcijaństwa” już nie. Sugeruje bowiem, że~religia chrześcijańska
była budowana, podczas gdy ona została już wzniesiona przez Chrystusa,
zaś budowane było właśnie „Christendom”, wspólnota ludzka żyjąca jej
prawami.

\vspace{\spaceFour}





% % ##################
% \CenterBoldFont{Uwagi do~konkretnych stron}





% ##################
\newpage

\CenterBoldFont{Błędy}


\begin{center}

  \begin{tabular}{|c|c|c|c|c|}
    \hline
    Strona & \multicolumn{2}{c|}{Wiersz} & Jest
                              & Powinno być \\ \cline{2-3}
    & Od góry & Od dołu & & \\
    \hline
    14  & 10 & & \textit{Eusebius}. & \textit{Eusebius}, \\
    14  & 11 & & \textit{A.D.324-344} & \textit{A.D. 324-344} \\
    14  & 11 & & \textit{problems} & \textit{Problems} \\
    14  & 12 & & \textit{cordoba} & \textit{Cordoba} \\
    14  & 12 & & \textit{council} & \textit{Council} \\
    14  & 13 & & studies & Studies \\
    14  & & 18 & \textit{church} & \textit{Church} \\
    15  & &  5 & s.90-91 & s.~90-91 \\
    15  & &  4 & s.117 & s.~117 \\
    15  & &  1 & 1971) & 1971 \\
    16  & &  5 & \textit{fourth} & \textit{Fourth} \\
    19  & &  7 & \textit{A.D} & \textit{A.D.} \\
    19  & &  2 & „Jerusalem” & \textit{Jerusalem} \\
    27  & & 17 & \textit{Egipt; the} & \textit{Egipt: The} \\
    27  & & 14 & s.126 & s.~126 \\
    29  & &  2 & 346;Smith & 346; Smith\\
    30  & & 10 & s.285 & s.~285 \\
    31  & &  4 & 341 (Kidd & 341; Kidd \\
    31  & &  3 & 67,71 & 67, 71 \\
    32  & &  6 & s.65 & s.~65 \\
    32  & &  5 & \textit{Antoniego}. & \textit{Antoniego}, \\
    32  & &  4 & 1987) & 1987). \\
    35  & &  5 & 82,380 & 82, 380 \\
    38  & &  1 & \textit{church} & \textit{Church} \\
    45  & & 10 & s.454 & s.~454 \\
    45  & &  9 & 30,52 & 30, 52 \\
    45  & &  9 & 68,76 & 68, 76 \\
    57  & &  7 & \textit{chrześcijaństwie}. & \textit{chrześcijaństwie}, \\
    66  & & 17 & \textit{Jerome, His} & \textit{Jerome: His} \\
    \hline
  \end{tabular}





  \newpage

  \begin{tabular}{|c|c|c|c|c|}
    \hline
    Strona & \multicolumn{2}{c|}{Wiersz} & Jest
                              & Powinno być \\ \cline{2-3}
    & Od góry & Od dołu & & \\
    \hline
    72  & &  4 & \textit{Chrysostom} & \textit{Chryzostom} \\
    74  & &  6 & \textit{saint} & \textit{Saint} \\
    76  & &  6 & Popes & \textit{Popes} \\
    76  & &  6 & Church & \textit{Church} \\
    79  & &  5 & \textit{Claudian; Poetry} & \textit{Claudian: Poetry} \\
    93  & &  6 & \textit{Jerome, His} & \textit{Jerome: His} \\
    106 & &  9 & \textit{Eusebius, bishop} & \textit{Eusebius: Bishop} \\
    114 & &  9 & \textit{Chalcedon, a~Historical}
           & \textit{Chalcedon: A~Historical} \\
    116 & &  4 & \textit{Arthur, a~History} & \textit{Arthur: A~History} \\
    116 & &  4 & \textit{350-} & \textit{350} \\
    116 & &  2 & 254---257 & 254-257 \\
    124 & &  8 & 266,277,294 & 266, 277, 294 \\
    124 & &  3 & \textit{Populi; Popular} & \textit{Populi: Popular} \\
    124 & &  2 & \textit{ontroversies} & \textit{Controversies} \\
    141 & & 12 & \textit{history} & \textit{History} \\
    141 & & 12 & Stevens. & Stevens, \\
    141 & &  8 & Stevens. & Stevens, \\
    154 & &  3 & \textit{Moddle} & \textit{Middle} \\
    155 & &  6 & \textit{Moddle} & \textit{Middle} \\
    156 & & 16 & \textit{Invasions; the} & \textit{Invasions: The} \\
    160 & &  4 & \textit{I, an~introduction}
           & \textit{I: An~Introduction} \\
    161 & &  8 & \textit{sixth} & \textit{Sixth} \\
    167 & 13 & & \textit{History}. & \textit{History}, \\
    168 & &  9 & \textit{I, an} & \textit{I:~An} \\
    197 & &  1 & \textit{Great, His} & \textit{Great: His} \\
    201 & & 21 & \textit{God; the} & \textit{God: The} \\
    235 & &  4 & Mann , & Mann, \\
    236 & &  7 & Mann , & Mann, \\
    238 & & 23 & \textit{conquests} & \textit{Conquests} \\
    249 & &  3 & \textit{before} & \textit{Before} \\
    \hline
  \end{tabular}





  % \begin{tabular}{|c|c|c|c|c|}
  %   \hline
  %   Strona & \multicolumn{2}{c|}{Wiersz} & Jest
  %   & Powinno być \\ \cline{2-3}
  %   & Od góry & Od dołu & & \\
  %   \hline
  %   %   & & & & \\
  %   %   & & & & \\
  %   %   & & & & \\
  %   %   & & & & \\
  %   %   & & & & \\
  %   %   & & & & \\
  %   %   & & & & \\
  %   %   & & & & \\
  %   %   & & & & \\
  %   %   & & & & \\
  %   %   & & & & \\
  %   %   & & & & \\
  %   %   & & & & \\
  %   %   & & & & \\
  %   %   & & & & \\
  %   %   & & & & \\
  %   %   & & & & \\
  %   %   & & & & \\
  %   %   & & & & \\
  %   %   & & & & \\
  %   %   & & & & \\
  %   %   & & & & \\
  %   %   & & & & \\
  %   \hline
  % \end{tabular}





  % \begin{tabular}{|c|c|c|c|c|}
  %   \hline
  %   Strona & \multicolumn{2}{c|}{Wiersz} & Jest
  %   & Powinno być \\ \cline{2-3}
  %   & Od góry & Od dołu & & \\
  %   \hline
  %   %   & & & & \\
  %   %   & & & & \\
  %   %   & & & & \\
  %   %   & & & & \\
  %   %   & & & & \\
  %   %   & & & & \\
  %   %   & & & & \\
  %   %   & & & & \\
  %   %   & & & & \\
  %   %   & & & & \\
  %   %   & & & & \\
  %   %   & & & & \\
  %   %   & & & & \\
  %   %   & & & & \\
  %   %   & & & & \\
  %   %   & & & & \\
  %   %   & & & & \\
  %   %   & & & & \\
  %   %   & & & & \\
  %   %   & & & & \\
  %   %   & & & & \\
  %   %   & & & & \\
  %   %   & & & & \\
  %   \hline
  % \end{tabular}





  \newpage

  \begin{tabular}{|c|c|c|c|c|}
    \hline
    Strona & \multicolumn{2}{c|}{Wiersz} & Jest
                              & Powinno być \\ \cline{2-3}
    & Od góry & Od dołu & & \\
    \hline
    254 & &  2 & V & t.~V \\
    270 & &  3 & 343-344 [ & 343-344. \\
    294 & &  4 & \textit{papal} & \textit{Papal} \\
    308 & &  4 & \textit{Century: a~Study} & \textit{Century: A~Study} \\
    310 & &  6 & \textit{continent} & \textit{Continent} \\
    314 & &  4 & \textit{is.} & s. \\
    315 & & 11 & \textit{during} & \textit{During} \\
    315 & &  3 & \textit{during} & \textit{During} \\
    388 & & 21 & \textit{Great, the} & \textit{Great: The} \\
    388 & & 19 & \textit{Dragon, Alfred} & \textit{Dragon: Alfred} \\
    388 & & 15 & A.Cotarelo & A.~Cotarelo \\
    388 & &  3 & \textit{Magno}) & \textit{Magno} \\
    388 & &  2 & \textit{Great: the} & \textit{Great: The} \\
    585 & &  7 & London. & London \\
    586 &  5 & & Struggle & \textit{Struggle} \\
    586 & 15 & & \textit{Chalcedon} & \textit{Chalcedon} \\
    586 & & 15 & \textit{1} & \textit{the~First} \\
    586 & & 10 & Danielou, Jean & Danielou Jean \\
    586 & & 10 & Henri Marrou & Marrou Henri \\
    587 & 13 & & \textit{A.D.} & \textit{A.D.}, \\
    587 & & 17 & \textit{Jerome,} & \textit{Jerome:} \\
    587 & &  8 & London, & London \\
    587 & &  2 & \textit{Moesia, a~History} & \textit{Moesia: History} \\
    588 &  2 & & \textit{Arthur, a~History} & \textit{Arthur: A~History} \\
    588 &  4 & & \textit{Invasion; the Making}
           & \textit{Invasion: The making} \\
    588 & 15 & & \textit{God; the Life} & \textit{God: The Life} \\
    588 & & 18 & \textit{Britain s} & \textit{Britain's} \\
    588 & & 17 & \textit{Chalcedon, a~Historical}
           & \textit{Chalcedon: A~Historical} \\
    590 & 14 & & \textit{Constantinople; Ecclesiastical}
           & \textit{Constantinople: Ecclesiastical} \\
    591 & & 12 & London, & London \\
    \hline
  \end{tabular}





  \newpage

  \begin{tabular}{|c|c|c|c|c|}
    \hline
    Strona & \multicolumn{2}{c|}{Wiersz}& Jest
                              & Powinno być \\ \cline{2-3}
    & Od góry & Od dołu & & \\
    \hline
    592 & 16 & & \textit{Lyons, Churchman} & \textit{Lyons: Churchman} \\
    592 & &  9 & \textit{Great, the~King} & \textit{Great: The~King} \\
    592 & &  8 & \textit{Canterbury; a~Study}
           & \textit{Canterbury: A~Study} \\
    592 & &  3 & \textit{Slavs; Saints} & \textit{Slavs: Saints} \\
    593 &  7 & & \textit{Empire; the~Arabs} & \textit{Empire: The~Arabs} \\
    593 & 12 & & \textit{Byzantium: the~Imperial}
           & \textit{Byzantium: The~Imperial} \\
    593 & 14 & & \textit{Kings; Their} & \textit{Kings: Their} \\
    593 & 16 & & \textit{England; a~History}
           & \textit{England: A~History} \\
    593 & 20 & & \textit{Great: the~Truth} & \textit{Great: The~Truth} \\
    593 & & 14 & \textit{State; the~Period} & \textit{State: The~Period} \\
    593 & & 12 & \textit{Dragon; Alfred} & \textit{Dragon: Alfred} \\
    593 & &  5 & \textit{St.~Peter; the~Birth}
           & \textit{St.~Peter: The~Birth} \\
    594 & 12 & & \textit{Dublin: the~History}
           & \textit{Dublin: The~History} \\
    594 & &  9 & \textit{Desiderius; Montecassino}
           & \textit{Desiderius: Montecassino} \\
    594 & &  1 & \textit{Empire; the~Arabs} & \textit{Empire: The~Arabs} \\
    595 &  1 & & \textit{Rufus; an~Investigation}
           & \textit{Rufus: An~Investigation} \\
    595 &  6 & & \textit{Byzantium: the~Imperial}
           & \textit{Byzantium: The~Imperial} \\
    595 &  8 & & \textit{England; a~History}
           & \textit{England: A~History} \\
    595 & 11 & & \textit{Kings; Their} & \textit{Kings: Their} \\
    595 & 16 & & \textit{State: the~Period} & \textit{State: The~Period} \\
    595 & 19 & & \textit{Tancred: a~Study} & \textit{Tancred: A~Study} \\
    595 & & 10 & \textit{Saint Peter; the~Reception}
           & \textit{Saint Peter: The~Reception} \\
    596 &  6 & & \textit{440} & 440 \\
    % Popraw dalsze błędy w indeksie
    \hline
  \end{tabular}

\end{center}

\vspace{\spaceTwo}


\noindent
\StrWd{3}{4} \\
\Jest www. WydawnictwoWektory.pl \\
\Powin www.WydawnictwoWektory.pl \\
\StrWd{13}{6} \\
\Jest „Alexander~of Alexandria” \\
\Powin \textit{Alexander~of Alexandria} \\



% ############################










% ############################
\newpage

\Work{ % Autor i tytuł dzieła
  Warren H.~Carroll \\
  \textit{Historia Chrześcijaństwa. Tom~IV: Podział Chrześcijaństwa},
  \cite{CarrollHistoriaChrzecijanstwaVolIV2011}}



% % ##################
% \CenterBoldFont{Uwagi}





% ##################
\newpage

\CenterBoldFont{Błędy}


\begin{center}

  \begin{tabular}{|c|c|c|c|c|}
    \hline
    Strona & \multicolumn{2}{c|}{Wiersz} & Jest
                              & Powinno być \\ \cline{2-3}
    & Od góry & Od dołu & & \\
    \hline
    26  & &  2 & \textit{war} & \textit{War} \\
    31  & &  6 & \textit{Blood; a} & \textit{Blood: A} \\
    33  & &  5 & 122=124 & 122-124 \\
    38  & &  2 & \textit{Conquistadors; First-Person}
           & \textit{Conquistadors; First-Person} \\
    39  & &  5 & (cytat) ; & (cytat); \\
    40  & &  2 & \textit{America; the} & \textit{America: The} \\
    41  & &  2 & s.384 & s.~384 \\
    48  & &  3 & 309; Zob. & 309; zob. \\
    49  & &  2 & s.99 & s.~99 \\
    67  & &  3 & s.84 & s.~84 \\
    78  & &  3 & & \\
    97  & &  2 & Marriman,\textit{Suleiman} & Marriman, \textit{Suleiman} \\
    97  & &  2 & 94;von & 94; von\\
    97  & &  2 & X,s.& X, s. \\
    110 & &  2 & \textit{Won; the} & \textit{Won: The} \\
    111 & &  5 & \textit{Cross. A} & \textit{Cross: A} \\
    112 & &  7 & \textit{Enemies. The} & \textit{Enemies: The} \\
    115 & &  5 & \textit{Master. A} & \textit{Master: A} \\
    130 & &  9 & s.278 & s.~278 \\
    139 & &  3 & s.31 & s.~31 \\
    149 & &  3 & \textit{Vasas. A} & \textit{Vasas: A} \\
    160 & &  5 & \textit{America. The} & \textit{America: The} \\
    163 & &  4 & \textit{is} & \textit{Is} \\
    176 & &  1 & Pastor,\textit{History} & Pastor, \textit{History} \\
    180 & &  2 & \textit{Towns. A} & \textit{Towns: A} \\
    181 & &  2 & \textit{Altars. Traditional}
           & \textit{Altars: Traditional} \\
    217 & & 14 & \textit{Calvin. The} & \textit{Calvin: The} \\
    217 & &  6 & \textit{Calvin, the} & \textit{Calvin: The} \\
    229 & &  1 & \textit{V, King} & \textit{V: King} \\
    \hline
  \end{tabular}





  \newpage

  \begin{tabular}{|c|c|c|c|c|}
    \hline
    Strona & \multicolumn{2}{c|}{Wiersz} & Jest
                              & Powinno być \\ \cline{2-3}
    & Od góry & Od dołu & & \\
    \hline
    230 & &  2 & \textit{V, King} & \textit{V: King} \\
    234 & &  5 & \textit{VI, the} & \textit{VI: The} \\
    236 & &  7 & \textit{VI, the} & \textit{VI: The} \\
    236 & &  5 & \textit{VI, the} & \textit{VI: The} \\
    236 & &  3 & \textit{VI, the} & \textit{VI: The} \\
    237 & &  6 & \textit{VI, the} & \textit{VI: The} \\
    237 & &  5 & \textit{and} & \textit{and the} \\
    238 & &  5 & \textit{VI, the} & \textit{VI: The} \\
    238 & &  3 & \textit{VI, the} & \textit{VI: The} \\
    239 & &  5 & \textit{VI, the} & \textit{VI: The} \\
    239 & &  3 & \textit{VI, the} & \textit{VI: The} \\
    249 & &  1 & \textit{Mass, and} & \textit{Mass and} \\
    \hline
  \end{tabular}





  % \begin{tabular}{|c|c|c|c|c|}
  %   \hline
  %   Strona & \multicolumn{2}{c|}{Wiersz} & Jest
  %                             & Powinno być \\ \cline{2-3}
  %   & Od góry & Od dołu & & \\
  %   \hline
  %   %   & & & & \\
  %   %   & & & & \\
  %   %   & & & & \\
  %   %   & & & & \\
  %   %   & & & & \\
  %   %   & & & & \\
  %   %   & & & & \\
  %   %   & & & & \\
  %   %   & & & & \\
  %   %   & & & & \\
  %   %   & & & & \\
  %   %   & & & & \\
  %   %   & & & & \\
  %   %   & & & & \\
  %   %   & & & & \\
  %   %   & & & & \\
  %   %   & & & & \\
  %   %   & & & & \\
  %   %   & & & & \\
  %   %   & & & & \\
  %   %   & & & & \\
  %   %   & & & & \\
  %   %   & & & & \\
  %   %   & & & & \\
  %   %   & & & & \\
  %   %   & & & & \\
  %   %   & & & & \\
  %   %   & & & & \\
  %   %   & & & & \\
  %   %   & & & & \\
  %   %   & & & & \\
  %   %   & & & & \\
  %   %   & & & & \\
  %   %   & & & & \\
  %   %   & & & & \\
  %   %   & & & & \\
  %   %   & & & & \\
  %   %   & & & & \\
  %   \hline
  % \end{tabular}





  % \begin{tabular}{|c|c|c|c|c|}
  %   \hline
  %   Strona & \multicolumn{2}{c|}{Wiersz} & Jest
  %                             & Powinno być \\ \cline{2-3}
  %   & Od góry & Od dołu & & \\
  %   \hline
  %   %   & & & & \\
  %   %   & & & & \\
  %   %   & & & & \\
  %   %   & & & & \\
  %   %   & & & & \\
  %   %   & & & & \\
  %   %   & & & & \\
  %   %   & & & & \\
  %   %   & & & & \\
  %   %   & & & & \\
  %   %   & & & & \\
  %   %   & & & & \\
  %   %   & & & & \\
  %   %   & & & & \\
  %   %   & & & & \\
  %   %   & & & & \\
  %   %   & & & & \\
  %   %   & & & & \\
  %   %   & & & & \\
  %   %   & & & & \\
  %   %   & & & & \\
  %   %   & & & & \\
  %   %   & & & & \\
  %   %   & & & & \\
  %   %   & & & & \\
  %   %   & & & & \\
  %   %   & & & & \\
  %   %   & & & & \\
  %   %   & & & & \\
  %   %   & & & & \\
  %   %   & & & & \\
  %   %   & & & & \\
  %   %   & & & & \\
  %   %   & & & & \\
  %   %   & & & & \\
  %   %   & & & & \\
  %   %   & & & & \\
  %   %   & & & & \\
  %   \hline
  % \end{tabular}





  \newpage

  \begin{tabular}{|c|c|c|c|c|}
    \hline
    Strona & \multicolumn{2}{c|}{Wiersz} & Jest
                              & Powinno być \\ \cline{2-3}
    & Od góry & Od dołu & & \\
    \hline
    816 & &  2 & 223- & 223, \\
    817 &  4 & & \textit{Towns; a} & \textit{Towns: The} \\
    817 &  7 & & \textit{II, King} & \textit{II: King} \\
    817 & 10 & & \textit{Northumberland; the}
           & \textit{Northumberland: The} \\
    817 & 12 & & Hilaire. & Hilaire, \\
    817 & 13 & & \textit{during} & \textit{During} \\
    817 & 14 & & \textit{Absolutism. A} & \textit{Absolutism: A} \\
    817 & & 11 & \textit{V, King} & \textit{V: King} \\
    817 & &  7 & Henrich. & Henrich, \\
    817 & &  5 & Bradford, & Bradford \\
    817 & &  2 & Anthony. & Anthony, \\
    817 & &  2 & \textit{Magnificent, Scourge}
           & \textit{Magnificent: Scourge} \\
    818 &  3 & & \textit{Bellarmine, Saint} & \textit{Bellarmine: Saint} \\
    818 &  4 & & \textit{Loyola; the} & \textit{Loyola: The} \\
    818 &  8 & & \textit{Darts. The} & \textit{Darts: The} \\
    818 & & 18 & A~\textit{History} & \textit{A~History} \\
    818 & & 12 & \textit{Playground. A} & \textit{Playground: A} \\
    818 & &  6 & \textit{Altars; Traditional}
           & \textit{Altars: Traditional} \\
    818 & &  6 & \textit{c.~1400-c. 1580} & \textit{1400-1580} \\
    818 & &  4 & E.H. & E.H., \\
    818 & &  1 & Philippe. & Philippe, \\
    % Popraw dalszą część bibliografii
    819 & &  9 & 1913 & 1913. \\
    820 & &  1 & 1992.. & 1992. \\
    823 &  2 & & \textit{1621--9} & \textit{1621--1629} \\
    823 &  6 & & \textit{1520--21} & \textit{1520--1521} \\
    823 & 17 & & (red.). & (red.), \\
    825 &  5 & & Charles. & Charles, \\
    825 & 11 & & John., & John, \\
    825 & & 15 & \textit{World; Our} & \textit{World: Our} \\
    825 & & 14 & \textit{the~Sea; the~Treasure}
           & \textit{the~Sea: The~Treasure} \\
    \hline
  \end{tabular}





  \begin{tabular}{|c|c|c|c|c|}
    \hline
    Strona & \multicolumn{2}{c|}{Wiersz} & Jest
                              & Powinno być \\ \cline{2-3}
    & Od góry & Od dołu & & \\
    \hline
    825 & &  5 & Carlos. & Carlos, \\
    825 & &  2 & Parkman, Francis. & Parkman Francis, \\
    825 & &  1 & Francis. & Francis, \\
    826 &  7 & & \textit{Letters} & \textit{Times} \\
    826 & 12 & & St.~Louis. & St.~Louis \\
    826 & &  8 & \textit{leyasu} & \textit{Ieyasu} \\
    826 & &  4 & R.S. & R.S., \\
    % & & & & \\
    % & & & & \\
    % & & & & \\
    % & & & & \\
    % & & & & \\
    % & & & & \\
    % & & & & \\
    % & & & & \\
    % & & & & \\
    % & & & & \\
    % & & & & \\
    % & & & & \\
    % & & & & \\
    % & & & & \\
    % & & & & \\
    % & & & & \\
    % & & & & \\
    % & & & & \\
    % & & & & \\
    % & & & & \\
    % & & & & \\
    % & & & & \\
    % & & & & \\
    % & & & & \\
    % & & & & \\
    % & & & & \\
    % & & & & \\
    % & & & & \\
    % & & & & \\
    % & & & & \\
    % & & & & \\
    % & & & & \\
    % & & & & \\
    % & & & & \\
    \hline
  \end{tabular}

\end{center}

\vspace{\spaceTwo}


\noindent
\StrWd{165}{7} \\
\Jest Bruce,\textit{AnneBoleyn},s.293,299-307,313-333;Scarisbrick,\textit{HenryVIII},s.349-350;Ridley,\textit{Cran-} \\
\Powin Bruce, \textit{Anne Boleyn}, s.~293, 299-307, 313-333; Scarisbrick,
\textit{Henry VIII}, s.~349-350; Ridley, \textit{Cran-} \\
\StrWd{165}{6} \\
\Jest
\textit{mer},s.106-111.AnnęBoleynstracono19maja1536roku.Miałazaledwiedwadzieściaosiemlat. \\
\Powin \textit{mer}, s.~106-111. Annę Boleyn stracono 19~maja 1536 roku.
Miała zaledwie dwadzieścia osiem lat. \\


% ############################










% ############################
\newpage

\Work{ % Autor i tytuł dzieła
  Warren H.~Carroll, Anne W. Carroll \\
  \textit{Historia Chrześcijaństwa. Tom~VI: Kryzys Chrześcijaństwa},
  \cite{CarrollCarrollHistoriaChrzecijanstwaVolVI2014}}

\vspace{0em}


% ##################
\CenterBoldFont{Uwagi do~konkretnych stron}

\vspace{0em}


\noindent
\Str{12} Wcięcia wszystkich akapitów poza pierwszy~są zbyt duże.

\vspace{\spaceFour}





\noindent
\StrWd{31}{4--2} Zdanie „Jestem zobowiązany Jamesowi H.~Billingtonowi,
\textit{Fire In the~Minds~of Man}, wielkiemu historykowi myśli
rewolucyjnej” po polsku brzmi źle i~jest trochę bez sensu. Nie wiem jednak
jak je~poprawić.

\vspace{\spaceFour}





\noindent
\Str{33} Jest dziwne, że~Lamennais jest tu nazwany „wielkim, choć czasami
błądzącym, francuskim duchownym”, skoro sama ta książka podaje na~43
stronie, że~odrzuci on najpierw wiarę katolicką, potem zaś chrześcijaństwo.
Możliwe, że~ta nielogiczność jest wyniki pośmiertnej edycji i~uzupełniania
tego dzieła oraz pracy tłumacza.

\vspace{\spaceFour}





\noindent
\Str{54} Pisze tu, że~bitwa pod Nowym Orleanem była decydującym momentem
w~Wojnie~1812 roku, powołując~się na książkę Paula Johnsona
\textit{Birth~of the~Modern}. Jednak w~tej pozycji Johnson przedstawia
zupełnie inną wersję wydarzeń. Bitwa ta rozegrała~się już po zawarciu
pokoju w~Londynie \red{Sprawdź miasto}, ale~przed tym jak statek
z~informacją o~tym dotarła do~USA, jej przebieg nie doprowadził jednak
do~kontynuacji działań wojennych. Tym samym, konkluduje Johnson, nie
wpłynęła na zawarcie pokój, ale~bardzo na~jego recepcję. Amerykanie
mogli~się bowiem czuć zwycięzcami wojny jako, że~wygrali ostatnią jej
bitwę.

\vspace{\spaceFour}





\noindent
\Str{63} Możliwe, że~informacje podane na tej i~na następnych stronach
dotyczące Ameryki Łacińskiej są poprawne, jednak napisane są w~sposób pełen
luk i~niejasności. Na~przykład na dole tej strony jest podane, że~Martin
skapitulował przed Monteverdim i~wyjechał z~Wenezueli, zaraz potem
zaś~został zdradzony, aresztowany i~wysłany przez Bolivara do~Hiszpanii
w~zamian za paszport, który umożliwi mu przyjazd do~Starego Kraju.
Wydaje~się mało prawdopodobne, by~Bolivar mógł aresztować Martina, gdyby
ten opuścił już Wenezuelę.

Poza tym, nie ma żadnego jasnego stwierdzenia, że~Bolivar wykorzystał
paszport i~udał~się do~Hiszpanii. Zaraz po~informacji, że~zdobył ten
dokument przenosimy~się do Trujillo dnia 15~czerwca 1813, co może
oznaczać miasto w~Hiszpania, ale~też jedno z~wielu o~takiej nazwie
w~Ameryce Południowej. Pierwszym pewnym miejsce w~którym go potem
widzimy, jest wenezuelska Barcelona.

\vspace{\spaceFour}





\noindent
\StrWd{67}{8} Po~tej linii powinien nastąpić odstęp między przypisami.

\vspace{\spaceFour}





\noindent
\Str{76} Następcą zmarłego w~1820~roku Jerzego~III Hanowerskiego był jego
najstarszy syn Jerzy~IV Hanowerski panujący w~latach 1820--1830. Dopiero
po~nim panował w~latach 1830--1837 panował Wilhelm~IV, który był młodszym
synem Jerzego~III, a~nie jego dalekim krewnym. Z~tego tej karygodnej
pomyłki wszelkie dalsze odniesienia do~działań tego monarchy mogą być
błędnie przypisanymi mu aktami Jerzego~IV, bądź źle umieszczone w~czasie.

\vspace{\spaceFour}





\noindent
\StrWd{83}{20--17} Zdanie „Tak samo było w~przypadku Lenina, kolejnego
wielkiego przywódcy rewolucji, który wychował~się w~pobożnej
chrześcijańskiej rodzinie, a~fakt, że~wedle jego własnego świadectwa,
utracił wiarę w~wieku szesnastu lat, nie miał na~to żadnego wpływu.”
źle brzmi i~bardzo trudno zrozumieć myśl jaką w~tym kontekście miało
przekazywać.

\vspace{\spaceFour}





\noindent
\StrWd{85}{11} Gwiazdka w~tej linii jest za~mała.

\vspace{\spaceFour}





\noindent
\Str{110} Na~tej stronie jest podane, że~gdy~w~1914 roku zamordowano
arcyksięcia Franciszka Ferdynanda i~jego żonę Zofię, Franciszkowi Józefowi
wyrwał~się raz jedyny okrzyk „Nie oszczędzono mi niczego!”, podczas gdy
na~stronie~115 jest napisane, iż~wykrzyknął on „Nie oszczędzono mi niczego
na~tej ziemni” w~momencie,gdy~dowiedział~się o~zamordowaniu swojej żony
Elżbiety. Te~dwa fragmenty zdają~się sobie przeczyć.

\vspace{\spaceFour}





\noindent
\Str{125} W~drugim paragrafie na~tej stronie jest trochę zamieszani.
Na~początku jest mowa o~zebraniu 87 osób szwajcarskim Vevey. Na~samym jego
końcu jest mowa o~głosowaniu w~kortezach i~ilości głosów jaka tam padła,
co~nie ma chyba nic wspólnego z~tym zebraniem i~ilością osób która na nim
była, nie~pamiętam zaś aby w~tej książce była podana ilość osób
zasiadających w~kortezach.

\vspace{\spaceFour}





\noindent
\StrWd{126}{8} Nie wiem czemu w~tej linii umieszczono słowa \textit{Dios!
  Patria! Fueros! Rey!}

\vspace{\spaceFour}





\noindent
\Str{135} Fragment utworu poety Grillparzera o~marszałku Radetzkim jest tu
cytowany z~innego źródła niż na~następnej stronie. Nie jest to żaden błąd,
jedynie trochę to dziwne.

\vspace{\spaceFour}





\noindent
\Str{145} Dwa ostatnie paragrafy nie~mają wcięcia w~tekście.

\vspace{\spaceFour}





\noindent
\Str{147} Stwierdzenie, że~to święty Piotr ustanowił papiestwo i~Kościół,
ten błąd jest szczególnie karygodny, jest sprzeczne z~wiarą katolicką.
Zapewne jest to herezja, lecz nie jestem na tyle kompetentny by~stwierdzić
to na 100\%. Jeśli jest to herezja, to wątpię by obarczała sumienie
Carrolla, który zapewne po prostu popełnił głupi błąd pisząc te słowa.

\vspace{\spaceFour}





\noindent
\Str{151} Przynajmniej w~mojej opinii na~tej stronie panuje pewne
zamieszanie. Nie potrafię na~przykład z~całą pewnością
stwierdzić, które z~wydarzeń opisanych w~ostatnim paragrafie
odnoszą~się do~pierwszego synodu, a~które do drugiego.

\vspace{\spaceFour}





\noindent
\StrWd{165}{14--12} Sens zdania „Wielu opuszczało ojczyznę, wypływając
do~USA z~niewielkich portów, a~ich nazwiska przetrwały tylko w~lokalnej
tradycji.” jest następujący. Pamięć o~tym, kto wówczas wypłynął do~Stanów
Zjednoczonych zachowała~się w~lokalnej tradycji ustnej, ale~nie
w~dokumentach z~tamtej epoki. W~tym sensie ich nazwiska nie przetrwały
w~źródłach, nie należy jednak przez to rozumieć, że~ich nazwiska zniknęły
z~użycia, co taka forma tego zdania może sugerować.

\vspace{\spaceFour}





\noindent
\Str{173} Mam problem ze zrozumieniem opisanych tu powodów wybuchu wojny
francusko-pruskiej. Dlaczego niby informacja o~tym, że~Niemcy obrażają
Francuzów wysłana do~króla Prus Wilhelma miała spowodować wypowiedzenie
wojny przez Napoleona~III.

\vspace{\spaceFour}





\noindent
\Str{218} Na~dole strony pozostawiono puste miejsce, które powinien
zajmować tekst z~następnej strony.

\vspace{\spaceFour}





\noindent
\StrWd{225}{3} Po tej linii następuje za~duży odstęp.

\vspace{\spaceFour}





\noindent
\Str{264} Dwa pierwsze paragrafy są źle sformatowane.

\vspace{\spaceFour}





\noindent
\Str{274} Na~dole strony pozostawiono puste miejsce, które powinien
zajmować tekst z~następnej strony.

\vspace{\spaceFour}





\noindent
\Str{277} Należy sprawdzić, czy w~czasie Powstania Tajpingów nie zginęło
na~polach bitew więcej osób, niż podczas I~Wojny Światowej. Uwaga którą tu
poczynił Carroll\footnote{Myślę, że~Anne W.~Carroll zgodziłaby~się
  na~przyznanie autorstwa jej mężowi Warrenowi.}, należy mieć na uwadze
czytając to~co pisze on~o~I~Wojnie Światowej na~stronach 867 i~873.

\vspace{\spaceFour}





\noindent
\StrWd{299}{1} Czcionka w~tej linii jest za~duża.

\vspace{\spaceFour}





\noindent
\StrWd{305}{4} Imię ojca Rasputina Efima, na~str.~313 jest pisane „Jefim”.

\vspace{\spaceFour}





\noindent
\StrWd{352}{1} Czcionka w~tej linii jest za~duża.

\vspace{\spaceFour}





\noindent
\Str{355} W~pierwszym paragrafie jest mowa o~głosowaniu które
zakończyło~się wynikiem siedem do~pięciu, później zaś, że~decyzja
o~pokoju z~Niemcami przeszła stosunkiem siedem do~czterech. Najpewniej
w~obu przypadkach mowa jest o~tym samym głosowaniu i~jeden z~podanych
wyników jest błędny.

\vspace{\spaceFour}





\noindent
\Str{383} Jeśli niczego nie przeoczyłem, to w~tym miejscu ostatni raz jest
mowa o~Denikinie i~jego armii, gdy wycofują~się na~Kubań i~Krym. Nie
dowiadujemy~się tym samym jakie były ich dalsze losy.

\vspace{\spaceFour}





\noindent
\Str{397} Ponieważ Polska, zapewne tak samo, jak kraje nadbałtyckie, nie
istniała w~1914~r., jest nieprawdopodobne, by~w~memorandum Erzberga była
mowa o~nich jako o~sąsiadujących z~Niemcami. Należy~się domyślać,
że~Erzberg chciał włączenia wszystkich ziem które można było uznać za
w~jakimś sensie polskie, analogicznie dla~państw nadbałtyckich,
do~Cesarskich Niemiec po~wygranej wojnie.

\vspace{\spaceFour}





\noindent
\StrWg{416}{22} Po tej linii powinien być większy odstęp.

\vspace{\spaceFour}





\noindent
\StrWd{432}{8} Na~podstawie wcześniejszej części książki nie jestem
w~stanie powiedzieć o~co chodziło w~sprawie nadużyć w~Gruzji.

\vspace{\spaceFour}





\noindent
\Str{435} Na~dole strony pozostawiono puste miejsce, które powinien
zajmować tekst z~następnej strony.

\vspace{\spaceFour}





\noindent
\StrWd{437}{8} Wydaje mi~się, że~spotkałem~się z~wersją, iż~Trocki został
zabity ciosem czekanem. Należy to sprawdzić jeszcze w~jakiejś innej pracy.

\vspace{\spaceFour}





\noindent
\StrWg{441}{1--2} Szacunki Carrollów, że~w~Chinach żyła jedna trzecia
ludności świata, budzą pewne moje wątpliwości. Po~pierwsze należałoby
ustalić o~jakim okresie czasu mowa, po~drugie należałoby sprawdzić, jak
rzeczywiście przedstawiał~się stosunek ludności Chin do ludności świata.

\vspace{\spaceFour}





\noindent
\StrWg{445}{17} Deng Xiaoping żył w~latach 1904--1997, zaś za moment
przejęcia jego władzy po~Mao Zedongu, który zmarł w~1976 roku, należy
chyba przyjąć rok~1978. Xiaoping miał więc wtedy nie dziewięćdziesiąt lecz
siedemdziesiąt cztery lata.

\vspace{\spaceFour}





\noindent
\StrWg{451}{8} Tu~można powtórzyć wątpliwości odnośnie podanej
ludności~Chin i~jej udziału w~ludności świata, które~są w~komentarzu
do~strony~441.

\vspace{\spaceFour}





\noindent
\StrWd{451}{6} Ten wiersz jest źle wcięty.

\vspace{\spaceFour}





\noindent
\Str{456} Na~dole strony pozostawiono puste miejsce, które powinien chyba
zajmować tekst z~następnej strony. Choć w~tym wypadku możliwe jest,
że~obecny wybór jest lepszy.

\vspace{\spaceFour}





\noindent
\StrWd{456}{11} Znak „*” jest w~tej linii za mały.

\vspace{\spaceFour}





\noindent
\StrWd{466}{18} Ponieważ ten cytata zaczyna~się z~małej litery, do~tego
zaraz następuje znak~„\ldots”, co sugeruje, że~ten cytat został błędnie
przytoczony. Jednak nie wiem jak go~poprawić.

\vspace{\spaceFour}





\noindent
\StrWd{466}{1} To~odwołanie bibliograficzne jest niedokończone.

\vspace{\spaceFour}





\noindent
\StrWd{478}{5--6} W~mojej opinii te~dwie linie~są źle sformatowane.

\vspace{\spaceFour}



\noindent
\Str{494} Stwierdzenie, że~Hitler i~Stalin byli w~owym czasie
najpotężniejszymi ludźmi na świecie jest dyskusyjne. Dorównywał ich
sile prezydent USA, można też dywagować, czy ludzie władający Japonią,
nie byli równie potężni\footnote{Nie jestem pewien, czy w~owym czasie
  cała Japońska władza należało do~cesarza Hirohito, stąd takie
  sformułowanie tego punktu.}.

\vspace{\spaceFour}





\noindent
\Str{497} Należy sprawdzić, czy~rzeczywiście największą grupą etniczną
w~Związku Radzieckim byli Ukraińcy.

\vspace{\spaceFour}





\noindent
\Str{497} Należy sprawdzić, jakie naprawdę temperatury~są, czy może raczej
były, zimą na~terenach Syberii. Temperatury rzędu -90\textcelsius
są~w~mojej ocenie mało prawdopodobne. Jeśli jednak Carrollom chodziło
o~-90 stopni Fahrenheita, to~oznaczałoby około -65\textcelsius, co~jest
już znacznie bardziej prawdopodobne.

\vspace{\spaceFour}





\noindent
\Str{499} Gdyby na~31 milinów Ukraińców przypadało 11~milionów ton
zboża to na głowę przypadałoby nie jak piszą Carrollowie
123~kilogramy, lecz~355~kilogramów zboża. Jeśli zaś byłoby to 10
milionów ton zboża, to~na jednego Ukraińca przypadałoby 322 kilogramów
zboża. Ktoś wyraźnie coś pomylił w~rachunkach.

\vspace{\spaceFour}





\noindent
\StrWg{503}{16} W~tym wersie znajduje~się malutkie wcięcie, którego
nie~powinno być. Osobiście uważam też, że~lepiej brzmiałby w~następującej
postaci: „Nadejdzie rok czarny, rok krwi i~pożarów”.

\vspace{\spaceFour}





\noindent
\StrWd{503}{2} Znajdujące~się na~końcu tej linii~„w:”, wygląda bardzo źle.
Należałoby je przenieść na~początek następnej linii.

\vspace{\spaceFour}





\noindent
\StrWd{507}{4} Ponad tą~linią powinien znajdować~się odstęp.

\vspace{\spaceFour}





\noindent
\StrWd{516}{6} W~tej linii jest wcięcie, którego według mnie nie~powinno
tu~być.

\vspace{\spaceFour}





\noindent
\StrWd{526}{6} Ciężko jest zrozumieć od~razu, kim był przywoływany w~tej
linii Jakub. Jest to zapewne karlistowski następca tronu, o~którym była
mowa w~rozdziale \textit{Zwycięstwo i~klęska   tradycjonalistów}.
\red{Sprawdź kiedyś jak~się on~dokładnie nazywał.}

\vspace{\spaceFour}





\noindent
\StrWg{527}{17} Wcięcie tego akapitu jest zbyt duże.

\vspace{\spaceFour}





\noindent
\Str{530} Według podany tu liczb prawica i~centrum zdobyły razem 210~miejsc
w~kortezach więc Front Ludowy z~263 miał 53, nie 26 miejsc przewagi. Albo
jakieś ugrupowanie zostało przemilczane, albo~te liczby zostały źle podane.

\vspace{\spaceFour}





\noindent
\StrWg{537}{6} Skala śmierci w Hiszpańskiej Wojnie Domowej jest zbyt mała,
by~można ją było nazwać holokaustem.

\vspace{\spaceFour}





\noindent
\StrWg{537}{6} Powinno tu~się znaleźć jawne potępienie zbrodni
nacjonalistów. Popełnienie ich w~reakcji nie znosi winy.

\vspace{\spaceFour}





\noindent
\StrWd{540}{7} Wcięcie tego akapitu jest zbyt duże.

\vspace{\spaceFour}





\noindent
\StrWd{543}{1} Do~formy całego przypisu niezbyt pasuje linia „Rafael Casa
de~la~Vega, \textit{Franco, żołnierz}, tłum.~J.~Chodorowski.”.

\vspace{\spaceFour}





\noindent
\Str{547} Byłoby dziwne, gdyby Churchill ostrzegał Wielką Brytanię przed
Hitlerem w~latach 1924--1932, skoro przez większość tego czasu był
on~człowiekiem zupełnie pozbawionym wpływów. Jednak w~historii zdarzały~się
już dziwniejsze rzeczy.

\vspace{\spaceFour}





\noindent
\Str{565} Tekst przypisów~69 i~70 trochę~się ze~sobą nie zgadzają.
W~przypisie 69~autorzy twierdzą, że~praca o~walkach na~Gudalcana Roberta
Leckiego jest pozycją niedoścignioną porównywalną tylko z~Tukidydesem.
Natomiast w~następnym, iż~najlepsza praca w~tym temacie to~ta autorstwa
Samuela Eliota Morisona.

\vspace{\spaceFour}





\noindent
\Str{566} Przypisy od~tłumacza~są źle ponumerowane ilością gwiazdek.

\vspace{\spaceFour}





\noindent
\StrWg{572}{12} Gwiazdka w~tej linii jest zbyt mała.

\vspace{\spaceFour}





\noindent
\StrWd{585}{17--15} Te~linie~są źle sformatowane.

\vspace{\spaceFour}



\noindent
\StrWd{609}{8} Gwiazdka w~tej linii jest za~mała.

\vspace{\spaceFour}





\noindent
\StrWg{623}{10} Gwiazdka w~tej linii jest zbyt mała.

\vspace{\spaceFour}





\noindent
\StrWd{778}{4--3} Te dwie linie, łącznie z~„251.” służącym za~odnośnik tego
przypisu powinny być częścią poprzedniego przypisu. Samo oznaczenie „251.”
jest częścią urwanych w~poprzedniej linii numerów stron: 250-251.

\vspace{\spaceFour}





\noindent
\StrWd{867}{17} Linia jest źle zedytowana. Drugie zdanie w~tej linii jest
początkiem następnej pozycji w~bibliografii, powinna więc być zgodnie z~tym
sformatowana.

\vspace{\spaceFour}





% ##################
\newpage

\CenterBoldFont{Błędy}


\begin{center}

  \begin{tabular}{|c|c|c|c|c|}
    \hline
    Strona & \multicolumn{2}{c|}{Wiersz} & Jest
                              & Powinno być \\ \cline{2-3}
    & Od góry & Od dołu & & \\
    \hline
    7   & &  4 & wszystko$^{ * }$ & wszystko \\
    7   & &  4 & 827 & 837 \\
    7   & &  3 & Rekonkwiście$^{ * }$ & Rekonkwiście \\
    23  & & 10 & \textit{Vhutch} & \textit{Church} \\
    24  & & 25 & Zbawiciela$^{ *^{ * } }$ & Zbawiciela$^{ ** }$ \\
    25  & &  9 & 1919 & 1819 \\
    32  & 11 & & dostosowania & do~stosowania \\
    50  & & 12 & za~panowania & rozpoczęta za~panowania \\
    55  & & 12 & piętnstu & piętnastu \\
    55  & &  7 & interesy & interesy Południa \\
    67  & 17 & & bezbożności”)$^{ 31 }$ & bezbożności”$^{ 31 }$) \\
    67  &  8 & & północy & południa \\
    68  & 21 & & siom & siłom \\
    85  &  1 & & Herald” Tribune” & Herald” \\
    96  & &  2 & W.H. Warren & W.H. Carroll \\
    97  & &  7 & „tak uważamy” & „Tak uważamy” \\
    104 & & 17 & i~związku & i~w~związku \\
    104 & &  2 & W.H.~Warren & W.H.~Carroll \\
    105 & & 11 & Counter-Revolution & Counter-Revolution” \\
    105 & &  5 & W.H.~Warren & W.H.~Carroll \\
    116 & & 15 & stał~się był & stał~się \\
    117 & & 23 & wyd.3,Boston & wyd.~3, Boston \\
    121 &  3 & & tonizowały & uspokajały \\
    126 & &  4 & Pampelunie. & Pampelunie). \\
    127 & 10 & & aż~przez & potem aż~przez \\
    135 & & 12 & doskonalej & doskonałej \\
    137 &  6 & & go & je \\
    139 & &  5 & \textit{1833} & \textit{1883} \\
    141 & &  8 & tom~VI, rozdział~XIV & rozdział~VIII, \\
    141 & &  7 & P\textit{olitical} & \textit{Political} \\
    \hline
  \end{tabular}





  \newpage

  \begin{tabular}{|c|c|c|c|c|}
    \hline
    Strona & \multicolumn{2}{c|}{Wiersz} & Jest
                              & Powinno być \\ \cline{2-3}
    & Od góry & Od dołu & & \\
    \hline
    141 & &  5 & rozdział zatytułowany & rozdział~II, \\
    150 &  5 & & dogmatach; & dogmatach, \\
    151 & &  8 & torturom... & torturom. \\
    151 & &  7 & miasta.. & miasta. \\
    152 & 22 & & i~i & i \\
    165 & &  7 & 2003) & 2003 \\
    170 & 16 & & potomek & bratanek \\
    170 & & 10 & potomka, „F\"{u}hrera & potomka „F\"{u}hrera \\
    171 & &  7 & skrajnym wręcz & wręcz skrajnym \\
    177 &  3 & & „byliście & „Byliście \\
    187 & 21 & & roku~Na & roku. Na \\
    190 & &  1 & Karl Marx & \textit{Karl Marx} \\
    194 & 12 & & etc.. & etc. \\
    194 & &  8 & spikerze & mówcy \\
    197 & &  7 & 1987) & 1987 \\
    198 & 16 & & wojny; & wojny \\
    199 & &  2 & 210 & 210. \\
    208 &  7 & & jest & jest natomiast \\
    208 & &  5 & wschodni, wschodni & zachodni, wschodni \\
    210 & &  6 & Pratt,, & Pratt, \\
    210 & &  5 & Carrol & Carroll \\
    223 & & 20 & dna & dnia \\
    228 & &  4 & 2005) & 2005 \\
    229 & & 14 & \textit{Westrn} & \textit{Western} \\
    233 & & 11 & piaty & piąty \\
    235 & 16 & & rzecz & Rzecz \\
    235 & & 15 & Uranu & Urana \\
    237 & &  1 & 1954) & 1954 \\
    239 & 12 & & wyrazili & nie~wyrazili \\
    239 & 15 & & z~zatem & a~zatem \\
    \hline
  \end{tabular}





  \begin{tabular}{|c|c|c|c|c|}
    \hline
    Strona & \multicolumn{2}{c|}{Wiersz} & Jest
                              & Powinno być \\ \cline{2-3}
    & Od góry & Od dołu & & \\
    \hline
    243 &  5 & & uczony; & uczony. \\
    243 &  5 & & roku1743 & roku 1743 \\
    248 & &  1 & 2008) & 2008 \\
    265 & & 13 & si & się \\
    267 & &  3 & (1944 ) & (1944) \\
    269 & &  5 & \textit{Germany, and} & \textit{Germany and} \\
    270 & &  1 & \textit{s.} & s. \\
    273 & &  3 & 1944 & 1994 \\
    294 &  7 & & światu”... & światu... \\
    299 & & 11 & Kołłnotaj & Kołłontaj \\
    299 & &  1 & 1988) & 1988 \\
    303 & &  4 & \textit{opończa}” & \textit{opończa} \\
    308 & & 19 & niszczysz & Niszczysz \\
    309 & 22 & & warstw & wszystkich warstw \\
    317 & &  3 & \textit{Kerensky; the} & \textit{Kerensky: The} \\
    320 & &  3 & Habsburg & \textit{Habsburg} \\
    324 & &  8 & eserowcow & eserowców \\
    327 &  7 & & w~coraz & ludzie w~coraz \\
    330 & &  1 & \textit{Wtnesses} & \textit{Witnesses} \\
    332 & & & Piotrogrodu,. & Piotrogrodu. \\ % Popraw tą linię
    338 & 22 & & roboty!, & roboty! \\
    349 & &  1 & \textit{s.} & s. \\
    351 &  4 & & 1917--1921 & 1914--1922 \\
    365 & &  5 & miasta ; & miasta; \\
    373 & 19 & & rok & rok. \\
    379 & &  2 & 1989) & 1989 \\
    380 &  7 & & zlej & złej \\
    380 & &  6 & destruktywna & destruktywną \\
    381 & &  1 & 1951) & 1951 \\
    385 & 10 & & dopływem & odpływem \\
    \hline
  \end{tabular}





  \begin{tabular}{|c|c|c|c|c|}
    \hline
    Strona & \multicolumn{2}{c|}{Wiersz} & Jest
                              & Powinno być \\ \cline{2-3}
    & Od góry & Od dołu & & \\
    \hline
    387 &  4 & & 1915--1922 & 1914--1922 \\
    387 & &  2 & 1989) & 1989 \\
    392 & &  3 & \textit{s.} & s. \\
    393 & &  6 & \textit{s.} & s. \\
    393 & &  2 & \textit{s.} & s. \\
    394 &  7 & & z & z~dala \\
    394 & &  1 & \textit{XV} , & \textit{XV}, \\
    396 & & 19 & Leonowi XII & Leonowi XIII \\
    408 & &  2 & London1971 & London 1971 \\
    409 & &  1 & \textit{Glory; Poland} & \textit{Glory: Poland} \\
    421 & & 17 & terroru\ldots. & terroru\ldots \\
    424 & &  5 & Radzieckiej.Trzon & Radzieckiej. Trzon \\
    430 & &  1 & \textit{under} & \textit{Under} \\
    431 & &  2 & \textit{s.} & s. \\
    432 & &  1 & \textit{s.} & s. \\
    433 & &  2 & \textit{s.} & s. \\
    435 &  2 & & krajem.. & krajem. \\
    436 & &  1 & \textit{s.} & s. \\
    442 &  6 & & imperium & imperium Czang \\
    443 &  7 & & doobra & dobra \\
    446 & 10 & & Baun & Braun \\
    448 & 15 & & Hunan Jiangxi. & Huan i~Jangxi \\
    457 & &  3 & \textit{war} & \textit{War} \\
    457 & &  2 & \textit{war} & \textit{War} \\
    459 & &  2 & \textit{s.} & s. \\
    460 & &  2 & \textit{s.} & s. \\
    461 & & 11 & \textit{s.} & s. \\
    462 & &  4 & \textit{Pro; Modern} & \textit{Pro. Modern} \\
    463 & &  7 & \textit{s.} & s. \\
    464 & &  8 & \textit{s.} & s. \\
    \hline
  \end{tabular}





  \newpage

  \begin{tabular}{|c|c|c|c|c|}
    \hline
    Strona & \multicolumn{2}{c|}{Wiersz} & Jest
                              & Powinno być \\ \cline{2-3}
    & Od góry & Od dołu & & \\
    \hline
    464 & &  3 & \textit{s.} & s. \\
    465 & &  9 & „Viva Cristo Rey” & \textit{Viva Cristo Rey} \\
    465 & &  8 & 194. 199. & 194, 199. \\
    468 &  6 & & \textit{Altars; Baltimore's}
           & \textit{Altars. Baltimore's} \\
    473 & 14 & & klepsydrze & klepsydrze. \\
    479 & &  6 & \textit{s.} & s. \\
    485 & 12 & & obserwują., wyciągając & obserwują. Wyciągając \\
    487 & & 14 & John a.~Ryan & John A.~Ryan \\
    488 & &  4 & pieniadze & pieniądze \\
    489 & 10 & & prac.$^{151}$ & prac$^{151}$. \\
    497 & 18 & & mniejszością & grupą \\
    498 &  1 & & od & do \\
    502 & 23 & & \textit{od} & \textit{of} \\
    506 & &  5 & zmienić & zmienić zdanie \\
    506 & &  1 & \textit{gwałtownie} & gwałtownie \\
    507 &  3 & & dłużej & długo \\
    507 & &  2 & \textit{Archipelago} III & \textit{Archipelago}, t.~III \\
    511 &  2 & & dotrzeć do~celu & dopłynąć do~celu \\
    514 & 11 & & Jarosławiu”)$^{ 22 }$ & Jarosławiu”$^{ 22 }$) \\
    514 & &  2 & \textit{labor} & \textit{Labor} \\
    517 & &  5 & Całkowita & całkowita \\
    520 & &  4 & Cronica de~Alfonso~III & \textit{Cronica de~Alfonso~III} \\
    521 & &  1 & wyd & wyd. \\
    522 & 18 & & kardynałem & kardynałem. \\
    523 & & 20 & roku & roku. \\
    525 &  8 & & z~gabinecie & w~gabinecie \\
    527 & & & Toledo$^{ 18 }$\textbf{.} & Toledo$^{ 18 }$. \\
    529 & 20 & & lewacki & lewicowy \\
    529 & &  3 & t.~Ivm Mardird & t.~I, Madrid \\
    530 & 21 & & \textit{Rey}” & \textit{Rey} \\
    \hline
  \end{tabular}





  \newpage

  \begin{tabular}{|c|c|c|c|c|}
    \hline
    Strona & \multicolumn{2}{c|}{Wiersz} & Jest
                              & Powinno być \\ \cline{2-3}
    & Od góry & Od dołu & & \\
    \hline
    531 & & 19 & Zamora\textbf{:} & Zamora: \\
    531 & & 18 & komunizmu”)$^{ 30 }$\textbf{.} & komunizmu”)$^{ 30 }$. \\
    531 & & 13 & W~spólnota & Wspólnota \\
    532 & 14 & & Cywilną$^{ 30 }$ & Cywilną \\
    532 & &  3 & roli,,  % ''
           & roli, \\
    535 & &  2 & \textit{mar tyrs} & \textit{Martyrs} \\
    536 & &  5 & \textit{into} & \textit{Into} \\
    537 & & 22 & Reyes)\textbf{.} & Reyes). \\
    537 & &  2 & \textit{1936} , & \textit{1936}, \\
    540 & &  9 & docenili & doceniliby \\
    541 &  8 & & (republikańskie & (Republikańskie \\
    541 & 10 & & komunistom). & komunistom.) \\
    541 & 15 & & faszyzmu”$^{ 55 }$\textbf{.} & faszyzmu”$^{ 55 }$. \\
    542 & & 21 & przywódca & Przywódca \\
    542 & & 21 & Boga” & Boga”. \\
    % Mogłem źle wyznaczyć początek cytatu.
    542 & &  6 & jednym & „jednym \\
    545 & &  4 & najzacieklejszych,. & najzacieklejszych \\
    545 & &  1 & red & red. \\
    548 & & 10 & 1987) & 1987 \\
    548 & &  5 & 1969) & 1969 \\
    549 & 18 & & obliczu & w~obliczu \\
    553 & &  3 & \textit{1939--1940} , & \textit{1939--1940}, \\
    555 & & 18 & także teraz & teraz także \\
    555 & &  5 & \textit{s.} & s. \\
    556 & &  4 & \textit{s.} & s. \\
    556 & &  1 & \textit{s.} & s. \\
    557 & &  4 & \textit{s.} & s. \\
    559 & & 12 & \textit{s.} & s. \\
    561 & &  4 & \textit{fate} & \textit{Fate} \\
    562 & &  1 & 1985) & 1985 \\
    \hline
  \end{tabular}





  % \begin{tabular}{|c|c|c|c|c|}
  %   \hline
  %   Strona & \multicolumn{2}{c|}{Wiersz} & Jest
  %   & Powinno być \\ \cline{2-3}
  %   & Od góry & Od dołu & & \\
  %   \hline
  %   %   & & & & \\
  %   %   & & & & \\
  %   %   & & & & \\
  %   %   & & & & \\
  %   %   & & & & \\
  %   %   & & & & \\
  %   %   & & & & \\
  %   %   & & & & \\
  %   %   & & & & \\
  %   %   & & & & \\
  %   %   & & & & \\
  %   %   & & & & \\
  %   %   & & & & \\
  %   %   & & & & \\
  %   %   & & & & \\
  %   %   & & & & \\
  %   %   & & & & \\
  %   %   & & & & \\
  %   %   & & & & \\
  %   %   & & & & \\
  %   %   & & & & \\
  %   %   & & & & \\
  %   %   & & & & \\
  %   %   & & & & \\
  %   %   & & & & \\
  %   %   & & & & \\
  %   %   & & & & \\
  %   %   & & & & \\
  %   %   & & & & \\
  %   %   & & & & \\
  %   %   & & & & \\
  %   %   & & & & \\
  %   %   & & & & \\
  %   %   & & & & \\
  %   %   & & & & \\
  %   %   & & & & \\
  %   %   & & & & \\
  %   %   & & & & \\
  %   \hline
  % \end{tabular}





  \newpage

  \begin{tabular}{|c|c|c|c|c|}
    \hline
    Strona & \multicolumn{2}{c|}{Wiersz} & Jest
                              & Powinno być \\ \cline{2-3}
    & Od góry & Od dołu & & \\
    \hline
    563 & 20 & & dwa & trzy \\
    564 &  1 & & samuraje & samurajowie \\
    565 & &  3 & kampanii; & kampanii. \\
    566 &  1 & & wspanialej & wspaniałej \\
    566 & 17 & & zaporami\ldots & zaopatrzeniem; \\
    566 & & 15 & Piekle!” & Piekle!”.” \\
    566 & & 12 & W~szakżeście& Wszakżeście \\
    570 &  4 & & wschodu & zachodu \\
    571 &  1 & & macDonald & MacDonald \\
    574 & &  5 & \textit{Preious} & \textit{Precious} \\
    575 & &  3 & 1997) & 1997 \\
    576 & &  1 & \textit{s.} & s. \\
    581 & &  1 & 212,. & 212. \\
    583 & &  7 & \textit{ThePrice} & \textit{The~Price} \\
    587 & & 10 & 1984) & 1984 \\
    591 & & 12 & \textit{Lost; American} & \textit{Lost: American} \\
    601 & &  3 & \textit{1949} & \textit{1949}, \\
    601 & &  2 & \textit{Confoct} & \textit{Conflict} \\
    605 & &  2 & \textit{Hungary from} & \textit{Hungary: From} \\
    607 & &  2 & \textit{between} & \textit{Between} \\
    609 & &  4 & kraju, W & kraju. W \\
    612 & &  3 & \textit{balance} & \textit{Balance} \\
    612 & &  2 & 1998) & 1998 \\
    617 & &  8 & Szpiegostwo & szpiegostwo \\
    634 & &  7 & \textit{s.} & s. \\
    646 & 19 & & \textit{Nie} & \textit{nie} \\
    685 & &  8 & roku Isaacs & R. Isaacs \\
    690 & &  5 & \textit{war} & \textit{War} \\
    693 & &  2 & Kambodży & z~Kambodży \\
    709 & &  6 & \textit{of} & \textit{to} \\
    \hline
  \end{tabular}





  % \begin{tabular}{|c|c|c|c|c|}
  %   \hline
  %   Strona & \multicolumn{2}{c|}{Wiersz} & Jest
  %   & Powinno być \\ \cline{2-3}
  %   & Od góry & Od dołu & & \\
  %   \hline
  %   %   & & & & \\
  %   %   & & & & \\
  %   %   & & & & \\
  %   %   & & & & \\
  %   %   & & & & \\
  %   %   & & & & \\
  %   %   & & & & \\
  %   %   & & & & \\
  %   %   & & & & \\
  %   %   & & & & \\
  %   %   & & & & \\
  %   %   & & & & \\
  %   %   & & & & \\
  %   %   & & & & \\
  %   %   & & & & \\
  %   %   & & & & \\
  %   %   & & & & \\
  %   %   & & & & \\
  %   %   & & & & \\
  %   %   & & & & \\
  %   %   & & & & \\
  %   %   & & & & \\
  %   %   & & & & \\
  %   %   & & & & \\
  %   %   & & & & \\
  %   %   & & & & \\
  %   %   & & & & \\
  %   %   & & & & \\
  %   %   & & & & \\
  %   %   & & & & \\
  %   %   & & & & \\
  %   %   & & & & \\
  %   %   & & & & \\
  %   %   & & & & \\
  %   %   & & & & \\
  %   %   & & & & \\
  %   %   & & & & \\
  %   %   & & & & \\
  %   \hline
  % \end{tabular}





  \newpage

  \begin{tabular}{|c|c|c|c|c|}
    \hline
    Strona & \multicolumn{2}{c|}{Wiersz} & Jest
                              & Powinno być \\ \cline{2-3}
    & Od góry & Od dołu & & \\
    \hline
    778 & &  5 & 250- & 250-251. \\
    785 & &  3 & Paul\_R\_Ehrlich & Paul\_R\_Ehrlich. \\
    790 & &  8 & Centrulo I~Amy & Centrulo i~Amy \\
    791 & &  4 & html & html. \\
    793 & &  7 & 145 & 145. \\
    793 & &  5 & \textit{s.} & s. \\
    797 & &  6 & Zob. & zob. \\
    797 & &  5 & Zob. & zob. \\
    797 & &  4 & Zob. & zob. \\
    797 & &  1 & Las & Last \\
    798 & 21 & & \textit{Kościół} & \textit{Kościół~są} \\
    798 & 22 & & \textit{są~Drogą} & \textit{Drogą} \\
    858 &  5 & & Pio Non (bł.~Pius~IX):
           & \textit{Pio Non (bł.~Pius~IX):} \\
    858 & 19 & & \textit{ofCatholic} & \textit{of Catholic} \\
    858 & 19 & & \textit{History,}(St.~Louis
           & \textit{History} (St.~Louis \\
    859 & 13 & & portugalskiej & portugalskiej. \\
    860 &  5 & & DuffDavid. & Duff David \\
    861 &  7 & & państwa.. & państwa. \\
    861 & 18 & & wyd.. & wyd. \\
    862 &  6 & & S. John Brown & S., \textit{John Brown} \\
    862 & &  2 & Jen. & Jen, \\
    864 &  5 & & York, & York \\
    864 & 20 & & \textit{against} & \textit{Against} \\
    864 & 21 & & York, & York \\
    866 & &  2 & FDR & \textit{FDR} \\
    867 & 15 & & York, & York \\
    867 & &  2 & York, & York \\
    868 &  3 & & 2004.. & 2004. \\
    868 & 15 & & York, & York \\
    868 & 23 & & York, & York \\
    \hline
  \end{tabular}





  \newpage

  \begin{tabular}{|c|c|c|c|c|}
    \hline
    Strona & \multicolumn{2}{c|}{Wiersz} & Jest
    & Powinno być \\ \cline{2-3}
    & Od góry & Od dołu & & \\
    \hline
    868 & 25 & & \textit{kardynał} & \textit{Cardinal} \\
    869 & 24 & & \textit{Denikin} & \textit{Denikin.} \\
    869 & & 12 & wojskowości.. & wojskowości \\
    870 &  8 & & 1937) & 1937). \\
    871 & 13 & & 1939 1961 & 1939, 1961 \\
    871 & 17 & & jedneaj & jednej \\
    871 & & 13 & \textit{Day 1918: World} & \textit{Day, 1918. World} \\
    871 & & 12 & \textit{its} & \textit{Its} \\
    871 & &  7 & \textit{under} & \textit{Under} \\
    873 &  1 & & York, & York \\
    873 &  9 & & York, & York \\
    873 & 10 & & York, & York \\
    873 & &  5 & 1958 1966 & 1958, 1966 \\
    874 & &  6 & Najlepsze I~najbardziej & Najlepsze i~najbardziej \\
    %   & & & & \\
    %   & & & & \\
    %   & & & & \\
    %   & & & & \\
    %   & & & & \\
    %   & & & & \\
    %   & & & & \\
    %   & & & & \\
    %   & & & & \\
    %   & & & & \\
    %   & & & & \\
    %   & & & & \\
    %   & & & & \\
    %   & & & & \\
    %   & & & & \\
    %   & & & & \\
    %   & & & & \\
    %   & & & & \\
    %   & & & & \\
    %   & & & & \\
    %   & & & & \\
    %   & & & & \\
    %   & & & & \\
    %   & & & & \\
    %   & & & & \\
    %   & & & & \\
    %   & & & & \\
    %   & & & & \\
    %   & & & & \\
    %   & & & & \\
    %   & & & & \\
    %   & & & & \\
    %   & & & & \\
    %   & & & & \\
    %   & & & & \\
    %   & & & & \\
    %   & & & & \\
    %   & & & & \\
    \hline
  \end{tabular}

\end{center}

\vspace{\spaceTwo}


\noindent
\StrWg{103}{5} \\
\Jest  mianem krucjaty (\textit{la~cruzada} ) określali \\
\Powin określali mianem krucjaty (\textit{la~cruzada}) \\
\StrWd{170}{10} \\
\Jest  „F\"{u}hrera z~Poczdamu”, ojca Fryderyka Wielkiego \\
\Powin „F\"{u}hrera z~Poczdamu”, Fryderyka Williama~I, ojca Fryderyka
Wielkiego \\
\StrWd{228}{5} \\
\Jest  The~Victory~of Reason: How Christianity Led to Freedom,
Capitalism and~Western Success \\
\Powin \textit{The~Victory~of Reason: How Christianity Led to Freedom,
  Capitalism and~Western Success} \\
\StrWg{234}{18} \\
\Jest  zapoczątkowujących teorię indukcji elektromagnetycznej \\
\Powin które doprowadziły do~powstania teorii indukcji
elektromagnetycznej \\
\StrWd{237}{1} \\
\Jest  Ford: The~Times, the~Man, and~the~Company \\
\Powin \textit{Ford: The~Times, the~Man, and~the~Company} \\
\StrWd{246}{9} \\
\Jest  Alexander Graham Bell and~the~Passion for~Invention \\
\Powin \textit{Alexander Graham Bell and~the~Passion for~Invention} \\
\StrWd{299}{2} \\
\Jest  Three Who Made a~Revolution \\
\Powin \textit{Three Who Made a~Revolution} \\
\StrWd{383}{18} \\
\Jest  Kołczak \\
\Powin Kołczak doszedł do wniosku \\
\StrWd{507}{14} \\
\Jest \textit{przeciwko zastosowaniu kary śmierci. Co~więcej,
  przekonał do~swego poglądu Politbiuro.} \\
\Powin przeciwko zastosowaniu kary śmierci. Co~więcej, przekonał
do~swego poglądu Politbiuro. \\
\StrWd{545}{10} \\
\Jest  i~dwutomowa \textit{Visions~of Glory} (Boston 1983),
\textit{Alone} \\
\Powin w~dwóch tomach: \textit{Visions~of Glory} (Boston 1983)
i~\textit{Alone} \\
\StrWd{566}{5} \\
\Jest  South Pacific Combat Air Transport \\
\Powin SCAT (\textit{South Pacific Combat Air Transport}) \\
\StrWd{566}{2} \\
\Jest  WAC~~(Women's Army Corps) \\
\Powin WAC~(\textit{Women's Army Corps}) \\
\StrWd{790}{3--1} \\
\Jest „Akcja afirmatywna w~orzecznictwie Sądu Najwyższego Stanów
Zjednoczonych”, Z~problemów bezpieczeństwa. Prawa człowieka \\
\Powin \textit{Akcja afirmatywna w~orzecznictwie Sądu Najwyższego Stanów
  Zjednoczonych}, w:~\textit{Z~problemów bezpieczeństwa. Prawa człowieka} \\
\StrWd{791}{3} \\
\Jest \textit{Infant Himicides through} \\
\Powin Bogomir Kuhar, \textit{Infant Homicides Through} \\
\StrWd{797}{6} \\
\Jest  „Why Can't We~Love Them Both?” \\
\Powin \textit{Why Can't We~Love Them Both?} \\
\StrWd{799}{5} \\
\Jest  „Pope John Paul~II's Encyclical \textit{Veritatis Splendor}” \\
\Powin \textit{Pope John Paul~II's Encyclical „Veritatis Splendor”} \\



% ############################










% ############################
\newpage

\Work{ % Autor i tytuł dzieła
  Ks. Bogusław Kumor \\
  \textit{Historia Kościoła. Tom~I: Starożytność chrześcijańska},
  \cite{KumorHistoriaKosciolaVolI2003}}


% ##################
\CenterBoldFont{Błędy}


\begin{center}

  \begin{tabular}{|c|c|c|c|c|}
    \hline
    Strona & \multicolumn{2}{c|}{Wiersz} & Jest
                              & Powinno być \\ \cline{2-3}
    & Od góry & Od dołu & & \\
    \hline
    14 & 11 & & rzeciwieństwie & przeciwieństwie \\
    14 & 15 & & jednej formy & jedną formę \\
    % & & & & \\
    % & & & & \\
    % & & & & \\
    % & & & & \\
    % & & & & \\
    % & & & & \\
    \hline
  \end{tabular}

\end{center}

\vspace{\spaceTwo}



% ############################










% ######################################
\newpage

\section{Historia świętej wiary, I~wiek przed i~I po Chrystusie}

\vspace{\spaceTwo}
% ######################################



% ############################
\Work{ % Autor i tytuł dzieła
  Wojciech Roszkowski \\
  \textit{Świat Chrystusa. Tom~I}, \cite{RoszkowskiSwiatChrystusVolI2016}}

\vspace{0em}


% ##################
\CenterBoldFont{Uwagi do~konkretnych stron}

\vspace{0em}


\noindent
\StrWg{63}{24} Wydaje mi się, że~zaraz przed zdaniem
„Oskarżycielem Mariammy\ldots” powinien znajdować~się tekst mówiący
o~tym, że~Herodes postawił ją przed sądem. Prawdopodobnie został on
zgubiony w~trakcie składania książki.

\vspace{\spaceFour}





\noindent
\StrWg{81}{4} Zdanie „miłość skupia i~uruchamia wszystkie cząstki\ldots” brzmi
bardzo niezręcznie i~powinno być jakoś poprawione.

\vspace{\spaceFour}





\noindent
\StrWg{84}{8} Podane tu twierdzenie jest prostym wnioskiem z~twierdzenia
Pitagorasa, więc nie wydaje~się, że~powinno ono uchodzić za~jakieś
szczególnie warte uwagi osiągnięcie Platona. Przynajmniej nie uchodziłoby
za takie dzisiaj, jednak ponieważ matematyka przeszła od V i~IV wieku przed
Chrystusem długą drogę, możliwe, iż w~tamtych czasach znalezienie takiej
konsekwencji twierdzenia Pitagorasa było bardzo ważnym odkryciem.

\vspace{\spaceFour}





\noindent
\StrWg{85}{10} Zdanie „przez pomiar kierunków, w~których punkt ten jest
widziany z~dwóch punktów o~znanych odległościach” brzmi niezręcznie i~chyba
warto byłoby je poprawione.

\vspace{\spaceFour}





\noindent
\StrWd{95}{5} Nie rozumiem co oznacza przytoczone słowo „kpy”. Może jest to
błędne zapisane słowo „kiepy”?

\vspace{\spaceFour}



% \start \Str{}

% \vspace{\spaceFour}

% \start \Str{}

% \vspace{\spaceFour}

% \start \Str{}

% \vspace{\spaceFour}

% \start \Str{}


% ##################
\newpage

\CenterBoldFont{Błędy}


\begin{center}

  \begin{tabular}{|c|c|c|c|c|}
    \hline
    Strona & \multicolumn{2}{c|}{Wiersz} & Jest
                              & Powinno być \\ \cline{2-3}
    & Od góry & Od dołu & & \\
    \hline
    6   & & 12 & ona na & na \\
    19  & 21 & & Ciągłą rachubę & Rachubę \\
    19  & &  5 & \textit{Time.The} & \textit{Time. The} \\
    31  & &  2 & passim, Kasjusz & passim; Kasjusz \\
    31  & &  2 & passim\textit{;} & passim; \\
    33  & &  3 & położona & położona jest \\
    34  &  1 & & jest na & na \\
    37  & &  1 & 2001) & 2001 \\
    38  & &  2 & \textit{Jasna} & \textit{Yasna} \\
    43  & &  7 & 1985) & 1985 \\
    43  & &  2 & W.Malandra & W.~Malandra \\
    43  & &  2 & //garodman.htm & /garodman.htm \\
    44  & &  7 & part\textbf{1} & part~1 \\
    45  & &  4 & \textit{H.Dubs} & \textit{H.~Dubs} \\
    % 47 Ware: Hertfordshire: Wordsworth?????
    49  & &  9 & Obyczajów & Obyczajów” \\
    58  & &  5 & Wojna żydowska & \textit{Wojna żydowska} \\
    58  & &  2 & J.Harrington & J. Harrington \\
    62  & &  2 & Życie Jezusa Chrystusa & \textit{Życie Jezusa Chrystusa} \\
    62  & &  2 & PX & PAX \\
    65  & &  1 & s.153--155; & s.~153--155. \\
    70  &  9 & & zdobyć Rzymowi & zdobyć \\
    70  & & 14 & bowiem także & bowiem \\
    72  & &  1 &  Dzieje od~założenia miasta
           & \textit{Dzieje od~założenia miasta} \\
    76  & & 14 & pierwszy & pierwszy patrycjusz \\
    77  &  5 & & żoną, Vipsanią, & żoną Vipsanią, \\
    77  &  5 & & syna, Drususa. & syna Drususa. \\
    78  &  3 & & Po & W~miejscach po \\
    % 79?????
    84  & &  2 & 1970) & 1970 \\
    84  & &  2 & 2000) & 2000 \\
    84  & &  1 & Słownik f\textit{ilozofów} & \textit{Słownik filozofów} \\
    \hline
  \end{tabular}





  \newpage

  \begin{tabular}{|c|c|c|c|c|}
    \hline
    Strona & \multicolumn{2}{c|}{Wiersz} & Jest
                              & Powinno być \\ \cline{2-3}
    & Od góry & Od dołu & & \\
    \hline
    87  & & 16 & strumienia & strumieniem \\
    99  & &  2 & s.188--194 & s.~188--194 \\
    99  & &  2 & J.Harrington & J. Harrington \\
    100 & &  2 & Wydawnictwo~m & Wydawnictwo~M \\
    105 & &  8 & Diocletianus & \textit{Diocletianus} \\
    108 & &  1 & Złota gałąź & \textit{Złota gałąź} \\
    110 & &  6 & PMich & P. Mich \\
    112 & & 12 & Tereny Alpes Maritimae, & Tereny \\
    112 & &  2 & Mówią wykopaliska & \textit{Mówią wykopaliska} \\
    115 & &  9 & Berenikę & Berenika \\
    126 & &  2 & \textit{A.D.},Guntur & \textit{A.D.}, Guntur \\
    126 & &  1 & P.G.Publishers & P.G. Publishers \\
    127 & &  2 & starożytnego Rzymu & \textit{starożytnego Rzymu} \\
    132 & &  1 & Życie codzienne & \textit{Życie codzienne} \\
    144 & &  2 & Wydawnic zy & Wydawniczy \\
    166 & &  1 & Psalmów,114 & Psalmów, 114 \\
    174 & &  1 & frg & frg. \\
    175 & &  5 & G.Miller & G. Miller \\
    175 & &  5 & Starożytni & \textit{Starożytni} \\
    175 & &  4 & W.Gołębiewski & W. Gołębiewski \\
    175 & &  3 & J.Stroynowski & J. Strounowski \\
    179 & &  1 & Starożytni olimpijczycy
           & \textit{Starożytni olimpijczycy} \\
    180 & &  2 & Wędrówki po~Helladzie & \textit{Wędrówki po~Helladzie} \\
    180 & &  2 & Starożytni olimpijczycy
           & \textit{Starożytni olimpijczycy} \\
    182 & &  2 & www1.fhw.gr & www.fhw.gr \\
    185 & &  3 & B.F.Gajdukiewicz & B.F. Gajdukiewicz \\
    185 & &  2 & M.I.Rostwcew & M.I. Rostwcew \\
    186 & &  1 & 1994) & 1994 \\
    187 & &  3 & B.Baranowski & B. Baranowski \\
    188 & &  2 & Historia Azerbejdżanu & \textit{Historia Azerbejdżanu} \\
    \hline
  \end{tabular}





  \newpage

  \begin{tabular}{|c|c|c|c|c|}
    \hline
    Strona & \multicolumn{2}{c|}{Wiersz} & Jest
                              & Powinno być \\ \cline{2-3}
    & Od góry & Od dołu & & \\
    \hline
    190 & &  1 & własne jęz. & własne z~jęz. \\
    191 & &  4 & H.Ananikian & H. Ananikian \\
    199 & &  1 & Starożytni Grecy i~Rzymianie
           & \textit{Starożytni Grecy i~Rzymianie} \\
    199 & &  1 & Dzieje od~założenia Miasta
           & \textit{Dzieje od~założenia Miasta} \\
    205 & &  5 & B.T.Batsford & B.T. Batsford \\
    226 & &  1 & 1989) & 1989 \\
    244 & &  4 & Rozmowy tuskulańskie
           & \textit{Rozmowy tuskulańskie} \\
    245 & &  6 & Ody & \textit{Ody} \\
    250 & &  1 & Osły & \textit{Osły} \\
    253 & &  2 & CAIS//History & CAIS/History \\
    255 & &  4 & D.P.M.Weerakkody & D.P.M. Weerakkody \\
    299 & &  1 & konfucjańskie & \textit{konfucjańskie} \\
    303 & &  6 & Dialogi konfucjańskie & \textit{Dialogi konfucjańskie} \\
    330 & &  2 & L.Shinne & L. Shinne \\
    332 & &  2 & prośba & prośba'' \\
    333 & &  4 & W.Malandra & W. Malandra \\
    346 & &  3 & Pieśni & \textit{Pieśni} \\
    346 & &  3 & Commentary on~Catullus & „Commentary on~Catullus” \\
    347 & &  3 & Dialogi konfucjańskie & \textit{Dialogi konfucjańskie} \\
    354 & &  2 & 1957) & 1957 \\
    360 & &  3 & Starożytni Celtowie & \textit{Starożytni Celtowie} \\
    367 & &  2 & A.Barton & A. Barton \\
    383 & &  6 & \textit{History} (Dublin: & \textit{History}, Dublin: \\
    387 & &  1 & O~wróżbiarstwie & \textit{O~wróżbiarstwie} \\
    387 & &  1 & Słowo jest cieniem & \textit{Słowo jest cieniem} \\
    394 & &  4 & A.Bard & A. Bard \\
    395 & &  2 & Poznaj Świat & „Poznaj Świat” \\
    % 396, PMich ????
    398 & &  6 & A.Bard & A. Bard \\
    408 & &  2 & 2010 & 2010) \\
    414 & &  4 & \textit{Boski, August} & \textit{Boski August} \\
    % & & & & \\
    % & & & & \\
    % & & & & \\
    % & & & & \\
    % & & & & \\
    % & & & & \\
    % & & & & \\
    % & & & & \\
    % & & & & \\
    % & & & & \\
    % & & & & \\
    % & & & & \\
    \hline
  \end{tabular}





  \newpage

  \begin{tabular}{|c|c|c|c|c|}
    \hline
    Strona & \multicolumn{2}{c|}{Wiersz} & Jest
                              & Powinno być \\ \cline{2-3}
    & Od góry & Od dołu & & \\
    \hline
    421 & &  3 & Starożytna Polska & \textit{Starożytna Polska} \\
    424 & &  1 & Mitologia germańska & \textit{Mitologia germańska} \\
    424 & &  1 & 1979) & 1979 \\
    429 & &  4 & Żaby & \textit{Żaby} \\
    429 & &  4 & \textit{komedie} & \textit{Komedie} \\
    430 & &  5 & C.G.Jung & C.G. Jung \\
    430 & &  5 & C.Kerenyi & C. Kerenyi \\
    %   & & & & \\
    %   & & & & \\
    %   & & & & \\
    %   & & & & \\
    %   & & & & \\
    %   & & & & \\
    %   & & & & \\
    %   & & & & \\
    %   & & & & \\
    %   & & & & \\
    %   & & & & \\
    %   & & & & \\
    %   & & & & \\
    %   & & & & \\
    %   & & & & \\
    %   & & & & \\
    %   & & & & \\
    %   & & & & \\
    %   & & & & \\
    %   & & & & \\
    %   & & & & \\
    %   & & & & \\
    %   & & & & \\
    %   & & & & \\
    %   & & & & \\
    %   & & & & \\
    %   & & & & \\
    \hline
  \end{tabular}





  % \begin{tabular}{|c|c|c|c|c|}
  %   \hline
  %   & \multicolumn{2}{c|}{} & & \\
  %   Strona & \multicolumn{2}{c|}{Wiersz} & Jest
  %                             & Powinno być \\ \cline{2-3}
  %   & Od góry & Od dołu & & \\
  %   \hline
  %   %   & & & & \\
  %   %   & & & & \\
  %   %   & & & & \\
  %   %   & & & & \\
  %   %   & & & & \\
  %   %   & & & & \\
  %   %   & & & & \\
  %   %   & & & & \\
  %   %   & & & & \\
  %   %   & & & & \\
  %   %   & & & & \\
  %   %   & & & & \\
  %   %   & & & & \\
  %   %   & & & & \\
  %   %   & & & & \\
  %   %   & & & & \\
  %   %   & & & & \\
  %   %   & & & & \\
  %   %   & & & & \\
  %   %   & & & & \\
  %   %   & & & & \\
  %   %   & & & & \\
  %   %   & & & & \\
  %   %   & & & & \\
  %   %   & & & & \\
  %   %   & & & & \\
  %   %   & & & & \\
  %   \hline
  % \end{tabular}
\end{center}

\vspace{\spaceTwo}


\noindent
\StrWd{18}{3} \\
\Jest  Dicta. Zbiór łacińskich sentencji, przysłów, zwrotów, powiedzeń
z~indeksem osobowym i~tematycznym, zebrał Czesław Michalunio~SJ \\
\Powin \textit{Dicta. Zbiór łacińskich sentencji, przysłów, zwrotów,
  powiedzeń, z~indeksem osobowym i~tematycznym,
  zebrał Czesław Michalunio~SJ} \\
\StrWd{19}{7} \\
\Jest  Geneza chrześcijańskiej rachuby lat (Tyniec: Wydawnictwo
Benedyktynów, 2000); \\
\Powin \textit{Geneza chrześcijańskiej rachuby lat}, Tyniec: Wydawnictwo
Benedyktynów, 2000; \\
\StrWd{37}{6} \\
\Jest  Religie świata rzymskiego \\
\Powin \textit{Religie świata rzymskiego} \\
\StrWd{37}{6} \\
\Jest  Historia wierzeń i~idei religijnych \\
\Powin \textit{Historia wierzeń i~idei religijnych} \\
\StrWd{37}{5} \\
\Jest  Historia wierzeń i~idei religijnych \\
\Powin \textit{Historia wierzeń i~idei religijnych} \\
\StrWd{37}{4} \\
\Jest  Starożytne cywilizacje. Wierzenia, mitologia, sztuka \\
\Powin \textit{Starożytne cywilizacje. Wierzenia, mitologia, sztuka} \\
\StrWd{37}{2} \\
\Jest  Wielki atlas mitów i~legend świata \\
\Powin \textit{Wielki atlas mitów i~legend świata} \\
\StrWd{51}{2} \\
\Jest  Starożytne cywilizacje. Wierzenia, mitologia, sztuka \\
\Powin \textit{Starożytne cywilizacje. Wierzenia, mitologia, sztuka} \\
\StrWd{51}{1} \\
\Jest  Historia wierzeń i~idei religijnych \\
\Powin \textit{Historia wierzeń i~idei religijnych} \\
\StrWd{58}{3} \\
\Jest  Biblia i~starożytny świat \\
\Powin \textit{Biblia i~starożytny świat} \\
\StrWd{58}{2} \\
\Jest  Od~Księgi Rodzaju do~Ewangelii \\
\Powin \textit{Od~Księgi Rodzaju do~Ewangelii} \\
\StrWd{66}{4} \\
\Jest  Księdze o~narodzinach Świetej Maryi \\
\Powin \textit{Księdze o~narodzinach Świętej Maryi} \\
\StrWd{70}{4} \\
\Jest  Mała encyklopedia kultury antycznej \\
\Powin \textit{Mała encyklopedia kultury antycznej} \\
\StrWd{73}{4} \\
\Jest  Starożytni Grecy i~Rzymianie w~życiu prywatnym i~państwowym \\
\Powin \textit{Starożytni Grecy i~Rzymianie w~życiu prywatnym
  i~państwowym} \\
\StrWd{73}{3} \\
\Jest  Życie codzienne w~Rzymie \\
\Powin \textit{Życie codzienne w~Rzymie} \\
\StrWd{73}{2} \\
\Jest  Census populi. Demografia starożytnego Rzymu \\
\Powin \textit{Census populi. Demografia starożytnego Rzymu} \\
\StrWd{76}{2} \\
\Jest  Starożytni Grecy i~Rzymianie w~życiu prywatnym i~państwowym \\
\Powin \textit{Starożytni Grecy i~Rzymianie w~życiu prywatnym
  i~państwowym} \\
\StrWd{91}{3} \\
\Jest  O~najwyższy dobru i~złu \\
\Powin \textit{O~najwyższym dobru i~złu} \\
\StrWd{92}{2} \\
\Jest  Grecy o~miłości, szczęściu i~życiu \\
\Powin \textit{Grecy o~miłości, szczęściu i~życiu} \\
\StrWd{96}{3} \\
\Jest  Grecy o~miłości, szczęściu i~życiu \\
\Powin \textit{Grecy o~miłości, szczęściu i~życiu} \\
\StrWd{96}{1} \\
\Jest  Grecy o~miłości, szczęściu i~życiu \\
\Powin \textit{Grecy o~miłości, szczęściu i~życiu} \\
\StrWd{99}{3} \\
\Jest  Życie Jezusa Chrystusa \\
\Powin \textit{Życie Jezusa Chrystus} \\
\StrWd{100}{3} \\
\Jest  Geneza chrześcijańskiej rachuby lat \\
\Powin \textit{Geneza chrześcijańskiej rachuby lat} \\
\StrWd{105}{7} \\
\Jest  Prowincje i~społeczeństwa prowincjonalne we~wschodniej części
basenu Morza Śródziemnego w~okresie od~Augusta do~Sewerów
(31 r.p.n.e. --~235 r.n.e) \\
\Powin \textit{Prowincje i~społeczeństwa prowincjonalne we~wschodniej
  części basenu Morza Śródziemnego w~okresie od~Augusta do~Sewerów
  (31 r.~p.n.e. -- 235 r.~n.e)} \\
\StrWd{105}{2} \\
\Jest  Bóstwa, kulty i~rytuały starożytnego Egiptu \\
\Powin \textit{Bóstwa, kulty i~rytuały starożytnego Egiptu} \\
\StrWd{109}{1} \\
\Jest  Bóstwa, kulty i~rytuały starożytnego Egiptu \\
\Powin \textit{Bóstwa, kulty i~rytuały starożytnego Egiptu} \\
\StrWd{110}{3} \\
\Jest  Bóstwa, kulty i~rytuały starożytnego Egiptu \\
\Powin \textit{Bóstwa, kulty i~rytuały starożytnego Egiptu} \\
\StrWd{110}{2} \\
\Jest  Starożytne cywilizacje. Wierzenia, mitologie, sztuka \\
\Powin \textit{Starożytne cywilizacje. Wierzenia, mitologie, sztuka} \\
\StrWd{110}{2} \\
\Jest  Wielki atlas mitów i~legend świata \\
\Powin \textit{Wielki atlas mitów i~legend świata} \\
\StrWd{130}{2} \\
\Jest  Starożytni Grecy i~Rzymianie \\
\Powin \textit{Starożytni Grecy i~Rzymianie} \\
\StrWd{134}{3} \\
\Jest  Historia społeczna starożytnego Rzymu \\
\Powin \textit{Historia społeczna starożytnego Rzymu} \\
\StrWd{135}{3} \\
\Jest  Miłość w~starożytnym Rzymie \\
\Powin \textit{Miłość w~starożytnym Rzymie} \\
\StrWd{150}{2} \\
\Jest  Grecy o~miłości, szczęściu i~życiu \\
\Powin \textit{Grecy o~miłości, szczęściu i~życiu} \\
\StrWd{152}{3} \\
\Jest  Encyklopedia archeologiczna Ziemi Świętej \\
\Powin \textit{Encyklopedia archeologiczna Ziemi Świętej} \\
\StrWd{174}{3} \\
\Jest  Historia wychowania w~starożytności \\
\Powin \textit{Historia wychowania w~starożytności} \\
\StrWd{174}{1} \\
\Jest  Liryka starożytnej Grecji \\
\Powin \textit{Liryka starożytnej Grecji} \\
\StrWd{175}{5} \\
\Jest  Sportowe życie antyczne Grecji \\
\Powin \textit{Sportowe życie antycznej Grecji} \\
\StrWd{175}{4} \\
\Jest  Dictionary~of Greek and Roman Antiquities \\
\Powin \textit{Dictionary~of Greek and Roman Antiquites} \\
\StrWd{199}{2} \\
\Jest  Historia wychowania w~starożytności \\
\Powin \textit{Historia wychowania w~starożytności} \\
\StrWd{208}{2} \\
\Jest  Barwny półświatek starożytnego Rzymu \\
\Powin \textit{Barwny półświatek starożytnego Rzymu} \\
\StrWd{217}{2} \\
\Jest  Życie codzienne w~Pompejach \\
\Powin \textit{Życie codzienne w~Pompejach} \\
\StrWd{226}{3} \\
\Jest  Buddyzm-dżinizm-religie ludów pierwotnych \\
\Powin \textit{Buddyzm-dżinizm-religie ludów pierwotnych} \\
\StrWd{230}{3} \\
\Jest  Życie codzienne Etrusków \\
\Powin \textit{Życie codzienne Etrusków} \\
% 236????
\StrWd{236}{2} \\
\Jest  Starożytni Grecy i~Rzymianie\ldots \\
\Powin \textit{Starożytni Grecy i~Rzymianie}\ldots \\
\StrWd{242}{2} \\
\Jest  Liryka Starożytnej Grecji \\
\Powin \textit{Liryka Starożytnej Grecji} \\
\StrWd{245}{2} \\
\Jest  Życie codzienne Palestynie w~czasach Chrystusa \\
\Powin \textit{Życie codzienne w~Palestynie w~czasach Chrystusa} \\
\StrWd{246}{4} \\
\Jest  Piekło. Oddalenie od~Domu Ojca \\
\Powin \textit{Piekło. Oddalenie od~Domu Ojca} \\
\StrWd{262}{2} \\
\Jest  Życie codzienne w~dawnych Indiach \\
\Powin \textit{Życie codzienne w~dawnych Indiach} \\
\StrWd{273}{23} \\
\Jest  Religie świata rzymskiego \\
\Powin \textit{Religie świata rzymskiego} \\
\StrWd{273}{23} \\
\Jest  Historia wierzeń i~idei religijnych \\
\Powin \textit{Historia wierzeń i~idei religijnych} \\
\StrWd{274}{2} \\
\Jest  Historia wierzeń i~idei religijnych \\
\Powin \textit{Historia wierzeń i~idei religijnych} \\
\StrWd{279}{7} \\
\Jest  Niezwykła technika starożytności \\
\Powin \textit{Niezwykła technika starożytności} \\
\StrWd{304}{1} \\
\Jest  Grecy o~miłości, szczęściu i~życiu \\
\Powin \textit{Grecy o~miłości, szczęściu i~życiu} \\
\StrWd{312}{5} \\
\Jest  Rzymska elegia miłosna, Wybór i~przekład Anna Świderska \\
\Powin \textit{Rzymska elegia miłosna. Wybór i~przekład Anna Świderska} \\
\StrWd{315}{1} \\
\Jest  Wędrówki po~Helladzie \\
\Powin \textit{Wędrówki po~Helladzie} \\
\StrWd{315}{1} \\
\Jest  Starożytni olimpijczycy \\
\Powin \textit{Starożytni olimpijczycy} \\
\StrWd{332}{3} \\
\Jest  Grecy o~miłości, szczęściu i~życiu \\
\Powin \textit{Grecy o~miłości, szczęściu i~życiu} \\
\StrWd{332}{2} \\
\Jest  Grecy o~miłości, szczęściu i~życiu \\
\Powin \textit{Grecy o~miłości, szczęściu i~życiu} \\
\StrWd{334}{1} \\
\Jest  miłości, szczęściu i~życiu \\
\Powin \textit{miłości, szczęściu i~życiu} \\
\StrWd{336}{4} \\
\Jest  Liryka starożytnej Grecji \\
\Powin \textit{Liryka starożytnej Grecji} \\
\StrWd{344}{2} \\
\Jest  Starożytni Grecy i~Rzymianie \\
\Powin \textit{Starożytni Grecy i~Rzymianie} \\
\StrWd{364}{2} \\
\Jest  O~najwyższy dobru i~złu \\
\Powin \textit{O~najwyższym dobru i~złu} \\
\StrWd{364}{2} \\
\Jest  Słowo jest cieniem czynu \\
\Powin \textit{Słowo jest cieniem czynu} \\
\StrWd{366}{1} \\
\Jest  Grecy o~miłości, szczęściu i~życiu \\
\Powin \textit{Grecy o~miłości, szczęściu i~życiu} \\
\StrWd{367}{6} \\
\Jest  Liryka starożytnej Grecji \\
\Powin \textit{Liryka starożytnej Grecji} \\
\StrWd{367}{4} \\
\Jest  \textit{Astronomic}a \\
\Powin \textit{Astronomica} \\
\StrWd{383}{1} \\
\Jest  Słowo jest cieniem czynu \\
\Powin \textit{Słowo jest cieniem czynu} \\
\StrWd{387}{2} \\
\Jest  Słowo jest cieniem czynu \\
\Powin \textit{Słowo jest cieniem czynu} \\
\StrWd{424}{3} \\
\Jest  Historia wierzeń i~idei religijnych \\
\Powin \textit{Historia wierzeń i~idei religijnych} \\
\StrWd{424}{2} \\
\Jest  Starożytne cywilizacje. Wierzenia, mitologia, sztuka \\
\Powin \textit{Starożytne cywilizacje. Wierzenia, mitologia, sztuka} \\
\StrWd{424}{2} \\
\Jest  Wielki atlas mitów i~legend świata \\
\Powin \textit{Wielki atlas mitów i~legend świata} \\



% ############################










% ######################################
\newpage

\section{Historia świętej wiary, VI i~VII wiek po Chrystusie}

\vspace{\spaceTwo}
% ######################################



% ############################
\Work{ % Autor i tytuł dzieła
  Łukasz Czarnecki \\
  \textit{Konstantynopol~626}, \cite{} % {CzarneckiKonstantynopol626Wyd2017}
}

\vspace{0em}


% ##################
\CenterBoldFont{Uwagi do~konkretnych stron}


\noindent
\StrWg{58}{18} Mam wątpliwość czy~wszystkie słowa wyróżnione tu~kursywą~są
częścią cytowanego fragmentu.

\vspace{\spaceFour}





% ##################
\CenterBoldFont{Błędy}


\begin{center}

  \begin{tabular}{|c|c|c|c|c|}
    \hline
    Strona & \multicolumn{2}{c|}{Wiersz} & Jest
                              & Powinno być \\ \cline{2-3}
    & Od góry & Od dołu & & \\
    \hline
    19  &  6 & & w~legła gruzach & legła w~gruzach \\
    19  & &  5 & \textit{TheChronicle} & \textit{The~Chronicle} \\
    22  & & 16 & na~bizantyńską & bizantyńską \\
    32  & &  2 & \textit{wiary, islam} & \textit{wiary. Islam} \\
    68  & &  1 & Dzieje Bizancjum & \textit{Dzieje Bizancjum} \\
    121 & & 13 & przed & przed nim \\
    126 & &  2 & The~Armenian & \textit{The~Armenian} \\
    143 & &  4 & wroga$^{ \textrm{\textit{20}} }$. & wroga$^{ 20 }$. \\
    178 &  5 & & tylko & ma~tylko \\
    202 & & 15 & \textit{wiary, islam} & \textit{wiary. Islam} \\
    203 &  3 & & \textit{history} & \textit{History} \\
    \hline
  \end{tabular}

\end{center}

\vspace{\spaceTwo}


% ############################










% ######################################
\newpage

\section{Historia świętej wiary, XVIII i~XIX wiek}

\vspace{\spaceTwo}
% ######################################



% ############################
\Work{ % Autor i tytuł dzieła
  Richard Butterwick \\
  \textit{Polska Rewolucja a~Kościół Katolicki 1788--1792},
  \cite{ButterwickPolskaRewolucjaAKosciolKatolicki2012}}

\vspace{0em}


% ##################
\CenterBoldFont{Uwagi do~konkretnych stron}

\vspace{0em}


\noindent
\Str{28} Euzebiusz \\



% ##################
\CenterBoldFont{Błędy}

\begin{center}

  \begin{tabular}{|c|c|c|c|c|}
    \hline
    Strona & \multicolumn{2}{c|}{Wiersz} & Jest
                              & Powinno być \\ \cline{2-3}
    & Od góry & Od dołu & & \\
    \hline
     % &   & &  &  \\
    \hline
  \end{tabular}

\end{center}

\vspace{\spaceTwo}


% ############################










% ######################################
\newpage

\section{Historia świętej wiary, XX i~XXI wiek}

\vspace{\spaceTwo}
% ######################################



% ############################
\Work{ % Autor i tytuł książki
  Red. E. Guerriero, M. Impagliazzo \\
  \textit{Najnowsza historia Kościoła. Katolicy i~kościoły
    chrześcijańskie} \\
  \textit{w~czasie pontyfikatu Jana Pawła II (1978--2005)},
  \cite{GuerrieroImpagliazzoNajnowszaHistoriaKosciola2006} }

\vspace{0em}


% ##################
\CenterBoldFont{Błędy}


\begin{center}

  \begin{tabular}{|c|c|c|c|c|}
    \hline
    Strona & \multicolumn{2}{c|}{Wiersz} & Jest
                              & Powinno być \\ \cline{2-3}
    & Od góry & Od dołu & & \\
    \hline
    6  & 10 & & religia miały & nauka miały \\
    6  & & 10 & do & od \\
    7  & & 11 & dużo & duże \\
    14 & &  3 & zgodne & zgadzające~się \\
    30 & &  4 & Afryki & Ameryki Południowej \\
    51 & & 17 & śś. & św. \\
    63 &  9 & & 1987 & 1986 \\
    \hline
  \end{tabular}

\end{center}

\vspace{\spaceTwo}


% ############################










% % ######################################
% \section{Dzieła świętych}

% \vspace{\spaceTwo}
% % ######################################








% % ######################################
% \newpage
% \section{Pozostali autorzy}

% \vspace{\spaceTwo}
% % ######################################










% #####################################################################
% #####################################################################
% Bibliografia

\bibliographystyle{plalpha}

\bibliography{DEUSBooks}{}





% ############################

% Koniec dokumentu
\end{document}
