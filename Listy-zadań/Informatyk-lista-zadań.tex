\documentclass[a4paper,11pt]{article}
\usepackage[polish]{babel}% Tłumaczy na polski teksty automatyczne LaTeXa i pomaga z typografią.
\usepackage[plmath,OT4,MeX]{polski}% Polska notacja we wzorach matematycznych. Ładne polskie
\usepackage[T1]{fontenc}% Pozwala pisać znaki diakrytyczne z języków innych niż polski.
\usepackage[utf8]{inputenc}% Pozwala pisać polskie znaki bezpośrednio.
\usepackage{indentfirst}% Sprawia, że jest wcięcie w pierwszym akapicie.
\frenchspacing% Wyłącza duże odstępy na końcu zdania. Pakiet babel polski robi to samo, ale to jest 
%zabezpieczenie jakibym chciał przestać go używać.
\usepackage{fullpage}% Mniejszse marginesy.
\usepackage{amsfonts}% Czcionki matematyczne od American Mathematic Society.
\usepackage{amsmath}% Dalsze wsparcie od AMS. Więc tego, co najlepsze w LaTeX, czyli trybu
%matematycznego.
\usepackage{amscd}% Jeszcze wsparcie od AMS.
\usepackage{latexsym}% Więcej symboli.
\usepackage{textcomp}% Pakiet z dziwnymi symbolami.
\usepackage{xy}% Pozwala rysować grafy.
\usepackage{tensor}% Pozwala prosto używać notacji tensorowej. Albo nawet pięknej notacji
%tensorowej:).
\usepackage{graphicx}% Pozwala wstawiać grafikę.
%\usepackage{url}% Pozwala pisać ładnie znak ~.
\newcommand{\cm}{\checkmark}
\newcommand{\bl}{(błąd?)}


\begin{document}



\begin{center}
INFORMATYKA. Zadania.
\end{center}



\begin{center}
B. W. Kernighan, D. M. Ritchie\\
,,Język ANSI C'',\\ \cite{BKDRJAC}.
\end{center}
\begin{itemize}
\item[--] \romannumeral4. 1\cm, 2\cm, 3\cm, 4\cm, 5\cm,
%\item[--]  (2.2,38). 1, 2, 3, 4, 5, 6, 7, 8, 9, 10, 11.
%\item[--]  \romannumeral4. 1, 2, 3, 4, 5, 6, 7.
%\item[--]  \romannumeral5. 1, 2, 3, 4, 5, 6, 7, 8.
%\item[--]  \romannumeral6. 1, 2, 3, 4, 5, 6, 7.
%\item[--]  \romannumeral7. 1, 2, 3, 4, 5, 6, 7, 8, 9, 10, 11. 
%\item[--]  \romannumeral8. 1, 2, 3, 4, 5, 6, 7, 8, 9, 10.
%\item[--]  \romannumeral9. 1, 2.
%\item[--] 
%\item[--] 
%\item[--] 
%\item[--] 
%\item[--] 
%\item[--] 
%\item[--] 
%\item[--] 
%\item[--] 
%\item[--] 
\end{itemize}



\bibliographystyle{ieeetr}
\bibliography{BibliographySBPL}{}



\end{document}