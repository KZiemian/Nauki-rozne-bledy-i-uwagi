% ------------------------------------------------------------------------------------------------------------------
% Basic configuration and packages
% ------------------------------------------------------------------------------------------------------------------
% Package for discovering wrong and outdated usage of LaTeX.
% More information to be found in l2tabu English version.
\RequirePackage[l2tabu, orthodox]{nag}
% Class of LaTeX document: {size of paper, size of font}[document class]
\documentclass[a4paper,11pt]{article}



% ------------------------------------------------------
% Packages not tied to particular normal language
% ------------------------------------------------------
% This package should improved spaces in the text
\usepackage{microtype}
% Add few important symbols, like text Celcius degree
\usepackage{textcomp}



% ------------------------------------------------------
% Polonization of LaTeX document
% ------------------------------------------------------
% Basic polonization of the text
\usepackage[MeX]{polski}
% Switching on UTF-8 encoding
\usepackage[utf8]{inputenc}
% Adding font Latin Modern
\usepackage{lmodern}
% Package is need for fonts Latin Modern
\usepackage[T1]{fontenc}



% ------------------------------------------------------
% Setting margins
% ------------------------------------------------------
\usepackage[a4paper, total={14cm, 25cm}]{geometry}



% ------------------------------------------------------
% Setting vertical spaces in the text
% ------------------------------------------------------
% Setting space between lines
\renewcommand{\baselinestretch}{1.1}

% Setting space between lines in tables
\renewcommand{\arraystretch}{1.4}



% ------------------------------------------------------
% Packages for scientific papers
% ------------------------------------------------------
% Switching off \lll symbol, that I guess is representing letter "Ł"
% It collide with `amsmath' package's command with the same name
\let\lll\undefined
% Basic package from American Mathematical Society (AMS)
\usepackage[intlimits]{amsmath}
% Equations are numbered separately in every section
\numberwithin{equation}{section}

% Other very useful packages from AMS
\usepackage{amsfonts}
\usepackage{amssymb}
\usepackage{amscd}
\usepackage{amsthm}

% Package with better looking calligraphy fonts
\usepackage{calrsfs}

% Package with better looking greek letters
% Example of use: pi -> \uppi
\usepackage{upgreek}
% Improving look of lambda letter
\let\oldlambda\Lambda
\renewcommand{\lambda}{\uplambda}




% ------------------------------------------------------
% BibLaTeX
% ------------------------------------------------------
% Package biblatex, with biber as its backend, allow us to handle
% bibliography entries that use Unicode symbols outside ASCII
\usepackage[
language=polish,
backend=biber,
style=alphabetic,
url=false,
eprint=true,
]{biblatex}

\addbibresource{LogikaITeoriaMnogosciBibliography.bib}





% ------------------------------------------------------
% Defining new environments (?)
% ------------------------------------------------------
% Defining enviroment "Wniosek"
\newtheorem{corollary}{Wniosek}
\newtheorem{definition}{Definicja}
\newtheorem{theorem}{Twierdzenie}





% ------------------------------------------------------
% Private packages
% You need to put them in the same directory as .tex file
% ------------------------------------------------------
% Contains various command useful for working with a text
\usepackage{latexgeneralcommands}
% Contains definitions useful for working with mathematical text
\usepackage{mathcommands}





% ------------------------------------------------------
% Package "hyperref"
% They advised to put it on the end of preambule
% ------------------------------------------------------
% It allows you to use hyperlinks in the text
\usepackage{hyperref}










% ------------------------------------------------------------------------------------------------------------------
% Title and author of the text
\title{DEUS: dzieła świętych i~błogosławionych \\
  {\Large Błędy i~uwagi}}

\author{Kamil Ziemian}


% \date{}
% ------------------------------------------------------------------------------------------------------------------










% ####################################################################
% Początek dokumentu
\begin{document}
% ####################################################################





% ######################################
\maketitle
% ######################################





% ######################################
\section{Sanctae Jan Chryzostom
  \textit{Homilie na~Księgę Rodzaju (seria pierwsza: Rdz 1--3)},
  \parencite{SancteJanChryzostomHomKsiegaRodzaju2008}}

% ######################################










% ######################################
\section{Dante \textit{Komedia}, \cite{DAK}}

% ######################################


% ##################
\CenterBoldFont{Błędy}

% \vspace{\spaceFive}


\begin{center}

  \begin{tabular}{|c|c|c|c|c|}
    \hline
    Strona & \multicolumn{2}{c|}{Wiersz} & Jest
                              & Powinno być \\ \cline{2-3}
    & Od góry & Od dołu & & \\
    \hline
    % 347 & 21 & & ,,Jestem & Jestem \\
    350 & & 9 & jed- nak & jednak \\
    % & & & & \\
    % & & & & \\
    % & & & & \\
    \hline
  \end{tabular}

\end{center}

\VerSpaceTwo


Piekło, Pieśń IV, 131: \ldots którzy wiedzą, Czyściec, Pieśń 1, 44: by was
z tej nocy\ldots


% ######################################










% ######################################
\section{Elio Guerriero \textit{Hans Urs von Balthasar},
  \cite{GuerrieroHansUrsVonBalthasarMonografia2004}}


% ######################################


% ##################
\CenterBoldFont{Uwagi do~konkretnych stron}

\vspace{0em}


\noindent
\Str{41} Pierwszy akapit na tej kończy się cudzysłowem zamykającym, lecz
nie ma w nim nigdzie cudzysłowu otwierającego.

\VerSpaceFour





% ##################
\CenterBoldFont{Błędy}


\begin{center}

  \begin{tabular}{|c|c|c|c|c|}
    \hline
    Strona & \multicolumn{2}{c|}{Wiersz} & Jest
                              & Powinno być \\ \cline{2-3}
    & Od góry & Od dołu & & \\
    \hline
    \hphantom{0}17 & 12 & & \textit{Karl. Barth.} & \textit{Karl Barth.} \\
    139 & & \hphantom{0}6 & Tage & \textit{Tage} \\
    153 & & \hphantom{0}2 & ($^{ 2 }$1988) & (1988) \\
    153 & & \hphantom{0}4 & (1972);(pol. & (1972); (pol. \\
    217 & & 14 & eologa & teologa \\
    % & & & & \\
    % & & & & \\
    % & & & & \\
    \hline
  \end{tabular}

\end{center}
% ############################











% % ######################################
% \newpage

% \section{Święta liturgia}

% \vspace{\spaceTwo}
% % ######################################



% % ############################
% \subsection{Święta liturgia po 1962~r.}

% \vspace{\spaceThree}
% % ############################



% ######################################
\section{Michael Davies \textit{Zniszczenie Mszy Świętej
    czyli~Godly Order Cranmera},
  \parencite{DaviesZniszczenieMszySwietej2016} }

% ######################################



% ##################
\CenterBoldFont{Uwagi}

\vspace{0em}


\noindent
Jak wiele książek z~tego wydawnictwa, ta~również ta nie przeszła
odpowiedniego procesu redakcji i~jest pod~wieloma względami źle zedytowana.

\VerSpaceFour





\noindent
W~procesie tworzenia wydania polskiego przepadła gdzieś bibliografia
i~indeks skrótów.





% ##################
\newpage

\CenterBoldFont{Błędy}


\begin{center}

  \begin{tabular}{|c|c|c|c|c|}
    \hline
    Strona & \multicolumn{2}{c|}{Wiersz} & Jest
                              & Powinno być \\ \cline{2-3}
    & Od góry & Od dołu & & \\
    \hline
    15  & \hphantom{0}2 & & z~zbiorach & w~zbiorach \\
    24  & & \hphantom{0}6 & stad & stąd \\
    24  & \hphantom{0}2 & & woli” & woli \\
    33  & \hphantom{0}9 & & ja & ją \\
    37  & \hphantom{0}5 & & zapomniało waszym & zapomniał o~waszym \\
    % & & & & \\
    % & & & & \\
    % & & & & \\
    \hline
  \end{tabular}

\end{center}

\VerSpaceTwo


\noindent
\textbf{Tylna okładka.} \\
\Jest  \textit{arcydziełem}”. \\
\PowinnoByc \textit{arcydziełem} \\


% ######################################










% ######################################
\section{Michael Davies
  \textit{Sobór Papieża Jana. Rewolucja liturgiczna},
  \parencite{}}

% ######################################


% ##################
\CenterBoldFont{Błędy}

\VerSpaceFive


\begin{center}

  \begin{tabular}{|c|c|c|c|c|}
    \hline
    Strona & \multicolumn{2}{c|}{Wiersz} & Jest
                              & Powinno być \\ \cline{2-3}
    & Od góry & Od dołu & & \\
    \hline
    \hphantom{0}11 & & \hphantom{0}2 & \textit{musimocna}
    & \textit{musi mocno} \\
    \hphantom{0}41 & 10 & & „Patrząc & Patrząc \\
    \hphantom{0}46 & & \hphantom{0}2 & Westminster & Westminster. \\
    % 81 & & 17 & ,,wyczuwalny & ,,Wyczuwalny \\
    \hphantom{0}97 & & \hphantom{0}3 & wprowadzen- ia & wprowadzenia \\
    176 & & \hphantom{0}9 & presji.„Czy & presji. „Czy \\
    178 & \hphantom{0}7 & & \textit{Novost}i & \textit{Novosti} \\
    183 & \hphantom{0}5 & & protestanckich & prawosławnych \\
    188 & 14 & & wary & wiary \\
    203 & 14 & & podporządkować. & podporządkować.” \\
    204 & & 15 & małżeństwa. & małżeństwa.” \\
    209 & \hphantom{0}3 & & zarazić. & zarazić.” \\
    209 & \hphantom{0}6 & & pracę.” & pracę. \\
    243 & & 17 & pominiecie & pominięcie \\
    245 & \hphantom{0}7 & & dokumentu. & dokumentu.” \\
    250 & 17 & & nie byłaby & byłaby \\
    263 & \hphantom{0}8 & & „<<Wzorcowa>>” & ”<<Wzorcowa>> \\
    265 & 13 & & tam, że & tam \\
    265 & 15 & & ewolucji, & ewolucji. \\
    310 & \hphantom{0}5 & & obór & Sobór \\
    323 & \hphantom{0}3 & & wyobrazili & wyobrażali \\
    323 & 11 & & nieuzasadnionym & nie uzasadnionym \\
    325 & & \hphantom{0}8 & Nie & „Nie \\
    326 & \hphantom{0}7 & & władzę. & władzę.” \\
    327 & \hphantom{0}3 & & 272 & 322 \\
    330 & 12 & & Podam -- nawet & Podam nawet \\
    330 & 13 & & to & jest \\
    359 & \hphantom{0}3 & & 59\% & 41\% \\
    \hline
  \end{tabular}

\end{center}

\VerSpaceTwo


\noindent
\StrWierszGora{44}{16} \\
\Jest  uprawnionych do głosowania \\
\PowinnoByc które można głosować \\
\StrWierszGora{174}{4} \\
\Jest  umożliwiaporozumieniezchrześcijaństwem.Strukturalnacałość\textit{DasKapital}
\\
\PowinnoByc umożliwia porozumienie z chrześcijaństwem. Strukturalna całość
\textit{Das Kapital} \\


% ######################################










% ######################################
\section{M. Davies \textit{Sobór Watykański II a~wolność
    religijna},
  \parencite{DaviesSoborAWolnoscReligina2002}}


% ##################
\CenterBoldFont{Błędy}

% \vspace{\spaceFive}


\begin{center}

  \begin{tabular}{|c|c|c|c|c|}
    \hline
    & \multicolumn{2}{c|}{} & & \\
    Strona & \multicolumn{2}{c|}{Wiersz} & Jest
                              & Powinno być \\ \cline{2-3}
    & Od góry & Od dołu & & \\
    \hline
    \hphantom{0}84 & \hphantom{0}2 & & społeczeństwie” & „społeczeństwie” \\
    109 & & 15 & nacjonalizmu & racjonalizmu \\
    118 & & 10 & był & został \\
    155 & \hphantom{0}9 & & nie & się \\
    193 & 17 & & nie są & są \\
    258 & & \hphantom{0}5 & \textit{rolą} & \textit{z rolą} \\
    262 & 17 & & \textit{upadku} & \textit{upadku.} \\
    269 & 12 & & \textit{Quanta cura} & \textit{Quas primas} \\
    269 & 17 & & \textit{Quanta cura} & \textit{Quas primas} \\
    333 & & 13 & jest & co jest \\
    % & & & & \\
    \hline
  \end{tabular}

\end{center}
% ######################################










% ######################################
\section{Tracey Rowland \textit{Wiara Ratzingera, teologia
    Benedykta XVI}, \parencite{RowlandWiaraRatzingera2010}}

\vspace{0em}


% ##################
\CenterBoldFont{Uwagi do~konkretnych stron}

\vspace{0em}


\noindent
\Str{50} Stwierdzenie, że~św. Jan Paweł II doszedł do~teodramatyki poprzez
tomizm i~fenomenologię, jest mocno wątpliwe. O~ile pamiętam w~pierwszym
artykule z~\cite{PoslugaMysleniaVolIX2011}, jest podane, że dopiero na
studiach seminaryjnych poprzez standardowy podręcznik metafizyki~(?), Karol
Wojtyła zetknął~się z~tomizmem. (Możliwe, że~w~tej pracy podane jest też,
kiedy pierwszy raz spotkał~się z~fenomenologią.) Natomiast co najmniej od
lat szkolnych był zafascynowany teatrem, str.~14--15
\cite{NowakJanPawelIIKronikaZyciaIPontyfikatu2015}. Zanim w~październiku
1942 r. podjął decyzję o zostaniu kapłanem był m.in.~od~kilku lat
zaangażowany w~działalność Teatru Rapsodycznego, z~którego twórcą
i~kierownikiem Mieczysławem Kotlarczykiem prowadził dyskusje, między innymi
na temat związku religii, filozofii, patriotyzmu i~sztuki, patrz
np.~str.~19--34 w~\cite{NowakJanPawelIIKronikaZyciaIPontyfikatu2015}.
Wydaje~się dużo bardziej prawdopodobne, że~właśnie z~jego doświadczeń
teatralnych wyrosła jego pasja do~teodramatyki.

\VerSpaceFour





\noindent
\Str{52} Choć nie znam twórczości Ryszarda Legutko, wciąż
wydaje mi~się bardzo wątpliwym stwierdzeniem określenie go mianem
teologa. Osobną sprawą jest jego związek z~tradycją tomistyczną.





% ##################
\CenterBoldFont{Błędy}

% \vspace{\spaceFive}


\begin{center}

  \begin{tabular}{|c|c|c|c|c|}
    \hline
    Strona & \multicolumn{2}{c|}{Wiersz} & Jest
                              & Powinno być \\ \cline{2-3}
    & Od góry & Od dołu & & \\
    \hline
    21 & 16 & & 1844--1970 & 1844--1900 \\
    27 & & \hphantom{0}9 & transcendentalne piękno
    & transcendentalnym pięknem \\
    46 & & 13 & 1947 & 1847 \\
    71 & & 11 & „przyjścia & >>przyjścia \\
    71 & & \hphantom{0}7 & domu” & domu<< \\
    78 & 12 & & ona z~tego & z~niej \\
    \hline
  \end{tabular}

\end{center}
% ######################################









% ######################################
\section{R.M. Wiltgen \textit{Ren wpada do~Tybru},
  \parencite{WiltgenRenWpadaDoTybru2010}}


% ##################
\CenterBoldFont{Błędy}


\begin{center}

  \begin{tabular}{|c|c|c|c|c|}
    \hline
    Strona & \multicolumn{2}{c|}{Wiersz} & Jest
                              & Powinno być \\ \cline{2-3}
    & Od góry & Od dołu & & \\
    \hline
    18  & \hphantom{0}5 & & dwunastu do czterech & dwunastu \\
    40  & & 16 & te & że \\
    42  & \hphantom{0}3 & & Kantonu(Dahomej) & Kantonu (Dahomej) \\
    44  & & \hphantom{0}6 & konferencji,tego & konferencji, tego \\
    % & & & & \\
    % & & & & \\
    % & & & & \\
    % & & & & \\
    \hline
  \end{tabular}

\end{center}
% ######################################










% ####################################################################
% ####################################################################
% Bibliography

\printbibliography





% ############################

% Koniec dokumentu
\end{document}
