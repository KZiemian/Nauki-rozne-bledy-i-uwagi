% ---------------------------------------------------------------------
% Basic configuration and packages
% ---------------------------------------------------------------------
% Package for discovering wrong and outdated usage of LaTeX.
% More information to be found in l2tabu English version.
\RequirePackage[l2tabu, orthodox]{nag}
% Class of LaTeX document: {size of paper, size of font}[document class]
\documentclass[a4paper,11pt]{article}



% ---------------------------------------
% Packages not tied to particular normal language
% ---------------------------------------
% This package should improved spaces in the text.
\usepackage{microtype}



% ---------------------------------------
% Polonization of LaTeX document
% ---------------------------------------
% Basic polonization of the text
\usepackage[MeX]{polski}
% Switching on UTF-8 encoding
\usepackage[utf8]{inputenc}
% Adding font Latin Modern
\usepackage{lmodern}
% Package is need for fonts Latin Modern
\usepackage[T1]{fontenc}



% ---------------------------------------
% Setting margins
% ---------------------------------------
% Package for easy settings of margins. Unit of measurement is inch.
\usepackage{vmargin}
\setmarginsrb
{ 0.7in}  % left margin
{ 0.6in}  % top margin
{ 0.7in}  % right margin
{ 0.8in}  % bottom margin
{  20pt}  % head height
{0.25in}  % head sep
{   9pt}  % foot height
{ 0.3in}  % foot sep



% ---------------------------------------
% Setting vertical spaces in the text
% ---------------------------------------
% Setting space between lines
\renewcommand{\baselinestretch}{1.1}

% Setting space between lines in tables
\renewcommand{\arraystretch}{1.4}





% ------------------------------
% Private packages
% You need to put them in the same directory as .tex file
% ------------------------------
% Contains various command useful for working with a text
\usepackage{latexgeneralcommands}





% ------------------------------
% Package ``hyperref''
% They advised to put it on the end of preambule
% ------------------------------
% It allows you to use hyperlinks in the text
\usepackage{hyperref}










% ---------------------------------------------------------------------
% Defining title and author of the text
\title{Historia Chrześcijaństwa Carrollów \\
  {\Large Błędy i~uwagi}}

\author{Kamil Ziemian}


% \date{}
% ---------------------------------------------------------------------










% ####################################################################
% Title of the text
\begin{document}
% ####################################################################





% ######################################
\maketitle  % Tytuł całego tekstu
% ######################################





% ######################################
\section{Warren H.~Carroll \textit{Historia Chrześcijaństwa.
    Tom~I: Narodziny Chrześcijaństwa},
  \cite{CarrollHistoriaChrzecijanstwaVolI2009}}
% ######################################



% ############################
\subsection{Uwagi do~konkretnych stron}
% ############################


\noindent
\StrWierszD{17}{4} Gwiazdka w~tej linii jest za~mała.

\VerSpaceFour





\noindent
\StrWierszD{41}{13} W~tej linii użyto złego rodzaju gwiazdki i~umieszczono
ją trochę krzywo.

\VerSpaceFour





\noindent
\StrWierszD{61}{20} Gwiazdka w~tej linii jest za~mała.

\VerSpaceFour





\noindent
\StrWierszD{65}{6} W~tej linii użyto złego rodzaju gwiazdki.

\VerSpaceFour





\noindent
\StrWierszD{73}{12} Gwiazdka w~tej linii jest trochę za~mała i~odrobinę
krzywo umieszczona.

\VerSpaceFour





\noindent
\StrWierszD{75}{4} Gwiazdka w~tej linii jest za mała.

\VerSpaceFour





\noindent
\StrWierszD{112}{2} Gwiazdka w~tej linii jest za mała.

\VerSpaceFour





\noindent
\StrWierszD{182}{2} Gwiazdka w~tej linii jest za mała.

\VerSpaceFour





\noindent
\textbf{Str.~199, wiersze 9, 8 (od~dołu).} Gwiazdki w~tych liniach~są zbyt
małe.

\VerSpaceFour





\noindent
\textbf{Str.~200, wiersze 2, 1 (od~dołu).} Gwiazdki w~tych liniach~są zbyt
małe.

\VerSpaceFour





\noindent
\StrWierszD{249}{1} Gwiazdka w~tej linii jest za mała.

\VerSpaceFour





\noindent
\StrWierszD{265}{2} Gwiazdka w~tej linii jest za mała.

\VerSpaceFour





\noindent
\StrWierszD{271}{1} Gwiazdka w~tej linii jest za mała.

\VerSpaceFour





\noindent
\StrWierszD{282}{14} Gwiazdka w~tej linii jest za mała.

\VerSpaceFour





\noindent
\StrWierszD{286}{4} Gwiazdka w~tej linii jest za mała.

\VerSpaceFour





\noindent
\textbf{Str.~290, wiersze 4, 3 (od~dołu).} Gwiazdki w~tych liniach~są zbyt
małe.

\VerSpaceFour





\noindent
\StrWierszD{366}{1} Gwiazdka w~tej linii jest za mała.

\VerSpaceFour





\noindent
\StrWierszG{378}{20} Gwiazdka w~tej linii jest za mała.

\VerSpaceFour





\noindent
\StrWierszD{495}{1} Gwiazdka w~tej linii jest za mała.

\VerSpaceFour





\noindent
\StrWierszD{538}{38} Gwiazdka w~tej linii jest za mała.

\VerSpaceFour





\noindent
\StrWierszD{562}{20} W~tej linii nie podano autorów publikacji. Nie
wiem czy to~błąd, czy~dla tej publikacji nie~trzeba podawać autorów.

\VerSpaceFour





\noindent
\StrWierszG{575}{4~od~góry i~1 od~dołu} Słowo „męczennik” jest w~bardzo
brzydki sposób podzielone między te dwie linie.

% 185???





% ##################
\newpage

\CenterBoldFont{Błędy}


\begin{center}

  \begin{tabular}{|c|c|c|c|c|}
    \hline
    Strona & \multicolumn{2}{c|}{Wiersz} & Jest
                              & Powinno być \\ \cline{2-3}
    & Od góry & Od dołu & & \\
    \hline
    \hphantom{0}19 & & 14 & \textit{Fossil~~Evidence}
    & \textit{Fossil Evidence} \\
    \hphantom{0}19 & & \hphantom{0}5 & \textit{Evolution -- the}
    & \textit{Evolution: The} \\
    \hphantom{0}20 & & 21 & le & Le \\
    \hphantom{0}29 & & \hphantom{0}2 & \textit{Huyuk, a}
    & \textit{Huyuk: A} \\
    \hphantom{0}30 & & \hphantom{0}3 & \textit{Sumerians, Their}
    & \textit{Sumerians: Their} \\
    \hphantom{0}33 & & \hphantom{0}3 & \textit{Babylon, a}
    & \textit{Babylon: A} \\
    \hphantom{0}35 & & \hphantom{0}5 & \textit{India, a}
    & \textit{India: A} \\
    \hphantom{0}42 & & \hphantom{0}2 & \textit{Elba, a}
    & \textit{Elba: A} \\
    \hphantom{0}44 & & \hphantom{0}1 & \textit{Hyksos, a}
    & \textit{Hyksos: A} \\
    \hphantom{0}45 & & 18 & \textit{Desert, a} & \textit{Desert: A} \\
    \hphantom{0}49 & & 20 & \textit{through} & \textit{Through} \\
    % A Path Through Genesis
    \hphantom{0}49 & & \hphantom{0}2 & \textit{through}
    & \textit{Through} \\
    \hphantom{0}53 & & 15 & \textit{Path through}
    & \textit{A~Path Through} \\
    \hphantom{0}56 & & \hphantom{0}5 & \textit{Akhenaten}„
           & \textit{Akhenaten}, \\
    \hphantom{0}61 & & \hphantom{0}2 & \textit{Israel,}
    & \textit{Israel:} \\
    \hphantom{0}63 & & 12 & \textit{Rames II, a} & \textit{Rames II: A} \\
    \hphantom{0}73 & & \hphantom{0}6 & \textit{into} & \textit{Into} \\
    \hphantom{0}73 & & \hphantom{0}6 & \textit{Past, the}
    & \textit{Past: The} \\
    \hphantom{0}81 & & \hphantom{0}3 & \textit{Israel, Its}
    & \textit{Israel: Its} \\
    \hphantom{0}83 & & 20 & \textit{Hazor, the} & \textit{Hazor: The} \\
    \hphantom{0}84 & & \hphantom{0}2 & \textit{Bible,} & \textit{Bible:} \\
    \hphantom{0}84 & & \hphantom{0}1 & \textit{a~Historical}
    & \textit{A~Historical} \\
    \hphantom{0}91 & & \hphantom{0}5 & \textit{Worlds}. 94-100
    & \textit{Worlds}, s.~94-100 \\
    \hphantom{0}92 & & \hphantom{0}6 & \textit{Testament},
    & \textit{Testament}, [w:] \\
    102 & & 16 & \textit{In} & \textit{in} \\
    102 & & \hphantom{0}4 & \textit{Israel, its} & \textit{Israel: Its} \\
    107 & & \hphantom{0}3 & \textit{Mesopotamia, Portrait}
           & \textit{Mesopotamia: Portrait} \\
    126 & 18 & & \textit{through} & \textit{Through} \\
    131 & & \hphantom{0}9 & Babylon & \textit{Babylon} \\
    \hline
  \end{tabular}





  \newpage

  \begin{tabular}{|c|c|c|c|c|}
    \hline
    Strona & \multicolumn{2}{c|}{Wiersz} & Jest
                              & Powinno być \\ \cline{2-3}
    & Od góry & Od dołu & & \\
    \hline
    132 & & \hphantom{0}6 & \textit{Israe}l & \textit{Israel} \\
    145 & & 12 & \textit{Greeks; the} & \textit{Greeks: The} \\
    145 & & \hphantom{0}3 & Babylon & \textit{Babylon} \\
    160 & & 21 & \textit{Personality, Its} & \textit{Personality: Its} \\
    166 & & \hphantom{0}5 & wspanialej & wspaniałej \\
    171 & & \hphantom{0}2 & \textit{Wisdom, the} & \textit{Wisdom: The} \\
    172 & & \hphantom{0}4 & \textit{Greeks, the} & \textit{Greeks: The} \\
    178 & & \hphantom{0}4 & Agamemnon & \textit{Agamemnon} \\
    182 & & \hphantom{0}5 & \textit{Greeks, a} & \textit{Greeks: A} \\
    193 & & \hphantom{0}1 & \textit{Great, King} & \textit{Great: King} \\
    203 & & \hphantom{0}3 & \textit{B.C} & \textit{B.C.} \\
    218 & & \hphantom{0}8 & \textit{146B.C.} & \textit{146~B.C.} \\
    220 & & \hphantom{0}2 & \textit{Africanus, Soldier}
           & \textit{Africanus: Soldier} \\
    223 & & \hphantom{0}5 & \textit{Carthage, a} & \textit{Carthage: A} \\
    228 & & 14 & \textit{B.~C} & \textit{B.C.} \\
    230 & & 15 & \textit{Empire, Rome's} & \textit{Empire: Rome's} \\
    230 & & \hphantom{0}4 & \textit{B.~C.} & \textit{B.C.} \\
    230 & & \hphantom{0}1 & \textit{146B.C.} & \textit{146~B.C.} \\
    233 & & \hphantom{0}9 & \textit{146B.C.} & \textit{146~B.C.} \\
    233 & & \hphantom{0}7 & \textit{146B.C.} & \textit{146~B.C.} \\
    234 & & \hphantom{0}5 & \textit{Syria, from} & \textit{Syria: From} \\
    238 & & \hphantom{0}7 & \textit{Ezra} & \textit{From Ezra} \\
    238 & & \hphantom{0}3 & \textit{Ezra} & \textit{From Ezra} \\
    241 & & \hphantom{0}5 & \textit{Ezra} & \textit{From Ezra} \\
    241 & & \hphantom{0}4 & \textit{Ezra} & \textit{From Ezra} \\
    241 & & \hphantom{0}1 & \textit{Ezra} & \textit{From Ezra} \\
    242 & & \hphantom{0}2 & \textit{Ezra} & \textit{From Ezra} \\
    242 & & \hphantom{0}2 & \textit{Ezra} & \textit{From Ezra} \\
    259 & & \hphantom{0}3 & \textit{Pompey, the} & \textit{Pompey: The} \\
    261 & & \hphantom{0}3 & \textit{Caesar, Politician}
           & \textit{Caesar: Politician} \\
    \hline
  \end{tabular}





  \newpage

  \begin{tabular}{|c|c|c|c|c|}
    \hline
    Strona & \multicolumn{2}{c|}{Wiersz} & Jest
    & Powinno być \\ \cline{2-3}
    & Od góry & Od dołu & & \\
    \hline
    262 & & \hphantom{0}6 & \textit{Pompey, the} & \textit{Pompey: The} \\
    266 & & \hphantom{0}6 & \textit{Pompey, the} & \textit{Pompey: The} \\
    267 & & \hphantom{0}2 & \textit{Rule, from} & \textit{Rule: From} \\
    268 & & \hphantom{0}3 & \textit{under} & \textit{Under} \\
    268 & & \hphantom{0}1 & \textit{under} & \textit{Under} \\
    275 & & \hphantom{0}7 & \textit{under} & \textit{Under} \\
    281 & & \hphantom{0}3 & \textit{the} & \textit{The} \\
    281 & & \hphantom{0}3 & \textit{Joseph, Their}
    & \textit{Joseph: Their} \\
    293 & & \hphantom{0}3 & \textit{Christ, His} & \textit{Christ: His} \\
    294 & & \hphantom{0}8 & \textit{St.~Matthew, a}
    & \textit{St.~Matthew: A} \\
    295 & & \hphantom{0}4 & \textit{Birth, an} & \textit{Birth: An} \\
    300 & & \hphantom{0}9 & \textit{Josephus, the}
    & \textit{Josephus: The} \\
    307 & & \hphantom{0}8 & \textit{Birth, an} & \textit{Birth: An} \\
    315 & & 20 & \textit{Bethlehem, an} & \textit{Bethlehem: An} \\
    315 & & 10 & \textit{niemal} & niemal \\
    316 & & \hphantom{0}8 & \textit{Dead, Studies}
    & \textit{Dead: Studies} \\
    320 & & 10 & Jesus Christ, \textit{His} & \textit{Jesus Christ: His} \\
    326 & & 12 & \textit{Antipas, a} & \textit{Antipas: A} \\
    365 & & \hphantom{0}9 & 1931 ) & 1931) \\
    374 & & 11 & 549;Belser & 549; Belser \\
    387 & & 10 & \textit{Doctor} & \textit{A~Doctor} \\
    387 & & \hphantom{0}9 & \textit{Doctor} & \textit{A~Doctor} \\
    387 & & \hphantom{0}5 & \textit{Doctor} & \textit{A~Doctor} \\
    388 & & 17 & \textit{Doctor} & \textit{A~Doctor} \\
    388 & & 13 & \textit{Doctor} & \textit{A~Doctor} \\
    388 & & \hphantom{0}1 & \textit{Doctor} & \textit{A~Doctor} \\
    390 & & \hphantom{0}7 & \textit{Doctor} & \textit{A~Doctor} \\
    391 & & 11 & \textit{Doctor} & \textit{A~Doctor} \\
    391 & & 10 & \textit{Doctor} & \textit{A~Doctor} \\
    392 & & \hphantom{0}3 & \textit{Doctor} & \textit{A~Doctor} \\
    \hline
  \end{tabular}





  \newpage

  \begin{tabular}{|c|c|c|c|c|}
    \hline
    Strona & \multicolumn{2}{c|}{Wiersz} & Jest
                              & Powinno być \\ \cline{2-3}
    & Od góry & Od dołu & & \\
    \hline
    393 & & \hphantom{0}8 & \textit{Doctor} & \textit{A~Doctor} \\
    402 & & \hphantom{0}9 & \textit{Antipas, a} & \textit{Antipas: A }\\
    407 & & \hphantom{0}6 & \textit{Greek, a} & \textit{Greek: A} \\
    408 & & \hphantom{0}2 & \textit{Claudius, the}
    & \textit{Claudius: The} \\
    410 & & \hphantom{0}3 & \textit{Luke, a} & \textit{Luke: A} \\
    419 & & 24 & \textit{Doctor} & \textit{A~Doctor} \\
    432 & & \hphantom{0}2 & \textit{Kerala, a} & \textit{Kerala: A} \\
    436 & & \hphantom{0}5 & \textit{Nero, Reality}
    & \textit{Nero: Reality} \\
    440 & & \hphantom{0}4 & \textit{Exile, a} & \textit{Exile: A} \\
    442 & & \hphantom{0}9 & \textit{Tertullian, a}
    & \textit{Tertullian: A} \\
    443 & & 17 & \textit{Jude, Introduction}
           & \textit{Jude: Introduction} \\
    444 & & \hphantom{0}9 & \textit{under} & \textit{Under} \\
    444 & & \hphantom{0}2 & \textit{under} & \textit{Under} \\
    449 & & \hphantom{0}7 & \textit{under} & \textit{Under} \\
    455 & & 25 & np., & np. \\
    458 & & \hphantom{0}4 & \textit{among} & \textit{Among} \\
    458 & & \hphantom{0}2 & \textit{Paul, Apostole}
    & \textit{Paul: Apostole} \\
    459 & & \hphantom{0}3 & \textit{Religion; the}
    & \textit{Religin: The} \\
    471 & & \hphantom{0}9 & \textit{Smyrna, a} & \textit{Smyrna: A} \\
    477 & & 17 & \textit{ofthe} & \textit{of the} \\
    478 & & \hphantom{0}7 & \textit{Severus, the} & \textit{Severus: The} \\
    490 & & 15 & \textit{Tertulian, a} & \textit{Tertulian: A} \\
    490 & & \hphantom{0}1 & cześć & część \\
    492 & & 13 & \textit{during} & \textit{During} \\
    492 & & \hphantom{0}2 & \textit{Mani, a} & \textit{Mani: A} \\
    508 & & \hphantom{0}4 & \textit{Edessa, the} & \textit{Edessa: The} \\
    533 & & \hphantom{0}4 & \textit{Constantine, a}
    & \textit{Constantine: A} \\
    561 & & 19 & \textit{Bible, a~Historical}
           & \textit{Bible: A~Historical} \\
    561 & & 16 & T. & T., \\
    561 & & 15 & \textit{Moses, the} & \textit{Moses: The} \\
    \hline
  \end{tabular}





  \newpage

  \begin{tabular}{|c|c|c|c|c|}
    \hline
    Strona & \multicolumn{2}{c|}{Wiersz} & Jest
                              & Powinno być \\ \cline{2-3}
    & Od góry & Od dołu & & \\
    \hline
    562 & \hphantom{0}4 & & \textit{Ezekiel: the} & \textit{Ezekiel: The} \\
    562 & 10 & & \textit{Abraham, Loved} & \textit{Abraham: Loved} \\
    562 & 11 & & \textit{Desert, a~History} & \textit{Desert: A~History} \\
    562 & 13 & & \textit{Abraham, Father} & \textit{Abraham: Father} \\
    562 & 15 & & \textit{Canaan: the~Ras} & \textit{Canaan: The~Ras} \\
    562 & & 14 & \textit{Judaea} & \textit{Judea} \\
    562 & & 13 & \textit{Israel, from} & \textit{Israel: From} \\
    562 & & 11 & \textit{Jerusalem; Excavating}
           & \textit{Jerusalem: Excavating} \\
    562 & & \hphantom{0}4 & \textit{Covenant, a~Study}
    & \textit{Covenant: A~Study} \\
    563 & \hphantom{0}4 & & \textit{Law; Studies} & \textit{Law: Studies} \\
    563 & \hphantom{0}6 & & \textit{1-39, Introduction}
           & \textit{1-39: Introduction} \\
    563 & & 18 & \textit{Joshua; Biblical} & \textit{Joshua: Biblical} \\
    563 & & 11 & \textit{Qumran, a} & \textit{Qumran:~A} \\
    563 & & \hphantom{0}8 & \textit{Israel, its} & \textit{Israel: Its} \\
    563 & & \hphantom{0}7 & T, & T. \\
    564 & \hphantom{0}1 & & \textit{Hazor, the} & \textit{Hazor: The} \\
    564 & \hphantom{0}8 & & \textit{Maccabees, with}
    & \textit{Maccabees: With} \\
    564 & 12 & & \textit{Akhenaten, Pharaoh}
           & \textit{Akhenaten, Pharaoh} \\
    564 & 13 & & \textit{before} & \textit{Before} \\
    564 & 16 & & \textit{Elba, a} & \textit{Elba: A} \\
    564 & 18 & & \textit{Greeks; the} & \textit{Greeks: The} \\
    564 & & 17 & \textit{Buddhism, its} & \textit{Buddhism: Its} \\
    564 & & \hphantom{0}7 & \textit{Darkness; a} & \textit{Darkness: A} \\
    564 & & \hphantom{0}3 & \textit{Sumerians; Their}
    & \textit{Sumerians: Their} \\
    565 & \hphantom{0}3 & & \textit{Huyuk, a} & \textit{Huyuk: A} \\
    565 & 13 & & \textit{Mesopotamia, Portrait}
           & \textit{Mesopotamia: Portrait} \\
    565 & 16 & & \textit{Babylon; a} & \textit{Babylon: A} \\
    565 & 18 & & \textit{Rameses~II, a} & \textit{Ramses~II: A} \\
    565 & 19 & & \textit{India; a} & \textit{India: A} \\
    565 & 17 & & \textit{Hyksos, a} & \textit{Hyksos: A} \\
    \hline
  \end{tabular}





  \newpage

  \begin{tabular}{|c|c|c|c|c|}
    \hline
    Strona & \multicolumn{2}{c|}{Wiersz} & Jest
                              & Powinno być \\ \cline{2-3}
    & Od góry & Od dołu & & \\
    \hline
    565 & 14 & & \textit{Egypt, an} & \textit{Egypt: An} \\
    565 & 13 & & \textit{Personality, Its} & \textit{Personality: Its} \\
    566 & \hphantom{0}3 & & \textit{B. C.} & \textit{B.C.} \\
    566 & \hphantom{0}7 & & \textit{Carthage, a} & \textit{Carthage: A} \\
    566 & \hphantom{0}9 & & \textit{Dead; Studies}
    & \textit{Dead: Studies} \\
    566 & 10 & & \textit{Empire; Rome's} & \textit{Empire: Rome's} \\
    566 & 11 & & \textit{Greeks, a} & \textit{Greeks: A} \\
    566 & 16 & & \textit{Caesar, Politician}
           & \textit{Caesar: Politician} \\
    566 & & 20 & \textit{Pompey, the} & \textit{Pompey: The} \\
    566 & & 19 & \textit{Pompey, the} & \textit{Pompey: The} \\
    566 & & 17 & \textit{Great, King} & \textit{Great: King} \\
    566 & & 13 & \textit{lonians} & \textit{Ionians} \\
    566 & & \hphantom{0}9 & A.H.M.,\textit{Sparta}
    & A.H.M., \textit{Sparta} \\
    566 & & \hphantom{0}5 & \textit{Past, the} & \textit{Past: The} \\
    566 & & \hphantom{0}2 & \textit{B. C.} & \textit{B.C.} \\
    566 & & \hphantom{0}1 & \textit{Wisdom; the} & \textit{Wisdom: The} \\
    567 & \hphantom{0}5 & & \textit{Library, Glory}
    & \textit{Library: Glory} \\
    567 & 19 & & \textit{under} & \textit{Under} \\
    567 & 19 & & \textit{Rule, from} & \textit{Rule: From} \\
    567 & 20 & & \textit{Cicero, a} & \textit{Cicero: A} \\
    567 & & \hphantom{0}4 & \textit{Passion, Death}
    & \textit{Passion: Death} \\
    568 & \hphantom{0}2 & & \textit{Matthew, a} & \textit{Matthew: A} \\
    568 & 15 & & \textit{Christ, a} & \textit{Christ: A} \\
    568 & 18 & & \textit{Antipas, a} & \textit{Antipas: A} \\
    568 & & 13 & \textit{Birth, an} & \textit{Birth: An} \\
    568 & & \hphantom{0}4 & \textit{among} & \textit{Among} \\
    568 & & \hphantom{0}4 & \textit{Peo-Pk} & \textit{People} \\
    569 & 12 & & \textit{Tertullian, a} & \textit{Tertullian: A} \\
    569 & & 21 & \textit{Smyrna, a} & \textit{Smyrna: A} \\
    \hline
  \end{tabular}





  \newpage

  \begin{tabular}{|c|c|c|c|c|}
    \hline
    Strona & \multicolumn{2}{c|}{Wiersz} & Jest
    & Powinno być \\ \cline{2-3}
    & Od góry & Od dołu & & \\
    \hline
    569 & & 13 & \textit{Jude, Introduction}
    & \textit{Jude: Introduction} \\
    569 & & \hphantom{0}1 & \textit{Greek, a} & \textit{Greek: A} \\
    570 & & 17 & \textit{Cecilia, Virgin} & \textit{Cecilia: Virgin} \\
    570 & & 15 & \textit{Exile, a} & \textit{Exile: A} \\
    570 & & \hphantom{0}5 & \textit{during} & \textit{During} \\
    571 & \hphantom{0}3 & & \textit{Religion; the}
    & \textit{Religion: The} \\
    571 & 16 & & \textit{Eusebian; Essay} & \textit{Eusebian: Essay} \\
    571 & 10 & & \textit{Rome; the} & \textit{Rome: The} \\
    571 & & \hphantom{0}7 & \textit{Mani; a} & \textit{Mani: A} \\
    571 & & \hphantom{0}1 & \textit{before} & \textit{Before} \\
    572 & \hphantom{0}2 & & 1959 & 1959. \\
    572 & \hphantom{0}6 & & \textit{Edessa, the} & \textit{Edessa: The} \\
    572 & 11 & & \textit{Josephus, the} & \textit{Josephus: The} \\
    572 & 17 & & \textit{Kerala, a} & \textit{Kerala: A} \\
    572 & \hphantom{0}7 & & \textit{Severus, the} & \textit{Severus: The} \\
    573 & \hphantom{0}1 & & \textit{Aurelius, His}
    & \textit{Aurelius: His} \\
    573 & \hphantom{0}8 & & \textit{Seneca, a} & \textit{Seneca: A} \\
    573 & 12 & & \textit{Constantine, a} & \textit{Constantine: A} \\
    573 & 15 & & 1948 & 1948. \\
    573 & 20 & & \textit{after} & \textit{After} \\
    573 & 18 & & \textit{Claudius, the} & \textit{Claudius: The} \\
    573 & & \hphantom{0}9 & \textit{138A.D.} & \textit{138~A.D.} \\
    573 & & \hphantom{0}3 & \textit{Meroe, a} & \textit{Meroe: A} \\
    573 & & \hphantom{0}2 & \textit{under} & \textit{Under} \\
    573 & & \hphantom{0}2 & \textit{Rule, from} & \textit{Rule: From} \\
    574 & \hphantom{0}5 & & \textit{Nero, Reality}
    & \textit{Nero: Reality} \\
    575 & \hphantom{0}4 & & \textit{422} & 422 \\
    575 & \hphantom{0}6 & & \textit{421} & 421 \\
    575 & \hphantom{0}7 & & pne.- 7 & pne.~-- 7 \\
    % Linia 8, sprawdź czy daty życia Agrypiny są poprawne
    575 & \hphantom{0}8 & & \textit{421, 422,} & 421, 422, \\
    \hline
  \end{tabular}





  \newpage

  \begin{tabular}{|c|c|c|c|c|}
    \hline
    Strona & \multicolumn{2}{c|}{Wiersz} & Jest
    & Powinno być \\ \cline{2-3}
    & Od góry & Od dołu & & \\
    \hline
    575 & \hphantom{0}8 & & \textit{508} & 508 \\
    575 & 14 & & \textit{40} & 40 \\
    575 & 14 & & \textit{68} & 68 \\
    575 & 15 & & \textit{91} & 91 \\
    575 & 16 & & \textit{407} & 407 \\
    575 & & \hphantom{0}7 & \textit{22} & 22 \\
    575 & & \hphantom{0}7 & \textit{517} & 517 \\
    575 & & \hphantom{0}5 & \textit{528} & 528 \\
    575 & & \hphantom{0}4 & \textit{518} & 518 \\
    575 & & \hphantom{0}1 & ii & i \\
    \hline
  \end{tabular}

\end{center}

\VerSpaceSix


\noindent
\StrWierszD{575}{4} \\
\Jest \hspace{5pt} nik \\
\Powin -nik \\

% ############################










% ######################################
\section{Warren H.~Carroll \textit{Historia Chrześcijaństwa.
    Tom~II: Budowanie Chrześcijaństwa},
  \cite{CarrollHistoriaChrzecijanstwaVolII2010}}
% ######################################



% ############################
\subsection{Uwagi}
% ############################



\noindent
Tłumaczenie podtytułu tego tomu „Budowanie Chrześcijaństwa” jest wyjątkowo
niezręczne. Należy zwrócić uwagę, że~Carroll nadał swojemu cyklowi tytuł
„History~of Christendom” nie „History~of Christianity”. „Chrisitianity”
tłumaczy się prosto jako „chrześcijaństwo”, „chistendom” nie ma chyba
odpowiednika w~języku polski, w~tym przypadku zaś można jego sens chyba
wyjaśnić, jako wspólnotę ludzi, której sposób życia definiuje
chrześcijaństwo. W~szczególności „Christendom” oznacza również sens
polityczny, jako zbioru państw, które~są połączone wspólną wiarą
chrześcijańską i~tym samym powinny działać jak różne członki jednego ciała.

Jakkolwiek więc tłumaczenie tytułu całego cyklu jako „Historia
Chrześcijaństwa” ma~sens, to tego podtytułu jako „Budowanie
Chrześcijaństwa” już nie. Sugeruje bowiem, że~religia chrześcijańska
była budowana, podczas gdy ona została już wzniesiona przez Chrystusa,
zaś budowane było właśnie „Christendom”, wspólnota ludzka żyjąca jej
prawami.

\VerSpaceFour





% % ##################
% \CenterBoldFont{Uwagi do~konkretnych stron}





% ##################
\newpage

\CenterBoldFont{Błędy}


\begin{center}

  \begin{tabular}{|c|c|c|c|c|}
    \hline
    Strona & \multicolumn{2}{c|}{Wiersz} & Jest
                              & Powinno być \\ \cline{2-3}
    & Od góry & Od dołu & & \\
    \hline
    \hphantom{0}14 & 10 & & \textit{Eusebius}. & \textit{Eusebius}, \\
    \hphantom{0}14 & 11 & & \textit{A.D.324-344} & \textit{A.D. 324-344} \\
    \hphantom{0}14 & 11 & & \textit{problems} & \textit{Problems} \\
    \hphantom{0}14 & 12 & & \textit{cordoba} & \textit{Cordoba} \\
    \hphantom{0}14 & 12 & & \textit{council} & \textit{Council} \\
    \hphantom{0}14 & 13 & & studies & Studies \\
    \hphantom{0}14 & & 18 & \textit{church} & \textit{Church} \\
    \hphantom{0}15 & & \hphantom{0}5 & s.90-91 & s.~90-91 \\
    \hphantom{0}15 & & \hphantom{0}4 & s.117 & s.~117 \\
    \hphantom{0}15 & & \hphantom{0}1 & 1971) & 1971 \\
    \hphantom{0}16 & & \hphantom{0}5 & \textit{fourth} & \textit{Fourth} \\
    \hphantom{0}19 & & \hphantom{0}7 & \textit{A.D} & \textit{A.D.} \\
    \hphantom{0}19 & & \hphantom{0}2 & „Jerusalem” & \textit{Jerusalem} \\
    \hphantom{0}27 & & 17 & \textit{Egipt; the} & \textit{Egipt: The} \\
    \hphantom{0}27 & & 14 & s.126 & s.~126 \\
    \hphantom{0}29 & & \hphantom{0}2 & 346;Smith & 346; Smith\\
    \hphantom{0}30 & & 10 & s.285 & s.~285 \\
    \hphantom{0}31 & & \hphantom{0}4 & 341 (Kidd & 341; Kidd \\
    \hphantom{0}31 & & \hphantom{0}3 & 67,71 & 67, 71 \\
    \hphantom{0}32 & & \hphantom{0}6 & s.65 & s.~65 \\
    \hphantom{0}32 & & \hphantom{0}5 & \textit{Antoniego}.
    & \textit{Antoniego}, \\
    \hphantom{0}32 & & \hphantom{0}4 & 1987) & 1987). \\
    \hphantom{0}35 & & \hphantom{0}5 & 82,380 & 82, 380 \\
    \hphantom{0}38 & & \hphantom{0}1 & \textit{church} & \textit{Church} \\
    \hphantom{0}45 & & 10 & s.454 & s.~454 \\
    \hphantom{0}45 & & \hphantom{0}9 & 30,52 & 30, 52 \\
    \hphantom{0}45 & & \hphantom{0}9 & 68,76 & 68, 76 \\
    \hphantom{0}57 & & \hphantom{0}7 & \textit{chrześcijaństwie}.
    & \textit{chrześcijaństwie}, \\
    \hphantom{0}66 & & 17 & \textit{Jerome, His} & \textit{Jerome: His} \\
    \hline
  \end{tabular}





  \newpage

  \begin{tabular}{|c|c|c|c|c|}
    \hline
    Strona & \multicolumn{2}{c|}{Wiersz} & Jest
                              & Powinno być \\ \cline{2-3}
    & Od góry & Od dołu & & \\
    \hline
    \hphantom{0}72 & & \hphantom{0}4 & \textit{Chrysostom}
    & \textit{Chryzostom} \\
    \hphantom{0}74 & & \hphantom{0}6 & \textit{saint} & \textit{Saint} \\
    \hphantom{0}76 & & \hphantom{0}6 & Popes & \textit{Popes} \\
    \hphantom{0}76 & & \hphantom{0}6 & Church & \textit{Church} \\
    \hphantom{0}79 & & \hphantom{0}5 & \textit{Claudian; Poetry}
    & \textit{Claudian: Poetry} \\
    \hphantom{0}93 & & \hphantom{0}6 & \textit{Jerome, His}
    & \textit{Jerome: His} \\
    106 & & \hphantom{0}9 & \textit{Eusebius, bishop}
    & \textit{Eusebius: Bishop} \\
    114 & & \hphantom{0}9 & \textit{Chalcedon, a~Historical}
           & \textit{Chalcedon: A~Historical} \\
    116 & & \hphantom{0}4 & \textit{Arthur, a~History}
    & \textit{Arthur: A~History} \\
    116 & & \hphantom{0}4 & \textit{350-} & \textit{350} \\
    116 & & \hphantom{0}2 & 254---257 & 254-257 \\
    124 & & \hphantom{0}8 & 266,277,294 & 266, 277, 294 \\
    124 & & \hphantom{0}3 & \textit{Populi; Popular}
    & \textit{Populi: Popular} \\
    124 & & \hphantom{0}2 & \textit{ontroversies}
    & \textit{Controversies} \\
    141 & & 12 & \textit{history} & \textit{History} \\
    141 & & 12 & Stevens. & Stevens, \\
    141 & & \hphantom{0}8 & Stevens. & Stevens, \\
    154 & & \hphantom{0}3 & \textit{Moddle} & \textit{Middle} \\
    155 & & \hphantom{0}6 & \textit{Moddle} & \textit{Middle} \\
    156 & & 16 & \textit{Invasions; the} & \textit{Invasions: The} \\
    160 & & \hphantom{0}4 & \textit{I, an~introduction}
           & \textit{I: An~Introduction} \\
    161 & & \hphantom{0}8 & \textit{sixth} & \textit{Sixth} \\
    167 & 13 & & \textit{History}. & \textit{History}, \\
    168 & & \hphantom{0}9 & \textit{I, an} & \textit{I:~An} \\
    197 & & \hphantom{0}1 & \textit{Great, His} & \textit{Great: His} \\
    201 & & 21 & \textit{God; the} & \textit{God: The} \\
    235 & & \hphantom{0}4 & Mann , & Mann, \\
    236 & & \hphantom{0}7 & Mann , & Mann, \\
    238 & & 23 & \textit{conquests} & \textit{Conquests} \\
    249 & & \hphantom{0}3 & \textit{before} & \textit{Before} \\
    \hline
  \end{tabular}





  \newpage

  \begin{tabular}{|c|c|c|c|c|}
    \hline
    Strona & \multicolumn{2}{c|}{Wiersz} & Jest
                              & Powinno być \\ \cline{2-3}
    & Od góry & Od dołu & & \\
    \hline
    254 & & \hphantom{0}2 & V & t.~V \\
    270 & & \hphantom{0}3 & 343-344 [ & 343-344. \\
    294 & & \hphantom{0}4 & \textit{papal} & \textit{Papal} \\
    308 & & \hphantom{0}4 & \textit{Century: a~Study}
    & \textit{Century: A~Study} \\
    310 & & \hphantom{0}6 & \textit{continent} & \textit{Continent} \\
    314 & & \hphantom{0}4 & \textit{is.} & s. \\
    315 & & 11 & \textit{during} & \textit{During} \\
    315 & & \hphantom{0}3 & \textit{during} & \textit{During} \\
    388 & & 21 & \textit{Great, the} & \textit{Great: The} \\
    388 & & 19 & \textit{Dragon, Alfred} & \textit{Dragon: Alfred} \\
    388 & & 15 & A.Cotarelo & A.~Cotarelo \\
    388 & & \hphantom{0}3 & \textit{Magno}) & \textit{Magno} \\
    388 & & \hphantom{0}2 & \textit{Great: the} & \textit{Great: The} \\
    585 & & \hphantom{0}7 & London. & London \\
    586 & \hphantom{0}5 & & Struggle & \textit{Struggle} \\
    586 & 15 & & \textit{Chalcedon} & \textit{Chalcedon} \\
    586 & & 15 & \textit{1} & \textit{the~First} \\
    586 & & 10 & Danielou, Jean & Danielou Jean \\
    586 & & 10 & Henri Marrou & Marrou Henri \\
    587 & 13 & & \textit{A.D.} & \textit{A.D.}, \\
    587 & & 17 & \textit{Jerome,} & \textit{Jerome:} \\
    587 & & \hphantom{0}8 & London, & London \\
    587 & & \hphantom{0}2 & \textit{Moesia, a~History}
    & \textit{Moesia: History} \\
    588 & \hphantom{0}2 & & \textit{Arthur, a~History}
    & \textit{Arthur: A~History} \\
    588 & \hphantom{0}4 & & \textit{Invasion; the Making}
           & \textit{Invasion: The making} \\
    588 & 15 & & \textit{God; the Life} & \textit{God: The Life} \\
    588 & & 18 & \textit{Britain s} & \textit{Britain's} \\
    588 & & 17 & \textit{Chalcedon, a~Historical}
           & \textit{Chalcedon: A~Historical} \\
    590 & 14 & & \textit{Constantinople; Ecclesiastical}
           & \textit{Constantinople: Ecclesiastical} \\
    591 & & 12 & London, & London \\
    \hline
  \end{tabular}





  \newpage

  \begin{tabular}{|c|c|c|c|c|}
    \hline
    Strona & \multicolumn{2}{c|}{Wiersz}& Jest
                              & Powinno być \\ \cline{2-3}
    & Od góry & Od dołu & & \\
    \hline
    592 & 16 & & \textit{Lyons, Churchman} & \textit{Lyons: Churchman} \\
    592 & & \hphantom{0}9 & \textit{Great, the~King}
    & \textit{Great: The~King} \\
    592 & & \hphantom{0}8 & \textit{Canterbury; a~Study}
           & \textit{Canterbury: A~Study} \\
    592 & & \hphantom{0}3 & \textit{Slavs; Saints}
    & \textit{Slavs: Saints} \\
    593 & \hphantom{0}7 & & \textit{Empire; the~Arabs}
    & \textit{Empire: The~Arabs} \\
    593 & 12 & & \textit{Byzantium: the~Imperial}
           & \textit{Byzantium: The~Imperial} \\
    593 & 14 & & \textit{Kings; Their} & \textit{Kings: Their} \\
    593 & 16 & & \textit{England; a~History}
           & \textit{England: A~History} \\
    593 & 20 & & \textit{Great: the~Truth} & \textit{Great: The~Truth} \\
    593 & & 14 & \textit{State; the~Period} & \textit{State: The~Period} \\
    593 & & 12 & \textit{Dragon; Alfred} & \textit{Dragon: Alfred} \\
    593 & & \hphantom{0}5 & \textit{St.~Peter; the~Birth}
           & \textit{St.~Peter: The~Birth} \\
    594 & 12 & & \textit{Dublin: the~History}
           & \textit{Dublin: The~History} \\
    594 & & \hphantom{0}9 & \textit{Desiderius; Montecassino}
           & \textit{Desiderius: Montecassino} \\
    594 & & \hphantom{0}1 & \textit{Empire; the~Arabs}
    & \textit{Empire: The~Arabs} \\
    595 & \hphantom{0}1 & & \textit{Rufus; an~Investigation}
           & \textit{Rufus: An~Investigation} \\
    595 & \hphantom{0}6 & & \textit{Byzantium: the~Imperial}
           & \textit{Byzantium: The~Imperial} \\
    595 & \hphantom{0}8 & & \textit{England; a~History}
           & \textit{England: A~History} \\
    595 & 11 & & \textit{Kings; Their} & \textit{Kings: Their} \\
    595 & 16 & & \textit{State: the~Period} & \textit{State: The~Period} \\
    595 & 19 & & \textit{Tancred: a~Study} & \textit{Tancred: A~Study} \\
    595 & & 10 & \textit{Saint Peter; the~Reception}
           & \textit{Saint Peter: The~Reception} \\
    596 & \hphantom{0}6 & & \textit{440} & 440 \\
    % Popraw dalsze błędy w indeksie
    \hline
  \end{tabular}

\end{center}

\VerSpaceSix


\noindent
\StrWierszD{3}{4} \\
\Jest  www. WydawnictwoWektory.pl \\
\Powin www.WydawnictwoWektory.pl \\
\StrWierszD{13}{6} \\
\Jest  „Alexander~of Alexandria” \\
\Powin \textit{Alexander~of Alexandria} \\



% ############################










% ######################################
\section{Warren H.~Carroll \textit{Historia Chrześcijaństwa.
    Tom~IV: Podział Chrześcijaństwa},
  \cite{CarrollHistoriaChrzecijanstwaVolIV2011}}
% ######################################






% ##################
\newpage

\CenterBoldFont{Błędy}


\begin{center}

  \begin{tabular}{|c|c|c|c|c|}
    \hline
    Strona & \multicolumn{2}{c|}{Wiersz} & Jest
                              & Powinno być \\ \cline{2-3}
    & Od góry & Od dołu & & \\
    \hline
    \hphantom{0}26 & & \hphantom{0}2 & \textit{war} & \textit{War} \\
    \hphantom{0}31 & & \hphantom{0}6 & \textit{Blood; a}
    & \textit{Blood: A} \\
    \hphantom{0}33 & & \hphantom{0}5 & 122=124 & 122-124 \\
    \hphantom{0}38 & & \hphantom{0}2 & \textit{Conquistadors; First-Person}
    & \textit{Conquistadors; First-Person} \\
    \hphantom{0}39 & & \hphantom{0}5 & (cytat) ; & (cytat); \\
    \hphantom{0}40 & & \hphantom{0}2 & \textit{America; the}
    & \textit{America: The} \\
    \hphantom{0}41 & & \hphantom{0}2 & s.384 & s.~384 \\
    \hphantom{0}48 & & \hphantom{0}3 & 309; Zob. & 309; zob. \\
    \hphantom{0}49 & & \hphantom{0}2 & s.99 & s.~99 \\
    \hphantom{0}67 & & \hphantom{0}3 & s.84 & s.~84 \\
    \hphantom{0}78 & & \hphantom{0}3 & & \\
    \hphantom{0}97 & & \hphantom{0}2 & Marriman,\textit{Suleiman}
    & Marriman, \textit{Suleiman} \\
    \hphantom{0}97 & & \hphantom{0}2 & 94;von & 94; von\\
    \hphantom{0}97 & & \hphantom{0}2 & X,s.& X, s. \\
    110 & & \hphantom{0}2 & \textit{Won; the} & \textit{Won: The} \\
    111 & & \hphantom{0}5 & \textit{Cross. A} & \textit{Cross: A} \\
    112 & & \hphantom{0}7 & \textit{Enemies. The} & \textit{Enemies: The} \\
    115 & & \hphantom{0}5 & \textit{Master. A} & \textit{Master: A} \\
    130 & & \hphantom{0}9 & s.278 & s.~278 \\
    139 & & \hphantom{0}3 & s.31 & s.~31 \\
    149 & & \hphantom{0}3 & \textit{Vasas. A} & \textit{Vasas: A} \\
    160 & & \hphantom{0}5 & \textit{America. The} & \textit{America: The} \\
    163 & & \hphantom{0}4 & \textit{is} & \textit{Is} \\
    176 & & \hphantom{0}1 & Pastor,\textit{History}
    & Pastor, \textit{History} \\
    180 & & \hphantom{0}2 & \textit{Towns. A} & \textit{Towns: A} \\
    181 & & \hphantom{0}2 & \textit{Altars. Traditional}
           & \textit{Altars: Traditional} \\
    217 & & 14 & \textit{Calvin. The} & \textit{Calvin: The} \\
    217 & & \hphantom{0}6 & \textit{Calvin, the} & \textit{Calvin: The} \\
    229 & & \hphantom{0}1 & \textit{V, King} & \textit{V: King} \\
    \hline
  \end{tabular}





  \newpage

  \begin{tabular}{|c|c|c|c|c|}
    \hline
    Strona & \multicolumn{2}{c|}{Wiersz} & Jest
                              & Powinno być \\ \cline{2-3}
    & Od góry & Od dołu & & \\
    \hline
    230 & & \hphantom{0}2 & \textit{V, King} & \textit{V: King} \\
    234 & & \hphantom{0}5 & \textit{VI, the} & \textit{VI: The} \\
    236 & & \hphantom{0}7 & \textit{VI, the} & \textit{VI: The} \\
    236 & & \hphantom{0}5 & \textit{VI, the} & \textit{VI: The} \\
    236 & & \hphantom{0}3 & \textit{VI, the} & \textit{VI: The} \\
    237 & & \hphantom{0}6 & \textit{VI, the} & \textit{VI: The} \\
    237 & & \hphantom{0}5 & \textit{and} & \textit{and the} \\
    238 & & \hphantom{0}5 & \textit{VI, the} & \textit{VI: The} \\
    238 & & \hphantom{0}3 & \textit{VI, the} & \textit{VI: The} \\
    239 & & \hphantom{0}5 & \textit{VI, the} & \textit{VI: The} \\
    239 & & \hphantom{0}3 & \textit{VI, the} & \textit{VI: The} \\
    249 & & \hphantom{0}1 & \textit{Mass, and} & \textit{Mass and} \\
    \hline
  \end{tabular}





  \newpage

  \begin{tabular}{|c|c|c|c|c|}
    \hline
    Strona & \multicolumn{2}{c|}{Wiersz} & Jest
                              & Powinno być \\ \cline{2-3}
    & Od góry & Od dołu & & \\
    \hline
    816 & & \hphantom{0}2 & 223- & 223, \\
    817 & \hphantom{0}4 & & \textit{Towns; a} & \textit{Towns: The} \\
    817 & \hphantom{0}7 & & \textit{II, King} & \textit{II: King} \\
    817 & 10 & & \textit{Northumberland; the}
           & \textit{Northumberland: The} \\
    817 & 12 & & Hilaire. & Hilaire, \\
    817 & 13 & & \textit{during} & \textit{During} \\
    817 & 14 & & \textit{Absolutism. A} & \textit{Absolutism: A} \\
    817 & & 11 & \textit{V, King} & \textit{V: King} \\
    817 & & \hphantom{0}7 & Henrich. & Henrich, \\
    817 & & \hphantom{0}5 & Bradford, & Bradford \\
    817 & & \hphantom{0}2 & Anthony. & Anthony, \\
    817 & & \hphantom{0}2 & \textit{Magnificent, Scourge}
    & \textit{Magnificent: Scourge} \\
    818 & \hphantom{0}3 & & \textit{Bellarmine, Saint}
    & \textit{Bellarmine: Saint} \\
    818 & \hphantom{0}4 & & \textit{Loyola; the} & \textit{Loyola: The} \\
    818 & \hphantom{0}8 & & \textit{Darts. The} & \textit{Darts: The} \\
    818 & & 18 & A~\textit{History} & \textit{A~History} \\
    818 & & 12 & \textit{Playground. A} & \textit{Playground: A} \\
    818 & & \hphantom{0}6 & \textit{Altars; Traditional}
    & \textit{Altars: Traditional} \\
    818 & & \hphantom{0}6 & \textit{c.~1400-c. 1580} & \textit{1400-1580} \\
    818 & & \hphantom{0}4 & E.H. & E.H., \\
    818 & & \hphantom{0}1 & Philippe. & Philippe, \\
    % Popraw dalszą część bibliografii
    819 & & \hphantom{0}9 & 1913 & 1913. \\
    820 & & \hphantom{0}1 & 1992.. & 1992. \\
    823 & \hphantom{0}2 & & \textit{1621--9} & \textit{1621--1629} \\
    823 & \hphantom{0}6 & & \textit{1520--21} & \textit{1520--1521} \\
    823 & 17 & & (red.). & (red.), \\
    825 & \hphantom{0}5 & & Charles. & Charles, \\
    825 & 11 & & John., & John, \\
    825 & & 15 & \textit{World; Our} & \textit{World: Our} \\
    825 & & 14 & \textit{the~Sea; the~Treasure}
    & \textit{the~Sea: The~Treasure} \\
    \hline
  \end{tabular}





  \newpage

  \begin{tabular}{|c|c|c|c|c|}
    \hline
    Strona & \multicolumn{2}{c|}{Wiersz} & Jest
                              & Powinno być \\ \cline{2-3}
    & Od góry & Od dołu & & \\
    \hline
    825 & & \hphantom{0}5 & Carlos. & Carlos, \\
    825 & & \hphantom{0}2 & Parkman, Francis. & Parkman Francis, \\
    825 & & \hphantom{0}1 & Francis. & Francis, \\
    826 & \hphantom{0}7 & & \textit{Letters} & \textit{Times} \\
    826 & 12 & & St.~Louis. & St.~Louis \\
    826 & & \hphantom{0}8 & \textit{leyasu} & \textit{Ieyasu} \\
    826 & & \hphantom{0}4 & R.S. & R.S., \\
    % & & & & \\
    % & & & & \\
    % & & & & \\
    % & & & & \\
    % & & & & \\
    % & & & & \\
    % & & & & \\
    % & & & & \\
    % & & & & \\
    % & & & & \\
    % & & & & \\
    % & & & & \\
    % & & & & \\
    % & & & & \\
    % & & & & \\
    % & & & & \\
    % & & & & \\
    % & & & & \\
    % & & & & \\
    % & & & & \\
    % & & & & \\
    % & & & & \\
    % & & & & \\
    % & & & & \\
    % & & & & \\
    % & & & & \\
    % & & & & \\
    % & & & & \\
    % & & & & \\
    % & & & & \\
    % & & & & \\
    % & & & & \\
    % & & & & \\
    % & & & & \\
    \hline
  \end{tabular}

\end{center}

\VerSpaceSix


\noindent
\StrWierszD{165}{7} \\
\Jest Bruce,\textit{AnneBoleyn},s.293,299-307,313-333;Scarisbrick,\textit{HenryVIII},s.349-350;Ridley,\textit{Cran-} \\
\Powin Bruce, \textit{Anne Boleyn}, s.~293, 299-307, 313-333; Scarisbrick,
\textit{Henry VIII}, s.~349-350; Ridley, \textit{Cran-} \\
\StrWierszD{165}{6} \\
\Jest
\textit{mer},s.106-111.AnnęBoleynstracono19maja1536roku.Miałazaledwiedwadzieściaosiemlat. \\
\Powin \textit{mer}, s.~106-111. Annę Boleyn stracono 19~maja 1536 roku.
Miała zaledwie dwadzieścia osiem lat. \\


% ############################










% ######################################
\section{Warren H.~Carroll, Anne W. Carroll
  \textit{Historia Chrześcijaństwa. Tom~VI: Kryzys Chrześcijaństwa},
  \cite{CarrollCarrollHistoriaChrzecijanstwaVolVI2014}}
% ######################################



% ############################
\subsection{Uwagi do~konkretnych stron}
% ############################



\noindent
\Str{12} Wcięcia wszystkich akapitów poza pierwszy~są zbyt duże.

\VerSpaceFour





\noindent
\StrWierszeD{31}{4--2} Zdanie „Jestem zobowiązany Jamesowi H.~Billingtonowi,
\textit{Fire In the~Minds~of Man}, wielkiemu historykowi myśli
rewolucyjnej” po polsku brzmi źle i~jest trochę bez sensu. Nie wiem jednak
jak je~poprawić.

\VerSpaceFour





\noindent
\Str{33} Jest dziwne, że~Lamennais jest tu nazwany „wielkim, choć czasami
błądzącym, francuskim duchownym”, skoro sama ta książka podaje na~43
stronie, że~odrzuci on najpierw wiarę katolicką, potem zaś chrześcijaństwo.
Możliwe, że~ta nielogiczność jest wyniki pośmiertnej edycji i~uzupełniania
tego dzieła oraz pracy tłumacza.

\VerSpaceFour





\noindent
\Str{54} Pisze tu, że~bitwa pod Nowym Orleanem była decydującym momentem
w~Wojnie~1812 roku, powołując~się na książkę Paula Johnsona
\textit{Birth~of the~Modern}. Jednak w~tej pozycji Johnson przedstawia
zupełnie inną wersję wydarzeń. Bitwa ta rozegrała~się już po zawarciu
pokoju w~Londynie ???Sprawdź miasto???, ale~przed tym jak statek
z~informacją o~tym dotarła do~USA, jej przebieg nie doprowadził jednak
do~kontynuacji działań wojennych. Tym samym, konkluduje Johnson, nie
wpłynęła na zawarcie pokój, ale~bardzo na~jego recepcję. Amerykanie
mogli~się bowiem czuć zwycięzcami wojny jako, że~wygrali ostatnią jej
bitwę.

\VerSpaceFour





\noindent
\Str{63} Możliwe, że~informacje podane na tej i~na następnych stronach
dotyczące Ameryki Łacińskiej są poprawne, jednak napisane są w~sposób pełen
luk i~niejasności. Na~przykład na dole tej strony jest podane, że~Martin
skapitulował przed Monteverdim i~wyjechał z~Wenezueli, zaraz potem
zaś~został zdradzony, aresztowany i~wysłany przez Bolivara do~Hiszpanii
w~zamian za paszport, który umożliwi mu przyjazd do~Starego Kraju.
Wydaje~się mało prawdopodobne, by~Bolivar mógł aresztować Martina, gdyby
ten opuścił już Wenezuelę.

Poza tym, nie ma żadnego jasnego stwierdzenia, że~Bolivar wykorzystał
paszport i~udał~się do~Hiszpanii. Zaraz po~informacji, że~zdobył ten
dokument przenosimy~się do Trujillo dnia 15~czerwca 1813, co może
oznaczać miasto w~Hiszpania, ale~też jedno z~wielu o~takiej nazwie
w~Ameryce Południowej. Pierwszym pewnym miejsce w~którym go potem
widzimy, jest wenezuelska Barcelona.

\VerSpaceFour





\noindent
\StrWierszD{67}{8} Po~tej linii powinien nastąpić odstęp między przypisami.

\VerSpaceFour





\noindent
\Str{76} Następcą zmarłego w~1820~roku Jerzego~III Hanowerskiego był jego
najstarszy syn Jerzy~IV Hanowerski panujący w~latach 1820--1830. Dopiero
po~nim panował w~latach 1830--1837 panował Wilhelm~IV, który był młodszym
synem Jerzego~III, a~nie jego dalekim krewnym. Z~tego tej karygodnej
pomyłki wszelkie dalsze odniesienia do~działań tego monarchy mogą być
błędnie przypisanymi mu aktami Jerzego~IV, bądź źle umieszczone w~czasie.

\VerSpaceFour





\noindent
\StrWierszeD{83}{20--17} Zdanie „Tak samo było w~przypadku Lenina, kolejnego
wielkiego przywódcy rewolucji, który wychował~się w~pobożnej
chrześcijańskiej rodzinie, a~fakt, że~wedle jego własnego świadectwa,
utracił wiarę w~wieku szesnastu lat, nie miał na~to żadnego wpływu.”
źle brzmi i~bardzo trudno zrozumieć myśl jaką w~tym kontekście miało
przekazywać.

\VerSpaceFour





\noindent
\StrWierszD{85}{11} Gwiazdka w~tej linii jest za~mała.

\VerSpaceFour





\noindent
\Str{110} Na~tej stronie jest podane, że~gdy~w~1914 roku zamordowano
arcyksięcia Franciszka Ferdynanda i~jego żonę Zofię, Franciszkowi Józefowi
wyrwał~się raz jedyny okrzyk „Nie oszczędzono mi niczego!”, podczas gdy
na~stronie~115 jest napisane, iż~wykrzyknął on „Nie oszczędzono mi niczego
na~tej ziemni” w~momencie,gdy~dowiedział~się o~zamordowaniu swojej żony
Elżbiety. Te~dwa fragmenty zdają~się sobie przeczyć.

\VerSpaceFour





\noindent
\Str{125} W~drugim paragrafie na~tej stronie jest trochę zamieszani.
Na~początku jest mowa o~zebraniu 87 osób szwajcarskim Vevey. Na~samym jego
końcu jest mowa o~głosowaniu w~kortezach i~ilości głosów jaka tam padła,
co~nie ma chyba nic wspólnego z~tym zebraniem i~ilością osób która na nim
była, nie~pamiętam zaś aby w~tej książce była podana ilość osób
zasiadających w~kortezach.

\VerSpaceFour





\noindent
\StrWierszD{126}{8} Nie wiem czemu w~tej linii umieszczono słowa
\textit{Dios! Patria! Fueros! Rey!}

\VerSpaceFour





\noindent
\Str{135} Fragment utworu poety Grillparzera o~marszałku Radetzkim jest tu
cytowany z~innego źródła niż na~następnej stronie. Nie jest to żaden błąd,
jedynie trochę to dziwne.

\VerSpaceFour





\noindent
\Str{145} Dwa ostatnie paragrafy nie~mają wcięcia w~tekście.

\VerSpaceFour





\noindent
\Str{147} Stwierdzenie, że~to święty Piotr ustanowił papiestwo i~Kościół,
ten błąd jest szczególnie karygodny, jest sprzeczne z~wiarą katolicką.
Zapewne jest to herezja, lecz nie jestem na tyle kompetentny by~stwierdzić
to na 100\%. Jeśli jest to herezja, to wątpię by obarczała sumienie
Carrolla, który zapewne po prostu popełnił głupi błąd pisząc te słowa.

\VerSpaceFour





\noindent
\Str{151} Przynajmniej w~mojej opinii na~tej stronie panuje pewne
zamieszanie. Nie potrafię na~przykład z~całą pewnością
stwierdzić, które z~wydarzeń opisanych w~ostatnim paragrafie
odnoszą~się do~pierwszego synodu, a~które do drugiego.

\VerSpaceFour





\noindent
\StrWierszD{165}{14--12} Sens zdania „Wielu opuszczało ojczyznę, wypływając
do~USA z~niewielkich portów, a~ich nazwiska przetrwały tylko w~lokalnej
tradycji.” jest następujący. Pamięć o~tym, kto wówczas wypłynął do~Stanów
Zjednoczonych zachowała~się w~lokalnej tradycji ustnej, ale~nie
w~dokumentach z~tamtej epoki. W~tym sensie ich nazwiska nie przetrwały
w~źródłach, nie należy jednak przez to rozumieć, że~ich nazwiska zniknęły
z~użycia, co taka forma tego zdania może sugerować.

\VerSpaceFour





\noindent
\Str{173} Mam problem ze zrozumieniem opisanych tu powodów wybuchu wojny
francusko-pruskiej. Dlaczego niby informacja o~tym, że~Niemcy obrażają
Francuzów wysłana do~króla Prus Wilhelma miała spowodować wypowiedzenie
wojny przez Napoleona~III.

\VerSpaceFour





\noindent
\Str{218} Na~dole strony pozostawiono puste miejsce, które powinien
zajmować tekst z~następnej strony.

\VerSpaceFour





\noindent
\StrWierszD{225}{3} Po tej linii następuje za~duży odstęp.

\VerSpaceFour





\noindent
\Str{264} Dwa pierwsze paragrafy są źle sformatowane.

\VerSpaceFour





\noindent
\Str{274} Na~dole strony pozostawiono puste miejsce, które powinien
zajmować tekst z~następnej strony.

\VerSpaceFour





\noindent
\Str{277} Należy sprawdzić, czy w~czasie Powstania Tajpingów nie zginęło
na~polach bitew więcej osób, niż podczas I~Wojny Światowej. Uwaga którą tu
poczynił Carroll\footnote{Myślę, że~Anne W.~Carroll zgodziłaby~się
  na~przyznanie autorstwa jej mężowi Warrenowi.}, należy mieć na uwadze
czytając to~co pisze on~o~I~Wojnie Światowej na~stronach 867 i~873.

\VerSpaceFour





\noindent
\StrWierszD{299}{1} Czcionka w~tej linii jest za~duża.

\VerSpaceFour





\noindent
\StrWierszD{305}{4} Imię ojca Rasputina Efima, na~str.~313 jest pisane
„Jefim”.

\VerSpaceFour





\noindent
\StrWierszD{352}{1} Czcionka w~tej linii jest za~duża.

\VerSpaceFour





\noindent
\Str{355} W~pierwszym paragrafie jest mowa o~głosowaniu które
zakończyło~się wynikiem siedem do~pięciu, później zaś, że~decyzja
o~pokoju z~Niemcami przeszła stosunkiem siedem do~czterech. Najpewniej
w~obu przypadkach mowa jest o~tym samym głosowaniu i~jeden z~podanych
wyników jest błędny.

\VerSpaceFour





\noindent
\Str{383} Jeśli niczego nie przeoczyłem, to w~tym miejscu ostatni raz jest
mowa o~Denikinie i~jego armii, gdy wycofują~się na~Kubań i~Krym. Nie
dowiadujemy~się tym samym jakie były ich dalsze losy.

\VerSpaceFour





\noindent
\Str{397} Ponieważ Polska, zapewne tak samo, jak kraje nadbałtyckie, nie
istniała w~1914~r., jest nieprawdopodobne, by~w~memorandum Erzberga była
mowa o~nich jako o~sąsiadujących z~Niemcami. Należy~się domyślać,
że~Erzberg chciał włączenia wszystkich ziem które można było uznać za
w~jakimś sensie polskie, analogicznie dla~państw nadbałtyckich,
do~Cesarskich Niemiec po~wygranej wojnie.

\VerSpaceFour





\noindent
\StrWierszG{416}{22} Po tej linii powinien być większy odstęp.

\VerSpaceFour





\noindent
\StrWierszD{432}{8} Na~podstawie wcześniejszej części książki nie jestem
w~stanie powiedzieć o~co chodziło w~sprawie nadużyć w~Gruzji.

\VerSpaceFour





\noindent
\Str{435} Na~dole strony pozostawiono puste miejsce, które powinien
zajmować tekst z~następnej strony.

\VerSpaceFour





\noindent
\StrWierszD{437}{8} Wydaje mi~się, że~spotkałem~się z~wersją, iż~Trocki
został zabity ciosem czekanem. Należy to sprawdzić jeszcze w~jakiejś innej
pracy.

\VerSpaceFour





\noindent
\StrWierszeG{441}{1--2} Szacunki Carrollów, że~w~Chinach żyła jedna trzecia
ludności świata, budzą pewne moje wątpliwości. Po~pierwsze należałoby
ustalić o~jakim okresie czasu mowa, po~drugie należałoby sprawdzić, jak
rzeczywiście przedstawiał~się stosunek ludności Chin do ludności świata.

\VerSpaceFour





\noindent
\StrWierszeG{445}{17} Deng Xiaoping żył w~latach 1904--1997, zaś za moment
przejęcia jego władzy po~Mao Zedongu, który zmarł w~1976 roku, należy
chyba przyjąć rok~1978. Xiaoping miał więc wtedy nie dziewięćdziesiąt lecz
siedemdziesiąt cztery lata.

\VerSpaceFour





\noindent
\StrWierszG{451}{8} Tu~można powtórzyć wątpliwości odnośnie podanej
ludności~Chin i~jej udziału w~ludności świata, które~są w~komentarzu
do~strony~441.

\VerSpaceFour





\noindent
\StrWierszD{451}{6} Ten wiersz jest źle wcięty.

\VerSpaceFour





\noindent
\Str{456} Na~dole strony pozostawiono puste miejsce, które powinien chyba
zajmować tekst z~następnej strony. Choć w~tym wypadku możliwe jest,
że~obecny wybór jest lepszy.

\VerSpaceFour





\noindent
\StrWierszD{456}{11} Znak „*” w~tej linii jest za mały.

\VerSpaceFour





\noindent
\StrWierszD{466}{18} Ponieważ ten cytata zaczyna~się z~małej litery, do~tego
zaraz następuje znak~„\ldots”, co sugeruje, że~ten cytat został błędnie
przytoczony. Jednak nie wiem jak go~poprawić.

\VerSpaceFour





\noindent
\StrWierszD{466}{1} To~odwołanie bibliograficzne jest niedokończone.

\VerSpaceFour





\noindent
\StrWierszD{478}{5--6} W~mojej opinii te~dwie linie~są źle sformatowane.

\VerSpaceFour





\noindent
\Str{494} Stwierdzenie, że~Hitler i~Stalin byli w~owym czasie
najpotężniejszymi ludźmi na świecie jest dyskusyjne. Dorównywał ich
sile prezydent USA, można też dywagować, czy ludzie władający Japonią,
nie byli równie potężni\footnote{Nie jestem pewien, czy w~owym czasie
  cała Japońska władza należało do~cesarza Hirohito, stąd takie
  sformułowanie tego punktu.}.

\VerSpaceFour





\noindent
\Str{497} Należy sprawdzić, czy~rzeczywiście największą grupą etniczną
w~Związku Radzieckim byli Ukraińcy.

\VerSpaceFour





\noindent
\Str{497} Należy sprawdzić, jakie naprawdę temperatury~są, czy może raczej
były, zimą na~terenach Syberii. Temperatury rzędu -90\textcelsius
są~w~mojej ocenie mało prawdopodobne. Jeśli jednak Carrollom chodziło
o~-90 stopni Fahrenheita, to~oznaczałoby około -65\textcelsius, co~jest
już znacznie bardziej prawdopodobne.

\VerSpaceFour





\noindent
\Str{499} Gdyby na~31 milinów Ukraińców przypadało 11~milionów ton
zboża to na głowę przypadałoby nie jak piszą Carrollowie
123~kilogramy, lecz~355~kilogramów zboża. Jeśli zaś byłoby to 10
milionów ton zboża, to~na jednego Ukraińca przypadałoby 322 kilogramów
zboża. Ktoś wyraźnie coś pomylił w~rachunkach.

\VerSpaceFour





\noindent
\StrWierszG{503}{16} W~tym wersie znajduje~się malutkie wcięcie, którego
nie~powinno być. Osobiście uważam też, że~lepiej brzmiałby w~następującej
postaci: „Nadejdzie rok czarny, rok krwi i~pożarów”.

\VerSpaceFour





\noindent
\StrWierszD{503}{2} Znajdujące~się na~końcu tej linii~„w:”, wygląda bardzo
źle. Należałoby je przenieść na~początek następnej linii.

\VerSpaceFour





\noindent
\StrWierszD{507}{4} Ponad tą~linią powinien znajdować~się odstęp.

\VerSpaceFour





\noindent
\StrWierszD{516}{6} W~tej linii jest wcięcie, którego według mnie
nie~powinno tu~być.

\VerSpaceFour





\noindent
\StrWierszD{526}{6} Ciężko jest zrozumieć od~razu, kim był przywoływany
w~tej linii Jakub. Jest to zapewne karlistowski następca tronu, o~którym
była mowa w~rozdziale \textit{Zwycięstwo i~klęska tradycjonalistów}.
????Sprawdź kiedyś jak~się on~dokładnie nazywał.

\VerSpaceFour





\noindent
\StrWierszG{527}{17} Wcięcie tego akapitu jest zbyt duże.

\VerSpaceFour





\noindent
\Str{530} Według podany tu liczb prawica i~centrum zdobyły razem 210~miejsc
w~kortezach więc Front Ludowy z~263 miał 53, nie 26 miejsc przewagi. Albo
jakieś ugrupowanie zostało przemilczane, albo~te liczby zostały źle podane.

\VerSpaceFour





\noindent
\StrWierszG{537}{6} Skala śmierci w Hiszpańskiej Wojnie Domowej jest zbyt
mała, by~można ją było nazwać holokaustem.

\VerSpaceFour





\noindent
\StrWierszG{537}{6} Powinno tu~się znaleźć jawne potępienie zbrodni
nacjonalistów. Popełnienie ich w~reakcji nie znosi winy.

\VerSpaceFour





\noindent
\StrWierszD{540}{7} Wcięcie tego akapitu jest zbyt duże.

\VerSpaceFour





\noindent
\StrWierszD{543}{1} Do~formy całego przypisu niezbyt pasuje linia „Rafael
Casa de~la~Vega, \textit{Franco, żołnierz}, tłum.~J.~Chodorowski.”.

\VerSpaceFour





\noindent
\Str{547} Byłoby dziwne, gdyby Churchill ostrzegał Wielką Brytanię przed
Hitlerem w~latach 1924--1932, skoro przez większość tego czasu był
on~człowiekiem zupełnie pozbawionym wpływów. Jednak w~historii zdarzały~się
już dziwniejsze rzeczy.

\VerSpaceFour





\noindent
\Str{565} Tekst przypisów~69 i~70 trochę~się ze~sobą nie zgadzają.
W~przypisie 69~autorzy twierdzą, że~praca o~walkach na~Gudalcana Roberta
Leckiego jest pozycją niedoścignioną porównywalną tylko z~Tukidydesem.
Natomiast w~następnym, iż~najlepsza praca w~tym temacie to~ta autorstwa
Samuela Eliota Morisona.

\VerSpaceFour





\noindent
\Str{566} Przypisy od~tłumacza~są źle ponumerowane ilością gwiazdek.

\VerSpaceFour





\noindent
\StrWierszG{572}{12} Gwiazdka w~tej linii jest zbyt mała.

\VerSpaceFour





\noindent
\StrWierszD{585}{17--15} Te~linie~są źle sformatowane.

\VerSpaceFour





\noindent
\StrWierszD{609}{8} Gwiazdka w~tej linii jest za~mała.

\VerSpaceFour





\noindent
\StrWierszG{623}{10} Gwiazdka w~tej linii jest zbyt mała.

\VerSpaceFour





\noindent
\StrWierszD{778}{4--3} Te dwie linie, łącznie z~„251.” służącym za~odnośnik
tego przypisu powinny być częścią poprzedniego przypisu. Samo oznaczenie
„251.” jest częścią urwanych w~poprzedniej linii numerów stron: 250-251.

\VerSpaceFour





\noindent
\StrWierszD{867}{17} Linia jest źle zedytowana. Drugie zdanie w~tej linii
jest początkiem następnej pozycji w~bibliografii, powinna więc być zgodnie
z~tym sformatowana.

\VerSpaceFour





% ##################
\newpage

\CenterBoldFont{Błędy}


\begin{center}

  \begin{tabular}{|c|c|c|c|c|}
    \hline
    Strona & \multicolumn{2}{c|}{Wiersz} & Jest
                              & Powinno być \\ \cline{2-3}
    & Od góry & Od dołu & & \\
    \hline
    \hphantom{00}7 & & \hphantom{0}4 & wszystko$^{ * }$ & wszystko \\
    \hphantom{00}7 & & \hphantom{0}4 & 827 & 837 \\
    \hphantom{00}7 & & \hphantom{0}3 & Rekonkwiście$^{ * }$
    & Rekonkwiście \\
    \hphantom{0}23 & & 10 & \textit{Vhutch} & \textit{Church} \\
    \hphantom{0}24 & & 25 & Zbawiciela$^{ *^{ * } }$ & Zbawiciela$^{ ** }$ \\
    \hphantom{0}25 & & \hphantom{0}9 & 1919 & 1819 \\
    \hphantom{0}32 & 11 & & dostosowania & do~stosowania \\
    \hphantom{0}50 & & 12 & za~panowania & rozpoczęta za~panowania \\
    \hphantom{0}55 & & 12 & piętnstu & piętnastu \\
    \hphantom{0}55 & &  7 & interesy & interesy Południa \\
    \hphantom{0}67 & 17 & & bezbożności”)$^{ 31 }$
    & bezbożności”$^{ 31 }$) \\
    \hphantom{0}67 & \hphantom{0}8 & & północy & południa \\
    \hphantom{0}68 & 21 & & siom & siłom \\
    \hphantom{0}85 & \hphantom{0}1 & & Herald” Tribune” & Herald” \\
    \hphantom{0}96 & & \hphantom{0}2 & W.H. Warren & W.H. Carroll \\
    \hphantom{0}97 & & \hphantom{0}7 & „tak uważamy” & „Tak uważamy” \\
    104 & & 17 & i~związku & i~w~związku \\
    104 & & \hphantom{0}2 & W.H.~Warren & W.H.~Carroll \\
    105 & & 11 & Counter-Revolution & Counter-Revolution” \\
    105 & & \hphantom{0}5 & W.H.~Warren & W.H.~Carroll \\
    116 & & 15 & stał~się był & stał~się \\
    117 & & 23 & wyd.3,Boston & wyd.~3, Boston \\
    121 & \hphantom{0}3 & & tonizowały & uspokajały \\
    126 & & \hphantom{0}4 & Pampelunie. & Pampelunie). \\
    127 & 10 & & aż~przez & potem aż~przez \\
    135 & & 12 & doskonalej & doskonałej \\
    137 & \hphantom{0}6 & & go & je \\
    139 & & \hphantom{0}5 & \textit{1833} & \textit{1883} \\
    141 & & \hphantom{0}8 & tom~VI, rozdział~XIV & rozdział~VIII, \\
    141 & & \hphantom{0}7 & P\textit{olitical} & \textit{Political} \\
    \hline
  \end{tabular}





  \newpage

  \begin{tabular}{|c|c|c|c|c|}
    \hline
    Strona & \multicolumn{2}{c|}{Wiersz} & Jest
                              & Powinno być \\ \cline{2-3}
    & Od góry & Od dołu & & \\
    \hline
    141 & & \hphantom{0}5 & rozdział zatytułowany & rozdział~II, \\
    150 & \hphantom{0}5 & & dogmatach; & dogmatach, \\
    151 & & \hphantom{0}8 & torturom... & torturom. \\
    151 & & \hphantom{0}7 & miasta.. & miasta. \\
    152 & 22 & & i~i & i \\
    165 & & \hphantom{0}7 & 2003) & 2003 \\
    170 & 16 & & potomek & bratanek \\
    170 & & 10 & potomka, „F\"{u}hrera & potomka „F\"{u}hrera \\
    171 & & \hphantom{0}7 & skrajnym wręcz & wręcz skrajnym \\
    177 & \hphantom{0}3 & & „byliście & „Byliście \\
    187 & 21 & & roku~Na & roku. Na \\
    190 & & \hphantom{0}1 & Karl Marx & \textit{Karl Marx} \\
    194 & 12 & & etc.. & etc. \\
    194 & & \hphantom{0}8 & spikerze & mówcy \\
    197 & & \hphantom{0}7 & 1987) & 1987 \\
    198 & 16 & & wojny; & wojny \\
    199 & & \hphantom{0}2 & 210 & 210. \\
    208 & \hphantom{0}7 & & jest & jest natomiast \\
    208 & & \hphantom{0}5 & wschodni, wschodni & zachodni, wschodni \\
    210 & & \hphantom{0}6 & Pratt,, & Pratt, \\
    210 & & \hphantom{0}5 & Carrol & Carroll \\
    223 & & 20 & dna & dnia \\
    228 & & \hphantom{0}4 & 2005) & 2005 \\
    229 & & 14 & \textit{Westrn} & \textit{Western} \\
    233 & & 11 & piaty & piąty \\
    235 & 16 & & rzecz & Rzecz \\
    235 & & 15 & Uranu & Urana \\
    237 & & \hphantom{0}1 & 1954) & 1954 \\
    239 & 12 & & wyrazili & nie~wyrazili \\
    239 & 15 & & z~zatem & a~zatem \\
    \hline
  \end{tabular}





  \begin{tabular}{|c|c|c|c|c|}
    \hline
    Strona & \multicolumn{2}{c|}{Wiersz} & Jest
                              & Powinno być \\ \cline{2-3}
    & Od góry & Od dołu & & \\
    \hline
    243 & \hphantom{0}5 & & uczony; & uczony. \\
    243 & \hphantom{0}5 & & roku1743 & roku 1743 \\
    248 & & \hphantom{0}1 & 2008) & 2008 \\
    265 & & 13 & si & się \\
    267 & & \hphantom{0}3 & (1944 ) & (1944) \\
    269 & & \hphantom{0}5 & \textit{Germany, and} & \textit{Germany and} \\
    270 & & \hphantom{0}1 & \textit{s.} & s. \\
    273 & & \hphantom{0}3 & 1944 & 1994 \\
    294 & \hphantom{0}7 & & światu”... & światu... \\
    299 & & 11 & Kołłnotaj & Kołłontaj \\
    299 & & \hphantom{0}1 & 1988) & 1988 \\
    303 & & \hphantom{0}4 & \textit{opończa}” & \textit{opończa} \\
    308 & & 19 & niszczysz & Niszczysz \\
    309 & 22 & & warstw & wszystkich warstw \\
    317 & & \hphantom{0}3 & \textit{Kerensky; the}
    & \textit{Kerensky: The} \\
    320 & & \hphantom{0}3 & Habsburg & \textit{Habsburg} \\
    324 & & \hphantom{0}8 & eserowcow & eserowców \\
    327 & \hphantom{0}7 & & w~coraz & ludzie w~coraz \\
    330 & & \hphantom{0}1 & \textit{Wtnesses} & \textit{Witnesses} \\
    332 & & & Piotrogrodu,. & Piotrogrodu. \\ % Popraw tą linię
    338 & 22 & & roboty!, & roboty! \\
    349 & & \hphantom{0}1 & \textit{s.} & s. \\
    351 & \hphantom{0}4 & & 1917--1921 & 1914--1922 \\
    365 & & \hphantom{0}5 & miasta ; & miasta; \\
    373 & 19 & & rok & rok. \\
    379 & & \hphantom{0}2 & 1989) & 1989 \\
    380 & \hphantom{0}7 & & zlej & złej \\
    380 & & \hphantom{0}6 & destruktywna & destruktywną \\
    381 & & \hphantom{0}1 & 1951) & 1951 \\
    385 & 10 & & dopływem & odpływem \\
    \hline
  \end{tabular}





  \begin{tabular}{|c|c|c|c|c|}
    \hline
    Strona & \multicolumn{2}{c|}{Wiersz} & Jest
                              & Powinno być \\ \cline{2-3}
    & Od góry & Od dołu & & \\
    \hline
    387 & \hphantom{0}4 & & 1915--1922 & 1914--1922 \\
    387 & & \hphantom{0}2 & 1989) & 1989 \\
    392 & & \hphantom{0}3 & \textit{s.} & s. \\
    393 & & \hphantom{0}6 & \textit{s.} & s. \\
    393 & & \hphantom{0}2 & \textit{s.} & s. \\
    394 & \hphantom{0}7 & & z & z~dala \\
    394 & & \hphantom{0}1 & \textit{XV} , & \textit{XV}, \\
    396 & & 19 & Leonowi XII & Leonowi XIII \\
    408 & & \hphantom{0}2 & London1971 & London 1971 \\
    409 & & \hphantom{0}1 & \textit{Glory; Poland}
    & \textit{Glory: Poland} \\
    421 & & 17 & terroru\ldots. & terroru\ldots \\
    424 & & \hphantom{0}5 & Radzieckiej.Trzon & Radzieckiej. Trzon \\
    430 & & \hphantom{0}1 & \textit{under} & \textit{Under} \\
    431 & & \hphantom{0}2 & \textit{s.} & s. \\
    432 & & \hphantom{0}1 & \textit{s.} & s. \\
    433 & & \hphantom{0}2 & \textit{s.} & s. \\
    435 & \hphantom{0}2 & & krajem.. & krajem. \\
    436 & & \hphantom{0}1 & \textit{s.} & s. \\
    442 & \hphantom{0}6 & & imperium & imperium Czang \\
    443 & \hphantom{0}7 & & doobra & dobra \\
    446 & 10 & & Baun & Braun \\
    448 & 15 & & Hunan Jiangxi. & Huan i~Jangxi \\
    457 & & \hphantom{0}3 & \textit{war} & \textit{War} \\
    457 & & \hphantom{0}2 & \textit{war} & \textit{War} \\
    459 & & \hphantom{0}2 & \textit{s.} & s. \\
    460 & & \hphantom{0}2 & \textit{s.} & s. \\
    461 & & 11 & \textit{s.} & s. \\
    462 & & \hphantom{0}4 & \textit{Pro; Modern} & \textit{Pro. Modern} \\
    463 & & \hphantom{0}7 & \textit{s.} & s. \\
    464 & & \hphantom{0}8 & \textit{s.} & s. \\
    \hline
  \end{tabular}





  \newpage

  \begin{tabular}{|c|c|c|c|c|}
    \hline
    Strona & \multicolumn{2}{c|}{Wiersz} & Jest
                              & Powinno być \\ \cline{2-3}
    & Od góry & Od dołu & & \\
    \hline
    464 & & \hphantom{0}3 & \textit{s.} & s. \\
    465 & & \hphantom{0}9 & „Viva Cristo Rey” & \textit{Viva Cristo Rey} \\
    465 & & \hphantom{0}8 & 194. 199. & 194, 199. \\
    468 & \hphantom{0}6 & & \textit{Altars; Baltimore's}
    & \textit{Altars. Baltimore's} \\
    473 & 14 & & klepsydrze & klepsydrze. \\
    479 & & \hphantom{0}6 & \textit{s.} & s. \\
    485 & 12 & & obserwują., wyciągając & obserwują. Wyciągając \\
    487 & & 14 & John a.~Ryan & John A.~Ryan \\
    488 & & \hphantom{0}4 & pieniadze & pieniądze \\
    489 & 10 & & prac.$^{151}$ & prac$^{151}$. \\
    497 & 18 & & mniejszością & grupą \\
    498 & \hphantom{0}1 & & od & do \\
    502 & 23 & & \textit{od} & \textit{of} \\
    506 & & \hphantom{0}5 & zmienić & zmienić zdanie \\
    506 & & \hphantom{0}1 & \textit{gwałtownie} & gwałtownie \\
    507 & \hphantom{0}3 & & dłużej & długo \\
    507 & & \hphantom{0}2 & \textit{Archipelago} III
    & \textit{Archipelago}, t.~III \\
    511 & \hphantom{0}2 & & dotrzeć do~celu & dopłynąć do~celu \\
    514 & 11 & & Jarosławiu”)$^{ 22 }$ & Jarosławiu”$^{ 22 }$) \\
    514 & & \hphantom{0}2 & \textit{labor} & \textit{Labor} \\
    517 & & \hphantom{0}5 & Całkowita & całkowita \\
    520 & & \hphantom{0}4 & Cronica de~Alfonso~III
    & \textit{Cronica de~Alfonso~III} \\
    521 & & \hphantom{0}1 & wyd & wyd. \\
    522 & 18 & & kardynałem & kardynałem. \\
    523 & & 20 & roku & roku. \\
    525 & \hphantom{0}8 & & z~gabinecie & w~gabinecie \\
    527 & & & Toledo$^{ 18 }$\textbf{.} & Toledo$^{ 18 }$. \\
    529 & 20 & & lewacki & lewicowy \\
    529 & & \hphantom{0}3 & t.~Ivm Mardird & t.~I, Madrid \\
    530 & 21 & & \textit{Rey}” & \textit{Rey} \\
    \hline
  \end{tabular}





  \newpage

  \begin{tabular}{|c|c|c|c|c|}
    \hline
    Strona & \multicolumn{2}{c|}{Wiersz} & Jest
                              & Powinno być \\ \cline{2-3}
    & Od góry & Od dołu & & \\
    \hline
    531 & & 19 & Zamora\textbf{:} & Zamora: \\
    531 & & 18 & komunizmu”)$^{ 30 }$\textbf{.} & komunizmu”)$^{ 30 }$. \\
    531 & & 13 & W~spólnota & Wspólnota \\
    532 & 14 & & Cywilną$^{ 30 }$ & Cywilną \\
    532 & & \hphantom{0}3 & roli,,  % ''
           & roli, \\
    535 & & \hphantom{0}2 & \textit{mar tyrs} & \textit{Martyrs} \\
    536 & & \hphantom{0}5 & \textit{into} & \textit{Into} \\
    537 & & 22 & Reyes)\textbf{.} & Reyes). \\
    537 & & \hphantom{0}2 & \textit{1936} , & \textit{1936}, \\
    540 & & \hphantom{0}9 & docenili & doceniliby \\
    541 & \hphantom{0}8 & & (republikańskie & (Republikańskie \\
    541 & 10 & & komunistom). & komunistom.) \\
    541 & 15 & & faszyzmu”$^{ 55 }$\textbf{.} & faszyzmu”$^{ 55 }$. \\
    542 & & 21 & przywódca & Przywódca \\
    542 & & 21 & Boga” & Boga”. \\
    % Mogłem źle wyznaczyć początek cytatu.
    542 & & \hphantom{0}6 & jednym & „jednym \\
    545 & & \hphantom{0}4 & najzacieklejszych,. & najzacieklejszych \\
    545 & & \hphantom{0}1 & red & red. \\
    548 & & 10 & 1987) & 1987 \\
    548 & & \hphantom{0}5 & 1969) & 1969 \\
    549 & 18 & & obliczu & w~obliczu \\
    553 & & \hphantom{0}3 & \textit{1939--1940} , & \textit{1939--1940}, \\
    555 & & 18 & także teraz & teraz także \\
    555 & & \hphantom{0}5 & \textit{s.} & s. \\
    556 & & \hphantom{0}4 & \textit{s.} & s. \\
    556 & & \hphantom{0}1 & \textit{s.} & s. \\
    557 & & \hphantom{0}4 & \textit{s.} & s. \\
    559 & & 12 & \textit{s.} & s. \\
    561 & & \hphantom{0}4 & \textit{fate} & \textit{Fate} \\
    562 & & \hphantom{0}1 & 1985) & 1985 \\
    \hline
  \end{tabular}





  % \begin{tabular}{|c|c|c|c|c|}
  %   \hline
  %   Strona & \multicolumn{2}{c|}{Wiersz} & Jest
  %   & Powinno być \\ \cline{2-3}
  %   & Od góry & Od dołu & & \\
  %   \hline
  %   %   & & & & \\
  %   %   & & & & \\
  %   %   & & & & \\
  %   %   & & & & \\
  %   %   & & & & \\
  %   %   & & & & \\
  %   %   & & & & \\
  %   %   & & & & \\
  %   %   & & & & \\
  %   %   & & & & \\
  %   %   & & & & \\
  %   %   & & & & \\
  %   %   & & & & \\
  %   %   & & & & \\
  %   %   & & & & \\
  %   %   & & & & \\
  %   %   & & & & \\
  %   %   & & & & \\
  %   %   & & & & \\
  %   %   & & & & \\
  %   %   & & & & \\
  %   %   & & & & \\
  %   %   & & & & \\
  %   %   & & & & \\
  %   %   & & & & \\
  %   %   & & & & \\
  %   %   & & & & \\
  %   %   & & & & \\
  %   %   & & & & \\
  %   %   & & & & \\
  %   %   & & & & \\
  %   %   & & & & \\
  %   %   & & & & \\
  %   %   & & & & \\
  %   %   & & & & \\
  %   %   & & & & \\
  %   %   & & & & \\
  %   %   & & & & \\
  %   \hline
  % \end{tabular}





  \newpage

  \begin{tabular}{|c|c|c|c|c|}
    \hline
    Strona & \multicolumn{2}{c|}{Wiersz} & Jest
                              & Powinno być \\ \cline{2-3}
    & Od góry & Od dołu & & \\
    \hline
    563 & 20 & & dwa & trzy \\
    564 & \hphantom{0}1 & & samuraje & samurajowie \\
    565 & & \hphantom{0}3 & kampanii; & kampanii. \\
    566 & \hphantom{0}1 & & wspanialej & wspaniałej \\
    566 & 17 & & zaporami\ldots & zaopatrzeniem; \\
    566 & & 15 & Piekle!” & Piekle!”.” \\
    566 & & 12 & W~szakżeście& Wszakżeście \\
    570 & \hphantom{0}4 & & wschodu & zachodu \\
    571 & \hphantom{0}1 & & macDonald & MacDonald \\
    574 & & \hphantom{0}5 & \textit{Preious} & \textit{Precious} \\
    575 & & \hphantom{0}3 & 1997) & 1997 \\
    576 & & \hphantom{0}1 & \textit{s.} & s. \\
    581 & & \hphantom{0}1 & 212,. & 212. \\
    583 & & \hphantom{0}7 & \textit{ThePrice} & \textit{The~Price} \\
    587 & & 10 & 1984) & 1984 \\
    591 & & 12 & \textit{Lost; American} & \textit{Lost: American} \\
    601 & & \hphantom{0}3 & \textit{1949} & \textit{1949}, \\
    601 & & \hphantom{0}2 & \textit{Confoct} & \textit{Conflict} \\
    605 & & \hphantom{0}2 & \textit{Hungary from}
    & \textit{Hungary: From} \\
    607 & & \hphantom{0}2 & \textit{between} & \textit{Between} \\
    609 & & \hphantom{0}4 & kraju, W & kraju. W \\
    612 & & \hphantom{0}3 & \textit{balance} & \textit{Balance} \\
    612 & & \hphantom{0}2 & 1998) & 1998 \\
    617 & & \hphantom{0}8 & Szpiegostwo & szpiegostwo \\
    634 & & \hphantom{0}7 & \textit{s.} & s. \\
    646 & 19 & & \textit{Nie} & \textit{nie} \\
    685 & & \hphantom{0}8 & roku Isaacs & R. Isaacs \\
    690 & & \hphantom{0}5 & \textit{war} & \textit{War} \\
    693 & & \hphantom{0}2 & Kambodży & z~Kambodży \\
    709 & & \hphantom{0}6 & \textit{of} & \textit{to} \\
    \hline
  \end{tabular}





  \newpage

  \begin{tabular}{|c|c|c|c|c|}
    \hline
    Strona & \multicolumn{2}{c|}{Wiersz} & Jest
                              & Powinno być \\ \cline{2-3}
    & Od góry & Od dołu & & \\
    \hline
    778 & & \hphantom{0}5 & 250- & 250-251. \\
    785 & & \hphantom{0}3 & Paul\_R\_Ehrlich & Paul\_R\_Ehrlich. \\
    790 & & \hphantom{0}8 & Centrulo I~Amy & Centrulo i~Amy \\
    791 & & \hphantom{0}4 & html & html. \\
    793 & & \hphantom{0}7 & 145 & 145. \\
    793 & & \hphantom{0}5 & \textit{s.} & s. \\
    797 & & \hphantom{0}6 & Zob. & zob. \\
    797 & & \hphantom{0}5 & Zob. & zob. \\
    797 & & \hphantom{0}4 & Zob. & zob. \\
    797 & & \hphantom{0}1 & Las & Last \\
    798 & 21 & & \textit{Kościół} & \textit{Kościół~są} \\
    798 & 22 & & \textit{są~Drogą} & \textit{Drogą} \\
    858 & \hphantom{0}5 & & Pio Non (bł.~Pius~IX):
           & \textit{Pio Non (bł.~Pius~IX):} \\
    858 & 19 & & \textit{ofCatholic} & \textit{of Catholic} \\
    858 & 19 & & \textit{History,}(St.~Louis
           & \textit{History} (St.~Louis \\
    859 & 13 & & portugalskiej & portugalskiej. \\
    860 & \hphantom{0}5 & & DuffDavid. & Duff David \\
    861 & \hphantom{0}7 & & państwa.. & państwa. \\
    861 & 18 & & wyd.. & wyd. \\
    862 & \hphantom{0}6 & & S. John Brown & S., \textit{John Brown} \\
    862 & & \hphantom{0}2 & Jen. & Jen, \\
    864 & \hphantom{0}5 & & York, & York \\
    864 & 20 & & \textit{against} & \textit{Against} \\
    864 & 21 & & York, & York \\
    866 & & \hphantom{0}2 & FDR & \textit{FDR} \\
    867 & 15 & & York, & York \\
    867 & & \hphantom{0}2 & York, & York \\
    868 & \hphantom{0}3 & & 2004.. & 2004. \\
    868 & 15 & & York, & York \\
    868 & 23 & & York, & York \\
    \hline
  \end{tabular}





  \newpage

  \begin{tabular}{|c|c|c|c|c|}
    \hline
    Strona & \multicolumn{2}{c|}{Wiersz} & Jest
    & Powinno być \\ \cline{2-3}
    & Od góry & Od dołu & & \\
    \hline
    868 & 25 & & \textit{kardynał} & \textit{Cardinal} \\
    869 & 24 & & \textit{Denikin} & \textit{Denikin.} \\
    869 & & 12 & wojskowości.. & wojskowości \\
    870 & \hphantom{0}8 & & 1937) & 1937). \\
    871 & 13 & & 1939 1961 & 1939, 1961 \\
    871 & 17 & & jedneaj & jednej \\
    871 & & 13 & \textit{Day 1918: World} & \textit{Day, 1918. World} \\
    871 & & 12 & \textit{its} & \textit{Its} \\
    871 & & \hphantom{0}7 & \textit{under} & \textit{Under} \\
    873 & \hphantom{0}1 & & York, & York \\
    873 & \hphantom{0}9 & & York, & York \\
    873 & 10 & & York, & York \\
    873 & & \hphantom{0}5 & 1958 1966 & 1958, 1966 \\
    874 & & \hphantom{0}6 & Najlepsze I~najbardziej
    & Najlepsze i~najbardziej \\
    %   & & & & \\
    %   & & & & \\
    %   & & & & \\
    %   & & & & \\
    %   & & & & \\
    %   & & & & \\
    %   & & & & \\
    %   & & & & \\
    %   & & & & \\
    %   & & & & \\
    %   & & & & \\
    %   & & & & \\
    %   & & & & \\
    %   & & & & \\
    %   & & & & \\
    %   & & & & \\
    %   & & & & \\
    %   & & & & \\
    %   & & & & \\
    %   & & & & \\
    %   & & & & \\
    %   & & & & \\
    %   & & & & \\
    %   & & & & \\
    %   & & & & \\
    %   & & & & \\
    %   & & & & \\
    %   & & & & \\
    %   & & & & \\
    %   & & & & \\
    %   & & & & \\
    %   & & & & \\
    %   & & & & \\
    %   & & & & \\
    %   & & & & \\
    %   & & & & \\
    %   & & & & \\
    %   & & & & \\
    \hline
  \end{tabular}

\end{center}

\VerSpaceSix


\noindent
\StrWierszG{103}{5} \\
\Jest  mianem krucjaty (\textit{la~cruzada} ) określali \\
\Powin określali mianem krucjaty (\textit{la~cruzada}) \\
\StrWierszD{170}{10} \\
\Jest  „F\"{u}hrera z~Poczdamu”, ojca Fryderyka Wielkiego \\
\Powin „F\"{u}hrera z~Poczdamu”, Fryderyka Williama~I, ojca Fryderyka
Wielkiego \\
\StrWierszD{228}{5} \\
\Jest  The~Victory~of Reason: How Christianity Led to Freedom,
Capitalism and~Western Success \\
\Powin \textit{The~Victory~of Reason: How Christianity Led to Freedom,
  Capitalism and~Western Success} \\
\StrWierszG{234}{18} \\
\Jest  zapoczątkowujących teorię indukcji elektromagnetycznej \\
\Powin które doprowadziły do~powstania teorii indukcji
elektromagnetycznej \\
\StrWierszD{237}{1} \\
\Jest  Ford: The~Times, the~Man, and~the~Company \\
\Powin \textit{Ford: The~Times, the~Man, and~the~Company} \\
\StrWierszD{246}{9} \\
\Jest  Alexander Graham Bell and~the~Passion for~Invention \\
\Powin \textit{Alexander Graham Bell and~the~Passion for~Invention} \\
\StrWierszD{299}{2} \\
\Jest  Three Who Made a~Revolution \\
\Powin \textit{Three Who Made a~Revolution} \\
\StrWierszD{383}{18} \\
\Jest  Kołczak \\
\Powin Kołczak doszedł do wniosku \\
\StrWierszD{507}{14} \\
\Jest \textit{przeciwko zastosowaniu kary śmierci. Co~więcej,
  przekonał do~swego poglądu Politbiuro.} \\
\Powin przeciwko zastosowaniu kary śmierci. Co~więcej, przekonał
do~swego poglądu Politbiuro. \\
\StrWierszD{545}{10} \\
\Jest  i~dwutomowa \textit{Visions~of Glory} (Boston 1983),
\textit{Alone} \\
\Powin w~dwóch tomach: \textit{Visions~of Glory} (Boston 1983)
i~\textit{Alone} \\
\StrWierszD{566}{5} \\
\Jest  South Pacific Combat Air Transport \\
\Powin SCAT (\textit{South Pacific Combat Air Transport}) \\
\StrWierszD{566}{2} \\
\Jest  WAC~~(Women's Army Corps) \\
\Powin WAC~(\textit{Women's Army Corps}) \\
\StrWierszD{790}{3--1} \\
\Jest „Akcja afirmatywna w~orzecznictwie Sądu Najwyższego Stanów
Zjednoczonych”, Z~problemów bezpieczeństwa. Prawa człowieka \\
\Powin \textit{Akcja afirmatywna w~orzecznictwie Sądu Najwyższego Stanów
  Zjednoczonych}, w:~\textit{Z~problemów bezpieczeństwa. Prawa człowieka} \\
\StrWierszD{791}{3} \\
\Jest \textit{Infant Himicides through} \\
\Powin Bogomir Kuhar, \textit{Infant Homicides Through} \\
\StrWierszD{797}{6} \\
\Jest  „Why Can't We~Love Them Both?” \\
\Powin \textit{Why Can't We~Love Them Both?} \\
\StrWierszD{799}{5} \\
\Jest  „Pope John Paul~II's Encyclical \textit{Veritatis Splendor}” \\
\Powin \textit{Pope John Paul~II's Encyclical „Veritatis Splendor”} \\



% ############################
















% % ######################################
% \newpage
% \section{Pozostali autorzy}

% \vspace{\spaceTwo}
% % ######################################










% #####################################################################
% #####################################################################
% Bibliography

\bibliographystyle{plalpha}

\bibliography{DEUSBooks}{}





% ############################

% End of the document
\end{document}
