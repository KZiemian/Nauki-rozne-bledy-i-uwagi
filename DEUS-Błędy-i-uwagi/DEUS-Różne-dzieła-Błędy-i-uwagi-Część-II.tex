% ------------------------------------------------------------------------------------------------------------------
% Basic configuration and packages
% ------------------------------------------------------------------------------------------------------------------
% Package for discovering wrong and outdated usage of LaTeX.
% More information to be found in l2tabu English version.
\RequirePackage[l2tabu, orthodox]{nag}
% Class of LaTeX document: {size of paper, size of font}[document class]
\documentclass[a4paper,11pt]{article}



% ------------------------------------------------------
% Packages not tied to particular normal language
% ------------------------------------------------------
% This package should improved spaces in the text
\usepackage{microtype}
% Add few important symbols, like text Celcius degree
\usepackage{textcomp}



% ------------------------------------------------------
% Polonization of LaTeX document
% ------------------------------------------------------
% Basic polonization of the text
\usepackage[MeX]{polski}
% Switching on UTF-8 encoding
\usepackage[utf8]{inputenc}
% Adding font Latin Modern
\usepackage{lmodern}
% Package is need for fonts Latin Modern
\usepackage[T1]{fontenc}



% ------------------------------------------------------
% Setting margins
% ------------------------------------------------------
\usepackage[a4paper, total={14cm, 25cm}]{geometry}



% ------------------------------------------------------
% Setting vertical spaces in the text
% ------------------------------------------------------
% Setting space between lines
\renewcommand{\baselinestretch}{1.1}

% Setting space between lines in tables
\renewcommand{\arraystretch}{1.4}



% ------------------------------------------------------
% Packages for scientific papers
% ------------------------------------------------------
% Switching off \lll symbol, that I guess is representing letter "Ł"
% It collide with `amsmath' package's command with the same name
\let\lll\undefined
% Basic package from American Mathematical Society (AMS)
\usepackage[intlimits]{amsmath}
% Equations are numbered separately in every section
\numberwithin{equation}{section}

% Other very useful packages from AMS
\usepackage{amsfonts}
\usepackage{amssymb}
\usepackage{amscd}
\usepackage{amsthm}

% Package with better looking calligraphy fonts
\usepackage{calrsfs}

% Package with better looking greek letters
% Example of use: pi -> \uppi
\usepackage{upgreek}
% Improving look of lambda letter
\let\oldlambda\Lambda
\renewcommand{\lambda}{\uplambda}




% ------------------------------------------------------
% BibLaTeX
% ------------------------------------------------------
% Package biblatex, with biber as its backend, allow us to handle
% bibliography entries that use Unicode symbols outside ASCII
\usepackage[
language=polish,
backend=biber,
style=alphabetic,
url=false,
eprint=true,
]{biblatex}

\addbibresource{LogikaITeoriaMnogosciBibliography.bib}





% ------------------------------------------------------
% Defining new environments (?)
% ------------------------------------------------------
% Defining enviroment "Wniosek"
\newtheorem{corollary}{Wniosek}
\newtheorem{definition}{Definicja}
\newtheorem{theorem}{Twierdzenie}





% ------------------------------------------------------
% Local packages
% You need to put them in the same directory as .tex file
% ------------------------------------------------------
% Package containing various command useful for working with a text
\usepackage{./Local-packages/general-commands}





% ------------------------------------------------------
% Package "hyperref"
% They advised to put it on the end of preambule
% ------------------------------------------------------
% It allows you to use hyperlinks in the text
\usepackage{hyperref}










% ------------------------------------------------------------------------------------------------------------------
% Title and author of the text
\title{DEUS: dzieła świętych i~błogosławionych \\
  {\Large Błędy i~uwagi. Część~I}}

\author{Kamil Ziemian}


% \date{}
% ------------------------------------------------------------------------------------------------------------------










% ####################################################################
% Początek dokumentu
\begin{document}
% ####################################################################





% ######################################
\maketitle  % Tytuł całego tekstu
% ######################################





% % ######################################
% \section{Święta wiara i~filozofia}

% \vspace{\spaceTwo}
% % ######################################



% % ############################
% \subsection{Dzieła powstałe w~XX i~XXI wieku}

% \vspace{\spaceThree}
% % ############################



% ######################################
% \newpage
\section{Dariusz Karłowisz
  \textit{Arcyparadoks śmierci. Męczeństwo jako kategoria
    filozoficzna~-- pytanie o~dowodową wartość męczeństwa},
  \cite{KarlowiczArcyparadoksSmierci2021}}


% ######################################


% ##################
\CenterBoldFont{Uwagi do~konkretnych stron}

\vspace{0em}


\noindent
\StrWierszeGora{20}{1--2} Nie rozumiem czemu w~tym wierszach pisze~się
o~„bardziej
cnotliwym” i~„doskonale szczęśliwym” życiu. Należałoby~się spodziewać,
że~tak jak jest mowa o~życiu bardziej cnotliwym, tak samo powinna być mowa
o~życiu bardziej szczęśliwym.

\VerSpaceFour





% \StrWd{}{}

% \vspace{\spaceFour}





% \StrWd{}{}

% \vspace{\spaceFour}









% ##################
\CenterBoldFont{Błędy}


\begin{center}

  \begin{tabular}{|c|c|c|c|c|}
    \hline
    Strona & \multicolumn{2}{c|}{Wiersz} & Jest
                              & Powinno być \\ \cline{2-3}
    & Od góry & Od dołu & & \\
    \hline
           & & & & \\
        % & & & & \\
            % & & & & \\
                % & & & & \\
                    % & & & & \\
                        % & & & & \\
                            % & & & & \\
    \hline
  \end{tabular}





  \begin{tabular}{|c|c|c|c|c|}
    \hline
    & \multicolumn{2}{c|}{} & & \\
    Strona & \multicolumn{2}{c|}{Wiersz} & Jest
                              & Powinno być \\ \cline{2-3}
    & Od góry & Od dołu & & \\
    \hline
    % & & & & \\
    % & & & & \\
    % & & & & \\
    % & & & & \\
    % & & & & \\
    % & & & & \\
    % & & & & \\
    \hline
  \end{tabular}

\end{center}

\VerSpaceTwo


\noindent
\textbf{Tylna okładka, wiersz 5.} \\
\Jest śmierci \\
\PowinnoByc \textit{śmierci} \\

% ############################










% ######################################
\newpage

\subsection{Refleksje nad świętą wiarą w~okresie po 1945~r.}

\VerSpaceThree
% ######################################



% ######################################
% \newpage
\section{Romano Amerio \textit{Iota unum. Analiza zmian
    w~Kościele Katolickim},
  \cite{AmerioIotaUnum}}

% ######################################

\vspace{0em}


% ##################
\CenterBoldFont{Uwagi}

\vspace{0em}


\noindent
W~tej książce sposób podawania tytułów przywoływanych dzieł
jest niejednolity. Zamiennie używane~są trzy konwencje: „Tytuł”,
\textit{Tytuł}, „\textit{Tytuł}”.

\VerSpaceFour





\noindent
\StrWierszDol{168}{10} W~tej linii odstępy między słowami~są zbyt duże.

\VerSpaceFour





\noindent
\StrWierszDol{180}{12} Tekst zaczynający~się po~myślniku jest odrobinę
za~mocno
wcięty.

\VerSpaceFour





\noindent
\StrWierszDol{231}{2} Nie wiem czy zaznaczanie kursywą pewnych słów
w~zaczynającym~się tu cytacie jest błędem czy nie?

\VerSpaceFour





\noindent
\StrWierszGora{310}{5} Użyte tu obraźliwe słowo „lewactwo” tak nie pasuje
do stylu
reszty książki, że~zapewne jest to efekt niedopuszczalnej swobody tłumacza.

\VerSpaceFour





\noindent
\StrWierszGora{331}{1} Zdanie „Idea \textit{socjalizmu chrześcijańskiego}
ma~niewątpliwie swoją rację bytu”, brzmi jakoś dziwnie. Może jego obecna
forma jest wynikiem pomyłki edytorskiej?

\VerSpaceFour





\noindent
\Str{425--426} W~dialogu \textit{Protagoras} Platon, Sokrates twierdzi,
że~na polityce każdy zna~się tak samo dobrze, choć dialog ten kończy~się
sytuacją, gdy dochodzi on do przeciwnego wniosku. W~świetle tego należałoby
głębiej zbadać dzieła Platona i~inne traktujące o~Sokratesie, by ustalić,
czy rzeczywiście uważał, iż~o~funkcjonowanie miasta należy pytać polityka.

\VerSpaceFour





\noindent
\StrWierszDol{439}{2} Po ostatnim słowie greckim powinien być zamykający
cudzysłów. W~tym momencie nie umiem pisać w~\LaTeX u~po alfabetem greckim
w~taki sposób, by napisać to słowo dobrze. Mam problem z czcionką
i~akcentami.

\VerSpaceFour





\noindent
\StrWierszGora{440}{1} Poza ostatnim słowem, ta linie jest powtórzenie
tekstu z~poprzedniej strony.

\VerSpaceFour





\noindent
\Str{452} Treść akapitu u~góry strony jest dziwna i~ciężka do~zrozumienia.
Należałoby dobrze przemyśleć temat, który porusza.

\VerSpaceFour





\noindent
\StrWierszDol{452}{5} W~tej linii odstępy między wyrazami~są zbyt duże.

\VerSpaceFour





\noindent
\StrWierszDol{493}{7} W~tej linii odstępy między wyrazami~są zbyt duże.

\VerSpaceFour





\noindent
\StrWierszDol{496}{4} Ponownie, użyte tu obraźliwe słowo „lewactwo” tak
nie pasuje do stylu reszty książki, że~zapewne jest to efekt
niedopuszczalnej swobody tłumacza.

\VerSpaceFour





\noindent
\StrWierszDol{513}{7} W~tej linii odstępy między wyrazami~są zbyt duże.

\VerSpaceFour





\noindent
\textbf{Tylna okładka.} Informacje tu podane są w~mojej ocenie wątpliwe
albo dyskusyjne. Przykładem tego jest uznanie tu Alessandro Manzoniego
za~największego filozofa i~poetę XVII~wieku.

\VerSpaceFour





% ##################
\newpage

\CenterBoldFont{Błędy}


\begin{center}

  \begin{tabular}{|c|c|c|c|c|}
    \hline
    Strona & \multicolumn{2}{c|}{Wiersz} & Jest
                              & Powinno być \\ \cline{2-3}
    & Od góry & Od dołu & & \\
    \hline
    \hphantom{0}30 & & \hphantom{0}8 & „franciszkanie” & „franciszkanów” \\
    \hphantom{0}42 & & 10 & „\textit{O~Rewolucji Francuskiej}”
    & „\textit{O~Rewolucji Francuskiej} \\
    \hphantom{0}89 & & 10 & SoboruWatykańskiego & Soboru Watykańskiego \\
    \hphantom{0}92 & \hphantom{0}5 & & Bożych & Bożych” \\
    \hphantom{0}94 & \hphantom{0}9 & & lat$^{ 62 }$” & lat”$^{ 62 }$ \\
    103 & & \hphantom{0}2 & tytułrm & tytułem \\
    110 & & \hphantom{0}4 & P. & R. \\
    112 & \hphantom{0}7 & & Liturgii & liturgii \\
    133 & & \hphantom{0}9 & (=\textit{ale}) & (= \textit{ale}) \\
    144 & \hphantom{0}8 & & Chrystusa$^{ 94 }$” & Chrystusa”$^{ 94 }$\\
    161 & & \hphantom{0}4 & [Kościoła]$^{ 114 }$?” & [Kościoła]?”$^{ 114 }$ \\
    166 & & \hphantom{0}6 & katolicyzmu$^{ 116 }$” & katolicyzmu”$^{ 116 }$ \\
    170 & & 15 & Rady & rady \\
    172 & & \hphantom{0}1 & Bogiem$^{ 121 }$” & Bogiem”$^{ 121 }$ \\
    175 & & 10 & Papieża$^{ 122 }$” & Papieża”$^{ 122 }$ \\
    175 & & \hphantom{0}8 & sensu$^{ 123 }$” & sensu”$^{ 123 }$ \\
    183 & & \hphantom{0}9 & niegodni$^{ 129 }$” & niegodni”$^{ 129 }$ \\
    232 & \hphantom{0}1 & & suos.$^{ 156 }$” & suos.”$^{ 156 }$ \\
    233 & 12 & & regułą$^{ 157 }$” & regułą”$^{ 157 }$ \\
    239 & & \hphantom{0}7 & \textit{flecti}$^{ 160 }$”
    & \textit{flecti}”$^{ 160 }$ \\
    242 & \hphantom{0}1 & & \textit{alsit}$^{ 162 }$”
    & \textit{alsit}”$^{ 162 }$ \\
    242 & & 13 & \textit{sbarro}$^{ 163 }$” & \textit{sbarro}”$^{ 163 }$ \\
    254 & & \hphantom{0}6
                              & \textit{domem}$^{ 165 }$
    & \textit{domem}”$^{ 165 }$ \\
    255 & \hphantom{0}2 & & \textit{rozwijac się}$^{ 166 }$”
           & \textit{rozwijac się}”$^{ 166 }$ \\
    256 & \hphantom{0}1 & & pracy$^{ 168 }$” & pracy”$^{ 168 }$ \\
    283 & 16 & & rasy” & rasy”$^{ 186 }$ \\
    283 & 16 & & 99)$^{ 186 }$ & 99) \\
    293 & 10 & & nagrodę$^{ 191 }$” & nagrodę”$^{ 191 }$ \\
    293 & 15 & & \textit{ciała}$^{ 192 }$” & \textit{ciała}”$^{ 192 }$ \\
    \hline
  \end{tabular}





  \newpage

  \begin{tabular}{|c|c|c|c|c|}
    \hline
    Strona & \multicolumn{2}{c|}{Wiersz} & Jest
                              & Powinno być \\ \cline{2-3}
    & Od góry & Od dołu & & \\
    \hline
    293 & & 10 & duszy$^{ 193 }$” & duszy”$^{ 193 }$ \\
    297 & 10 & & skażona~$^{ 196 }$ & skażona$^{ 196 }$ \\
    301 & \hphantom{0}5 & & zaniedbać$^{ 197 }$” & zaniedbać”$^{ 197 }$ \\
    318 & & 14 & socjalistyczną$^{ 202 }$” & socjalistyczną”$^{ 202 }$ \\
    319 & 14 & & Ducha$^{ 203 }$” & Ducha”$^{ 203 }$ \\
    339 & & 10 & absolutnych? & absolutnych?” \\
    363 & & \hphantom{0}6 & wydaw\textbf{n}. & wydawn. \\
    381 & \hphantom{0}2 & & Notre- Dame & Notre-Dame \\
    382 & \hphantom{0}2 & & Świętym$^{ 236 }$” & Świętym”$^{ 236 }$ \\
    382 & & 12 & Pana$^{ 237 }$” & Pana”$^{ 237 }$ \\
    382 & \hphantom{0}2 & & Świętym$^{ 236 }$” & Świętym”$^{ 236 }$ \\
    382 & & 12 & Pana$^{ 237 }$” & Pana”$^{ 237 }$ \\
    383 & 12 & & płaszczyźnie$^{ 240 }$” & płaszczyźnie”$^{ 240 }$ \\
    385 & & \hphantom{0}8 & bezpośredni$^{ 243 }$ & bezpośredni \\
    385 & & \hphantom{0}7 & Dogmatów” & Dogmatów”$^{ 243 }$ \\
    388 & \hphantom{0}8 & & katolika$^{ 245 }$” & katolika”$^{ 245 }$ \\
    394 & \hphantom{0}6 & & 190$^{ 251 }$) & 190)$^{ 251 }$ \\
    394 & & \hphantom{0}1 & stosowany\ldots & stosowany. \\
    % 395?
    406 & 12 & & przekonany”\ldots & przekonany”. \\
    408 & \hphantom{0}3 & & wolę$^{ 261 }$” & wolę”$^{ 261 }$ \\
    413 & & \hphantom{0}9 & prawdy$^{ 263 }$” & prawdy”$^{ 263 }$ \\
    415 & & \hphantom{0}8 & miejscu$^{ 266 }$ & miejscu” \\
    415 & & \hphantom{0}8 & 610).” & 610). \\
    % 423
    425 & \hphantom{0}5 & & człowieka & człowieka” \\
    % 425?????
    442 & & \hphantom{0}9 & kocha$^{ 280 }$” & kocha”$^{ 280 }$ \\
    % 443??????
    451 & & 14 & widzialne$^{ 283 }$” & widzialne”$^{ 283 }$ \\
    475 & 17 & & \textit{bezpodstawne}$^{ 295 }$”
           & \textit{bezpodstawne}”$^{ 295 }$ \\
           %
    495 & & 11 & człowiek & człowiek” \\
    % 495??????
    505 & & \hphantom{0}6 & \textit{Osservatore} & \textit{L'Osservatore} \\
    517 & & 10 & szkodę\ldots & szkodę \\
    % 533??????
    % & & & & \\
    % & & & & \\
    % & & & & \\
    % & & & & \\
    % & & & & \\
    % & & & & \\
    % & & & & \\
    \hline
  \end{tabular}

\end{center}

\VerSpaceTwo


\noindent
\StrWierszGora{460}{14} \\
\Jest czyniąc z~nadziei córkę wiary \\
\PowinnoByc czyniąc z~wiary córkę nadziei \\

% ############################










% ######################################
\section{ % Autor i tytuł dzieła
  Dietrich von~Hildebrand \\
  \textit{Koń trojański w~Mieście Boga},
  \parencite{HildebrandKonTrojanski2006}}

% ######################################


% ##################
\CenterBoldFont{Uwagi do konkretnych stron}


\noindent
\StrWierszDol{80}{5--4} Czy chodzi tu o~to, że~po ogołoceniu z~religii życie
usycha? Czy też, że~po pozbawieniu świętości usycha religia?

\VerSpaceFour





\noindent
\Str{104} Nie wiem czy dało~się to zrobić lepiej, ale~strona na~której
znajduje~się tylko treść jednego przypisu, nie~wygląda dobrze.

\VerSpaceFour





\noindent
\Str{140} „Jednakże prawda o~istnieniu Boga powinna to panowanie zdobyć,
a~gdy je~zdobędzie, wówczas jej realność społeczna uzyskuje zupełnie nowych
charakter.” Hildebrandowi chodziło chyba o~to, że~jeśli prawda o~istnieniu
Boga zapanuje w~sferze międzyludzkiej, wówczas to społeczność uzyskuje
zupełnie nową jakość bycia.

\VerSpaceFour





\noindent
\Str{187} Nie potrafię zrozumieć jaki sens miał mieć fragment: „Liczni
spośród tych, co~przenoszą liturgię nad modlitwę prywatną, ponieważ
ta~ostatnia nie sprzyja rzekomo wspólnocie między ludźmi, sprawiają
wrażenie, że~stracili z~pola widzenia ten głęboko wspólnotowy aspekt
modlitwy liturgicznej”.





% ##################
\VerSpaceTwo


\CenterBoldFont{Błędy}


\begin{center}

  \begin{tabular}{|c|c|c|c|c|}
    \hline
    Strona & \multicolumn{2}{c|}{Wiersz} & Jest
                              & Powinno być \\ \cline{2-3}
    & Od góry & Od dołu & & \\
    \hline
    \hphantom{0}62 & & \hphantom{0}4 & Kościół & i~Kościół \\
    \hphantom{0}78 & \hphantom{0}4 & & \textit{ludzie} z~\textit{zewnątrz}
           & \textit{ludzie z~zewnątrz} \\
    \hphantom{0}85 & & \hphantom{0}1 & Chicago1977 & Chicago 1977 \\
    106 &  4 & & Tematyczność prawdy$^{ 42 }$
           & Tematyczność$^{ 42 }$ prawdy \\
    224 &  2 & & \textit{]a} & \textit{Ja} \\
    231 & &  7 & \textit{jak} & jak \\
    245 &  8 & & \textit{wiary}~w & \textit{wiary~w} \\
    257 & 11 & & Teil-harda & Teilharda \\
    294 & & 10 & św.Piotra & św.~Piotra \\
    304 &  6 & & pastorów & proboszczów \\
    374 & &  5 & nowi \textit{moraliści} & \textit{nowi moraliści} \\
    379 &  8 & & Bóg jest & \textit{Bóg jest} \\
    \hline
  \end{tabular}

\end{center}

\VerSpaceTwo


\noindent
\StrWierszGora{241}{2} \\
\Jest Hildebrand,\textit{Ethics},Franciscan \\
\PowinnoByc Hildebrand, \textit{Ethics}, Franciscan \\
\StrWierszDol{297}{4} \\
\Jest żywotności,niezależnej od~przypadków,jakie \\
\PowinnoByc żywotności, niezależnej od~przypadków, jakie \\
\StrWierszDol{297}{3} \\
\Jest świecie,przed \\
\PowinnoByc świecie, przed \\

% ######################################










% ######################################
\newpage

\section{Dietrich von~Hildebrand \textit{Spustoszona
    winnica}, \cite{HildebrandSpustoszonaWinnica2006}}

% ######################################


% ##################
\CenterBoldFont{Uwagi do konkretnych stron}


\noindent
\StrWierszDol{99}{7} Zdanie „Wytrwanie w~miłości do~innego człowieka
w~stanie
wiecznej niedojrzałości”, nie jest najprostsze do~zrozumienia. Możliwe,
że~powinno ono brzmieć „Wytrwanie w~miłości do~innego człowieka, będącego
w~stanie wiecznej niedojrzałości”.

\VerSpaceFour





\noindent
\Str{114} Stwierdzenie „o~ile Stary Testament uznaje~się za~prawdziwe
objawienie Boga” bardzo odstaje od~treści tej książki. Hildebrand ciężko
posądzać o~podważanie tego, iż~Stary Testament jest objawieniem Boga, a~to
zdanie sprawia wrażenie, że~jest to dopuszczalne. Prawdopodobnie miało ono
brzmieć „o~ile Stary Testament uznają za~prawdziwe objawienie Boga”,
są~bowiem żydzi, którzy odrzucają Boga oraz~Stary Testament.

\VerSpaceFour





\noindent
\Str{152} Zdanie „Dochodzi do~tego jeszcze jeden problem: hasło
\textit{totus homo} bynajmniej nie znaczy, iż~pozycja Chrystusa~--
Jego posłannictwo, Jego święte kapłaństwo, Jego święty urząd
nauczycielki, Jego charakter jako Króla królów --~odróżnia Go w~sposób
szczególny od~wszystkich ludzi, którzy są tylko ludźmi i~wynosi
Go~ponad nich.” brzmi bardzo dziwnie w~zestawieniu z~resztą książki.
Zapewne został tu~popełniony jakiś błąd.

\VerSpaceFour





\noindent
\Str{177} Użycie słów „odnajdywanie” oraz~„znajdowanie” w~prowadzonej
analizie, jest odrobinę mylące i~należałoby je zmienić w~taki sposób,
by~stała~się ona jaśniejsza.

\VerSpaceFour





\noindent
\Str{216} Paragraf poświęcony relacji jednostki i~wspólnoty jest napisany
w~dziwny sposób, czyniący go dość niejednoznacznym. Możliwie, że~to wynik
pomyłki tłumacza.

\VerSpaceFour





\noindent
\Str{216} Możliwe, że~zamiast „w~połączeniu w~jedną jedyną substancję”
powinno być „połączonych w~jedną jedyną substancję”. Nie potrafię jednak
rozstrzygnąć, która wersja jest poprawna.

% \vspace{\spaceFour}





% ##################
\newpage

\CenterBoldFont{Błędy}


\begin{center}

  \begin{tabular}{|c|c|c|c|c|}
    \hline
    Strona & \multicolumn{2}{c|}{Wiersz} & Jest
                              & Powinno być \\ \cline{2-3}
    & Od góry & Od dołu & & \\
    \hline
    \hphantom{00}5 & \hphantom{0}2 & & w & \textit{w} \\
    % 71 & & 3 & 24a.25 & 24 a. 25 \\ ???
    \hphantom{0}81 & & \hphantom{0}1 & CLXIX. & CLXIX, \\
    125 & \hphantom{0}3 & & z & \textit{z} \\
    129 & & 11 & L'\textit{Osservatore} & \textit{L'Osservatore} \\
    132 & & \hphantom{0}1 & 4,20) & 4,20). \\
    137 & & \hphantom{0}6 & m.\hspace{1em} in. & m.~in.\\
    271 & \hphantom{0}6 & & et & \textit{et} \\
    273 & & \hphantom{0}5 & zwraca & zwracamy \\
    282 & & \hphantom{0}5 & wydaje~się zmian & zmian wydaje~się \\
    283 & & \hphantom{0}8 & Bądźcie & „Bądźcie \\
    % & & & & \\
    \hline
  \end{tabular}

\end{center}

\VerSpaceTwo


\noindent
\textbf{Grzbiet.} \\
\Jest Hildebrand„Spustoszona \\
\PowinnoByc Hildebrand „Spustoszona \\
\StrWierszDol{215}{3} \\
\Jest  Tegoż,\textit{Journuals},TorchBooks/Harper{\&}Row,NewYork(b.d.w.),s.187. \\
\PowinnoByc Tegoż, \textit{Journuals}, Torch Books/Harper \& Row, New York
(b.d.w.), s. 187. \\

% ############################










% ######################################
\newpage

\section{Święta wiara i~filozofia~-- dzieła podejrzane}

\VerSpaceTwo
% ######################################



% ############################
\subsection{Refleksje nad świętą wiarą po 1945~r.}

\VerSpaceThree
% ############################



% ######################################
\section{Jacques Maritain \textit{Wieśniak znad Garonny.
    Stary świecki chrześcijanin snuje
    refleksje} \`{a}~propos \textit{czasów współczesnych},
  \cite{MaritainWiesniakZnadGaronny2017}}

% ######################################


% ##################
\CenterBoldFont{Błędy}


\begin{center}

  \begin{tabular}{|c|c|c|c|c|}
    \hline
    Strona & \multicolumn{2}{c|}{Wiersz} & Jest
                              & Powinno być \\ \cline{2-3}
    & Od góry & Od dołu & & \\
    \hline
    22  & &  3 & słowa. & słowa). \\
    % & & & & \\
    % & & & & \\
    % & & & & \\
    % & & & & \\
    % & & & & \\
    % & & & & \\
    % & & & & \\
    % & & & & \\
    % & & & & \\
    \hline
  \end{tabular}

\end{center}

% \vspace{\spaceTwo}


% \noindent
% \StrWg{241}{2} \\
% \Jest  \\
% \Powin \\

% ############################










% ######################################
\newpage

\section{Katolickie praktyki i~zwyczaje}

\VerSpaceTwo
% ######################################



% ######################################
\section{Scott Hahn \textit{Znaki życia}, \parencite{}}

% ######################################


% ##################
\CenterBoldFont{Błędy}


Str. 75. \ldots aby obyć się bez dokładki.


\VerSpaceOne
% ############################










% ######################################
\newpage

\section{Biografie, autobiografie i~inne działa poświęcone
  różnym postaciom ważnym dla świętej wiary}

\VerSpaceTwo
% ######################################



% ############################
\subsection{XIX i~XX wiek po Chrystusie}

\VerSpaceThree
% ############################



% ######################################
\section{Dale Ahlquist \textit{Apostoł zdrowego rozsądku},
  \parencite{}}

% ######################################


% ##################
\CenterBoldFont{Błędy}


Str. 124. Nie można było zagrozić zagłodzeniem\ldots


\VerSpaceTwo
% ############################










% ############################
\subsection{XX i~XXI wiek po Chrystusie}

\VerSpaceThree
% ############################



% ######################################
\section{Scott i~Kimberly Hahn \textit{Rome sweet home},
  \parencite{}}

% ######################################


% ##################
\CenterBoldFont{Błędy}


Str. 219. \ldots przyciągnąć. -- Chociaż pielęgniarki\ldots


\VerSpaceTwo
% ############################










% ######################################
\newpage

\section{Dzieła na temat świętej wiary, wątpliwej jakości}

\VerSpaceTwo
% ######################################





% ######################################
\section{Adrienne von Speyr \textit{Trzy kobiety i~Pan},
  \cite{SpeyrTrzyKobietyIPan1998}}

% ######################################

% \vspace{0em}


% % ##################
% \CenterBoldFont{Uwagi do konkretnych stron}

% \vspace{0em}







% ##################
\CenterBoldFont{Błędy}

\VerSpaceFive


\begin{center}

  \begin{tabular}{|c|c|c|c|c|}
    \hline
    Strona & \multicolumn{2}{c|}{Wiersz} & Jest
                              & Powinno być \\ \cline{2-3}
    & Od góry & Od dołu & & \\
    \hline
    % & & & & \\
    % & & & & \\
    89 & &  1 & 1998$^{ 3 }$ & 1998 \\
    \hline
  \end{tabular}

\end{center}

\VerSpaceTwo


% ############################










% ####################################################################
% ####################################################################
% Bibliography

\printbibliography





% ############################

% Koniec dokumentu
\end{document}
