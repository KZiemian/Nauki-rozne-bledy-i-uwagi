% ------------------------------------------------------------------------------------------------------------------
% Basic configuration and packages
% ------------------------------------------------------------------------------------------------------------------
% Package for discovering wrong and outdated usage of LaTeX.
% More information to be found in l2tabu English version.
\RequirePackage[l2tabu, orthodox]{nag}
% Class of LaTeX document: {size of paper, size of font}[document class]
\documentclass[a4paper,11pt]{article}



% ------------------------------------------------------
% Packages not tied to particular normal language
% ------------------------------------------------------
% This package should improved spaces in the text
\usepackage{microtype}
% Add few important symbols, like text Celcius degree
\usepackage{textcomp}



% ------------------------------------------------------
% Polonization of LaTeX document
% ------------------------------------------------------
% Basic polonization of the text
\usepackage[MeX]{polski}
% Switching on UTF-8 encoding
\usepackage[utf8]{inputenc}
% Adding font Latin Modern
\usepackage{lmodern}
% Package is need for fonts Latin Modern
\usepackage[T1]{fontenc}



% ------------------------------------------------------
% Setting margins
% ------------------------------------------------------
\usepackage[a4paper, total={14cm, 25cm}]{geometry}



% ------------------------------------------------------
% Setting vertical spaces in the text
% ------------------------------------------------------
% Setting space between lines
\renewcommand{\baselinestretch}{1.1}

% Setting space between lines in tables
\renewcommand{\arraystretch}{1.4}



% ------------------------------------------------------
% Packages for scientific papers
% ------------------------------------------------------
% Switching off \lll symbol, that I guess is representing letter "Ł"
% It collide with `amsmath' package's command with the same name
\let\lll\undefined
% Basic package from American Mathematical Society (AMS)
\usepackage[intlimits]{amsmath}
% Equations are numbered separately in every section
\numberwithin{equation}{section}

% Other very useful packages from AMS
\usepackage{amsfonts}
\usepackage{amssymb}
\usepackage{amscd}
\usepackage{amsthm}

% Package with better looking calligraphy fonts
\usepackage{calrsfs}

% Package with better looking greek letters
% Example of use: pi -> \uppi
\usepackage{upgreek}
% Improving look of lambda letter
\let\oldlambda\Lambda
\renewcommand{\lambda}{\uplambda}




% ------------------------------------------------------
% BibLaTeX
% ------------------------------------------------------
% Package biblatex, with biber as its backend, allow us to handle
% bibliography entries that use Unicode symbols outside ASCII
\usepackage[
language=polish,
backend=biber,
style=alphabetic,
url=false,
eprint=true,
]{biblatex}

\addbibresource{LogikaITeoriaMnogosciBibliography.bib}





% ------------------------------------------------------
% Defining new environments (?)
% ------------------------------------------------------
% Defining enviroment "Wniosek"
\newtheorem{corollary}{Wniosek}
\newtheorem{definition}{Definicja}
\newtheorem{theorem}{Twierdzenie}





% ------------------------------------------------------
% Private packages
% You need to put them in the same directory as .tex file
% ------------------------------------------------------
% Contains various command useful for working with a text
\usepackage{latexgeneralcommands}
% Contains definitions useful for working with mathematical text
\usepackage{mathcommands}





% ------------------------------------------------------
% Package "hyperref"
% They advised to put it on the end of preambule
% ------------------------------------------------------
% It allows you to use hyperlinks in the text
\usepackage{hyperref}










% ------------------------------------------------------------------------------------------------------------------
% Title and author of the text
\title{DEUS: dzieła świętych i~błogosławionych \\
  {\Large Błędy i~uwagi}}

\author{Kamil Ziemian}


% \date{}
% ------------------------------------------------------------------------------------------------------------------










% ####################################################################
% Początek dokumentu
\begin{document}
% ####################################################################





% ######################################
\maketitle
% ######################################





% ######################################
\section{Sanctae Jan Chryzostom
  \textit{Homilie na~Księgę Rodzaju (seria pierwsza: Rdz 1--3)},
  \parencite{SancteJanChryzostomHomKsiegaRodzaju2008}}

% ######################################


% ##################
\CenterBoldFont{Uwagi}

\vspace{0em}


\noindent
\Str{78} Na~końcu pierwszego akapitu jest jeden nieotwarty
cudzysłów, nie wiadomo więc, gdzie~się powinien zaczynać, a~gdzie
kończyć.

\VerSpaceFour





\noindent
\Str{106} Od końca pierwszego akapitu do~końca tej homilii,
tekst staję~się chaotyczny i~w~pewnym stopniu nielogiczny.
Należałoby~by sprawdzić, czy~został poprawnie przetłumaczony.





% ##################
\CenterBoldFont{Błędy}

\VerSpaceFive


\begin{center}

  \begin{tabular}{|c|c|c|c|c|}
    \hline
    Strona & \multicolumn{2}{c|}{Wiersz} & Jest
                              & Powinno być \\ \cline{2-3}
    & Od góry & Od dołu & & \\
    \hline
    73 & & 11 & w pokoju & pokoju \\
    % & & & & \\
    % & & & & \\
    \hline
  \end{tabular}

\end{center}

\VerSpaceTwo


% ######################################










% ######################################
\section{Sanctae Jak Klimak \textit{Drabina raju},
  \parencite{SancteJanKlimakDrabinaRaju2011}}

% ######################################


% ##################
\CenterBoldFont{Uwagi ogólne}

\vspace{0em}


\noindent
W~tym wydaniu przyjęto dziwną regułę, że~liczba odsyłająca do~przypisu nie
jest zaraz przy wyrazie, lecz znajduje~się między nimi odstęp. Na~przykład
na stronie~101 jest „głowo $^{ 1 }$”, a~nie „głowo$^{ 1 }$”. Zdecydowanie
wolę tę drugą konwencję.



\VerSpaceTwo


% ######################################










% ######################################
\section{Sanctae Augustyn z~Hippony
  \textit{Państw Boże}, \cite{}}
% ######################################


% ##################
\CenterBoldFont{Uwagi}


% ##################
\CenterBoldFont{Błędy}

% \vspace{\spaceFive}


\begin{center}

  \begin{tabular}{|c|c|c|c|c|}
    \hline
    Strona & \multicolumn{2}{c|}{Wiersz} & Jest
                              & Powinno być \\ \cline{2-3}
    & Od góry & Od dołu & & \\
    \hline
    \hphantom{0}51 & \hphantom{0}8 & & zostawia” & zostawia \\
    \hphantom{0}92 & 16 & & Od & „Od \\
    122 & \hphantom{0}9 & & punicka. & punicka.” \\
    % & & & & \\
    % & & & & \\
    % & & & & \\
    \hline
  \end{tabular}

\end{center}

\VerSpaceTwo
% ######################################










% ############################
\section{Sanctae Tomasz z Akwinu\textit{Suma Teologiczna.
    Tom~I},
  \parencite{SancteTomaszZAkwinuSumaTeologicznaTomI1960}}

% ######################################


% ##################
% \CenterBoldFont{Uwagi ogólne}





% ##################
\newpage

\CenterBoldFont{Błędy}


\begin{center}

  \begin{tabular}{|c|c|c|c|c|}
    \hline
    Strona & \multicolumn{2}{c|}{Wiersz} & Jest
    & Powinno być \\ \cline{2-3}
    & Od góry & Od dołu & & \\
    \hline
    28 & \hphantom{0}3 & & ( 1 & (1 \\
    31 & & \hphantom{0}7 & \textit{WIEDZĄ~?} & \textit{WIEDZĄ?} \\
    32 & & 13 & \textit{WIEDZĄ~?} & \textit{WIEDZĄ?} \\
    32 & & \hphantom{0}3 & bo-wiem & bowiem \\
    32 & & \hphantom{0}2 & [Mądrość]dała & [Mądrość] dała \\
    33 & 12 & & n~i~ż~s~z~e~~władze & n~i~ż~s~z~e\, władze \\
    33 & 14 & & n~a~d~r~z~ę~d~n~e~j~~władzy
           & n~a~d~r~z~ę~d~n~e~j\, władzy \\
    33 & & 11 & \textit{PRAKTYCZNĄ~?} & \textit{PRAKTYCZNĄ?} \\
    34 & & \hphantom{0}1 & za & ta \\
    35 & 13 & & ludzkiego”. & ludzkiego. \\
    35 & & \hphantom{0}9 & \textit{MĄDROŚCIĄ~?} & \textit{MĄDROŚCIĄ?} \\
    36 & \hphantom{0}3 & & jest & nie jest \\
    37 & 19 & & \textit{BÓG~?} & \textit{BÓG?} \\
    37 & & 14 & jest$^{ 2 }$ & jest''$^{ 2 }$ \\
    37 & \hphantom{0}9 & & n & na \\
    38 & & 15 & \textit{UZASADNIAĆ~?} & \textit{UZASADNIAĆ~?} \\
    38 & & 13 & Gdzie & „Gdzie \\
    38 & & \hphantom{0}3 & biskupie & o~biskupie \\
    39 & & \hphantom{0}1 & teologii~! & teologii! \\
    40 & & 16 & \textit{PRZENOŚNI~?} & \textit{PRZENOŚNI?} \\
    41 & 15 & & niewykształconych”$^{ 5 }$
           & niewykształconych”$^{ 5 }$) \\
    41 & 20 & & i~~d~l~a & i{}\, d~l~a \\
    42 & \hphantom{0}5 & & \textit{ZNACZEŃ~?} & \textit{ZNACZEŃ?} \\
    44 & \hphantom{0}1 & & \textit{JEST~?} & \textit{JEST~?} \\
    44 & 17 & & \textit{SIEBIE~?} & \textit{SIEBIE?} \\
    45 & & \hphantom{0}6 & pond & ponad \\
    46 & 10 & & \textit{BOGA~?} & \textit{BOGA?} \\
    47 & & 13 & \textit{ISTNIEJE~?} & \textit{ISTNIEJ?} \\
    48 & & 16 & druga & Druga \\
    \hline
  \end{tabular}





  \newpage

  \begin{tabular}{|c|c|c|c|c|}
    \hline
    Strona & \multicolumn{2}{c|}{Wiersz} & Jest
                              & Powinno być \\ \cline{2-3}
    & Od góry & Od dołu & & \\
    \hline
    49 & & 14 & poznania,{ }, & poznania, \\
    60 & & 16 & zak & tak \\
    % & & & & \\
    % & & & & \\
    % & & & & \\
    % & & & & \\
    % & & & & \\
    \hline
  \end{tabular}

\end{center}

\VerSpaceTwo



% ######################################










% ######################################
\section{Sanctae Tomasz z Akwinu
  \textit{O~królowaniu~-- królowi Cypru},
  \parencite{SancteTomaszZAkwinuOKrolowaniu2006}}

% ######################################


% ##################
\CenterBoldFont{Uwagi ogólne}

\vspace{0em}


\noindent
Marginesy od~strony brzegu~są zbyt małe, co często utrudnia czytanie.

% \vspace{\spaceFour}





% ##################
\CenterBoldFont{Uwagi do~konkretnych stron}

\vspace{0em}


\noindent
\StrWierszGora{20}{16} Słowa św.~Grzegorza~są tu~przytaczane jako
\textit{Czymże jest władza na~szczycie, jeśli nie burzą umysłu?},
podczas gdy~w~paragrafie~10.5 na~stronie 91 brzmią one \textit{Czymże
  jest burza na~morzu, jeśli nie burzą umysłu?}. Odnośnie tego
problemu warto zobaczyć komentarze tłumacza na stronach 224--225.

\VerSpaceFour





\noindent
\StrWierszGora{32}{2} Obecność wcięcia w~tym wierszu, jest zapewne błędem
składu.

\VerSpaceFour





\noindent
\StrWierszGora{49}{1} Zdanie „Dalej: rzeczy zgodne z~naturą mają
w~sobie doskonałość, bowiem natura posługuje~się jednostkami --~i~tak
jest najlepiej” powinno według mnie brzmieć raczej „Rzeczy zgodne
z~naturą mają w~sobie doskonałość: natura posługuje~się jednostkami
i~tak jest najlepiej”. Jednak bez~znajomości łaciny, nie da~się tego
problemu rozstrzyganą w~sposób merytoryczny. To~samo odnosi~się
do~wszystkich następnych uwaga o~sposobie tłumaczenie tekstu, chyba
że~powiedziano inaczej.

\VerSpaceFour





\noindent
\Str{63} Treść tej~strony sprawia dużo problemów. Po~pierwsze
w~punkcie~6.2 jest mowa o~„dobre tyranii”, a~zgodnie z prowadzoną
klasyfikacją, nie może być czegoś takiego jak tyrania, która jest
dobra. Po~drugie, w~tym samym punkcie jest mowa, że~„dobra tyrania”
nie unicestwia pokoju jaki panuje w~społeczności. Jednak w~punkcie
4.9~na~stronie~55, pisze św.~Tomasz, że~tyrani dla zachowania władzy
zasiewają między poddanymi niezgodę, a~tą która już istnieje
podsycają.

Jak wspomina tłumacz we~„Wstępie”, św.~Tomasz nie ukończył, a~tym
bardziej nie poprawił dzieła, stąd błędy te mogą być tego wynikiem.
Inna możliwość jest taka, że~św.~Tomasz wyłożył tu~błędną, wzajemnie
sprzeczną doktrynę. W~tej sprawie zobacz również uwagi tłumacza
na~stronie~180.

\VerSpaceFour





\noindent
\StrWierszDol{89}{3} Wydaj mi~się, że~zamiast „jeśli jest cnotliwe”
powinno być „jeśli jest własnością cnoty”.

\VerSpaceFour





\noindent
\StrWierszDol{91}{7} Popierając~się pobieżną i~niefachową analizą tekstu
łacińskiego, doszedłem do~wniosku, że~zamiast „ten sam porzuca zwyczaj
czynienia dobrze” powinno być bardziej sensowne w~tym kontekście zdanie
„ten sam porzuca zwyczaj czynienia dobrze w~czasie zamętu”.

\VerSpaceFour





\noindent
\StrWierszDol{101}{13} Jestem słaby z~interpunkcji, mimo tego wydaje
mi~się, że~zamiast „a~kiedy już potrzeba więcej, oni dają królom
z~własnej woli” powinno być „a~kiedy już potrzeba, więcej oni dają
królom z~własnej woli”.

\VerSpaceFour





\noindent
\StrWierszGora{103}{5} Zwykle w~tym wydaniu cytaty łacińskie~są
albo~przytaczane po łacinie, albo tłumaczone na~polski. W~tej linii jednak,
cytat jest przytoczony do~połowy po~łacinie, dalej jest tłumaczenie całości
na~polski.

\VerSpaceFour





\noindent
\StrWierszGora{107}{6} Sytuacja taka sama jak na stronie~103, z~tym,
że~teraz po~łacinie przytoczona jest~druga część cytatu.

\VerSpaceFour





\noindent
\textbf{Str.~192, wiersze 15, 14 (od~dołu).} Odstęp między tymi
liniami jest za~duży.

\VerSpaceFour





\noindent
\StrWierszGora{260}{15 i~33} Pod~tymi liniami nie powinien znajdować~się
odstęp.

\VerSpaceFour





\noindent
\StrWierszGora{267}{10 i~31} Pod~tymi liniami nie powinien znajdować~się
odstęp.

\VerSpaceFour





\noindent
\Str{309} Począwszy od~następującej strony, brak jest numeracji kolejnych
stron.

\VerSpaceFour





\noindent
\StrWierszGora{312}{16} Po~tej linii w~tekście powinien znajdować~się
odstęp.

\VerSpaceFour





% ##################
\newpage

\CenterBoldFont{Błędy}


\begin{center}

  \begin{tabular}{|c|c|c|c|c|}
    \hline
    Strona & \multicolumn{2}{c|}{Wiersz} & Jest
                              & Powinno być \\ \cline{2-3}
    & Od góry & Od dołu & & \\
    \hline
    \hphantom{0}13 & \hphantom{0}6 & & panowala & pawnowała \\
    \hphantom{0}23 & & 11 & sądzić & wierzyć \\
    \hphantom{0}24 & \hphantom{0}8 & & tu & tu to \\
    \hphantom{0}29 & & 12 & zawartą & zawartej \\
    \hphantom{0}90 & & 10 & impune & \textit{impune} \\
    \hphantom{0}91 & & \hphantom{0}1 & impune & \textit{impune} \\
    109 & & 11 & boskiego) & boskiego \\
    111 & & \hphantom{0}1 & wszyst & wszyst- \\
    129 & & 11 & Policraticus & \textit{Policraticus} \\
    130 & & 19 & \textit{ypocrita} & \textit{hypocrita} \\
    146 & \hphantom{0}5 & & \textit{politeia} & politeia \\
    146 & \hphantom{0}6 & & \textit{Politeia}& Politeia \\
    146 & & \hphantom{0}9 & \textit{politei} & politei \\
    162 & 11 & & \textit{z~zasady} & z~zasady \\
    163 & & \hphantom{0}5 & tyranię & w~tyranię \\
    187 & \hphantom{0}3 & & rozdział , & rozdział, \\
    189 & & 17 & \textit{Twarda} & Twarda \\
    209 & & \hphantom{0}9 & civitates & \textit{civitates} \\
    209 & & \hphantom{0}6 & - \textit{Ibidem.} & \textit{Ibidem.} \\
    240 & & 17 & \textit{ypocrita} & \textit{hypocrita} \\
    240 & & \hphantom{0}9 & \textit{ypocrita} & \textit{hypocrita} \\
    241 & & \hphantom{0}1 & DeMalo & \textit{De malo} \\
    241 & & \hphantom{0}3 & \textit{DeMalo} & \textit{De malo} \\
    244 & 12 & & zasłuży. & zasłuży, \\
    268 & & \hphantom{0}2 & \ldots InSent & InSent \\
    286 & 12 & & ma~bowiem był~on & jest \\
    315 & & \hphantom{0}2 & Dzieje Polski w~zarysie
    & \textit{Dzieje Polski w~zarysie} \\
    317 & 12 & & Własność & własność \\
    317 & 12 & & Podatki & podatki \\
    \hline
  \end{tabular}





  \newpage

  \begin{tabular}{|c|c|c|c|c|}
    \hline
    Strona & \multicolumn{2}{c|}{Wiersz} & Jest
                              & Powinno być \\ \cline{2-3}
    & Od góry & Od dołu & & \\
    \hline
    317 & 13 & & Gospodarcza & gospodarcza \\
    317 & 13 & & Korupcja & korupcja \\
    317 & 14 & & Granice & granice \\
    % & & & & \\
    % & & & & \\
    \hline
  \end{tabular}

\end{center}

\VerSpaceTwo



% ######################################










% ######################################
\section{Sancte Franciszek Salezy \textit{Filotea},
  \parencite{SancteFranciszekSalezyFilotea}}

% ######################################

\vspace{0em}


% ##################
\CenterBoldFont{Błędy}

% \vspace{\spaceFive}


\begin{center}

  \begin{tabular}{|c|c|c|c|c|}
    \hline
    Strona & \multicolumn{2}{c|}{Wiersz} & Jest
                              & Powinno być \\ \cline{2-3}
    & Od góry & Od dołu & & \\
    \hline
    \hphantom{0}10 & \hphantom{0}8 & & prowadzićna & prowadzić na \\
    \hphantom{0}39 & \hphantom{0}3 & & \textit{we mnie jest}
    & \textit{we mnie} \\
    \hphantom{0}39 & & \hphantom{0}7 & rzeź- wić & rzeźwić \\
    \hphantom{0}40 & 10 & & szystkim & wszystkim \\
    \hphantom{0}45 & & \hphantom{0}7 & twojemu & ich \\
    \hphantom{0}56 & & \hphantom{0}9 & Ojczyznę.O, & Ojczyznę. O, \\
    \hphantom{0}59 & & 11 & światowych. & światowych.” \\
    185 & & 12 & chciwy1 & chciwy$^{ 1 }$ \\
    263 & & \hphantom{0}4 & ŚwiętyTomasz & Święty Tomasz \\
    % & & & & \\
    % & & & & \\
    % & & & & \\
    \hline
  \end{tabular}

\end{center}

\VerSpaceTwo


% ######################################










% ######################################
\section{Bł. John Henry Newman \textit{Apologia pro vita
    sua}, \parencite{NewmanApologia2009}}

% ######################################


% ##################
\CenterBoldFont{Uwagi do~konkretnych stron}

\vspace{0em}


\noindent
\StrWierszDol{37}{11} Kiedy ten sam fragment jest cytowany na~stronie~33
święty Alfons de~Liguori jest tam określany jako święty, tutaj zaś~jako
błogosławiony. Ponieważ jego kanonizacja odbyła~się w~1849 roku, zapewne
w~oryginalnej przytaczanego tekstu jest on określany jako święty.

\VerSpaceFour





\noindent
\StrWierszDol{49}{1} Przez „straszydło na~wróble” to~pewnie tłumaczenie
angielskiego „straw man”. Jeśli tak należy pamiętać, że~w~języku
angielskim funkcjonuje „straw man argument”, choć nie wiem, czy był już
wtedy znany pod tą~nazwą.

\VerSpaceFour





\noindent
\Str{127} Nie jestem w~stanie stwierdzić, co miało znaczyć zdanie „I~przez
tak długi czas nie było możliwe złamanie go w~tym stanie rzeczy, gdyby
wielka zmiana nie zaszła w~warunkach ruchu przeciwnego, który już~się
rozpoczął, aby~mu~stawić opór.”

\VerSpaceFour





\noindent
\StrWierszDol{177}{6} Jeśli Churton zmarł w~1792 roku, w~jaki sposób mógł
być zwolennikiem ruchu oksfordzkiego, który powstał w~XIX~wieku?
Tu~musi być jakiś błąd.

\VerSpaceFour





% ##################
\newpage

\CenterBoldFont{Błędy}

\VerSpaceFive


\begin{center}

  \begin{tabular}{|c|c|c|c|c|}
    \hline
    Strona & \multicolumn{2}{c|}{Wiersz} & Jest
                              & Powinno być \\ \cline{2-3}
    & Od góry & Od dołu & & \\
    \hline
    \hphantom{0}35 & 12 & & \textit{Części} & części \\
    \hphantom{0}47 & \hphantom{0}3 & & skondensowana & skondensowania \\
    \hphantom{0}57 & \hphantom{0}7 & & tej & tych \\
    \hphantom{0}61 & \hphantom{0}7 & & starta & wytarta \\
    \hphantom{0}88 & & \hphantom{0}8 & pomagającymi & wspomagającymi \\
    117 & 13 & & 3) A~dalej & A~dalej \\
    119 & & 12 & 4) & 3) \\
    122 & & 11 & świat?” & świat? \\
    122 & & 11 & „Ten & >>Ten \\
    122 & & 10 & godzien” & godzien<<  % >>
    \\
    122 & & 10 & „nauczyć & >>nauczyć \\
    122 & & \hphantom{0}9 & wyrok” & wyrok<<  % >>
    \\
    130 & & \hphantom{0}1 & ukazały, dopóty & ukazały, \\
    134 & 15 & & działo~się & było \\
    139 & & 10 & „Mądrości Bożej” & >>Mądrości Bożej<<”  % >>
    \\
    145 & \hphantom{0}1 & & fakt; & fakt \\
    160 & & 11 & co~innego & coś~innego \\
    160 & & \hphantom{0}9 & znajdziemy & znajdziemy tam \\
    167 & \hphantom{0}2 & & ustępowałem; & ustępowałem, \\
    174 & 12 & & uważa & uważam \\
    176 & \hphantom{0}3 & & praktycznego”\ldots „Bogate
    & praktycznego”, „Bogate \\
    176 & & \hphantom{0}5 & przeciwnik & czynnik \\
    178 & & \hphantom{0}8 & tak & tak~to \\
    182 & \hphantom{0}9 & & intuicje & intuicje odnośnie \\
    182 & 13 & & znosić & zaprzeczać \\
    182 & & 11 & przeciwieństwo & przeciwieństwa \\
    182 & & \hphantom{0}1 & „tak” & >>tak<<  % >>
    \\
    182 & & \hphantom{0}1 & „nie” & >>nie<<”  % >>
    \\
    183 & 10 & & życzący & nie życzący \\
    \hline
  \end{tabular}





  \newpage

  \begin{tabular}{|c|c|c|c|c|}
    \hline
    Strona & \multicolumn{2}{c|}{Wiersz} & Jest
                              & Powinno być \\ \cline{2-3}
    & Od góry & Od dołu & & \\
    \hline
    193 & & 11 & dzieciństwem” & dzieciństwem \\
    194 & & \hphantom{0}8 & punkt & argument \\
    194 & & \hphantom{0}7 & punktem & argumentem \\
    194 & & \hphantom{0}7 & punktem & argumentem \\
    195 & \hphantom{0}3 & & pewien punkt & pewne rzeczy \\
    199 & 11 & & 1156 & 1556 \\
    203 & & \hphantom{0}2 & stale & stałe \\
    221 & & \hphantom{0}5 & puseyizmu”. & puseyizmu. \\
    224 & & 12 & wnioskiem. & wnioskiem”. \\
    227 & & 11 & Liście & liście \\
    250 & \hphantom{0}5 & & nieistniejącego & istniejącego \\
    254 & \hphantom{0}9 & & więzieniu. & więzieniu.” \\
    254 & 10 & & Boże & „Boże \\
    254 & 10 & & „Śniłem & Śniłem \\
    259 & & \hphantom{0}3 & mnie & niej \\
    % & & & & \\
    % & & & & \\
    % & & & & \\
    % & & & & \\
    % & & & & \\
    % & & & & \\
    % & & & & \\
    % & & & & \\
    % & & & & \\
    % & & & & \\
    % & & & & \\
    % & & & & \\
    \hline
  \end{tabular}

\end{center}

\VerSpaceTwo


\noindent
\StrWierszGora{39}{7--8} \\
\Jest zdobyła~się jedynie na~słowa \\
\PowinnoByc usłyszała jedynie słowa \\


% ############################




































% ####################################################################
% ####################################################################
% Bibliography

\printbibliography





% ############################

% Koniec dokumentu
\end{document}
