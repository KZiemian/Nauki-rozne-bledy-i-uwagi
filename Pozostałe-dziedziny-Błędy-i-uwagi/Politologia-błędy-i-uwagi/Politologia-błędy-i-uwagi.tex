% ---------------------------------------------------------------------
% Podstawowe ustawienia i pakiety
% ---------------------------------------------------------------------
\RequirePackage[l2tabu, orthodox]{nag}  % Wykrywa przestarzałe i niewłaściwe
% sposoby używania LaTeXa. Więcej jest w l2tabu English version.
\documentclass[a4paper,11pt]{article}
% {rozmiar papieru, rozmiar fontu}[klasa dokumentu]
\usepackage[MeX]{polski}  % Polonizacja LaTeXa, bez niej będzie pracował
% w języku angielskim.
\usepackage[utf8]{inputenc}  % Włączenie kodowania UTF-8, co daje dostęp
% do polskich znaków.
\usepackage{lmodern}  % Wprowadza fonty Latin Modern.
\usepackage[T1]{fontenc}  % Potrzebne do używania fontów Latin Modern.



% ------------------------------
% Podstawowe pakiety (niezwiązane z ustawieniami języka)
% ------------------------------
\usepackage{microtype}  % Twierdzi, że poprawi rozmiar odstępów w tekście.
% \usepackage{graphicx}  % Wprowadza bardzo potrzebne komendy do wstawiania
% % grafiki.
% \usepackage{verbatim}  % Poprawia otoczenie VERBATIME.
% \usepackage{textcomp}  % Dodaje takie symbole jak stopnie Celsiusa,
% wprowadzane bezpośrednio w tekście.
\usepackage{vmargin}  % Pozwala na prostą kontrolę rozmiaru marginesów,
% za pomocą komend poniżej. Rozmiar odstępów jest mierzony w calach.
% ------------------------------
% MARGINS
% ------------------------------
\setmarginsrb
{ 0.7in} % left margin
{ 0.6in} % top margin
{ 0.7in} % right margin
{ 0.8in} % bottom margin
{  20pt} % head height
{0.25in} % head sep
{   9pt} % foot height
{ 0.3in} % foot sep



% ------------------------------
% Często używane pakiety
% ------------------------------
% \usepackage{csquotes}  % Pozwala w prosty sposób wstawiać cytaty do tekstu.
\usepackage{xcolor}  % Pozwala używać kolorowych czcionek (zapewne dużo
% więcej, ale ja nie potrafię nic o tym powiedzieć).





% ---------------------------------------------------------------------
% Dodatkowe ustawienia dla języka polskiego
% ---------------------------------------------------------------------
\renewcommand{\thesection}{\arabic{section}.}
% Kropki po numerach rozdziału (polski zwyczaj topograficzny)
\renewcommand{\thesubsection}{\thesection\arabic{subsection}}
% Brak kropki po numerach podrozdziału



% ------------------------------
% Pakiety napisane przez użytkownika.
% Mają być w tym samym katalogu to ten plik .tex
% ------------------------------
\usepackage{latexgeneralcommands}



% ------------------------------
% Ustawienia różnych parametrów tekstu
% ------------------------------
\renewcommand{\baselinestretch}{1.1}

\renewcommand{\arraystretch}{1.4}  % Ustawienie szerokości odstępów między
% wierszami w tabelach.



% ------------------------------
% Pakiet "hyperref"
% Polecano by umieszczać go na końcu preambuły.
% ------------------------------
\usepackage{hyperref}  % Pozwala tworzyć hiperlinki i zamienia odwołania
% do bibliografii na hiperlinki.










% ---------------------------------------------------------------------
% Tytuł i autor tekstu
\title{Politologia \\
  {\Large Błędy i~uwagi}}

\author{Kamil Ziemian}


% \date{}
% ---------------------------------------------------------------------










% ####################################################################
% Początek dokumentu
\begin{document}
% ####################################################################





% ######################################
\maketitle  % Tytuł całego tekstu
% ######################################





% % ######################################
% \section{}

% \vspace{\spaceTwo}
% % ######################################





% ############################
\Work{ % Autor i tytuł dzieła
  Adam Wielomski \\
  \textit{Sojusz ekstremów w~epoce globalizacji} \\
  \textit{Jak neoliberałowie i~neomarksiści budują nam nowy świat},
  \cite{WielomskiSojuszEkstremowETC2021}}

\vspace{0em}


% ##################
\CenterBoldFont{Uwagi do~konkretnych stron}

\vspace{0em}


\noindent
\textbf{Str. 10, wiersze 5--6.} Zdanie „Albo kapitału nie sposób jednak
pogodzić z~komunizmem, podobnie jak ognia z~wodą?” brzmi dziwnie,
jak autor popełni jakiś błąd w~trakcie pisania. Może powinno ono brzmieć
„Ale czy kapitału nie sposób jednak pogodzić z~komunizmem, podobnie jak
ognia z~wodą?”?

\vspace{\spaceFour}





\noindent
\StrWd{15}{14} Zdanie „Reszta nie ma jutra.” brzmi jak anglicyzm,
pochodzący od zwrotu „For them there is not tomorrow”, i~jakoś nie
najlepiej pasuje do tekstu w~języku polskim. Proponowałbym zamienić je na
„Reszta nie ma przyszłości.”.

\vspace{\spaceFour}





\noindent
\Str{41} W~przypisie 51 odwołanie do pracy Sławomira Mentza \textit{John
  Maynard Keynes a~Ludwig von Mises} wydaje~się nie posiadać pełnych danych
bibliograficznych.

\vspace{\spaceFour}





\noindent
\StrWd{41}{6} Nawias kwadratowy otaczający datę „1936” jest zbyt duży.

\vspace{\spaceFour}










% ##################
\newpage

\CenterBoldFont{Błędy}


\begin{center}

  \begin{tabular}{|c|c|c|c|c|}
    \hline
    Strona & \multicolumn{2}{c|}{Wiersz} & Jest
                              & Powinno być \\ \cline{2-3}
    & Od góry & Od dołu & & \\
    \hline
    14  & & 12 & albo coraz & coraz \\
    15  &  2 & & tylko & nie tylko \\
    66  & & 14 & przyświeca & podoba~się \\
    %   & &  & &  \\
    %  & & & & \\
    %  & &  & &  \\
    %   & & &  &  \\
    %   &  & &  &  \\
    %   & &  &  &  \\
    % & &   & &  \\
    %  & &   &  &  \\
    \hline
  \end{tabular}

\end{center}

\vspace{\spaceTwo}


\noindent
\textbf{Str. 44, wiersze 7--8.} \\
\Jest  Swoista uniwersalizacja liberalizmu anglosaskiego z~liberalizmem
jako takim wynikła stąd, \\
\Powin Swoiste utożsamienie liberalizmu anglosaskiego z~liberalizmem
jako taki wynikło stąd, \\




% ############################









































% ####################################################################
% ####################################################################
% Bibliografia

\bibliographystyle{plalpha}

\bibliography{VariousFieldsBooks}{}





% ############################

% Koniec dokumentu
\end{document}