% ------------------------------------------------------------------------------------------------------------------
% Basic configuration and packages
% ------------------------------------------------------------------------------------------------------------------
% Package for discovering wrong and outdated usage of LaTeX.
% More information to be found in l2tabu English version.
\RequirePackage[l2tabu, orthodox]{nag}
% Class of LaTeX document: {size of paper, size of font}[document class]
\documentclass[a4paper,11pt]{article}



% ------------------------------------------------------
% Packages not tied to particular normal language
% ------------------------------------------------------
% This package should improved spaces in the text
\usepackage{microtype}
% Add few important symbols, like text Celcius degree
\usepackage{textcomp}



% ------------------------------------------------------
% Polonization of LaTeX document
% ------------------------------------------------------
% Basic polonization of the text
\usepackage[MeX]{polski}
% Switching on UTF-8 encoding
\usepackage[utf8]{inputenc}
% Adding font Latin Modern
\usepackage{lmodern}
% Package is need for fonts Latin Modern
\usepackage[T1]{fontenc}



% ------------------------------------------------------
% Setting margins
% ------------------------------------------------------
\usepackage[a4paper, total={14cm, 25cm}]{geometry}



% ------------------------------------------------------
% Setting vertical spaces in the text
% ------------------------------------------------------
% Setting space between lines
\renewcommand{\baselinestretch}{1.1}

% Setting space between lines in tables
\renewcommand{\arraystretch}{1.4}



% ------------------------------------------------------
% Packages for scientific papers
% ------------------------------------------------------
% Switching off \lll symbol, that I guess is representing letter "Ł"
% It collide with `amsmath' package's command with the same name
\let\lll\undefined
% Basic package from American Mathematical Society (AMS)
\usepackage[intlimits]{amsmath}
% Equations are numbered separately in every section
\numberwithin{equation}{section}

% Other very useful packages from AMS
\usepackage{amsfonts}
\usepackage{amssymb}
\usepackage{amscd}
\usepackage{amsthm}

% Package with better looking calligraphy fonts
\usepackage{calrsfs}

% Package with better looking greek letters
% Example of use: pi -> \uppi
\usepackage{upgreek}
% Improving look of lambda letter
\let\oldlambda\Lambda
\renewcommand{\lambda}{\uplambda}




% ------------------------------------------------------
% BibLaTeX
% ------------------------------------------------------
% Package biblatex, with biber as its backend, allow us to handle
% bibliography entries that use Unicode symbols outside ASCII
\usepackage[
language=polish,
backend=biber,
style=alphabetic,
url=false,
eprint=true,
]{biblatex}

\addbibresource{Logika-i-teoria-mnogości-Bibliography.bib}





% ------------------------------------------------------
% Defining new environments (?)
% ------------------------------------------------------
% Defining enviroment "Wniosek"
\newtheorem{corollary}{Wniosek}
\newtheorem{definition}{Definicja}
\newtheorem{theorem}{Twierdzenie}





% ------------------------------------------------------
% Wonderful package PGF/TikZ
% ------------------------------------------------------
% \usepackage{tikz}

% % Loading concrete TikZ libraries
% \usetikzlibrary{decorations.markings}

% % Package with TikZ pics
% \usepackage{TikZPics}

% % Package with additional TikZ styles
% \usepackage{TikZStyles}





% ------------------------------------------------------
% Private packages
% You need to put them in the same directory as .tex file
% ------------------------------------------------------
% Contains various command useful for working with a text
\usepackage{latexgeneralcommands}
% Contains definitions useful for working with mathematical text
% \usepackage{mathcommands}





% ------------------------------------------------------
% Package "hyperref"
% They advised to put it on the end of preambule
% ------------------------------------------------------
% It allows you to use hyperlinks in the text
\usepackage{hyperref}










% ------------------------------------------------------------------------------------------------------------------
% Title and author of the text
\title{Kultura \\
  {\Large Błędy i~uwagi}}

\author{Kamil Ziemian}


% \date{}
% ------------------------------------------------------------------------------------------------------------------










% ####################################################################
% Początek dokumentu
\begin{document}
% ####################################################################





% ######################################
\maketitle  % Tytuł całego tekstu
% ######################################





% ######################################
\section{Kultura od~XVII wieku do~dziś}

\VerSpaceTwo
% ######################################










% ############################
\subsection{Kultura XIX i~XX wieku}

\VerSpaceTwo
% ############################



% ############################
\section{ % Autor i tytuł dzieła
  E. Michael Jones \\
  \textit{Zdeprawowani moderniści},
  \cite{EMichaelJonesZdeprawowaniModernisci2014}}


% ##################
\CenterBoldFont{Błędy}


\begin{center}

  \begin{tabular}{|c|c|c|c|c|}
    \hline
    Strona & \multicolumn{2}{c|}{Wiersz} & Jest
                              & Powinno być \\ \cline{2-3}
    & Od góry & Od dołu & & \\
    \hline
    16  & & 19 & Claya & Gaya \\
    24  &  4 & & człowieczeństwa & człowieczeństwa” \\
    25  & 11 & & Samoa & „Samoa \\
    25  & & 14 & wyłączności & Wyłączności \\
    27  & &  1 & beztroskich>> & beztroskich>>” \\
    29  & 12 & & roku & roku. \\
    30  & &  9 & <<młodej studentki” % >>
           & <<młodej studentki>> \\
    30  & &  8 & bawełniane sukienki>> & <<bawełniane sukienki>> \\
    37  & 11 & & wówczas” & wówczas \\
    37  & & 16 & niego & z~niego \\
    38  & &  2 & zbagatelizowałaś, & zbagatelizowałaś. \\
    49  & 12 & & za & z \\
    56  & 17 & & którym & których \\
    56  &  3 & & Wilberforce'a** & Wilberforce'a* \\
    67  & & 13 & \textit{Whay} & \textit{What} \\
    71  & 14 & & „ Wartości & „Wartości \\
    85  & & 13 & \textit{lalek} & \textit{lalek}” \\
    91  & &  4 & a oni & „a oni \\
    99  &  9 & & maja & mają \\
    101 & & 11 & naukowym & z naukowym \\
    103 & & 13 & ATA & ATS \\
    105 & 19 & & mniej silna & silniejsza \\
    107 & 19 & & Voris & Vorisem \\
    108 &  2 & & rzeczywistości & o~rzeczywistości \\
    111 & 21 & & Indiana, & Indiana. \\
    111 & & 20 & w na & na \\
    119 & & 11 & oszukiwałam”$^{ 2 }$.W & oszukiwałam”$^{ 2 }$. W \\
    122 & 20 & & od & do \\
    161 & &  2 & wyraźn0e & wyraźne \\
    165 & &  3 & ktrego & którego \\
    \hline
  \end{tabular}





  \begin{tabular}{|c|c|c|c|c|}
    \hline
    Strona & \multicolumn{2}{c|}{Wiersz} & Jest
                              & Powinno być \\ \cline{2-3}
    & Od góry & Od dołu & & \\
    \hline
    171 & 12 & & zajmującysię & zajmujący~się \\
    186 & &  5 & umożliwiła & uniemożliwiła \\
    199 & & 15 & wiary & utraty wiary \\
    200 & &  1 & Mogło & Mogła \\
    201 & & 14 & roku & roku. \\
    205 & & 19 & Laetesa & Klaudiusza \\
    207 &  7 & & taka & taką \\
    211 & & 18 & „Szczyt & Szczyt \\
    214 &  5 & & [ w~nagrodę] & [w~nagrodę] \\
    214 &  6 & & najwyraźniej .. & najwyraźniej\ldots \\
    217 & 22 & & Fread & Freuda \\
    218 & &  2 & od~z & z \\
    226 & 10 & & Mannung & Manning \\
    228 & & 17 & pytanie:, & pytanie: \\
    230 & 12 & & 1963 roku & 1963 roku. \\
    230 & 13 & & 1425 & 1525 \\
    237 & &  4 & 560 & 1560 \\
    \hline
  \end{tabular}

\end{center}

\VerSpaceTwo


\noindent
\StrWierszDol{17}{10} \\
\Jest nieczystości pierworodnej córki jest ślepotą ducha. \\
\PowinnoByc pierworodną córką nieczystości jest ślepota ducha. \\
\StrWierszGora{201}{8} \\
\Jest otorbił~się w~kokonie psychoanalizy samego siebie\ldots \\
\PowinnoByc otorbił samego siebie w~kokonie psychoanalizy\ldots \\
\StrWierszGora{209}{20} \\
\Jest \textit{Gdyby potrafił być perwersyjny, byłby zdrowy, podobnie jak
  ojciec$^{ 202 }$}. \\
\PowinnoByc „Gdyby potrafił być perwersyjny, byłby zdrowy,
podobnie jak ojciec”$^{ 202 }$. \\



% ############################










% ######################################
\newpage

\section{Zachodni półwysep Eurazji}

\VerSpaceTwo
% ######################################



% ############################
\subsection{Zagadnienia ogólne}

\VerSpaceThree
% ############################










% ############################
\subsection{Kultura polska i~analizy tego kraju}

\VerSpaceThree
% ############################



% ############################
\section{ % Autor i tytuł dzieła
  Maria Janion \\
  \textit{Niesamowita Słowiańszczyzna. Fantazmaty literatury},
  \cite{JanionNiesamowitaSlowianszczyzna2006} }


% ##################
\CenterBoldFont{Uwagi}


\begin{center}

  \begin{tabular}{|c|c|c|c|c|}
    \hline
    & \multicolumn{2}{c|}{} & & \\
    Strona & \multicolumn{2}{c|}{Wiersz} & Jest
                              & Powinno być \\ \cline{2-3}
    & Od góry & Od dołu & & \\
    \hline
    10  & & 12 & postkolonialnej?$^{ 10 }$. & postkolonialnej$^{ 10 }$? \\
    60  & &  6 & puścizny & spuścizny \\
    113 & 18 & & i~\textit{Masław} & \textit{i~Masław} \\
    175 & 16 & & ciemięzców & ciemiężców \\
    175 & 19 & & ciemięzców & ciemiężców \\
    % & & & & \\
    % & & & & \\
    % & & & & \\
    % & & & & \\
    % & & & & \\
    % & & & & \\
    % & & & & \\
    \hline
  \end{tabular}

\end{center}

\VerSpaceTwo


\noindent
\StrWierszDol{72}{2} \\
\Jest Frankreich 1871, Deutschland 1918 \\
\PowinnoByc \textit{Frankreich 1871}, \textit{Deutschland 1918} \\



% ############################










% ############################
\section{ % Autor i tytuł dzieła
  Jan Sowa \\
  \textit{Fantomowe ciało króla. Peryferyjne zmagania z~nowoczesną formą},
  \cite{SowaFantomoweCialoKrola2011}}

\vspace{0em}


% ##################
\CenterBoldFont{Uwagi do~konkretnych stron}

\vspace{0em}


\noindent
\StrWierszGora{36}{15} Zdanie „jego aktywność ściśle wiąże~się
związana ze społeczną\ldots”, powinno brzmieć „jego aktywność ściśle
wiąże~się ze społeczną\ldots” lub „jego aktywność jest ściśle
związana ze społeczną\ldots”. Choć na podstawie tekstu, nie można
wybrać wersji zamierzonej przez autora, nie~jest to jednak problemem,
bo obie przekazują tą samą treść.

\VerSpaceFour





\StrWierszGora{63}{5--9} Zawarty tu tekst, jest źle skonstruowany
gramatycznie.





% ##################
\CenterBoldFont{Błędy}


\begin{center}

  \begin{tabular}{|c|c|c|c|c|}
    \hline
    Strona & \multicolumn{2}{c|}{Wiersz} & Jest
                              & Powinno być \\ \cline{2-3}
    & Od góry & Od dołu & & \\
    \hline
    92  & &  6 & 1140 & 1440 \\
    102 & 11 & & a bo & bo \\
    118 &  8 & & dawały & nie dawały \\
    300 & 17 & & mogły & mogło \\
    362 & 12 & & torrusa & torusa \\
    367 & 15 & & c\textbf{oś} & \textbf{coś} \\
    % & & & & \\
    \hline
  \end{tabular}

\end{center}


\VerSpaceTwo
% ############################










% ######################################
\section{Kultura japońska}

\VerSpaceTwo
% ######################################



% ############################
\section{ % Autor i tytuł dzieła
  Renata Iwicka \\
  \textit{Źródła klasycznej demonologii japońskiej},
  \cite{IwickaZrodlaKlasycznejDemonologiJaponskiej2017}}

\vspace{0em}


% ##################
\CenterBoldFont{Uwagi do~konkretnych stron}

\vspace{0em}


\Str{26} Zdjęcie na tej stronie powinno być lepiej opisane.

\VerSpaceFour





\Str{49} Użyto tu przydomka „Raiko” % Raik\={o}
by odnieść~się do~postaci Minamoto no~Yorumitsu jednak~to, że~ten
przydomek odnosi~się do~niego stanie~się jasne dopiero na~stronie~52.

\VerSpaceFour





\StrWierszGora{94}{14--15} Według tego co~znalazłem, shamisen
to~podobny do~gitary trójstrunowy instrument\footnote{Jednak nie mogę
  gwarantować, że~jest to poprane wyjaśnienie tej nazwy.}. Myślę,
że~czym on~jest powinni być tu~wyjaśnione, wszak nie każdy czytelnik
tej książki, jest takim znawcą kultury japońskiej, by~to wiedzieć.
A~jak znaleźć wiarygodne źródło w~takiej specjalistycznej sprawie, też
nie jest całkiem proste.

\VerSpaceFour





\StrWierszDol{95}{3} W~tej książce normalnie zapisuje~się nazwę tej
wyspy jako „Kyushu”. % Ky\={u}sh\={u}

\VerSpaceFour





\StrWierszGora{128}{7} To~zdanie jest sformułowane w~taki sposób,
że~nie rozumiem ani~jak położona jest oś~o~której mowa, ani jak
nazywają~się leżące na~niej bramy.

\VerSpaceFour





\StrWierszGora{129}{12--16} Fragment ten jest napisany w~taki sposób,
że~nie byłem w~stanie zrozumieć z~niego, która pisownia nazwy bramy
jest pierwotna, a~która powstała później. Z~reszty książki wynika,
że~pierwotna nazwa to „Rajomon”, % Raj\={o}mon
późniejsza zaś~to „Rashomon”. % Rash\={o}mon

\VerSpaceFour





\StrWierszDol{139}{15} Aby~zachować płynność przytoczonej
tu~opowieści, należałoby wcześniej wspomnieć, że~dwa bohaterowie
dotarli do jakiejś rezydencji.

\VerSpaceFour





\textbf{Str. 148, przypis 441.} Jestem przekonany, że~ten przypis
zamiast mówić jak obecnie, iż~\textit{Rituale Romanum} było używane prze
niektórych egzorcystów od~1999, miał stwierdzać, że~był używany
do~tego roku.

\VerSpaceFour





\StrWierszDol{169}{10} Całkiem możliwe, że~zamiast „wykryto dużo
wcześniej” powinno pisać „wykryto bardzo wcześnie”.

\VerSpaceFour





\StrWierszGora{170}{} Nie jest dla mnie o~jaki z~wymienionych systemów
jest mowa w~tym zdaniu. Czy chodzi o~\textit{shugendo}, % shugend\={o}
czy też o~\textit{on'yodo}? % on'y\={o}d\={o}

\VerSpaceFour





\Str{181} Podtytuł rozdziału jest zapisany tu jako „\textit{Tabu
  w}~shinto”, % shint\={o}
podczas gdy~w~spisie treści na~stronie~6 jako „Tabu
w~\textit{shinto}”. % shint\={o}
Warto byłoby ten zapis ujednolicić, choć może to~być sprzeczne
z~konwencją zapisu tytułów odpowiednich części tekstu.

\VerSpaceFour





\Str{187} Drugi paragraf jest nie najlepiej skonstruowany
pod~względem językowym. Jego sens miał być zapewne taki. Gdyby
przyporządkowanie stron świata i~zwierząt różnym wojownikom bazowało
na~ich osiągnięciach i~roli odgrywanej w~historii, to podział ten
wyglądałby inaczej niż ten~który~się przyjął.

\VerSpaceFour





\Str{215} Podpis pod tym rysunkiem powinien być obszerniejszy.

\VerSpaceFour





\StrWierszDol{222}{9} Może zamiast „użył znaku oznaczającego
kobietę” powinno być „stworzył znak oznaczający kobietę”?

\VerSpaceFour





\StrWierszDol{222}{2} Według opisu na angielskiej Wikipedii, główny
bohater mangi
\href{https://en.wikipedia.org/wiki/XxxHolic}{\textit{xxxHOLiC}} jest
nawiedzany przez istoty określane jako \textit{yokai}
% y\={o}kai
i~\textit{ayakashi}.

\VerSpaceFour





\Str{223} W~książce tej zwykle zapisuje~się nazwę przywoływanej
tu~wyspy jako „Kyushu”, % Ky\={u}sh\={u}
nie „Kiusiu”.





% ##################
\CenterBoldFont{Błędy}


\begin{center}

  \begin{tabular}{|c|c|c|c|c|}
    \hline
    Strona & \multicolumn{2}{c|}{Wiersz} & Jest
                              & Powinno być \\ \cline{2-3}
    & Od góry & Od dołu & & \\
    \hline
    10  & &  2 & Kioto, 2012 & Kioto 2012 \\
    12  & & 12 & J.~Tubielewicz~J. & J.Tubielewicz \\
    12  & & 11 & J.~Tubielewicz~J. & J.Tubielewicz \\
    13  & &  6 & \textit{Yokai} Database: & \textit{Yokai Database}: \\
    % Nad ,,o'' w ,,Yokai'' powinna być kreska akcentu.
    17  & 17 & & Samą przestrzeń & O~samej przestrzeni \\
    63  & 13 & & „późniejszym światem & „późniejszym światem” \\
    64  & 18 & & rezydencja obok & rezydencja \\
    72  & &  4 & narodzin & narodzonej \\
    125 & &  6 & spostrzegania & postrzegania \\
    141 & 20 & & tych samych & tych \\
    154 & & 12 & )] & ] \\
    168 & & 15 & kamienia strzałek & kamieni strzałowych \\
    172 & 19 & & obecne & obecny \\
    179 & & 11 & schowku & schowku. \\
    186 & &  2 & pozbawione cudzysłowu & bez cudzysłowów \\
    197 & & 20 & schemat zachowań ludzkich & ludzki schemat zachowań \\
    198 &  7 & & & \textit{yokai$\,^{647}$} \\  % y\={o}kai
    214 & &  2 & księżycu & \textit{księżycu} \\
    230 & &  2 & bogów ludzi & bogów, ludzi \\
    231 & & 17 & pewną & pewnymi \\
    \hline
  \end{tabular}

\end{center}

\VerSpaceTwo


\noindent
\StrWierszGora{169}{19} \\
\Jest a~zwłaszcza \\
\PowinnoByc których szczególnym uosobieniem w~tym kontekście jest \\
\StrWierszGora{189}{6} \\
\Jest zarysowaną postać \\
\PowinnoByc zarysowanie tej postaci \\
\StrWierszGora{198}{10} \\
\Jest obejmują dokładnie tę~samą kategorię \\
\PowinnoByc są objęte tą~samą kategorią \\



% ############################





























% ######################################
\newpage

\section{Kino}

\VerSpaceTwo
% ######################################



% ############################
\section{ % Redaktor i tytuł dzieła
  Red. Piotr Kletowski \\
  „Europejskie kino gatunków”,
  \cite{RedKletowskiEuropejskieKinoGatunkow2016}}


% ##################
\CenterBoldFont{Uwagi}


\StrWierszDol{129}{1} W~filmie templariuszom nie wyłupiono oczu, lecz
stracono i~powieszono na drzewach. Tam dzikie ptaki wyjadły ich oczy.





% ##################
\CenterBoldFont{Błędy}


\begin{center}

  \begin{tabular}{|c|c|c|c|c|}
    \hline
    & \multicolumn{2}{c|}{} & & \\
    Strona & \multicolumn{2}{c|}{Wiersz} & Jest
                              & Powinno być \\ \cline{2-3}
    & Od góry & Od dołu & & \\
    \hline
    87 & 11 & & wybraną & jedną wybraną \\
    % & & & & \\
    % & & & & \\
    % & & & & \\
    % & & & & \\
    \hline
  \end{tabular}

\end{center}

\VerSpaceTwo


\noindent
\StrWierszGora{160}{12} \\
\Jest \textit{Za kilka dolarów więcej} Monco \\
\PowinnoByc filmu \textit{Dobry, zły i~brzydki} Blondie \\



% ############################










% ############################
\section{ % Autor i tytuł dzieła
  Tadeusz Szczepański \\
  „Zwierciadło Bergmana”, \cite{SzczepanskiZwierciadloBergmana2007} }


% ##################
\CenterBoldFont{Uwagi}


\StrWierszGora{131}{14} Postać pastora zagrał Gunnar Olsson.

\VerSpaceFour





\StrWierszDol{132}{17} Protestancki biskup Edvard Verg\'{e}rus
to~jeden z~bohaterów filmu Bergmana \textit{Fanny i~Aleksander}.

\VerSpaceFour





\Str{146} Ponieważ Monika nie tylko udało~się uciec
z~mieszczańskiej willi, bez żadnych konsekwencji dla jej dalszego
życia, ale także zdobyć pieczeń po~którą tam poszła,
uważam~że~nazwanie jej działań „sromotnie nieudanymi”, jest
co~najmniej nietrafne.

\VerSpaceFour





\Str{220} W~samym filmie \textit{Siódma pieczęć} nie znalazłem
niczego co~podtrzymywałoby przedstawioną tu interpretację, że~końcowa
scena z~rodziną kuglarzy rozgrywa~się w~czyjejś wyobraźni. Wręcz
przeciwnie, cała narracja filmu przekonuje mnie, że~wydarzyło~się
to~naprawdę.

\VerSpaceFour





\Str{247} Na~tym zdjęciu bardzo ciężko dojrzeć, że~malowidło
na~ścianie przedstawie Śmierć z~szachami. Na~szczęście na~następnej
stronie jest wyjaśnione, iż~przedstawia ono Śmierć z~szachami
pod~ręką, która prowadzi koński zaprzęg.

\VerSpaceFour





% ##################
\CenterBoldFont{Błędy}


\begin{center}

  \begin{tabular}{|c|c|c|c|c|}
    \hline
    & \multicolumn{2}{c|}{} & & \\
    Strona & \multicolumn{2}{c|}{Wiersz} & Jest
                              & Powinno być \\ \cline{2-3}
    & od góry & od dołu & & \\
    \hline
    31  & & 15 & zamiłowania ch & zamiłowaniach \\
    101 & 19 & & przypowieść & opowieść \\
    139 & 12 & & pozamał & pozamał\dywiz \\
    214 &  2 & & (\r{A}ke Fridell) Tubal & \r{A}ke Fridell (Tubal) \\
    % & & & & \\
    % & & & & \\
    457 & 18 & & Łódź . & Łódź. \\
    % & & & & \\
    \hline
  \end{tabular}

\end{center}

\VerSpaceTwo





% ############################









% ######################################
\newpage
\section{Literatura}

\VerSpaceTwo
% ######################################



% ############################
\section{ % Autor i tytuł dzieła
  Teodor Parnicki \\
  „Szkice literackie”, \cite{ParnickiSzkiceLiterackie1979}}


% ##################
\CenterBoldFont{Uwagi do~konkretnych stron}


\Str{88, 220} Na stronie 88 tytuł tej części to „O~humanizmie
katolicyzmu”, zaś na 220 to „O~humanizmie katolickim”. Sens tych dwóch
tytułów jest oczywiście diametralnie różny. Wydaje się, że~poprawny to
„O~humanizmie katolickim” bo takie słowa są użyte w~tytule szkicu
o~Zofii Kossak.

\VerSpaceFour





\Str{69} W~zaczynającym się tu szkicu „Aleksander Puszkin
(w~stulecie śmierci)” wyróżniona jest część pierwsza i~trzecia, ale
nigdzie nie ma drugiej.

\VerSpaceFour





% ##################
\CenterBoldFont{Błędy}


\begin{center}

  \begin{tabular}{|c|c|c|c|c|}
    \hline
    Strona & \multicolumn{2}{c|}{Wiersz} & Jest
                              & Powinno być \\ \cline{2-3}
    & od góry & od dołu & & \\
    \hline
    XXI   &  6 & & \textit{Juliana}\ldots$^{ 26 }$
           & \textit{Juliana}\ldots”$^{ 26 }$ \\
    XXIII & 15 & Lenowa. & Lenowa). \\
    91  &  9 & & upadku; & upadku. \\
    92  & 23 & & XX & XIX \\
    119 & & 7 & stałe & całe \\
    174 & 13 & & słynnych --- encyklikach & słynnych encyklikach \\
    174 & 15 & & XX. & XX.) \\
    % & & & & \\
    % & & & & \\
    % & & & & \\
    % & & & & \\
    % & & & & \\
    \hline
  \end{tabular}

\end{center}

\VerSpaceTwo





% ############################










% ######################################
\newpage

\section{Malarstwo}

\VerSpaceTwo
% ######################################



% ############################
\section{ % Autor i tytuł dzieła
  Władysław Strzemiński \\
  \textit{Teoria widzenia}, \cite{StrzeminskiTeoriaWidzenia2016}}


% ##################
\CenterBoldFont{Uwagi do~konkretnych stron}


\StrWierszGora{33}{13} Cudzysłów wystający na~lewy margines wygląd mało
estetycznie. W~przypadku dłuższych cytatów, których lewy margines jest
większy, nie~jest to problem.

\VerSpaceFour





\StrWierszDol{39}{19} Cudzysłów wystający na~lewy margines wygląd mało
estetycznie.

\VerSpaceFour





% \Str{146} Ponieważ Monika nie tylko udało~się uciec
% z~mieszczańskiej willi, bez żadnych konsekwencji dla jej dalszego
% życia, ale także zdobyć pieczeń po~którą tam poszła,
% uważam~że~nazwanie jej działań ,,sromotnie nieudanymi'', jest
% co~najmniej nietrafne.

% \VerSpaceFour





% \start \Str{220} W~samym filmie \textit{Siódma pieczęć} nie znalazłem
% niczego co~podtrzymywałoby przedstawioną tu interpretację,
% że~końcowa scena z~rodziną kuglarzy rozgrywa~się w~czyjejś
% wyobraźni. Wręcz przeciwnie, cała narracja filmu przekonuje mnie,
% że~wydarzyło~się to~naprawdę.

% \vspace{\spaceFour}





% \Str{247} Na~tym zdjęciu bardzo ciężko dojrzeć, że~malowidło
% na~ścianie przedstawie Śmierć z~szachami. Na~szczęście na~następnej
% stronie jest wyjaśnione, iż~przedstawia ono Śmierć z~szachami
% pod~ręką, która prowadzi koński zaprzęg.

% \vspace{\spaceFour}





% ##################
\CenterBoldFont{Błędy}


\begin{center}

  \begin{tabular}{|c|c|c|c|c|}
    \hline
    Strona & \multicolumn{2}{c|}{Wiersz} & Jest
                              & Powinno być \\ \cline{2-3}
    & od góry & od dołu & & \\
    \hline
    15  & & 12 & 1709 & 1709) \\
    15  & &  3 & na & nad \\
    16  & & 16 & \textit{Teoria widzenia:} & \textit{Teoria widzenia}: \\
    44  &  3 & & rewolucyjny radykalizm & rewolucyjnego radykalizmu \\
    % & & & & \\
    % & & & & \\
    % & & & & \\
    % & & & & \\
    % & & & & \\
    \hline
  \end{tabular}

\end{center}

\VerSpaceTwo



% ############################










% ######################################
\newpage

\section{Muzyka}

\VerSpaceTwo
% ######################################



% ############################
\section{ % Autor i tytuł dzieła
  Franciszek Wesołowski \\
  \textit{Zasady muzyki}, \cite{WesolowskiZasadyMuzyki2017}}


% \CenterTB{Uwagi}

% \start \StrWd{}{}


% ##################
\CenterBoldFont{Błędy}


\begin{center}

  \begin{tabular}{|c|c|c|c|c|}
    \hline
    Strona & \multicolumn{2}{c|}{Wiersz} & Jest
                              & Powinno być \\ \cline{2-3}
    & Od góry & Od dołu & & \\
    \hline
    12  & &  4 & $440,\! 2$ & $440 \times 2$ \\
    12  & &  3 & $440,\! 3$ & $440 \times 3$ \\
    12  & &  3 & $440,\! 4$ & $440 \times 4$ \\
    % & & & & \\
    % & & & & \\
    % & & & & \\
    % & & & & \\
    \hline
  \end{tabular}

\end{center}

\VerSpaceTwo



% ############################










% ############################
\section{ % Autor i tytuł dzieła
  Franciszek Wesołowski \\
  \textit{Zasady muzyki},
  \cite{WesolowskiZasadyMuzyki2021}}


% \CenterTB{Uwagi}

% \start \StrWd{}{}


% ##################
\CenterBoldFont{Błędy}


\begin{center}

  \begin{tabular}{|c|c|c|c|c|}
    \hline
    Strona & \multicolumn{2}{c|}{Wiersz} & Jest
                              & Powinno być \\ \cline{2-3}
    & Od góry & Od dołu & & \\
    \hline
    12  & &  4 & $440,\! 2$ & $440 \times 2$ \\
    12  & &  3 & $440,\! 3$ & $440 \times 3$ \\
    12  & &  3 & $440,\! 4$ & $440 \times 4$ \\
    % & & & & \\
    % & & & & \\
    % & & & & \\
    % & & & & \\
    \hline
  \end{tabular}

\end{center}

\VerSpaceTwo



% ############################










% ####################################################################
% ####################################################################
% Bibliography

\printbibliography





% ############################
% End of the document

\end{document}
