% ------------------------------------------------------------------------------------------------------------------
% Basic configuration and packages
% ------------------------------------------------------------------------------------------------------------------
% Package for discovering wrong and outdated usage of LaTeX.
% More information to be found in l2tabu English version.
\RequirePackage[l2tabu, orthodox]{nag}
% Class of LaTeX document: {size of paper, size of font}[document class]
\documentclass[a4paper,11pt]{article}



% ------------------------------------------------------
% Packages not tied to particular normal language
% ------------------------------------------------------
% This package should improved spaces in the text
\usepackage{microtype}
% Add few important symbols, like text Celcius degree
\usepackage{textcomp}



% ------------------------------------------------------
% Polonization of LaTeX document
% ------------------------------------------------------
% Basic polonization of the text
\usepackage[MeX]{polski}
% Switching on UTF-8 encoding
\usepackage[utf8]{inputenc}
% Adding font Latin Modern
\usepackage{lmodern}
% Package is need for fonts Latin Modern
\usepackage[T1]{fontenc}



% ------------------------------------------------------
% Setting margins
% ------------------------------------------------------
\usepackage[a4paper, total={14cm, 25cm}]{geometry}



% ------------------------------------------------------
% Setting vertical spaces in the text
% ------------------------------------------------------
% Setting space between lines
\renewcommand{\baselinestretch}{1.1}

% Setting space between lines in tables
\renewcommand{\arraystretch}{1.4}



% ------------------------------------------------------
% Packages for scientific papers
% ------------------------------------------------------
% Switching off \lll symbol, that I guess is representing letter "Ł"
% It collide with `amsmath' package's command with the same name
\let\lll\undefined
% Basic package from American Mathematical Society (AMS)
\usepackage[intlimits]{amsmath}
% Equations are numbered separately in every section
\numberwithin{equation}{section}

% Other very useful packages from AMS
\usepackage{amsfonts}
\usepackage{amssymb}
\usepackage{amscd}
\usepackage{amsthm}

% Package with better looking calligraphy fonts
\usepackage{calrsfs}

% Package with better looking greek letters
% Example of use: pi -> \uppi
\usepackage{upgreek}
% Improving look of lambda letter
\let\oldlambda\Lambda
\renewcommand{\lambda}{\uplambda}




% ------------------------------------------------------
% BibLaTeX
% ------------------------------------------------------
% Package biblatex, with biber as its backend, allow us to handle
% bibliography entries that use Unicode symbols outside ASCII
\usepackage[
language=polish,
backend=biber,
style=alphabetic,
url=false,
eprint=true,
]{biblatex}

\addbibresource{Logika-i-teoria-mnogości-Bibliography.bib}





% ------------------------------------------------------
% Defining new environments (?)
% ------------------------------------------------------
% Defining enviroment "Wniosek"
\newtheorem{corollary}{Wniosek}
\newtheorem{definition}{Definicja}
\newtheorem{theorem}{Twierdzenie}





% ------------------------------------------------------
% Wonderful package PGF/TikZ
% ------------------------------------------------------
% \usepackage{tikz}

% % Loading concrete TikZ libraries
% \usetikzlibrary{decorations.markings}

% % Package with TikZ pics
% \usepackage{TikZPics}

% % Package with additional TikZ styles
% \usepackage{TikZStyles}





% ------------------------------------------------------
% Private packages
% You need to put them in the same directory as .tex file
% ------------------------------------------------------
% Contains various command useful for working with a text
\usepackage{latexgeneralcommands}
% Contains definitions useful for working with mathematical text
% \usepackage{mathcommands}





% ------------------------------------------------------
% Package "hyperref"
% They advised to put it on the end of preambule
% ------------------------------------------------------
% It allows you to use hyperlinks in the text
\usepackage{hyperref}










% ------------------------------------------------------------------------------------------------------------------
% Title and author of the text
\title{Politologia \\
  {\Large Błędy i~uwagi}}

\author{Kamil Ziemian}


% \date{}
% ------------------------------------------------------------------------------------------------------------------










% ####################################################################
% Początek dokumentu
\begin{document}
% ####################################################################





% ######################################
\maketitle  % Tytuł całego tekstu
% ######################################





% % ######################################
% \section{}

% \vspace{\spaceTwo}
% % ######################################





% ############################
\section{ % Autor i tytuł dzieła
  Adam Wielomski \\
  \textit{Sojusz ekstremów w~epoce globalizacji} \\
  \textit{Jak neoliberałowie i~neomarksiści budują nam nowy świat},
  \cite{WielomskiSojuszEkstremowETC2021}}

\vspace{0em}


% ##################
\CenterBoldFont{Uwagi do~konkretnych stron}

\vspace{0em}


\noindent
\textbf{Str. 10, wiersze 5--6.} Zdanie „Albo kapitału nie sposób jednak
pogodzić z~komunizmem, podobnie jak ognia z~wodą?” brzmi dziwnie,
jak autor popełni jakiś błąd w~trakcie pisania. Może powinno ono brzmieć
„Ale czy kapitału nie sposób jednak pogodzić z~komunizmem, podobnie jak
ognia z~wodą?”?

\VerSpaceFour





\noindent
\StrWierszDol{15}{14} Zdanie „Reszta nie ma jutra.” brzmi jak anglicyzm,
pochodzący od zwrotu „For them there is not tomorrow”, i~jakoś nie
najlepiej pasuje do tekstu w~języku polskim. Proponowałbym zamienić je na
„Reszta nie ma przyszłości.”.

\VerSpaceFour





\noindent
\Str{41} W~przypisie 51 odwołanie do pracy Sławomira Mentza \textit{John
  Maynard Keynes a~Ludwig von Mises} wydaje~się nie posiadać pełnych danych
bibliograficznych.

\VerSpaceFour





\noindent
\StrWierszDol{41}{6} Nawias kwadratowy otaczający datę „1936” jest zbyt duży.

\VerSpaceFour










% ##################
\newpage

\CenterBoldFont{Błędy}


\begin{center}

  \begin{tabular}{|c|c|c|c|c|}
    \hline
    Strona & \multicolumn{2}{c|}{Wiersz} & Jest
                              & Powinno być \\ \cline{2-3}
    & Od góry & Od dołu & & \\
    \hline
    14  & & 12 & albo coraz & coraz \\
    15  &  2 & & tylko & nie tylko \\
    66  & & 14 & przyświeca & podoba~się \\
    %   & &  & &  \\
    %  & & & & \\
    %  & &  & &  \\
    %   & & &  &  \\
    %   &  & &  &  \\
    %   & &  &  &  \\
    % & &   & &  \\
    %  & &   &  &  \\
    \hline
  \end{tabular}

\end{center}

\VerSpaceTwo


\noindent
\textbf{Str. 44, wiersze 7--8.} \\
\Jest Swoista uniwersalizacja liberalizmu anglosaskiego z~liberalizmem
jako takim wynikła stąd, \\
\PowinnoByc Swoiste utożsamienie liberalizmu anglosaskiego z~liberalizmem
jako taki wynikło stąd, \\




% ############################









































% ####################################################################
% ####################################################################
% Bibliography

\printbibliography





% ############################
% End of the document

\end{document}
