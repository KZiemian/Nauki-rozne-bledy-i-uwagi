% ------------------------------------------------------------------------------------------------------------------
% Basic configuration and packages
% ------------------------------------------------------------------------------------------------------------------
% Package for discovering wrong and outdated usage of LaTeX.
% More information to be found in l2tabu English version.
\RequirePackage[l2tabu, orthodox]{nag}
% Class of LaTeX document: {size of paper, size of font}[document class]
\documentclass[a4paper,11pt]{article}



% ------------------------------------------------------
% Packages not tied to particular normal language
% ------------------------------------------------------
% This package should improved spaces in the text
\usepackage{microtype}
% Add few important symbols, like text Celcius degree
\usepackage{textcomp}



% ------------------------------------------------------
% Polonization of LaTeX document
% ------------------------------------------------------
% Basic polonization of the text
\usepackage[MeX]{polski}
% Switching on UTF-8 encoding
\usepackage[utf8]{inputenc}
% Adding font Latin Modern
\usepackage{lmodern}
% Package is need for fonts Latin Modern
\usepackage[T1]{fontenc}



% ------------------------------------------------------
% Setting margins
% ------------------------------------------------------
\usepackage[a4paper, total={14cm, 25cm}]{geometry}



% ------------------------------------------------------
% Setting vertical spaces in the text
% ------------------------------------------------------
% Setting space between lines
\renewcommand{\baselinestretch}{1.1}

% Setting space between lines in tables
\renewcommand{\arraystretch}{1.4}



% ------------------------------------------------------
% Packages for scientific papers
% ------------------------------------------------------
% Switching off \lll symbol, that I guess is representing letter "Ł"
% It collide with `amsmath' package's command with the same name
\let\lll\undefined
% Basic package from American Mathematical Society (AMS)
\usepackage[intlimits]{amsmath}
% Equations are numbered separately in every section
\numberwithin{equation}{section}

% Other very useful packages from AMS
\usepackage{amsfonts}
\usepackage{amssymb}
\usepackage{amscd}
\usepackage{amsthm}

% Package with better looking calligraphy fonts
\usepackage{calrsfs}

% Package with better looking greek letters
% Example of use: pi -> \uppi
\usepackage{upgreek}
% Improving look of lambda letter
\let\oldlambda\Lambda
\renewcommand{\lambda}{\uplambda}




% ------------------------------------------------------
% BibLaTeX
% ------------------------------------------------------
% Package biblatex, with biber as its backend, allow us to handle
% bibliography entries that use Unicode symbols outside ASCII
\usepackage[
language=polish,
backend=biber,
style=alphabetic,
url=false,
eprint=true,
]{biblatex}

\addbibresource{Cybernetyka-Bibliography.bib}





% ------------------------------------------------------
% Defining new environments (?)
% ------------------------------------------------------
% Defining enviroment "Wniosek"
\newtheorem{corollary}{Wniosek}
\newtheorem{definition}{Definicja}
\newtheorem{theorem}{Twierdzenie}





% ------------------------------------------------------
% Local packages
% You need to put them in the same directory as .tex file
% ------------------------------------------------------
% Package containing various command useful for working with a text
\usepackage{./Local-packages/general-commands}
% Package containing commands and other code useful for working with
% mathematical text
\usepackage{./Local-packages/math-commands}





% ------------------------------------------------------
% Package "hyperref"
% They advised to put it on the end of preambule
% ------------------------------------------------------
% It allows you to use hyperlinks in the text
\usepackage{hyperref}










% ------------------------------------------------------------------------------------------------------------------
% Title and author of the text
\title{Cybernetyka \\
  {\Large Błędy i~uwagi}}

\author{Kamil Ziemian}


% \date{}
% ------------------------------------------------------------------------------------------------------------------










% ####################################################################
\begin{document}
% ####################################################################





% ################################################
\maketitle % Tytuł całego tekstu
% ################################################





% ######################################
\section{Norbert Wiener \\
  \textit{Cybernetics, or~Control and~Communiication~in \\
    the~Animal and the~Machine},
  \parencite{Wiener-Cybernetics-Second-edition-Pub-2016}}

\vspace{0em}
% ######################################


% ##################
\CenterBoldFont{Uwagi do~rozdziałów}

\vspace{0em}


\noindent

\textbf{Rozdział~III.} Na stronie 62 wprowadzona jest dość nieszczęśliwy
sposób numerowania wzorów. Dokładniej oznaczenie „(3.081)” nie oznacza wcale
wzoru numer~81 w~rozdziale~3, ale \textit{pierwszy} wzór, będący wnioskiem
lub~komentarze do~wzoru numer~8 w~rozdziale~3. Sens tych oznaczeń byłby
bardziej zrozumiały, gdyby zostały one zapisane jako „(3.08.1)”. Ze~względu
na ilość potencjalnych nieporozumień, jakie powoduje stosowana w~książce
notacja, w~tych notatkach podanym oznaczenia wszystkich problematycznych
wzorów w~postaci~(X.Y.Z).

\VerSpaceFour










% ##################
\CenterBoldFont{Uwagi do~konkretnych stron}

\vspace{0em}


\noindent
\Str{53} Napotykamy tu problem, że~wzory (2.24) i~(2.25) różnią~się tylko
tym, że~pierwszy zawiera symbol „$\text{l.i.m.}$”, a~w~drugim na jego
miejscu stoi „$\lim$”. Ponieważ symbol „$\text{l.i.m.}$” nie został
nigdzie zdefiniowany, nie wiadomo co on oznacza, tym samym nie wiemy, czym
różnią~się wnioski wynikające z~dwóch twierdzeń ergodycznych. Niestety,
wyjaśnienie tego problemu przekracza naszą wiedzę z~teorii ergodycznej.

\VerSpaceFour





\noindent
\StrWierszGora{60}{15} Wiener używa tu nazwy „error~of measurement”, którą
przetłumaczylibyśmy na „błąd pomiaru”. Jednak znacznie lepiej jest
używać przyjętej w~nowszej literaturze nazwy „niepewność pomiarowa”,
po~angielsku „measurement uncertanity”, gdyż lepiej oddaje treść tego
pojęcia. Dlatego też, w~tych notatkach będziemy zawsze używać tego drugiego
zwrotu.

\VerSpaceFour





\noindent
\StrWierszeGora{60}{15--17} Przyjrzyjmy~się zdaniu „uniformly distributed
error lying over a~range~of length $b_{ 1 } b_{ 2 } \ldots b_{ n }\ldots$, where
$b_{ \kappa }$ is the~first digit not equal to~$0$”. Prawie na~pewno należy je
rozumieć w~następujący sposób. Niepewność pomiarowa jest dana przez ułamek
z~rozwinięciem binarnym $0.b_{ 1 } b_{ 2 } b_{ 3 }\ldots$, gdzie $b_{ i }$,
$i = 1, 2, 3, \ldots$, to kolejne cyfry binarne tego ułamka. Liczba $\kappa$ to
najmniejsza liczba naturalna większa od zera, taka że~$b_{ \kappa } \neq 0$.
Inaczej mówiąc $b_{ \kappa }$ jest pierwszą cyfrą po przecinku, która jest
równa~$1$.

\VerSpaceFour





\noindent
\textbf{Str.~60, wiersz~1 od dołu, str.~61, wiersz~1.} Wiener pisze,
że~zakładamy o~zmiennych, iż~mają „fundamental equpartition”. Na podstawie
dalszego tekstu można wnioskować, że~chodzi mu o~to, iż~zawsze będziemy
przyjmować, że~funkcje zmiennej~$x$ należy zawsze całkować względem miary
$dx$, nie zaś miary takiej jak $d\mu( x ) = x^{ 3 } dx$.

\VerSpaceFour





\noindent
\Str{61} Treść tej strony jest wyjątkowo trudna do zrozumienia,
a~jednocześnie kluczowa dla tego rozdziału, jeśli nie dla całej książki,
dlatego należy~się ją możliwie dokładnie przeanalizować. Pytanie na które
Wiener szuka tutaj odpowiedzi, jest następujące. Niech~$f( x )$ będzie
funkcją gęstości prawdopodobieństwa, określoną na prostej rzeczywistej. Jak
zmierzyć ilość informacji jaką ta funkcja zawiera?

Zacznijmy od~dyskusji własności, jakie funkcja $f( x )$ ma posiadać. Funkcja
ta musi być w~odpowiednim sensie całkowalna, jednak dla potrzeb niniejszej
dyskusji wygodnie jest przyjąć, iż~jest ona ciągła na całym $\Rbb$, poza
skończoną liczbą punktów. Warunek ten można osłabić i~wyniki wciąż pozostaną
w~mocy, ale~takie uproszczenie koncepcyjnej strony problemu jest obecnie
pożądane. Po drugie, funkcja ta musi być unormowana w~następującym sensie
\begin{equation}
  \label{eq:Winere-Cybernetics-ETC-01}
  \int_{ -\infty }^{ +\infty } f( x ) \, dx = 1.
\end{equation}
Ponieważ $f( x )$ ma być gęstością prawdopodobieństwa, więc musi być funkcją
nieujemną $f( x ) \geq 0$.

Przyjrzymy~się teraz słowom „the~average logarithm~of the~breadth
$f_{ 1 }( x )$ may be~considered as~some sort~of average~of the~height~of
the~logarithm~of the~reciprocal~of $f_{ 1 }( x )$”. W~naszej ocenie Wiener
odwołuje~się tutaj do~idei z~teorii informacji, której nie sformułował
jawnie, która zaś wiąże prawdopodobieństwo zdarzenia z~ilości informacji
jaką jego zajście nam daje. Ideę tą można wysłowić w~następujący sposób.





% ##################
\begin{quote}

  Im mniej prawdopodobne jest dane wydarzenie, tym więcej informacji jego
  zajście nam dostarcza. Zajście zdarzenia pewnego posiada zerową ilość
  informacji, jeśli zaś prawdopodobieństwo zdarzenia monotonicznie malej
  do zera, to ilość informacji dąży do nieskończoności.

\end{quote}
% ##################





Definicja ta nie jest w~pełni ścisła, lecz jej doprecyzowanie jest dla
naszych potrzeb zbędne. Jeśli teraz oznaczymy prawdopodobieństwo danego
zdarzenia przez $P$, to funkcja dana wzorem
\begin{equation}
  \label{eq:Winere-Cybernetics-ETC-02}
  \log_{ 2 }\!\left( \frac{ 1 }{ P } \right) = -\log_{ 2 }( P ),
\end{equation}
posiada wszystkie własności, które wedle podanej przed chwilą idei, powinny
charakteryzować ilość informacji odpowiadającą zdarzeniu~$P$.
Wzór ten jest w~zasadzie identyczny z~wzorem (3.02) z~książki Wienera
i~stanowi jedną ze standardowych definicji ilości informacji.

Wróćmy teraz do słów Wienera o~braniu średniej z~logarytmu wyrażenia
$1 / f( x )$ (używa on tu angielskiego słowa „reciprocal”). Widzimy teraz,
że~chodzi mu tu~o~ilość informacji przypisaną zdarzeniu o~prawdopodobieństwu
równemu $f( x )$. Tutaj wchodzi jednak pewien niuans teorii
prawdopodobieństwa, taki że~samo $f( x )$ nie jest prawdopodobieństwem
żadnego zdarzenia, gdyż zbiór $\{ x \}$ nie reprezentuje żadnego „sensownego”
zdarzenia. Dopiero wyrażenie
\begin{equation}
  \label{eq:Winere-Cybernetics-ETC-03}
  f( x ) \, \Delta x,
\end{equation}
jest przybliżonym prawdopodobieństwem zdarzenia, polegającym na tym,
że~zaszło jedno z~„bezsensownych” zdarzeń zawartych w~przedziale
$[ x, x + \Delta x ]$. Idąc dalej tą drogą wzór
\begin{equation}
  \label{eq:Winere-Cybernetics-ETC-04}
  \log_{ 2 }\!\left( \tfrac{ 1 }{ f( x ) } \right) =
  -\log_{ 2 }\!\big( f( x ) \big) \, f( x ) \, \Delta x,
\end{equation}
interpretujemy jako przybliżoną ilość informacja, którą dostarcza nam
fakt, że~zaszło zdarzenie $x_{ 1 }$ znajdują~się w~przedziale
$[ x, x + \Delta x ]$. Z~tego też powodu pełną informację zawartą w~funkcji
gęstości prawdopodobieństwa $f( x )$ definiujemy jako
\begin{equation}
  \label{eq:Winere-Cybernetics-ETC-05}
  -\int_{ -\infty }^{ +\infty } \log_{ 2 }\!\big( f( x ) \big) \, f( x ) \, dx.
\end{equation}
Z~dokładnością do znaku, jest to ten sam wzór na~ilości informacji,
którego używa Wiener, przy czym zmienna znaku z~plusa na~minus i~odwrotnie,
nie ma wielkiego znaczenia dla rozwijanej teorii. To co ma znaczenie, to
fakt, że~aby wzór \eqref{eq:Winere-Cybernetics-ETC-05} mógł być stosowany
do~szerokiej klasy rozkładów prawdopodobieństwa, musimy przyjąć wcale nie
oczywistą konwencję
\begin{equation}
  \label{eq:Winere-Cybernetics-ETC-06}
  0 \log_{ 2 }( 0 ) = 0.
\end{equation}

\VerSpaceFour










% \noindent
% \Str{23} Na tej stronie podany jest wzór
% \begin{equation}
%   \label{eq:Gancarzewicz-Arytmetyka-01}
%   a_{ n } =
%   \frac{ 1 }{ \sqrt{ 5 } }
%   \Big\{ \Big( \frac{ 1 + \sqrt{ 5 } }{ 2 } \Big)^{ n }
%   - \Big( \frac{ 1 - \sqrt{5} }{ 2 } \Big)^{ n } \Big\}
% \end{equation}
% nie wygląda najlepiej, ze względu na to jak wysokość nawiasów ma~się
% do wysokości zawartych w~nich wyrażeń. Według mnie, znaczniej lepiej
% wyglądały on, gdyby został zapisany w~następujący sposób
% \begin{equation}
%   \label{eq:Gancarzewicz-Arytmetyka-01}
%   a_{ n } =
%   \frac{ 1 }{ \sqrt{ 5 } }
%   \left\{ \left( \frac{ 1 + \sqrt{ 5 } }{ 2 } \right)^{ n }
%   - \left( \frac{ 1 - \sqrt{5} }{ 2 } \right)^{ n } \right\},
% \end{equation}
% ewentualnie
% \begin{equation}
%   \label{eq:Gancarzewicz-Arytmetyka-02}
%   a_{ n } =
%   \frac{ 1 }{ \sqrt{ 5 } }
%   \left\{ \left( \tfrac{ 1 + \sqrt{ 5 } }{ 2 } \right)^{ n }
%   - \left( \tfrac{ 1 - \sqrt{5} }{ 2 } \right)^{ n } \right\}.
% \end{equation}

% W~dalszych ciągu tych notatek, poza wyjątkowymi sytuacjami nie będziemy
% odnosić~się do tego typu problemów typograficznych.

% \VerSpaceFour





% \noindent
% \Str{23} W~przeprowadzanym tutaj wyprowadzaniu wzoru (2.6) pojawia się
% następujący problem. Czy dopuszczamy by zmienne $a$ i~$b$ w~nim występujące
% mogły przyjmować wartość $0$? Jeśli odpowiedź na to pytanie jest twierdząca,
% to stajemy przed problemem tego, iż~we wzorze (2.6) pojawiają~się
% człony $a^{ 0 }$ i~$b^{ 0 }$. Ponieważ jeśli $b = 0$ to poszukiwana zależność
% redukuje się do wyrażenia $a^{ n }$ (analogicznie dla $a = 0$), proponuję by
% przyjąć, iż rozważamy tylko sytuację, gdy~$a \neq 0$ i~$b \neq 0$.

% \VerSpaceFour





% \Str{25--26} W~opis trójkąta Pascala wkradła się pewna nieścisłość.
% Mianowicie by
% narysować pierwszą linię zawierającą tylko symbol $\binom{ 0 }{ 0 }$ musimy
% wiedzieć
% ile on wynosi, a~nie jest wcale jasne, czy jego wartość wynika
% przeprowadzonych
% do tej pory rozważań. Jeśli nie to należy przyjąć, że~z~definicji
% \begin{equation}
%   \label{Gancarzewicz-Arytmetyka-03}
%   \binom{ 0 }{ 0 } := 1.
% \end{equation}

% Drugi problem dotyczy stwierdzenia, że~gdy znamy $n$-tą linię, to następną
% $( n + 1 )$-szą linię budujemy tak, że~dodajemy po dwa kolejne współczynniki
% z~poprzedniej linii. Bardziej poprawne byłoby stwierdzenie, iż~na początku
% i~końcu $( n + 1 )$-szej linii zapisujemy wartość symboli
% $\binom{ n + 1 }{ 0 } = \binom{ n + 1 }{ n + 1 } = 1$, pozostałe zaś
% współczynniki otrzymujemy dodając dwa kolejne współczynniki z~linii $n$-tej,
% tak jak pokazano to na rysunku. Kwestia tego, gdzie jest początek i~koniec
% linii, gdzie musimy wpisać wartości symboli $\binom{ n + 1 }{ 0 }$
% i~$\binom{ n + 1 }{ n + 1 }$ w~praktyce nie stanowi problemu, dlatego nie
% będziemy~się w~ten problem zagłębiać.

% \VerSpaceFour





% \noindent
% \Str{26} Przedstawiony tu rysunek z~trójkątem Pascala można by uczynić
% ładniejszym na wiele różnych sposobów.

% \VerSpaceFour





% \noindent
% \Str{41} Od tego miejsca rozróżnienie między cyfrą, a~liczbą składającą~się
% z~jednej cyfry zaczyna nabierać specjalnego znaczenia, choć ta książeczka
% nie poświęca temu wiele miejsca. Oczywiście, często dla wygody nie
% rozróżnia~się tych dwóch bytów, warto jednak uprzednio przyjrzeć się temu
% zagadnieniu dokładnie.

% Cyfry to symbole z~pewnego ustalonego zbioru, które będziemy notować
% w~nawiasach. Czyli „$0$” jest jedną spośród cyfr arabskich. Samym cyfrom
% nie przypisujemy sensu liczbowego, sens ten posiadają dopiero ciągi cyfr
% skonstruowane według odpowiednich reguł. Przykładowo, często za ciągi cyfr
% reprezentujące liczby uważa~się tylko te które zaczynają~się, czytając od
% lewej do prawej, od cyfry różnej od~„$0$”, z~wyjątkiem ciągu składającego
% się tylko z~jednego elementu: $0$. Taki ciąg oznacza liczbę zero. Inny
% przykładem liczby skonstruowanej z~cyfr arabskich jest $17$, która
% jest dwuelementowym ciągiem cyfr „$1$”, „$7$”.

% Wśród ciągów cyfr występują ciągi o~długości jeden, takie jak wspomniany już
% ciąg $0$ lub $1$, istnieje jednak różnica między cyfrą „$1$” i~liczbą $1$.
% Cyfra „$1$” to symbol który nie oznacza żadnej wartości liczbowej, podczas
% gdy $1$ to jednoelementowy ciąg cyfr oznaczający liczbę jeden.

% W~tych rozważaniach nie zajęliśmy~się problemem, czy dany poprawny ciąg
% cyfr, powiedzmy $17$, \textit{jest} liczbą siedemnaście, czy
% \textit{oznacza} liczbę siedemnaście? Jest to ważkie zagadnienie z~zakresu
% filozofii matematyki i~zapewne innych działów ludzkiej wiedzy, lecz w~tym
% momencie nasza ignorancja nie pozwala nam zabrać na jego temat głosu.

% \VerSpaceFour





% \noindent
% \StrWierszGora{41}{11} Odwołanie do przypisu w~tej linii wygląda naprawdę
% brzydko, ale nie wiem jak można byłoby to poprawić.

% \VerSpaceFour





% \noindent
% \Str{42} Według mnie symbol $\mod$ oznaczający \textbf{równość dwóch liczb
%   modulo $k$} wygląda znacznie lepiej, niż używany w~tej książce symbol
% $mod$. Dlatego też w~tych notatkach zawsze będę~się posługiwał pierwszą
% z~podanych wersji.

% \VerSpaceFour





% \noindent
% \Str{43} Kiedy już wiemy, że istnieje jedna i~tylko jedna liczba całkowita
% $k$, dla której zachodzi
% \begin{equation}
%   \label{eq:Gancarzewicz-Arytmetyka-04}
%   b = k a + r,
% \end{equation}
% to możemy określić $r = b - k a$. tylko czy to dowodzi jednoznaczności $r$?
% Jeśli nawet nie, to łatwo ten fakt udowodnić. Załóżmy, że~zachodzi
% \begin{subequations}
%   \begin{align}
%     \label{eq:Gancarzewicz-Arytmetyka-05-A}
%     b = k a + r_{ 1 }, \\
%     \label{eq:Gancarzewicz-Arytmetyka-05-B}
%     b = k a + r_{ 2 }.
%   \end{align}
% \end{subequations}
% Od~razu otrzymujemy
% \begin{equation}
%   \label{eq:Gancarzewicz-Arytmetyka-06}
%   k a + r_{ 1 } = k a + r_{ 2 } \quad \iff \quad r_{ 1 } = r_{ 2 }.
% \end{equation}

% \VerSpaceFour





% \Str{43} Analiza problemu tego czy liczba $b$ jest podzielna przez
% liczbę~$a$, który jest podana w~dalszej części książeczki, byłby prostsza,
% gdyby w~tym miejscu zostało napisane jawnie, że~z~definicji podzielność
% liczby $b$ przez liczbę $a$ jest równoważna warunkowi
% \begin{equation}
%   \label{eq:Gancarzewicz-Arytmetyka-07}
%   b = 0 ( \mod a ).
% \end{equation}
% Ten zaś warunek jest równoważny
% \begin{equation}
%   \label{eq:Gancarzewicz-Arytmetyka-08}
%   0 = b ( \mod a ).
% \end{equation}

% \VerSpaceFour





% \noindent
% \Str{47} Na tej stronie podana jest informacja, że~za poprawnie
% skonstruowany ciąg cyfr, który reprezentuje liczbę uznaje~się ciąg, którego
% pierwsza cyfra, czytając od lewej do prawej jest różna od „$0$”.
% Dodaje też, że~niekiedy rezygnuje~się z~tego warunku. Pominięto jednak
% uwagę, że~ciąg jednoelementowy $0$ zawsze uważa~się za poprawny ciąg cyfr,
% reprezentujący liczbę zero.

% \VerSpaceFour





% \noindent
% \Str{51} Na tej stronie Gancarzewicz pisze o~„alfabecie (staro-)chińskim”.
% Wydaje mi~się, że~z~punktu widzenia językoznawstwa, nie jest to alfabet,
% ale pewien system hieroglificzny. Ponieważ jest to jednak książka
% do~matematyki, nie do językoznawstwa, możemy wybaczyć autorowi ewentualne
% nieścisłości.

% \VerSpaceFour

















% ##################
\newpage

\CenterBoldFont{Błędy}

\VerSpaceFive


\begin{center}

  \begin{tabular}{|c|c|c|c|c|}
    \hline
    Strona & \multicolumn{2}{c|}{Wiersz} & Jest
    & Powinno być \\ \cline{2-3}
    & Od góry & Od dołu & & \\
    \hline
    50 & \hphantom{0}5 & & 2.05 & 2.06 \\
    60 & & \hphantom{0}4 & $f_{ 1 }( x ) \;\: dx$ & $f_{ 1 }( x ) \, dx$ \\
    60 & & \hphantom{0}4 & $f_{ 2 }( x ) \;\: dx$ & $f_{ 2 }( x ) \, dx$ \\
    62 & \hphantom{0}2 & & (3.081) & (3.08.1) \\
    62 & \hphantom{0}4 & & (3.082) & (3.08.2) \\
    62 & \hphantom{0}6 & & (3.083) & (3.08.3) \\
    62 & \hphantom{0}8 & & (3.084) & (3.08.4) \\
    62 & & \hphantom{0}5
    & $\exp^{ ( -x^{ 2 } / 2b ) \, dx / \sqrt{ 2 \pi b } }$
    & $\exp( -x^{ 2 } / 2b ) \frac{ dx }{ \sqrt{ 2 \pi b } }$ \\
    62 & & \hphantom{0}7
    & $\exp^{ ( -x^{ 2 } / 2a ) \, dx / \sqrt{ 2 \pi a } }$
    & $\exp( -x^{ 2 } / 2a ) \frac{ dx }{ \sqrt{ 2 \pi a } }$ \\
    & \hphantom{0}9 & & (3.091) & (3.09.1) \\
    % & & & & \\
    \hline
  \end{tabular}

\end{center}

\VerSpaceTwo


% \StrWierszeDol{24}{4--5} \\
% \Jest jeszcze raz zmienić wskaźnik sumacyjny, tym razem $i$ zamieniamy
% na~$j$, \\
% \PowinnoByc zmienić nazwę wskaźnika sumacyjnego z~$i$ na~$j$ \\
% \textbf{Str. 49, trzecia kolumna tabelki, wiersz 7.} \\
% \Jest $= 1632$ \\
% \PowinnoByc $= 1642$ \\
% \StrWierszeDol{54}{14} \\
% \Jest \textit{są jedną z~następującyc par: $00$, $25$. $50$, $75$} \\
% \PowinnoByc \textit{tworzą jedną z~następujących liczb: $00$, $25$, $50$,
%   $75$} \\
% ############################










% ####################################################################
% ####################################################################
% Bibliography

\printbibliography





% ############################
% End of the document

\end{document}
