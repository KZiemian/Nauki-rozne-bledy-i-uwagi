% ------------------------------------------------------------------------------------------------------------------
% Basic configuration and packages
% ------------------------------------------------------------------------------------------------------------------
% Package for discovering wrong and outdated usage of LaTeX.
% More information to be found in l2tabu English version.
\RequirePackage[l2tabu, orthodox]{nag}
% Class of LaTeX document: {size of paper, size of font}[document class]
\documentclass[a4paper,11pt]{article}



% ------------------------------------------------------
% Packages not tied to particular normal language
% ------------------------------------------------------
% This package should improved spaces in the text
\usepackage{microtype}
% Add few important symbols, like text Celcius degree
\usepackage{textcomp}



% ------------------------------------------------------
% Polonization of LaTeX document
% ------------------------------------------------------
% Basic polonization of the text
\usepackage[MeX]{polski}
% Switching on UTF-8 encoding
\usepackage[utf8]{inputenc}
% Adding font Latin Modern
\usepackage{lmodern}
% Package is need for fonts Latin Modern
\usepackage[T1]{fontenc}



% ------------------------------------------------------
% Setting margins
% ------------------------------------------------------
\usepackage[a4paper, total={14cm, 25cm}]{geometry}



% ------------------------------------------------------
% Setting vertical spaces in the text
% ------------------------------------------------------
% Setting space between lines
\renewcommand{\baselinestretch}{1.1}

% Setting space between lines in tables
\renewcommand{\arraystretch}{1.4}



% ------------------------------------------------------
% Packages for scientific papers
% ------------------------------------------------------
% Switching off \lll symbol, that I guess is representing letter "Ł"
% It collide with `amsmath' package's command with the same name
\let\lll\undefined
% Basic package from American Mathematical Society (AMS)
\usepackage[intlimits]{amsmath}
% Equations are numbered separately in every section
\numberwithin{equation}{section}

% Other very useful packages from AMS
\usepackage{amsfonts}
\usepackage{amssymb}
\usepackage{amscd}
\usepackage{amsthm}

% Package with better looking calligraphy fonts
\usepackage{calrsfs}

% Package with better looking greek letters
% Example of use: pi -> \uppi
\usepackage{upgreek}
% Improving look of lambda letter
\let\oldlambda\Lambda
\renewcommand{\lambda}{\uplambda}




% ------------------------------------------------------
% BibLaTeX
% ------------------------------------------------------
% Package biblatex, with biber as its backend, allow us to handle
% bibliography entries that use Unicode symbols outside ASCII
\usepackage[
language=polish,
backend=biber,
style=alphabetic,
url=false,
eprint=true,
]{biblatex}

\addbibresource{Logika-i-teoria-mnogości-Bibliography.bib}





% ------------------------------------------------------
% Defining new environments (?)
% ------------------------------------------------------
% Defining enviroment "Wniosek"
\newtheorem{corollary}{Wniosek}
\newtheorem{definition}{Definicja}
\newtheorem{theorem}{Twierdzenie}





% ------------------------------------------------------
% Wonderful package PGF/TikZ
% ------------------------------------------------------
% \usepackage{tikz}

% % Loading concrete TikZ libraries
% \usetikzlibrary{decorations.markings}

% % Package with TikZ pics
% \usepackage{TikZPics}

% % Package with additional TikZ styles
% \usepackage{TikZStyles}





% ------------------------------------------------------
% Private packages
% You need to put them in the same directory as .tex file
% ------------------------------------------------------
% Contains various command useful for working with a text
\usepackage{latexgeneralcommands}
% Contains definitions useful for working with mathematical text
% \usepackage{mathcommands}





% ------------------------------------------------------
% Package "hyperref"
% They advised to put it on the end of preambule
% ------------------------------------------------------
% It allows you to use hyperlinks in the text
\usepackage{hyperref}










% ------------------------------------------------------------------------------------------------------------------
% Title and author of the text
\title{Ekonomia \\
  {\Large Błędy i~uwagi}}

\author{Kamil Ziemian}


% \date{}
% ------------------------------------------------------------------------------------------------------------------










% ####################################################################
% Początek dokumentu
\begin{document}
% ####################################################################





% ######################################
\maketitle % Tytuł całego tekstu
% ######################################





% ######################################
\section{Bardzo wpływowe dzieła}

\VerSpaceTwo
% ######################################



% ############################
\section{ % Autor i tytuł dzieła
  Adam Smith \\
  \textit{Badania nad naturą i~przyczynami bogactw narodów. Tom~I},
  \cite{} }


% ##################
\CenterBoldFont{Błędy}


\begin{center}

  \begin{tabular}{|c|c|c|c|c|}
    \hline
    Strona & \multicolumn{2}{c|}{Wiersz} & Jest
                              & Powinno być \\ \cline{2-3}
    & Od góry & Od dołu & & \\
    \hline
    VII & & 1 & (1859) & (1759) \\
    % & & & & \\
    \hline
  \end{tabular}

\end{center}

\VerSpaceTwo

% ############################











% ######################################
\section{Austriacka szkoła ekonomii}

\VerSpaceTwo
% ######################################



% ############################
\section{ % Autor i tytuł dzieła
  J\"{o}rg Guido H\"{u}lsmann \\
  \textit{Etyka produkcji pieniądza},
  \cite{HulsmannEtykaProdukcjiPieniadza2014}}

\vspace{0em}


% ##################
\CenterBoldFont{Uwagi do~konkretnych stron}

\vspace{0em}


\noindent
\Str{II} Logo firmy NowePrzetargi.pl zostało wydrukowane w słabej jakości.

\VerSpaceFour





\noindent
\StrWierszDol{XXII}{9--10} Według tego co jest tu napisane, dzieło Habigera na
temat okresu 1891--1991 ukazało~się w~roku 1990. To zapewne wynik błędu
drukarskiego.

\VerSpaceFour





\noindent
\StrWierszGora{4}{9--10} Może wydawać~się dziwne, że w tym kontekście
wymienione jest
ósme przykazanie\footnote{Używany jest tu katolicki podział Dekalogu.}. Może
to być błąd, możliwe też, że~H\"{u}lsmann miał tu na myśli związek jak ma to
przykazanie z~kwestią produkcji pieniądza, co omówił na stronie 23
niniejszej książki

\VerSpaceFour





\noindent
\StrWierszGora{57}{6} W~tej linii pojawia~się anglicyzm „run na bank”. Jak
jest
wyjaśnione dalej w~tym samym paragrafie pochodzi to od angielskiego „bank
run”. Wydaje mi~się, że~znacznie lepszym terminem byłby „nalot na bank” albo
„najazd na bank”.

\VerSpaceFour





\noindent
\textbf{Tylna okładka, wiersz 3 od dołu.} Odstępy w tym wierszu są
zbyt duże.








% ##################
\CenterBoldFont{Błędy}


\begin{center}

  \begin{tabular}{|c|c|c|c|c|}
    \hline
    Strona & \multicolumn{2}{c|}{Wiersz} & Jest
                              & Powinno być \\ \cline{2-3}
    & Od góry & Od dołu & & \\
    \hline
    XXIV  & & 20 & pieniędzy & pieniędzy” \\
    XXIV  & & 20 & s.~160)”. & s.~160). \\
    XXVII & & 10 & Juan de Mariana: & \textit{Juan de Mariana:} \\
    % & & & & \\
    \hline
  \end{tabular}





  % \begin{tabular}{|c|c|c|c|c|}
  %   \hline
  %   & \multicolumn{2}{c|}{} & & \\
  %   Strona & \multicolumn{2}{c|}{Wiersz} & Jest
  %   & Powinno być \\ \cline{2-3}
  %   & Od góry & Od dołu & & \\
  %   \hline
  %   %   & & & & \\
  %   %   & & & & \\
  %   %   & & & & \\
  %   %   & & & & \\
  %   %   & & & & \\
  %   %   & & & & \\
  %   %   & & & & \\
  %   %   & & & & \\
  %   %   & & & & \\
  %   %   & & & & \\
  %   %   & & & & \\
  %   %   & & & & \\
  %   %   & & & & \\
  %   %   & & & & \\
  %   %   & & & & \\
  %   %   & & & & \\
  %   %   & & & & \\
  %   %   & & & & \\
  %   %   & & & & \\
  %   %   & & & & \\
  %   %   & & & & \\
  %   %   & & & & \\
  %   %   & & & & \\
  %   %   & & & & \\
  %   %   & & & & \\
  %   %   & & & & \\
  %   %   & & & & \\
  %   %   & & & & \\
  %   %   & & & & \\
  %   %   & & & & \\
  %   %   & & & & \\
  %   %   & & & & \\
  %   %   & & & & \\
  %   %   & & & & \\
  %   %   & & & & \\
  %   %   & & & & \\
  %   %   & & & & \\
  %   %   & & & & \\
  %   \hline
  % \end{tabular}





  % \begin{tabular}{|c|c|c|c|c|}
  %   \hline
  %   Strona & \multicolumn{2}{c|}{Wiersz} & Jest
  %   & Powinno być \\ \cline{2-3}
  %   & Od góry & Od dołu & & \\
  %   \hline
  %   %   & & & & \\
  %   %   & & & & \\
  %   %   & & & & \\
  %   %   & & & & \\
  %   %   & & & & \\
  %   %   & & & & \\
  %   %   & & & & \\
  %   %   & & & & \\
  %   %   & & & & \\
  %   %   & & & & \\
  %   %   & & & & \\
  %   %   & & & & \\
  %   %   & & & & \\
  %   %   & & & & \\
  %   %   & & & & \\
  %   %   & & & & \\
  %   %   & & & & \\
  %   %   & & & & \\
  %   %   & & & & \\
  %   %   & & & & \\
  %   %   & & & & \\
  %   %   & & & & \\
  %   %   & & & & \\
  %   %   & & & & \\
  %   %   & & & & \\
  %   %   & & & & \\
  %   %   & & & & \\
  %   %   & & & & \\
  %   %   & & & & \\
  %   %   & & & & \\
  %   %   & & & & \\
  %   %   & & & & \\
  %   %   & & & & \\
  %   %   & & & & \\
  %   %   & & & & \\
  %   %   & & & & \\
  %   %   & & & & \\
  %   %   & & & & \\
  %   \hline
  % \end{tabular}

\end{center}

\VerSpaceTwo


\noindent
\StrWierszDol{13}{17} \\
\Jest jako naturalni przywódcy \\
\PowinnoByc jako będący naturalnymi przywódcami \\
\StrWierszDol{58}{24}
\Jest ofiar pokusy zarządców magazynów pieniędzy i~bankierów. \\
\PowinnoByc zarządców magazynów pieniędzy i~bankierów, którzy stali~się
ofiarami rządy zysku. \\
\StrWierszGora{59}{6} \\
\Jest depozytów otrzymanych w~pożyczkach \\
\PowinnoByc przetrzymywanych depozytów w~pożyczki \\
% \StrWd{11}{5} \\
% \Jest   \\
% \Powin  \\
% \StrWd{11}{1} \\
% \Jest   \\
% \Powin  \\
% \StrWg{12}{11} \\
% \Jest   \\
% \Powin  \\
% \StrWd{12}{18} \\
% \Jest   \\
% \Powin  \\
% \StrWd{12}{8} \\
% \Jest  (\textit{Froce Ouvri\'{e}re} ) \\
% \Powin (\texit{Froce Ouvri\'{e}re} --~Główna Konfederacja Pracy
% --~Siły
% Pracy) \\
% \StrWg{13}{10} \\
% \Jest   \\
% \Powin  \\
% \StrWd{13}{4} \\
% \Jest   \\
% \Powin  \\
% \StrWg{15}{1} \\
% \Jest  Polish American Congress Charitable Foundation \\
% \Powin \textit{Polish American Congress Charitable Foundation} \\
% \StrWg{15}{3} \\
% \Jest  Polish-American Enterprise Fund \\
% \Powin \textit{Polish-American Enterprise Fund} \\
% \StrWg{15}{10} \\
% \Jest  Postal, Telegraph and~Telephone International \\
% \Powin \textit{Postal, Telegraph and~Telephone International} \\
% \StrWg{43}{2} \\
% \Jest  \textit{Univeristy~of Illinois at~Urbana Champaign, 1998} \\
% \Powin Univeristy~of Illinois at~Urbana-Champaign, 1998 \\



% ############################










% ############################
\section{ % Autor i tytuł dzieła
  Murray N. Rothbard \\
  \textit{Złoto, banki, ludzie. Krótka historia pieniądza},
  \cite{RothbardZlotoBankiLudzie2016}}


% ##################
\CenterBoldFont{Uwagi}





% ##################
\CenterBoldFont{Uwagi do~konkretnych stron}


% \vspace{\spaceFour}





% ##################
\CenterBoldFont{Błędy}

\begin{center}

  \begin{tabular}{|c|c|c|c|c|}
    \hline
    Strona & \multicolumn{2}{c|}{Wiersz} & Jest
                              & Powinno być \\ \cline{2-3}
    & Od góry & Od dołu & & \\
    \hline
    157 & 11 & & poprawianym[...] & poprawianym [...] \\
    159 & & 16 & Busha & H.W. Busha \\
    159 & &  5 & www.cato.org & www.cato.org. \\
    % & & & & \\
    165 & &  6 & 19-29 & 19--29 \\
    165 & &  1 & (1723-1790) & \textit{(1723--1790)} \\
    166 & &  1 & 74-80 & 74--80 \\
    169 & &  3 & Bohm-Bawerka & B\"{o}hm-Bawerka \\
    170 & &  2 & 1929-1933 & 1929--1933 \\
    % & & & & \\
    \hline
  \end{tabular}





  % \begin{tabular}{|c|c|c|c|c|}
  %   \hline
  %   & \multicolumn{2}{c|}{} & & \\
  %   Strona & \multicolumn{2}{c|}{Wiersz} & Jest
  %   & Powinno być \\ \cline{2-3}
  %   & Od góry & Od dołu & & \\
  %   \hline
  %   %   & & & & \\
  %   %   & & & & \\
  %   %   & & & & \\
  %   %   & & & & \\
  %   %   & & & & \\
  %   %   & & & & \\
  %   %   & & & & \\
  %   %   & & & & \\
  %   %   & & & & \\
  %   %   & & & & \\
  %   %   & & & & \\
  %   %   & & & & \\
  %   %   & & & & \\
  %   %   & & & & \\
  %   %   & & & & \\
  %   %   & & & & \\
  %   %   & & & & \\
  %   %   & & & & \\
  %   %   & & & & \\
  %   %   & & & & \\
  %   %   & & & & \\
  %   %   & & & & \\
  %   %   & & & & \\
  %   %   & & & & \\
  %   %   & & & & \\
  %   %   & & & & \\
  %   %   & & & & \\
  %   %   & & & & \\
  %   %   & & & & \\
  %   %   & & & & \\
  %   %   & & & & \\
  %   %   & & & & \\
  %   %   & & & & \\
  %   %   & & & & \\
  %   %   & & & & \\
  %   %   & & & & \\
  %   %   & & & & \\
  %   %   & & & & \\
  %   \hline
  % \end{tabular}





  % \begin{tabular}{|c|c|c|c|c|}
  %   \hline
  %   & \multicolumn{2}{c|}{} & & \\
  %   Strona & \multicolumn{2}{c|}{Wiersz} & Jest
  %   & Powinno być \\ \cline{2-3}
  %   & Od góry & Od dołu & & \\
  %   \hline
  %   %   & & & & \\
  %   %   & & & & \\
  %   %   & & & & \\
  %   %   & & & & \\
  %   %   & & & & \\
  %   %   & & & & \\
  %   %   & & & & \\
  %   %   & & & & \\
  %   %   & & & & \\
  %   %   & & & & \\
  %   %   & & & & \\
  %   %   & & & & \\
  %   %   & & & & \\
  %   %   & & & & \\
  %   %   & & & & \\
  %   %   & & & & \\
  %   %   & & & & \\
  %   %   & & & & \\
  %   %   & & & & \\
  %   %   & & & & \\
  %   %   & & & & \\
  %   %   & & & & \\
  %   %   & & & & \\
  %   %   & & & & \\
  %   %   & & & & \\
  %   %   & & & & \\
  %   %   & & & & \\
  %   %   & & & & \\
  %   %   & & & & \\
  %   %   & & & & \\
  %   %   & & & & \\
  %   %   & & & & \\
  %   %   & & & & \\
  %   %   & & & & \\
  %   %   & & & & \\
  %   %   & & & & \\
  %   %   & & & & \\
  %   %   & & & & \\
  %   \hline
  % \end{tabular}

\end{center}

\VerSpaceTwo


\noindent
\StrWierszDol{156}{2} \\
\Jest Prekursorzy Nowej Lewicy. Studia z~myśli społecznej XIX i~XX wieku \\
\PowinnoByc \textit{Prekursorzy Nowej Lewicy. Studia z~myśli społecznej XIX
  i~XX wieku} \\


% \Jest  Amerian Seafarers Union \\
% \Powin \textit{Amerian Seafarers Union} \\



% \StrWd{11}{5} \\
% \Jest  Council~of Economic Advisers \\
% \Powin \textit{Council~of Economic Advisers} \\
% \StrWd{11}{1} \\
% \Jest  Council on~Foregin Relations \\
% \Powin \textit{Council on~Foregin Relations} \\
% \StrWg{12}{11} \\
% \Jest  \textit{Congress}) --~Kanadyjski Kongres Związków Zawodowych \\
% \Powin \textit{Congress} --~Kanadyjski Kongres Związków Zawodowych) \\
% \StrWd{12}{18} \\
% \Jest  Emergency Committe for~Aid to~Poland \\
% \Powin \textit{Emergency Committe for~Aid to~Poland} \\
% \StrWd{12}{8} \\
% \Jest  (\textit{Froce Ouvri\'{e}re} ) \\
% \Powin (\textit{Froce Ouvri\'{e}re} --~Główna Konfederacja Pracy
% --~Siły Pracy) \\
% \StrWg{13}{10} \\
% \Jest Generalized System~of Preferences \\
% \Powin  \textit{Generalized System~of Preferences} \\
% \StrWd{13}{4} \\
% \Jest
% \Powin
% Katolickich Uniwersytetu Leuven \\
% \StrWg{15}{1} \\
% \Jest
% \Powin
% \StrWg{15}{3} \\
% \Jest
% \Powin
% \StrWg{15}{10} \\
% \Jest
% \Powin
% \StrWg{43}{2} \\
% \Jest
% \Powin


% ############################












% ######################################
\newpage

\section{Historia ekonomii}

\VerSpaceTwo
% ######################################




% ############################
\section{ % Autor i tytuł dzieła
  Red. Randall G. Holcombe \\
  \textit{15 wielkich austriackiej szkoły ekonomii},
  \cite{}}

\vspace{0em}


% ##################
\CenterBoldFont{Uwagi}

\vspace{0em}


Gdzieś powinna być starsza wersja uwag do tej książki.????

\VerSpaceFour








% ##################
\CenterBoldFont{Uwagi do~konkretnych stron}


\noindent
\StrWierszDol{14}{2--3} Należy zwrócić uwagę na to, że~według informacji
które można znaleźć na~stronie 209 niniejszej książki, wspomniany tu tekst
Murray’a N.
Rothbarda \textit{Ludwig von~Mises: Scholar, Creator, Hero}
w~przeredagowanej wersji został do niej włączony pod tytułem \textit{Ludwig
  von~Mises. Przywódca szkoły austriackiej}.

% \vspace{\spaceFour}







% ##################
\newpage

\CenterBoldFont{Błędy}

\begin{center}

  \begin{tabular}{|c|c|c|c|c|}
    \hline
    Strona & \multicolumn{2}{c|}{Wiersz} & Jest
    & Powinno być \\ \cline{2-3}
    & Od góry & Od dołu & & \\
    \hline
    10  & &  4 & należące & nie należące \\
    12  & &  3 & Rothbard [w:] & Rothbard [w:] \\
    %   & & & & \\
    %   & & & & \\
    %   & & & & \\
    %   & & & & \\
    %   & & & & \\
    %   & & & & \\
    %   & & & & \\
    %   & & & & \\
    %   & & & & \\
    %   & & & & \\
    %   & & & & \\
    %   & & & & \\
    %   & & & & \\
    %   & & & & \\
    %   & & & & \\
    %   & & & & \\
    %   & & & & \\
    %   & & & & \\
    %   & & & & \\
    %   & & & & \\
    %   & & & & \\
    %   & & & & \\
    %   & & & & \\
    %   & & & & \\
    %   & & & & \\
    %   & & & & \\
    %   & & & & \\
    %   & & & & \\
    %   & & & & \\
    %   & & & & \\
    %   & & & & \\
    %   & & & & \\
    %   & & & & \\
    %   & & & & \\
    %   & & & & \\
    %   & & & & \\
    \hline
  \end{tabular}





  %% \begin{tabular}{|c|c|c|c|c|}
  %%   \hline
  %%   & \multicolumn{2}{c|}{} & & \\
  %%   Strona & \multicolumn{2}{c|}{Wiersz} & Jest
  %%   & Powinno być \\ \cline{2-3}
  %%   & Od góry & Od dołu & & \\
  %%   \hline
  %%   %   & & & & \\
  %%   %   & & & & \\
  %%   %   & & & & \\
  %%   %   & & & & \\
  %%   %   & & & & \\
  %%   %   & & & & \\
  %%   %   & & & & \\
  %%   %   & & & & \\
  %%   %   & & & & \\
  %%   %   & & & & \\
  %%   %   & & & & \\
  %%   %   & & & & \\
  %%   %   & & & & \\
  %%   %   & & & & \\
  %%   %   & & & & \\
  %%   %   & & & & \\
  %%   %   & & & & \\
  %%   %   & & & & \\
  %%   %   & & & & \\
  %%   %   & & & & \\
  %%   %   & & & & \\
  %%   %   & & & & \\
  %%   %   & & & & \\
  %%   %   & & & & \\
  %%   %   & & & & \\
  %%   %   & & & & \\
  %%   %   & & & & \\
  %%   %   & & & & \\
  %%   %   & & & & \\
  %%   %   & & & & \\
  %%   %   & & & & \\
  %%   %   & & & & \\
  %%   %   & & & & \\
  %%   %   & & & & \\
  %%   %   & & & & \\
  %%   %   & & & & \\
  %%   %   & & & & \\
  %%   %   & & & & \\
  %%   \hline
  %% \end{tabular}

\end{center}

\VerSpaceTwo


\noindent
% \Jest  Amerian Seafarers Union \\
% \Powin \textit{Amerian Seafarers Union} \\



% \StrWd{11}{5} \\
% \Jest  Council~of Economic Advisers \\
% \Powin \textit{Council~of Economic Advisers} \\
% \StrWd{11}{1} \\
% \Jest  Council on~Foregin Relations \\
% \Powin \textit{Council on~Foregin Relations} \\
% \StrWg{12}{11} \\
% \Jest  \textit{Congress}) --~Kanadyjski Kongres Związków Zawodowych \\
% \Powin \textit{Congress} --~Kanadyjski Kongres Związków Zawodowych) \\
% \StrWd{12}{18} \\
% \Jest  Emergency Committe for~Aid to~Poland \\
% \Powin \textit{Emergency Committe for~Aid to~Poland} \\
% \StrWd{12}{8} \\
% \Jest  (\textit{Froce Ouvri\'{e}re} ) \\
% \Powin (\textit{Froce Ouvri\'{e}re} --~Główna Konfederacja Pracy
% --~Siły Pracy) \\
% \StrWg{13}{10} \\
% \Jest Generalized System~of Preferences \\
% \Powin  \textit{Generalized System~of Preferences} \\
% \StrWd{13}{4} \\
% \Jest
% \Powin
% Katolickich Uniwersytetu Leuven \\
% \StrWg{15}{1} \\
% \Jest
% \Powin
% \StrWg{15}{3} \\
% \Jest
% \Powin
% \StrWg{15}{10} \\
% \Jest
% \Powin
% \StrWg{43}{2} \\
% \Jest
% \Powin


% ############################








% ####################################################################
% ####################################################################
% Bibliography

\printbibliography





% ############################
% End of the document

\end{document}
