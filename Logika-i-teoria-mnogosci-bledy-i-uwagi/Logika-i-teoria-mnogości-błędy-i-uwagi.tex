% Autor: Kamil Ziemian

% ---------------------------------------------------------------------
% Podstawowe ustawienia i pakiety
% ---------------------------------------------------------------------
\RequirePackage[l2tabu, orthodox]{nag}  % Wykrywa przestarzałe i niewłaściwe
% sposoby używania LaTeXa. Więcej jest w l2tabu English version.
\documentclass[a4paper,11pt]{article}
% {rozmiar papieru, rozmiar fontu}[klasa dokumentu]
\usepackage[MeX]{polski}  % Polonizacja LaTeXa, bez niej będzie pracował
% w języku angielskim.
\usepackage[utf8]{inputenc}  % Włączenie kodowania UTF-8, co daje dostęp
% do polskich znaków.
\usepackage{lmodern}  % Wprowadza fonty Latin Modern.
\usepackage[T1]{fontenc}  % Potrzebne do używania fontów Latin Modern.



% ------------------------------
% Podstawowe pakiety (niezwiązane z ustawieniami języka)
% ------------------------------
\usepackage{microtype}  % Twierdzi, że poprawi rozmiar odstępów w tekście.
% \usepackage{graphicx}  % Wprowadza bardzo potrzebne komendy do wstawiania
% % grafiki.
% \usepackage{verbatim}  % Poprawia otoczenie VERBATIME.
% \usepackage{textcomp}  % Dodaje takie symbole jak stopnie Celsiusa,
% % wprowadzane bezpośrednio w tekście.
\usepackage{vmargin}  % Pozwala na prostą kontrolę rozmiaru marginesów,
% za pomocą komend poniżej. Rozmiar odstępów jest mierzony w calach.
% ------------------------------
% MARGINS
% ------------------------------
\setmarginsrb
{ 0.7in} % left margin
{ 0.6in} % top margin
{ 0.7in} % right margin
{ 0.8in} % bottom margin
{  20pt} % head height
{0.25in} % head sep
{   9pt} % foot height
{ 0.3in} % foot sep



% ------------------------------
% Często przydatne pakiety
% ------------------------------
% \usepackage{csquotes}  % Pozwala w prosty sposób wstawiać cytaty do tekstu.
\usepackage{xcolor}  % Pozwala używać kolorowych czcionek (zapewne dużo
% więcej, ale ja nie potrafię nic o tym powiedzieć).



% ------------------------------
% Pakiety do tekstów z nauk przyrodniczych
% ------------------------------
\let\lll\undefined  % Amsmath gryzie się z pakietami do języka
% polskiego, bo oba definiują komendę \lll. Aby rozwiązać ten problem
% oddefiniowuję tę komendę, ale może tym samym pozbywam się dużego Ł.
\usepackage[intlimits]{amsmath}  % Podstawowe wsparcie od American
% Mathematical Society (w skrócie AMS)
\usepackage{amsfonts, amssymb, amscd, amsthm}  % Dalsze wsparcie od AMS
% \usepackage{siunitx}  % Do prostszego pisania jednostek fizycznych
\usepackage{upgreek}  % Ładniejsze greckie litery
% Przykładowa składnia: pi = \uppi
% \usepackage{slashed}  % Pozwala w prosty sposób pisać slash Feynmana.
\usepackage{calrsfs}  % Zmienia czcionkę kaligraficzną w \mathcal
% na ładniejszą. Może w innych miejscach robi to samo, ale o tym nic
% nie wiem.



% ##########
% Tworzenie otoczeń "Twierdzenie", "Definicja", "Lemat", etc.
\newtheorem{twr}{Twierdzenie}  % Komenda wprowadzająca otoczenie
% „twr” do pisania twierdzeń matematycznych
\newtheorem{defin}{Definicja}  % Analogicznie jak powyżej
\newtheorem{wni}{Wniosek}



% ------------------------------
% Pakiety napisane przez użytkownika.
% Mają być w tym samym katalogu to ten plik .tex
% ------------------------------
\usepackage{latexshortcuts}
\usepackage{mathshortcuts}



% ---------------------------------------------------------------------
% Dodatkowe ustawienia dla języka polskiego
% ---------------------------------------------------------------------
\renewcommand{\thesection}{\arabic{section}.}
% Kropki po numerach rozdziału (polski zwyczaj topograficzny)
\renewcommand{\thesubsection}{\thesection\arabic{subsection}}
% Brak kropki po numerach podrozdziału



% ------------------------------
% Ustawienia różnych parametrów tekstu
% ------------------------------
\renewcommand{\arraystretch}{1.2}  % Ustawienie szerokości odstępów między
% wierszami w tabelach.



% ------------------------------
% Pakiet "hyperref"
% Polecano by umieszczać go na końcu preambuły.
% ------------------------------
\usepackage{hyperref}  % Pozwala tworzyć hiperlinki i zamienia odwołania
% do bibliografii na hiperlinki.





% ---------------------------------------------------------------------
% Tytuł, autor, data
\title{Logika i~teoria mnogości --~błędy i~uwagi}

% \author{}
% \date{}
% ---------------------------------------------------------------------





% ####################################################################
% Początek dokumentu
\begin{document}
% ####################################################################



% ######################################
\maketitle  % Tytuł całego tekstu
% ######################################



% ######################################
\section{Logika}

\vspace{\spaceTwo}
% ######################################



% ##################
\Work{ % Autor i tytuł dzieła
  Józef W.~Bremer \\
  „Wprowadzenie do~logiki”, \cite{BremerWprowadzenieDoLogiki2004} }


\CenterTB{Uwagi}

\start \StrWg{132}{1} Zdanie
,,$p \vee q ( ( p \land q ) \to ( p \land q ) )$'' nie ma żadnego sensu
logicznego, musiał zostać zgubiony spójnik logiczny po~pierwszym~$q$.
Niestety nie wiem który należy tam umieścić.


\CenterTB{Błędy}

\begin{center}
  \begin{tabular}{|c|c|c|c|c|}
    \hline
    & \multicolumn{2}{c|}{} & & \\
    Strona & \multicolumn{2}{c|}{Wiersz} & Jest
                              & Powinno być \\ \cline{2-3}
    & Od góry & Od dołu & & \\
    \hline
    4   &  8 & & LATEX & \LaTeX \\
    13  & &  8 & \emph{formalnej} & \emph{formalnej}; \\
    20  & & 15 & Ockhama''$^{ 7 }$. & Ockhama''$^{ 7 }$, \\
    21  & 14 & & Współczesnych & współczesnych \\
    26  & &  2 & \emph{Lwowsko-\! Warszawska} & \emph{Lwowsko-Warszawska} \\
    45  & & 10 & konkretna & konkretną \\
    86  & 10 & & przypadku.. & przypadku. \\
    88  & 11 & & średniego & pośredniego \\  % ???
    98  & &  9 & $P \ul{ e } S$ & $S \ul{ e } P$ \\
    114 &  9 & & zwane\emph{prawo} & zwane \emph{prawo} \\
    117 &  6 & & jedzie & jadą \\
    119 & & 14 & współczesnej~. & współczesnej. \\
    122 & &  4 & wniosek $\equiv P$) & wniosek $\equiv P$)'' \\
    125 & & 14 & ,,nie-analityczne  % ''
           & ,,nie-analityczne'' \\
    131 &  6 & & ,,$\neg p \vee \neg q''$ & ,,$\neg p \vee \neg q$'' \\
    131 &  6 & & $\equiv$,\hspace{2pt},$\neg ( p \land q )$''
           & $\equiv$ ,,$\neg ( p \land q )$'' \\
    132 &  8 & & $\neg( \neg p\;\;\; \land\; \neg q ) $
           & $\neg( \neg p \land \neg q ) $ \\
    132 & 17 & & $p\quad \to \quad q$ & $p \to q$ \\
    132 & & 16 & $p\quad \to \quad q$ & $p \to q$ \\
    % & & & & \\
    % & & & & \\
    % & & & & \\
    % & & & & \\
    % & & & & \\
    \hline
  \end{tabular}
\end{center}

\vspace{\spaceTwo}
% ##################










% ######################################
\newpage
\section{Logika matematyczna}

\vspace{\spaceTwo}
% ######################################



% ##################
\Work{ % Autor i tytuł dzieła
  Andrzej Grzegorczyk \\
  „Zarys logiki matematycznej”,
  \cite{GrzegorczykZarysLogikiMatematycznej1975} }


\CenterTB{Błędy}

\begin{center}

  \begin{tabular}{|c|c|c|c|c|}
    \hline
    & \multicolumn{2}{c|}{} & & \\
    Strona & \multicolumn{2}{c|}{Wiersz} & Jest
                              & Powinno być \\ \cline{2-3}
    & Od góry & Od dołu & & \\
    \hline
    29  & & 13 & $\forall x W$ & $\forall x \exists W$ \\
    % & & & & \\
    % & & & & \\
    % & & & & \\
    % & & & & \\
    % & & & & \\
    % & & & & \\
    % & & & & \\
    % & & & & \\
    % & & & & \\
    \hline
  \end{tabular}

\end{center}

\vspace{\spaceTwo}
% ##################





% ##################
\Work{ % Autor i tytuł dzieła
  Willard Van Orman Quine \\
  ,,Logika matematyczna'', \cite{QuineLogikaMatematyczna1974} }


\CenterTB{Błędy}

\begin{center}
  \begin{tabular}{|c|c|c|c|c|}
    \hline
    & \multicolumn{2}{c|}{} & & \\
    Strona & \multicolumn{2}{c|}{Wiersz} & Jest
                              & Powinno być \\ \cline{2-3}
    & Od góry & Od dołu & & \\
    \hline
    % & & & & \\
    19  & &  2 & ęzyka & języka \\
    21  & & 10 & o fałszywych poprzednikach i & o \\
    22  & 18 & & ,,jeżeli --- to ---  % ''
           & ,,jeżeli --- to ---'' \\
    27  &  5 & & ,,jeżeli --- to'' & ,,jeżeli --- to ---'' \\
    % & & & & \\
    % & & & & \\
    \hline
  \end{tabular}
\end{center}

\vspace{\spaceTwo}
% ##################










% ######################################
\newpage
\section{Teoria mnogości}

\vspace{\spaceTwo}
% ######################################



% ##################
\Work{ % Autor i tytuł dzieła
  Kazimierz Kuratowski \\
  ,,Wstęp do~teorii mnogości i~topologii'',
  \cite{KuratowskiWstepTeoriiMnogosciITopologii2004} }


\CenterTB{Uwagi}

\start \Str{10} Należy uwzględnić dwa typy dowodu używające negacji.
Pierwszy to~dowód przez \textbf{kontrapozycję}, opierający~się
na~tożsamości
\begin{equation}
  \label{eq:Kuratowski-01}
  ( \alpha \Rightarrow \beta ) \equiv ( \neg \beta \Rightarrow \neg \alpha).
\end{equation}
Drugi to~dowód \textbf{nie wprost}, opierający~się na~tożsamości
\begin{equation}
  \label{eq:Kuratowski-02}
  ( \alpha \Rightarrow \beta ) \equiv \neg ( \alpha \land \neg \beta ).
\end{equation}

\vspace{\spaceFour}





\CenterTB{Błędy}

\begin{center}
  \begin{tabular}{|c|c|c|c|c|}
    \hline
    & \multicolumn{2}{c|}{} & & \\
    Strona & \multicolumn{2}{c|}{Wiersz} & Jest
                              & Powinno być \\ \cline{2-3}
    & Od góry & Od dołu & & \\
    \hline
    11 &  3 & & \textbf{7.} & \textbf{7a.} \\
    11 &  4 & & \textbf{7a.} & \textbf{7b.} \\
    % & & & & \\
    % & & & & \\
    % & & & & \\
    % & & & & \\
    % & & & & \\
    % & & & & \\
    % & & & & \\
    % & & & & \\
    % & & & & \\
    % & & & & \\
    % & & & & \\
    % & & & & \\
    % & & & & \\
    % & & & & \\
    % & & & & \\
    % & & & & \\
    % & & & & \\
    % & & & & \\
    % & & & & \\
    % & & & & \\
    % & & & & \\
    % & & & & \\
    % & & & & \\
    % & & & & \\
    % & & & & \\
    % & & & & \\
    % & & & & \\
    % & & & & \\
    % & & & & \\
    % & & & & \\
    % & & & & \\
    % & & & & \\
    % & & & & \\
    % & & & & \\
    % & & & & \\
    % & & & & \\
    \hline
  \end{tabular}
\end{center}

\vspace{\spaceTwo}
% ##################










% ####################################################################
% ####################################################################
% Bibliografia
\bibliographystyle{plalpha} \bibliography{LibMathInfo}{}


% ############################

% Koniec dokumentu
\end{document}
