% ---------------------------------------------------------------------
% Podstawowe ustawienia i pakiety
% ---------------------------------------------------------------------
\RequirePackage[l2tabu, orthodox]{nag} % Wykrywa przestarzałe i niewłaściwe
% sposoby używania LaTeXa. Więcej jest w l2tabu English version.


\documentclass[a4paper,11pt]{article}
% {rozmiar papieru, rozmiar fontu}[klasa dokumentu]
\usepackage[MeX]{polski} % Polonizacja LaTeXa, bez niej będzie pracował
% w języku angielskim.
\usepackage[utf8]{inputenc} % Włączenie kodowania UTF-8, co daje dostęp
% do polskich znaków.
\usepackage[T1]{fontenc} % Potrzebne do używania fontów Latin Modern.
\usepackage{lmodern} % Wprowadza fonty Latin Modern.



% ------------------------------
% Podstawowe pakiety (niezwiązane z ustawieniami języka)
% ------------------------------
\usepackage{microtype} % Twierdzi, że poprawi rozmiar odstępów w tekście.
\usepackage{graphicx} % Wprowadza bardzo potrzebne komendy do wstawiania
% grafiki.
\usepackage{verbatim} % Poprawia otoczenie VERBATIME.
\usepackage{textcomp} % Dodaje takie symbole jak stopnie Celsiusa,
% wprowadzane bezpośrednio w tekście.
\usepackage{vmargin} % Pozwala na prostą kontrolę rozmiaru marginesów,
% za pomocą komend poniżej. Rozmiar odstępów jest mierzony w calach.
% ------------------------------
% MARGINS
% ------------------------------
\setmarginsrb
{ 0.7in}  % left margin
{ 0.6in}  % top margin
{ 0.7in}  % right margin
{ 0.8in}  % bottom margin
{  20pt}  % head height
{0.25in}  % head sep
{   9pt}  % foot height
{ 0.3in}  % foot sep



% ------------------------------
% Często przydatne pakiety
% ------------------------------
% \usepackage{csquotes} % Pozwala w prosty sposób wstawiać cytaty do tekstu.
\usepackage{xcolor} % Pozwala używać kolorowych czcionek (zapewne dużo
% więcej, ale ja nie potrafię nic o tym powiedzieć).



% ------------------------------
% Pakiety do tekstów z nauk przyrodniczych
% ------------------------------
\let\lll\undefined % Amsmath gryzie się z pakietami do języka
% polskiego, bo oba definiują komendę \lll. Aby rozwiązać ten problem
% oddefiniowuję tę komendę, ale może tym samym pozbywam się dużego Ł.
\usepackage[intlimits]{amsmath} % Podstawowe wsparcie od American
% Mathematical Society (w skrócie AMS)
\usepackage{amsfonts, amssymb, amscd, amsthm} % Dalsze wsparcie od AMS
\usepackage{bm}  % Daję komendę \bm do pogrubionej czcionki matematycznej
% \usepackage{siunitx} % Do prostszego pisania jednostek fizycznych
\usepackage{upgreek} % Ładniejsze greckie litery
% Przykładowa składnia: pi = \uppi
\usepackage{slashed} % Pozwala w prosty sposób pisać slash Feynmana.
\usepackage{calrsfs} % Zmienia czcionkę kaligraficzną w \mathcal
% na ładniejszą. Może w innych miejscach robi to samo, ale o tym nic
% nie wiem.



% ------------------------------
% Tworzenie środowisk (?) „Twierdzenie”, „Definicja”, „Lemat”, etc.
% ------------------------------
% Komenda wprowadzająca otoczenie „theorem” do pisania twierdzeń
% matematycznych.
\newtheorem{theorem}{Twierdzenie}
% Analogicznie jak powyżej
\newtheorem{definition}{Definicja}
\newtheorem{corollary}{Wniosek}



% ------------------------------
% Pakiety napisane przez użytkownika.
% Mają być w tym samym katalogu to ten plik .tex
% ------------------------------
\usepackage{latexgeneralcommands}
\usepackage{mathcommands}

 % Pakiet napisany między innymi dla tego pliku.
\usepackage{ODEcommands}





% ---------------------------------------------------------------------
% Dodatkowe ustawienia dla języka polskiego
% ---------------------------------------------------------------------
\renewcommand{\thesection}{\arabic{section}.}
% Kropki po numerach rozdziału (polski zwyczaj topograficzny)
\renewcommand{\thesubsection}{\thesection\arabic{subsection}}
% Brak kropki po numerach podrozdziału



% ------------------------------
% Ustawienia różnych parametrów tekstu
% ------------------------------
\renewcommand{\baselinestretch}{1.1}

% Ustawienie szerokości odstępów między wierszami w tabelach.
\renewcommand{\arraystretch}{1.4}



% ------------------------------
% Pakiet "hyperref"
% Polecano by umieszczać go na końcu preambuły.
% ------------------------------
\usepackage{hyperref} % Pozwala tworzyć hiperlinki i zamienia odwołania
% do bibliografii na hiperlinki.










% ---------------------------------------------------------------------
% Tytuł, autor, data
\title{Równania różniczkowe zwyczajne \\
  {\Large Błędy i~uwagi}}

\author{Kamil Ziemian}


% \date{}
% ---------------------------------------------------------------------










% ####################################################################
\begin{document}
% ####################################################################





% ######################################
\maketitle % Tytuł całego tekstu
% ######################################





% ##############################
\Work{ % Autor i tytuł dzieła
  Władimir Igoriewicz Arnold \\
  \textit{Równania różniczkowe zwyczajne},
  \cite{ArnoldRownaniaRozniczkoweZwyczajne1975}}


% ##################
\newpage

\CenterBoldFont{Błędy}


\begin{center}

  \begin{tabular}{|c|c|c|c|c|}
    \hline
    Strona & \multicolumn{2}{c|}{Wiersz} & Jest
                              & Powinno być \\ \cline{2-3}
    & Od góry & Od dołu & & \\
    \hline
    5   & &  7 & 1968 - 196 & 1968 - 1969 \\
    11  & 17 & & mechanice klasycznej & mechanice kwantowej \\
    15  & & 16 & rozdziale 6 & rozdziale 5 \\
    28  & 15 & & wzór (8) & wzór \\
    34  &  9 & & $\dot{ x }_{ 1 } = x_{ 2 }$ & $\dot{ x }_{ 1 } = x_{ 1 }$ \\
    47  & & 12 & $x_{ i } = \varphi_{ i }( x_{ 1 }, \ldots, x_{ n } )$
           & $x_{ i } = \varphi_{ i }( y_{ 1 }, \ldots, y_{ n } )$ \\
    53  & & 14 & obrót & obrót krzywych całkowych \\
    56  & 12 & & osobliwym & nieosobliwym \\
    61  &  7 & & $\vecxbold$???, $\vecalphabold_{ 0 }$
           & $\vecxbold$, $\vecalphabold$ \\
    64  & & 11 & ????$\vecgbold( t_{ 2 }, t_{ 1 }, \vecxbold )
                = \vecgbold^{ t_{ 2 } }_{ t_{ 1 } }( \vecxbold, t_{ 1 } )$
           & $\vecxbold^{ t_{ 2 } }_{ t_{ 1 } }( \vecxbold, t_{ 1 } )
             = \vecgbold( t_{ 2 }, t_{ 1 }, \vecxbold )$ \\
    64  & & 10 & $( \vecvarphibold( t ), t )$
           & $( t, \vecvarphibold( t ) )$ \\[0.3em]
    66  & 17 & & $\vecvbold( t, \vecxbold, \dot{ \vecalphabold } )$
           & $\vecvbold( t, \vecxbold, \vecalphabold )$??? \\[0.3em]
    70  & & 13 & $\dot{ p }_{ i } = \frac{ \partial H }{ \partial q_{ i } }$
           & $\dot{ p }_{ i } = -\frac{ \partial H }{ \partial q_{ i } }$ \\[0.3em]
    71  & & 15 & $\frac{ \partial \vecvbold_{ 0 } }{ \vecxbold }$
           & $\frac{ \partial \vecvbold_{ 0 } }{ \partial \vecxbold }$ \\[0.4em]
    72  &  4 & & „niezaburzonego” & „zaburzonego” \\
    73  &  5 & & $\vecxbold( 0$ & $\vecxbold( 0 )$ \\
    90  & &  3 & \textit{wraz z pochodną dla} $x = 0$
           & \textit{dla} $x = 0$ \\
    92  &  3 & & $U( x( O ) )$ & $U( x( 0 ) )$ \\[0.3em]
    123 &  6 & & $^{ \Rbb }A : \Cbb^{ m } \to { }^{ \Rbb } \Cbb^{ n }$
           & ${ }^{ \Rbb }A : { }^{ \Rbb } \Cbb^{ m }
             \to { }^{ \Rbb } \Cbb^{ n }$ \\
    125 &  5 & & $\mathbf{I}$ & $I$ \\
    % & & & & \\
    % & & & & \\
    % & & & & \\
    % & & & & \\
    \hline
  \end{tabular}

\end{center}

\vspace{\VerSpaceSix}


\noindent
\StrWierszD{66}{7} \\
\Jest  \textit{tyłu do~brzegu} \\
\Powin \textit{tyłu nieograniczenie albo~do~brzegu} \\
\StrWierszG{110}{9} \\
\Jest  sumą częściową szeregu --~iloczynu \\
\Powin jest sumą części wyrazów iloczynu \\



% ############################










% ############################
\Work{ % Autor i tytuł dzieła
  N. M. Matwiejew \\
  \textit{Metody całkowania równana różniczkowych zwyczajnych},
  \cite{MatwiejewMetodyCalkowaniaRownanRozniczkowychZwyczajnych1982}}

\vspace{0em}


% ##################
\CenterBoldFont{Błędy}


\begin{center}

  \begin{tabular}{|c|c|c|c|c|}
    \hline
    Strona & \multicolumn{2}{c|}{Wiersz} & Jest
                              & Powinno być \\ \cline{2-3}
    & Od góry & Od dołu & & \\
    \hline
    5   & &  9 & Dodzimy & Dowodzimy \\
    5   & &  8 & potkowych & początkowych \\
    5   & &  7 & poąątkowych & początkowych \\
    10  & & 19 & damy & mamy \\
    % 15  & & & & \\
    % & & & & \\
    % & & & & \\
    \hline
  \end{tabular}

\end{center}

\vspace{\VerSpaceSix}


\noindent
\StrWierszG{15}{13}
\Jest  w~sensie ustępu \\
\Powin w~sensie zdefiniowanym w~ustępie \\
\StrWierszD{20}{2} \\
\Jest  i~nie ma rozwiązania określonego w~tym samym przedziale
nie~identycznego z~rozwiązaniem $y = y( x )$ chociażby w~jednym
punkcie przedziału $\absOne{ x - x_{ 0 } } \leq h$ różnym
od~punktu $x = x_{ 0 }$. \\
\Powin i~nie istnieje inne rozwiązanie określone w~przedziale
$\absOne{ x - x_{ 0 } } \leq h_{ 1 } \leq h$ które nie byłoby równe
rozwiązaniu $y = y( x )$ w~każdym punkcie przedziału
$\absOne{ x - x_{ 0 } } \leq h_{ 1 }$. \\


% ############################










% ############################
\newpage

\Work{ % Autor i tytuł dzieła
  N.M. Matwiejew \\
  \textit{Metody całkowania równana różniczkowych zwyczajnych},
  \cite{MatwiejewMetodyCalkowaniaRownanRozniczkowychZwyczajnych1986}}

\vspace{0em}


% ##################
\CenterBoldFont{Uwagi}

\vspace{0em}


\noindent
Książka ta podąża za zwyczajem, że~jeśli argumenty danej funkcji są
odpowiednio oczywiste, to można je opuścić. Przykładowo na stronie~7
mamy równanie (1) postaci:
\begin{equation}
  \label{eq:Matwiejew-Metody-calkowania-ETC-01}
  F\left( x, y, y', y'', \ldots, y^{ ( n ) } \right) = 0,
\end{equation}
choć bardziej precyzyjne byłoby napisanie $y( x )$, $y'( x )$, etc. W~tych
notatkach, by uniknąć nieporozumień, we~wszystkich miejscach gdzie mowa jest
o~wartościach jakie przyjmują funkcje dla zadanych wartości argumentów,
będziemy~się starać podawać owe argumenty w~sposób jawny.

Ty samym, jeśli będziemy rozważali w~sposób ogólny funkcje
$y : \Rbb \to \Rbb$ to będziemy ją oznaczać symbolem $y$. Natomiast
w~przypadkach takich jak równanie
\begin{equation}
  \label{eq:Matwiejew-Metody-calkowania-ETC-02}
  y( x )^{ 2 } = x,
\end{equation}
będziemy~się starali zawsze wymienić wszystkie argumenty w~sposób jawny.

\vspace{\VerSpaceFour}










% ##################
\CenterBoldFont{Uwagi do~konkretnych stron}

\vspace{0em}


\noindent
\Str{7} W~tym miejscu warto~się zastanowić nad tym, jak w~ścisły sposób
zdefiniować równanie różniczkowe zwyczajne? Jak każdy taki ważny
problem w~matematyce, podaną definicję trzeba będzie niewątpliwie rozszerzać
i~modyfikować, tak aby objąć jej nowymi wersjami kolejne ważne problemy
matematyczne\footnote{Przez „problem matematyczny” rozumiemy tu zagadnienie
  matematyczne, które jawnie wymaga od nas znalezienia jego rozwiązania.
  Przykładowo, problemem matematyczny jest zagadnienie wyznaczenia
  funkcji $y( x )$, takiej że~jej pochodna w~punkcie $x$ jest równa
  $\sin\!\big( y( x ) \big)$. }. Niezależnie jednak od tego, warto spróbować
podać teraz ścisłą definicję które będzie obejmowała większość najbardziej
podstawowych problemów matematycznych jakie napotykamy w~teorii równań
różniczkowych zwyczajnych, nawet jeśli wiele innych będzie wykluczała.

Na tej stronie możemy znaleźć informacje, że~jeśli nie jest powiedziane
inaczej, to zakładamy, że~zarówno dziedzina jak i~przeciwdziedzina funkcji
$y$ są podzbiorami liczb rzeczywistych. Jeśli chodzi o~przeciwdziedzinę
to możemy przyjąć, że~zawsze jest ona równa $\Rbb$, bo wybór ten nie
powinien nigdzie grać roli. Jeśli chodzi o~dziedzinę danej funkcji, to
sprawa jest bardziej złożona.

Niech $A$ oznacza dziedzinę funkcji $y$. W~niniejszej książce naszym
podstawowym wymogiem jest to, by w~każdym punkcie dziedziny $A$ istniała
pochodna funkcji $y$, więc $A$ musi być dobrany w~taki sposób, by pojęcie
pochodnej w~punkcie $x \in A$ miało sens. Wbrew pozorom podanie klasy zbiorów
o~tej własności nie jest takie proste. Czy przykładowo funkcja
$f : \Qbb \to \Qbb$ dana zależnością
\begin{equation}
  \label{eq:Matwiejew-Metody-calkowania-ETC-03}
  f( x ) = x^{ 2 },
\end{equation}
jest różniczkowalna?

Pomimo problemów z~określeniem klasy tych zbiorów, w~większości przypadków
przyjmuje~się, że~zbiór $A$ jest odcinkiem otwartym $( a, b )$, gdzie
dopuszczamy sytuację, że $a$ i~$b$ mogą przyjmować wartość $\pm \infty$. Jak
zauważono na~stronie~14 omawianej książki, nie nastręcza również problemu
określenie funkcji~$y$ na odcinkach domkniętych z~jednej, bądź obu stron.
W~przypadku odcinka $[ a, b )$ przez pochodną w~punkcie~$a$ rozumiemy
pochodną prawostronną funkcji $y$ i~analogicznie postępuję w~przypadku
odcinka $( a, b ]$. Przy czym jeśli dany jest odcinek prawostronnie
domknięty $( a, b ]$ to musi zachodzić $-\infty < b < +\infty$, natomiast $a$ może
przyjąć wartość $-\infty$. Analogicznie zasady stosują~się do pozostały pozostał
przypadków odcinków jednostronnie, bądź obustronnie domkniętych.
Do~problemu kształtu dziedziny powrócimy niedługo.

Niech teraz $O$ będzie otwartym podzbiorem $\Rbb^{ n + 2 }$, przy czym
elementy $\Rbb^{ n + 2 }$ będziemy oznaczać przez
$( x_{ 0 }, x_{ 1 }, \ldots, x_{ n - 1 }, x_{ n }, x_{ n + 1 } )$
lub $x_{ 1 }, x_{ 2 }, \ldots, x_{ n + 1 }, x_{ n + 2 }$. Niech dana będzie funkcja
$F : O \to R$. Przyjmujemy, że zależy ona w~sposób \textbf{istotny}
(w~książce używa~się terminu „jawny”) od zmiennej $x_{ n }$, przez co
rozumiemy następującą własność. Istnieją takie liczby
$x_{ 0 }, x_{ 1 }, \ldots, x_{ n }, x_{ n + 1 }, x_{ n + 1 }'$, że~zachodzi
\begin{equation}
  \label{eq:Matwiejew-Metody-calkowania-ETC-04}
  F( x_{ 0 }, x_{ 1 }, \ldots, x_{ n }, x_{ n + 1 } ) \neq
  F( x_{ 0 }, x_{ 1 }, \ldots, x_{ n }, x_{ n + 1 }' ).
\end{equation}
Niech teraz $A_{ 0 }$ będzie rzutem zbioru $O$ na oś $x_{ 0 }$, czyli zbiorem
takim że~jeśli $x \in A_{ 0 }$ to istnieją takie liczby
$x_{ 1 }, x_{ 2 }, \ldots, x_{ n }, x_{ n + 1 }$,
że~$( x, x_{ 1 }, x_{ 2 }, \ldots, x_{ n }, x_{ n + 1 } ) \in O$. Zbiór ten
będziemy również oznaczać przez $\proj_{ 0 } O$.

\textbf{Równaniem różniczkowym zwyczajnym rzędu~$n$} nazywamy wyrażenie
\begin{equation}
  \label{eq:Matwiejew-Metody-calkowania-ETC-05}
  F\left( x, y( x ), y'( x ), y''( x ), \ldots, y^{ ( n ) }( x ) \right) = 0.
\end{equation}
\textbf{Rozwiązaniem równania \eqref{eq:Matwiejew-Metody-calkowania-ETC-02}}
nazywamy funkcję $y_{ 1 } : A_{ 1 } \to \Rbb$, gdzie $A_{ 1 } \subset A_{ 0 }$, która
posiada pochodne do rzędu $n$ włącznie w~każdym punkcie swojej dziedziny
i~dla której zachodzi
\begin{equation}
  \label{eq:Matwiejew-Metody-calkowania-ETC-06}
  F\left( x, y_{ 1 }( x ), y_{ 1 }'( x ), \ldots, y_{ 1 }^{ ( n ) }( x ) \right) =
  0, \qquad
  \forall x \in A_{ 1 }.
\end{equation}

Potrzebujemy teraz powrócić do problemu określenia dziedziny $A_{ 1 }$
funkcji $y$. Jak powiedzieliśmy wcześniej, standardowo przyjmuje~się,
że~$A_{ 1 } = ( a, b )$. Zazwyczaj żąda~się dodatkowo, by $A_{ 1 }$ był
największym odcinkiem otwarty zawartym w~zbiorze $A_{ 0 }$, w~którym jesteśmy
w~stanie zdefiniować rozwiązanie równania
\eqref{eq:Matwiejew-Metody-calkowania-ETC-05}. Co jednak zrobić w~przypadku
równań takich jak równanie $y'( x ) = y( x )^{ 2 }$ omawiane na stronie~15,
które posiada rozwiązanie postaci
\begin{equation}
  \label{eq:Matwiejew-Metody-calkowania-ETC-07}
  y( x ) = \frac{ 1 }{ 1 - x }.
\end{equation}
Funkcja ta jest określona na zbiorze $( -\infty, 1 ) \cup ( 1, +\infty )$ i~nie możemy
w~żaden sposób wybrać jednego z~tych przedziałów jako większego od drugiego.

Problem rozwiązań takich jak \eqref{eq:Matwiejew-Metody-calkowania-ETC-06}
będzie omawiany w~dalszym ciągu książki, wtedy ewentualnie powrócimy do
pojawiające~się przy nich dokładniej. W~chwili obecnej, motywowani tym
przykładem, poprzestaniemy na stwierdzeniu, że~jeśli to jest możliwe to
będziemy szukali rozwiązań określonych na całym zbiorze $A_{ 0 }$, co jest
równoważne stwierdzeniu $A_{ 1 } = A_{ 0 }$, co~wyklucza wszelkie
niejednoznaczności w~sposobie określenia dziedziny rozwiązania.
Niewątpliwie, ze względu na swoją prostotę interpretacyjną, optymalną
sytuacją jest $A_{ 1 } = A_{ 0 } = \Rbb$.

\vspace{\VerSpaceFour}





\noindent
\Str{8} Na tej stronie napotykamy po raz pierwszy konkretne równanie
różniczkowe
\begin{equation}
  \label{eq:Matwiejew-Metody-calkowania-ETC-08}
  y'( x ) - 2 x = 0.
\end{equation}
Z~równania tego w~prosty sposób odczytujemy funkcję $F$:
\begin{equation}
  \label{eq:Matwiejew-Metody-calkowania-ETC-09}
  F( x_{ 0 }, x_{ 1 }, x_{ 2 } ) = x_{ 2 } - 2 x_{ 0 }.
\end{equation}
Stajemy tu jednak przed problemem, jaka jest dziedzina funkcji~$F$?
W~dalszym ciągu jeśli nie powiedziano inaczej, za dziedzinę funkcji $F$
definiującej równanie różniczkowe będziemy uważać największy zbiór na~którym
jesteśmy w~stanie ją określić. Ponieważ zgodnie z~tym co powiedziano
wcześniej, dziedzina funkcji $F$ jest podzbiorem $\Rbb^{ n + 2 }$, w~naszym
wypadku $n = 1$, więc dla funkcji danej przez
\eqref{eq:Matwiejew-Metody-calkowania-ETC-07} jest ona równa $\Rbb^{ 3 }$.
Łatwo~się też przekonać, że~rozwiązaniem równania
\eqref{eq:Matwiejew-Metody-calkowania-ETC-08} jest funkcja
$y( x ) = x^{ 2 } + C$, której dziedzina to $\Rbb$.

Jednocześnie równanie \eqref{eq:Matwiejew-Metody-calkowania-ETC-08} zapisane
jako
\begin{equation}
  \label{eq:Matwiejew-Metody-calkowania-ETC-10}
  y'( x ) = 2 x,
\end{equation}
jest przykładem równania pierwszego rzędu w~postaci normalnej (zob. str.~13
omawianej książki). Ogólna forma równania w~postaci normalnej to
\begin{equation}
  \label{eq:Matwiejew-Metody-calkowania-ETC-11}
  y^{ ( n ) }( x ) = f\big( x, y( x ), y'( x ), \ldots, y^{ ( n - 1 ) }( x ) \big).
\end{equation}
W~sposób elementarny przekształcamy je do równania w~postaci ogólnej
\begin{equation}
  \label{eq:Matwiejew-Metody-calkowania-ETC-12}
  F\big( x, y( x ), y'( x ), \ldots, y^{ ( n ) }( x ) \big) =
  f\big( x, y( x ), y'( x ), \ldots, y^{ ( n - 1 ) }( x ) \big) -
  y^{ ( n - 1 ) }( x ).
\end{equation}
Z~tego powodu, jeśli równanie różniczkowe jest nam dane pierwotnie nie
w~postaci ogólnej, lecz normalnej, to przy analizie problemu dziedziny jego
rozwiązań możemy stosować dokładnie te same zasady, co dla równania
w~postaci ogólnej.

Jeśli chodzi o~ustalenie dziedziny funkcji $f$ stosować będziemy te same
zasady co dla funkcji $F$. Należy jednak pamiętać, że~jeśli równanie jest
pierwotnie wyrażone w~formie ogólnej
\begin{equation}
  \label{eq:Matwiejew-Metody-calkowania-ETC-13}
  F\big( x, y( x ), y'( x ), \ldots, y^{ ( n ) }( x ) \big) = 0,
\end{equation}
to jego postać normalna
\begin{equation}
  \label{eq:Matwiejew-Metody-calkowania-ETC-14}
  y^{ ( n ) }( x ) = f\big( x, y( x ), y'( x ), \ldots, y^{ ( n - 1 ) }( x ) \big),
\end{equation}
może nie obejmować wszystkich przypadków, dla których ma sens równanie
\eqref{eq:Matwiejew-Metody-calkowania-ETC-13}. Ponieważ dziedzina $F$ jest
podzbiorem $\Rbb^{ n + 2 }$, a~$f$ podzbiorem $\Rbb^{ n + 1 }$, wysłowienie
co dokładnie oznacza, że~równanie
\eqref{eq:Matwiejew-Metody-calkowania-ETC-13}
jest ogólniejsze od równania \eqref{eq:Matwiejew-Metody-calkowania-ETC-14}
może być dość zawiłe, zamiast tego odwołamy~się do przykładu.

Rozpatrzmy równanie w~postaci ogólnej
\begin{equation}
  \label{eq:Matwiejew-Metody-calkowania-ETC-15}
  y( x ) \frac{ d y( x ) }{ d x } - x = 0.
\end{equation}
Odczytujemy z~niego funkcję
$F( x_{ 0 }, x_{ 1 }, x_{ 2 } ) = x_{ 2 } x_{ 1 } - x_{ 0 }$, której dziedzina
jest równa $\Rbb^{ 3 }$. Po przekształceniu do postaci normalnej dostajemy
równanie
\begin{equation}
  \label{eq:Matwiejew-Metody-calkowania-ETC-16}
  \frac{ d y( x ) }{ d x } = \frac{ x }{ y( x ) }.
\end{equation}
Funkcja $f( x_{ 0 }, x_{ 1 } ) = x_{ 0 } / x_{ 1 }$ ma dziedzinę
$\Rbb \times ( \Rbb \setminus \{ 0 \} )$. Podstawiając $y( x ) = 0$
do~\eqref{eq:Matwiejew-Metody-calkowania-ETC-13} od razu sprawdzamy,
że~funkcja ta nie jest rozwiązaniem tego równania, podczas gdy podstawienie
tej funkcji do \eqref{eq:Matwiejew-Metody-calkowania-ETC-16} prowadzi do
wyrażenie pozbawionego sensu\footnote{W~książce są podane sposoby radzenia
  sobie z~tego typu problemami, nie jest to jednak ważne w~obecnych
  rozważaniach.}.

\vspace{\VerSpaceFour}





\noindent
\Str{10} Przy okazji wyprowadzania równania różniczkowe dla rodziny
wszystkich okręgów na płaszczyźnie $xy$ napotykamy po raz pierwszy na
pewien problem, który powróci do nas w przyszłości. Naszym punktem wyjścia
jest równania
\begin{equation}
  \label{eq:Matwiejew-Metody-calkowania-ETC-17}
  ( x - a )^{ 2 } + ( y - b )^{ 2 } = R^{ 2 }.
\end{equation}
Różniczkując je dwa razy otrzymujemy dwa następujące równania.

\vspace{\negVerSpaceFour}


\begin{subequations}

  \begin{align}
    \label{eq:Matwiejew-Metody-calkowania-ETC-18-A}
    1 + \big( y'( x ) \big)^{ 2 } + \big( y( x ) - b \big) y''( x )
    &= 0, \\
    \label{eq:Matwiejew-Metody-calkowania-ETC-18-B}
    3 y'( x ) y''( x ) + \big( y( x ) - b \big) y'''( x )
    &= 0.
  \end{align}

\end{subequations}


\noindent
Aby usunąć z~równania \eqref{eq:Matwiejew-Metody-calkowania-ETC-18-B}
parametr $b$ przyjmujemy, że~$y''( x ) \neq 0$, więc możemy przepisać równanie
\eqref{eq:Matwiejew-Metody-calkowania-ETC-18-A} jako
\begin{equation}
  \label{eq:Matwiejew-Metody-calkowania-ETC-19}
  y( x ) - b =
  -\frac{ 1 + \big( y'( x ) \big)^{ 2 } }{ y''( x ) }.
\end{equation}
Podstawiając tą zależność do \eqref{eq:Matwiejew-Metody-calkowania-ETC-18-B}
dostajemy
\begin{equation}
  \label{eq:Matwiejew-Metody-calkowania-ETC-20}
  3 y'( x ) y''( x ) -
  \frac{ 1 + \big( y'( x ) \big)^{ 2 } }{ y''( x ) } y'''( x ) = 0.
\end{equation}
Po pomnożeniu obustronnie przez $y''( x )$ dostajemy
\begin{equation}
  \label{eq:Matwiejew-Metody-calkowania-ETC-21}
  3 y'( x ) \big( y''( x ) \big)^{ 2 } -
  \Big( 1 + \big( y'( x ) \big)^{ 2 } \Big) y'''( x ) = 0.
\end{equation}
Choć równanie wyprowadziliśmy przy założeniu, że~$y''( x ) \neq 0$, to końcowa
jego postać jest dobrze określona również, gdy funkcja ta przyjmuje wartość
zero. Powstaje więc pytania, jaki jest zakres obowiązywania tego równania?

Równanie \eqref{eq:Matwiejew-Metody-calkowania-ETC-18-A} możemy przekształci
przy założeniu $y( x ) - b \neq 0$ do postaci
\begin{equation}
  \label{eq:Matwiejew-Metody-calkowania-ETC-22}
  y''( x ) =
  -\frac{ 1 + \big( y'( x ) \big)^{ 2 } }{ y( x ) - b },
\end{equation}
widzimy więc, że~$y''( x ) \neq 0$ jeśli tylko $y( x ) \neq b$, w~przeciwnym
wypadku jest ona nieokreślona. Możemy stąd wyciągnąć wniosek,
że~na mocy swojego wyprowadzenia równanie
\eqref{eq:Matwiejew-Metody-calkowania-ETC-21} obowiązuje dla wszystkich
wartości $x$ dla których $y( x ) \neq b$. To~zaś prowadzi do~kilku problemów
wartych przedyskutowania.

Po~pierwsze, równanie \eqref{eq:Matwiejew-Metody-calkowania-ETC-21} nie
zawiera stałej~$b$, gdyż naszym celem było wyprowadzenie równania
pozbawionego stałych obecnych w~równaniu
\eqref{eq:Matwiejew-Metody-calkowania-ETC-17}. Biorąc więc za~punkt wyjścia
\eqref{eq:Matwiejew-Metody-calkowania-ETC-20} nie jesteśmy w~stanie
sformułować na jego podstawie warunku $y( x ) \neq b$. Możliwe jednak, że~jeśli
rozwiążemy to równanie, to jego rozwiązania będą nieokreślone jeśli $y( x )$
przyjmie pewną wartość $D$, specyficzną dla danego rozwiązania. Już teraz
możemy stwierdzić, iż~jest to prawdą, przynajmniej dla pewnej klasy
rozwiązań tego równania, które w~postaci uwikłanej (zob. str.~15 omawianej
książki\footnote{Na tej stronie omawiane jest rozwiązanie w~postaci
  uwikłanej dla równania pierwszego rzędu, a~równanie
  \eqref{eq:Matwiejew-Metody-calkowania-ETC-20}. Pomimo tego mamy nadzieję,
  że~to w~jakim sensie równanie
  \eqref{eq:Matwiejew-Metody-calkowania-ETC-17} przedstawia rozwiązanie
  równania \eqref{eq:Matwiejew-Metody-calkowania-ETC-21} jest wystarczająco
  zrozumiałe i~nie wymaga dalszych komentarzy.}) jest dane przez naszą
wyjściową zależność
\eqref{eq:Matwiejew-Metody-calkowania-ETC-17}. Nad analizą tej sytuacji
zatrzymamy~się przez chwilę.

Dla $x \in ( a - R, a + R )$ jesteśmy w~stanie je odwikłać na funkcję
$y( x )$ otrzymując dwie funkcje
\begin{equation}
  \label{eq:Matwiejew-Metody-calkowania-ETC-23}
  y_{ \pm }( x ) = \pm\sqrt{ R^{ 2 } - ( x - a )^{ 2 } } + b.
\end{equation}
Nie jest możliwe odwikłanie tego równania na funkcję $y( x )$ w~sposób
matematycznie spójny na~przedziale $( a + R - \delta, a + R ]$, dla pewnego
$\delta > 0$, przy czym analogiczna sytuacja zachodzi dla przedziału
$[ a - R, a - R + \delta )$. W~tematykę tego, czy w~takim razie nie powinniśmy
również rozważać równania różniczkowego na funkcję $x( y )$ (zob. str.~13
tej książki) nie będziemy~się zagłębiać. Niemniej widzimy, że
\begin{equation}
  \label{eq:Matwiejew-Metody-calkowania-ETC-24}
  \lim_{ x \nearrow a + R } y_{ \pm }( x ) = b,
\end{equation}
więc funkcje te możemy przedłużyć do funkcji ciągłych na przedziale
$( a - R, a + R ]$ (analogicznie możemy postąpić dla punktu $a - R$).
Tak przedłużone funkcje będziemy oznaczać przez $\yTilde_{ \pm }( x )$.
Przyjmują one wartość $b$ tylko dla $x = a + R$, ale~żadna z~nich nie jest
różniczkowalna w~punkcie $x = a + R$, bo~zachowuje~się w~ich otoczeniu jak
$\sqrt{ x }$ w~otoczeniu punktu $x = 0$. Powyższe rozważania pomogą nam
naświetlić drugi ważny problem.

Zauważmy, że~choć równanie \eqref{eq:Matwiejew-Metody-calkowania-ETC-21}
obowiązuje dla wszystkich $x$ taki, że~$y( x ) \neq b$, to nawet jeśli ten
warunek jest nam z~góry znany, to zanim nie rozwiążemy tego równania, nie
wiem dla jakich wartości zmiennej $x$ rozwiązanie przyjmie wartość $b$.
Musimy więc najpierw rozwiązać w~odpowiedni sposób to równanie, a~następnie
usunąć z~dziedziny rozwiązania te punkty, dla których $y( x ) = b$. Jeśli
takie punkty rzeczywiście istnieją to powstaje następujące pytanie. Sokor
otrzymaliśmy rozwiązanie również dla tych wartości $x$ dla
których nie jest spełniony warunek $y( x ) \neq b$, to czy nie oznacza to,
że~choć warunek ten był potrzebny do wyprowadzenia
\eqref{eq:Matwiejew-Metody-calkowania-ETC-21}, to jest on ostatecznie
restrykcyjny i~należy przyjąć bardziej ogólną koncepcję rozwiązania?

W~omawianym przykładzie, funkcje $y_{ \pm }( x )$ były zdefiniowane tylko na
przedziale $( a - R, a + R )$, jednak dało~się je przedłużyć do funkcji
$\yTilde_{ \pm }( x )$ ciągłych na $( a - R, a + R ]$, które są jednak
nieróżniczkowalne w~punkcie $a + R$. Czy w~takim wypadku należy uważać
funkcje $\yTilde_{ \pm }( x )$ za rozwiązanie odpowiedniego równania
różniczkowego? Zwróćmy uwagę, że~rozwiązanie w~postaci uwikłanej
\begin{equation}
  \label{eq:Matwiejew-Metody-calkowania-ETC-25}
  ( x - a )^{ 2 } + ( y - b )^{ 2 } = R^{ 2 },
\end{equation}
przedstawia okrąg, czyli krzywą gładką w~każdym swoim punkcie, leżącą
w~płaszczyźnie $xy$, która nie wykazuje żadnej osobliwości w~punkcie
o~współrzędnych $( a + R, b )$. Z~tego
punktu widzenia funkcja $\yTilde_{ + }( x )$ jest obcięciami rozwiązania
danego funkcją uwikłaną do zbioru $( a - R, a + R ]$, które spełnia warunek
$y \geq 0$, więc należy je uważa za pełnoprawne rozwiązanie równania
\eqref{eq:Matwiejew-Metody-calkowania-ETC-21}. Niewątpliwie, fakt,
że~rozwiązania w~postaci uwikłanej pozwalają nam na objęciem pojęciem
rozwiązania równania różniczkowego takich krzywych jak okręgi dane równaniem
\eqref{eq:Matwiejew-Metody-calkowania-ETC-25}, jest jednym z~powodów, dla
których to pojęcie zostało wprowadzone.

Przedstawiona powyżej analiza pozwala uzmysłowić, jak skomplikowanym
zadaniem może być analiza zakresu obowiązywania danego równania
różniczkowego oraz jego rozwiązań. Do omawiania tych zagadnień zapewne
powrócimy jeszcze wielokrotnie w~tych notatkach.

\vspace{\VerSpaceFour}





\noindent
\Str{15} Równanie (10) na tej stronie zostało bardzo elegancko wyprowadzone,
przy założeniu, że~funkcja uwikłana zależna od zmiennej $x$ dana związkiem
$\Phi( x, y ) = 0$, spełnia równanie różniczkowe
$y'( x ) = f\big( x, y( x ) \big)$. Nie poruszono jednak kwestii tego, czemu
mamy uważać tą funkcję uwikłaną za rozwiązanie badanego równania?
W~szczególności, jeśli jest rozwiązanie równania $\Phi( x, y ) = 0$ na funkcję
$y( x )$, to czy ona spełnia równanie $y'( x ) = f\big( x, y( x ) \big)$?
Przy standardowych założeniach o~funkcji $\Phi( x, y )$ odpowiedź na to
ostatnie pytanie jest twierdząca i~dzięki temu możemy uznać funkcję daną
równaniem uwikłanym za rozwiązanie badanego równania.

Dowód tego faktu jest następujący. Załóżmy, że~funkcja $\Phi( x, y )$ spełnia
założenia o~ciągłości i~istnieniu pochodnych, które są wymagane
w~standardowej wersji twierdzenia o~funkcji uwikłanej. Przyjmijmy dodatkowo,
że~$\Phi_{ y }'\big( x, y \big) \neq 0$ w~interesującym nas zakresie wartości
zmiennej~$x$, który oznaczymy~$A$. Tym samym możemy odwikłać funkcję
$y( x )$ dla wszystkich wartości $x \in A$. Przepiszmy teraz równanie (10)
w~bardziej jawnej formie jako
\begin{equation}
  \label{eq:Matwiejew-Metody-calkowania-ETC-26}
  \Phi_{ x }'\big( x, y( x ) \big) +
  \Phi_{ y }'\big( x, y( x ) \big) f\big( x, y( x ) \big) = 0.
\end{equation}
Równanie to przekształcamy w~oczywisty sposób do postaci
\begin{equation}
  \label{eq:Matwiejew-Metody-calkowania-ETC-27}
  -\frac{ \Phi_{ x }'\big( x, y( x ) \big) }
  { \Phi_{ y }'\big( x, y( x ) \big) } =
  f\big( x, y( x ) \big).
\end{equation}
Jak dobrze wiadomo, lewa strona tej równości przedstawia pochodną funkcji
uwikłanej $y'( x )$.

Na koniec możemy zauważyć, że~zarówno wzór
\eqref{eq:Matwiejew-Metody-calkowania-ETC-26} jak i~dobrze znaną zależność
\begin{equation}
  \label{eq:Matwiejew-Metody-calkowania-ETC-28}
  y'( x ) =
  -\frac{ \Phi_{ x }'\big( x, y( x ) \big) }{ \Phi_{ y }'\big( x, y'( x ) \big) },
\end{equation}
można wyprowadzić w~ten sam sposób, poprzez zróżniczkowanie
i~przekształcenie zależności $\Phi\big( x, y( x ) \big) = 0$. Choć z~tego
punktu widzenia równoważność równania
\eqref{eq:Matwiejew-Metody-calkowania-ETC-26}
i~$y'( x ) = f\big( x, y( x ) \big)$ może~się wydawać oczywista, woleliśmy
przedyskutować to zagadnienie możliwie dokładnie.

\vspace{\VerSpaceFour}





\noindent
\Str{16} Należy przedyskutować dwa zagadnienia dotyczące rozwiązania
równania różniczkowego
\begin{equation}
  \label{eq:Matwiejew-Metody-calkowania-ETC-29}
  y'( x ) = f\big( x, y( x ) \big),
\end{equation}
w~postaci parametrycznej. Po pierwsze, czy jeśli $y( x )$ jest rozwiązaniem
tego równania, to czy można podać jego rozwiązanie w~postaci parametrycznej?
Po drugie, jak uzasadnić, że~funkcje $\varphi( t )$ i~$\psi( t )$ spełniające
równanie
\begin{equation}
  \label{eq:Matwiejew-Metody-calkowania-ETC-30}
  \frac{ \psi'( t ) }{ \varphi'( t ) } =
  f\big( \varphi( t ), \psi( t ) \big),
\end{equation}
przedstawiając rozwiązanie równania
\eqref{eq:Matwiejew-Metody-calkowania-ETC-29}?

Zanim odpowiemy na te pytania, zrobimy małą uwagę odnośnie oznaczeń.
W~całym tym punkcie, jeśli nie powiedziano inaczej, symbol prim oznacza
pochodną danej funkcji. W~szczególności we wzorze
\eqref{eq:Matwiejew-Metody-calkowania-ETC-30}, co przyjmowaliśmy domyślnie,
symbol $\psi'( t )$ oznacza pochodną funkcji $\psi( t )$ po~czasie.

Przejdźmy teraz do odpowiedzi na~pierwsze pytanie z postawionych wyżej
pytań. Równanie różniczkowe \eqref{eq:Matwiejew-Metody-calkowania-ETC-29}
jest określone na płaszczyźnie~$xy$, więc funkcje $\varphi( t )$ i~$\psi( t )$ możemy
rozumieć jako odwzorowanie $\Psi( t ) : ( t_{ 0 }, t_{ 1 } ) \to \Rbb^{ 2 }$, dane
przez
\begin{equation}
  \label{eq:Matwiejew-Metody-calkowania-ETC-31}
  \Psi( t ) = \big( \varphi( t ), \psi( t ) \big).
\end{equation}
Korzystając z~interpretacji równania
\eqref{eq:Matwiejew-Metody-calkowania-ETC-29} za pomocą wprowadzonego dalej
na tej stronie i~następnych pojęcia pola kierunków oraz krzywej całkowej,
można stwierdzić, że~funkcje $\varphi( t )$ i~$\psi( t )$ są rozwiązaniem
rozpatrywanego równania w~postaci parametrycznej, jeśli $\Psi( t )$ wyznacza
krzywą całkową. Jest to bardzo ładny atrakcyjny sposób rozumienia
rozwiązania parametrycznego, dla porządku podamy alternatywną interpretacje.

Mianowicie, jeśli $y( x )$ jest rozwiązaniem równania określonym dla
$x \in ( a, b )$, to parę funkcji $\varphi( t )$, $\psi( t )$ określonych
dla~$t \in I = ( t_{ 0 }, t_{ 1 } )$ nazywamy postacią parametryczną tego
równania jeśli zachodzi $\varphi( I ) \subset ( a, b )$ oraz
\begin{equation}
  \label{eq:Matwiejew-Metody-calkowania-ETC-32}
  \psi( t ) = y\big( \varphi( t ) \big).
\end{equation}

Wzór \eqref{eq:Matwiejew-Metody-calkowania-ETC-32}
pozwala nam stwierdzić, że~jeśli mamy dane rozwiązanie $y( x )$, określone
przy tych samych warunkach co powyżej, to zawsze możemy utworzyć odpowiednie
rozwiązanie parametryczne. Wystarczy przyjąć, że~$I = ( a, b )$,
$\varphi( t ) = t$ (zwykle zapisuje to jako $x = t$) i~zdefiniować $\psi( t )$ jako
$y\big( \varphi( t ) \big)$. Zauważmy, że~zachodzi wówczas
\begin{equation}
  \label{eq:Matwiejew-Metody-calkowania-ETC-33}
  \frac{ d \psi( t ) }{ d t } =
  \frac{ d y( x ) }{ d x }\bigg|_{ x = \varphi( t ) }
  \frac{ d \varphi( t ) }{ d t }.
\end{equation}
Oznaczając symbolem prim pochodną po czasie i~zakładając,
że~$\varphi'( t ) \neq 0$ otrzymujemy zależność
\begin{equation}
  \label{eq:Matwiejew-Metody-calkowania-ETC-34}
  \frac{ d y( x ) }{ d x }\bigg|_{ x = \varphi( t ) } =
  \frac{ \psi'( t ) }{ \varphi'( t ) }.
\end{equation}
Razem z~równaniami \eqref{eq:Matwiejew-Metody-calkowania-ETC-29}
i~\eqref{eq:Matwiejew-Metody-calkowania-ETC-32} prowadzi do równania
\begin{equation}
  \label{eq:Matwiejew-Metody-calkowania-ETC-35}
  \frac{ \psi'( t ) }{ \varphi'( t ) } =
  f\big( \varphi( t ), \psi( t ) \big).
\end{equation}
To kończy rozważanie na temat wyrażenia znanego rozwiązania w~postaci
parametrycznej.

Przejdźmy teraz do problemu, dlaczego możemy uważać rozwiązanie równania
\eqref{eq:Matwiejew-Metody-calkowania-ETC-30} za rozwiązania równania
\eqref{eq:Matwiejew-Metody-calkowania-ETC-29}. Przyjmijmy, że~funkcja
$\varphi( t )$ jest różniczkowalna w~sposób ciągły. Na podstawie równania
\eqref{eq:Matwiejew-Metody-calkowania-ETC-30} wiemy, że~musi zachodzić
$\varphi'( t )$. Na podstawie standardowych twierdzeń z~analizy matematycznej
(zob. przykładowo \cite{FichtenholzRachunekRozniczkowyETCVolI2005}) oraz
dobrze znanego wzoru na pochodną funkcji odwrotnej:
\begin{equation}
  \label{eq:Matwiejew-Metody-calkowania-ETC-36}
  \frac{ d f^{ -1 }( y ) }{ d y } =
  \frac{ d f( t ) }{ d t }\bigg|_{ t = f^{ -1 }( y ) },
\end{equation}
w~pewnych otoczeniu wybranego punktu $t$, oznaczmy je
$J = ( t - \delta_{ 1 }, t + \delta_{ 2 } )$, gdzie $\delta_{ 1 }, \delta_{ 2 } > 0$, istniej
funkcja odwrotna do $\varphi( t )$, oznaczać ją będziemy $\eta( x )$, która jest
różniczkowalna i~jej pochodna wynosi
\begin{equation}
  \label{eq:Matwiejew-Metody-calkowania-ETC-37}
  \frac{ d \eta( x ) }{ d x } =
  \frac{ 1 }{ \varphi'\big( \eta( x ) \big) }.
\end{equation}
Dla większej przejrzystość obliczeń użyliśmy tutaj symbolu
$\varphi'\big( \eta( x ) \big)$ na oznaczenia wartości pochodnej $\varphi'( t )$ obliczonej
w~punkcie $\eta( x )$.

Rozpatrzmy teraz funkcję $\psi\big( \eta( x ) \big)$. Jej pochodna po zmiennej $x$
jest równa
\begin{equation}
  \label{eq:Matwiejew-Metody-calkowania-ETC-38}
  \frac{ d \psi\big( \eta( x ) \big) }{ d x } =
  \frac{ d \psi( t ) }{ d t }\bigg|_{ t = \eta( x ) }
  \frac{ d \eta( x ) }{ d x } =
  \frac{ d \psi( t ) }{ d t }\bigg|_{ t = \eta( x ) }
  \frac{ 1 }{ \varphi'\big( \eta( x ) \big)}.
\end{equation}
Jeżeli teraz obliczymy wartość obu stron równania
\eqref{eq:Matwiejew-Metody-calkowania-ETC-30} dla wartości $t = \eta( x )$
otrzymamy
\begin{equation}
  \label{eq:Matwiejew-Metody-calkowania-ETC-39}
  \frac{ \psi'\big( \eta( x ) \big) }{ \varphi'\big( \eta( x ) \big) } =
  f\Big( x, \psi\big( \eta( x ) \big) \Big).
\end{equation}
Korzystając z~\eqref{eq:Matwiejew-Metody-calkowania-ETC-38} możemy przepisać
powyższe równanie jako
\begin{equation}
  \label{eq:Matwiejew-Metody-calkowania-ETC-40}
  \frac{ d \psi\big( \eta( x ) \big) }{ d x } =
  f\Big( x, \psi\big( \eta( x ) \big) \big).
\end{equation}
Widzimy więc, że~funkcja $y( x ) = \psi\big( \eta( x ) \big)$ jest rozwiązaniem
równania \eqref{eq:Matwiejew-Metody-calkowania-ETC-29} określonym na zbiorze
$\varphi( J )$. To wyjaśnia czemu funkcje $\varphi( t )$ i~$\psi( t )$ możemy uznać za
rozwiązanie rozważanego równania różniczkowego.

\vspace{\VerSpaceFour}





\noindent
\Str{16--17} Warto dodać kilka słów komentarza do wprowadzonego tutaj
pojęcia pola kierunków oraz na relacji jaki pojęcie to ma do używanego
przez Władimira Arnolda pojęcia pola wektorowego definiujące równanie
różniczkowe \cite{ArnoldRownaniaRozniczkoweZwyczajne1975}. Samo pole
kierunków oraz inne pojęcia dla niego wprowadzone, takie jak izokliny
(por. str.~17 i~18), w~tych notatkach będziemy rozpatrywali, chyba
że~powiedziano inaczej, tylko w~kontekście najprostszego przypadku, dla
którego są wprowadzone w~tym fragmencie książki. Ten najprostszy przypadek
przedstawia~się następująco. Szukana funkcja $y$ jest odwzorowaniem
$y : A \to \Rbb$, gdzie $A \subset \Rbb$, a~równanie które ma spełniać podane
jest w~standardowej postaci normalnej:
\begin{equation}
  \label{eq:Matwiejew-Metody-calkowania-ETC-41}
  y'( x ) = f\big( x, y( x ) \big).
\end{equation}
W~związku z~tym pole kierunków jest zdefiniowane na płaszczyźnie $xy$,
gdzie $x$ odpowiada zmiennej niezależnej, a~zmienna $y$ wartościom
przyjmowanym przez funkcję $y( x )$.

Zaczniemy od pewnych uwag terminologicznych. O~równaniu
\eqref{eq:Matwiejew-Metody-calkowania-ETC-41} i~różnych jego
uogólnieniach będziemy mówili, że~wyznacza ono
\textbf{ruch zadanego punktu}. Przez ruch punktu będziemy rozumieć
rozwiązanie tego równania $y$. Natomiast przez \textbf{tor ruchu} będziemy
rozumieć zbiór $y( A )$.

Przejdźmy teraz do pola kierunków. W~jego opisie pojawiają~się długości
odcinków i~kąty, stąd wniosek, że~określamy je na płaszczyźnie, na~której
jest zdefiniowany iloczyn skalarny. Inaczej mówiąc, mamy do czynienia
z~dwuwymiarową
przestrzenią euklidesową $\Ebold^{ 2 }$, z~wybranym w~niej układem
współrzędnych. Moglibyśmy więc podać formalizacje pola kierunków za~pomocą
wiązki stycznej $T \Ebold^{ 2 }$, lecz nie widzimy takiej potrzeby.

W~dalszym ciągu potrzebne nam będą następujące oznaczenia. Przez $v$
będziemy oznaczali pole kierunków określone na $\Ebold^{ 2 }$, przez
$\vecv$ będziemy określali pole wektorowe w~takim sensie jak rozumie je
w~cytowanej wyżej pozycji Arnold, po więcej informacji odsyłamy do tej
książki.

Podanie kierunku jest równoważne z~podaniem pewnej prostej, aby~zdefiniować
kierunek wystarczy każdemu punktowi $P$ przyporządkować dowolny wektor
kierunkowy prostej wyznaczający dany kierunek. Tym samym pole wektorowe
$\vecv$ określone w~obszarze $G \subset \Ebold^{ 2 }$ odpowiednie pole kierunków,
jednak w~ogólności pole kierunków nie wyznacza w~sposób jednoznaczny pola
wektorowego. Mając dane pole kierunków możemy zawsze podać nieskończoną
ilość pól wektorowych, takich że wektor $\vecv( x, y )$ ma kierunek równy
$v( x, y )$. Wystarczy bowiem znaleźć jedno takie pole $\vecv( x, y )$,
następnie dla $\lambda \neq 0$ określić pole
\begin{equation}
  \label{eq:Matwiejew-Metody-calkowania-ETC-42}
  \vecv_{ \, \lambda }( x, y ) = \lambda \HorSpaceThree \vecv( x, y ).
\end{equation}
Sposób budowy pola wektorowego odpowiadającego danemu polu kierunków podamy
dalej.

Powstaje pytanie, czemu w~tym kontekście wystarczające jest rozważanie pola
kierunków, podczas gdy Arnold musi rozważać pola wektorowe? Wynika to
z~tego, że~tutaj analizujemy tylko najprostszy przypadek równania, w~którym
szukana jest funkcja $y : A \to \Rbb$, $A \subset \Rbb$, podczas, gdy formalizm
Arnolda obejmuje sytuacje, w~których szukana funkcja jest typu
$y : A \to \Rbb^{ n }$. Przeanalizujemy teraz dokładniej oba te przypadki.

Dla uproszczenia notacji, będziemy od teraz przyjmować, że~we wszystkich
przypadkach mamy $A = ( x_{ 0 }, x_{ 1 } )$. Nie ma to wielkiego znaczenia
dla ogólności przedstawionych rozważań. Niech szukana funkcja będzie postaci
$y : ( x_{ 0 }, x_{ 1 } ) \to \Rbb$. Wówczas krzywa całkowa
zawarta w~$G \subset \Ebold^{ 2 }$ i~styczna do pola kierunków
$v( x, y ) = f( x, y )$ wyznacza nie tylko kształt toru ruchu punktu,
którego równanie ruchu ma postać
\eqref{eq:Matwiejew-Metody-calkowania-ETC-41},
ale też jaki dokładnie sposób w~jaki ten tor pokonuje. Tor ruchu dla
badanego przypadku jest zawsze bardzo prosty, tworzy on bowiem odcinek
\begin{equation}
  \label{eq:Matwiejew-Metody-calkowania-ETC-43}
  y\big( ( x_{ 0 }, x_{ 1 } ) \big) = ( a, b )
\end{equation}
 (ewentualnie jeden z~odcinków
$( a, b ]$, $[ a, b )$, $[ a, b ]$), dla pewnych $a, b \in \Rbb$,
ewentualnie $a = -\infty$, $b = +\infty$. Wynika to z~tego, że~funkcja $y_{ 1 }( x )$
jest ciągła, a~obraz odcinka (zbioru spójnego) przez odwzorowanie ciągłe
jest też odcinek.

Jeśli zaś chodzi o~dokładny przebieg ruchu, to~z~krzywych całkowych pola
kierunków możemy odczytać czy jest to na przykład ruch oscylujący typu
$y_{ 1 }( x ) = A \sin( x )$, $A \in \Rbb$, czy ruch liniowy
$y_{ 1 }( x ) = A x$, etc. Wynika to z~tego, że~krzywa fazowa odpowiada
definicji funkcji jako zbioru par uporządkowanych odpowiedniego iloczynu
kartezjańskiego.

Wskażmy jeszcze na jeden ważny fakt. W~tej konkretnej sytuacji do pełnego
scharakteryzowania krzywych całkowych wystarczy znajomość kierunku do
którego mają one być styczna. Wiemy bowiem, że~krzywa całkowa zawsze
„biegnie od lewej do prawej”, bo to jest kierunek w~których zmienna $x$.
Załóżmy teraz, że~przez punkt $( x, y )$ ma przechodzić gładka krzywa
całkowa, dana przez funkcję $y$. Równaniem tej krzywej jest postaci
\begin{equation}
  \label{eq:Matwiejew-Metody-calkowania-ETC-44}
  K : ( x - \delta, x + \delta ) \to \Rbb^{ 2 }, \quad
  K( x ) = \big( x, y( x ) \big),
\end{equation}
gdzie $\delta > 0$. Wobec tego wektor $\vecv$ styczny do~tej krzywej całkowej
w~tym punkcie jest dany przez
\begin{equation}
  \label{eq:Matwiejew-Metody-calkowania-ETC-45}
  \vecv \HorSpaceThree ( x, y ) =
  \big[ 1, y'( x ) \big].
\end{equation}
Z~analizy matematycznej wiemy, że~jeśli $\alpha$ jest kątem pod jakim styczna do
krzywej w~punkcie $\big( x, y( x ) \big)$, to zachodzi
\begin{equation}
  \label{eq:Matwiejew-Metody-calkowania-ETC-46}
  \tan( \alpha ) = y'( x ).
\end{equation}
Ponieważ zaś $\tan( \alpha )$ jest wyznaczony przez pole kierunków, pozwala nam
to jednoznacznie wyznaczyć wektor styczny do krzywej całkowej. Możemy teraz
skorzystać z~\eqref{eq:Matwiejew-Metody-calkowania-ETC-41} i~otrzymać wzór
na pole wektorowe odpowiadające temu równaniu różniczkowemu:
\begin{equation}
  \label{eq:Matwiejew-Metody-calkowania-ETC-47}
  \vecv \HorSpaceThree ( x, y ) =
  \big[ 1, f( x, y ) \big].
\end{equation}
Możemy teraz odwrócić rozumowanie i~stwierdzić, że~rozwiązanie równania
różniczkowego oznacza znalezienie krzywych całkowych na płaszczyźnie $xy$,
takich że w~każdym ich punkcie ich wektor styczny jest dany przez pole
wektorowe zdefiniowane wzorem \eqref{eq:Matwiejew-Metody-calkowania-ETC-47}.

Przyjrzyjmy~się jeszcze raz temu rozumowaniu. Ponieważ w~omawianym
przypadku wiemy, że~pierwsza pola wektorowego zadanego przez równanie
\eqref{eq:Matwiejew-Metody-calkowania-ETC-42} musi~się zawsze równać jeden:
\begin{equation}
  \label{eq:Matwiejew-Metody-calkowania-ETC-48}
  \vecv_{ x }( x, y ) = 1,
\end{equation}
więc biorąc to pod uwagę, jesteśmy w~stanie znając wartości pola kierunków
w~punkcie $( x, y )$, jesteśmy w~stanie odtworzyć wektor $\vecv( x, y )$.
Warunek ten wyklucza również możliwość zastosowania transformacji takich jak
\eqref{eq:Matwiejew-Metody-calkowania-ETC-42}.

Rozpatrzmy teraz sytuację analizowaną przez Arnolda. Niech więc będzie dane
pole wektorowe $\vecv : G \to \Rbb^{ 2 }$, $G \subset \Rbb^{ 2 }$. Szukać będziemy
funkcji postaci $\vecy : ( t_{ 0 }, t_{ 1 } ) \to G$, spełniający następujący
równanie
\begin{equation}
  \label{eq:Matwiejew-Metody-calkowania-ETC-49}
  \frac{ d \vecy( t ) }{ d t } =
  \vecf \HorSpaceOne \big( t, \vecy( t ) \big) =
  \vecv \HorSpaceThree \big( \vecy( t ) \big),
\end{equation}
będące odpowiednikiem \eqref{eq:Matwiejew-Metody-calkowania-ETC-41}.
Zwyczajowo użyliśmy symbolu $t$ dla oznaczenia parametru od którego zależy
rozwiązanie $\vecy$. Dodatkowo współrzędne kartezjańskie w~zbiorze $G$
będziemy oznaczać $y_{ 1 }$ i~$y_{ 2 }$. Ostatnie równanie można
również zapisać jako
\begin{equation}
  \label{eq:Matwiejew-Metody-calkowania-ETC-50}
  \begin{bmatrix}
    y'_{ 1 }( t ) \\
    y'_{ 2 }( t )
  \end{bmatrix} =
  \begin{bmatrix}
    v_{ 1 }\big( y_{ 1 }( t ), y_{ 2 }( t ) \big) \\
    v_{ 2 }\big( y_{ 1 }( t ), y_{ 2 }( t ) \big)
  \end{bmatrix}\!.
\end{equation}

Jeśli teraz weźmiemy pole wektorowe $\vecv$ i~narysujemy zarówno je samo,
jak i~krzywe których wektory styczne w~każdym punkcie są równe wektorom
pola $\vecv$ w~tym samym punkcie, to tym razem nie otrzymamy pełnej
informacji o~ruchu punktu, który jest opisany równaniem
\eqref{eq:Matwiejew-Metody-calkowania-ETC-48}, bowiem te krzywe będą
obrazować tylko tor danego ruchu. Przykładowo, krzywa widoczna na rysunku
????
pokazuje nam, że~tor ruchu punktu tworzy pewną krzywą zamkniętą, ale nic
nie mówi nam o tym, czy punkt porusza się po niej ze stałą prędkością,
czy też raz zwalnia, raz przyśpiesza?

Aby uzyskać jak poprzednio pełną informację o~ruchu powinniśmy rozpatrywać
krzywe w~zbiorze $( t_{ 0 }, t_{ 1 } ) \times G$, na którym określone jest pole
wektorowe
\begin{equation}
  \label{eq:Matwiejew-Metody-calkowania-ETC-51}
  \vecv_{ \, \textrm{ext} }( t, y_{ 1 }, y_{ 2 } ) =
  \begin{bmatrix}
    1 \\
    v_{ 1 }( y_{ 1 }, y_{ 2 } ) \\
    v_{ 2 }( y_{ 1 }, y_{ 2 } )
  \end{bmatrix}\!.
\end{equation}
Dopiero znalezienie odpowiednika krzywych całkowych dla tego pola da nam
pełną informację o~ruchu, z~tego samego powodu co poprzednio: odpowiednik
krzywej całkowej w~zbiorze $( t_{ 0 }, t_{ 1 } ) \times G$ wyznacza funkcje
wedle definicji danej przez teorię mnogości. Powyżej podaliśmy pole
wektorowe dla tej przestrzeni, kwestią czy można na nią uogólnić pojęcie
pola kierunków, nie będziemy~się zajmować.

Różnica między równaniem \eqref{eq:Matwiejew-Metody-calkowania-ETC-41}
i~\eqref{eq:Matwiejew-Metody-calkowania-ETC-48} wynika głównie z~rodzaju
funkcji, które są ich rozwiązaniami. Dokładniej z~tego, że~funkcje typu
$y : ( t_{ 0 }, t_{ 1 } ) \to \Rbb$ są niezwykle proste, gdyż ich
przeciwdziedzina jest niezwykle prosta. W~tym przypadku torem ruchu jest
zawsze odcinek $( a, b )$, ewentualnie odcinek domknięty z jednej lub obu
stron, podczas gdy w~wypadku funkcji
$\vecy : ( t_{ 0 }, t_{ 1 } ) \to \Rbb^{ 2 }$ torami ruchu są
gładkie krzywe, czyli obiekty geometryczne o~nieporównywalnie większym
bogactwie kształtów i~zachowań niż odcinki.

Na koniec zwróćmy uwagę na rzecz stosunkowo oczywistą. Ze~względu na to,
że~w~poprzednim przypadku płaszczyzna $xy$ zamierała zarówno informacje
o~przebiegu ruchu jak i~o~jego torze, to wystarczyło określić na niej pole
kierunków. W~obecnie rozważanym przypadku, krzywe w~obszarze $G$ zawierają
tylko informacje o~torze ruchu, skutkiem czego gdybyśmy określili pole
kierunków, pole odcinków o~długości jeden, rozwiązania ruchu nie byłyby
jednoznacznie wyznaczone.

Możemy to zilustrować prostym przykładem. Weźmy $G = \Rbb^{ 2 }$ i~pole
wektorowe $\vecv( y_{ 1 }, y_{ 2 } ) = ( 1, 0 )$. Łatwo sprawdzić,
że~$\vecy_{ 1 }( t ) = [ t, 0 ]$ jest rozwiązaniem równania
\eqref{eq:Matwiejew-Metody-calkowania-ETC-49},
ale~$\vecy_{ 2 }( t ) = [ -t, 0 ]$ już nim nie jest. Gdybyśmy zamiast tego
rozpatrywali pole odcinków jednostkowy, takie że~w~punkcie
$( y_{ 1 }, y_{ 2 } )$ odcinek wyznacza ten sam kierunek na którym leży
wektor $\vecv( y_{ 1 }, y_{ 2 } )$ i~żądali od~rozwiązania tylko,
by~wektor $\vecy'( t )$ znajdujący~się w~punkcie $\vecy( t )$ posiadał
kierunek wyznaczony przez pole kierunków, wówczas $\vecy_{ 1 }( t )$
i~$\vecy_{ 2 }( t )$~byłby rozwiązaniami tak postawionego problemu.

Do tych rozważań należy któregoś dnia wrócić, poprawić je i~uzupełnić wedle
tego co jest napisane w~dziele Arnolda
\cite{ArnoldRownaniaRozniczkoweZwyczajne1975}.

\vspace{\VerSpaceFour}





\noindent
\Str{17} W~świetle przykładu 3 ze~strony 19, warto zatrzymać~się dłużej
nad podstawowymi własnościami izoklin. Przyjmijmy, że~izoklina dana jest
przez funkcję $h( x )$ określoną na zbiorze $( x_{ 0 }, x_{ 1 } )$. Tym
samym równanie izokliny jako krzywej jest postaci
\begin{equation}
  \label{eq:Matwiejew-Metody-calkowania-ETC-52}
  K( x ) = \big( x, h( x ) \big).
\end{equation}
Zgodnie z~definicją izokliny, dla pewnego $k \in \Rbb$ zachodzi
\begin{equation}
  \label{eq:Matwiejew-Metody-calkowania-ETC-53}
  k = f\big( x, h( x ) \big).
\end{equation}
Zauważmy, że~aby izoklina była rozwiązaniem danego równania różniczkowego
musiałoby ponadto zachodzi
\begin{equation}
  \label{eq:Matwiejew-Metody-calkowania-ETC-54}
  h'( x ) = k.
\end{equation}
Przykład~3 pokazuje, że~izoklina może być rozwiązaniem równania
różniczkowego które ją definiuje, ale jest to sytuacja dość szczególna.
Przykład~1 ze strony~17 jasno pokazuje, że~żadna z~izoklin równania
\begin{equation}
  \label{eq:Matwiejew-Metody-calkowania-ETC-55}
  y'( x ) = 2 x,
\end{equation}
nie jest jego rozwiązaniem.

\vspace{\VerSpaceFour}




\noindent
\Str{18} Ta strona jest napisana w~trochę bałaganiarski sposób. W~tym samym
paragrafie w~którym analizuje~się czy krzywe całkowe są rosnące czy
malejące, wprowadza~się pojęcie \textbf{linii ekstremów}, za~to
w~paragrafie o~tym czy krzywe całkowe są wypukłe czy wklęsłe, wprowadza~się
pojęcie \textbf{linii punktów przegięcia}. Te pojęcia można byłoby
wprowadzić w~bardziej elegancki i~łatwiejszy w~zrozumieniu sposób.

Najpierw w~jednym paragrafie rozważamy prawą stronę równania
\begin{equation}
  \label{eq:Matwiejew-Metody-calkowania-ETC-56}
  y'( x ) = f\big( x, y( x ) \big),
\end{equation}
dochodząc do wniosku, że~jeśli w~danym obszarze $\Ocal$ płaszczyzny $x y$
jest ona dodatnia (ujemna), to wszystkie krzywe całkowe w~tym obszarze
są skierowane ku górze (ku dołowi). Następnie możemy przeprowadzić
podobną analizę drugiej pochodnej funkcji $y( x )$, różniczkują
po $x$ równania \eqref{eq:Matwiejew-Metody-calkowania-ETC-56}:
\begin{equation}
  \label{eq:Matwiejew-Metody-calkowania-ETC-57}
  y''( x ) =
  \frac{ \partial f\big( x, y( x ) \big)}{ \partial x } +
  \frac{ \partial f\big( x, y \big) }{ \partial y }\Big|_{ y = y( x ) } \,
  f\big( x, y( x ) \big).
\end{equation}
Jeśli prawa strona tego wzoru ma znak dodatni (ujemny) w~obszarze $\Ocal$
płaszczyzny to~funkcja $y( x )$ jest w~tym obszarze wypukła (wklęsła).

W~następnym paragrafie zdefiniowalibyśmy pojęcie linii ekstremów i~linii
punktów przegięcia, teraz zaś podamy tu bardziej sformalizowaną definicję
linii ekstremów, mając nadzieję, że~pomoże to naświetlić lepiej omawiane
problemy. Niech $h( x )$ będzie funkcją określoną na odcinku
$( x_{ 1 }, x_{ 2 } )$.
Mówimy, że~$h( x )$ wyznacza linię ekstremów równania
\eqref{eq:Matwiejew-Metody-calkowania-ETC-56} jeśli przez każdy
punkt $( x_{ 0 }, y_{ 0 } )$, gdzie $y_{ 0 } = h( x_{ 0 } )$,
$x_{ 0 } \in ( x_{ 1 }, x_{ 2 } )$, przechodzi
krzywa całkowa dana funkcją $y( x )$, która to funkcja jest określona
przynajmniej na pewnym przedziale $( x_{ 0 } - \delta, x_{ 0 } + \delta )$, gdzie
$\delta > 0$, która ma w~punkcie $x_{ 0 }$ ekstremu. Jeśli w~jakimś otoczeniu
punktu $x_{ 0 }$ krzywa dana funkcją $y( x )$ pokrywa~się tą daną przez
$h( x )$, to wedle tego co napisano w~książce, nie zachodzi potrzeba by
funkcja $y( x )$ miała w~punkcie $x_{ 0 }$ ekstremu. Trzeba jednak
przyznać, że~fragment książki który mówi o~pokrywaniu się krzywej całkowej
z~linią ekstremów jest trochę niejednoznaczny i~powyższe interpretacja nie
musi być poprawna.

W~przypadku linii punktów przegięcia, nie ma żadnego komentarza odnośnie
przypadku, gdy~krzywa całkowa pokrywa~się tą linią. W~tym momencie nie
umiemy powiedzieć, czy definicja w~obecnej formie jest poprawna, czy
też~jej część została przypadkiem pominięta.

\vspace{\VerSpaceFour}





\noindent
\Str{19} Równanie
\begin{equation}
  \label{eq:Matwiejew-Metody-calkowania-ETC-58}
  \frac{ d y( x ) }{ d x } =
  \frac{ y( x ) }{ x }
\end{equation}
analizowane w~przykładzie~3 na tej stronie można rozwiązać w~inny sposób,
który pod pewnymi względami może być prostszy do~zrozumienia, dlatego
uważamy, iż~warto go tu zaprezentować. Z~drugiej
strony, prezentując poniżej to podejście wchodzimy w~pewne subtelności
pojęcia dziedziny rozwiązań równania różniczkowego i~izoklin, więc niniejsze
rozważania mogą okazać~się trudniejsze do zrozumienia, niż~to co można
znaleźć w~książce.

Mianowicie, zaczynamy od zbadania jego izoklin, czyli szukamy rozwiązań
równania
\begin{equation}
  \label{eq:Matwiejew-Metody-calkowania-ETC-59}
  k = \frac{ y_{ \, \izo }( x ) }{ x }, \quad
  k \in \Rbb.
\end{equation}
Dla zaznaczenia, że~izoklina nie musi być rozwiązanie równania
\eqref{eq:Matwiejew-Metody-calkowania-ETC-58}, szukaną funkcję oznaczyliśmy
$y_{ \, \izo }( x )$, a~nie $y( x )$. Równanie to jest niezmiernie proste,
więc od razu dostajemy
\begin{equation}
  \label{eq:Matwiejew-Metody-calkowania-ETC-60}
  y_{ \, \izo }( x ) = k \, x, \quad
  k \in \Rbb.
\end{equation}
To co wymaga większej uwagi, jest to, że~$x = 0$ nie należy do dziedziny
równania \eqref{eq:Matwiejew-Metody-calkowania-ETC-59}, więc izokliny są
określone tylko dla $x \in ( -\infty, 0 ) \cup ( 0, +\infty )$. Czy jednak powinnyśmy
dopuszczać jako izokliny krzywe reprezentowane przez funkcje, których
dziedziną nie jest przedział $( a, b ) \subset \Rbb$ (ewentualnie przedział jedno- lub obustronnie domknięty)?

Jak wspomniano wyżej, zwykle chcemy by rozwiązanie równania różniczkowego
było przedziałem otwartym $( a, b ) \subset \Rbb$, a~jak zaraz~się okaże,
izokliny tego równania są również jego rozwiązaniami, więc przyjmiemy dla
nich podobną konwencję. Będziemy więc szukać izoklin, których funkcja
$y_{ \, \izo }( x )$ ma za dziedzinę możliwie największy przedział otwarty
zawarty w~$\Rbb$. Na mocy tego, rozbijamy funkcję daną przez równanie
\eqref{eq:Matwiejew-Metody-calkowania-ETC-60} na dwie funkcje:

\vspace{\negVerSpaceFour}


\begin{subequations}

  \begin{align}
    \label{eq:Matwiejew-Metody-calkowania-ETC-61-A}
    y_{ \, \izo, \, + }( x )
    &=
      k \, x, \quad
      x > 0, \\
    \label{eq:Matwiejew-Metody-calkowania-ETC-61-B}
    y_{ \, \izo, \, - }( x )
    &=
      k \, x, \quad
      x < 0.
  \end{align}

\end{subequations}


\noindent
Tym samy za izokliny równania \eqref{eq:Matwiejew-Metody-calkowania-ETC-58}
będziemy uważać nie proste przechodzące przez początek układu
współrzędnych, lecz odpowiednie półproste wychodzące z~tego punktu. Może to
być zgodne z~intencjami Matwiejewa wyłożonymi w~opisie omawianego
przykładu, lecz fragment o~tym, że~te półproste są zarówno krzywymi
całkowymi jak i~izoklinami, można interpretować na kilka sposobów.

Łatwo zauważyć, że~$y'( x ) = k$ i~$y( x ) / x = k$, więc mogłoby~się
izokliny są zarazem rozwiązaniami równania
\eqref{eq:Matwiejew-Metody-calkowania-ETC-58}. Tak jak w~przypadku izoklin,
rozwiązania równania są określone na zbiorze $( -\infty, 0 )$
albo $( 0, +\infty )$. Zgodność z~tekstem książki jest prawie pewna,
bo~Matwiejew tylko półproste wychodzące z~układu współrzędnych nazywa
krzywymi całkowymi badanego równania. Niemniej warto zaznaczyć, że~funkcja
\begin{equation}
  \label{eq:Matwiejew-Metody-calkowania-ETC-62}
  y_{ - }( x ) = k \, x, \quad
  x < 0,
\end{equation}
opisuje raczej półprostą wchodzącą do~środka układu współrzędnych. Jest
to~jednak drobna szczegół nazewniczy, który nie powinien prowadzić
do~nieporozumień.

Dla porządku przeanalizujemy jeszcze przypadek $x = 0$, $y \neq 0$. W~takiej
sytuacji musimy rozpatrzyć równanie
\begin{equation}
  \label{eq:Matwiejew-Metody-calkowania-ETC-63}
  \frac{ d x( y ) }{ d y } =
  \frac{ x( y ) }{ y }.
\end{equation}
Równanie izoklin rozwiązujemy tak jak poprzednio dostając

\vspace{\negVerSpaceFour}


\begin{subequations}

  \begin{align}
    \label{eq:Matwiejew-Metody-calkowania-ETC-64-A}
    x_{ \, \izo, \, + }( y )
    &=
      k_{ 1 } \, y, \quad
      y > 0, \\
    \label{eq:Matwiejew-Metody-calkowania-ETC-64-B}
    x_{ \, \izo, \, - }( y )
    &=
      k_{ 1 } \, y, \quad
      y < 0.
  \end{align}

\end{subequations}


\noindent
Tak jak poprzednio łatwo sprawdzamy, że~izokliny są też rozwiązaniami
równania \eqref{eq:Matwiejew-Metody-calkowania-ETC-63}. W~szczególności,
rozwiązaniami naszego równania są dwie półproste

\vspace{\negVerSpaceFour}


\begin{subequations}

  \begin{align}
    \label{eq:Matwiejew-Metody-calkowania-ETC-65-A}
    x_{ + }( y )
    &=
      0, \quad
      y > 0, \\
    \label{eq:Matwiejew-Metody-calkowania-ETC-65-B}
    x_{ - }( y )
    &=
      0, \quad
      y < 0.
  \end{align}

\end{subequations}


Na koniec zwróćmy uwagę na~następujący prosty fakt. Jeśli zapiszemy równania
\eqref{eq:Matwiejew-Metody-calkowania-ETC-61-A}-\eqref{eq:Matwiejew-Metody-calkowania-ETC-61-B} i~\eqref{eq:Matwiejew-Metody-calkowania-ETC-64-A}-\eqref{eq:Matwiejew-Metody-calkowania-ETC-64-B} jako równania uwikłane na funkcje:

\vspace{\negVerSpaceFour}


\begin{subequations}

  \begin{align}
    \label{eq:Matwiejew-Metody-calkowania-ETC-66-A}
    y - k \, x = 0, \\
    \label{eq:Matwiejew-Metody-calkowania-ETC-66-B}
    x - k_{ 1 } \, y = 0,
  \end{align}

\end{subequations}


\noindent
gdzie $k, k_{ 1 } \in \Rbb$, a~równania te chcemy rozwiązać osobno w~zbiorze
$\Rbb \setminus ( 0, \, 0 )$, rozwiązania zaś mają być określone na zbiorach postaci
$( a, b )$, to możemy zaobserwować, iż~równania
\eqref{eq:Matwiejew-Metody-calkowania-ETC-66-A}
i~\eqref{eq:Matwiejew-Metody-calkowania-ETC-66-B} są sobie w~oczywisty
sposób częściowo równoważne. Jeśli bowiem $k \neq 0$, to równanie
\eqref{eq:Matwiejew-Metody-calkowania-ETC-66-A} przechodzi w~równanie
\eqref{eq:Matwiejew-Metody-calkowania-ETC-66-B}
z~$k_{ 1 } = \frac{ 1 }{ k }$ i~odwrotnie.

Warunek, że~rozwiązania równań
\eqref{eq:Matwiejew-Metody-calkowania-ETC-66-A}
i~\eqref{eq:Matwiejew-Metody-calkowania-ETC-66-B} mają być określone na
zbiorach postaci $( a, b )$ nie jest potrzebny by omawiana równoważność
zachodziła. Dodaliśmy ją, by rozwiązania równania uwikłanego miały tę samą
dziedzinę co izokliny analizowane wcześniej.

\vspace{\VerSpaceFour}





\noindent
\Str{21} Przyjrzyjmy~się następującemu tekstowi jaki możemy znaleźć na tej
stronie.
\begin{quote}

  Przede wszystkim z~kursu analizy wiadomo, że~rozwiązanie (19) jest funkcją
  zmiennej niezależnej~$x$ różniczkowalną w~sposób ciągły (${}^{ 2 }$).
  Geometrycznie oznacza to, że~przez punkt $( x_{ 0 }, y_{ 0 } )$ przechodzi
  jedna i~tylko jedna krzywa całkowa.

\end{quote}
Wzór (19) o~którym tu mowa to
\begin{equation}
  \label{eq:Matwiejew-Metody-calkowania-ETC-67}
  y( x ) =
  \int_{ x_{ 0 } }^{ x } f( x ) \, dx + y_{ 0 }.
\end{equation}
Jakoś nie widzę, żeby z~tego, iż~powyższa funkcja jest różniczkowalna
w~sposób ciągły implikowało to, że~przez każdy punkt $( x_{ 0 }, y_{ 0 } )$
przechodziło tylko jedna krzywa całkowa. Tak w~istocie jest, co wiem na
podstawie twierdzenia Picarda, które w~tej książce pierwszy razy zostanie
wspomniane na stronie~24.

\vspace{\VerSpaceFour}

{\Large Od strony 20 należy zacząć ponowne czytanie książki.}







\noindent
\Str{31} ????




















% ##################
\newpage

\CenterBoldFont{Błędy}


\begin{center}

  \begin{tabular}{|c|c|c|c|c|}
    \hline
    Strona & \multicolumn{2}{c|}{Wiersz} & Jest
                              & Powinno być \\ \cline{2-3}
    & Od góry & Od dołu & & \\
    \hline
    5   & &  3 & wierdzenia & twierdzenia \\
    15  & & & ono określa & określa ono \\
    % & & & & \\
    % & & & & \\
    \hline
  \end{tabular}

\end{center}

\vspace{\VerSpaceSix}


\noindent
\StrWierszG{15}{13} \\
\Jest  w~sensie ustępu \\
\Powin w~sensie zdefiniowanym w~ustępie \\
\StrWierszD{20}{2} \\
\Jest  i~nie ma rozwiązania określonego w~tym samym przedziale
nie~identycznego z~rozwiązaniem $y = y( x )$ chociażby w~jednym
punkcie przedziału $| x - x_{ 0 } | \leq h$ różnym od~punktu $x = x_{ 0 }$. \\
\Powin i~nie istnieje rozwiązanie określone w~przedziale
$| x - x_{ 0 } | \leq h_{ 1 } \leq h$, $h_{ 1 } > 0$, spełniające ten warunek
początkowy, które nie byłoby równe rozwiązaniu $y = y( x )$ w~każdym
punkcie tego przedziału. \\
\StrWierszG{21}{2} \\
\Jest  nie jedno rozwiązanie \\
\Powin ma co najmniej dwa rozwiązania \\


% ############################







% ####################################################################
% ####################################################################
% Bibliografia

\bibliographystyle{plalpha}

\bibliography{MathematicsBooks}{}





% ############################

% Koniec dokumentu
\end{document}
