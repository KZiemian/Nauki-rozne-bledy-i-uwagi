% ---------------------------------------------------------------------
% Podstawowe ustawienia i pakiety
% ---------------------------------------------------------------------
\RequirePackage[l2tabu, orthodox]{nag} % Wykrywa przestarzałe i niewłaściwe
% sposoby używania LaTeXa. Więcej jest w l2tabu English version.


\documentclass[a4paper,11pt]{article}
% {rozmiar papieru, rozmiar fontu}[klasa dokumentu]
\usepackage[MeX]{polski} % Polonizacja LaTeXa, bez niej będzie pracował
% w języku angielskim.
\usepackage[utf8]{inputenc} % Włączenie kodowania UTF-8, co daje dostęp
% do polskich znaków.
\usepackage[T1]{fontenc} % Potrzebne do używania fontów Latin Modern.
\usepackage{lmodern} % Wprowadza fonty Latin Modern.



% ----------------------------
% Podstawowe pakiety (niezwiązane z ustawieniami języka)
% ----------------------------
\usepackage{microtype} % Twierdzi, że poprawi rozmiar odstępów w tekście.
\usepackage{graphicx} % Wprowadza bardzo potrzebne komendy do wstawiania
% grafiki.
\usepackage{verbatim} % Poprawia otoczenie VERBATIME.
\usepackage{textcomp} % Dodaje takie symbole jak stopnie Celsiusa,
% wprowadzane bezpośrednio w tekście.
\usepackage{vmargin} % Pozwala na prostą kontrolę rozmiaru marginesów,
% za pomocą komend poniżej. Rozmiar odstępów jest mierzony w calach.
% ----------------------------
% MARGINS
% ----------------------------
\setmarginsrb
{ 0.7in}  % left margin
{ 0.6in}  % top margin
{ 0.7in}  % right margin
{ 0.8in}  % bottom margin
{  20pt}  % head height
{0.25in}  % head sep
{   9pt}  % foot height
{ 0.3in}  % foot sep



% ------------------------------
% Często przydatne pakiety
% ------------------------------
% \usepackage{csquotes} % Pozwala w prosty sposób wstawiać cytaty do tekstu.
\usepackage{xcolor} % Pozwala używać kolorowych czcionek (zapewne dużo
% więcej, ale ja nie potrafię nic o tym powiedzieć).



% ------------------------------
% Pakiety do tekstów z nauk przyrodniczych
% ------------------------------
\let\lll\undefined % Amsmath gryzie się z pakietami do języka
% polskiego, bo oba definiują komendę \lll. Aby rozwiązać ten problem
% oddefiniowuję tę komendę, ale może tym samym pozbywam się dużego Ł.
\usepackage[intlimits]{amsmath} % Podstawowe wsparcie od American
% Mathematical Society (w skrócie AMS)
\usepackage{amsfonts, amssymb, amscd, amsthm} % Dalsze wsparcie od AMS
\usepackage{bm}  % Daję komendę \bm do pogrubionej czcionki matematycznej
% \usepackage{siunitx} % Do prostszego pisania jednostek fizycznych
% \usepackage{upgreek} % Ładniejsze greckie litery
% Przykładowa składnia: pi = \uppi
\usepackage{slashed} % Pozwala w prosty sposób pisać slash Feynmana.
\usepackage{calrsfs} % Zmienia czcionkę kaligraficzną w \mathcal
% na ładniejszą. Może w innych miejscach robi to samo, ale o tym nic
% nie wiem.



% ------------------------------
% Tworzenie środowisk (?) „Twierdzenie”, „Definicja”, „Lemat”, etc.
% ------------------------------
% Komenda wprowadzająca otoczenie „theorem” do pisania twierdzeń
% matematycznych.
\newtheorem{theorem}{Twierdzenie}
% Analogicznie jak powyżej
\newtheorem{definition}{Definicja}
\newtheorem{corollary}{Wniosek}



% ------------------------------
% Pakiety napisane przez użytkownika.
% Mają być w tym samym katalogu to ten plik .tex
% ------------------------------
% \usepackage{ODE} % Pakiet napisany między innymi dla tego pliku.
\usepackage{latexgeneralcommands}
\usepackage{mathcommands}




% --------------------------------------------------------------------
% Dodatkowe ustawienia dla języka polskiego
% --------------------------------------------------------------------
\renewcommand{\thesection}{\arabic{section}.}
% Kropki po numerach rozdziału (polski zwyczaj topograficzny)
\renewcommand{\thesubsection}{\thesection\arabic{subsection}}
% Brak kropki po numerach podrozdziału



% ------------------------------
% Ustawienia różnych parametrów tekstu
% ------------------------------
\renewcommand{\baselinestretch}{1.1}

% Ustawienie szerokości odstępów między wierszami w tabelach.
\renewcommand{\arraystretch}{1.4}



% ------------------------------
% Pakiet "hyperref"
% Polecano by umieszczać go na końcu preambuły.
% ------------------------------
\usepackage{hyperref} % Pozwala tworzyć hiperlinki i zamienia odwołania
% do bibliografii na hiperlinki.










% ---------------------------------------------------------------------
% Tytuł, autor, data
\title{Równania różniczkowe zwyczajne \\
  {\Large Błędy i~uwagi}}

\author{Kamil Ziemian}


% \date{}
% ---------------------------------------------------------------------










% ####################################################################
\begin{document}
% ####################################################################





% ######################################
\maketitle % Tytuł całego tekstu
% ######################################





% ##############################
\Work{ % Autor i tytuł dzieła
  Władimir Igoriewicz Arnold \\
  \textit{Równania różniczkowe zwyczajne},
  \cite{ArnoldRownaniaRozniczkoweZwyczajne1975}}


% ##################
\newpage

\CenterBoldFont{Błędy}


\begin{center}

  \begin{tabular}{|c|c|c|c|c|}
    \hline
    Strona & \multicolumn{2}{c|}{Wiersz} & Jest
                              & Powinno być \\ \cline{2-3}
    & Od góry & Od dołu & & \\
    \hline
    5   & &  7 & 1968 - 196 & 1968 - 1969 \\
    11  & 17 & & mechanice klasycznej & mechanice kwantowej \\
    15  & & 16 & rozdziale 6 & rozdziale 5 \\
    28  & 15 & & wzór (8) & wzór \\
    34  &  9 & & $\dot{ x }_{ 1 } = x_{ 2 }$ & $\dot{ x }_{ 1 } = x_{ 1 }$ \\
    47  & & 12 & $x_{ i } = \varphi_{ i }( x_{ 1 }, \ldots, x_{ n } )$
           & $x_{ i } = \varphi_{ i }( y_{ 1 }, \ldots, y_{ n } )$ \\
    53  & & 14 & obrót & obrót krzywych całkowych \\
    56  & 12 & & osobliwym & nieosobliwym \\
    61  &  7 & & $\vecxbold$???, $\vecalphabold_{ 0 }$
           & $\vecxbold$, $\vecalphabold$ \\
    64  & & 11 & ????$\vecgbold( t_{ 2 }, t_{ 1 }, \vecxbold )
                = \vecgbold^{ t_{ 2 } }_{ t_{ 1 } }( \vecxbold, t_{ 1 } )$
           & $\vecxbold^{ t_{ 2 } }_{ t_{ 1 } }( \vecxbold, t_{ 1 } )
             = \vecgbold( t_{ 2 }, t_{ 1 }, \vecxbold )$ \\
    64  & & 10 & $( \vecvarphibold( t ), t )$
           & $( t, \vecvarphibold( t ) )$ \\[0.3em]
    66  & 17 & & $\vecvbold( t, \vecxbold, \dot{ \vecalphabold } )$
           & $\vecvbold( t, \vecxbold, \vecalphabold )$??? \\[0.3em]
    70  & & 13 & $\dot{ p }_{ i } = \frac{ \partial H }{ \partial q_{ i } }$
           & $\dot{ p }_{ i } = -\frac{ \partial H }{ \partial q_{ i } }$ \\[0.3em]
    71  & & 15 & $\frac{ \partial \vecvbold_{ 0 } }{ \vecxbold }$
           & $\frac{ \partial \vecvbold_{ 0 } }{ \partial \vecxbold }$ \\[0.4em]
    72  &  4 & & „niezaburzonego” & „zaburzonego” \\
    73  &  5 & & $\vecxbold( 0$ & $\vecxbold( 0 )$ \\
    90  & &  3 & \textit{wraz z pochodną dla} $x = 0$
           & \textit{dla} $x = 0$ \\
    92  &  3 & & $U( x( O ) )$ & $U( x( 0 ) )$ \\[0.3em]
    123 &  6 & & $^{ \Rbb }A : \Cbb^{ m } \to { }^{ \Rbb } \Cbb^{ n }$
           & ${ }^{ \Rbb }A : { }^{ \Rbb } \Cbb^{ m }
             \to { }^{ \Rbb } \Cbb^{ n }$ \\
    125 &  5 & & $\mathbf{I}$ & $I$ \\
    % & & & & \\
    % & & & & \\
    % & & & & \\
    % & & & & \\
    \hline
  \end{tabular}

\end{center}

\vspace{\VerSpaceSix}


\noindent
\StrWd{66}{7} \\
\Jest  \textit{tyłu do~brzegu} \\
\Powin \textit{tyłu nieograniczenie albo~do~brzegu} \\
\StrWg{110}{9} \\
\Jest  sumą częściową szeregu --~iloczynu \\
\Powin jest sumą części wyrazów iloczynu \\



% ############################










% ############################
\Work{ % Autor i tytuł dzieła
  N. M. Matwiejew \\
  \textit{Metody całkowania równana różniczkowych zwyczajnych},
  \cite{MatwiejewMetodyCalkowaniaRownanRozniczkowychZwyczajnych1982}}

\vspace{0em}


% ##################
\CenterBoldFont{Błędy}


\begin{center}

  \begin{tabular}{|c|c|c|c|c|}
    \hline
    Strona & \multicolumn{2}{c|}{Wiersz} & Jest
                              & Powinno być \\ \cline{2-3}
    & Od góry & Od dołu & & \\
    \hline
    5   & &  9 & Dodzimy & Dowodzimy \\
    5   & &  8 & potkowych & początkowych \\
    5   & &  7 & poąątkowych & początkowych \\
    10  & & 19 & damy & mamy \\
    % 15  & & & & \\
    % & & & & \\
    % & & & & \\
    \hline
  \end{tabular}

\end{center}

\vspace{\VerSpaceSix}


\noindent
\StrWg{15}{13}
\Jest  w~sensie ustępu \\
\Powin w~sensie zdefiniowanym w~ustępie \\
\StrWd{20}{2} \\
\Jest  i~nie ma rozwiązania określonego w~tym samym przedziale
nie~identycznego z~rozwiązaniem $y = y( x )$ chociażby w~jednym
punkcie przedziału $\absOne{ x - x_{ 0 } } \leq h$ różnym
od~punktu $x = x_{ 0 }$. \\
\Powin i~nie istnieje inne rozwiązanie określone w~przedziale
$\absOne{ x - x_{ 0 } } \leq h_{ 1 } \leq h$ które nie byłoby równe
rozwiązaniu $y = y( x )$ w~każdym punkcie przedziału
$\absOne{ x - x_{ 0 } } \leq h_{ 1 }$. \\


% ############################










% ############################
\newpage

\Work{ % Autor i tytuł dzieła
  N. M. Matwiejew \\
  \textit{Metody całkowania równana różniczkowych zwyczajnych},
  \cite{MatwiejewMetodyCalkowaniaRownanRozniczkowychZwyczajnych1986}}

\vspace{0em}


% ##################
\CenterBoldFont{Uwagi}

\vspace{0em}

\noindent
Książka ta podąża za zwyczajem, że~jeśli argumenty danej funkcji są
odpowiednio oczywiste, to można je opuścić. Przykładowo na stronie~7
mamy równanie (1) postaci:
\begin{equation}
  \label{eq:Matwiejew-Metody-calkowania-ETC-01}
  F\left( x, y, y', y'', \ldots, y^{ ( n ) } \right) = 0,
\end{equation}
choć bardziej precyzyjne byłoby napisanie $y( x )$, $y'( x )$, etc. W~tych
notatkach, by uniknąć nieporozumień, we~wszystkich miejscach gdzie mowa jest
o~wartościach jakie przyjmują funkcje dla zadanych wartości argumentów,
będziemy~się starać podawać owe argumenty w~sposób jawny.

Ty samym, jeśli będziemy rozważali w~sposób ogólny funkcje
$y : \Rbb \to \Rbb$ to będziemy ją oznaczać symbolem $y$. Natomiast
w~przypadkach takich jak równanie
\begin{equation}
  \label{eq:Matwiejew-Metody-calkowania-ETC-02}
  y( x )^{ 2 } = x,
\end{equation}
będziemy~się starali zawsze wymienić wszystkie argumenty w~sposób jawny.

\vspace{\VerSpaceFour}










% ##################
\CenterBoldFont{Uwagi do~konkretnych stron}

\vspace{0em}


\noindent
\Str{7} W~tym miejscu warto~się zastanowić nad tym, jak w~ścisły sposób
zdefiniować równanie różniczkowe zwyczajne? Jak każdy taki ważny
problem w~matematyce, podaną definicję trzeba będzie niewątpliwie rozszerzać
i~modyfikować, tak aby objąć jej nowymi wersjami kolejne ważne problemy
matematyczne\footnote{Przez „problem matematyczny” rozumiemy tu zagadnienie
  matematyczne, które jawnie wymaga od nas znalezienia jego rozwiązania.
  Przykładowo, problemem matematyczny jest zagadnienie wyznaczenia
  funkcji $y( x )$, takiej że~jej pochodna w~punkcie $x$ jest równa
  $\sin\!\big( y( x ) \big)$. }. Niezależnie jednak od tego, warto spróbować
podać teraz ścisłą definicję które będzie obejmowała większość najbardziej
podstawowych problemów matematycznych jakie napotykamy w~teorii równań
różniczkowych zwyczajnych, nawet jeśli wiele innych będzie wykluczała.

Na tej stronie możemy znaleźć informacje, że~jeśli nie jest powiedziane
inaczej, to zakładamy, że~zarówno dziedzina jak i~przeciwdziedzina funkcji
$y$ są podzbiorami liczb rzeczywistych. Jeśli chodzi o~przeciwdziedzinę
to możemy przyjąć, że~zawsze jest ona równa $\Rbb$, bo wybór ten nie
powinien nigdzie grać roli. Jeśli chodzi o~dziedzinę danej funkcji, to
sprawa jest bardziej złożona.

Niech $A$ oznacza dziedzinę funkcji $y$. W~niniejszej książce naszym
podstawowym wymogiem jest to, by w~każdym punkcie dziedziny $A$ istniała
pochodna funkcji $y$, więc $A$ musi być dobrany w~taki sposób, by pojęcie
pochodnej w~punkcie $x \in A$ miało sens. Wbrew pozorom podanie klasy zbiorów
o~tej własności nie jest takie proste. Czy przykładowo funkcja
$f : \Qbb \to \Qbb$ dana zależnością
\begin{equation}
  \label{eq:Matwiejew-Metody-calkowania-ETC-01}
  f( x ) = x^{ 2 },
\end{equation}
jest różniczkowalna?

Pomimo problemów z~określeniem klasy tych zbiorów, w~większości przypadków
przyjmuje~się, że~zbiór $A$ jest odcinkiem otwartym $( a, b )$, gdzie
dopuszczamy sytuację, że $a$ i~$b$ mogą przyjmować wartość $\pm \infty$. Jak
zauważono na stronie ???, nie nastręcza również problemu określenie
funkcji~$y$ na odcinkach domkniętych z~jednej, bądź obu stron. W~przypadku
odcinka $[ a, b )$ przez pochodną w~punkcie~$a$ rozumiemy pochodną
prawostronną funkcji $y$ i~analogicznie postępuję w~przypadku odcinka
$( a, b ]$. Przy czym jeśli dany jest odcinek prawostronnie domknięty
$( a, b ]$ to musi zachodzić $-\infty < b < +\infty$, natomiast $a$ może przyjąć
wartość $-\infty$. Analogicznie zasady stosują~się do pozostały pozostał
przypadków odcinków jednostronnie, bądź obustronnie domkniętych. Do problemu
kształtu dziedziny powrócimy niedługo.

Niech teraz $O$ będzie otwartym podzbiorem $\Rbb^{ n + 1 }$, przy czym
elementy $\Rbb^{ n }$ będziemy oznaczać przez
$( x_{ 0 }, x_{ 1 }, \ldots, x_{ n - 1 }, x_{ n } )$
lub $x_{ 1 }, x_{ 2 }, \ldots, x_{ n }, x_{ n + 1 }$. Niech dana będzie funkcja
$F : O \to R$. Przyjmujemy, że zależy ona w~sposób \textbf{istotny}
(w~książce używa~się terminu „jawny”) od zmiennej $x_{ n }$, przez co
rozumiemy następującą własność. Istnieją takie liczby
$x_{ 0 }, x_{ 1 }, \ldots, x_{ n - 1 }, x_{ n }, x_{ n }'$, że~zachodzi
\begin{equation}
  \label{eq:Matwiejew-Metody-calkowania-ETC-02}
  F( x_{ 0 }, x_{ 1 }, \ldots, x_{ n - 1 }, x_{ n } ) \neq
  F( x_{ 0 }, x_{ 1 }, \ldots, x_{ n - 1 }, x_{ n }' ).
\end{equation}
Niech teraz $A_{ 0 }$ będzie rzutem zbioru $O$ na oś $x_{ 0 }$, czyli zbiorem
takim że~jeśli $x \in A_{ 0 }$ to istnieją takie liczby
$x_{ 1 }, x_{ 2 }, \ldots, x_{ n - 1 }, x_{ n }$,
że~$( x, x_{ 1 }, x_{ 2 }, \ldots, x_{ n - 1 }, x_{ n } ) \in O$. Zbiór ten będziemy
również oznaczać przez $\proj_{ 0 } O$.

\textbf{Równaniem
  różniczkowym zwyczajnym} nazywamy wyrażenie
\begin{equation}
  \label{eq:Matwiejew-Metody-calkowania-ETC-03}
  F\left( x, y( x ), y'( x ), y''( x ), \ldots, y^{ ( n ) }( x ) \right) = 0.
\end{equation}
\textbf{Rozwiązaniem równania \eqref{eq:Matwiejew-Metody-calkowania-ETC-02}}
nazywamy funkcję $y_{ 1 } : A_{ 1 } \to \Rbb$, gdzie $A_{ 1 } \subset A_{ 0 }$, która
posiada pochodne do rzędu $n$ włącznie w~każdym punkcie swojej dziedziny
i~dla której zachodzi
\begin{equation}
  \label{eq:Matwiejew-Metody-calkowania-ETC-04}
  F\left( x, y_{ 1 }( x ), y_{ 1 }'( x ), \ldots, y_{ 1 }^{ ( n ) }( x ) \right) =
  0, \qquad
  \forall x \in A_{ 1 }.
\end{equation}

Potrzebujemy teraz powrócić do problemu określenia dziedziny $A_{ 1 }$
funkcji $y$. Jak powiedzieliśmy wcześniej, standardowo przyjmuje~się,
że~$A_{ 1 } = ( a, b )$. Zazwyczaj żąda~się dodatkowo, by $A_{ 1 }$ był
największym odcinkiem otwarty zawartym w~zbiorze $A_{ 0 }$, w~którym jesteśmy
w~stanie zdefiniować rozwiązanie równania
\eqref{eq:Matwiejew-Metody-calkowania-ETC-03}. Co jednak zrobić w~przypadku
równań takich jak równanie $y'( x ) = y( x )^{ 2 }$ omawiane na stronie (?),
które posiada rozwiązanie postaci
\begin{equation}
  \label{eq:Matwiejew-Metody-calkowania-ETC-05}
  y( x ) = \frac{ 1 }{ 1 - x }.
\end{equation}
Funkcja ta jest określona na zbiorze $( -\infty, 1 ) \cup ( 1, +\infty )$ i~nie możemy
w~żaden sposób wybrać jednego z~tych przedziałów jako większego od drugiego.

Problem rozwiązań takich jak \eqref{eq:Matwiejew-Metody-calkowania-ETC-06}
będzie omawiany w~dalszym ciągu książki, wtedy ewentualnie powrócimy do
pojawiające~się przy nich dokładniej. W~chwili obecnej, motywowani tym
przykładem, poprzestaniemy na stwierdzeniu, że~jeśli to jest możliwe to
będziemy szukali rozwiązań określonych na całym zbiorze $A_{ 0 }$, co jest
równoważne stwierdzeniu $A_{ 1 } = A_{ 0 }$, co~wyklucza wszelkie
niejednoznaczności w~sposobie określenia dziedziny rozwiązania.
Niewątpliwie, ze względu na swoją prostotę interpretacyjną, optymalną
sytuacją jest $A_{ 1 } = A_{ 0 } = \Rbb$.

\vspace{\VerSpaceFour}





\noindent
\Str{8} Na tej stronie napotykamy po raz pierwszy konkretne równanie
różniczkowe
\begin{equation}
  \label{eq:Matwiejew-Metody-calkowania-ETC-06}
  y'( x ) - 2 x = 0.
\end{equation}
Z~równania tego w~prosty sposób odczytujemy funkcję $F$:
\begin{equation}
  \label{eq:Matwiejew-Metody-calkowania-ETC-07}
  F( x_{ 0 }, x_{ 1 }, x_{ 2 } ) = x_{ 2 } - 2 x_{ 0 }.
\end{equation}
Stajemy tu jednak przed problemem, jaka jest dziedzina funkcji~$F$?
W~dalszym ciągu jeśli nie powiedziano inaczej, za dziedzinę funkcji $F$
będziemy uważać największy zbiór na~którym jesteśmy w~stanie ją określić.
Ponieważ zgodnie z~tym co powiedziano wcześniej, dziedzina funkcji $F$ ma
być podzbiorem $\Rbb^{ n + 1 }$ dla pewnego~$n$, dziedziną funkcji określonej
przez \eqref{eq:Matwiejew-Metody-calkowania-ETC-05} jest $\Rbb^{ 3 }$.

Łatwo~się też przekonać, że~rozwiązaniem równania
\eqref{eq:Matwiejew-Metody-calkowania-ETC-06} jest funkcja
$y( x ) = x^{ 2 } + C$, której dziedzina wynosi $\Rbb$.

\vspace{\VerSpaceFour}





\noindent
\Str{10} Przy okazji wyprowadzania równania różniczkowe dla rodziny
wszystkich okręgów na płaszczyźnie $x y$ napotykamy po raz pierwszy na
pewien problem, który powróci do nas w przyszłości.

\vspace{\negVerSpaceThree}


\begin{subequations}

  \begin{align}
    \label{eq:Matwiejew-Metody-calkowania-ETC-08-A}
    1 + \big( y'( x ) \big)^{ 2 } + \big( y( x ) - b \big) y''( x )
    &= 0, \\
    \label{eq:Matwiejew-Metody-calkowania-ETC-08-B}
    3 y'( x ) y''( x ) + \big( y( x ) - b \big) y'''( x )
    &= 0.
  \end{align}

\end{subequations}


\noindent
Aby usunąć z~równania \eqref{eq:Matwiejew-Metody-calkowania-ETC-08-B}
parametr $b$ przyjmujemy, że~$y''( x ) \neq 0$, więc możemy przepisać równanie
\eqref{eq:Matwiejew-Metody-calkowania-ETC-08-A} jako
\begin{equation}
  \label{eq:Matwiejew-Metody-calkowania-ETC-09}
  y( x ) - b =
  -\frac{ 1 + \big( y'( x ) \big)^{ 2 } }{ y''( x ) }.
\end{equation}
Podstawiając tą zależność do \eqref{eq:Matwiejew-Metody-calkowania-ETC-08-B}
dostajemy
\begin{equation}
  \label{eq:Matwiejew-Metody-calkowania-ETC-10}
  3 y'( x ) y''( x ) -
  \frac{ 1 + \big( y'( x ) \big)^{ 2 } }{ y''( x ) } y'''( x ) = 0.
\end{equation}
Po pomnożeniu obustronnie przez $y''( x )$ dostajemy
\begin{equation}
  \label{eq:Matwiejew-Metody-calkowania-ETC-11}
  3 y'( x ) \big( y''( x ) \big)^{ 2 } -
  \Big( 1 + \big( y'( x ) \big)^{ 2 } \Big) y'''( x ) = 0.
\end{equation}
Choć równanie wyprowadziliśmy przy założeniu, że~$y''( x ) \neq 0$, to końcowa
jego postać jest dobrze określona również, gdy funkcja ta przyjmuje wartość
zero. Powstaje więc pytania, jaki jest zakres obowiązywania tego równania?



\vspace{\VerSpaceFour}





\noindent
\Str{15} Równanie (10) na tej stronie zostało bardzo elegancko wyprowadzone,
przy założeniu, że~funkcja uwikłana zależna od zmiennej $x$ dana związkiem
$\Phi( x, y ) = 0$, spełnia równanie różniczkowe
$y'( x ) = f\big( x, y( x ) \big)$. Nie poruszono jednak kwestii tego, czemu
mamy uważać tą funkcję uwikłaną za rozwiązanie badanego równania?
W~szczególności, jeśli jest rozwiązanie równania $\Phi( x, y ) = 0$ na funkcję
$y( x )$, to czy ona spełnia równanie $y'( x ) = f\big( x, y( x ) \big)$?
Przy standardowych założeniach o~funkcji $\Phi( x, y )$ odpowiedź na to
nostatnie pytanie jest twierdząca i~dzięki temu możemy uznać funkcję daną
równaniem uwikłanym za rozwiązanie badanego równania.

Dowód tego faktu jest następujący. Załóżmy, że~funkcja $\Phi( x, y )$ spełnia
założenia o~ciągłości i~istnieniu pochodnych, które są wymagane
w~standardowej wersji twierdzenia o~funkcji uwikłanej. Przyjmijmy dodatkowo,
że~$\Phi_{ y }'\big( x, y \big) \neq 0$ w~interesującym nas zakresie wartości
zmiennej~$x$, który oznaczymy~$A$. Tym samym możemy odwikłać funkcję
$y( x )$ dla wszystkich wartości $x \in A$. Przepiszmy teraz równanie (10)
w~bardziej jawnej formie jako
\begin{equation}
  \label{eq:Matwiejew-Metody-calkowania-ETC-12}
  \Phi_{ x }'\big( x, y( x ) \big) +
  \Phi_{ y }'\big( x, y( x ) \big) f\big( x, y( x ) \big) = 0.
\end{equation}
Równanie to przekształcamy w~oczywisty sposób do postaci
\begin{equation}
  \label{eq:Matwiejew-Metody-calkowania-ETC-13}
  -\frac{ \Phi_{ x }'\big( x, y( x ) \big) }
  { \Phi_{ y }'\big( x, y( x ) \big) } =
  f\big( x, y( x ) \big).
\end{equation}
Jak dobrze wiadomo, lewa strona tej równości przedstawia pochodną funkcji
uwikłanej $y'( x )$.

Na koniec możemy zauważyć, że~zarówno wzór
\eqref{eq:Matwiejew-Metody-calkowania-ETC-12} jak i~dobrze znaną zależność
\begin{equation}
  \label{eq:Matwiejew-Metody-calkowania-ETC-14}
  y'( x ) =
  -\frac{ \Phi_{ x }'\big( x, y( x ) \big) }{ \Phi_{ y }'\big( x, y'( x ) \big) },
\end{equation}
można wyprowadzić w~ten sam sposób, poprzez zróżniczkowanie
i~przekształcenie zależności $\Phi\big( x, y( x ) \big) = 0$. Choć z~tego
punktu widzenia równoważność równania
\eqref{eq:Matwiejew-Metody-calkowania-ETC-12}
i~$y'( x ) = f\big( x, y( x ) \big)$ może~się wydawać oczywista, to woleliśmy
przedyskutować to zagadnienie możliwie dokładnie.

\vspace{\VerSpaceFour}





\noindent
\Str{16}  \\

\vspace{\VerSpaceFour}





% \noindent
% \Str{31} \Dok




% ##################
\newpage

\CenterBoldFont{Błędy}


\begin{center}

  \begin{tabular}{|c|c|c|c|c|}
    \hline
    Strona & \multicolumn{2}{c|}{Wiersz} & Jest
                              & Powinno być \\ \cline{2-3}
    & Od góry & Od dołu & & \\
    \hline
    5   & &  3 & wierdzenia & twierdzenia \\
    % 10  & & 19 & damy & mamy \\
    15  & & & ono określa & określa ono \\
    % & & & & \\
    % & & & & \\
    \hline
  \end{tabular}

\end{center}

\vspace{\VerSpaceSix}


\noindent
\StrWg{15}{13}
\Jest  w~sensie ustępu \\
\Powin w~sensie zdefiniowanym w~ustępie \\
% \StrWd{20}{2} \\
% \Jest  i~nie ma rozwiązania określonego w~tym samym przedziale
% nie~identycznego z~rozwiązaniem $y = y( x )$ chociażby w~jednym
% punkcie przedziału $\absOne{ x - x_{ 0 } } \leq h$ różnym
% od~punktu $x = x_{ 0 }$. \\
% \Powin i~nie istnieje inne rozwiązanie określone w~przedziale
% $\absOne{ x - x_{ 0 } } \leq h_{ 1 } \leq h$ które nie byłoby równe
% rozwiązaniu $y = y( x )$ w~każdym punkcie przedziału
% $\absOne{ x - x_{ 0 } } \leq h_{ 1 }$. \\


% ############################







% ####################################################################
% ####################################################################
% Bibliografia

\bibliographystyle{plalpha}

\bibliography{MathComScienceBooks}{}





% ############################

% Koniec dokumentu
\end{document}
