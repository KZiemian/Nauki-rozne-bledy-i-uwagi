% ---------------------------------------------------------------------
% Podstawowe ustawienia i pakiety
% ---------------------------------------------------------------------
\RequirePackage[l2tabu, orthodox]{nag} % Wykrywa przestarzałe i niewłaściwe
% sposoby używania LaTeXa. Więcej jest w l2tabu English version.


\documentclass[a4paper,11pt]{article}
% {rozmiar papieru, rozmiar fontu}[klasa dokumentu]
\usepackage[MeX]{polski} % Polonizacja LaTeXa, bez niej będzie pracował
% w języku angielskim.
\usepackage[utf8]{inputenc} % Włączenie kodowania UTF-8, co daje dostęp
% do polskich znaków.
\usepackage[T1]{fontenc} % Potrzebne do używania fontów Latin Modern.
\usepackage{lmodern} % Wprowadza fonty Latin Modern.



% ------------------------------
% Podstawowe pakiety (niezwiązane z ustawieniami języka)
% ------------------------------
\usepackage{microtype} % Twierdzi, że poprawi rozmiar odstępów w tekście.
\usepackage{graphicx} % Wprowadza bardzo potrzebne komendy do wstawiania
% grafiki.
\usepackage{verbatim} % Poprawia otoczenie VERBATIME.
\usepackage{textcomp} % Dodaje takie symbole jak stopnie Celsiusa,
% wprowadzane bezpośrednio w tekście.
\usepackage{vmargin} % Pozwala na prostą kontrolę rozmiaru marginesów,
% za pomocą komend poniżej. Rozmiar odstępów jest mierzony w calach.
% ------------------------------
% MARGINS
% ------------------------------
\setmarginsrb
{ 0.7in}  % left margin
{ 0.6in}  % top margin
{ 0.7in}  % right margin
{ 0.8in}  % bottom margin
{  20pt}  % head height
{0.25in}  % head sep
{   9pt}  % foot height
{ 0.3in}  % foot sep



% ------------------------------
% Często przydatne pakiety
% ------------------------------
% \usepackage{csquotes} % Pozwala w prosty sposób wstawiać cytaty do tekstu.
\usepackage{xcolor} % Pozwala używać kolorowych czcionek (zapewne dużo
% więcej, ale ja nie potrafię nic o tym powiedzieć).



% ------------------------------
% Pakiety do tekstów z nauk przyrodniczych
% ------------------------------
\let\lll\undefined % Amsmath gryzie się z pakietami do języka
% polskiego, bo oba definiują komendę \lll. Aby rozwiązać ten problem
% oddefiniowuję tę komendę, ale może tym samym pozbywam się dużego Ł.
\usepackage[intlimits]{amsmath} % Podstawowe wsparcie od American
% Mathematical Society (w skrócie AMS)
\usepackage{amsfonts, amssymb, amscd, amsthm} % Dalsze wsparcie od AMS
\usepackage{bm}  % Daję komendę \bm do pogrubionej czcionki matematycznej
% \usepackage{siunitx} % Do prostszego pisania jednostek fizycznych
% \usepackage{upgreek} % Ładniejsze greckie litery
% Przykładowa składnia: pi = \uppi
\usepackage{slashed} % Pozwala w prosty sposób pisać slash Feynmana.
\usepackage{calrsfs} % Zmienia czcionkę kaligraficzną w \mathcal
% na ładniejszą. Może w innych miejscach robi to samo, ale o tym nic
% nie wiem.



% ------------------------------
% Tworzenie środowisk (?) „Twierdzenie”, „Definicja”, „Lemat”, etc.
% ------------------------------
% Komenda wprowadzająca otoczenie „theorem” do pisania twierdzeń
% matematycznych.
\newtheorem{theorem}{Twierdzenie}
% Analogicznie jak powyżej
\newtheorem{definition}{Definicja}
\newtheorem{corollary}{Wniosek}



% ------------------------------
% Pakiety napisane przez użytkownika.
% Mają być w tym samym katalogu to ten plik .tex
% ------------------------------
% \usepackage{ODE} % Pakiet napisany między innymi dla tego pliku.
\usepackage{latexgeneralcommands}
\usepackage{mathcommands}




% ---------------------------------------------------------------------
% Dodatkowe ustawienia dla języka polskiego
% ---------------------------------------------------------------------
\renewcommand{\thesection}{\arabic{section}.}
% Kropki po numerach rozdziału (polski zwyczaj topograficzny)
\renewcommand{\thesubsection}{\thesection\arabic{subsection}}
% Brak kropki po numerach podrozdziału



% ------------------------------
% Ustawienia różnych parametrów tekstu
% ------------------------------
\renewcommand{\baselinestretch}{1.1}

% Ustawienie szerokości odstępów między wierszami w tabelach.
\renewcommand{\arraystretch}{1.4}



% ------------------------------
% Pakiet "hyperref"
% Polecano by umieszczać go na końcu preambuły.
% ------------------------------
\usepackage{hyperref} % Pozwala tworzyć hiperlinki i zamienia odwołania
% do bibliografii na hiperlinki.










% ---------------------------------------------------------------------
% Tytuł, autor, data
\title{Równania różniczkowe zwyczajne \\
  {\Large Błędy i~uwagi}}

\author{Kamil Ziemian}


% \date{}
% ---------------------------------------------------------------------










% ####################################################################
\begin{document}
% ####################################################################





% ######################################
\maketitle % Tytuł całego tekstu
% ######################################





% ##############################
\Work{ % Autor i tytuł dzieła
  Władimir Igoriewicz Arnold \\
  \textit{Równania różniczkowe zwyczajne},
  \cite{ArnoldRownaniaRozniczkoweZwyczajne1975}}


% ##################
\newpage

\CenterBoldFont{Błędy}


\begin{center}

  \begin{tabular}{|c|c|c|c|c|}
    \hline
    Strona & \multicolumn{2}{c|}{Wiersz} & Jest
                              & Powinno być \\ \cline{2-3}
    & Od góry & Od dołu & & \\
    \hline
    5   & &  7 & 1968 - 196 & 1968 - 1969 \\
    11  & 17 & & mechanice klasycznej & mechanice kwantowej \\
    15  & & 16 & rozdziale 6 & rozdziale 5 \\
    28  & 15 & & wzór (8) & wzór \\
    34  &  9 & & $\dot{ x }_{ 1 } = x_{ 2 }$ & $\dot{ x }_{ 1 } = x_{ 1 }$ \\
    47  & & 12 & $x_{ i } = \varphi_{ i }( x_{ 1 }, \ldots, x_{ n } )$
           & $x_{ i } = \varphi_{ i }( y_{ 1 }, \ldots, y_{ n } )$ \\
    53  & & 14 & obrót & obrót krzywych całkowych \\
    56  & 12 & & osobliwym & nieosobliwym \\
    61  &  7 & & $\vecxbold$???, $\vecalphabold_{ 0 }$
           & $\vecxbold$, $\vecalphabold$ \\
    64  & & 11 & ????$\vecgbold( t_{ 2 }, t_{ 1 }, \vecxbold )
                = \vecgbold^{ t_{ 2 } }_{ t_{ 1 } }( \vecxbold, t_{ 1 } )$
           & $\vecxbold^{ t_{ 2 } }_{ t_{ 1 } }( \vecxbold, t_{ 1 } )
             = \vecgbold( t_{ 2 }, t_{ 1 }, \vecxbold )$ \\
    64  & & 10 & $( \vecvarphibold( t ), t )$
           & $( t, \vecvarphibold( t ) )$ \\[0.3em]
    66  & 17 & & $\vecvbold( t, \vecxbold, \dot{ \vecalphabold } )$
           & $\vecvbold( t, \vecxbold, \vecalphabold )$??? \\[0.3em]
    70  & & 13 & $\dot{ p }_{ i } = \frac{ \partial H }{ \partial q_{ i } }$
           & $\dot{ p }_{ i } = -\frac{ \partial H }{ \partial q_{ i } }$ \\[0.3em]
    71  & & 15 & $\frac{ \partial \vecvbold_{ 0 } }{ \vecxbold }$
           & $\frac{ \partial \vecvbold_{ 0 } }{ \partial \vecxbold }$ \\[0.4em]
    72  &  4 & & „niezaburzonego” & „zaburzonego” \\
    73  &  5 & & $\vecxbold( 0$ & $\vecxbold( 0 )$ \\
    90  & &  3 & \textit{wraz z pochodną dla} $x = 0$
           & \textit{dla} $x = 0$ \\
    92  &  3 & & $U( x( O ) )$ & $U( x( 0 ) )$ \\[0.3em]
    123 &  6 & & $^{ \Rbb }A : \Cbb^{ m } \to { }^{ \Rbb } \Cbb^{ n }$
           & ${ }^{ \Rbb }A : { }^{ \Rbb } \Cbb^{ m }
             \to { }^{ \Rbb } \Cbb^{ n }$ \\
    125 &  5 & & $\mathbf{I}$ & $I$ \\
    % & & & & \\
    % & & & & \\
    % & & & & \\
    % & & & & \\
    \hline
  \end{tabular}

\end{center}

\vspace{\VerSpaceSix}


\noindent
\StrWierszD{66}{7} \\
\Jest  \textit{tyłu do~brzegu} \\
\Powin \textit{tyłu nieograniczenie albo~do~brzegu} \\
\StrWierszG{110}{9} \\
\Jest  sumą częściową szeregu --~iloczynu \\
\Powin jest sumą części wyrazów iloczynu \\



% ############################










% ############################
\Work{ % Autor i tytuł dzieła
  N. M. Matwiejew \\
  \textit{Metody całkowania równana różniczkowych zwyczajnych},
  \cite{MatwiejewMetodyCalkowaniaRownanRozniczkowychZwyczajnych1982}}

\vspace{0em}


% ##################
\CenterBoldFont{Błędy}


\begin{center}

  \begin{tabular}{|c|c|c|c|c|}
    \hline
    Strona & \multicolumn{2}{c|}{Wiersz} & Jest
                              & Powinno być \\ \cline{2-3}
    & Od góry & Od dołu & & \\
    \hline
    5   & &  9 & Dodzimy & Dowodzimy \\
    5   & &  8 & potkowych & początkowych \\
    5   & &  7 & poąątkowych & początkowych \\
    10  & & 19 & damy & mamy \\
    % 15  & & & & \\
    % & & & & \\
    % & & & & \\
    \hline
  \end{tabular}

\end{center}

\vspace{\VerSpaceSix}


\noindent
\StrWierszG{15}{13}
\Jest  w~sensie ustępu \\
\Powin w~sensie zdefiniowanym w~ustępie \\
\StrWierszD{20}{2} \\
\Jest  i~nie ma rozwiązania określonego w~tym samym przedziale
nie~identycznego z~rozwiązaniem $y = y( x )$ chociażby w~jednym
punkcie przedziału $\absOne{ x - x_{ 0 } } \leq h$ różnym
od~punktu $x = x_{ 0 }$. \\
\Powin i~nie istnieje inne rozwiązanie określone w~przedziale
$\absOne{ x - x_{ 0 } } \leq h_{ 1 } \leq h$ które nie byłoby równe
rozwiązaniu $y = y( x )$ w~każdym punkcie przedziału
$\absOne{ x - x_{ 0 } } \leq h_{ 1 }$. \\


% ############################










% ############################
\newpage

\Work{ % Autor i tytuł dzieła
  N. M. Matwiejew \\
  \textit{Metody całkowania równana różniczkowych zwyczajnych},
  \cite{MatwiejewMetodyCalkowaniaRownanRozniczkowychZwyczajnych1986}}

\vspace{0em}


% ##################
\CenterBoldFont{Uwagi}

\vspace{0em}

\noindent
Książka ta podąża za zwyczajem, że~jeśli argumenty danej funkcji są
odpowiednio oczywiste, to można je opuścić. Przykładowo na stronie~7
mamy równanie (1) postaci:
\begin{equation}
  \label{eq:Matwiejew-Metody-calkowania-ETC-01}
  F\left( x, y, y', y'', \ldots, y^{ ( n ) } \right) = 0,
\end{equation}
choć bardziej precyzyjne byłoby napisanie $y( x )$, $y'( x )$, etc. W~tych
notatkach, by uniknąć nieporozumień, we~wszystkich miejscach gdzie mowa jest
o~wartościach jakie przyjmują funkcje dla zadanych wartości argumentów,
będziemy~się starać podawać owe argumenty w~sposób jawny.

Ty samym, jeśli będziemy rozważali w~sposób ogólny funkcje
$y : \Rbb \to \Rbb$ to będziemy ją oznaczać symbolem $y$. Natomiast
w~przypadkach takich jak równanie
\begin{equation}
  \label{eq:Matwiejew-Metody-calkowania-ETC-02}
  y( x )^{ 2 } = x,
\end{equation}
będziemy~się starali zawsze wymienić wszystkie argumenty w~sposób jawny.

\vspace{\VerSpaceFour}










% ##################
\CenterBoldFont{Uwagi do~konkretnych stron}

\vspace{0em}


\noindent
\Str{7} W~tym miejscu warto~się zastanowić nad tym, jak w~ścisły sposób
zdefiniować równanie różniczkowe zwyczajne? Jak każdy taki ważny
problem w~matematyce, podaną definicję trzeba będzie niewątpliwie rozszerzać
i~modyfikować, tak aby objąć jej nowymi wersjami kolejne ważne problemy
matematyczne\footnote{Przez „problem matematyczny” rozumiemy tu zagadnienie
  matematyczne, które jawnie wymaga od nas znalezienia jego rozwiązania.
  Przykładowo, problemem matematyczny jest zagadnienie wyznaczenia
  funkcji $y( x )$, takiej że~jej pochodna w~punkcie $x$ jest równa
  $\sin\!\big( y( x ) \big)$. }. Niezależnie jednak od tego, warto spróbować
podać teraz ścisłą definicję które będzie obejmowała większość najbardziej
podstawowych problemów matematycznych jakie napotykamy w~teorii równań
różniczkowych zwyczajnych, nawet jeśli wiele innych będzie wykluczała.

Na tej stronie możemy znaleźć informacje, że~jeśli nie jest powiedziane
inaczej, to zakładamy, że~zarówno dziedzina jak i~przeciwdziedzina funkcji
$y$ są podzbiorami liczb rzeczywistych. Jeśli chodzi o~przeciwdziedzinę
to możemy przyjąć, że~zawsze jest ona równa $\Rbb$, bo wybór ten nie
powinien nigdzie grać roli. Jeśli chodzi o~dziedzinę danej funkcji, to
sprawa jest bardziej złożona.

Niech $A$ oznacza dziedzinę funkcji $y$. W~niniejszej książce naszym
podstawowym wymogiem jest to, by w~każdym punkcie dziedziny $A$ istniała
pochodna funkcji $y$, więc $A$ musi być dobrany w~taki sposób, by pojęcie
pochodnej w~punkcie $x \in A$ miało sens. Wbrew pozorom podanie klasy zbiorów
o~tej własności nie jest takie proste. Czy przykładowo funkcja
$f : \Qbb \to \Qbb$ dana zależnością
\begin{equation}
  \label{eq:Matwiejew-Metody-calkowania-ETC-03}
  f( x ) = x^{ 2 },
\end{equation}
jest różniczkowalna?

Pomimo problemów z~określeniem klasy tych zbiorów, w~większości przypadków
przyjmuje~się, że~zbiór $A$ jest odcinkiem otwartym $( a, b )$, gdzie
dopuszczamy sytuację, że $a$ i~$b$ mogą przyjmować wartość $\pm \infty$. Jak
zauważono na stronie ???, nie nastręcza również problemu określenie
funkcji~$y$ na odcinkach domkniętych z~jednej, bądź obu stron. W~przypadku
odcinka $[ a, b )$ przez pochodną w~punkcie~$a$ rozumiemy pochodną
prawostronną funkcji $y$ i~analogicznie postępuję w~przypadku odcinka
$( a, b ]$. Przy czym jeśli dany jest odcinek prawostronnie domknięty
$( a, b ]$ to musi zachodzić $-\infty < b < +\infty$, natomiast $a$ może przyjąć
wartość $-\infty$. Analogicznie zasady stosują~się do pozostały pozostał
przypadków odcinków jednostronnie, bądź obustronnie domkniętych. Do problemu
kształtu dziedziny powrócimy niedługo.

Niech teraz $O$ będzie otwartym podzbiorem $\Rbb^{ n + 2 }$, przy czym
elementy $\Rbb^{ n + 2 }$ będziemy oznaczać przez
$( x_{ 0 }, x_{ 1 }, \ldots, x_{ n - 1 }, x_{ n }, x_{ n + 1 } )$
lub $x_{ 1 }, x_{ 2 }, \ldots, x_{ n + 1 }, x_{ n + 2 }$. Niech dana będzie funkcja
$F : O \to R$. Przyjmujemy, że zależy ona w~sposób \textbf{istotny}
(w~książce używa~się terminu „jawny”) od zmiennej $x_{ n }$, przez co
rozumiemy następującą własność. Istnieją takie liczby
$x_{ 0 }, x_{ 1 }, \ldots, x_{ n }, x_{ n + 1 }, x_{ n + 1 }'$, że~zachodzi
\begin{equation}
  \label{eq:Matwiejew-Metody-calkowania-ETC-04}
  F( x_{ 0 }, x_{ 1 }, \ldots, x_{ n }, x_{ n + 1 } ) \neq
  F( x_{ 0 }, x_{ 1 }, \ldots, x_{ n }, x_{ n + 1 }' ).
\end{equation}
Niech teraz $A_{ 0 }$ będzie rzutem zbioru $O$ na oś $x_{ 0 }$, czyli zbiorem
takim że~jeśli $x \in A_{ 0 }$ to istnieją takie liczby
$x_{ 1 }, x_{ 2 }, \ldots, x_{ n }, x_{ n + 1 }$,
że~$( x, x_{ 1 }, x_{ 2 }, \ldots, x_{ n }, x_{ n + 1 } ) \in O$. Zbiór ten
będziemy również oznaczać przez $\proj_{ 0 } O$.

\textbf{Równaniem różniczkowym zwyczajnym rzędu~$n$} nazywamy wyrażenie
\begin{equation}
  \label{eq:Matwiejew-Metody-calkowania-ETC-05}
  F\left( x, y( x ), y'( x ), y''( x ), \ldots, y^{ ( n ) }( x ) \right) = 0.
\end{equation}
\textbf{Rozwiązaniem równania \eqref{eq:Matwiejew-Metody-calkowania-ETC-02}}
nazywamy funkcję $y_{ 1 } : A_{ 1 } \to \Rbb$, gdzie $A_{ 1 } \subset A_{ 0 }$, która
posiada pochodne do rzędu $n$ włącznie w~każdym punkcie swojej dziedziny
i~dla której zachodzi
\begin{equation}
  \label{eq:Matwiejew-Metody-calkowania-ETC-06}
  F\left( x, y_{ 1 }( x ), y_{ 1 }'( x ), \ldots, y_{ 1 }^{ ( n ) }( x ) \right) =
  0, \qquad
  \forall x \in A_{ 1 }.
\end{equation}

Potrzebujemy teraz powrócić do problemu określenia dziedziny $A_{ 1 }$
funkcji $y$. Jak powiedzieliśmy wcześniej, standardowo przyjmuje~się,
że~$A_{ 1 } = ( a, b )$. Zazwyczaj żąda~się dodatkowo, by $A_{ 1 }$ był
największym odcinkiem otwarty zawartym w~zbiorze $A_{ 0 }$, w~którym jesteśmy
w~stanie zdefiniować rozwiązanie równania
\eqref{eq:Matwiejew-Metody-calkowania-ETC-05}. Co jednak zrobić w~przypadku
równań takich jak równanie $y'( x ) = y( x )^{ 2 }$ omawiane na stronie~15,
które posiada rozwiązanie postaci
\begin{equation}
  \label{eq:Matwiejew-Metody-calkowania-ETC-07}
  y( x ) = \frac{ 1 }{ 1 - x }.
\end{equation}
Funkcja ta jest określona na zbiorze $( -\infty, 1 ) \cup ( 1, +\infty )$ i~nie możemy
w~żaden sposób wybrać jednego z~tych przedziałów jako większego od drugiego.

Problem rozwiązań takich jak \eqref{eq:Matwiejew-Metody-calkowania-ETC-06}
będzie omawiany w~dalszym ciągu książki, wtedy ewentualnie powrócimy do
pojawiające~się przy nich dokładniej. W~chwili obecnej, motywowani tym
przykładem, poprzestaniemy na stwierdzeniu, że~jeśli to jest możliwe to
będziemy szukali rozwiązań określonych na całym zbiorze $A_{ 0 }$, co jest
równoważne stwierdzeniu $A_{ 1 } = A_{ 0 }$, co~wyklucza wszelkie
niejednoznaczności w~sposobie określenia dziedziny rozwiązania.
Niewątpliwie, ze względu na swoją prostotę interpretacyjną, optymalną
sytuacją jest $A_{ 1 } = A_{ 0 } = \Rbb$.

\vspace{\VerSpaceFour}





\noindent
\Str{8} Na tej stronie napotykamy po raz pierwszy konkretne równanie
różniczkowe
\begin{equation}
  \label{eq:Matwiejew-Metody-calkowania-ETC-08}
  y'( x ) - 2 x = 0.
\end{equation}
Z~równania tego w~prosty sposób odczytujemy funkcję $F$:
\begin{equation}
  \label{eq:Matwiejew-Metody-calkowania-ETC-09}
  F( x_{ 0 }, x_{ 1 }, x_{ 2 } ) = x_{ 2 } - 2 x_{ 0 }.
\end{equation}
Stajemy tu jednak przed problemem, jaka jest dziedzina funkcji~$F$?
W~dalszym ciągu jeśli nie powiedziano inaczej, za dziedzinę funkcji $F$
definiującej równanie różniczkowe będziemy uważać największy zbiór na~którym
jesteśmy w~stanie ją określić. Ponieważ zgodnie z~tym co powiedziano
wcześniej, dziedzina funkcji $F$ jest podzbiorem $\Rbb^{ n + 2 }$, w~naszym
wypadku $n = 1$, więc dla funkcji danej przez
\eqref{eq:Matwiejew-Metody-calkowania-ETC-07} jest ona równa $\Rbb^{ 3 }$.
Łatwo~się też przekonać, że~rozwiązaniem równania
\eqref{eq:Matwiejew-Metody-calkowania-ETC-08} jest funkcja
$y( x ) = x^{ 2 } + C$, której dziedzina to $\Rbb$.

Jednocześnie równanie \eqref{eq:Matwiejew-Metody-calkowania-ETC-08} zapisane
jako
\begin{equation}
  \label{eq:Matwiejew-Metody-calkowania-ETC-10}
  y'( x ) = 2 x,
\end{equation}
jest przykładem równania pierwszego rzędu w~postaci normalnej (zob. str.~13
omawianej książki). Ogólna forma równania w~postaci normalnej to
\begin{equation}
  \label{eq:Matwiejew-Metody-calkowania-ETC-11}
  y^{ ( n ) }( x ) = f\big( x, y( x ), y'( x ), \ldots, y^{ ( n - 1 ) }( x ) \big).
\end{equation}
W~sposób elementarny przekształcamy je do równania w~postaci ogólnej
\begin{equation}
  \label{eq:Matwiejew-Metody-calkowania-ETC-12}
  F\big( x, y( x ), y'( x ), \ldots, y^{ ( n ) }( x ) \big) =
  f\big( x, y( x ), y'( x ), \ldots, y^{ ( n - 1 ) }( x ) \big) -
  y^{ ( n - 1 ) }( x ).
\end{equation}
Z~tego powodu, jeśli równanie różniczkowe jest nam dane pierwotnie nie
w~postaci ogólnej, lecz normalnej, to przy analizie problemu dziedziny jego
rozwiązań możemy stosować dokładnie te same zasady, co dla równania
w~postaci ogólnej.

Jeśli chodzi o~ustalenie dziedziny funkcji $f$ stosować będziemy te same
zasady co dla funkcji $F$. Należy jednak pamiętać, że~jeśli równanie jest
pierwotnie wyrażone w~formie ogólnej
\begin{equation}
  \label{eq:Matwiejew-Metody-calkowania-ETC-13}
  F\big( x, y( x ), y'( x ), \ldots, y^{ ( n ) }( x ) \big) = 0,
\end{equation}
to jego postać normalna
\begin{equation}
  \label{eq:Matwiejew-Metody-calkowania-ETC-14}
  y^{ ( n ) }( x ) = f\big( x, y( x ), y'( x ), \ldots, y^{ ( n - 1 ) }( x ) \big),
\end{equation}
może nie obejmować wszystkich przypadków, dla których ma sens równanie
\eqref{eq:Matwiejew-Metody-calkowania-ETC-13}. Ponieważ dziedzina $F$ jest
podzbiorem $\Rbb^{ n + 2 }$, a~$f$ podzbiorem $\Rbb^{ n + 1 }$, wysłowienie
co dokładnie oznacza, że~równanie
\eqref{eq:Matwiejew-Metody-calkowania-ETC-13}
jest ogólniejsze od równania \eqref{eq:Matwiejew-Metody-calkowania-ETC-14}
może być dość zawiłe, zamiast tego odwołamy~się do przykładu.

Rozpatrzmy równanie w~postaci ogólnej
\begin{equation}
  \label{eq:Matwiejew-Metody-calkowania-ETC-15}
  y( x ) \frac{ d y( x ) }{ d x } - x = 0.
\end{equation}
Odczytujemy z~niego funkcję
$F( x_{ 0 }, x_{ 1 }, x_{ 2 } ) = x_{ 2 } x_{ 1 } - x_{ 0 }$, której dziedzina
jest równa $\Rbb^{ 3 }$. Po przekształceniu do postaci normalnej dostajemy
równanie
\begin{equation}
  \label{eq:Matwiejew-Metody-calkowania-ETC-16}
  \frac{ d y( x ) }{ d x } = \frac{ x }{ y( x ) }.
\end{equation}
Funkcja $f( x_{ 0 }, x_{ 1 } ) = x_{ 0 } / x_{ 1 }$ ma dziedzinę
$\Rbb \times ( \Rbb \setminus \{ 0 \} )$. Podstawiając $y( x ) = 0$
do~\eqref{eq:Matwiejew-Metody-calkowania-ETC-13} od razu sprawdzamy,
że~funkcja ta nie jest rozwiązaniem tego równania, podczas gdy podstawienie
tej funkcji do \eqref{eq:Matwiejew-Metody-calkowania-ETC-16} prowadzi do
wyrażenie pozbawionego sensu\footnote{W~książce są podane sposoby radzenia
  sobie z~tego typu problemami, nie jest to jednak ważne w~obecnych
  rozważaniach.}.

\vspace{\VerSpaceFour}





\noindent
\Str{10} Przy okazji wyprowadzania równania różniczkowe dla rodziny
wszystkich okręgów na płaszczyźnie $xy$ napotykamy po raz pierwszy na
pewien problem, który powróci do nas w przyszłości. Naszym punktem wyjścia
jest równania
\begin{equation}
  \label{eq:Matwiejew-Metody-calkowania-ETC-17}
  ( x - a )^{ 2 } + ( y - b )^{ 2 } = R^{ 2 }.
\end{equation}
Różniczkując je dwa razy otrzymujemy dwa następujące równania.

\vspace{\negVerSpaceThree}


\begin{subequations}

  \begin{align}
    \label{eq:Matwiejew-Metody-calkowania-ETC-18-A}
    1 + \big( y'( x ) \big)^{ 2 } + \big( y( x ) - b \big) y''( x )
    &= 0, \\
    \label{eq:Matwiejew-Metody-calkowania-ETC-18-B}
    3 y'( x ) y''( x ) + \big( y( x ) - b \big) y'''( x )
    &= 0.
  \end{align}

\end{subequations}


\noindent
Aby usunąć z~równania \eqref{eq:Matwiejew-Metody-calkowania-ETC-18-B}
parametr $b$ przyjmujemy, że~$y''( x ) \neq 0$, więc możemy przepisać równanie
\eqref{eq:Matwiejew-Metody-calkowania-ETC-18-A} jako
\begin{equation}
  \label{eq:Matwiejew-Metody-calkowania-ETC-19}
  y( x ) - b =
  -\frac{ 1 + \big( y'( x ) \big)^{ 2 } }{ y''( x ) }.
\end{equation}
Podstawiając tą zależność do \eqref{eq:Matwiejew-Metody-calkowania-ETC-16-B}
dostajemy
\begin{equation}
  \label{eq:Matwiejew-Metody-calkowania-ETC-20}
  3 y'( x ) y''( x ) -
  \frac{ 1 + \big( y'( x ) \big)^{ 2 } }{ y''( x ) } y'''( x ) = 0.
\end{equation}
Po pomnożeniu obustronnie przez $y''( x )$ dostajemy
\begin{equation}
  \label{eq:Matwiejew-Metody-calkowania-ETC-21}
  3 y'( x ) \big( y''( x ) \big)^{ 2 } -
  \Big( 1 + \big( y'( x ) \big)^{ 2 } \Big) y'''( x ) = 0.
\end{equation}
Choć równanie wyprowadziliśmy przy założeniu, że~$y''( x ) \neq 0$, to końcowa
jego postać jest dobrze określona również, gdy funkcja ta przyjmuje wartość
zero. Powstaje więc pytania, jaki jest zakres obowiązywania tego równania?

Równanie \eqref{eq:Matwiejew-Metody-calkowania-ETC-18-A} możemy przekształci
przy założeniu $y( x ) - b \neq 0$ do postaci
\begin{equation}
  \label{eq:Matwiejew-Metody-calkowania-ETC-22}
  y''( x ) =
  -\frac{ 1 + \big( y'( x ) \big)^{ 2 } }{ y( x ) - b },
\end{equation}
widzimy więc, że~$y''( x ) \neq 0$ jeśli tylko $y( x ) \neq b$, w~przeciwnym
wypadku jest ona nieokreślona. Możemy stąd wyciągnąć wniosek,
że~na mocy swojego wyprowadzenia równanie
\eqref{eq:Matwiejew-Metody-calkowania-ETC-21} obowiązuje dla wszystkich
wartości $x$ dla których $y( x ) \neq b$. To~zaś prowadzi do~kilku problemów
wartych przedyskutowania.

Po~pierwsze, równanie \eqref{eq:Matwiejew-Metody-calkowania-ETC-21} nie
zawiera stałej~$b$, gdyż naszym celem było wyprowadzenie równania
pozbawionego stałych obecnych w~równaniu
\eqref{eq:Matwiejew-Metody-calkowania-ETC-17}. Biorąc więc za~punkt wyjścia
\eqref{eq:Matwiejew-Metody-calkowania-ETC-20} nie jesteśmy w~stanie
sformułować na jego podstawie warunku $y( x ) \neq b$. Możliwe jednak, że~jeśli
rozwiążemy to równanie, to jego rozwiązania będą nieokreślone jeśli $y( x )$
przyjmie pewną wartość $D$, specyficzną dla danego rozwiązania. Już teraz
możemy stwierdzić, iż~jest to prawdą, przynajmniej dla pewnej klasy
rozwiązań tego równania, które w~postaci uwikłanej (zob. str.~15 omawianej
książki\footnote{Na tej stronie omawiane jest rozwiązanie w~postaci
  uwikłanej dla równania pierwszego rzędu, a~równanie
  \eqref{eq:Matwiejew-Metody-calkowania-ETC-20}. Pomimo tego mamy nadzieję,
  że~to w~jakim sensie równanie
  \eqref{eq:Matwiejew-Metody-calkowania-ETC-17} przedstawia rozwiązanie
  równania \eqref{eq:Matwiejew-Metody-calkowania-ETC-21} jest wystarczająco
  zrozumiałe i~nie wymaga dalszych komentarzy.}) jest dane przez naszą
wyjściową zależność
\eqref{eq:Matwiejew-Metody-calkowania-ETC-17}. Nad analizą tej sytuacji
zatrzymamy~się przez chwilę.

Dla $x \in ( a - R, a + R )$ jesteśmy w~stanie je odwikłać na funkcję
$y( x )$ otrzymując dwie funkcje
\begin{equation}
  \label{eq:Matwiejew-Metody-calkowania-ETC-23}
  y_{ \pm }( x ) = \pm\sqrt{ R^{ 2 } - ( x - a )^{ 2 } } + b.
\end{equation}
Nie jest możliwe odwikłanie tego równania na funkcję $y( x )$ w~sposób
matematycznie spójny na~przedziale $( a + R - \delta, a + R ]$, dla pewnego
$\delta > 0$, przy czym analogiczna sytuacja zachodzi dla przedziału
$[ a - R, a - R + \delta )$. W~tematykę tego, czy w~takim razie nie powinniśmy
również rozważać równania różniczkowego na funkcję $x( y )$ (zob. str.~13
tej książki) nie będziemy~się zagłębiać. Niemniej widzimy, że
\begin{equation}
  \label{eq:Matwiejew-Metody-calkowania-ETC-24}
  \lim_{ x \nearrow a + R } y_{ \pm }( x ) = b,
\end{equation}
więc funkcje te możemy przedłużyć do funkcji ciągłych na przedziale
$( a - R, a + R ]$ (analogicznie możemy postąpić dla punktu $a - R$).
Tak przedłużone funkcje będziemy oznaczać przez $\yTilde_{ \pm }( x )$.
Przyjmują one wartość $b$ tylko dla $x = a + R$, ale~żadna z~nich nie jest
różniczkowalna w~punkcie $x = a + R$, bo~zachowuje~się w~ich otoczeniu jak
$\sqrt{ x }$ w~otoczeniu punktu $x = 0$. Powyższe rozważania pomogą nam
naświetlić drugi ważny problem.

Zauważmy, że~choć równanie \eqref{eq:Matwiejew-Metody-calkowania-ETC-21}
obowiązuje dla wszystkich $x$ taki, że~$y( x ) \neq b$, to nawet jeśli ten
warunek jest nam z~góry znany, to zanim nie rozwiążemy tego równania, nie
wiem dla jakich wartości zmiennej $x$ rozwiązanie przyjmie wartość $b$.
Musimy więc najpierw rozwiązać w~odpowiedni sposób to równanie, a~następnie
usunąć z~dziedziny rozwiązania te punkty, dla których $y( x ) = b$. Jeśli
takie punkty rzeczywiście istnieją to powstaje następujące pytanie. Sokor
otrzymaliśmy rozwiązanie również dla tych wartości $x$ dla
których nie jest spełniony warunek $y( x ) \neq b$, to czy nie oznacza to,
że~choć warunek ten był potrzebny do wyprowadzenia
\eqref{eq:Matwiejew-Metody-calkowania-ETC-21}, to jest on ostatecznie
restrykcyjny i~należy przyjąć bardziej ogólną koncepcję rozwiązania?

W~omawianym przykładzie, funkcje $y_{ \pm }( x )$ były zdefiniowane tylko na
przedziale $( a - R, a + R )$, jednak dało~się je przedłużyć do funkcji
$\yTilde_{ \pm }( x )$ ciągłych na $( a - R, a + R ]$, które są jednak
nieróżniczkowalne w~punkcie $a + R$. Czy w~takim wypadku należy uważać
funkcje $\yTilde_{ \pm }( x )$ za rozwiązanie odpowiedniego równania
różniczkowego? Zwróćmy uwagę, że~rozwiązanie w~postaci uwikłanej
\begin{equation}
  \label{eq:Matwiejew-Metody-calkowania-ETC-25}
  ( x - a )^{ 2 } + ( y - b )^{ 2 } = R^{ 2 },
\end{equation}
przedstawia okrąg, czyli krzywą gładką w~każdym swoim punkcie, leżącą
w~płaszczyźnie $xy$, która nie wykazuje żadnej osobliwości w~punkcie
o~współrzędnych $( a + R, b )$. Z~tego
punktu widzenia funkcja $\yTilde_{ + }( x )$ jest obcięciami rozwiązania
danego funkcją uwikłaną do zbioru $( a - R, a + R ]$, które spełnia warunek
$y \geq 0$, więc należy je uważa za pełnoprawne rozwiązanie równania
\eqref{eq:Matwiejew-Metody-calkowania-ETC-21}. Niewątpliwie, fakt,
że~rozwiązania w~postaci uwikłanej pozwalają nam na objęciem pojęciem
rozwiązania równania różniczkowego takich krzywych jak okręgi dane równaniem
\eqref{eq:Matwiejew-Metody-calkowania-ETC-25}, jest jednym z~powodów, dla
których to pojęcie zostało wprowadzone.

Przedstawiona powyżej analiza pozwala uzmysłowić, jak skomplikowanym
zadaniem może być analiza zakresu obowiązywania danego równania
różniczkowego oraz jego rozwiązań. Do omawiania tych zagadnień zapewne
powrócimy jeszcze wielokrotnie w~tych notatkach.

\vspace{\VerSpaceFour}





\noindent
\Str{15} Równanie (10) na tej stronie zostało bardzo elegancko wyprowadzone,
przy założeniu, że~funkcja uwikłana zależna od zmiennej $x$ dana związkiem
$\Phi( x, y ) = 0$, spełnia równanie różniczkowe
$y'( x ) = f\big( x, y( x ) \big)$. Nie poruszono jednak kwestii tego, czemu
mamy uważać tą funkcję uwikłaną za rozwiązanie badanego równania?
W~szczególności, jeśli jest rozwiązanie równania $\Phi( x, y ) = 0$ na funkcję
$y( x )$, to czy ona spełnia równanie $y'( x ) = f\big( x, y( x ) \big)$?
Przy standardowych założeniach o~funkcji $\Phi( x, y )$ odpowiedź na to
ostatnie pytanie jest twierdząca i~dzięki temu możemy uznać funkcję daną
równaniem uwikłanym za rozwiązanie badanego równania.

Dowód tego faktu jest następujący. Załóżmy, że~funkcja $\Phi( x, y )$ spełnia
założenia o~ciągłości i~istnieniu pochodnych, które są wymagane
w~standardowej wersji twierdzenia o~funkcji uwikłanej. Przyjmijmy dodatkowo,
że~$\Phi_{ y }'\big( x, y \big) \neq 0$ w~interesującym nas zakresie wartości
zmiennej~$x$, który oznaczymy~$A$. Tym samym możemy odwikłać funkcję
$y( x )$ dla wszystkich wartości $x \in A$. Przepiszmy teraz równanie (10)
w~bardziej jawnej formie jako
\begin{equation}
  \label{eq:Matwiejew-Metody-calkowania-ETC-26}
  \Phi_{ x }'\big( x, y( x ) \big) +
  \Phi_{ y }'\big( x, y( x ) \big) f\big( x, y( x ) \big) = 0.
\end{equation}
Równanie to przekształcamy w~oczywisty sposób do postaci
\begin{equation}
  \label{eq:Matwiejew-Metody-calkowania-ETC-27}
  -\frac{ \Phi_{ x }'\big( x, y( x ) \big) }
  { \Phi_{ y }'\big( x, y( x ) \big) } =
  f\big( x, y( x ) \big).
\end{equation}
Jak dobrze wiadomo, lewa strona tej równości przedstawia pochodną funkcji
uwikłanej $y'( x )$.

Na koniec możemy zauważyć, że~zarówno wzór
\eqref{eq:Matwiejew-Metody-calkowania-ETC-26} jak i~dobrze znaną zależność
\begin{equation}
  \label{eq:Matwiejew-Metody-calkowania-ETC-28}
  y'( x ) =
  -\frac{ \Phi_{ x }'\big( x, y( x ) \big) }{ \Phi_{ y }'\big( x, y'( x ) \big) },
\end{equation}
można wyprowadzić w~ten sam sposób, poprzez zróżniczkowanie
i~przekształcenie zależności $\Phi\big( x, y( x ) \big) = 0$. Choć z~tego
punktu widzenia równoważność równania
\eqref{eq:Matwiejew-Metody-calkowania-ETC-26}
i~$y'( x ) = f\big( x, y( x ) \big)$ może~się wydawać oczywista, woleliśmy
przedyskutować to zagadnienie możliwie dokładnie.

\vspace{\VerSpaceFour}





\noindent
\Str{16} Należy przedyskutować dwa zagadnienia dotyczące rozwiązania
równania różniczkowego
\begin{equation}
  \label{eq:Matwiejew-Metody-calkowania-ETC-29}
  y'( x ) = f\big( x, y( x ) \big),
\end{equation}
w~postaci parametrycznej. Po pierwsze, czy jeśli $y( x )$ jest rozwiązaniem
tego równania, to czy można podać jego rozwiązanie w~postaci parametrycznej?
Po drugie, jak uzasadnić, że~funkcje $\varphi( t )$ i~$\psi( t )$ spełniające
równanie
\begin{equation}
  \label{eq:Matwiejew-Metody-calkowania-ETC-30}
  \frac{ \psi'( t ) }{ \varphi'( t ) } =
  f\big( \varphi( t ), \psi( t ) \big),
\end{equation}
przedstawiając rozwiązanie równania
\eqref{eq:Matwiejew-Metody-calkowania-ETC-29}?

Zanim odpowiemy na te pytania, zrobimy małą uwagę odnośnie oznaczeń.
W~całym tym punkcie, jeśli nie powiedziano inaczej, symbol prim oznacza
pochodną danej funkcji. W~szczególności we wzorze
\eqref{eq:Matwiejew-Metody-calkowania-ETC-30}, co przyjmowaliśmy domyślnie,
symbol $\psi'( t )$ oznacza pochodną funkcji $\psi( t )$ po~czasie.

Przejdźmy teraz do odpowiedzi na~pierwsze pytanie z postawionych wyżej
pytań. Równanie różniczkowe \eqref{eq:Matwiejew-Metody-calkowania-ETC-29}
jest określone na płaszczyźnie~$xy$, więc funkcje $\varphi( t )$ i~$\psi( t )$ możemy
rozumieć jako odwzorowanie $\Psi( t ) : ( t_{ 0 }, t_{ 1 } ) \to \Rbb^{ 2 }$, dane
przez
\begin{equation}
  \label{eq:Matwiejew-Metody-calkowania-ETC-31}
  \Psi( t ) = \big( \varphi( t ), \psi( t ) \big).
\end{equation}
Korzystając z~interpretacji równania
\eqref{eq:Matwiejew-Metody-calkowania-ETC-29} za pomocą wprowadzonego dalej
na tej stronie i~następnych pojęcia pola kierunków oraz krzywej całkowej,
można stwierdzić, że~funkcje $\varphi( t )$ i~$\psi( t )$ są rozwiązaniem
rozpatrywanego równania w~postaci parametrycznej, jeśli $\Psi( t )$ wyznacza
krzywą całkową. Jest to bardzo ładny atrakcyjny sposób rozumienia
rozwiązania parametrycznego, dla porządku podamy alternatywną interpretacje.

Mianowicie, jeśli $y( x )$ jest rozwiązaniem równania określonym dla
$x \in ( a, b )$, to parę funkcji $\varphi( t )$, $\psi( t )$ określonych
dla~$t \in I = ( t_{ 0 }, t_{ 1 } )$ nazywamy postacią parametryczną tego
równania jeśli zachodzi $\varphi( I ) \subset ( a, b )$ oraz
\begin{equation}
  \label{eq:Matwiejew-Metody-calkowania-ETC-32}
  \psi( t ) = y\big( \varphi( t ) \big).
\end{equation}

Wzór \eqref{eq:Matwiejew-Metody-calkowania-ETC-32}
pozwala nam stwierdzić, że~jeśli mamy dane rozwiązanie $y( x )$, określone
przy tych samych warunkach co powyżej, to zawsze możemy utworzyć odpowiednie
rozwiązanie parametryczne. Wystarczy przyjąć, że~$I = ( a, b )$,
$\varphi( t ) = t$ (zwykle zapisuje to jako $x = t$) i~zdefiniować $\psi( t )$ jako
$y\big( \varphi( t ) \big)$. Zauważmy, że~zachodzi wówczas
\begin{equation}
  \label{eq:Matwiejew-Metody-calkowania-ETC-33}
  \frac{ d \psi( t ) }{ d t } =
  \frac{ d y( x ) }{ d x }\bigg|_{ x = \varphi( t ) }
  \frac{ d \varphi( t ) }{ d t }.
\end{equation}
Oznaczając symbolem prim pochodną po czasie i~zakładając,
że~$\varphi'( t ) \neq 0$ otrzymujemy zależność
\begin{equation}
  \label{eq:Matwiejew-Metody-calkowania-ETC-34}
  \frac{ d y( x ) }{ d x }\bigg|_{ x = \varphi( t ) } =
  \frac{ \psi'( t ) }{ \varphi'( t ) }.
\end{equation}
Razem z~równaniami \eqref{eq:Matwiejew-Metody-calkowania-ETC-29}
i~\eqref{eq:Matwiejew-Metody-calkowania-ETC-32} prowadzi do równania
\begin{equation}
  \label{eq:Matwiejew-Metody-calkowania-ETC-35}
  \frac{ \psi'( t ) }{ \varphi'( t ) } =
  f\big( \varphi( t ), \psi( t ) \big).
\end{equation}
To kończy rozważanie na temat wyrażenia znanego rozwiązania w~postaci
parametrycznej.

Przejdźmy teraz do problemu, dlaczego możemy uważać rozwiązanie równania
\eqref{eq:Matwiejew-Metody-calkowania-ETC-30} za rozwiązania równania
\eqref{eq:Matwiejew-Metody-calkowania-ETC-29}. Przyjmijmy, że~funkcja
$\varphi( t )$ jest różniczkowalna w~sposób ciągły. Na podstawie równania
\eqref{eq:Matwiejew-Metody-calkowania-ETC-30} wiemy, że~musi zachodzić
$\varphi'( t )$. Na podstawie standardowych twierdzeń z~analizy matematycznej
(zob. przykładowo \cite{FichtenholzRachunekRozniczkowyETCVolI2005}) oraz
dobrze znanego wzoru na pochodną funkcji odwrotnej:
\begin{equation}
  \label{eq:Matwiejew-Metody-calkowania-ETC-36}
  \frac{ d f^{ -1 }( y ) }{ d y } =
  \frac{ d f( t ) }{ d t }\bigg|_{ t = f^{ -1 }( y ) },
\end{equation}
istnieje w~pewnych otoczeniu wybranego punktu $t$ istniej funkcja odwrotna
do $\varphi( t )$, którą będziemy oznaczać $\eta( x )$, która jest różniczkowalna
i~jej pochodna wynosi
\begin{equation}
  \label{eq:Matwiejew-Metody-calkowania-ETC-37}
  \frac{ d \eta( x ) }{ d x } =
  \frac{ 1 }{ \varphi'\big( \eta( x ) \big) }.
\end{equation}
Dla większej przejrzystość obliczeń użyliśmy tutaj symbolu
$\varphi'\big( \eta( x ) \big)$ na oznaczenia wartości pochodnej $\varphi'( t )$ obliczonej
w~punkcie $\eta( x )$.

Rozpatrzmy teraz funkcję $\psi\big( \eta( x ) \big)$. Jej pochodna po zmiennej $x$
jest równa
\begin{equation}
  \label{eq:Matwiejew-Metody-calkowania-ETC-38}
  \frac{ d \psi\big( \eta( x ) \big) }{ d x } =
  \frac{ d \psi( t ) }{ d t }\bigg|_{ t = \eta( x ) }
  \frac{ d \eta( x ) }{ d x } =
  \frac{ d \psi( t ) }{ d t }\bigg|_{ t = \eta( x ) }
  \frac{ 1 }{ \varphi'\big( \eta( x ) \big)}.
\end{equation}
Jeżeli teraz obliczymy wartość obu stron równania
\eqref{eq:Matwiejew-Metody-calkowania-ETC-30} dla wartości $t = \eta( x )$
otrzymamy
\begin{equation}
  \label{eq:Matwiejew-Metody-calkowania-ETC-39}
  \frac{ \psi'\big( \eta( x ) \big) }{ \varphi'\big( \eta( x ) \big) } =
  f\Big( x, \psi\big( \eta( x ) \big) \Big).
\end{equation}
Dzięki równaniu

\vspace{\VerSpaceFour}





\noindent
\Str{31}




















% ##################
\newpage

\CenterBoldFont{Błędy}


\begin{center}

  \begin{tabular}{|c|c|c|c|c|}
    \hline
    Strona & \multicolumn{2}{c|}{Wiersz} & Jest
                              & Powinno być \\ \cline{2-3}
    & Od góry & Od dołu & & \\
    \hline
    5   & &  3 & wierdzenia & twierdzenia \\
    % 10  & & 19 & damy & mamy \\
    15  & & & ono określa & określa ono \\
    % & & & & \\
    % & & & & \\
    \hline
  \end{tabular}

\end{center}

\vspace{\VerSpaceSix}


\noindent
\StrWierszG{15}{13} \\
\Jest  w~sensie ustępu \\
\Powin w~sensie zdefiniowanym w~ustępie \\
% \StrWd{20}{2} \\
% \Jest  i~nie ma rozwiązania określonego w~tym samym przedziale
% nie~identycznego z~rozwiązaniem $y = y( x )$ chociażby w~jednym
% punkcie przedziału $\absOne{ x - x_{ 0 } } \leq h$ różnym
% od~punktu $x = x_{ 0 }$. \\
% \Powin i~nie istnieje inne rozwiązanie określone w~przedziale
% $\absOne{ x - x_{ 0 } } \leq h_{ 1 } \leq h$ które nie byłoby równe
% rozwiązaniu $y = y( x )$ w~każdym punkcie przedziału
% $\absOne{ x - x_{ 0 } } \leq h_{ 1 }$. \\


% ############################







% ####################################################################
% ####################################################################
% Bibliografia

\bibliographystyle{plalpha}

\bibliography{MathematicsBooks}{}





% ############################

% Koniec dokumentu
\end{document}
