% ---------------------------------------------------------------------
% Podstawowe ustawienia i pakiety
% ---------------------------------------------------------------------
\RequirePackage[l2tabu, orthodox]{nag} % Wykrywa przestarzałe i niewłaściwe
% sposoby używania LaTeXa. Więcej jest w l2tabu English version.


\documentclass[a4paper,11pt]{article}
% {rozmiar papieru, rozmiar fontu}[klasa dokumentu]
\usepackage[MeX]{polski} % Polonizacja LaTeXa, bez niej będzie pracował
% w języku angielskim.
\usepackage[utf8]{inputenc} % Włączenie kodowania UTF-8, co daje dostęp
% do polskich znaków.
\usepackage[T1]{fontenc} % Potrzebne do używania fontów Latin Modern.
\usepackage{lmodern} % Wprowadza fonty Latin Modern.



% ----------------------------
% Podstawowe pakiety (niezwiązane z ustawieniami języka)
% ----------------------------
\usepackage{microtype} % Twierdzi, że poprawi rozmiar odstępów w tekście.
\usepackage{graphicx} % Wprowadza bardzo potrzebne komendy do wstawiania
% grafiki.
\usepackage{verbatim} % Poprawia otoczenie VERBATIME.
\usepackage{textcomp} % Dodaje takie symbole jak stopnie Celsiusa,
% wprowadzane bezpośrednio w tekście.
\usepackage{vmargin} % Pozwala na prostą kontrolę rozmiaru marginesów,
% za pomocą komend poniżej. Rozmiar odstępów jest mierzony w calach.
% ----------------------------
% MARGINS
% ----------------------------
\setmarginsrb
{ 0.7in}  % left margin
{ 0.6in}  % top margin
{ 0.7in}  % right margin
{ 0.8in}  % bottom margin
{  20pt}  % head height
{0.25in}  % head sep
{   9pt}  % foot height
{ 0.3in}  % foot sep



% ------------------------------
% Często przydatne pakiety
% ------------------------------
% \usepackage{csquotes} % Pozwala w prosty sposób wstawiać cytaty do tekstu.
\usepackage{xcolor} % Pozwala używać kolorowych czcionek (zapewne dużo
% więcej, ale ja nie potrafię nic o tym powiedzieć).



% ------------------------------
% Pakiety do tekstów z nauk przyrodniczych
% ------------------------------
\let\lll\undefined % Amsmath gryzie się z pakietami do języka
% polskiego, bo oba definiują komendę \lll. Aby rozwiązać ten problem
% oddefiniowuję tę komendę, ale może tym samym pozbywam się dużego Ł.
\usepackage[intlimits]{amsmath} % Podstawowe wsparcie od American
% Mathematical Society (w skrócie AMS)
\usepackage{amsfonts, amssymb, amscd, amsthm} % Dalsze wsparcie od AMS
\usepackage{bm}  % Daję komendę \bm do pogrubionej czcionki matematycznej
% \usepackage{siunitx} % Do prostszego pisania jednostek fizycznych
% \usepackage{upgreek} % Ładniejsze greckie litery
% Przykładowa składnia: pi = \uppi
\usepackage{slashed} % Pozwala w prosty sposób pisać slash Feynmana.
\usepackage{calrsfs} % Zmienia czcionkę kaligraficzną w \mathcal
% na ładniejszą. Może w innych miejscach robi to samo, ale o tym nic
% nie wiem.



% ------------------------------
% Tworzenie środowisk (?) „Twierdzenie”, „Definicja”, „Lemat”, etc.
% ------------------------------
% Komenda wprowadzająca otoczenie „theorem” do pisania twierdzeń
% matematycznych.
\newtheorem{theorem}{Twierdzenie}
% Analogicznie jak powyżej
\newtheorem{definition}{Definicja}
\newtheorem{corollary}{Wniosek}



% ------------------------------
% Pakiety napisane przez użytkownika.
% Mają być w tym samym katalogu to ten plik .tex
% ------------------------------
% \usepackage{ODE} % Pakiet napisany między innymi dla tego pliku.
\usepackage{latexgeneralcommands}
\usepackage{mathcommands}




% --------------------------------------------------------------------
% Dodatkowe ustawienia dla języka polskiego
% --------------------------------------------------------------------
\renewcommand{\thesection}{\arabic{section}.}
% Kropki po numerach rozdziału (polski zwyczaj topograficzny)
\renewcommand{\thesubsection}{\thesection\arabic{subsection}}
% Brak kropki po numerach podrozdziału



% ------------------------------
% Ustawienia różnych parametrów tekstu
% ------------------------------
\renewcommand{\baselinestretch}{1.1}

% Ustawienie szerokości odstępów między wierszami w tabelach.
\renewcommand{\arraystretch}{1.4}



% ------------------------------
% Pakiet "hyperref"
% Polecano by umieszczać go na końcu preambuły.
% ------------------------------
\usepackage{hyperref} % Pozwala tworzyć hiperlinki i zamienia odwołania
% do bibliografii na hiperlinki.










% ---------------------------------------------------------------------
% Tytuł, autor, data
\title{Równania różniczkowe zwyczajne \\
  {\Large Błędy i~uwagi}}

\author{Kamil Ziemian}


% \date{}
% ---------------------------------------------------------------------










% ####################################################################
\begin{document}
% ####################################################################





% ######################################
\maketitle % Tytuł całego tekstu
% ######################################





% ##############################
\Work{ % Autor i tytuł dzieła
  Władimir Igoriewicz Arnold \\
  \textit{Równania różniczkowe zwyczajne},
  \cite{ArnoldRownaniaRozniczkoweZwyczajne1975}}


% ##################
\newpage

\CenterBoldFont{Błędy}


\begin{center}

  \begin{tabular}{|c|c|c|c|c|}
    \hline
    Strona & \multicolumn{2}{c|}{Wiersz} & Jest
                              & Powinno być \\ \cline{2-3}
    & Od góry & Od dołu & & \\
    \hline
    5   & &  7 & 1968 - 196 & 1968 - 1969 \\
    11  & 17 & & mechanice klasycznej & mechanice kwantowej \\
    15  & & 16 & rozdziale 6 & rozdziale 5 \\
    28  & 15 & & wzór (8) & wzór \\
    34  &  9 & & $\dot{ x }_{ 1 } = x_{ 2 }$ & $\dot{ x }_{ 1 } = x_{ 1 }$ \\
    47  & & 12 & $x_{ i } = \varphi_{ i }( x_{ 1 }, \ldots, x_{ n } )$
           & $x_{ i } = \varphi_{ i }( y_{ 1 }, \ldots, y_{ n } )$ \\
    53  & & 14 & obrót & obrót krzywych całkowych \\
    56  & 12 & & osobliwym & nieosobliwym \\
    61  &  7 & & $\vecxbold$???, $\vecalphabold_{ 0 }$
           & $\vecxbold$, $\vecalphabold$ \\
    64  & & 11 & ????$\vecgbold( t_{ 2 }, t_{ 1 }, \vecxbold )
                = \vecgbold^{ t_{ 2 } }_{ t_{ 1 } }( \vecxbold, t_{ 1 } )$
           & $\vecxbold^{ t_{ 2 } }_{ t_{ 1 } }( \vecxbold, t_{ 1 } )
             = \vecgbold( t_{ 2 }, t_{ 1 }, \vecxbold )$ \\
    64  & & 10 & $( \vecvarphibold( t ), t )$
           & $( t, \vecvarphibold( t ) )$ \\[0.3em]
    66  & 17 & & $\vecvbold( t, \vecxbold, \dot{ \vecalphabold } )$
           & $\vecvbold( t, \vecxbold, \vecalphabold )$??? \\[0.3em]
    70  & & 13 & $\dot{ p }_{ i } = \frac{ \partial H }{ \partial q_{ i } }$
           & $\dot{ p }_{ i } = -\frac{ \partial H }{ \partial q_{ i } }$ \\[0.3em]
    71  & & 15 & $\frac{ \partial \vecvbold_{ 0 } }{ \vecxbold }$
           & $\frac{ \partial \vecvbold_{ 0 } }{ \partial \vecxbold }$ \\[0.4em]
    72  &  4 & & „niezaburzonego” & „zaburzonego” \\
    73  &  5 & & $\vecxbold( 0$ & $\vecxbold( 0 )$ \\
    90  & &  3 & \textit{wraz z pochodną dla} $x = 0$
           & \textit{dla} $x = 0$ \\
    92  &  3 & & $U( x( O ) )$ & $U( x( 0 ) )$ \\[0.3em]
    123 &  6 & & $^{ \Rbb }A : \Cbb^{ m } \to { }^{ \Rbb } \Cbb^{ n }$
           & ${ }^{ \Rbb }A : { }^{ \Rbb } \Cbb^{ m }
             \to { }^{ \Rbb } \Cbb^{ n }$ \\
    125 &  5 & & $\mathbf{I}$ & $I$ \\
    % & & & & \\
    % & & & & \\
    % & & & & \\
    % & & & & \\
    \hline
  \end{tabular}

\end{center}

\vspace{\VerSpaceSix}


\noindent
\StrWd{66}{7} \\
\Jest  \textit{tyłu do~brzegu} \\
\Powin \textit{tyłu nieograniczenie albo~do~brzegu} \\
\StrWg{110}{9} \\
\Jest  sumą częściową szeregu --~iloczynu \\
\Powin jest sumą części wyrazów iloczynu \\



% ############################










% ############################
\Work{ % Autor i tytuł dzieła
  N. M. Matwiejew \\
  \textit{Metody całkowania równana różniczkowych zwyczajnych},
  \cite{MatwiejewMetodyCalkowaniaRownanRozniczkowychZwyczajnych1982}}

\vspace{0em}


% ##################
\CenterBoldFont{Uwagi}

\vspace{0em}


\noindent
\Str{15} Równanie (10) zostało bardzo elegancko wyprowadzone, przy
założeniu, że~funkcja uwikłana dana równaniem (9) spełnia
równanie różniczkowe (1), nie odpowiada to jednak na pytanie czy jeśli
funkcja $y( x )$ spełnia równanie (10), to spełnia też interesujące
nas równanie wyjściowe. Dowód tego twierdzenia jest następujący.
Załóżmy, że $\Phi_{ y }' \neq 0$, tak by było zapewnione istnienie
funkcji uwikłanej. Jeżeli teraz spełnione jest równanie (10), to można
je przekształcić do postaci:
\begin{equation}
  \label{MatwiejewMCRRZ-01}
  -\frac{ \Phi_{ x }' }{ \Phi_{ y }' } = f( x, y ).
\end{equation}
Lewa strona tej równości jest równa pochodnej funkcji uwikłanej
$y( x )$, określonej wzorem (9).

\vspace{\VerSpaceThree}





\noindent
\Str{16} \Dok \\

\vspace{\VerSpaceThree}





\noindent
\Str{31} \Dok




% ##################
\CenterBoldFont{Błędy}


\begin{center}

  \begin{tabular}{|c|c|c|c|c|}
    \hline
    Strona & \multicolumn{2}{c|}{Wiersz} & Jest
                              & Powinno być \\ \cline{2-3}
    & Od góry & Od dołu & & \\
    \hline
    5   & &  9 & Dodzimy & Dowodzimy \\
    5   & &  8 & potkowych & początkowych \\
    5   & &  7 & poąątkowych & początkowych \\
    10  & & 19 & damy & mamy \\
    15  & & & & \\
    % & & & & \\
    % & & & & \\
    \hline
  \end{tabular}

\end{center}

\vspace{\VerSpaceSix}


\noindent
\StrWg{15}{13}
\Jest  w~sensie ustępu \\
\Powin w~sensie zdefiniowanym w~ustępie \\
\StrWd{20}{2} \\
\Jest  i~nie ma rozwiązania określonego w~tym samym przedziale
nie~identycznego z~rozwiązaniem $y = y( x )$ chociażby w~jednym
punkcie przedziału $\absOne{ x - x_{ 0 } } \leq h$ różnym
od~punktu $x = x_{ 0 }$. \\
\Powin i~nie istnieje inne rozwiązanie określone w~przedziale
$\absOne{ x - x_{ 0 } } \leq h_{ 1 } \leq h$ które nie byłoby równe
rozwiązaniu $y = y( x )$ w~każdym punkcie przedziału
$\absOne{ x - x_{ 0 } } \leq h_{ 1 }$. \\


% ############################










% ############################
\newpage

\Work{ % Autor i tytuł dzieła
  N. M. Matwiejew \\
  \textit{Metody całkowania równana różniczkowych zwyczajnych},
  \cite{MatwiejewMetodyCalkowaniaRownanRozniczkowychZwyczajnych1986}}

\vspace{0em}


% ##################
\CenterBoldFont{Uwagi do~konkretnych stron}

\vspace{0em}


\noindent
\Str{7} W~tym miejscu warto~się zastanowić nad tym, jak w~ścisły sposób
zdefiniować równanie różniczkowe zwyczajne? Jak każdy taki ważny
problem w~matematyce, podaną definicję trzeba będzie niewątpliwie rozszerzać
i~modyfikować, tak aby objąć jej nowymi wersjami kolejne ważne problemy
matematyczne\footnote{Przez „problem matematyczny” rozumiemy tu zagadnienie
  matematyczne, które jawnie wymaga od nas znalezienia jego rozwiązania.
  Przykładowo, problemem matematyczny jest zagadnienie wyznaczenia
  funkcji $y( x )$, takiej że~jej pochodna w~punkcie $x$ jest równa
  $\sin\!\big( y( x ) \big)$. }. Niezależnie jednak od tego, warto spróbować
podać teraz ścisłą definicję które będzie obejmowała większość najbardziej
podstawowych problemów matematycznych jakie napotykamy w~teorii równań
różniczkowych zwyczajnych, nawet jeśli wiele innych będzie wykluczała.

Na tej stronie możemy znaleźć informacje, że~jeśli nie jest powiedziane
inaczej, to zakładamy, że~zarówno dziedzina jak i~przeciwdziedzina funkcji
$y( x )$ są podzbiorami liczb rzeczywistych. Jeśli chodzi o~przeciwdziedzinę
to możemy przyjąć, że~zawsze jest ona równa $\Rbb$, bo wybór ten nie
powinien nigdzie grać roli. Jeśli chodzi o~dziedzinę, do sprawy tej
powrócimy. ????

Niech teraz $O$ będzie otwartym podzbiorem $\Rbb^{ n + 1 }$, przy czym
elementy $\Rbb^{ n }$ będziemy oznaczać przez
$( x_{ 0 }, x_{ 1 }, \ldots, x_{ n - 1 }, x_{ n } )$
lub $x_{ 1 }, x_{ 2 }, \ldots, x_{ n }, x_{ n + 1 }$. Niech dana będzie funkcja
$F : O \to R$. Przyjmujemy, że zależy ona w~sposób \textbf{istotny}
(w~książce używa~się terminu „jawny”) od zmiennej $x_{ n }$, przez co
rozumiemy następującą własność. Istnieją takie liczby
$x_{ 0 }, x_{ 1 }, \ldots, x_{ n - 1 }, x_{ n }, x_{ n }'$, że~zachodzi
\begin{equation}
  \label{eq:Matwiejew-Metody-calkowania-ETC-01}
  F( x_{ 0 }, x_{ 1 }, \ldots, x_{ n - 1 }, x_{ n } ) \neq
  F( x_{ 0 }, x_{ 1 }, \ldots, x_{ n - 1 }, x_{ n }' ).
\end{equation}
Niech teraz $A_{ 0 }$ będzie rzutem zbioru $O$ na oś $x_{ 0 }$, czyli zbiorem
takim że~jeśli $x \in A_{ 0 }$ to istnieją takie liczby
$x_{ 1 }, x_{ 2 }, \ldots, x_{ n - 1 }, x_{ n }$,
że~$( x, x_{ 1 }, x_{ 2 }, \ldots, x_{ n - 1 }, x_{ n } ) \in O$. Zbiór ten będziemy
oznaczać przez $\proj_{ 0 } O$.

\textbf{Równaniem
  różniczkowym zwyczajnym} nazywamy wyrażenie
\begin{equation}
  \label{eq:Matwiejew-Metody-calkowania-ETC-02}
  F\left( x, y( x ), y'( x ), y''( x ), \ldots, y^{ ( n ) }( x ) \right) = 0.
\end{equation}
\textbf{Rozwiązaniem równania \eqref{eq:Matwiejew-Metody-calkowania-ETC-02}}
nazywamy funkcję $y_{ 1 } : A_{ 1 } \to \Rbb$, gdzie $A_{ 1 } \subset A_{ 0 }$, która
posiada pochodne w~do rzędu $n$ włącznie w~każdym punkcie swojej dziedziny
i~dla której zachodzi
\begin{equation}
  \label{eq:Matwiejew-Metody-calkowania-ETC-03}
  F\left( x, y_{ 1 }( x ), y_{ 1 }'( x ), \ldots, y_{ 1 }^{ ( n ) }( x ) \right) =
  0, \qquad
  \forall x \in A_{ 1 }.
\end{equation}

\vspace{\VerSpaceThree}





\noindent
\Str{8} Na tej stronie napotykamy po raz pierwszy konkretne równanie
różniczkowe
\begin{equation}
  \label{eq:Matwiejew-Metody-calkowania-ETC-04}
  y'( x ) - 2 x = 0.
\end{equation}
Z~równania tego w~prosty sposób odczytujemy funkcję $F$:
\begin{equation}
  \label{eq:Matwiejew-Metody-calkowania-ETC-05}
  F( x_{ 0 }, x_{ 1 }, x_{ 2 } ) = x_{ 2 } - 2 x_{ 0 }.
\end{equation}
Stajemy tu jednak przed problemem, jaka jest dziedzina funkcji~$F$?
W~dalszym ciągu jeśli nie powiedziano inaczej, za dziedzinę funkcji $F$
będziemy uważać największy zbiór na~którym jesteśmy w~stanie ją określić.
Ponieważ zgodnie z~tym co powiedziano wcześniej, dziedzina funkcji $F$ ma
być podzbiorem $\Rbb^{ n + 1 }$ dla pewnego $n$, dziedziną funkcji określonej
przez \eqref{eq:Matwiejew-Metody-calkowania-ETC-05} jest $\Rbb^{ 3 }$.







% \noindent
% \Str{15} Równanie (10) zostało bardzo elegancko wyprowadzone, przy
% założeniu, że~funkcja uwikłana dana równaniem (9) spełnia
% równanie różniczkowe (1), nie odpowiada to jednak na pytanie czy jeśli
% funkcja $y( x )$ spełnia równanie (10), to spełnia też interesujące
% nas równanie wyjściowe. Dowód tego twierdzenia jest następujący.
% Załóżmy, że $\Phi_{ y }' \neq 0$, tak by było zapewnione istnienie
% funkcji uwikłanej. Jeżeli teraz spełnione jest równanie (10), to można
% je przekształcić do postaci:
% \begin{equation}
%   \label{MatwiejewMCRRZ-01}
%   -\frac{ \Phi_{ x }' }{ \Phi_{ y }' } = f( x, y ).
% \end{equation}
% Lewa strona tej równości jest równa pochodnej funkcji uwikłanej
% $y( x )$, określonej wzorem (9).

% \vspace{\VerSpaceThree}





% \noindent
% \Str{16} \Dok \\

% \vspace{\spaceFour}





% \noindent
% \Str{31} \Dok




% ##################
\newpage

\CenterBoldFont{Błędy}


\begin{center}

  \begin{tabular}{|c|c|c|c|c|}
    \hline
    Strona & \multicolumn{2}{c|}{Wiersz} & Jest
                              & Powinno być \\ \cline{2-3}
    & Od góry & Od dołu & & \\
    \hline
    5   & &  3 & wierdzenia & twierdzenia \\
    % 5   & &  8 & potkowych & początkowych \\
    % 5   & &  7 & poąątkowych & początkowych \\
    % 10  & & 19 & damy & mamy \\
    % 15  & & & & \\
    % & & & & \\
    % & & & & \\
    \hline
  \end{tabular}

\end{center}

\vspace{\VerSpaceSix}


% \noindent
% \StrWg{15}{13}
% \Jest  w~sensie ustępu \\
% \Powin w~sensie zdefiniowanym w~ustępie \\
% \StrWd{20}{2} \\
% \Jest  i~nie ma rozwiązania określonego w~tym samym przedziale
% nie~identycznego z~rozwiązaniem $y = y( x )$ chociażby w~jednym
% punkcie przedziału $\absOne{ x - x_{ 0 } } \leq h$ różnym
% od~punktu $x = x_{ 0 }$. \\
% \Powin i~nie istnieje inne rozwiązanie określone w~przedziale
% $\absOne{ x - x_{ 0 } } \leq h_{ 1 } \leq h$ które nie byłoby równe
% rozwiązaniu $y = y( x )$ w~każdym punkcie przedziału
% $\absOne{ x - x_{ 0 } } \leq h_{ 1 }$. \\


% ############################







% ####################################################################
% ####################################################################
% Bibliografia

\bibliographystyle{plalpha}

\bibliography{MathComScienceBooks}{}





% ############################

% Koniec dokumentu
\end{document}
