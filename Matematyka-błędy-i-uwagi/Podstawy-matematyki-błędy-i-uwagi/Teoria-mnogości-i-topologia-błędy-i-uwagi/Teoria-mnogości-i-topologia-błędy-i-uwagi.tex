% ---------------------------------------------------------------------
% Podstawowe ustawienia i pakiety
% ---------------------------------------------------------------------
\RequirePackage[l2tabu, orthodox]{nag}  % Wykrywa przestarzałe i niewłaściwe
% sposoby używania LaTeXa. Więcej jest w l2tabu English version.
\documentclass[a4paper,11pt]{article}
% {rozmiar papieru, rozmiar fontu}[klasa dokumentu]
\usepackage[MeX]{polski}  % Polonizacja LaTeXa, bez niej będzie pracował
% w języku angielskim.
\usepackage[utf8]{inputenc} % Włączenie kodowania UTF-8, co daje dostęp
% do polskich znaków.
\usepackage{lmodern}  % Wprowadza fonty Latin Modern.
\usepackage[T1]{fontenc}  % Potrzebne do używania fontów Latin Modern.



% ------------------------------
% Podstawowe pakiety (niezwiązane z ustawieniami języka)
% ------------------------------
\usepackage{microtype}  % Twierdzi, że poprawi rozmiar odstępów w tekście.
% \usepackage{graphicx}  % Wprowadza bardzo potrzebne komendy do wstawiania
% grafiki.
\usepackage{vmargin}  % Pozwala na prostą kontrolę rozmiaru marginesów,
% za pomocą komend poniżej. Rozmiar odstępów jest mierzony w calach.
% ------------------------------
% MARGINS
% ------------------------------
\setmarginsrb
{ 0.7in}  % left margin
{ 0.6in}  % top margin
{ 0.7in}  % right margin
{ 0.8in}  % bottom margin
{  20pt}  % head height
{0.25in}  % head sep
{   9pt}  % foot height
{ 0.3in}  % foot sep



% ------------------------------
% Często używane pakiety
% ------------------------------
% \usepackage{csquotes}  % Pozwala w prosty sposób wstawiać cytaty do tekstu.
\usepackage{xcolor}  % Pozwala używać kolorowych czcionek (zapewne dużo
% więcej, ale ja nie potrafię nic o tym powiedzieć).



% ------------------------------
% Pakiety do tekstów z nauk przyrodniczych
% ------------------------------
\let\lll\undefined  % Amsmath gryzie się z pakietami do języka
% polskiego, bo oba definiują komendę \lll. Aby rozwiązać ten problem
% oddefiniowuję tę komendę, ale może tym samym pozbywam się dużego Ł.
\usepackage[intlimits]{amsmath}  % Podstawowe wsparcie od American
% Mathematical Society (w skrócie AMS)
\usepackage{amsfonts, amssymb, amscd, amsthm}  % Dalsze wsparcie od AMS
\usepackage{upgreek}  % Ładniejsze greckie litery
% Przykładowa składnia: pi = \uppi
\usepackage{calrsfs}  % Zmienia czcionkę kaligraficzną w \mathcal
% na ładniejszą. Może w innych miejscach robi to samo, ale o tym nic
% nie wiem.



% ------------------------------
% Wspaniały pakiet PGF/TikZ
% ------------------------------
% \usepackage{tikz}

% \usetikzlibrary{decorations.markings}  % Włączenie konkretnych bibliotek
% % pakietu TikZ



% ---------------
% Tworzenie otoczeń "Twierdzenie", "Definicja", "Lemat", etc.
% ---------------
\newtheorem{theorem}{Twierdzenie}  % Komenda wprowadzająca otoczenie
% „theorem” do pisania twierdzeń matematycznych
\newtheorem{definition}{Definicja}  % Analogicznie jak powyżej
\newtheorem{corollary}{Wniosek}



% ------------------------------
% Pakiety których pliki *.sty mają być w tym samym katalogu co ten plik
% ------------------------------
\usepackage{latexgeneralcommands}
\usepackage{mathcommands}
% \usepackage{calculuscommands}




% ---------------------------------------------------------------------
% Dodatkowe ustawienia dla języka polskiego
% ---------------------------------------------------------------------
\renewcommand{\thesection}{\arabic{section}.}
% Kropki po numerach rozdziału (polski zwyczaj topograficzny)
\renewcommand{\thesubsection}{\thesection\arabic{subsection}}
% Brak kropki po numerach podrozdziału



% ------------------------------
% Ustawienia różnych parametrów tekstu
% ------------------------------
\renewcommand{\baselinestretch}{1.1}

\renewcommand{\arraystretch}{1.4}  % Ustawienie szerokości odstępów między
% wierszami w tabelach





% ------------------------------
% Pakiet „hyperref”
% Polecano by umieszczać go na końcu preambuły
% ------------------------------
\usepackage{hyperref}  % Pozwala tworzyć hiperlinki i zamienia odwołania
% do bibliografii na hiperlinki










% ---------------------------------------------------------------------
% Tytuł i autor tekstu
\title{Teoria mnogości i~topologia \\
  {\Large Błędy i~uwagi}}

\author{Kamil Ziemian}


% \date{}
% ---------------------------------------------------------------------










% ####################################################################
\begin{document}
% ####################################################################





% ######################################
\maketitle  % Tytuł całego tekstu
% ######################################





% % ######################################
% \section{XIX szkoła analizy matematycznej}

% \vspace{\spaceTwo}
% % ######################################





% ############################
\Work{ % Autor i tytuł dzieła
  Kazimierz Kuratowski \\
  \textit{Wstęp do~teorii mnogości i~topologii},
  \cite{KuratowskiWstepTeoriiMnogosciITopologii2004}}

\vspace{0em}


% ##################
\CenterBoldFont{Uwagi}

\vspace{0em}


\noindent
Brak ,,gramatyki'' pisania zdań złożonych logicznych. Jest to jednak usprawiedliwione względami dydaktycznymi.

\vspace{\spaceFour}











% ##################
\CenterBoldFont{Uwagi do konkretnych stron}

\vspace{0em}



Nie wspomniano o rozdzielności dodawania zbiorów względem mnożenia. Jest to w zadaniach.


\Str{10} Należy uwzględnić dwa typy dowodu używające negacji logicznej.
Pierwszy to~dowód przez \textbf{kontrapozycję}, opierający~się
na~tożsamości
\begin{equation}
  \label{eq:Kuratowski-Wstep-do-teorii-mnogosci-ETC-01}
  ( \alpha \Rightarrow \beta ) \equiv ( \neg \beta \Rightarrow \neg \alpha).
\end{equation}
Drugi to~dowód \textbf{nie wprost}, opierający~się na~tożsamości
\begin{equation}
  \label{eq:Kuratowski-Wstep-do-teorii-mnogosci-ETC-02}
  ( \alpha \Rightarrow \beta ) \equiv \neg ( \alpha \land \neg \beta ).
\end{equation}

Pomimo tego, Kuratowski przypisał wszystkie dowody „niewprost” zasadzie kontrapozycji.

\vspace{\spaceFour}





% ##################
\CenterBoldFont{Błędy}

\vspace{\spaceFive}


\begin{center}

  \begin{tabular}{|c|c|c|c|c|}
    \hline
    Strona & \multicolumn{2}{c|}{Wiersz} & Jest
                              & Powinno być \\ \cline{2-3}
    & Od góry & Od dołu & & \\
    \hline
    5   & & 19 & porządkowyh & porządkowych \\
    5   & &  6 & to & do \\
    11  &  3 & & \textbf{7.} & \textbf{7a.} \\
    11  &  4 & & \textbf{7a.} & \textbf{7b.} \\
    % & & & & \\
    % & & & & \\
    % & & & & \\
    % & & & & \\
    % & & & & \\
    % & & & & \\
    % & & & & \\
    % & & & & \\
    % & & & & \\
    % & & & & \\
    % & & & & \\
    % & & & & \\
    % & & & & \\
    % & & & & \\
    % & & & & \\
    % & & & & \\
    % & & & & \\
    % & & & & \\
    % & & & & \\
    % & & & & \\
    % & & & & \\
    % & & & & \\
    % & & & & \\
    % & & & & \\
    % & & & & \\
    % & & & & \\
    % & & & & \\
    % & & & & \\
    % & & & & \\
    % & & & & \\
    % & & & & \\
    % & & & & \\
    % & & & & \\
    % & & & & \\
    % & & & & \\
    \hline
  \end{tabular}

\end{center}

\vspace{\spaceTwo}


Nie wspomniano o rozdzielności dodawania zbiorów względem mnożenia. Jest to w zadaniach.

% ############################



% % ##################
% \newpage

% \CenterBoldFont{Błędy}

% \vspace{\spaceFive}


% \begin{center}

%   \begin{tabular}{|c|c|c|c|c|}
%     \hline
%     Strona & \multicolumn{2}{c|}{Wiersz} & Jest
%                               & Powinno być \\ \cline{2-3}
%     & Od góry & Od dołu & & \\
%     \hline
%     % & & & & \\
%     % & & & & \\
%     % & & & & \\
%     % & & & & \\
%     % & & & & \\
%     % & & & & \\
%     % & & & & \\
%     \hline
%   \end{tabular}

% \end{center}

\vspace{\spaceTwo}


% ############################

















% % ##################
% \CenterBoldFont{Błędy}


% \begin{center}

%   \begin{tabular}{|c|c|c|c|c|}
%     \hline
%     & \multicolumn{2}{c|}{} & & \\
%     Strona & \multicolumn{2}{c|}{Wiersz} & Jest
%                               & Powinno być \\ \cline{2-3}
%     & Od góry & Od dołu & & \\
%     \hline
%     % & & & & \\
%     % & & & & \\
%     \hline
%   \end{tabular}





%   % \begin{tabular}{|c|c|c|c|c|}
%   %   \hline
%   %   & \multicolumn{2}{c|}{} & & \\
%   %   Strona & \multicolumn{2}{c|}{Wiersz} & Jest
%   %                             & Powinno być \\ \cline{2-3}
%   %   & Od góry & Od dołu & & \\
%   %   \hline
%   %   & & & & \\
%   %   \hline
%   % \end{tabular}

% \end{center}


% \noindent
% \StrWd{}{} \\
% \Jest   \\
% \Powin  \\


% \vspace{\spaceTwo}
% % ############################










% % ############################
% \newpage

% \Work{ % Autorzy i tytuł dzieła
%   J. Jost, X. Li-Jost \\
%   \textit{Calculus of Variations},
%   \cite{JostLiJostCalculusOfVariations1998}}

% \vspace{0em}

% % ##################
% \CenterBoldFont{Uwagi}

% \vspace{0em}


% \Str{6} Dowód podstawowego lematu rachunku wariacyjnego jest
% poprawny, acz mało elegancko zrobiony.


% % % ##################
% % \CenterBoldFont{Błędy}


% % \begin{center}

% %   \begin{tabular}{|c|c|c|c|c|}
% %     \hline
% %     & \multicolumn{2}{c|}{} & & \\
% %     Strona & \multicolumn{2}{c|}{Wiersz} & Jest
% %                               & Powinno być \\ \cline{2-3}
% %     & Od góry & Od dołu & & \\
% %     \hline
% %     & & & & \\
% %     & & & & \\
% %     \hline
% %   \end{tabular}

% % \end{center}

% \vspace{\spaceTwo}


% % ############################















% ####################################################################
% ####################################################################
% Bibliografia
\bibliographystyle{plalpha}

\bibliography{MathComScienceBooks}{}





% ############################

% Koniec dokumentu
\end{document}
