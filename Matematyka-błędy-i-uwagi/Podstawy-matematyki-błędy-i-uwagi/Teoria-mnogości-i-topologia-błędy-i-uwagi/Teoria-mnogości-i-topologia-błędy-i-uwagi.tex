% ---------------------------------------------------------------------
% Podstawowe ustawienia i pakiety
% ---------------------------------------------------------------------
\RequirePackage[l2tabu, orthodox]{nag}  % Wykrywa przestarzałe i niewłaściwe
% sposoby używania LaTeXa. Więcej jest w l2tabu English version.
\documentclass[a4paper,11pt]{article}
% {rozmiar papieru, rozmiar fontu}[klasa dokumentu]
\usepackage[MeX]{polski}  % Polonizacja LaTeXa, bez niej będzie pracował
% w języku angielskim.
\usepackage[utf8]{inputenc} % Włączenie kodowania UTF-8, co daje dostęp
% do polskich znaków.
\usepackage{lmodern}  % Wprowadza fonty Latin Modern.
\usepackage[T1]{fontenc}  % Potrzebne do używania fontów Latin Modern.



% ------------------------------
% Podstawowe pakiety (niezwiązane z ustawieniami języka)
% ------------------------------
\usepackage{microtype}  % Twierdzi, że poprawi rozmiar odstępów w tekście.
% \usepackage{graphicx}  % Wprowadza bardzo potrzebne komendy do wstawiania
% grafiki.
\usepackage{vmargin}  % Pozwala na prostą kontrolę rozmiaru marginesów,
% za pomocą komend poniżej. Rozmiar odstępów jest mierzony w calach.
% ------------------------------
% MARGINS
% ------------------------------
\setmarginsrb
{ 0.7in}  % left margin
{ 0.6in}  % top margin
{ 0.7in}  % right margin
{ 0.8in}  % bottom margin
{  20pt}  % head height
{0.25in}  % head sep
{   9pt}  % foot height
{ 0.3in}  % foot sep



% ------------------------------
% Często używane pakiety
% ------------------------------
% \usepackage{csquotes}  % Pozwala w prosty sposób wstawiać cytaty do tekstu.
\usepackage{xcolor}  % Pozwala używać kolorowych czcionek (zapewne dużo
% więcej, ale ja nie potrafię nic o tym powiedzieć).



% ------------------------------
% Pakiety do tekstów z nauk przyrodniczych
% ------------------------------
\let\lll\undefined  % Amsmath gryzie się z pakietami do języka
% polskiego, bo oba definiują komendę \lll. Aby rozwiązać ten problem
% oddefiniowuję tę komendę, ale może tym samym pozbywam się dużego Ł.
\usepackage[intlimits]{amsmath}  % Podstawowe wsparcie od American
% Mathematical Society (w skrócie AMS)
\usepackage{amsfonts, amssymb, amscd, amsthm}  % Dalsze wsparcie od AMS
\usepackage{upgreek}  % Ładniejsze greckie litery
% Przykładowa składnia: pi = \uppi
\usepackage{calrsfs}  % Zmienia czcionkę kaligraficzną w \mathcal
% na ładniejszą. Może w innych miejscach robi to samo, ale o tym nic
% nie wiem.



% ------------------------------
% Wspaniały pakiet PGF/TikZ
% ------------------------------
% \usepackage{tikz}

% \usetikzlibrary{decorations.markings}  % Włączenie konkretnych bibliotek
% % pakietu TikZ



% ------------------------------
% Tworzenie środowisk (?) „Twierdzenie”, „Definicja”, „Lemat”, etc.
% ------------------------------
% Komenda wprowadzająca otoczenie „theorem” do pisania twierdzeń
% matematycznych.
\newtheorem{theorem}{Twierdzenie}
% Analogicznie jak powyżej
\newtheorem{definition}{Definicja}
\newtheorem{corollary}{Wniosek}



% ------------------------------
% Pakiety których pliki *.sty mają być w tym samym katalogu co ten plik
% ------------------------------
\usepackage{latexgeneralcommands}
\usepackage{mathcommands}
% \usepackage{calculuscommands}




% ---------------------------------------------------------------------
% Dodatkowe ustawienia dla języka polskiego
% ---------------------------------------------------------------------
\renewcommand{\thesection}{\arabic{section}.}
% Kropki po numerach rozdziału (polski zwyczaj topograficzny)
\renewcommand{\thesubsection}{\thesection\arabic{subsection}}
% Brak kropki po numerach podrozdziału



% ------------------------------
% Ustawienia różnych parametrów tekstu
% ------------------------------
\renewcommand{\baselinestretch}{1.1}

% Ustawienie szerokości odstępów między wierszami w tabelach.
\renewcommand{\arraystretch}{1.4}



% ------------------------------
% Pakiet „hyperref”
% Polecano by umieszczać go na końcu preambuły
% ------------------------------
\usepackage{hyperref}  % Pozwala tworzyć hiperlinki i zamienia odwołania
% do bibliografii na hiperlinki










% ---------------------------------------------------------------------
% Tytuł i autor tekstu
\title{Teoria mnogości i~topologia \\
  {\Large Błędy i~uwagi}}

\author{Kamil Ziemian}


% \date{}
% ---------------------------------------------------------------------










% ####################################################################
\begin{document}
% ####################################################################





% ######################################
\maketitle  % Tytuł całego tekstu
% ######################################





% % ######################################
% \section{XIX szkoła analizy matematycznej}

% \vspace{\spaceTwo}
% % ######################################





% ############################
\Work{ % Autor i tytuł dzieła
  Kazimierz Kuratowski \\
  \textit{Wstęp do~teorii mnogości i~topologii},
  \cite{KuratowskiWstepTeoriiMnogosciITopologii2004}}

\vspace{0em}


% ##################
\CenterBoldFont{Uwagi do~konkretnych rozdziałów}

\vspace{0em}


\noindent
\textbf{Rozdział I.} Ten bardzo krótki rozdział jest tylko skromnym zarysem
pewnego działu logiki matematycznej, dlatego powinniśmy podchodzić do jego
treści z~wyrozumiałością i~przechodzić do porządku dziennego nad większością
rzeczy, które możemy uznać za jego niedociągnięcia. W~pewnych szczególnych
przypadkach, gdy uproszczenia w~nim zawarte uznamy za zbyt wielkie, omówione
to zostanie w~„Uwagach do konkretnych stron”.

\vspace{\spaceFour}











% ##################
\CenterBoldFont{Uwagi do konkretnych stron}

\vspace{0em}



Nie wspomniano o rozdzielności dodawania zbiorów względem mnożenia. Jest to
w zadaniach.

\noindent
\Str{10} Z~jakiegoś powodu \textbf{zasadą kontrapozycji} została tu nazwana
tożsamość
\begin{equation}
  \label{eq:Kuratowski-Wstep-do-teorii-mnogosci-ETC-01}
  ( \neg \beta \Rightarrow \neg \alpha ) \Rightarrow ( \alpha \Rightarrow \beta ),
\end{equation}
podczas gdy zachodzi jej mocniejsze wersja
\begin{equation}
  \label{eq:Kuratowski-Wstep-do-teorii-mnogosci-ETC-02}
  ( \alpha \Rightarrow \beta ) \equiv ( \neg \beta \Rightarrow \neg \alpha ).
\end{equation}
Przez „zasadę kontrapozycji” powinno~się rozumieć, tą drugą, mocniejszą
wersję tożsamości.

\vspace{\spaceFour}





\noindent
\Str{10} W~tym miejscu wszystkie dowody
„nie wprost”\footnote{Stosujemy tu nazewnictwo różniące~się od tego
  w~książce. Mamy nadzieję, że~będzie on prostsze w~zrozumieniu i~użyciu.},
nazwane dowodami „przez sprowadzenie do niedorzeczności”, przypisano
zasadzie kontrapozycji:
\begin{equation}
  \label{eq:Kuratowski-Wstep-do-teorii-mnogosci-ETC-03}
  ( \alpha \Rightarrow \beta ) \equiv ( \neg \beta \Rightarrow \neg \alpha).
\end{equation}
W~rzeczywistości pod nazwą „dowody nie wprost” kryją się dwa typy dowodów,
z~których jeden opiera~się na podanej powyżej tożsamości. Drugi typ dowodu,
który będziemy nazywać\textbf{dowodem przez sprowadzenie
  do~niedorzeczności}, opierają się na poniższej tożsamości
\begin{equation}
  \label{eq:Kuratowski-Wstep-do-teorii-mnogosci-ETC-04}
  ( \alpha \Rightarrow \beta ) \equiv \neg \big( \alpha \land ( \neg \beta ) \big).
\end{equation}

\vspace{\spaceFour}





\noindent
\StrWg{10}{13} Odniesienie do przypisu numer~$1$ w~tym wierszu wygląda dość
brzydko. W~tym momencie nie mam dobrej idei jak to poprawić.

\vspace{\spaceFour}





\noindent
\Str{14} Często przyjmuje~się inną konwencję nazewniczą i~przez
\textbf{podzbiory właściwe zbioru~$A$} uznaje~się wszystkie podzbiory~$A$
różne od~$A$ i~$\emptyset$.

\vspace{\spaceFour}





\noindent
\Str{14} Na tej stronie podana jest następująca implikacja
\begin{equation}
  \label{eq:Kuratowski-Wstep-do-teorii-mnogosci-ETC-05}
  ( A \subset B ) \land ( B \subset A ) \Rightarrow ( A = B ).
\end{equation}
W~istocie od razu możemy otrzymać mocniejszy twierdzenie:
\begin{equation}
  \label{eq:Kuratowski-Wstep-do-teorii-mnogosci-ETC-06}
  ( A \subset B ) \land ( B \subset A ) \equiv ( A = B ).
\end{equation}

\vspace{\spaceFour}





\noindent
\Str{15--16} Gdy wprowadzamy pojęcie „przestrzeni” w~której, czyli takiego
zbioru, że~wszystkie rozważane zbiory są jego podzbiorami, pojawia~się
dodatkowym problem logiczny. Oznaczmy naszą przestrzeń przez $X$ i~rozważmy
formułę
\begin{equation}
  \label{eq:Kuratowski-Wstep-do-teorii-mnogosci-ETC-06}
  x \in A.
\end{equation}
Wydaje się, że~poprawna postać tej formuły to
\begin{equation}
  \label{eq:Kuratowski-Wstep-do-teorii-mnogosci-ETC-07}
  ( x \in X ) \land ( x \in A ),
\end{equation}
jednak dla wygody zwykle opuszcza~się pierwszy człon powyższego iloczynu
logicznego.

Powstaje jednak pytanie, czy praca w~ramach danej przestrzeni nie wymaga
doprecyzowania innych formuł logicznych? Oraz czy wzór
\eqref{eq:Kuratowski-Wstep-do-teorii-mnogosci-ETC-07}
zawiera wszystkie potrzebne modyfikacje, jakie musimy wprowadzić dla formuły
$x \in A$.

\vspace{\spaceFour}





\Str{16} Niech $X$ oznacza przestrzeń w~której pracujemy. Wprowadzamy
oznaczenie
\begin{equation}
  \label{eq:Kuratowski-Wstep-do-teorii-mnogosci-ETC-08}
  A^{ c } := X \setminus A,
\end{equation}
gdzie $\setminus$ oznacza różnicę zbiorów. Teraz możemy zapisać wzór (28)
obecny na tej stronie jako
\begin{equation}
  \label{eq:Kuratowski-Wstep-do-teorii-mnogosci-ETC-09}
  ( B^{ c } \subset A^{ c } ) \Rightarrow ( A \subset B ).
\end{equation}
Tak jak w~poprzednich przypadkach wypadkach, w~prosty sposób można
udowodnić silniejszą tożsamość:
\begin{equation}
  \label{eq:Kuratowski-Wstep-do-teorii-mnogosci-ETC-10}
  ( B^{ c } \subset A^{ c } ) \equiv ( A \subset B ).
\end{equation}

\vspace{\spaceFour}





\Str{18} Zarysowano tu związek między rachunkiem zdań, algebrą zbiorów
i~algebrą Boole'a. Obawiam~się, że~taka skrótowa forma tych relacji może
wprowadzać w~błąd i~prowadzić do nieporozumień. Zanim więc przystąpi~się
do~wyciągania z~tego konkretnych wniosków, musimy bardziej szczegółowo
przebadać te relacje.

\vspace{\spaceFour}













% ##################
\CenterBoldFont{Błędy}

\vspace{\spaceFive}


\begin{center}

  \begin{tabular}{|c|c|c|c|c|}
    \hline
    Strona & \multicolumn{2}{c|}{Wiersz} & Jest
                              & Powinno być \\ \cline{2-3}
    & Od góry & Od dołu & & \\
    \hline
    5   & & 19 & porządkowyh & porządkowych \\
    5   & &  6 & to & do \\
    11  &  3 & & \textbf{7.} & \textbf{7a.} \\
    11  &  4 & & \textbf{7a.} & \textbf{7b.} \\
    15  & 15 & & 7a, & 7b, \\
    16  & 12 & & (3), (23) i~(2) & (2), (3) i~(23) \\
    % & & & & \\
    24  &  9 & & $\lor a \in \{ x : \psi( x ) \} ]$ & $\lor [ a \in \{ x : \psi( x ) \} ]$ \\
    % & & & & \\
    51  & 17 & & $f( A_{ l } )$ & $f( A_{ 1 } )$ \\
    % & & & & \\
    % & & & & \\
    % & & & & \\
    % & & & & \\
    % & & & & \\
    % & & & & \\
    % & & & & \\
    % & & & & \\
    % & & & & \\
    % & & & & \\
    % & & & & \\
    % & & & & \\
    % & & & & \\
    % & & & & \\
    % & & & & \\
    % & & & & \\
    % & & & & \\
    % & & & & \\
    % & & & & \\
    % & & & & \\
    % & & & & \\
    % & & & & \\
    % & & & & \\
    % & & & & \\
    % & & & & \\
    % & & & & \\
    % & & & & \\
    % & & & & \\
    % & & & & \\
    \hline
  \end{tabular}





  % \newpage

  %   \begin{tabular}{|c|c|c|c|c|}
%     \hline
%     Strona & \multicolumn{2}{c|}{Wiersz} & Jest
%                               & Powinno być \\ \cline{2-3}
%     & Od góry & Od dołu & & \\
%     \hline
%     % & & & & \\
%     % & & & & \\
%     % & & & & \\
%     % & & & & \\
%     % & & & & \\
%     % & & & & \\
%     % & & & & \\
%     \hline
%   \end{tabular}

\end{center}

\vspace{\spaceTwo}


\noindent
\StrWg{13}{2} \\
\Jest  Mamy więc na rysunku 2 $A \cap B = \emptyset$, a~na rysunku 3 $B - A = \emptyset$. \\
\Powin Dla rysunku 2 mam tym $A \cap B = \emptyset$, a~dla rysunku 3 zachodzi $B - A = \emptyset$. \\

% ############################



















% % ##################
% \CenterBoldFont{Błędy}


% \begin{center}

%   \begin{tabular}{|c|c|c|c|c|}
%     \hline
%     Strona & \multicolumn{2}{c|}{Wiersz} & Jest
%                               & Powinno być \\ \cline{2-3}
%     & Od góry & Od dołu & & \\
%     \hline
%     % & & & & \\
%     % & & & & \\
%     \hline
%   \end{tabular}





%   % \begin{tabular}{|c|c|c|c|c|}
%   %   \hline
%   %   & \multicolumn{2}{c|}{} & & \\
%   %   Strona & \multicolumn{2}{c|}{Wiersz} & Jest
%   %                             & Powinno być \\ \cline{2-3}
%   %   & Od góry & Od dołu & & \\
%   %   \hline
%   %   & & & & \\
%   %   \hline
%   % \end{tabular}

% \end{center}


% \noindent
% \StrWd{}{} \\
% \Jest   \\
% \Powin  \\


% \vspace{\spaceTwo}
% % ############################










% % ############################
% \newpage

% \Work{ % Autorzy i tytuł dzieła
%    \\
%   \textit{},
%   \cite{}}

% \vspace{0em}


% % ##################
% \CenterBoldFont{Uwagi}

% \vspace{0em}


% \Str{}


% % % ##################
% % \CenterBoldFont{Błędy}


% % \begin{center}

% %   \begin{tabular}{|c|c|c|c|c|}
% %     \hline
% %     Strona & \multicolumn{2}{c|}{Wiersz} & Jest
% %                               & Powinno być \\ \cline{2-3}
% %     & Od góry & Od dołu & & \\
% %     \hline
% %     & & & & \\
% %     & & & & \\
% %     \hline
% %   \end{tabular}

% % \end{center}

% \vspace{\spaceTwo}


% % ############################















% ####################################################################
% ####################################################################
% Bibliografia

\bibliographystyle{plalpha}

\bibliography{MathComScienceBooks}{}





% ############################

% Koniec dokumentu
\end{document}
