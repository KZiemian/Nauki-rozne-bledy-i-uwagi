% ---------------------------------------------------------------------
% Podstawowe ustawienia i pakiety
% ---------------------------------------------------------------------
\RequirePackage[l2tabu, orthodox]{nag} % Wykrywa przestarzałe i niewłaściwe
% sposoby używania LaTeXa. Więcej jest w l2tabu English version.
\documentclass[a4paper,11pt]{article}
% {rozmiar papieru, rozmiar fontu}[klasa dokumentu]
\usepackage[MeX]{polski} % Polonizacja LaTeXa, bez niej będzie pracował
% w języku angielskim.
\usepackage[utf8]{inputenc} % Włączenie kodowania UTF-8, co daje dostęp
% do polskich znaków.
\usepackage{lmodern} % Wprowadza fonty Latin Modern.
\usepackage[T1]{fontenc} % Potrzebne do używania fontów Latin Modern.



% ------------------------------
% Podstawowe pakiety (niezwiązane z ustawieniami języka)
% ------------------------------
\usepackage{microtype} % Twierdzi, że poprawi rozmiar odstępów w tekście.
\usepackage{graphicx} % Wprowadza bardzo potrzebne komendy do wstawiania
% grafiki.
\usepackage{verbatim} % Poprawia otoczenie VERBATIME.
\usepackage{textcomp} % Dodaje takie symbole jak stopnie Celsiusa,
% wprowadzane bezpośrednio w tekście.
\usepackage{vmargin} % Pozwala na prostą kontrolę rozmiaru marginesów,
% za pomocą komend poniżej. Rozmiar odstępów jest mierzony w calach.
% ------------------------------
% MARGINS
% ------------------------------
\setmarginsrb
{ 0.7in}  % left margin
{ 0.6in}  % top margin
{ 0.7in}  % right margin
{ 0.8in}  % bottom margin
{  20pt}  % head height
{0.25in}  % head sep
{   9pt}  % foot height
{ 0.3in}  % foot sep



% ------------------------------
% Często przydatne pakiety
% ------------------------------
\usepackage{csquotes} % Pozwala w prosty sposób wstawiać cytaty do tekstu.
\usepackage{xcolor} % Pozwala używać kolorowych czcionek (zapewne dużo
% więcej, ale ja nie potrafię nic o tym powiedzieć).



% ------------------------------
% Pakiety do tekstów z nauk przyrodniczych
% ------------------------------
\let\lll\undefined % Amsmath gryzie się z językiem pakietami do języka
% polskiego, bo oba definiują komendę \lll. Aby rozwiązać ten problem
% oddefiniowuję tę komendę, ale może tym samym pozbywam się dużego Ł.
\usepackage[intlimits]{amsmath} % Podstawowe wsparcie od American
% Mathematical Society (w skrócie AMS)
\usepackage{amsfonts, amssymb, amscd, amsthm} % Dalsze wsparcie od AMS
% \usepackage{siunitx} % Dla prostszego pisania jednostek fizycznych
\usepackage{upgreek} % Ładniejsze greckie litery
% Przykładowa składnia: pi = \uppi
\usepackage{slashed} % Pozwala w prosty sposób pisać slash Feynmana.
\usepackage{calrsfs} % Zmienia czcionkę kaligraficzną w \mathcal
% na ładniejszą. Może w innych miejscach robi to samo, ale o tym nic
% nie wiem.



% ---------------
% Wspaniały pakiet PGF/TikZ
% ---------------
\usepackage{tikz}

% Włączenie konkretnych bibliotek pakietu TikZ.
\usetikzlibrary{decorations.markings}

% Picsy TikZa
\usepackage{TikZPics}

% Dodatkowe style TikZa
\usepackage{TikZStyles}



% ------------------------------
% Tworzenie środowisk (?) „Twierdzenie”, „Definicja”, „Lemat”, etc.
% ------------------------------
% Komenda wprowadzająca otoczenie „theorem” do pisania twierdzeń
% matematycznych.
\newtheorem{theorem}{Twierdzenie}
% Analogicznie jak powyżej
\newtheorem{definition}{Definicja}
\newtheorem{corollary}{Wniosek}



% ---------------------------------------
% Pakiety napisane przez użytkownika.
% Mają być w tym samym katalogu to ten plik .tex
% ---------------------------------------
\usepackage{latexgeneralcommands}
\usepackage{mathcommands}



% ---------------------------------------------------------------------
% Dodatkowe ustawienia dla języka polskiego
% ---------------------------------------------------------------------
\renewcommand{\thesection}{\arabic{section}.}
% Kropki po numerach rozdziału (polski zwyczaj topograficzny)
\renewcommand{\thesubsection}{\thesection\arabic{subsection}}
% Brak kropki po numerach podrozdziału



% ------------------------------
% Ustawienia różnych parametrów tekstu
% ------------------------------
\renewcommand{\baselinestretch}{1.1}

% Ustawienie szerokości odstępów między wierszami w tabelach.
\renewcommand{\arraystretch}{1.4}



% ------------------------------
% Pakiet „hyperref”
% Polecano by umieszczać go na końcu preambuły.
% ------------------------------
\usepackage{hyperref} % Pozwala tworzyć hiperlinki i zamienia odwołania
% do bibliografii na hiperlinki.










% ---------------------------------------------------------------------
% Tytuł i autor tekstu
\title{Teoria kategorii \\
  {\Large Błędy i~uwagi}}

\author{Kamil Ziemian}


% \date{}
% ---------------------------------------------------------------------










% ####################################################################
\begin{document}
% ####################################################################





% ######################################
\maketitle % Tytuł całego tekstu
% ######################################





% ############################
\Work{ % Autorzy i tytuł dzieła
  Zbigniew Semadeni, Antoni Wiweger \\
  \textit{Wstęp do~teorii kategorii i~funktorów},
  \cite{SemadeniWiwegerWstepDoTeoriiKategoriiETC1972}}

\vspace{0em}


% ##################
\CenterBoldFont{Uwagi do~konkretnych stron}

\vspace{0em}

% \noi \tb{Konkretne strony}

% \vspace{\spaceFour}


\noindent
\StrWierszeD{8}{11 i~12} Wzory brzydko wystają ponad i~poniżej nawiasów,
może da się to jakoś poprawić?

\vspace{\spaceFour}





\noindent
\Str{????} Należy wspomnieć, że~współcześnie w~teorii kategorii stosuje~się
również inną, często dużo wygodniejszą od standardowej, konwencję zapisu
składania odwzorowań. Wedle niej odwzorowania zapisuje~się od lewej do
prawej w~kolejności która odpowiada temu w~jakiej kolejności napotkalibyśmy
dane odwzorowanie przechodząc po diagramie zgodnie z~kierunek strzałek.

Zgodnie z~nią złożenie odwzorowań odpowiadające diagramowi (a) na
rysunku~(1) \eqref{fig:Semadeni-Wiweger-WstepETC-01} zapisalibyśmy wedle
standardowej konwencji jako $\gamma \beta \alpha$, bowiem
\begin{equation}
  \label{eq:Semadeni-Wiweger-WstepETC-01}
  ( \gamma \beta \alpha )( x ) = \gamma\big( \beta\big( \alpha( x ) \big) \big).
\end{equation}
Wedle zaś notacji strzałkowej to odwzorowanie zapiszemy jako $\alpha \beta \gamma$.
W~przypadku diagramu (b) na~tym samym rysunku, sytuacja ma~się analogicznie.

Należy zwrócić uwagę, że~jak zostało to w~książce wyjaśnione w~teorii
kategorii starami~się unikać obliczania wartości odwzorowania dla danego
argumentu~$x$, zamiast tego skupiamy~się na składaniu odpowiednich
odwzorowań. W~przypadku zaś składania odwzorowań notacja strzałkowa nie jest
wygodniejsza od standardowej i~nie prowadzi do żadnych nieporozumień.










% ##################
\begin{figure}

  \centering

  \label{fig:Semadeni-Wiweger-WstepETC-01}

  \begin{tikzpicture}

    \node (A 1) at (0,0) {$A$};

    \node (B 1) at (1.75,0) {$B$};

    \draw[diagram arrow 1] (A 1) -- (B 1);

    \node at (0.875,-0.5) {$\alpha$};



    \node (C 1) at (3.5,0) {$C$};

    \draw[diagram arrow 1] (B 1) -- (C 1);

    \node at (2.625,-0.5) {$\beta$};


    \node (D 1) at (5.25,0) {$D$};

    \draw[diagram arrow 1] (C 1) -- (D 1);

    \node at (4.375,-0.5) {$\gamma$};



    \node at (2.625,-1.5) {(a) Funkcja $\alpha \beta \gamma$};





    \begin{scope}[xshift=8cm]

      \node (A 2) at (0,0) {$A$};

      \node (B 2) at (1.75,0) {$B$};

      \draw[diagram arrow 1] (B 2) -- (A 2);

      \node at (0.875,-0.5) {$\alpha$};



      \node (C 2) at (3.5,0) {$C$};

      \draw[diagram arrow 1] (C 2) -- (B 2);

      \node at (2.625,-0.5) {$\beta$};



      \node (D 2) at (5.25,0) {$D$};

      \draw[diagram arrow 1] (D 2) -- (C 2);

      \node at (4.375,-0.5) {$\gamma$};



      \node at (2.625,-1.5) {(b) Funkcja $\gamma \beta \alpha$};

    \end{scope}

  \end{tikzpicture}

  \caption{Przykłady zapisu składania odwzorowań w~notacji strzałkowej.}

\end{figure}
% ##################








% ##################
\CenterBoldFont{Błędy}


\begin{center}

  \begin{tabular}{|c|c|c|c|c|}
    \hline
    Strona & \multicolumn{2}{c|}{Wiersz} & Jest
                              & Powinno być \\ \cline{2-3}
    & Od góry & Od dołu & & \\
    \hline
    % & & & & \\
    % & & & & \\
    % & & & & \\
    % & & & & \\
    % & & & & \\
    % & & & & \\
    \hline
  \end{tabular}

\end{center}

\vspace{\spaceTwo}




% ############################










% #####################################################################
% #####################################################################
% Bibliografia

\bibliographystyle{plalpha}

\bibliography{MathComScienceBooks}{}





% ############################

% Koniec dokumentu
\end{document}
