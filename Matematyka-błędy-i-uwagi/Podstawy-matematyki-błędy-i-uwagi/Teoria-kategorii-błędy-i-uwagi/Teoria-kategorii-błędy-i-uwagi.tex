


% ------------------------------
% Wonderful package PGF/TikZ
% ------------------------------
\usepackage{tikz}

% Loading concrete TikZ libraries
\usetikzlibrary{decorations.markings}

% Package with TikZ pics
\usepackage{TikZPics}

% Package with additional TikZ styles
\usepackage{TikZStyles}










% ---------------------------------------------------------------------
% Defining title and author of the text
\title{Teoria kategorii \\
  {\Large Błędy i~uwagi}}

\author{Kamil Ziemian}


% \date{}
% ---------------------------------------------------------------------










% ####################################################################
\begin{document}
% ####################################################################





% ######################################
% Title of the text
\maketitle
% ######################################





% ############################
\Work{ % Autorzy i tytuł dzieła
  Zbigniew Semadeni, Antoni Wiweger \\
  \textit{Wstęp do~teorii kategorii i~funktorów},
  \cite{SemadeniWiwegerWstepDoTeoriiKategoriiETC1972}}

\vspace{0em}


% ##################
\CenterBoldFont{Uwagi}

\vspace{0em}


\noindent
\textbf{Podrozdział 1.7.} Należy wspomnieć, że~współcześnie w~teorii
kategorii stosuje~się również inną, często dużo wygodniejszą
od~standardowej, konwencję zapisu składania odwzorowań. Wedle niej
odwzorowania zapisuje~się od lewej do~prawej w~kolejności która odpowiada
temu w~jakiej kolejności napotkalibyśmy dane odwzorowanie przechodząc
po~diagramie zgodnie z~kierunek strzałek.

Zgodnie z~nią złożenie odwzorowań odpowiadające diagramowi~(a) na
rysunku~(1) \eqref{fig:Semadeni-Wiweger-Wstep-ETC-01} zapisalibyśmy wedle
standardowej konwencji jako $\gamma \beta \alpha$, bowiem
\begin{equation}
  \label{eq:Semadeni-Wiweger-Wstep-ETC-01}
  ( \gamma \beta \alpha )( x ) = \gamma\big( \beta\big( \alpha( x ) \big) \big).
\end{equation}
Wedle zaś notacji strzałkowej to odwzorowanie zapiszemy jako $\alpha \beta \gamma$.
W~przypadku diagramu~(b) na~tym samym rysunku, sytuacja ma~się analogicznie.

Należy zwrócić uwagę, że~jak zostało to w~książce wyjaśnione w~teorii
kategorii starami~się unikać obliczania wartości odwzorowania dla danego
argumentu~$x$, zamiast tego skupiamy~się na składaniu odpowiednich
odwzorowań. W~przypadku zaś składania odwzorowań notacja strzałkowa nie jest
wygodniejsza od standardowej i~nie prowadzi do żadnych nieporozumień.

Ponieważ w~tej książce notacja strzałkowa nie jest używana, więc by uniknąć
nieporozumień w~tych notatkach również będziemy jej unikać. Jeśli będziemy
ją w~danej sytuacji stosować zostanie to w~jawny sposób zaznaczone.





% ##################
\begin{figure}

  \centering

  \label{fig:Semadeni-Wiweger-Wstep-ETC-01}

  \begin{tikzpicture}

    \path (0,0) -- (0,1);



    \node (A 1) at (0,0) {$A$};

    \node (B 1) at (1.75,0) {$B$};

    \draw[diagram arrow 1] (A 1) -- (B 1);

    \node at (0.875,-0.5) {$\alpha$};



    \node (C 1) at (3.5,0) {$C$};

    \draw[diagram arrow 1] (B 1) -- (C 1);

    \node at (2.625,-0.5) {$\beta$};



    \node (D 1) at (5.25,0) {$D$};

    \draw[diagram arrow 1] (C 1) -- (D 1);

    \node at (4.375,-0.5) {$\gamma$};



    \node at (2.625,-1.5) {(a)~Funkcja $\alpha \beta \gamma$};





    \begin{scope}[xshift=8cm]

      \node (A 2) at (0,0) {$A$};

      \node (B 2) at (1.75,0) {$B$};

      \draw[diagram arrow 1] (B 2) -- (A 2);

      \node at (0.875,-0.5) {$\alpha$};



      \node (C 2) at (3.5,0) {$C$};

      \draw[diagram arrow 1] (C 2) -- (B 2);

      \node at (2.625,-0.5) {$\beta$};



      \node (D 2) at (5.25,0) {$D$};

      \draw[diagram arrow 1] (D 2) -- (C 2);

      \node at (4.375,-0.5) {$\gamma$};



      \node at (2.625,-1.5) {(b)~Funkcja $\gamma \beta \alpha$};

    \end{scope}

  \end{tikzpicture}

  \caption{Przykłady zapisu składania odwzorowań w~notacji strzałkowej.}

\end{figure}
% ##################


\vspace{\VerSpaceFour}










% ##################
\CenterBoldFont{Uwagi do~konkretnych stron}

\vspace{0em}


\noindent
\StrWierszeD{8}{11 i~12} Wzory brzydko wystają ponad i~poniżej nawiasów,
może da się to jakoś poprawić?

\vspace{\VerSpaceFour}





\noindent
\Str{14}

\vspace{\VerSpaceFour}





\noindent
\Str{21} Zastanówmy~się nad znaczeniem użytego na tej stronie wyrażenia
\begin{equation}
  \label{eq:Semadeni-Wiweger-Wstep-ETC-02}
  v_{ B_{ 1 }, \, B_{ 2 }, \, B_{ 3 } }^{ \Bfrak } \subset
  v_{ B_{ 1 }, \, B_{ 2 }, \, B_{ 3 } }^{ \Ufrak },
\end{equation}
która ma obowiązywać dla dowolnych $B_{ 1 }, B_{ 2 }, B_{ 3 } \in \Bfrak^{ 0 }$.
Według tego co napisano na stronie 15 tej książki,
$v_{ A, \, B, \, C }^{ \Ufrak }$, gdzie $A, B, C \in \Ufrak$ to przekształcenie
\begin{equation}
  \label{eq:Semadeni-Wiweger-Wstep-ETC-03}
  v_{ A, \, B, \, C }^{ \Ufrak } :
  \langle A, B \rangle_{ \Ufrak } \times \langle B, C \rangle_{ \Ufrak } \to
  \langle A, C \rangle_{ \Ufrak }.
\end{equation}
Ponieważ na tejże stronie 15 jest stwierdzone, że~dla wszystkich
$A, B \in \Ufrak$, $\langle A, B \rangle_{ \Ufrak }$ jest zbiorem. Tym samym przekształcenie
$v_{ A, \, B, \, C }^{ \Ufrak }$ (por. str.~17 książki) można utożsamić z~trójką
\begin{equation}
  \label{eq:Semadeni-Wiweger-Wstep-ETC-04}
  \bigg( \left\{ \left(\alpha, \beta, v_{ A, \, B, \, C }^{ \Ufrak }( \alpha, \beta )
    \right) \, \Big| \,
    \alpha \in \langle A, B \rangle_{ \Ufrak }, \, \beta \in \langle B, C \rangle_{ \Ufrak } \right\}, \,
  \langle A, B \rangle_{ \Ufrak } \times \langle B, C \rangle_{ \Ufrak }, \,
  \langle A, C \rangle_{ \Ufrak } \bigg).
\end{equation}
Wobec tego wyrażenie \eqref{eq:Semadeni-Wiweger-Wstep-ETC-02} należy rozumieć
jako skrótowy zapis tego, że~dziedziny i~kodziedziny
$v_{ B_{ 1 }, \, B_{ 2 }, \, B_{ 3 } }^{ \Ufrak }$
oraz~$v_{ B_{ 1 }, \, B_{ 2 }, \, B_{ 3 } }^{ \Ufrak }$ są równe, oraz tego~że
\begin{equation}
  \label{eq:Semadeni-Wiweger-Wstep-ETC-05}
  \begin{split}
    &\left\{ \left( \alpha, \beta,
      v_{ B_{ 1 }, \, B_{ 2 }, \, B_{ 3 } }^{ \Bfrak }( \alpha, \beta )
      \right)
      \, \Big| \,
      \alpha \in \langle B_{ 1 }, \, B_{ 2 } \rangle_{ \Bfrak }, \,
      \beta \in \langle B_{ 2 }, \, B_{ 3 } \rangle_{ \Bfrak } \right\}
      \subset \\
    &\quad
      \subset
      \left\{ \left( \gamma, \delta,
      v_{ B_{ 1 }, \, B_{ 2 }, \, B_{ 3 } }^{ \Ufrak }( \gamma, \delta )
      \right)
      \, \Big| \,
      \gamma \in \langle B_{ 1 }, \, B_{ 2 } \rangle_{ \Ufrak }, \,
      \delta \in \langle B_{ 2 }, \, B_{ 3 } \rangle_{ \Ufrak } \right\}.
  \end{split}
\end{equation}
Warto zauważyć, iż~z~powyższej zależności wynika od razu warunek numer
(\romannumeral2) na to by kategoria $\Bfrak$ była podkategorią $\Ufrak$,
czyli~że
\begin{equation}
  \label{eq:Semadeni-Wiweger-Wstep-ETC-06}
  \langle B_{ 1 }, B_{ 2 } \rangle_{ \Bfrak } \subset \langle B_{ 1 }, B_{ 3 } \rangle_{ \Ufrak }
\end{equation}
dla dowolnych $B_{ 1 }, B_{ 2 } \in \Bfrak^{ 0 }$. Tym samym widzimy,
że~warunek (\romannumeral3) implikuje warunek (\romannumeral2).

Jest jednak dość oczywiste, że~w~umyśle autorów to inny wniosek z~zależności
\eqref{eq:Semadeni-Wiweger-Wstep-ETC-05} jest kluczowy. Ten mianowicie, że
\begin{equation}
  \label{eq:Semadeni-Wiweger-Wstep-ETC-07}
  v_{ B_{ 1 }, \, B_{ 2 }, \, B_{ 3 } }^{ \Bfrak }( \alpha, \beta ) =
  \beta \alpha =
  v_{ B_{ 1 }, \, B_{ 2 }, \, B_{ 3 } }^{ \Ufrak }( \alpha, \beta ),
\end{equation}
dla wszystkich $B_{ 1 }, B_{ 2 }, B_{ 3 } \in \Bfrak^{ 0 }$ i~dla wszystkich
$\alpha \in \langle B_{ 1 }, B_{ 2 } \rangle_{ \Bfrak }$, $\beta \in \langle B_{ 2 }, B_{ 3 } \rangle_{ \Bfrak }$.
Inaczej mówiąc, jeśli można złożyć przekształcenia $\alpha$, $\beta$ w~kategorii
$\Bfrak$ to można je też złożyć w~kategorii $\Ufrak$ i~w~obu przypadkach
otrzymujemy jako wynik to samo przekształcenie, które oznaczamy przez
$\beta \alpha$. To kończy naszą dyskusję, jak sens ma wzór
\eqref{eq:Semadeni-Wiweger-Wstep-ETC-02}.

\vspace{\VerSpaceFour}





\noindent
\Str{26} Przyjrzyjmy~się uważniej znajdującemu~się na tej stronie zdaniu
„Widać stąd, że~diagram należy interpretować nie jako zbiór obiektów
i~morfizmów kategorii, lecz jako funkcję ze~schematu do~kategorii.”. Choć na
poziomie intuicyjnym jest dość jasne o~co chodzi, to jednak próby uściślenia
tego stwierdzania napotykają pewne problemy. Jeżeli bowiem diagram jest
pewno funkcją ze schematu do kategorii, to co jest jej wartością? Obiekt
kategorii? Kilka obiektów kategorii? Morfizm~$f$? Zbiór morfizmów
$\langle A, B \rangle_{ \Ufrak }$?

W~tym momencie nie jestem w~stanie udzielić satysfakcjonującej odpowiedzi na
to pytanie, więc pozostaniemy na poziomie intuicyjnego zrozumienia czym jest
diagram. W~większości przypadków powinno być ono całkowicie zadowalające.

\vspace{\VerSpaceFour}





\noindent
\Str{27} W~tym miejscu jest mowa, że~funktor to dowolna funkcja $\Phi$, która
morfizmom kategorii $\Ufrak$ przyporządkowuje morfizmy kategorii~$\Bfrak$,
\begin{equation}
  \label{eq:Semadeni-Wiweger-Wstep-ETC-08}
  \Phi : \Ufrak \to \Bfrak.
\end{equation}
Zwróćmy uwagę, że~według tego co napisano na końcu strony~15 tej książki,
należy~się spodziewać, iż~morfizmy danej kategorii nie tworzą zbioru tylko
klasę. Czy jednak można w~sposób ścisły mówić o~funkcji która ma jako
dziedzinę i~kodziedzinę nie zbiory tylko klasy?

Nie potrafię na~to pytanie odpowiedzieć w~sposób pewny, lecz zgodnie
z~informacjami podanymi na~stronie~19, w~systemie aksjomatycznym G\"{o}dla
można zdefiniować iloczyn kartezjański dwóch i~tym samym zdefiniowania
funkcji między dwoma klasami powinno przebiegać podobnie jak w~standardowym
systemie teorii mnogości. Problem wymaga jednak dalszych badań.




























% ##################
\newpage

\CenterBoldFont{Błędy}


\begin{center}

  \begin{tabular}{|c|c|c|c|c|}
    \hline
    Strona & \multicolumn{2}{c|}{Wiersz} & Jest
                              & Powinno być \\ \cline{2-3}
    & Od góry & Od dołu & & \\
    \hline
    26  & & \hphantom{0}2 & ( B) & (B) \\
    26  & & \hphantom{0}3 & k ategorią & kategorią \\
    26  & & \hphantom{0}4 & czwór ką & czwórką \\
    26  & & \hphantom{0}6 & ka tegorii & kategorii \\
    26  & & \hphantom{0}7 & morfi zmy & morfizmy \\
    26  & & \hphantom{0}8 & kateg orię & kategorię \\
    26  & & 10 & funk torami & funktorami \\
    26  & & 11 & stam y & stamy \\
    26  & & 12 & zawi ła & zawiła \\
    26  & & 13 & Ścisł a & Ścisła \\
    % & & & & \\
    % & & & & \\
    % & & & & \\
    % & & & & \\
    % & & & & \\
    \hline
  \end{tabular}

\end{center}

\vspace{\spaceTwo}




% ############################










% #####################################################################
% #####################################################################
% Bibliography

\bibliographystyle{plalpha}

\bibliography{MathematicsBooks}{}





% ############################

% End of the document
\end{document}
