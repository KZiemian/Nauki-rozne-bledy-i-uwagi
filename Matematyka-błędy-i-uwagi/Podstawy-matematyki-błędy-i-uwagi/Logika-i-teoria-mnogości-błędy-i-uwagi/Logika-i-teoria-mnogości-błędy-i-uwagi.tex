% Autor: Kamil Ziemian

% ---------------------------------------------------------------------
% Basic configuraton of LaTeX document and Polish language
% ---------------------------------------------------------------------
\RequirePackage[l2tabu, orthodox]{nag}  % Find out deprecated part of LaTeX.
% More information in l2tabu English version.


\documentclass[a4paper,11pt]{article}
% {rozmiar papieru, rozmiar fontu}[klasa dokumentu]
\usepackage[utf8]{inputenc} % Włączenie kodowania UTF-8, co daje dostęp
% do polskich znaków.
\usepackage{lmodern} % Wprowadza fonty Latin Modern.
\usepackage[T1]{fontenc} % Potrzebne do używania fontów Latin Modern.

\usepackage[MeX]{polski} % Polonizacja LaTeXa, bez niej będzie pracował
% w języku angielskim.









% ---------------------------------------
% MARGINS
% ---------------------------------------
\usepackage{vmargin}  % Pozwala na prostą kontrolę rozmiaru marginesów,
% za pomocą komend poniżej. Rozmiar odstępów jest mierzony w calach.
\setmarginsrb
{ 0.7in}  % left margin
{ 0.6in}  % top margin
{ 0.7in}  % right margin
{ 0.8in}  % bottom margin
{  20pt}  % head height
{0.25in}  % head sep
{   9pt}  % foot height
{ 0.3in}  % foot sep







% ---------------------------------------
% Basic packages
% ---------------------------------------
\usepackage{microtype}  % Twierdzi, że poprawi rozmiar odstępów w tekście.
% \usepackage{graphicx}  % Wprowadza bardzo potrzebne komendy do wstawiania
% % grafiki.
% \usepackage{verbatim}  % Poprawia otoczenie VERBATIME.
% \usepackage{textcomp}  % Dodaje takie symbole jak stopnie Celsiusa,
% % wprowadzane bezpośrednio w tekście.



% ---------------------------------------
% Often used packages
% ---------------------------------------
\usepackage{xcolor}  % Pozwala używać kolorowych czcionek (zapewne dużo
% więcej, ale ja nie potrafię nic o tym powiedzieć).







% ---------------------------------------
% Packages for scientific writing
% ---------------------------------------
\let\lll\undefined  % Amsmath gryzie się z pakietami do języka
% polskiego, bo oba definiują komendę \lll. Aby rozwiązać ten problem
% oddefiniowuję tę komendę, ale może tym samym pozbywam się dużego Ł.
\usepackage[intlimits]{amsmath}  % Podstawowe wsparcie od American
% Mathematical Society (w skrócie AMS)
\usepackage{amsfonts, amssymb, amscd, amsthm}  % Dalsze wsparcie od AMS
% \usepackage{siunitx}  % Do prostszego pisania jednostek fizycznych
\usepackage{upgreek}  % Ładniejsze greckie litery
% Przykładowa składnia: pi = \uppi
% \usepackage{slashed}  % Pozwala w prosty sposób pisać slash Feynmana.
\usepackage{calrsfs}  % Zmienia czcionkę kaligraficzną w \mathcal
% na ładniejszą. Może w innych miejscach robi to samo, ale o tym nic
% nie wiem.



% Tworzenie otoczeń "Twierdzenie", "Definicja", "Lemat", etc.
\newtheorem{twr}{Twierdzenie}  % Komenda wprowadzająca otoczenie
% „twr” do pisania twierdzeń matematycznych
\newtheorem{defin}{Definicja}  % Analogicznie jak powyżej
\newtheorem{wni}{Wniosek}







% ---------------------------------------
% Various setting for this document
% ---------------------------------------
\renewcommand{\arraystretch}{1.2}  % Ustawienie szerokości odstępów między
% wierszami w tabelach.







% ---------------------------------------
% Packages written by us
% ---------------------------------------
\usepackage{latexgeneralcommands}
% \usepackage{mathshortcuts}







% ---------------------------------------------------------------------
% Additional setting for Polish language
% ---------------------------------------------------------------------
\renewcommand{\thesection}{\arabic{section}.}
% Kropki po numerach rozdziału (polski zwyczaj topograficzny)
\renewcommand{\thesubsection}{\thesection\arabic{subsection}}
% Brak kropki po numerach podrozdziału










% ------------------------------
% Pakiet „hyperref”
% Polecano by umieszczać go na końcu preambuły.
% ------------------------------
\usepackage{hyperref}  % Pozwala tworzyć hiperlinki i zamienia odwołania
% do bibliografii na hiperlinki.










% ---------------------------------------------------------------------
% Tytuł, autor, data
\title{Logika i~teoria mnogości~-- błędy i~uwagi}

% \author{}
% \date{}
% ---------------------------------------------------------------------










% ####################################################################
% Początek dokumentu
\begin{document}
% ####################################################################





% ######################################
\maketitle  % Tytuł całego tekstu
% ######################################





% ######################################
\section{Logika}

\vspace{\spaceTwo}
% ######################################



% ############################
\Work{ % Autor i tytuł dzieła
  Józef W.~Bremer \\
  „Wprowadzenie do~logiki”, \cite{BremerWprowadzenieDoLogiki2004} }


% ##################
\CenterBoldFont{Uwagi}


\start \StrWg{132}{1} Zdanie
„$p \vee q ( ( p \land q ) \to ( p \land q ) )$” nie ma żadnego sensu
logicznego, musiał zostać zgubiony spójnik logiczny po~pierwszym~$q$.
Niestety nie wiem który należy tam umieścić.





% ##################
\CenterBoldFont{Błędy}


\begin{center}

  \begin{tabular}{|c|c|c|c|c|}
    \hline
    & \multicolumn{2}{c|}{} & & \\
    Strona & \multicolumn{2}{c|}{Wiersz} & Jest
                              & Powinno być \\ \cline{2-3}
    & Od góry & Od dołu & & \\
    \hline
    4   &  8 & & LATEX & \LaTeX \\
    13  & &  8 & \textit{formalnej} & \textit{formalnej}; \\
    20  & & 15 & Ockhama”$^{ 7 }$. & Ockhama”$^{ 7 }$, \\
    21  & 14 & & Współczesnych & współczesnych \\
    26  & &  2 & \textit{Lwowsko-\! Warszawska}
           & \textit{Lwowsko-Warszawska} \\
    45  & & 10 & konkretna & konkretną \\
    86  & 10 & & przypadku.. & przypadku. \\
    88  & 11 & & średniego & pośredniego \\  % ???
    98  & &  9 & $P \underline{ e } S$ & $S \underline{ e } P$ \\
    114 &  9 & & zwane\textit{prawo} & zwane \textit{prawo} \\
    117 &  6 & & jedzie & jadą \\
    119 & & 14 & współczesnej~. & współczesnej. \\
    122 & &  4 & wniosek $\equiv P$) & wniosek $\equiv P$)” \\
    125 & & 14 & „nie-analityczne & „nie-analityczne” \\
    131 &  6 & & „$\neg p \vee \neg q''$ & „$\neg p \vee \neg q$” \\
    131 &  6 & & $\equiv$,\hspace{2pt},$\neg ( p \land q )$”
           & $\equiv$ „$\neg ( p \land q )$” \\
    132 &  8 & & $\neg( \neg p\;\;\; \land\; \neg q ) $ & $\neg( \neg p \land \neg q ) $ \\
    132 & 17 & & $p\quad \to \quad q$ & $p \to q$ \\
    132 & & 16 & $p\quad \to \quad q$ & $p \to q$ \\
    % & & & & \\
    % & & & & \\
    % & & & & \\
    % & & & & \\
    % & & & & \\
    \hline
  \end{tabular}

\end{center}


\vspace{\spaceTwo}
% ############################










% ######################################
\newpage
\section{Logika matematyczna}

\vspace{\spaceTwo}
% ######################################



% ############################
\Work{ % Autor i tytuł dzieła
  Andrzej Grzegorczyk \\
  „Zarys logiki matematycznej”,
  \cite{GrzegorczykZarysLogikiMatematycznej1975} }



% ##################
\CenterBoldFont{Uwagi}


\noindent
\Str{16} Na tej stronie Grzegorczyk swoim wyborem nazw dla
własności funkcji stworzył problem, który czyni język jego książki
niejednoznacznym. Zdecydowała się nadać funkcji $f : X \to Y$
różnowartościowej nazwę \textit{funkcji wzajemnie
  jednoznaczną}\footnote{Nie cytuję dosłownie sformułowań z~książki,
  bo wymagają one poprawienia i~ujednoznacznienia.}, a funkcji która
jest różnowartościowa i~spełnia dodatkowo warunek $f( X ) = Y$ nazwę
\textit{funkcji wzajemnie jednoznacznej odwzorowującej zbiór $X$ na cały
  zbiór $Y$}.

Już na stronie następnej stronie (str. 17) stosuje to nazewnictwo
niekonsekwentnie. Pisze, że „relacja $R$ wyznacza odwzorowanie
wzajemnie jednoznaczne zbioru $X$ na zbiór $Y$”, gdzie chodzi mu
o~odwzorowanie $X$ na cały zbiór $Y$, czyli funkcję $f$ taką, że
$f( X ) = Y$. Aby z tego wybrnąć przyjmujemy umowę, że~sformułowania
„na zbiór $Y$” i~„na cały zbiór $Y$” oba oznaczają to sama, czyli że
dane odwzorowanie pokrywa swoimi wartościami cały zbiór $Y$
(równoważnie $f( X ) = Y$). Jeśli $f( X ) \neq Y$ będziemy mówić, że
\textit{funkcja $f$ odwzorowuje zbiór $X$ w~zbiór $Y$}.

Dalej na stronie 17 pisze, że jeśli zawęzimy potęgowanie oraz
pierwiastkowanie do liczb dodatnich, to wówczas obie funkcje są
„wzajemnie jednoznaczne” i~„jedna jest odwrotnością drugiej”. W~tym
miejscu prawie na pewno przez stwierdzenie, że tak określone
potęgowanie i~pierwiastkowanie są funkcjami „wzajemnie jednoznacznymi”
należy rozumieć nie tylko to, że~są one różnowartościowe, ale też
że~odwzorowują zbiór liczby dodatnich na cały zbiór liczb dodatnich.

Dodatkowo na stronie 16 pojawia się pojęcie „relacji jednoznacznej”,
przez co autor rozumie chyba relację która wyznacza pewną funkcję,
czyli spełnia warunki (2) i~(3) ze str.~10. Na stronie 17 jest zaś
mowa o „relacji wzajemnie jednoznacznej” co z kolei oznacza relację
$R$ taką, że zarówno relacja $R$ jak i~relacja do niej odwrotna
wyznaczają funkcje. Jak widać pojęcie jednoznaczności dla funkcji
i~dla relacji oznacza zupełnie coś innego.

Proponuję następujące wyjście z tej sytuacji. Funkcję $f: X \to Y$
będziemy nazywać \textit{jednoznaczną} jeśli jest różnowartościowa.
Funkcję która jest różnowartościowa i~dla której zachodzi
$f( X ) = Y$, będziemy nazywali \textit{wzajemnie jednoznaczną}. Pojęć
„relacja jednoznaczna” i~„relacja wzajemnie jednoznaczna” będziemy
unikać.

Dalsza lektura pokaże, czy książka pozwala na stosowanie tej
terminologi, czy też okaże~się to zbyt skomplikowane w~praktyce. Do
tego jednak momentu będziemy poprawiać jej fragmenty w~zgodzie z~tym
co zostało powyżej napisane.

\vspace{\spaceFour}





% ##################
\CenterBoldFont{Błędy}


\begin{center}

  \begin{tabular}{|c|c|c|c|c|}
    \hline
    & \multicolumn{2}{c|}{} & & \\
    Strona & \multicolumn{2}{c|}{Wiersz} & Jest
                              & Powinno być \\ \cline{2-3}
    & Od góry & Od dołu & & \\
    \hline
    12  &  3 & & \emph{ciałem} & \emph{ciałem uporządkowanym} \\
    16  & 12 & & str. 12 & str. 10 \\
    18  & 20 & & (dla & \big(dla \\
    20  & 21 & & \big($f \in Y$)\big) & \big($f \in Y$\big)\big) \\
    20  & &  6 & liczby{ } $\neq 0$ & liczby $\neq 0$ \\
    22  & & 11 & zwrotność & samozwrotność \\
    22  & &  3 & wzajemnej jednoznaczności & jednoznaczności \\
    22  & &  2 & przeciwobraz & przeciwobraz obrazu \\
    25  &  4 & & (21); & (21)); \\
    29  & & 13 & $\forall x W$ & $\forall x \exists W$ \\
    34  &  1 & & (23) & (23') \\
    34  & 20 & & $\forall u$ & $x \in \bigcap X \equiv \forall u$ \\
    % & & & & \\
    % & & & & \\
    % & & & & \\
    % & & & & \\
    % & & & & \\
    % & & & & \\
    \hline
  \end{tabular}

\end{center}

\vspace{\spaceTwo}
% ############################










% ############################
\Work{ % Autor i tytuł dzieła
  Willard Van Orman Quine \\
  „Logika matematyczna”, \cite{QuineLogikaMatematyczna1974} }


% ##################
\CenterBoldFont{Błędy}


\begin{center}

  \begin{tabular}{|c|c|c|c|c|}
    \hline
    & \multicolumn{2}{c|}{} & & \\
    Strona & \multicolumn{2}{c|}{Wiersz} & Jest
                              & Powinno być \\ \cline{2-3}
    & Od góry & Od dołu & & \\
    \hline
    % & & & & \\
    19  & &  2 & ęzyka & języka \\
    21  & & 10 & o fałszywych poprzednikach i & o \\
    22  & 18 & & „jeżeli --- to --- & „jeżeli --- to ---” \\
    27  &  5 & & „jeżeli --- to” & „jeżeli --- to ---” \\
    % & & & & \\
    % & & & & \\
    \hline
  \end{tabular}

\end{center}

\vspace{\spaceTwo}
% ############################










% ######################################
\newpage
\section{Teoria mnogości}

\vspace{\spaceTwo}
% ######################################



% ############################
\Work{ % Autor i tytuł dzieła
  Kazimierz Kuratowski \\
  „Wstęp do~teorii mnogości i~topologii”,
  \cite{KuratowskiWstepTeoriiMnogosciITopologii2004} }


% ##################
\CenterBoldFont{Uwagi}


\start \Str{10} Należy uwzględnić dwa typy dowodu używające negacji.
Pierwszy to~dowód przez \textbf{kontrapozycję}, opierający~się
na~tożsamości
\begin{equation}
  \label{eq:KuratowskiWTMiT-01}
  ( \alpha \Rightarrow \beta ) \equiv ( \neg \beta \Rightarrow \neg \alpha).
\end{equation}
Drugi to~dowód \textbf{nie wprost}, opierający~się na~tożsamości
\begin{equation}
  \label{eq:KuratowskiWTMiT-02}
  ( \alpha \Rightarrow \beta ) \equiv \neg ( \alpha \land \neg \beta ).
\end{equation}

\vspace{\spaceFour}





% ##################
\CenterBoldFont{Błędy}


\begin{center}

  \begin{tabular}{|c|c|c|c|c|}
    \hline
    & \multicolumn{2}{c|}{} & & \\
    Strona & \multicolumn{2}{c|}{Wiersz} & Jest
                              & Powinno być \\ \cline{2-3}
    & Od góry & Od dołu & & \\
    \hline
    11 &  3 & & \textbf{7.} & \textbf{7a.} \\
    11 &  4 & & \textbf{7a.} & \textbf{7b.} \\
    % & & & & \\
    % & & & & \\
    % & & & & \\
    % & & & & \\
    % & & & & \\
    % & & & & \\
    % & & & & \\
    % & & & & \\
    % & & & & \\
    % & & & & \\
    % & & & & \\
    % & & & & \\
    % & & & & \\
    % & & & & \\
    % & & & & \\
    % & & & & \\
    % & & & & \\
    % & & & & \\
    % & & & & \\
    % & & & & \\
    % & & & & \\
    % & & & & \\
    % & & & & \\
    % & & & & \\
    % & & & & \\
    % & & & & \\
    % & & & & \\
    % & & & & \\
    % & & & & \\
    % & & & & \\
    % & & & & \\
    % & & & & \\
    % & & & & \\
    % & & & & \\
    % & & & & \\
    % & & & & \\
    \hline
  \end{tabular}

\end{center}

\vspace{\spaceTwo}
% ############################










% ####################################################################
% ####################################################################
% Bibliografia
\bibliographystyle{plalpha}

\bibliography{MathComScienceBooks}{}





% ############################

% Koniec dokumentu
\end{document}
