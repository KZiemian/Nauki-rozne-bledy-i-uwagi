% ---------------------------------------------------------------------
% Podstawowe ustawienia i pakiety
% ---------------------------------------------------------------------
\RequirePackage[l2tabu, orthodox]{nag} % Wykrywa przestarzałe i niewłaściwe
% sposoby używania LaTeXa. Więcej jest w l2tabu English version.
\documentclass[a4paper,11pt]{article}
% {rozmiar papieru, rozmiar fontu}[klasa dokumentu]
\usepackage[MeX]{polski} % Polonizacja LaTeXa, bez niej będzie pracował
% w języku angielskim.
\usepackage[utf8]{inputenc} % Włączenie kodowania UTF-8, co daje dostęp
% do polskich znaków.

% Umożliwia korzystanie z fontów Latin Modern.
\usepackage{lmodern}
\usepackage[T1]{fontenc} % Potrzebne do używania fontów Latin Modern.



% ------------------------------
% Podstawowe pakiety (niezwiązane z ustawieniami języka)
% ------------------------------
\usepackage{microtype} % Twierdzi, że poprawi rozmiar odstępów w tekście.
\usepackage{graphicx} % Wprowadza bardzo potrzebne komendy do wstawiania
% grafiki.
\usepackage{verbatim} % Poprawia otoczenie VERBATIME.
\usepackage{textcomp} % Dodaje takie symbole jak stopnie Celsiusa,
% wprowadzane bezpośrednio w tekście.
% `vmargin' -- paczka pozwalająca na prostą kontrolę marginesów w pliku
% wyjściowym, za pomocą komend widniejących poniżej. Wielkość marginesu
% jest mierzona w calach (cal ~ 2.54 cm).
\usepackage{vmargin}
% ------------------------------
% MARGINS
% ------------------------------
\setmarginsrb
{ 0.7in}  % left margin
{ 0.6in}  % top margin
{ 0.7in}  % right margin
{ 0.8in}  % bottom margin
{  20pt}  % head height
{0.25in}  % head sep
{   9pt}  % foot height
{ 0.3in}  % foot sep



% ------------------------------
% Często przydatne pakiety
% ------------------------------
% \usepackage{csquotes} % Pozwala w prosty sposób wstawiać cytaty do tekstu.
\usepackage{xcolor} % Pozwala używać kolorowych czcionek (zapewne dużo
% więcej, ale ja nie potrafię nic o tym powiedzieć).



% ------------------------------
% Pakiety do tekstów z nauk przyrodniczych
% ------------------------------
\let\lll\undefined % Amsmath gryzie się z językiem pakietami do języka
% polskiego, bo oba definiują komendę \lll. Aby rozwiązać ten problem
% oddefiniowuję tę komendę, ale może tym samym pozbywam się dużego Ł.
\usepackage[intlimits]{amsmath} % Podstawowe wsparcie od American
% Mathematical Society (w skrócie AMS)
\usepackage{amsfonts, amssymb, amscd, amsthm} % Dalsze wsparcie od AMS
% \usepackage{siunitx} % Dla prostszego pisania jednostek fizycznych
\usepackage{upgreek} % Ładniejsze greckie litery
% Przykładowa składnia: pi = \uppi
\usepackage{slashed} % Pozwala w prosty sposób pisać slash Feynmana.
% \usepackage{calrsfs} % Zmienia czcionkę kaligraficzną w \mathcal
% na ładniejszą. Może w innych miejscach robi to samo, ale o tym nic
% nie wiem.



% ------------------------------
% Tworzenie środowisk (?) „Twierdzenie”, „Definicja”, „Lemat”, etc.
% ------------------------------
% Komenda wprowadzająca otoczenie „theorem” do pisania twierdzeń
% matematycznych.
\newtheorem{theorem}{Twierdzenie}
% Analogicznie jak powyżej
\newtheorem{definition}{Definicja}
\newtheorem{corollary}{Wniosek}



% ---------------------------------------
% Pakiety napisane przez użytkownika.
% Mają być w tym samym katalogu to ten plik .tex
% ---------------------------------------
\usepackage{latexgeneralcommands}
\usepackage{mathcommands}
\newcommand{\conca}{\textrm{conca}}




% ---------------------------------------------------------------------
% Dodatkowe ustawienia dla języka polskiego
% ---------------------------------------------------------------------
\renewcommand{\thesection}{\arabic{section}.}
% Kropki po numerach rozdziału (polski zwyczaj topograficzny)
\renewcommand{\thesubsection}{\thesection\arabic{subsection}}
% Brak kropki po numerach podrozdziału



% ------------------------------
% Ustawienia różnych parametrów tekstu
% ------------------------------
\renewcommand{\baselinestretch}{1.1}

% Ustawienie szerokości odstępów między wierszami w tabelach.
\renewcommand{\arraystretch}{1.4}





% ------------------------------
% Pakiet „hyperref”
% Polecano by umieszczać go na końcu preambuły.
% ------------------------------
\usepackage{hyperref} % Pozwala tworzyć hiperlinki i zamienia odwołania
% do bibliografii na hiperlinki.










% ---------------------------------------------------------------------
% Tytuł i autor tekstu
\title{Teoria automatów i~języków formalnych \\
  {\Large Błędy i~uwagi}}

\author{Kamil Ziemian}


% \date{}
% ---------------------------------------------------------------------










% ####################################################################
\begin{document}
% ####################################################################





% ######################################
\maketitle % Tytuł całego tekstu
% ######################################





% ############################
\Work{ % Autorzy i tytuł dzieła
  Maria Foryś, Wit Foryś \\
  \textit{Teoria automatów i~języków formalnych},
  \cite{ForysForysTeoriaAutomatowIJezykowFormalnych2005}}

\vspace{0em}


% ##################
\CenterBoldFont{Uwagi}

\vspace{0em}


\noindent
Zawartość tej książeczki jest pod pewnym względami większa, pod
innymi mniejsza, niż kursu online \textit{Języki, automaty
  i~obliczania}, autorstwa, notabene, Marii Foryś, Wita Forysia
i~Adama Romana. Warto więc przerabiać go równolegle z~tą pozycją. Kurs
ten można znaleźć pod tym
\colorhref{https://wazniak.mimuw.edu.pl/index.php?title=J\%C4\%99zyki,\_automaty\_i\_obliczenia}{linkiem}\footnote{Pełna
  forma linku:
  \href{https://wazniak.mimuw.edu.pl/index.php?title=J\%C4\%99zyki,\_automaty\_i\_obliczenia}
  {https://wazniak.mimuw.edu.pl/index.php?title=J\%C4\%99zyki,\_automaty\_i\_obliczenia}.}.

\vspace{\spaceFour}





\noindent
W~tej książce używany jest cudzysłów w~formie ”cytowany tekst”,
ale polskie standardy typograficzna mówią, że~powinno~się stosować
formę: „cytowany tekst”. Dodatkowo, wyjątek warto uczynić dla ciągów
symboli (liter z~danego alfabetu $A$), nazywanych napisami bądź
stringami (ang. \textit{strings}), które zgodnie z~przyjętą
w~informatyce konwencją będziemy oznaczać jako
$\texttt{"} abc \texttt{"}$.

Więcej o~napisach powiemy przy okazji omawiania wolnych monoidów
słownych.

\vspace{\spaceFour}





\noindent
Zapisywanie dowodów twierdzeń za pomocą kursywy, jak to~się
robi w~tej książeczce, naprawdę nie jest dobrym pomysłem. Dlatego
w~tych notatkach jeśli będę cytował fragment treści dowodu, to będzie
on pisany kursywą, ale poprawka do niego już czcionką
prostą\footnote{Należy te notatki przejrzeć i~zobaczyć, czy wszędzie
  ta konwencja jest przestrzega.}.

\vspace{\spaceFour}





% ##################
\CenterBoldFont{Uwagi do~konkretnych stron}


\noindent
\Str{5} Warto zauważyć, że~ze względu na łączność działania
w~monoidzie $M$, łączne jest również działanie w~monoidzie
$\Pcal( M )$.
\begin{equation}
  \label{eq:Forys-Forys-01}
  ( A \cdot B ) \cdot C = A \cdot ( B \cdot C ), \quad
  A, B, C \in \Pcal( M ).
\end{equation}

\vspace{\spaceFour}





\noindent
\Str{5} Czy podana tu definicja mnożenia w~monoidzie
$\Pcal( M )$, gdzie $M$ jest dowolnym monoidem, obejmuje też
przypadek, gdy jeden ze zbiorów jest pusty? Wydaje mi~się, że~tak, ale
całkowitej pewności nie mam. Niezależnie od tego, jest jasne jaki ma
być wynik tej operacji. Mianowicie, dla dowolnego $A \subset M$ ma zachodzić
\begin{equation}
  \label{eq:Forys-Forys-02}
  \emptyset \cdot A = A \cdot \emptyset := \emptyset.
\end{equation}

\vspace{\spaceFour}





\noindent
\Str{6} Według mnie symbol „$\textrm{mod}_{ 6 }$” wygląda
znacznie lepiej niż używany w tej książce „$mod_{ 6 }$”.

\vspace{\spaceFour}





\noindent
\Str{6} Na tej stronie znajdujemy informację, że~element
$x \in M$ jest odwracalny w~monoidzie $M$ wtedy i tylko wtedy, gdy
istnieje takie $y \in M$, że $x \cdot y = 1_{ M }$, zaś zbiór taki
elementów tworzy grupę, zwaną grupą jedynki. Jednak monoid nie musi
być przemienny, więc nie widzę powodu by zachodziła też równość
$y \cdot x = 1_{ M }$, co~stawia pod znakiem zapytania to, czy zbiór
takich elementów tworzy grupę.

Co więcej, wolne monoidy językowe, które zdefiniujemy dalej, a~które
są dla nas szczególnie interesujące, nie są przemienne, więc problem
jest tym poważniejszy.

Wydaje mi, że jest to zwykłe przeoczenie ze strony autorów i~trzeba po
prostu przyjąć, że~element $x \in M$ nazywamy odwracalny, gdy istnieje
taki element $y \in M$, że~zachodzi
\begin{equation}
  \label{eq:Forys-Forys-03}
  x \cdot y = y \cdot x = 1_{ M }.
\end{equation}

\vspace{\spaceFour}





\noindent
\Str{6} Na tej stronie znajdujemy stwierdzenia, że „zbiór
elementów odwracalnych monoidu $M$ stanowi podgrupę tego monoidu”.
Zwykle przez podgrupę rozumie~się podzbiór $G_{ 1 }$ grupy $G_{ 0 }$,
taki że~wraz z~działaniem\footnote{Używając terminologi z~książki
  Białynickiego-Biruli, mówilibyśmy o~trzech różnych działaniach:
  dwuargumentowym działaniu wewnętrznym, jednoargumentowym działaniu
  brania elementu odwrotnego i~zeroargumentowym działaniu zwracającym
  element neutralny. Zob. \cite{BialynickiBirulaZarysAlgebry1987}.}
odziedziczonym z~$G_{ 0 }$ jest on grupą. Tutaj jednak jest mowa,
że~grupa jedynki, oznaczmy ją $U$ (od ang. \textit{unity}), jest
podgrupą monoidu, który to monoid nie musi być grupą.

Sens tego zwrotu jest dość oczywisty. Mianowicie, że trójka
uporządkowana $( U, \cdot, 1 )$, gdzie $\cdot$ jest działaniem
odziedziczonym po monoidzie $M$, zaś $1 \in U$, jest elementem
neutralnym tego monoidu, jest grupą. Jednak takie rozciąganie
standardowego rozumienia pojęcia podgrupy czyni cały system pojęciowy
trochę mniej ścisłym.

Możemy doprecyzować ten formalizm za pomocą pojęcia algebry ogólnej
(zob. \cite{BialynickiBirulaZarysAlgebry1987}) i~wydaje~się,
że~w~ostatecznym rozrachunku jest to najbardziej ekonomiczne
podejście.

Niech $( A, \varphi_{ 1 }, \varphi_{ 2 }, \ldots, \varphi_{ n } )$
będzie algebrą ogólną typu $\alpha$, np. monoidem, i~niech
$B \subset A$. Jeżeli system
$( B, \varphi_{ 1 }|_{ B }, \varphi_{ 2 }|_{ B }, \ldots, \varphi_{ n
}|_{ B } )$ jest algebrą ogólną typu $\beta$, to mówimy, że~$B$ jest
podalgebrą ogólną $A$ typu $\beta$. Przykładowo, jeśli
$( M, \cdot, 1_{ M } )$ jest monoidem, a~$( G, \cdot, 1_{ M } )$,
gdzie $G \subset M$, jest grupą, to mówimy, że~$G$ jest podgrupą
monoidu $M$. Tak jak zazwyczaj, będziemy używali też bardziej
zwięzłych wyrażeń językowych jak „$G$ jest podgrupą $M$”, nie
stwierdzając jawnie, że~$M$ jest monoidem.

Idea jaka stoi za tymi pojęciami jest jasna. Po zawężeniu działań
wewnętrznych danej algebry ogólnej
$( A, \varphi_{ 1 }, \varphi_{ 2 }, \ldots, \varphi_{ n } )$ do
mniejszego zbioru $B \subset A$, mogą się zmienić własności tych
działań i~tym samym typ algebry. Podzbiór $\Fbb$ pierścienia $\Pcal$
może okazać~się ciałem, bo wszystkie elementy nieodwracalne,
z~wyjątkiem zera, należą do zbioru $\Pcal \setminus \Fbb$.

Wszystkie te rozważania moglibyśmy ująć w~sposób bardziej
sformalizowany, jednak wydaje~się, że~obecna ich forma jest
wystarczająco precyzyjna.

\vspace{\spaceFour}





\noindent
\Str{6} W~dowodzie twierdzenia 1.1.1 brakuje mi zdania typu
„Jeśli $S$ nie jest monoidem, to rozszerzamy go do monoidu, za pomocą
procedury którą zaprezentujemy poniżej.”.

\vspace{\spaceFour}





\noindent
\Str{6} Dowód twierdzenia 1.1.1 zawiera dodatkową informację,
nad którą warto~się zatrzymać. Mianowicie że~każdą półgrupę można
rozszerzyć do monoidu. Wymaga to jedynie
oczywistego/nieoczywistego\footnote{Zależy to od przekonań
  filozoficzno-matematycznych konkretnej osoby.} założenia,
że~istnieje element $1 \notin S$.

W~kontekście samego dowodu twierdzenia 1.1.1, warto~się zastanowić nad
tym dlaczego dokonujemy rozszerzenia półgrupy $( S, \cdot )$ do
monoidu $( S^{ 1 }, \cdot, 1 )$. Dzięki temu mamy rodzinę odwzorowań
$\rho_{ a }$, przy czym dla każdego ustalonego $a$ zachodzi
$\rho_{ a } : S^{ 1 } \to S^{ 1 }$, a~ponieważ $S^{ 1 }$ posiada
element neutralny, to otrzymujemy $\rho_{ a }( 1 ) = a$. Ta równość
pozwala pokazać, że~z~$\rho_{ a_{ 1 } } = \rho_{ a_{ 2 } }$ wynika
$a_{ 1 } = a_{ 2 }$. Z tego zaś od razu wynika, że odwzorowanie
$h : S \to ( \{ \rho_{ a } \}_{ a \in S } )$, $h( a ) = \rho_{ a }$
jest iniektywne. Poza tym rozszerzenie półgrupy do monoidu nie wydaje
się nigdzie indziej potrzebne.

Pytanie, czy istnieje sposób pokazania, że~odwzorowanie $h$ jest
iniektywne, bez rozszerzania półgrupy $S$ do monoidu $S^{ 1 }$? Nawet
jeżeli tak, to prostota przeprowadzonego w~książce dowodu sprawia, że
warto przy nim pozostać.

\vspace{\spaceFour}





\noindent
\Str{7} Tak jak w~przypadku symbolu $\textrm{mod}_{ 6 }$,
wydaje mi~się, że~lepiej wyglądałby symbol $\textrm{Ker}_{ h }$.

\vspace{\spaceFour}





\noindent
\StrWd{7}{15} Użyty w~podanej tu formule $\rho \subset S^{ 2 }$
symbol $S^{ 2 }$, jest niejednoznaczny. W~tym kontekście $S^{ 2 }$ nie
oznacza $S \cdot S$ w~sensie mnożenia w~półgrupie $\Pcal( S )$, ale
iloczyn kartezjański $S$ z~samym sobą: $S^{ 2 } = S \times S$.

\vspace{\spaceFour}





\noindent
\Str{8} Rysunki na tej stronie są zrobione dość niechlujnie.
Wystarczy zwrócić uwagę, że~ten sam symbol $h$ znajduje~się na nich
w~różnych odległości od lewego marginesu, podczas, gdy powinien być
zawsze w~tej samej odległości.

\vspace{\spaceFour}





\Str{8} Wedle podanej tu definicji wolny monoid $M$ nad
alfabetem $A$, który będziemy oznaczać również symbolem $A^{ * }$,
jest obiektem o~pewnej bardzo konkretnej strukturze ontologicznej.
Sprawa ta prowadzi do pewnych niejasności w~dalszym wykładzie
materiału.

Mianowicie bierzemy pewien zbiór $A$, którego elementy noszą nazwę
„liter” lub „symboli” (patrz jedna z~uwag do str.~9) następnie
tworzymy zbiór wszystkich ciągów tych symboli, który będziemy oznaczać
symbolem $A^{ * }$.
\begin{equation}
  \label{eq:Forys-Forys-04}
  A^{ * } =
  \{ ( a_{ 1 }, a_{ 2 }, a_{ 3 }, \ldots, a_{ n } ) \;\; : \;\; n \geq 0, \;
  a_{ i } \in A \}.
\end{equation}
Wedle tej definicji do $A^{ * }$ należy ciąg pusty, oznaczany symbolem
$1 \equiv \texttt{""}$. Taki ciąg intuicyjnie jest prosty do pojęcia,
zaś~bardziej formalnie jest to odwzorowanie $f : \emptyset \to A$.
W~zbiorze $A^{ * }$ wprowadzamy też działanie konkatenacjami oznaczane
$\cdot$ wzorem
\begin{equation}
  \label{eq:Froys-Forys-05}
  ( a_{ 1 }, a_{ 2 }, \ldots, a_{ n } ) \cdot ( b_{ 1 }, b_{ 2 }, \ldots, b_{ m } ) =
  ( a_{ 1 }, a_{ 2 }, \ldots, a_{ n }, b_{ 1 }, b_{ 2 }, \ldots, b_{ m } ).
\end{equation}

W~dalszym ciągu książki dowiadujemy~się, że klasycznym monoidem wolnym
jest $( \Nbb_{ 0 }, +, 0 )$. Powstaje teraz pytanie, czy aby móc
powiedzieć, że~zbiór liczb naturalnych z~$0$ jest monoidem wolnym, nie
musimy go przedstawić jako zbioru ciągów jakiś symboli, tak jak jest
to przedstawione w~powyższej definicji? Takie podejście rodziłoby
jednak pewne problemy.

Zauważmy, że~możemy wprawdzie przyjąć\footnote{Nie będziemy tu
  poruszać niezwykle interesującego i~ważnego zagadnienie, tego
  mianowicie, że~w~pewnych teoriach typów „obiekty izomorficzne są
  identyczne” i~tym samym nie ma dwóch różnych „modeli liczb
  naturalnych” z~zerem czy bez. Zawędrowalibyśmy tym sposobem zbyt
  daleko od właściwego tematu.}, że~zbiór $A = \{ | \}$, liczba $0$ to
pusty ciąg, $1 = ( | )$, $2 = ( |, | )$, $3 = ( |, |, | )$, etc.,
jednak podstawowa intuicja matematyczna mówi, że~wprowadzanie takiej
konstrukcji nie powinno być konieczne. Zanim przejdziemy do dyskusji
tego zagadnienia, zauważmy, że~w tak podanym wolnym monoidzie
$A^{ * }$ konkatenacja ciągów rzeczywiście prowadzi do tego co
intuicyjnie rozpoznajemy jako dodawanie liczb naturalnych. Przykładowo
\begin{equation}
  \label{eq:Forys-Forys-06}
  1 + 2 = ( | ) \cdot ( |, | ) = ( |, |, | ) = 3.
\end{equation}

Wróćmy teraz do problemu, czemu takie podejście do liczby naturalnych
jawi~się intuicji matematycznej jako niekonieczne. Powinno być bowiem
możliwe myślenie o~liczbach naturalnych jako „po prostu o~liczbach”,
jako autonomicznych bytach, nie zaś jako ciągach jakiegoś symbolu.
Dodatkowo, taka definicji to błędne koło, samo bowiem pojęcie ciągu
o~skończonej długości wymaga uznania, że~wiemy czym jest zbiór liczb
$\{ 1, 2, \ldots, n \}$. Skończony ciąg symboli to bowiem nic innego
jak odwzorowanie $f : \{ 1, 2, \ldots, n \} \to A$. Z~tego powodu
potrzebujemy definicji liczb naturalnych, która nie bazuje na pojęciu
ciągu, jedną z~nich jest ta, która korzysta z~pojęcia zbioru pustego.

Rozważmy teraz inny przykład. Weźmy zbiór wszystkich macierzy kwadratowych
wymiaru $n$, o~wyrazach rzeczywistych\footnote{Typ wyrazów danej macierzy
  nie ma wielkiego znaczenia dla naszych rozważań. Przyjęliśmy, że~są to
  liczby rzeczywiste głównie dla ustalenia uwagi.}, który będziemy oznaczać
symbolem $\GL( m, \Rbb )$, możemy rozpatrywać jako monoid
$( \GL( m, \Rbb ), \cdot, \matUnit )$, gdzie $\cdot$ to mnożenie macierzy, zaś
$\matUnit$ to macierz jednostkowa $m \times m$. Czy jest więc w~ogóle sens
w~pytaniu~się, czy ten monoid zawiera jakiś podmonoid wolny, rozumiany
jako zbiór ciągów z~działaniem konkatenacji? Wszak mając dwie macierze
$A$ i~$B$ należące do $\GL( m, \Rbb )$ musielibyśmy z~jednej strony rozważać
ich iloczyn $A \cdot B$, z~drugiej ciąg dwuelementowy $( A, B )$, które
jawią~się jako dwa zupełnie odmienne byty.

Nie jest powiedziane, że~nie można utożsamić konkretnej macierzy
kwadratowej z~ciągiem elementów jakiegoś zbioru macierzy, acz taka
konstrukcja wydaje~się znacznie mniej naturalna, niż ta przedstawiona
powyżej dla liczb naturalnych.

W~świetle tego proponuję następujące wyjście z~sytuacji.
\textbf{Wolnym monoidem słownym $A^{ * }$ nad alfabetem $A$} będziemy
nazywać wcześniej określony zbiór ciągów wraz z~działaniem
konkatenacji. Jest więc to obiekt o~konkretnej strukturze
ontologicznej.

Poza tym terminologię ze~stron 8-9 opisującą wolne monoidy
pozostawiamy prawienie niezmienioną i~będziemy ją stosować również
do~wolnych monoidów słownych. Będziemy więc mówić o~„bazie (alfabecie)
$A$ wolnego monoidu słownego $A^{ * }$”, elementy zbioru $A$ będziemy
nazywać „literami”, bądź jak zostało to wyjaśnione w~jednym z~dalszych
komentarzy, „symbolami”. Ciągi liter (symboli) będziemy nazywali
„słowami”, a~dowolny podzbiór monoidu $A^{ * }$ będziemy nazywali
„językiem”. W~razie potrzeby będziemy do~tego zbioru dodawać
odpowiednie synonimy wszystkich tych pojęć.

Wybiegając trochę w~przyszłość, zdefiniujemy teraz monoid wolny,
korzystając z~pojęcia kanonicznego homomorfizmu. Monoid
$( M, \cdot, 1_{ M } )$ jest \textbf{monoidem wolnym}, jeśli jest
istnieje taki zbiór $A \subset M$, że~wolny monoid słowny $A^{ * }$
nad alfabetem $A$ jest kanonicznie izomorficzny z~$M$. Nie widać więc
powodu by monoid $M$ miał bardziej ustaloną strukturę ontologiczną,
niż np. grupa abelowa.

Aby uczynić rozważania łatwiejszymi w~zrozumieniu, wolnego monoidu
słownego będziemy często zapisywać jako ciągi symboli wewnątrz
cudzysłowu \texttt{" "}. Przykładowo $\texttt{"} abc \texttt{"}$. Mamy
nadzieję, że~pomoże to uniknąć pewnych nieporozumień.

Analogiczna terminologia do tej omawianej na stronach 8-9 tej
książeczki i~powyżej, stosuje~się też do półgrup. Mamy więc
\textbf{wolne półgrupy słowne nad alfabetem $A$} i~\textbf{wolne
  półgrupy}, ponieważ jednak te definicje oraz własności tych obiektów
są bardzo podobne do omówionych powyżej własności monoidów, nie
będziemy ich dokładnie opisywać.

Równoważnym sposobem wysłowienia definicji wolnego monoidu słownego
jest powiedzenie, że~jeśli dany monoid $( M, \cdot, 1_{ M } )$ jest
monoidem wolnym, to istnieje taki zbiór $B$, że~każdy element
$m \in M$, $m \neq 1_{ M }$ można przedstawić w~jeden i~tylko jeden
sposób jako
\begin{equation}
  \label{eq:Forys-Forys-07}
  b_{ 1 } \cdot b_{ 2 } \cdot \ldots \cdot b_{ n }, \quad
  b_{ i } \in B, n > 0.
\end{equation}
Zauważmy, że~w~powyższym wyrażeniu może zachodzić $b_{ i } = b_{ j }$
dla $i \neq j$. W~szczególności oznacza to, że~$B$ jest bazą wolnego
monoidu~$M$.

Ponieważ podane wyżej sformułowanie jest równoważne definicji wolnego
monoidu używającej pojęcia kanonicznego homomorfizmu~$h$, mogliśmy
przyjąć je za definicję tego obiektu. Nie poszliśmy tą drogą, bo
precyzyjne zapisanie w~ten sposób definicji wolnego monoidu, wymagało
wprowadzenia uprzednio tylu pojęć, że~wydaje~się, iż taki wywód nie
byłby bardziej przejrzysty od tego, a~wiele pojęć tu podanych wciąż
wymagałoby późniejszego wprowadzenia.

\vspace{\spaceFour}





\Str{8} Korzystając z~wprowadzonych wcześniej pojęć, możemy
teraz zająć~się niejasnościami, jakie wynikają, z~użycia w~tej książce
symbolu $A^{ * }$ w~dwóch różnych znaczeniach. Jeżeli bowiem
$A \subset M$, gdzie $M$ jest pewnym monoidem, to symbol $A^{ * }$
może oznaczać zarówno wolny monoid słowny nad alfabetem $A$, jak
i~podmonoid monoidu $M$ generowany przez $A$, a~są to dwa różne byty.

Jeśli przyjmiemy, że~$M$ to omawiany już wcześniej monoid macierzy
kwadratowych o~wyrazach rzeczywistych $( \GL( m, \Rbb ), \cdot, \matUnit )$,
zaś $A = \{ B_{ 1 }, B_{ 2 }, B_{ 3 }, \ldots, B_{ n } \}$, to symbol
$A^{ * }$ możne oznaczać zbiór ciągów macierzy kwadratowych,
np.~$( B_{ 1 }, B_{ 2 }, B_{ 3 } )$, jak i~zbiór macierzy kwadratowych
utworzonych jako iloczyny macierzowy macierzy ze zbioru $A$. W~tym
drugim przypadku do tego zbioru należy choćby macierz
$B_{ 1 } \cdot B_{ 2 } \cdot B_{ 3 }$.

By usunąć tego typu dwuznaczności przyjmiemy następującą konwencję.
Jeżeli jest jawnie powiedziane, że~$A$ jest podzbiorem monoidu $M$, to
przez symbol $A^{ * }$ będziemy rozumieli najmniejszy podmonoid $M$
zawierający $A$. Taki obiekt będziemy nazywali \textbf{podmonoidem
  algebraicznym}. Jeżeli symbol ten będzie oznaczać wolny monoid słowny,
zostanie to jawnie zaznaczone w~tekście.

Jeżeli zbiór $A$ jest podany, ale bez wskazania, że jest to podzbiór
pewnego monoidu $M$, to w~większość przypadków przez symbol $A^{ * }$
będziemy rozumieć wolny monoid słowny nad alfabetem $A$. Niestety sposób
w~jaki ta książeczka używa tej notacji, nie pozwala nam przyjąć reguły,
że~$A^{ * }$ oznacza wolny monoid słowny, zawsze gdy nie jest powiedziane,
iż~$A$ jest podzbiorem pewnego monoidu. Aby więc uniknąć nieporozumień,
będziemy~się starali zawsze zaznaczyć jawnie, że~chodzi nam o~wolny monoid
słowny.

W~przypadku gdy~$A$ jest podzbiorem monoidu $M$, ale~rozważać będziemy
wolny monoid słowny nad $A$, zostanie to zaznaczone jawnie. Szczególny
przypadkiem, acz nietrudnym do zrozumienia, jest ten, gdy $A \subset M$ i~$M$
jest monoidem słownym. Wówczas obie definicje symbolu $A^{ * }$~się
pokrywają.

W~świetle powyższych ustaleń wprowadzimy odpowiednie poprawki do tej
książeczki.

\vspace{\spaceFour}





\Str{8} Dysponując już tak uporządkowanym słownictwem,
przeanalizujmy strukturę wolnego monoidu słownego $A^{ * }$ na
alfabetem $A$. Jeśli potrzebny będzie nam przykład wolnego monoidu
słownego, sięgniemy po niezwykle ważny w~informatyce wolny monoid
słowny $\{ 0, 1 \}^{ * }$. Łatwo spostrzec, że~jest to monoid
nieprzemienny, przykładowo jeśli $u = 00$, $v = 11$, to
\begin{equation}
  \label{eq:Forys-Forys-08}
  u v = 0011 \neq 1100 = vu.
\end{equation}
Nasuwa~się naturalne pytanie: co można powiedzieć o~zbiorze $C$ takim,
że dla wszystkich $u, v \in C$ zachodzi
\begin{equation}
  \label{eq:Forys-Forys-09}
  uv = vu.
\end{equation}
Problem ten wymaga dalszej analizy.

Zauważmy, że~dla monoidu słownego zachodzi równość
\begin{equation}
  \label{eq:Forys-Forys-10}
  | u v | = | u | + | v |.
\end{equation}
Wynika ona w~prosty sposób z~definicji konkatenacji. Z~tej zależności
i~z~tego, że~$| u | \geq 0, u \in M$, wynika, że grupa jedynki
w~wolnym monoidzie słownym jest trywialna. Choć wniosek ten jest
prosty do zrozumienia, bardziej sformalizowany dowód wymagałby użycia
nierówności
\begin{equation}
  \label{eq:Forys-Forys-11}
  | u v | \geq | u |,
\end{equation}
które wynika wprost z~dwóch wspomnianych własności.

\vspace{\spaceFour}





\Str{8} Zwróćmy uwagę, że~jeśli $A \subset M$, gdzie $M$ jest
monoidem, to podmonoid algebraiczny $A^{ * }$ i~wolnym monoid słowny
$A^{ * }$ i~alfabetem $A$ mogą być radykalnie różnymi obiektami.

Dla przykładu weźmy monoid $( \Nbb_{ 0 }, +, 0)$ i~niech
$A = \{ 0 \}$. Wówczas jeżeli przez symbol $A^{ * }$ rozumiemy
podmonoid algebraiczny, to $A^{ * } = \{ 0 \}$, bowiem $0 + 0 = 0$.
Jeżeli, przez $A^{ * }$ rozumiemy wolny podmonoid słowny to wówczas
zachodzi
\begin{equation}
  \label{eq:Forys-Forys-12}
  A^{ * } = \{ \texttt{""}, \texttt{"} 0 \texttt{"},
  \texttt{"} 00 \texttt{"}, \texttt{"} 000 \texttt{"},
  \texttt{"} 0000 \texttt{"}, \ldots \}.
\end{equation}
Powyższy monoid słowny jest izomorficzny z~całym monoidem
$( \Nbb_{ 0 }, +, 0)$. Ten izomorfizm zadany jest przez relacje
\begin{equation}
  \label{eq:Forys-Forys-13}
  \begin{split}
    h( \texttt{""} ) &= 0, \\
    h( \texttt{"} 0 \texttt{"} ) &= 1, \\
    h( \texttt{"} 0 0 \texttt{"} ) &= 2, \\
    h( \texttt{"} 0 0 0 \texttt{"} ) &= 3, \\
                     &\vdots
  \end{split}
\end{equation}

\vspace{\spaceFour}





\Str{8--9} By uniknąć nieporozumień, należy jasno stwierdzić,
że~gdy będziemy mówić o~„zbiorze ciągów liter z~alfabetu $A$” to
w~domyśle do tego zbioru zawsze należy też ciąg pusty \texttt{""}.

Aby uniknąć innych dwuznaczności, związanych z~ciągami pustymi,
wprowadzimy teraz dodatkową notację. Niech $A$ będzie dowolnym
zbiorem. Jeżeli będziemy rozważali monoid, niekoniecznie wolny,
zbudowany z~pewnych ciągów elementów zbioru $A$, wraz z~działaniem
konkatenacja, to dla $a \in A$ określmy $a^{ 0 }$ jako
\begin{equation}
  \label{eq:Forys-Forys-14}
  a^{ 0 } = \texttt{""}.
\end{equation}

Analogicznie, jeśli dany jest wolny monoid słowny $A^{ * }$, to dla
$w \in A^{ * }$ określamy
\begin{equation}
  \label{eq:Forys-Forys-15}
  w^{ 0 } = \texttt{""} \in A^{ * }.
\end{equation}

Warto tutaj dodać pewnie komentarz. Jak to jest możliwe, że~monoid
składający~się pewnych ciągów elementów zbioru $A$ wraz z~działaniem
konkatenacji, może nie być monoidem wolnym? To zagadnienie omówimy
bardziej szczegółowo trochę dalej, teraz poprzestaniemy na zauważeniu,
że~wolny monoid słowny składa~się, ze~\textit{wszystkich} ciągów
elementów zbioru~$A$. Jeżeli dany monoid jest utworzony nie ze
wszystkich, lecz tylko~niektórych ciągów elementów zbioru~$A$, to nie
musi on być monoidem wolny.

\vspace{\spaceFour}





\Str{9} Wedle przyjętej na tej stronie nomenklatury elementy
alfabetu będziemy nazywać „literami” i~będziemy mówić, że~„słowa są
ciągami liter”. Jednak ze względu na wygodę, warto przyjąć, że słowo
„symbol” jest synonimem słowa „litera”, dzięki czemu będziemy mogli
powiedzieć „elementami alfabetu są symbole” i~„słowa są ciągami
symboli”.

\vspace{\spaceFour}





\Str{9} Dowód twierdzenia 1.2.1 byłby znacznie prostszy do zrozumienia,
gdyby został wskazane, że~jeśli $h : A^{ * } \to M$ jest rozszerzeniem
odwzorowania $f : A \to M$ do homomorfizmu musi spełniać następujące trzy
warunki
\begin{subequations}
  \begin{align}
    \label{eq:Forys-Forys-16-A}
    h( 1 ) &= 1_{ M }, \\
    \label{eq:Forys-Forys-16-B}
    h( a ) &= f( a ), \quad \forall \, a \in A, \\
    \label{eq:Forys-Forys-16-C}
    h( a_{ 1 } \ldots a_{ n } )
           &= f( a_{ 1 } ) \cdot \ldots \cdot f( a_{ n } ),
  \end{align}
\end{subequations}
$\forall \, a_{ i } \in A$, $i = 1, 2, \ldots, n$. Dla większej
przejrzystości działanie wewnętrzne w~monoidzie $M$ zapisaliśmy tu za
pomocą symbolu $\cdot$. Równania te gwarantują nam, że~może istnieć co
najwyżej jedno odwzorowanie $h : A^{ * } \to M$, dla których są one
spełnione.

Możemy teraz użyć równań
\eqref{eq:Forys-Forys-16-A}-\eqref{eq:Forys-Forys-16-C}, do
\textit{zdefiniowania} homomorfizmu $h$. Ponieważ sprawdzenie, że~tak
zadane odwzorowanie jest homeomorfizmem jest stosunkowo
proste\footnote{Jak to niektórzy powiadają: „Nie ma czego dowodzić”.}
opuścimy dyskusję tego problemu.

\vspace{\spaceFour}





\Str{9} W~celu uproszczenia rozważań wprowadzimy teraz dodatkową
terminologię. Homomorfizm $h : A^{ * } \to M$, $A \subset M$, prowadzący
z~wolnego monoidu słownego $A^{ * }$ do monoidu $M$, jednoznacznie
wyznaczony przez warunki
\begin{subequations}
  \begin{align}
    \label{eq:Forys-Forys-17-A}
    h( \texttt{""} ) &= 1_{ M }, \\
    \label{eq:Forys-Forys-17-B}
    h( \texttt{"} a \texttt{"} ) &= a, \quad
                                   \forall \, a \in A,
  \end{align}
\end{subequations}
będziemy nazywali \textbf{kanonicznym homomorfizmem wolnego monoidu
  $A^{ * }$ słownego w~monoid $M$}, lub po prostu \textbf{kanonicznym
  homomorfizmem}, gdy zbiór $A$ i~monoid $M$ będą wynikać z~kontekstu.
Jeżeli dodatkowo $h$ jest izomorfizmem, to będziemy oprócz oczywistych
sformułowań takich jak „kanoniczny izomorfizm”, będziemy również
mówili, że~$A^{ * }$ i~$M$ są kanonicznie izomorficzne.

Nie potrafię powiedzieć, czy ten homomorfizm jest kanoniczny w~sensie
teorii kategorii, jednak w~tym momencie i~tak nie potrafię wymyślić
dla niego lepszej nazwy.

\vspace{\spaceFour}





\Str{9} Warto zwrócić uwagę na to, że~omawiany tu homomorfizm
kanoniczny wolnego monoidu słownego $A^{ * }$ z~monoidem $M$, gdzie
$A \subset M$ jest zbiorem generatorów $M$, nie musi być izomorfizmem.
Zanim przejdziemy do dokładniejszego omówienia tej kwestii,
przypomnijmy, że~homomorfizm kanoniczny jest jednoznacznie wyznaczony
przez warunki \eqref{eq:Forys-Forys-17-A}--\eqref{eq:Forys-Forys-17-B}.

By to pokazać rozważmy ponownie monoid $( \Nbb_{ 0 }, +, 0 )$ i~wolny
monoid słowny $A^{ * } = \{ 0, 1 \}^{ * }$. Jest oczywiste, że
\begin{equation}
  \label{eq:Forys-Forys-18}
  A^{ * } =
  \{ \texttt{""}, \texttt{"} 0 \texttt{"}, \texttt{"} 1 \texttt{"},
  \texttt{"} 0 0 \texttt{"}, \texttt{"} 0 1 \texttt{"},
  \texttt{"} 1 1 \texttt{"}, \ldots \}.
\end{equation}
Kanoniczny homomorfizm $h : A^{ * } \to \Nbb_{ 0 }$ na mocy wzorów
\eqref{eq:Forys-Forys-16-A}-\eqref{eq:Forys-Forys-16-C}
i~\eqref{eq:Forys-Forys-17-A}-\eqref{eq:Forys-Forys-17-B}, posiada
następujące własności
\begin{subequations}
  \begin{align}
    \label{eq:Forys-Forys-19-A}
    h( \texttt{""} ) &= 0, \\
    \label{eq:Forys-Forys-19-B}
    h( \texttt{"} 0 \texttt{"} ) &= 0, \quad
                                   h( \texttt{"} 1 \texttt{"} ) = 1, \\
    \label{eq:Forys-Forys-19-C}
    h( \texttt{"} a_{ 1 } a_{ 2 } \ldots a_{ n } \texttt{"} )
                     &= h( \texttt{"} a_{ 1 } \texttt{"} )
                       + h( \texttt{"} a_{ 2 } \texttt{"} ) + \ldots
                       + h( \texttt{"} a_{ n } \texttt{"} ) = \\
                       % \skiplabel
                     &= a_{ 1 } + a_{ 2 } + \ldots + a_{ n },
  \end{align}
\end{subequations}
$\forall \, a_{ i } \in \{ 0, 1 \}$, $i = 1, 2, \ldots, n$.
Przykładowo
\begin{equation}
  \label{eq:Forys-Forys-20}
  h( \texttt{"} 0100110 \texttt{"} ) =
  0 + 1 + 0 + 0 + 1 + 1 + 0 = 3.
\end{equation}
Łatwo zauważyć, że~odwzorowanie $h$ nie jest izomorfizmem. Świadczy
o~tym choćby to, że
\begin{equation}
  \label{eq:Forys-Forys-21}
  h( \texttt{""} ) = h( \texttt{"} 0 \texttt{"} ) = 0.
\end{equation}

Na koniec zauważmy, że~wedle definicji wolnego monoidu musi istnieć
\textit{co najmniej jeden} taki zbiór $B \subset M$, że~$M$ jest
kanonicznie izomorficzny z~wolnym monoidem słownym $B^{ * }$. Dla
monoidu $( \Nbb, +, 0 )$ taki zbiorem jest $B = \{ 1 \}$, więc jest to
monoid wolny.

\vspace{\spaceFour}





\Str{9} Potrzebujemy tu poczynić pewną uwagę odnośnie notacji.
Przyjmijmy, że $A$ jest podzbiorem monoidu $M$. W~skutek wybranych
oznaczeń zapis $M = A^{ * }$ może oznaczać, że~podmonoid algebraiczny
$A^{ * }$ jest równy $M$. Jeśli zaś kanoniczny homomorfizm
$h : A^{ * } \to M$ jest izomorfizmem, to wówczas przez $M = A^{ * }$,
będzie rozumieć skrótowy zapisem tego faktu. Obie te sytuacje nie są
wcale równoważne.

Rozpatrzmy ponownie monoid $( \Nbb_{ 0 }, +, 0 )$ i~wolny monoid
słowny $A^{ * } = \{ 0, 1 \}^{ * }$. Jest oczywiste, że~jeśli symbol
$A^{ * }$ rozumiemy jako podmonoid algebraiczny to zachodzi
$M = A^{ * }$, bo~$n = 1 + 1 + \ldots + 1$ (suma $n$ jedynek), dla
każdego $n \in \Nbb$. Jeśli jednak przez $A^{ * }$ rozumiemy wolny
monoid słowny, to jak zostało wykazane w~poprzednich komentarzach,
homomorfizm $h$ \textit{nie jest} izomorfizmem, więc nie możemy
napisać równości „$M = A^{ * }$”.

Na koniec zauważmy, że~jeśli $A = \{ 1 \}$ to wówczas wolny monoid
słowny $A^{ * }$ jest kanonicznie izomorficzny
z~$( \Nbb_{ 0 }, +, 0)$. Ten przykład powinien uczynić zrozumiałymi
pewne subtelności tej teorii.

\vspace{\spaceFour}





\Str{9} Niech $M$ będzie monoidem wolnym, $B \subset M$ jego bazą,
a~$h : B^{ * } \to M$ kanonicznym izomorfizmem. Dla $a \in M$
wprowadzamy następującą notację
\begin{equation}
  \label{eq:Forys-Forys-22}
  \texttt{"} a \texttt{"} :=
  \texttt{"} b_{ 1 } b_{ 2 } \ldots b_{ n } \texttt{"}, \quad
  \textrm{dla} \quad
  a = h( \texttt{"} b_{ 1 } b_{ 2 } \ldots b_{ n } \texttt{"} ), \,
  b_{ i } \in B,\, i = 1, 2, \ldots, n.
\end{equation}
Mamy nadzieję, że ta notacja pozwoli w~przejrzysty sposób zapisywać
pewne dość abstrakcyjne kroki rachunkowe.

\vspace{\spaceFour}





\Str{10} Dowód drugiej części twierdzenia 1.2.2 byłby bardziej zrozumiały,
gdyby wzór przedstawiający działania kanonicznego homomorfizmu zapisać jako
\begin{equation}
  \label{eq:Forys-Forys-23}
  h( \texttt{"} a_{ 1 } a_{ 2 } \ldots a_{ n } \texttt{"} ) =
  h( \texttt{"} a_{ 1 } \texttt{"} ) h( \texttt{"} a_{ 2 } \texttt{"} ) \ldots
  h( \texttt{"} a_{ n } \texttt{"} ) = a_{ 1 } a_{ 2 } \ldots a_{ n }.
\end{equation}

\vspace{\spaceFour}





\Str{10} W~dowodzie twierdzenia 1.2.2 użyte jest mowa o~„długości elementu
$m \in M$ w~$B^{ * }$”. Sens tego jest następujący. Niech będzie dany zbiór
$B$, niekoniecznie zawarty w~$M$, i~izomorfizm $f$, nie musi być to
izomorfizm kanoniczny, między wolnym monoidem słownym $B^{ * }$ a~$M$.
Dla każdego elementu $m \in M$ istnieje więc $w \in B^{ * }$, taki~że
$f( w ) = m$. Możemy więc przypisać elementowi $m$ liczbę z~$\Nbb_{ 0 }$,
którą nazywamy jego długością, za pomocą wzoru
\begin{equation}
  \label{eq:Forys-Forys-24}
  | m | := | w |.
\end{equation}
Długość $m$ może zależeć od wyboru zbioru $B$ i~odwzorowanie $F$, ale
problemem tym nie będziemy~się tu zajmować. Powyższy fakt wystarczy
w~zupełności do~udowodnienia twierdzenia~1.2.2.

Wybiegnijmy teraz trochę w~przyszłość. Jeśli monoid $M$ jest monoidem
wolnym, to~z~definicji jest on kanonicznie izomorficzny z~wolnym
monoidem słownym $A^{ * }$, dla pewnego $A \subset M$. Dzięki
wnioskowi 1.2.2 i~dyskusji przeprowadzonej w~tych notatkach na jego
temat, zbiór $A$ jest jednoznacznie wyznaczony przez relację
\begin{equation}
  \label{eq:Forys-Forys-25}
  A = S \setminus S^{ 2 }, \quad
  S = M \setminus \{ 1_{ M } \}.
\end{equation}
W~każdym monoidzie wolnym istnieje więc kanoniczne pojęcie długości
elementu $m \in M$. Mianowicie, niech $h : A^{ * } \to M$ będzie
kanonicznym izomorfizmem odpowiednich monoidów i~niech $m = h( w )$.
Przez \textbf{długość elementu $m \in M$} rozumiemy liczbę całkowitą
nieujemną określoną przez wzór \eqref{eq:Forys-Forys-20}.

\vspace{\spaceFour}





\Str{10--11} W~kontekście podanych tu rozważań, należy zatrzymać~się
zatrzymać nad pewnymi własnościami potęg podzbiorów w~monoidach
i~półgrupach. Niech $S$ będzie dowolną półgrupą, zachodzi wówczas
\begin{equation}
  \label{eq:Forys-Forys-26}
  S \supset S^{ 2 } \supset S^{ 3 } \supset \ldots
\end{equation}
Ogólniej $S^{ k } \supset S^{ k + 1 }$ i~może zachodzić
$S^{ k } \neq S^{ k + 1 }$. Choć dowód tego twierdzenia jest bardzo
prosty to przeprowadzimy go powoli i~szczegółowo, by dokładnie
zrozumieć o~co w~nim chodzi.

Rozpatrzmy przypadek $S \supset S^{ 2 }$. Każdy element jest $S^{ 2 }$ jest
postaci $ab$, gdzie $a, b \in S$. Z~definicji działania wewnętrznego
mamy $ab = c$, $c \in S$, z~czego wynika, że~$S \supset S^{ 2 }$.

Weźmy teraz przypadek $S^{ 2 } \supset S^{ 3 }$. Rozpatrzymy element
\begin{equation}
  \label{eq:Forys-Forys-27}
  abc \in S^{ 3 }, \quad a, b, c \in S.
\end{equation}
Znowu, korzystając z~podstawowych własności działania wewnętrznego
w~półgrupie mamy
\begin{equation}
  \label{eq:Forys-Forys-28}
  abc = ( ab ) c.
\end{equation}
Oczywiście zachodzi $ab = a_{ 1 }$, dla pewnego $a_{ 1 } \in S$, więc
\begin{equation}
  \label{eq:Forys-Forys-29}
  abc = ( ab ) c = a_{ 1 } c,
\end{equation}
a~wyrażenie $a_{ 1 } c$ należy do zbioru $S^{ 2 }$ z~definicji.

Analogicznie przeprowadzamy dowód dla dowolnego $k$. Weźmy iloczyn
\begin{equation}
  \label{eq:Forys-Forys-30}
  a_{ 1 } a_{ 2 } a_{ 3 } a_{ 4 } \ldots a_{ k } a_{ k + 1 } \in S^{ k + 1 }, \quad
  a_{ i } \in S,\, i = 1, 2, \ldots, k + 1.
\end{equation}
Przekształcamy go w~sposób analogiczny co poprzednio.
\begin{equation}
  \label{eq:Forys-Forys-31}
  a_{ 1 } a_{ 2 } a_{ 3 } a_{ 4 } \ldots a_{ k } a_{ k + 1 } =
  ( a_{ 1 } a_{ 2 } ) a_{ 3 } a_{ 4 } \ldots a_{ k } a_{ k + 1 } =
  b a_{ 3 } a_{ 4 } \ldots a_{ k } a_{ k + 1 } \in S^{ k },
\end{equation}
gdzie $b = a_{ 1 } a_{ 2 } \in S$. Tym samym dowiedliśmy,
że~$S^{ k } \supset S^{ k + 1 }$.

Gdybyśmy rozważali monoid $M$, to nie zaobserwowalibyśmy nic
ciekawego. Wynika to z~tego, że $1_{ M } \in M$, więc $M^{ 2 } = M$.
Sprawa robi~się ciekawsza, gdy rozważamy zbiór
$S = M \setminus \{ 1_{ M } \}$. Nie możemy zagwarantować,
że~$S \supset S^{ 2 }$, bo może~się zdarzyć, że~dla dwóch elementów
$a, b \in S$ zachodzi $a b = 1_{ M }$. Możemy jednak zauważyć,
że~zachodzi następujące twierdzenie.





% #############
\begin{theorem}
  \label{thm:Forys-Forys-01}

  Niech $M$ będzie monoidem. Zbiór $S = M \setminus \{ 1_{ M } \}$ jest
  podpółgrupą wtedy i~tylko wtedy, gdy grupa jedynki monoidu $M$ jest
  trywialna.

\end{theorem}



\begin{proof}

  Dowód tego twierdzenia jest w~istocie bardzo prosty. Po pierwsze
  przypomnijmy, że~z~definicji zbiór $S = M \setminus \{ 1_{ M } \}$
  jest półgrupą wtedy i~tylko wtedy, gdy $S \supset S^{ 2 }$.

  $\Rightarrow$ Wystarczy pokazać, że~jeśli $a b = 1_{ M }$ to
  $a = 1_{ M }$ i~tym samym $b = 1_{ M }$. Ponieważ $1_{ M } \notin S$
  i~$S^{ 2 } \subset S$, więc jeśli istniej taka para $a, b \in M$,
  że~$ab = 1_{ M }$ to $a$ lub $b$ musi należeć do zbioru
  $M \setminus S = \{ 1_{ M } \}$, co kończy tę część dowodu.

  $\Leftarrow$ Dowód przeprowadzimy przez kontrapozycję. Jeżeli grupa jedynki
  nie jest trywialna, to istnieją dwa takie elementy,
  że~$a b = 1_{ M }$, przy czym $a \neq 1_{ M }$ i~$b \neq 1_{ M }$.
  Tym samym mielibyśmy $a, b \in S$, $ab = 1_{ M }$
  czyli~$ab \notin S$. Dowód został zakończony.

\end{proof}
% #############





Widzimy więc, że~jeśli $M$ jest monoidem, to aby dla zbioru
$S = M \setminus \{ 1_{ M } \}$ zachodził relacja
\eqref{eq:Forys-Forys-26}, kluczowe jest to by~$S$ było półgrupą.
Jeśli tak nie jest, to nie tylko może zachodzić
\begin{equation}
  \label{eq:Forys-Forys-32}
  S \nsupseteq S^{ 2 },
\end{equation}
ale może też istnieć wiele $k > 1$, takich~że
\begin{equation}
  \label{eq:Forys-Forys-33}
  S^{ k } \nsupseteq S^{ k + 1 }.
\end{equation}

\vspace{\spaceFour}





\Str{10} Zatrzymajmy~się na chwilę nad wnioskiem 1.2.2.
Stwierdzanie, że~baza $B$ wolnego monoidu (wolnej półgrupy) $M$ jest
minimalnym zbiorem generatorów, oznaczy tyle, że~dla każdego zbioru
generatorów $D$ zachodzi $B \subset D$. Z~dowodu twierdzenia 1.2.2
wynika, że~$A \subset B$, gdzie $A = S \setminus S^{ 2 }$. Jeśli ponadto pokazalibyśmy,
że~$B \subset D$, dla każdego zbioru generatorów $D$, to otrzymamy
równość~$A = B$. W~tym momencie jedyny dowód inkluzji $B \subset D$ jaki
potrafię wymyślić, przebiega w~następujący sposób.

Jeżeli zbiór $D$ zawiera element neutralny, to możemy go usunąć
z~niego bez straty ogólności, por. wzór na stronie~7 omawianej
książki. Od tego momentu przyjmujemy więc, że~$1_{ M } \notin D$.

Każdy element $d$ zbioru $D$ jest w~sposób koniczny izomorficzny
z~pewny ciągiem elementów zbioru $B$, czyli
$\texttt{"} d \texttt{"} = h^{ -1 }( d ) = \texttt{"} b_{ 1 } b_{ 2 }
\ldots b_{ n } \texttt{"} \in B^{ * }$, gdzie symbol $B^{ * }$ oznacza
wolny monoid słowny. Tym samym jeśli
$b = d_{ 1 } d_{ 2 } \ldots d_{ m }$, $d_{ i } \in D$,
$i = 1, 2, \ldots, m$, to musi zachodzić
\begin{equation}
  \label{eq:Forys-Forys-34}
  \texttt{"} d_{ 1 } d_{ 2 } \ldots d_{ m } \texttt{"} =
  \texttt{"} b \texttt{"} = h^{ -1 }( b ) \in B^{ * }.
\end{equation}
Ponieważ każdy $\texttt{"} d_{ i } \texttt{"}$ sam jest postaci
$\texttt{"} b_{ i,\, 1 } b_{ i,\, 2 } \ldots b_{ i,\, n_{ i } }
\texttt{"}$, gdzie $b_{ i,\, j } \in B$, $i = 1, 2, \ldots, m$,
$j = 1, 2, \ldots, n_{ i }$, to na mocy tego co zostało powiedziane przy
okazji omawiania wzoru \eqref{eq:Forys-Forys-09}, powyższa równość
może zachodzić wtedy i~tylko wtedy, gdy $m = 1$
i~$d_{ 1 } = \texttt{"} b \texttt{"}$. Czyli
$\texttt{"} b \texttt{"} \in h^{ -1 }( D )$, co jest równoważne temu,
że~$h( \texttt{"} b \texttt{"} ) = b \in D$. Tym samym dowiedliśmy,
że~$B \subset D$.

Wykazaliśmy nie tylko, że~baza $B$ jest nie tylko najmniejszym
w~sensie inkluzji zbiorem generatorów wolnego monoidu $M$, ale też,
że~jest ona wyznaczona jednoznacznie przez warunek $B = S \setminus S^{ 2 }$.

\vspace{\spaceFour}





\Str{10} Wzorując~się na przykładzie 1.2.1 wprowadzimy teraz dwa monoidy,
które mogą okazać~się przydatne przy podawania przykładów i~kontrprzykładów
dotyczących własności tych struktur algebraicznych.

Niech będzie dany dowolny\footnote{Nie będziemy~się tu wgłębiać
  w~dyskusje nad ontologiczną naturą symbolu~$a$.} symbol $a$.
Definiujemy wolne monoidy słowne
\begin{subequations}
  \begin{align}
    \label{eq:Forys-Forys-35-A}
    \Acal_{ 1 } &:= \{ a^{ n } : n = 0, 1, 2, \ldots \}, \\
    \label{eq:Forys-Forys-35-B}
    \Acal_{ 2 } &:= \{ \texttt{""} \} \cup \{ a^{ n } : n = 2, 3, \ldots \}.
  \end{align}
\end{subequations}

Monoid $\Acal_{ 1 }$ jest wolny monoidem słownym nad~alfabetem
$B = \{ a \}$. Monoid $\Acal_{ 2 }$ nie jest monoidem wolnym,
bo~$S \setminus S^{ 2 } = \{ a^{ 2 }, a^{ 3 } \}$,
$S = \Acal_{ 2 } \setminus \{ 1 \}$ zaś rozkład na~elementy $a^{ 2 }$
i~$a^{ 3 }$ nie jest jednoznaczny. Przykładowo
\begin{equation}
  \label{eq:Forys-Forys-36}
  a^{ 6 } = a^{ 2 } a^{ 2 } a^{ 2 } = a^{ 3 } a^{ 3 }.
\end{equation}

\vspace{\spaceFour}





\Str{10} W~kontekście wniosku~1.2.2 należy zauważyć, że~przeprowadzając
drobną modyfikację fragmentu dowodu twierdzenia~1.2.2, można udowodnić
następujące twierdzenie.





% #############
\begin{theorem}
  \label{thm:Forys-Forys-02}

  Niech $M$ będzie monoidem,~a~$D$ zbiorem jego generatorów. Wówczas
  zachodzi
  \begin{equation}
    \label{eq:Forys-Forys-37}
    A \subset D,
  \end{equation}
  gdzie $A = S \setminus S^{ 2 }$, $S = M \setminus \{ 1_{ M } \}$.

\end{theorem}



\begin{proof}

  Dla dowolnego $a \in A$ zachodzi
  \begin{equation}
    \label{eq:Forys-Forys-38}
    a = d_{ 1 } d_{ 2 } \ldots d_{ n }, \quad
    d_{ i } \in D,\; i = 1, 2, \ldots, n.
  \end{equation}
  Jeżeli $n > 1$, to~wówczas jak już dyskutowaliśmy $a \in S^{ 2 }$
  i~tym samym\footnote{Zastanowić się czy to jest dowód ad absurdum.}
  $a \notin A$. Wobec tego musi zachodzi $n = 1$, $a = d_{ 1 }$, stąd
  $a \in D$.

\end{proof}
% #############





\Str{11} W~dowodzie twierdzenia 1.2.4 jest napisane „Jeśli monoid jest
wolny, to na mocy twierdzenia 1.2.2 spełnia wszystkich powyższe warunki”,
gdzie warunki o~których mowa to to, że~monoid $M$ jest równo podzielny,
$S = M \setminus \{ 1_{ M } \}$ jest podgrupą i~$\cap_{ n \in \Nbb } \, S^{ n } = \emptyset$. Jednak
według mnie nie wynikają one z~tego twierdzenia, lecz z~sformułowanej
jawnie w~tych notatkach definicji wolnego monoidu $M$.

Dokładniej wszystkie te własności posiada każdy wolny monoid słownego
i~na mocy kanonicznego izomorfizmu przenoszą~się one z~wolnego monoidu
słownego $B^{ * }$ na monoid $M$, gdzie $B \subset M$. Dowody
potrzebnych własności dla wolnego monoidu słownego przedstawimy
poniżej.

Zaczniemy od~tego, że~dla dowolnego wolnego monoidu słownego $M$ zbiór
$S = M \setminus \{ 1_{ M } \}$ jest podgrupą. W~zasadzie cała praca
jest już wykonana, bo zgodnie z~twierdzeniem \ref{thm:Forys-Forys-01}
$S$ jest podpółgrupą wtedy i~tylko wtedy, gdy grupa jedynki $M$ jest
trywialna, a~ten fakt udowodniliśmy przy okazji podanie równości
\eqref{eq:Forys-Forys-09}.

Przejdźmy teraz do dowodu tego, że wolny monoid słownym jest
równopodzielny. Weźmy takie elementy $a, b, c, d \in M$, że~zachodzi
\begin{equation}
  \label{eq:Forys-Forys-39}
  a b = c d.
\end{equation}
Z~definicji wolnego monoidu słownego wynika, że~możemy przedstawić te
elementy jako
\begin{subequations}
  \begin{align}
    \label{eq:Forys-Forys-40-A}
    a &= f_{ a,\, 1 } f_{ a,\, 2 } \ldots f_{ a,\, n_{ 1 } }, \\
    \label{eq:Forys-Forys-40-B}
    b &= f_{ b,\, 1 } f_{ b,\, 2 } \ldots f_{ b,\, n_{ 2 } }, \\
    \label{eq:Forys-Forys-40-C}
    c &= f_{ c,\, 1 } f_{ c,\, 2 } \ldots f_{ c,\, n_{ 3 } }, \\
    \label{eq:Forys-Forys-40-D}
    d &= f_{ d,\, 1 } f_{ d,\, 2 } \ldots f_{ d,\, n_{ 4 } },
  \end{align}
\end{subequations}
gdzie wszystkie $f_{ a,\, i }$, $f_{ b,\, j }$, $f_{ c,\, k }$,
$f_{ d,\, l }$ należą do bazy $B$ wolnego monoidu słownego~$M$. Tym
samym mamy
\begin{subequations}
  \begin{align}
    \label{eq:Forys-Forys-41-A}
    a b
    &=
      f_{ a,\, 1 } f_{ a,\, 2 } \ldots f_{ a,\, n_{ 1 } } f_{ b,\, 1 }
      f_{ b,\, 2 } \ldots f_{ b,\, n_{ 2 } }, \\
    \label{eq:Forys-Forys-41-B}
    cd
    &=
      f_{ c,\, 1 } f_{ c,\, 2 } \ldots f_{ c,\, n_{ 3 } } f_{ d,\, 1 }
      f_{ d,\, 2 } \ldots f_{ d,\, n_{ 4 } }.
  \end{align}
\end{subequations}
Ponieważ elementy wolnego monoidu słownego są ciągami symboli, wobec
tego równość \eqref{eq:Forys-Forys-30} może zachodzić wtedy i~tylko
wtedy, gdy $n_{ 1 } + n_{ 2 } = n_{ 3 } + n_{ 4 }$
i~$f_{ a,\, 1 } = f_{ c,\, 1 }$, $f_{ a,\, 2 } = f_{ c,\, 2 }$, etc.
Rozpatrzmy teraz możliwe przypadki.

Jeśli $n_{ 1 } = n_{ 3 }$ to $a = b$ i~$c = d$, więc równości $a = cu$
i~$d = ub$ zachodzą dla $u = \texttt{""}$. Może przejść do analizy
następnego przypadku.

Jeśli $n_{ 1 } > n_{ 3 }$, to możemy zapisać
\begin{equation}
  \label{eq:Forys-Forys-42}
  c = f_{ a,\, 1 } f_{ a,\, 2 } \ldots f_{ a,\, n_{ 3 } },
\end{equation}
i~tym samym
\begin{equation}
  \label{eq:Forys-Forys-43}
  a =
  ( f_{ a,\, 1 } f_{ a,\, 2 } \ldots f_{ a,\, n_{ 3 } } ) \cdot ( f_{ a,\, n_{ 3 } + 1 }
  f_{ a,\, n_{ 3 } + 2 } \ldots f_{ a,\, n_{ 1 } } )
  =
  c ( f_{ a,\, n_{ 3 } + 1 } f_{ a,\, n_{ 3 } + 2 } \ldots f_{ a,\, n_{ 1 } } ).
\end{equation}
Kładąc
$u = f_{ a,\, n_{ 3 } + 1 } f_{ a,\, n_{ 3 } + 2 } \ldots f_{ a,\, n_{ 1 }
}$, dostajemy równość $a = cu$. Implikuje to również,
że~$f_{ d,\, 1 } = f_{ a,\, n_{ 3 } + 1 }$,
$f_{ d,\, 2 } = f_{ a,\, n_{ 3 } + 2 }$, etc. Tym samym dostajemy
równość $d = ub$ i~dowód tego przypadku został zakończony.

Pozostał nam do rozpatrzenia przypadek $n_{ 1 } < n_{ 3 }$, dowód
przebiega analogicznie co dla nierówności w~drugą stronę. Tym razem
mamy
\begin{equation}
  \label{eq:Forys-Forys-44}
  a = f_{ c,\, 1 } f_{ c,\, 2 } \ldots f_{ c,\, n_{ 1 } },
\end{equation}
i~analogicznie
\begin{equation}
  \label{eq:Forys-Forys-45}
  c =
  ( f_{ c,\, 1 } f_{ c,\, 2 } \ldots f_{ c,\, n_{ 1 } })
  \cdot ( f_{ c,\, n_{ 1 } + 1 } f_{ c,\, n_{ 1 } + 2 } \ldots f_{ c,\, n_{ 3 } } )
  =
  a \cdot ( f_{ c,\, n_{ 1 } + 1 } f_{ c,\, n_{ 1 } + 2 } \ldots f_{ c,\, n_{ 3 } } )
\end{equation}
Oznaczmy
$v = f_{ c,\, n_{ 1 } + 1 } f_{ c,\, n_{ 2 } + 2 } \ldots f_{ c,\, n_{ 3 }
}$, stąd $c = a v$. Tak jak poprzednio mamy ciąg równości
$f_{ b,\, 1 } = f_{ c,\, n_{ 1 } + 1 }$,
$f_{ b,\, 2 } = f_{ c,\, n_{ 1 } + 2 }$, etc., co prowadzi do równości
$b = v d$. To kończy dowód tej własności.

Pozostaje nam udowodnić, że~dla wolnego monoidu słownego $M$ zachodzi
$\bigcap_{ n \in \Nbb } S^{ n } = \emptyset$. Zauważmy, że~z~definicji
zbioru $S$, jeśli $w$ należy do $S$ to jego długość jest co najmniej
równa jeden: $| w | \geq 1$. Tym samym jeśli $w_{ 1 } \in S^{ 2 }$, to
jego długość jest co najmniej równa $2$, bo istnieją takie
$s_{ 1 }, s_{ 2 } \in S$, takie że $w = s_{ 1 } s_{ 2 }$, więc
\begin{equation}
  \label{eq:Forys-Forys-46}
  | w | = | s_{ 1 } s_{ 2 } | = | s_{ 1 } | + | s_{ 2 } | \geq 2.
\end{equation}
Rozumują w~ten sposób i~korzystając z~zasady indukcji w~prosty sposób
otrzymujemy, że
\begin{equation}
  \label{eq:Forys-Forys-47}
  | w | \geq n, \quad \forall w \in S^{ n }.
\end{equation}

Teraz możemy udowodnić pożądaną własność. Oczywista jest zależność
\begin{equation}
  \label{eq:Forys-Forys-48}
  \bigcap_{ n \in \Nbb } \, S^{ n } \subset S.
\end{equation}
Niech teraz $w \in S$ i~$| w | = n$. W~takim razie
$w \notin S^{ n + 1 }$, więc
$w \notin \bigcap_{ n \in \Nbb } S^{ n }$. Jest to prawdą dla każdego
$w \in S$ i~stąd $\bigcap_{ n \in \Nbb } S^{ n } = \emptyset$.

\vspace{\spaceFour}





\Str{11} W~dowodzie twierdzenia 1.2.4 fakt, że~grupa jedynki
monoidu $M$ jest trywialna, nie wydaje~się potrzebny, więc tą część
tego dowodu możemy zupełnie pominąć.

Grupa jedynki monoidu $M$ rozważanego w~tym twierdzeniu jest w~istocie
trywialna. Wynika to~na mocy twierdzenia \ref{thm:Forys-Forys-01}
z~tego, że~$S$ jest podpółgrupą, co przyjęliśmy w~założeniach tego
twierdzenia. Do dowodu tej własności nie jest więc zupełnie potrzebny
fakt, że~$\bigcap_{ n \in \Nbb } S^{ n } = \emptyset$.

\vspace{\spaceFour}





\Str{11} W~dowodzie twierdzenia 1.2.4 jest mowa o~tym, że ciąg inkluzji
\begin{equation}
  \label{eq:Forys-Forys-49}
  S \supset S^{ 2 } \supset \ldots \supset S^{ k } \supset S^{ k + 1 } \supset S^{ k + 2 } \supset \ldots
\end{equation}
jest „skończony”, bądź „nieskończony”, jednak nie jest wcale jasne, co
słowo „skończony” oznacza w~tym kontekście.

W~istocie „skończony ciąg inkluzji” oznacza po prostu, że dla pewnego
$k$ zachodzi $S^{ k } = S^{ k + 1 }$, przyjmijmy przy tym, że~jest to
najmniejsze $k$ o~tej własności. Tym samym, dzięki łączności mnożenia
podzbiorów monoidu, otrzymujemy
\begin{equation}
  \label{eq:Forys-Forys-50}
  S^{ k + 2 } = ( S^{ k + 1 } ) S = S^{ k } S = S^{ k + 1 } = S^{ k }.
\end{equation}
Stąd wynika, że~w~takiej sytuacji ciąg \eqref{eq:Forys-Forys-49}
przyjmuje postać
\begin{equation}
  \label{eq:Forys-Forys-51}
  S \supsetneq S^{ 2 } \supsetneq S^{ 3 } \supsetneq \ldots \supsetneq S^{ k } = S^{ k + 1 } = S^{ k + 2 } = \ldots
\end{equation}
Małe słowo komentarza. Ponieważ $k$ zostało wybrane jako najmniejsza
liczba, taka że zachodzi $S^{ k } = S^{ k + 1 }$, stąd wynika, że~dla
$l < k$ jest prawdą, iż~$S^{ l } \supsetneq S^{ l + 1 }$, więc powyższy wzór
jest poprawny.

„Nieskończony ciąg inkluzji” oznacza więc sytuację, gdy
\begin{equation}
  \label{eq:Forys-Forys-52}
  S \supsetneq S^{ 2 } \supsetneq S^{ 3 } \supsetneq \ldots \supsetneq S^{ k } \supsetneq S^{ k + 1 } \supsetneq S^{ k + 2 } \supsetneq \ldots
\end{equation}

\vspace{\spaceFour}





\Str{11} Powinniśmy~się przyjrzeć lepiej stwierdzeniu, że~jeśli
$m \in S^{ k } \setminus S^{ k + 1 }$, to można go przedstawić jako
$s_{ 1 } s_{ 2 } \ldots s_{ k }$, gdzie
$s_{ i } \in S \setminus S^{ 2 }$, $i = 1, 2, \ldots, k$. Choć jest to fakt
stosunkowo prosty, omówimy go w~sposób szczegółowy, by pozostawić
możliwie mało miejsca na niejasności.

Istnienie ciągu $s_{ i } \in S$, $i = 1, 2, \ldots, k$ wynika wprost z~tego,
że~$s \in S^{ k }$. Pozostaje pokazać, że~jeżeli
$m = s_{ 1 } s_{ 2 } \ldots s_{ k } \notin S^{ k + 1 }$, to dla każdego
$i = 1, 2, \ldots, k$ mamy $s_{ 1 } \notin S^{ 2 }$. Na początek
przypomnijmy, że~jeśli elementu $u \in S$ można przedstawić jako
$u = u_{ 1 } u_{ 2 } \ldots u_{ n }$, $n \geq 2$, to można go przedstawić
jako $u = u_{ 1 } U$, przyjmując $U = u_{ 2 } u_{ 3 } \ldots u_{ n }$. Jest
to równoważne stwierdzeniu, że~$S^{ n } \subset S^{ 2 }$, gdy
$n \geq 2$, jednak przedstawione w~tej formie powinno uczynić
zrozumienie tego problemu prostszym.

Dowód rzeczonej implikacji można teraz przeprowadzić bardzo prosto,
korzystając z~zasady kontrapozycji. Jeśli istnieje element
$s_{ i } \in S^{ 2 }$ to można go zapisać jako iloczyn dwóch lub
więcej elementów. Ponieważ dla dowodu dokładna liczba czynników na
które rozkłada~się $s_{ i }$ nie ma znaczenia, więc przyjmiemy,
że~zachodzi najprostsza możliwość: $s_{ i } = t_{ 1 } t_{ 2 }$,
$t_{ 1 }, t_{ 2 } \in S$. Więc
$m = s_{ 1 } s_{ 2 } \ldots s_{ i - 1 } t_{ 1 } t_{ 2 } s_{ i + 1 } \ldots s_{ k
}$, więc $m \in S^{ k + 1 }$, czyli nie jest prawdą,
iż~$m \notin S^{ k + 1 }$, co kończy dowód.

Powyższe rozważania można podsumować stwierdzając,
że~$m \in S^{ k } \setminus S^{ k + 1 }$, wtedy i~tylko wtedy, gdy
można go zapisać jako $m = s_{ 1 } s_{ 2 } \ldots s_{ k }$,
$s_{ i } \in S \setminus S^{ 2 }$, $i = 1, \ldots, k$.

\vspace{\spaceFour}





\StrWd{11}{11} Według mnie zapis $\min\{ k, l \}$ wyglądałoby znacznie
ładniej, niż użyty tutaj $min\{ k, l \}$.

\vspace{\spaceFour}





\Str{12} Zwróćmy uwagę na to, że~homomorfizm o~którym mowa w~twierdzeniu
1.2.5 nie musi być izomorfizmem. Rozważmy bowiem następujący monoid. Niech
$a$ będzie dowolnym symbolem\footnote{Nie będziemy~się tu wgłębiać
  w~ontologiczną naturę tego symbolu.} i~określmy następujący wolny monoid
słowny:
\begin{equation}
  \label{eq:Forys-Forys-53}
  \Acal = \{ a^{ n } : n = 0, 2, 4, \ldots \}.
\end{equation}
Inaczej mówiąc, monoid $\Acal$ składa~się, że~wszystkich ciągów
symbolu $a$, które mają parzystą długość. Monoid ten jest wolny, co
wynika choćby z~tego, że~$B = S \setminus S^{ 2 } = \{ a^{ 2 } \}$,
$S = \Acal \setminus \{ 1_{ \Acal } \}$ i~tym samym każdy element $S$
ma jednoznaczny rozkład na elementy zbioru~$B$.

Dla dowolnego $w = a^{ n } \in \Acal$, określamy homomorfizm $l$
za~pomocą wzoru
\begin{equation}
  \label{eq:Forys-Forys-54}
  l( w ) = l( a^{ n } ) := n \in \Nbb.
\end{equation}
Jest oczywiste, że~ten homomorfizm spełnia wszystkie założenie
twierdzenia i~nie jest bijekcją, bowiem zbiór $l( \Acal )$ jest równy
zbiorowi liczb parzystych.

\vspace{\spaceFour}





\noindent
\textbf{Str.~12, wiersze 2 i~3.} Wyrażenie $l^{ -1 }( 0 ) = \{ 1_{ M } \}$ jest
bardzo brzydko podzielone między te dwa wiersze.

\vspace{\spaceFour}





\Str{12} Dowód drugiej implikacji z~twierdzenia 1.2.5 jest przeprowadzony
w~sposób dość niechlujny, postaram się tutaj przedstawić go w~sposób
bardziej uporządkowany. Plan dowodu jest jasny. Ponieważ założyliśmy,
że~monoid $M$ jest równopodzielny, to jeśli na podstawie własności
homomorfizmu $l$ dowiedziemy, że~$S = M \setminus \{ 1_{ M } \}$ jest półgrupą
i~$\bigcap_{ n \in \Nbb } S^{ n } = \emptyset$, to na podstawie twierdzenia 1.2.4 monoid~$M$
jest monoidem wolnym.

Pierwszy krok polega na pokazaniu, że~$S$ jest półgrupą. Zgodnie
z~twierdzeniem \eqref{thm:Forys-Forys-01} jest to równoważne temu,
że~grupa jedynki monoidu $M$ jest trywialna. Ten ostatni fakt teraz
udowodnimy.

Niech $x$ należy do grupy jedynki monoidu $M$. W~takim wypadku
istnieje element $y$, taki~że
\begin{equation}
  \label{eq:Forys-Forys-55}
  x y = 1_{ M }.
\end{equation}
Obliczając wartość homomorfizmu $l$ na obu stronach tej równości
dostajemy
\begin{equation}
  \label{eq:Forys-Forys-56}
  l( x y ) = l( x ) + l( y ) = l( 1_{ M } ) = 0.
\end{equation}
Ta równość może zachodzić wtedy i~tylko wtedy, gdy
$l( x ) = l( y ) = 0$. Ponieważ $l^{ -1 }( 0 ) = \{ 1_{ M } \}$, więc
$x = y = 1_{ M }$, czyli grupa jedynki jest trywialna.

Drugi krok polega na pokazaniu,
że~$\bigcap_{ n \in \Nbb } S^{ n } = \emptyset$. Jeżeli
$M = \{ 1_{ M } \}$, to $S = \emptyset$ i~tym samym
$\bigcap_{ n \in \Nbb } S^{ n } = \emptyset$. Przejdziemy teraz do
drugiego przypadku. Udowodnimy, że~dla każdego $m \in S$, jeśli
$c = l( m )$, to $m \notin S^{ c + 1 }$. By to zrobić posłużymy~się
zasadą kontrapozycji\footnote{Kamil, przemyśl to, że zasada
  kontrapozycji w tym przypadku działa w następujący sposób:
  $\forall \, m ( p( m ) \Rightarrow q( m ) )$ przechodzi
  w~$\forall \, m ( \not q( m ) \Rightarrow \sim p( m ) )$.}. Jeżeli
$m \in S^{ c + 1 }$ to istnieją takie
$s_{ 1 }, s_{ 2 }, \ldots, s_{ c + 1 }$, zawarte w~$S$, że~zachodzi
\begin{equation}
  \label{eq:Forys-Forys-57}
  m = s_{ 1 } s_{ 2 } \ldots s_{ c + 1 }.
\end{equation}
Skorzystajmy teraz z~własności homomorfizmu $l$.
\begin{equation}
  \label{eq:Forys-Forys-58}
  l( m ) = l( s_{ 1 } s_{ 2 } \ldots s_{ c + 1 } ) =
  l( s_{ 1 } ) + l( s_{ 2 } ) + \ldots + l( s_{ c + 1 } )
\end{equation}
Ponieważ $l^{ -1 }( 0 ) = \{ 1_{ M } \}$, więc $l( s_{ i } ) \geq 1$,
dla~$i = 1, 2, \ldots, c + 1$. Tym samym, $l( m ) \geq c + 1$ co jest
sprzeczne z~tym, że~$l( m ) = c$.

Udowodniliśmy tym samym, że~dla każdego $m \in S$ jest prawdą,
że~$m \notin S^{ c + 1 }$, gdzie $c = l( m )$ i~co za tym idzie
$\bigcap_{ n \in \Nbb } S^{ n } = \emptyset$. Dowód twierdzenia jest zakończony.

\vspace{\spaceFour}





\Str{12} W~dowodzie twierdzenia 1.2.6 panuje podobny bałagan,
jak w~dowodzie twierdzenia 1.2.4. Tak jak poprzednio postaramy~się go
uporządkować.

Na początku czytamy, że~monoid wolny ma trywialną grupę jedynki, jest
równopodzielny, skracalny i~każdy jego element $m \neq 1_{ M }$
posiada skończoną liczbę nietrywialnych lewych faktorów, co ma wynikać
z~twierdzenia~1.2.5. Ponownie nie jest to prawdą. Wszystkie te cechy
wynikają z~tego, że~posiada je wolny monoid słowny i~dzięki istnieniu
kanonicznego izomorfizmu cechy te przenoszą~się na monoid wolny.
Wykazaliśmy poprzednio, że~wolny monoid słowny ma trywialną grupę
jedynki i~jest równopodzielny, teraz wykażemy dwie pozostałe
własności.

Niech $B^{ * }$ będzie wolnym monidem słowny. Załóżmy, że~dla
$a, x, y \in B^{ * }$ zachodzi
\begin{equation}
  \label{eq:Forys-Forys-59}
  a x = a y.
\end{equation}
Ponieważ $B^{ * }$ jest wolnym monoidem słownym, więc zachodzą
poniższe zależności
\begin{subequations}
  \begin{align}
    \label{eq:Forys-Forys-60-A}
    a &= b_{ a,\, 1 } b_{ a,\, 2 } \ldots b_{ a,\, n_{ 1 } }, \\
    \label{eq:Forys-Forys-60-B}
    x &= b_{ x,\, 1 } b_{ x,\, 2 } \ldots b_{ x,\, n_{ 2 } }, \\
    \label{eq:Forys-Forys-60-C}
    y &= b_{ y,\, 1 } b_{ y,\, 2 } \ldots b_{ y,\, n_{ 3 } },
  \end{align}
\end{subequations}
gdzie $b_{ a,\, i }, b_{ x,\, j }, b_{ y,\, k } \in B$, dla
$i = 1, 2, \ldots, n_{ 1 }$, $j = 1, 2, \ldots, n_{ 2 }$
i~$k = 1, 2, \ldots, n_{ 3 }$. Wstawiając te rozkłady do równości
\eqref{eq:Forys-Forys-59} dostajemy
\begin{equation}
  \label{eq:Forys-Forys-61}
  b_{ a,\, 1 } b_{ a,\, 2 } \ldots b_{ a,\, n_{ 1 } } b_{ x,\, 1 }
  b_{ x,\, 2 } \ldots b_{ x,\, n_{ 2 } }
  =
  b_{ a,\, 1 } b_{ a,\, 2 } \ldots b_{ a,\, n_{ 1 } } b_{ y,\, 1 }
  b_{ y,\, 2 } \ldots b_{ y,\, n_{ 3 } }.
\end{equation}
Ta równość może zachodzić wtedy i~tylko wtedy, gdy
$n_{ 1 } + n_{ 2 } = n_{ 1 } + n_{ 3 }$, czyli $n_{ 2 } = n_{ 3 }$
oraz
$b_{ x,\, 1 } = b_{ y,\, 1 }, b_{ x,\, 2 } = b_{ y,\, 2 }, \ldots, b_{ x,\,
  n_{ 1 } } = b_{ y,\, n_{ 1 } }$. To zaś dowodzi, że $x = y$ tym samy
wolny monoid słowny jest lewoskracalny. Dowód prawoskracalności
przebiega w~ten sam sposób do tego stopnia, że~nie mam potrzeby go
przytaczać.

Przejdziemy teraz do dowodu tego, że~w~wolnym monoidzie słownym $M$
każdym element $m \in B^{ * }$ posiada tylko skończoną ilość
nietrywialnych lewych faktorów, przy czym analogiczna własność
zachodzi dla prawych faktorów.

Metoda dowodu jest analogiczna do tej której zastosowaliśmy dla
dowodzenia lewoskracalności. Niech dla $m \in B^{ * }$ zachodzi
\begin{equation}
  \label{eq:Forys-Forys-62}
  m = s t.
\end{equation}
Jak poprzednio, obowiązują następujące równości
\begin{subequations}
  \begin{align}
    \label{eq:Forys-Forys-63-A}
    m &= b_{ m,\, 1 } b_{ m,\, 2 } \ldots b_{ m,\, n }, \\
    \label{eq:Forys-Forys-63-B}
    s &= b_{ s,\, 1 } b_{ s,\, 2 } \ldots b_{ s,\, n_{ 1 } }, \\
    \label{eq:Forys-Forys-63-C}
    t &= b_{ t,\, 1 } b_{ t,\, 2 } \ldots b_{ t,\, n_{ 2 } }.
  \end{align}
\end{subequations}
gdzie, $n = | m |$, $b_{ m,\, i }, b_{ s,\, j }, b_{ s,\, k } \in B$,
dla $i = 1, 2, \ldots, n$, $j = 1, 2, \ldots, n_{ 1 }$,
$k = 1, 2, \ldots, n_{ 2 }$. Wstawiając powyższe równości do
\eqref{eq:Forys-Forys-62} otrzymujemy
\begin{equation}
  \label{eq:Forys-Forys-64}
  b_{ m,\, 1 } b_{ m,\, 2 } \ldots b_{ m,\, n } =
  b_{ s,\, 1 } b_{ s,\, 2 } \ldots b_{ s,\, n_{ 1 } }
  b_{ t,\, 1 } b_{ t,\, 2 } \ldots b_{ t,\, n_{ 2 } }.
\end{equation}
Ponownie, z~tej równości wynika, że~$n = n_{ 1 } + n_{ 2 }$ oraz
$b_{ m,\, 1 } = b_{ s,\, 1 }, b_{ m,\, 2 } = b_{ s,\, 2 }, \ldots, b_{ m,\,
  n_{ 1 } } = b_{ s,\, n_{ 1 } }$,
$b_{ m,\, n_{ 1 } + 1 } = b_{ t,\, 1 }, b_{ m,\, n_{ 1 } + 2 } = b_{
  t,\, 2 }, \ldots, b_{ m,\, n } = b_{ t,\, n_{ 2 } }$. Z~tego wynika,
że~każdy z~nietrywialnych lewych faktorów musi być postaci
\begin{equation}
  \label{eq:Forys-Forys-65}
  s = b_{ m,\, 1 } b_{ m,\, 2 } \ldots b_{ m,\, i }, \quad
  1 \leq i \leq n.
\end{equation}
Jest ich więc dokładnie~$n$, przy czym~$n = | m |$ jest długością
słowa~$m$. To~kończy dowód pierwszej części twierdzenia.

Druga część twierdzenia wymaga w~większości tylko dobranych poprawek,
które są zamieszczone w~części „Błędy” tych notatek. Tutaj zapiszemy
kilka tylko wyjątków od tej reguły.

Strategia tej części dowodu jest taka sama, jak w~dowodzie twierdzenia
1.2.5. Zakładamy, że~monoid $M$ jest równopodzielny, skracalny, ma
trywialną grupę jedynki oraz dowolny $m \in M$, $m \neq 1_{ M }$ ma
tylko skończoną liczbę nietrywialnych lewych faktorów. Ponieważ
z~założenia $M$ jest równopodzielny, to jeśli dowiedziemy,
że~z~pozostałych warunków wynika, iż~$S = M \setminus \{ 1_{ M } \}$
jest podpółgrupą i~$\bigcap_{ n \in \Nbb } S^{ n } = \emptyset$, to na
mocy twierdzenia 1.2.4, monoid $M$ będzie monoidem wolnym.

To, że~$S$ jest podpółgrupą wynika wprost z~tego, że~grupa jedynki
jest trywialna, tego faktu używaliśmy już nieraz w~tych notatkach. Jak
pokazano w~książce korzystając z~tego, że~monoid $M$ ma trywialną
grupę jedynki, jest skracalny i~każdy element $m \neq 1_{ M }$ ma
skończoną liczbę nietrywialnych lewych faktorów, można udowodnić,
iż~$\bigcap_{ n \in \Nbb } S^{ n } = \emptyset$. To zaś kończy dowód twierdzenia.

Ostatnią rzeczą która mocniej kłuje w~oczy, jest zdanie „Zatem element
$m$ ma dla każdego $n \in \Nbb$, $n$ różnych lewych faktorów, co jest
sprzeczne z~założeniem.”. Bardziej zgrabnym wysłowieniem tego samego
byłoby stwierdzenie, że~dla każdego $n \in \Nbb$ można podać $n$
różnych, nietrywialnych faktorów elementu~$m$, co jest sprzeczne
z~założeniem tym, że~istnieje ich tylko skończona ilość.

\vspace{\spaceFour}





\Str{12} Twierdzenie 1.2.6 powinno być dalej prawdą, gdy zamiast skończonej
ilości nietrywialny lewych faktorów, będziemy mówić o~skończonej ilości
prawych faktorów.

\vspace{\spaceFour}





\Str{13} Może wydawać~się zastanawiające, że~podmonoid wolnego
monoid, nie musi sam być monoidem wolny, dlatego zatrzymamy~się nad tą
kwestią nieco dłużej. Rozpatrzmy monoid $\Acal_{ 1 }$ i~$\Acal_{ 2 }$
zdefiniowane wedle wzorów
\eqref{eq:Forys-Forys-35-A}-\eqref{eq:Forys-Forys-35-B}. Jak już
zostało powiedziane $\Acal_{ 1 }$ jest wolnym monoidem słownym, jest
więc w~oczywisty sposób monoidem wolnym, $\Acal_{ 2 }$ nie jest
monoidem wolnym, pomimo tego, że~$\Acal_{ 2 }$ jest podmonoidem
$\Acal_{ 1 }$.

Nieformalne wyjaśnienie dlaczego $\Acal_{ 2 }$ nie jest monoidem
wolnym, nasuwa~się samo. Nie zawiera on swojej bazy
$B = \{ a \} \nsubseteq \Acal_{ 2 }$. Podchodząc do problemu bardziej
formalnie, aby $M$ był monidem wolnym, musi posiadać podzbiór
$B \subset M$, taki że kanoniczny homomorfizm ustala izomorfizm między
wolnym monoidem językowym $B^{ * }$ i~monoidem~$M$. Ujmując to
inaczej, za pomocą kanonicznego homomorfizmu \textit{każdy} ciąg
elementów zbioru $B$ można utożsamić z~jednym i~tylko jednym
elementem~$M$.

Niech dany będzie wolny monoid $M_{ 1 }$ i~jego podmonoid właściwy
$M_{ 2 } \nsubseteq M_{ 1 }$. Niech $B$ oznacza bazę monoidu
$M_{ 1 }$. Wówczas $B \nsubseteq M_{ 2 }$. Gdyby bowiem było
$B \subset M_{ 2 }$, to ponieważ $M_{ 2 }$ jest zamknięty ze~względu
na działanie wewnętrzne, mielibyśmy $M_{ 1 } \subset M_{ 2 }$. Nie
powinno nas więc dziwić, że~jeśli nawet monoid $M_{ 1 }$ posiada
podzbiór $B$ który stanowi jego bazą, to podmonoid $M_{ 2 }$ może już
podzbioru o~takich własnościach nie posiadać. Inaczej mówiąc,
niezależnie od tego jaki zbiór $C \subset M_{ 2 }$ to albo nie każdy
element $M_{ 2 }$ będzie można przedstawić za pomocą kanonicznego
homomorfizmu jako ciąg elementów $C$, albo dwa różne ciągi będą
przedstawiać ten sam element. W~przypadku monoidu $\Acal_{ 2 }$ mamy
do czynienia z~drugim przypadkiem.

\vspace{\spaceFour}





\StrWg{13}{8} Nie rozumiem czemu w~tej linii wprowadzono
symbol~$T$ na oznaczenie półgrupy $A^{ * } \setminus 1_{ A^{ * } }$,
która do tej pory była oznaczana symbolem~$S$. By nie wprowadzić
jeszcze więcej zamieszania, będziemy podążali śladem tej książeczki
i~używali symbolu~$T$ w~tych samych miejscach, w~których ona też go
używa.

\vspace{\spaceFour}





\Str{13} Dowód twierdzenia~1.3.1 można przedstawić w~prostszy
sposób, co tutaj zrobimy. Jego plan jest dość prosty. Pokażemy,
że~jeśli $A^{ * }$ jest wolnym monoidem, to dowolny jego podmonoid
$M \subset A^{ * }$ posiada dwie następujące własności. Po pierwsze
$M^{ + } = M \setminus \{ 1_{ A^{ * } } \}$ jest półgrupą. Po drugie,
$\bigcap_{ n \in \Nbb } ( M^{ + } )^{ n } = \emptyset$. Na~mocy tego,
że~$M^{ + }$ jest półgrupą zachodzi następujący ciąg inkluzji.
\begin{equation}
  \label{eq:Forys-Forys-66}
  M^{ + } \supset ( M^{ + } )^{ 2 } \supset ( M^{ + } )^{ 3 } \supset \ldots
\end{equation}
Ciąg ten analizujemy z~wykorzystaniem własności
$\bigcap_{ n \in \Nbb } ( M^{ + } )^{ n } = \emptyset$, w~ten sam
sposób co dowodzie twierdzenia 1.2.4, tym samym uzyskując dowodząc,
że~monoid $M$ jest generowane przez zbiór
$X = M^{ + } \setminus ( M^{ + } )^{ 2 }$. Ostatnim krokiem, jest pokazanie,
że~$X$ jest istotnie najmniejszym zbiorem generatorów.

Jak wiemy z twierdzenia \eqref{thm:Forys-Forys-01} fakt, że~$M^{ + }$ jest
półgrupą jest równoważny temu, że~grupa jedynki monoidu $M$~jest trywialna.
Ponieważ grupa jedynki $M$ musi zawierać~się w~grupie jedynki $A^{ * }$, to
ponieważ ta~ostatnia jest trywialna, ta pierwsza też musi taką być.

Dowód drugiej własności opiera się na zauważeniu, że
\begin{equation}
  \label{eq:Forys-Forys-67}
  \bigcap_{ n \in \Nbb } ( M^{ + } )^{ n } \subset \bigcap_{ n \in \Nbb } T^{ n } = \emptyset,
\end{equation}
gdzie $T = A^{ * } \setminus \{ 1_{ A^{ * } } \}$. Ponieważ dowód, że~z~tych dwóch
własności wynika, iż~każdy element $M$ można rozłożyć na elementy zbioru
$X$, jest kopią rozumowania przedstawionego w~dowodzie twierdzenia 1.2.4,
nie będziemy go tu przytaczać.

Ostatnia własność, że~$X = ( M^{ + } ) \setminus ( M^{ + } )^{ 2 }$ jest minimalnym
zbiorem generatorów, wynika z~tego, co zostało przez nas wykazane, przy
okazji omawiania wniosku~1.2.2. Mianowicie, że~dla dowolnego monoidu $M$
(półgrupy), każdy zbiór generatorów musi zawierać zbiór $S \setminus S^{ 2 }$,
$S = M \setminus \{ 1_{ M } \}$.

Na koniec zauważmy, że~gdyby podmonoid $M$ był równopodzielny, to~na~mocy
twierdzenia~1.2.4 byłby monoidem wolnym. Jak pokazuje przykład monoidów
$\Acal_{ 1 }$ i~$\Acal_{ 2 }$
(zob.~wzory~\eqref{eq:Forys-Forys-35-A}-\eqref{eq:Forys-Forys-35-B}),
nie musi to wcale zachodzić. Monoid $\Acal_{ 1 }$ jest wolny,
$\Acal_{ 2 } \subset \Acal_{ 1 }$, natomiast podmonoid $\Acal_{ 2 }$ nie jest
równopodzielny.

Rozważmy bowiem elementy $b_{ 1 } = a^{ 2 }$, $b_{ 2 } = a^{ 4 }$,
$d_{ 1 } = d_{ 2 } = a^{ 3 }$. Zachodzi
\begin{equation}
  \label{eq:Forys-Forys-68}
  b_{ 1 } b_{ 2 } = a^{ 6 } = d_{ 1 } d_{ 2 }.
\end{equation}
Ponieważ $| b_{ 1 } | < | d_{ 1 } |$, więc sytuacja $b_{ 1 } = d_{ 1 } u$,
$u \in \Acal_{ 2 }$, jest wykluczona. W~takim wypadku pozostaje rozpatrzyć
przypadek $d_{ 1 } = b_{ 1 } v$. Oznacza to, że musiałby istnieć element
$v \in \Acal_{ 2 }$, taki~że
\begin{equation}
  \label{eq:Forys-Forys-69}
  a^{ 3 } = a^{ 3 } v.
\end{equation}
W~oczywisty sposób, taki element nie istnieje w~$\Acal_{ 2 }$, więc ten
monoid nie jest równopodzielny.

\vspace{\spaceFour}





\Str{13} Użyte na tej stronie oznaczenie $w M$ jest tradycyjnym
zapisem, używanym dla specjalnego przypadku mnożenia w~monoidzie (półgrupie)
$\Pcal( M )$. Znaczenie tych symboli jest następujące:
\begin{equation}
  \label{eq:Forys-Forys-70}
  w \cdot A := \{ w \} \cdot A, \quad
  A \in \Pcal( M ).
\end{equation}

\vspace{\spaceFour}





\Str{13--14} Fragment dowodu twierdzenia 1.3.2, poświęcony implikacji
$3 \Rightarrow 1$ jest w~istocie dowodem implikacji $2 \Rightarrow 1$. Niestety, w~tej chwili
nie potrafię podać dowodu implikacji $3 \Rightarrow 1$, muszę więc poprzestać
na~dokładnym omówieniu przedstawionego dowodu i~wyjaśnieniu gdzie leży
problem.

Zaczniemy od przyjrzenia~się warunkom 2 i~3.
\begin{itemize}

\item[2)] Dla dowolnego $w \in A^{ * }$ jeśli $M w \cap M \neq \emptyset$
  i~$w M \cap M \neq \emptyset$, to~$w \in $M.

\item[3)] Dla dowolnego $w \in A^{ * }$ jeśli $M w \cap M \cap w M \neq \emptyset$,
  to~$w \in M$.

\end{itemize}
W~przypadku warunku 2 z~tego, że~istnieją takie $m_{ 1 } \in M$
i~$m_{ 2 } \in M$, że~$m_{ 1 } w \in M$ oraz $w m_{ 2 } \in M$, możemy wnioskować,
iż~$w \in M$. Dla przypadku 3, wymagane jest by~istniały takie elementy
$m_{ 1 }, m_{ 2 } \in M$, że~$m_{ 1 } w = w m_{ 2 } \in M$ i~dopiero na mocy tego
możemy wnioskować, iż~$w \in M$.

Widzimy więc, że~warunek~3 jest ściśle mocniejszy, niż warunek~2 oraz że
jeśli przyjmiemy warunek 2 jako prawdziwy, jesteśmy w~stanie dowieść
prawdziwości warunku~3.

Przejdźmy teraz do dowodu implikacji $2 \Rightarrow 1$, który jest przedstawiony
jako dowód implikacji $3 \Rightarrow 1$. Dowód ten prowadzony jest metodą
\textit{reductio ad~absurdum}, czyli korzystając z~reguły
\begin{equation}
  \label{eq:Forys-Forys-71}
  ( p \Rightarrow q ) \Leftrightarrow \; \sim ( p \wedge ( \sim q ) ).
\end{equation}
Przyjmijmy więc, że~$M$~nie jest monoidem wolnym. Ponieważ jest on
podmonoidem monoidu wolnego $A^{ * }$ posiada on bazę~$C$, w~takim razie
jedyna możliwość by nie był on wolny, jest tak, że~pewien element posiada
dwa różne rozkłady na elementy tej bazy. Niech $w \in C^{ + }$ będzie
elementem z~dwoma różnymi rozkładami, który ma najmniejszą długość w~$M$.

Przez to, że~$w$ jest elementem o~najmniejszej długości, rozumiemy to,
że~jeśli przez $d \in \Nbb$ oznaczy długość najkrótszego z~rozkładów~$w$
na~elementy $C$, to nie istnieje element $v \in M$, taki że co najmniej
jeden z~jego rozkładów na elementy zbioru $C$ ma długość silnie mniejszą
od~$d$.

Niech teraz
\begin{equation}
  \label{eq:Forys-Forys-72}
  w = c_{ i_{ 1 } } c_{ i_{ 2 } } \ldots c_{ i_{ p } } =
  c_{ j_{ 1 } } c_{ j_{ 2 } } \ldots c_{ j_{ p } },
\end{equation}
gdzie $c_{ i_{ k } }, c_{ j_{ l } } \in C$, $k = 1, \ldots, p$, $l = 1, \ldots, q$ oraz
zachodzi $k > 1$, $l > 1$. Gdyby bowiem $k = l = 1$ to oba rozkłady byłby
identyczne. Gdyby zaś $k = 1$ i~$l > 1$, to oznaczałoby,
że~istnieje $c_{ i_{ 1 } } \in C = ( M^{ + } ) \setminus ( M^{ + } )^{ 2 }$, które należy
do~$( M^{ + } )^{ 2 }$, bo~każdy wyraz postaci
$c_{ j_{ 1 } } c_{ j_{ 2 } } \ldots c_{ j_{ q } }$ należy do $( M^{ + } )^{ 2 }$,
i~otrzymujemy sprzeczność. Analogiczne rozumowanie da~się przeprowadzić
w~stosunku do $l = 1$ i~$k > 1$, którego nie będziemy jawnie przeprowadzać,
co kończy dowód tego, iż~$k > 1$ i~$l > 1$.

Ponadto bez ograniczania ogólności możemy przyjąć,
że~$c_{ i_{ 1 } } \neq c_{ j_{ 1 } }$ i~$c_{ i_{ p } } \neq c_{ j_{ q } }$. Gdyby było
inaczej, przykładowo $c_{ i_{ 1 } } = c_{ j_{ 1 } }$, to na mocy tego,
że~$A^{ * }$ jest monoidem skracalnym, po skróceniu pierwszych wyrazów
w~dwóch rozkładach $w$ ze~wzoru \eqref{eq:Forys-Forys-72}, otrzymalibyśmy
\begin{equation}
  \label{eq:Forys-Forys-73}
  w' = c_{ i_{ 2 } } c_{ i_{ 3 } } \ldots c_{ i_{ p } } =
  c_{ j_{ 2 } } c_{ j_{ 3 } } \ldots c_{ j_{ q } }.
\end{equation}
Czyli $w'$ jest element zawartym w~$C^{ + }$, który ma dwa różne rozkłady
na~elementy $C$ i~jeden z~tych rozkładów jest krótszy od~najkrótszego
rozkładu~$w$. Ta uwaga powinna wyjaśnić, czemu w~książeczce w~linii 12
na~stronie 14 znajdujemy wzór $c_{ i_{ 1 } } \neq c_{ j_{ 1 } }$.

W~następnym kroku zapisujemy \eqref{eq:Forys-Forys-72} jako
\begin{equation}
  \label{eq:Forys-Forys-74}
  c_{ i_{ 1 } } ( c_{ i_{ 2 } } c_{ i_{ 3 } } \ldots c_{ i_{ p } } ) =
  c_{ j_{ 1 } } ( c_{ j_{ 2 } } c_{ j_{ 3 } } \ldots c_{ j_{ q } }).
\end{equation}
Korzystając z~równopodzielności $A^{ * }$, oznacza to, że~zachodzi jedna
z~dwóch możliwości. Albo
\begin{subequations}
  \begin{align}
    \label{eq:Forys-Forys-75-A}
    c_{ i_{ 1 } } &= c_{ j_{ 1 } } u, \\
    \label{eq:Forys-Forys-75-B}
    c_{ j_{ 2 } } c_{ j_{ 3 } } \ldots c_{ j_{ q } }
                  &= u ( c_{ i_{ 2 } } c_{ i_{ 3 } } \ldots c_{ i_{ p } } ),
  \end{align}
\end{subequations}
albo
\begin{subequations}
  \begin{align}
    \label{eq:Forys-Forys-76-A}
    c_{ j_{ 1 } } &= c_{ i_{ 1 } } v, \\
    \label{eq:Forys-Forys-76-B}
    c_{ i_{ 2 } } c_{ i_{ 3 } } \ldots c_{ i_{ p } }
                  &= v ( c_{ j_{ 2 } } c_{ j_{ 3 } } \ldots c_{ j_{ q } } ).
  \end{align}
\end{subequations}
Rozpatrzmy pierwszy przypadek. Jest oczywiste,
że~$c_{ i_{ 1 } }, c_{ j_{ 1 } }, c_{ i_{ 2 } } c_{ i_{ 3 } } \ldots c_{ i_{ p } },
c_{ j_{ 2 } } c_{ j_{ 3 } } \ldots c_{ j_{ q } } \in C^{ + } \subset M$. Ponieważ
\begin{subequations}
  \begin{align}
    \label{eq:Forys-Forys-77-A}
    c_{ i_{ 1 } } &\in M u \cap M \neq \emptyset, \\
    \label{eq:Forys-Forys-77-B}
    c_{ j_{ 2 } } c_{ j_{ 3 } } \ldots c_{ j_{ q } } &\in M \cap u M \neq \emptyset,
  \end{align}
\end{subequations}
więc na mocy warunku 2, $u \in M$.

Zatrzymajmy~się na~chwilę w~tym miejscu, bowiem wedle mojego rozeznania,
to jest moment kiedy dowód implikacji $3 \Rightarrow 1$ załamał~się. Wedle dowodu,
jeśli przyjmiemy warunek~3 za prawdziwy, to wzory
\eqref{eq:Forys-Forys-76-A}--\eqref{eq:Forys-Forys-76-B}
i~\eqref{eq:Forys-Forys-77-A}--\eqref{eq:Forys-Forys-77-B} pozwalają nam
stwierdzić, iż~$u \in M$. Ale tak nie jest, bo jak wskazywaliśmy powyżej, by
skorzystać z~warunku trzy musielibyśmy pokazać,~że
\begin{equation}
  \label{eq:Forys-Forys-78}
  c_{ i_{ 1 } } = c_{ j_{ 2 } } c_{ j_{ 3 } } \ldots c_{ j_{ q } },
\end{equation}
bo wówczas $c_{ i_{ 1 } } \in M u \cap M \cap u M \neq \emptyset$ i~to pozwala nam stwierdzić,
iż~$u \in M$. Niestety nie wiem jak można poprawić to rozumowanie.

Będziemy teraz kontynuowali dowód implikacji $2 \Rightarrow 1$. Mamy teraz dwie
możliwości\footnote{Zauważmy, że~tym miejscu korzystamy z~zasady
  wyłączonego środka, więc nie byłoby to akceptowalne w~matematyce
  konstrukcyjnej.}: $u \neq 1_{ A^{ * } }$ lub $u = 1_{ A^{ * } }$. W~pierwszym
przypadku $c_{ i_{ 1 } } \in C = ( M^{ + } ) \setminus ( M^{ + } )^{ 2 }$, które
jednocześnie należy do $( M^{ + } )^{ 2 }$, otrzymujemy więc sprzeczność.
Warto zwrócić uwagę, że~już wcześniej w~tym dowodzie napotkaliśmy tego typu
problem.

Analizy przypadku $u = 1_{ A^{ * } }$ jest bardzo prosta. Mamy wówczas
$c_{ i _{ 1 } } = c_{ j_{ 1 } }$, co wobec tego co zostało powiedziane
wcześniej, nie jest możliwe i~otrzymana sprzeczność kończy dowód implikacji
$2 \Rightarrow 1$.

\vspace{\spaceFour}





\StrWg{14}{12} Odstępy w~tej linii są stanowczo za~duże.

\vspace{\spaceFour}





\StrWd{14}{9} Odstęp nad tą linią jest zbyt duży.

\vspace{\spaceFour}





\Str{16} Warto zauważyć, że~dopuszczamy by ciąg przepisujący słowo
$x$ na~słowo $y$ miał długość 0. Dokładniej, oznacza to, że~każdym systemie
przepisującym słowo $w$ można przepisać na nie samo. Jest to bardzo rozsądne
założenie.

\vspace{\spaceFour}





\Str{16} Oznaczenie „$RS$” pochodzi zapewne od angielskiego
\textit{rewriting system}. Poza tym, w~mojej opini zapis „$\textrm{RS}$”
wyglądałby lepiej.

\vspace{\spaceFour}





\Str{16} W~oznaczeniu $L_{ gen }( RS, I )$, „gen” pochodzi zapewne
od~angielskiego \textit{generated}. Standardowo, w~mojej opinii zapis
\begin{equation}
  \label{eq:Forys-Forys-79}
  L_{ \textrm{gen} }( \textrm{RS}, I )
\end{equation}
wyglądałoby znacznie lepiej, niż ten stosowany w~tej książeczce.

\vspace{\spaceFour}





\Str{16} W~oznaczeniu $L_{ acc }( RS, I )$, „acc” pochodzi zapewne
od~angielskiego \textit{accepted}, choć jestem tego mniej pewien, niż tego
jak należy rozwinąć skrót „gen”. Standardowo, w~mojej opinii zapis
\begin{equation}
  \label{eq:Forys-Forys-80}
  L_{ \textrm{acc} }( \textrm{RS}, I )
\end{equation}
wyglądałby znacznie lepiej, niż ten stosowany w~tej książeczce.

\vspace{\spaceFour}





\Str{17} W~tej książeczce brakuje szerszego objaśnienia jaki jest
sens symboli terminalnych i~nieterminalnych, które pozwoliłoby wyrobić
sobie podstawową intuicję na ich temat. W~tym momencie nie jestem w~stanie
samemu podać takiego objaśnienia.

\vspace{\spaceFour}





\Str{17} Ponieważ w~gramatyce zbiór praw przepisujących~$P$ jest
podzbiorem $( V_{ N } \cup V_{ T } )^{ + } \cup ( V_{ N } \cup V_{ T } )^{ * }$,
definicja przepisywania słów wymaga tutaj dodatkowego uściślenia, przy czym
wystarczy ograniczyć~się do~doprecyzowania przypadku bezpośredniego
przepisywania. Definicja przepisywania pośredniego dla gramatyki jest
bowiem bardzo  oczywistą modyfikacją analogicznej definicji dla systemu
przepisującego.

Dla gramatyki $G = ( V_{ N }, V_{ T }, P, v_{ 0 } )$ definiujemy bezpośrednie
przepisywanie słów, należących do wolnego monoidu językowego
$( V_{ N } \cup V_{ T } )^{ * }$ w~następujący sposób. Niech
$x, y \in ( V_{ N } \cup V_{ T } )^{ * }$. Mówimy, że~gramatyka przepisuje
bezpośrednio słowo $x$ na $y$, co zapisuje jak poprzednio przez
\begin{equation}
  \label{eq:Forys-Forys-81}
  x \mapsto y,
\end{equation}
jeśli istnieje reguła $v \to w \in P$ i~takie słowa
$w_{ 1 }, w_{ 2 } \in ( V_{ N } \cup V_{ T } )^{ * }$, że $x = w_{ 1 } u w_{ 2 }$,
$y = w_{ 1 } v w_{ 2 }$.

Definicja przepisywania pośredniego jest tak oczywistą modyfikacją
definicji podanej dla systemów przepisujących, że~nie będziemy jej
formułować

\vspace{\spaceFour}





\Str{18} Przyjmiemy, że~dla danej gramatyki będziemy używać
oznaczenia $M = ( V_{ N } \cup V_{ T } )^{ * }$ i~tym samym słowo puste będziemy
oznaczać przez $1_{ M }$. Zgodnie z~tą zasadą poprawiny oznaczenia
pojawiające~się po raz pierwszy na tej stronie i,~jeśli będzie trzeba,
w~dalszej części tej książeczki.

\vspace{\spaceFour}





\StrWd{20}{10} W~tym miejscu, tak ja na stronie~7, symbol
$( A^{ * } )^{ 2 }$, nie oznacza wyniku mnożenia w~monoidzie
$\Pcal( A^{ * } )$, czyli $A^{ * } \cdot A^{ * }$, lecz iloczyn kartezjański
$A^{ * } \times A^{ * }$. W~dalszym ciągu zwykle nie będziemy już opisywać co
w~danym miejscu oznacza ten symbol, licząc że czytelnik będzie w~stanie sam
to wywnioskować. Postaramy~się jednak komentować miejsca, gdzie zrozumienie
poprawnego sensu tego symbolu jest wyjątkowo trudne.

\vspace{\spaceFour}





\StrWg{21}{7} W~tej linii znajdujemy tajemniczo wyglądający wzór
\begin{equation}
  \label{eq:Forys-Forys-82}
  \sim_{ \Acal } \; \subseteq \; \sim_{ \Bcal },
\end{equation}
gdzie symbolu $\sim_{ \Acal }$ (analogicznie $\sim_{ \Bcal }$) używamy do zapisania,
że~dwa elementy są ze sobą w~relacji $\Acal$. Przykładowo fakt, że~elementy
$a$ i~$b$ są w~relacji $\Acal$, zapisujemy jako $a \sim_{ \Acal } b$. Jak
jednak można napisać, że~symbol $\sim_{ \Acal }$ zawiera~się w~symbolu
$\sim_{ \Bcal }$? Aby to wyjaśnić wprowadzimy dodatkową notację.

W~systemie teorii mnogości, w~którym pracujemy w~tej książeczce, relacja
w~zbiorze $X$ jest pewnym podzbiorem iloczynu kartezjańskiego~$X \times X$.
Oznaczmy więc przez $R \subset X \times X$ zbiór wyznaczający tę relację, zaś przez
$\sim_{ R }$ symbol którym będziemy oznaczać, że~dwa elementy~są w~zadanej
relacji. Jeśli więc $a$ i~$b$ są w relacji $R$ to będziemy pisać
\begin{equation}
  \label{eq:Forys-Forys-83}
  a \sim_{ R } b.
\end{equation}

Niech będą dane dwie relacje $R_{ 1 }$ i~$R_{ 2 }$, przy czym
$R_{ 1 }, R_{ 2 } \subset X \times X$. Jeśli
\begin{equation}
  \label{eq:Forys-Forys-84}
  R_{ 1 } \subset R_{ 2 },
\end{equation}
to mówimy, że relacja $R_{ 2 }$ jest mocniejsza od relacji $R_{ 1 }$ lub
że~relacja $R_{ 1 }$ jest słabsza od relacji $R_{ 2 }$. W~takim razie zapis
\begin{equation}
  \label{eq:Forys-Forys-85}
  \sim_{ R_{ 1 } } \; \subset \; \sim_{ R_{ 2 } },
\end{equation}
jest tylko inną, chyba mniej poprawną, wersją wzoru
\eqref{eq:Forys-Forys-84}.














% ##################
\newpage

\CenterBoldFont{Błędy}

\vspace{\spaceFive}


\begin{center}

  \begin{tabular}{|c|c|c|c|c|}
    \hline
    Strona & \multicolumn{2}{c|}{Wiersz} & Jest
                              & Powinno być \\ \cline{2-3}
    & Od góry & Od dołu & & \\
    \hline
    5  & 11 & & $\forall x, y, z \in S$ & $\forall \, x, y, z \in S,$ \\
    5  & 15 & & $\forall x \in M$ & $\forall \, x \in M,$ \\
    5  & &  1 & $x \in \textrm{\textbf{M}}$ & $x \in M$ \\
    5  & &  1 & $\exists b \in B\; ,$ & $\exists \, b \in B,$ \\
    5  & &  3 & $\in \!\! \Pcal( M )$ & $\in \Pcal( M )$ \\
    5  & & 11 & $x_{ 1 } ... x_{ n }$ & $x_{ 1 } \ldots x_{ n }$ \\
    5  & & 11 & $x ... x$ & $x \ldots x$ \\
    5  & & 12 & $( \textrm{\textbf{M}}, \cdot, 1_{ \textrm{\textbf{M}}} )$
           & $( M, \cdot, 1_{ M } )$ \\
    5  & & 13 & $( \textrm{\textbf{S}}, \cdot )$ & $( S, \cdot )$ \\
    6  & &  1 & \textit{dla} $x \in S^{ 1 }$ $\rho_{ a }( x )= a x$.
                & $\rho_{ a }( x ) = a x$ dla $x \in S^{ 1 }$. \\
    6  &  9 & & $( S, \cdot )${ }, { }{ }$( S', \ast )$
              & $( S, \cdot )$, $( S', \ast )$ \\
    6  & 11 & & $\forall x, y \in S$ & $\forall \, x, y \in S,$ \\
    6  & 14 & & $\forall x, y \in M$ & $\forall \, x, y \in M,$ \\
    6  & & 14 & $x \cdot y = 1$ & $x \cdot y = 1_{ M }$ \\
    7  &  8 & & $\{ 1 \} \bigcup X \bigcup X^{ 2 } \bigcup ... \bigcup X^{ n } \bigcup \ldots$
           & $\{ 1_{ M } \} \bigcup X \bigcup X^{ 2 } \bigcup \ldots \bigcup X^{ n } \bigcup \ldots$ \\
    7  & 15 & & $X \bigcup X^{ 2 } \bigcup ... \bigcup X^{ n } \bigcup \ldots$
           & $X \bigcup X^{ 2 } \bigcup \ldots \bigcup X^{ n } \bigcup \ldots$ \\
    7  & &  3 & $M / \rho$ & $M_{ / \rho }$ \\
    7  & &  4 & $S / \rho$ & $S_{ / \rho }$ \\
    7  & &  8 & $\forall x, y, z \in S$ & $\forall \, x, y, z \in S,$ \\
    7  & & 10 & $\forall x, y, z \in S$ & $\forall \, x, y, z \in S,$ \\
    7  & & 12 & $\forall x, y, z \in S$ & $\forall \, x, y, z \in S,$ \\
    8  & &  4 & $( a_{ 1 }, ..., a_{ n } )$ & $( a_{ 1 }, \ldots, a_{ n } )$ \\
    8  & &  4 & $a_{ 1 } ... a_{ n }$ & $a_{ 1 } \ldots a_{ n }$ \\
    8  & &  7 & $n \!\! = \!\! 0$ & $n = 0$ \\
    8  & &  8 & $( a_{ 1 }, ..., a_{ n } ) \cdot ( b_{ 1 }, ..., b_{ m } )$
           & $( a_{ 1 }, \ldots, a_{ n } ) \cdot ( b_{ 1 }, \ldots, b_{ m } )$ \\
    8  & &  8 & $( a_{ 1 }, ..., a_{ n }, b_{ 1 }, ..., b_{ m } )$
           & $( a_{ 1 }, \ldots, a_{ n }, b_{ 1 }, \ldots, b_{ m } )$ \\
    8  & & 10 & $( a_{ 1 }, ..., a_{ n } )$ & $( a_{ 1 }, \ldots, a_{ n } )$ \\
    8  & & 10 & $n \geq 0\;\;,$ & $n \geq 0,$ \\
    9  &  1 & & wolną półgrupę & wolną półgrupę słowną \\
    \hline
  \end{tabular}





  \newpage

  \begin{tabular}{|c|c|c|c|c|}
    \hline
    Strona & \multicolumn{2}{c|}{Wiersz} & Jest
                              & Powinno być \\ \cline{2-3}
    & Od góry & Od dołu & & \\
    \hline
    9  &  2 & & wolny monoid & wolny monoid słowny \\
    9  &  2 & & wolna półgrupa & wolna półgrupa słowna \\
    9  &  3 & & wolnych monoidów (półgrup)
                & wolnych monoidów (półgrup) słownych \\
    9  &  3 & & nazywamy & słownego nazywamy \\
    9  &  4 & & nazywamy & słownego nazywamy \\
    9  &  6 & & \textit{monoidu} $A^{ * }$
    & \textit{monoidu słownego} $A^{ * }$ \\
    9  &  6 & & nazywamy & słownego nazywamy \\
    9  &  6 & & monoid wolny & słowny monoid wolny \\
    9  &  6 & & półgrupa wolna & słowny półgrupa wolna \\
    9  & 12 & & wolnych monoidów & wolnych monoidów słownych \\
    9  & 12 & & wolnych półgrupach & wolnych półgrupach słownych \\
    9  & 14 & & monoidzie & monoidzie słownym \\
    9  & 19 & & \textit{monoidu $A^{ * }$}
    & wolnego monoidu słownego $A^{ * }$ \\
    9  & 20 & & $\{ 1 \}$ & $\{ 1_{ M } \}$ \\
    9  & 22 & & $f : A \mapsto M$ & $f : A \to M$ \\
    9  & 23 & & $h : A^{ * } \mapsto M$ & $h : A^{ * } \to M$ \\
    9  & 25 & & \textit{homomorfizm}
    & homomorfizm z~wolnego monoidu słownego w~dowolny monoid \\
    9  & &  9 & $\to \textrm{\textbf{M}}$ & $\to M$ \\
    10 & 12 & & $b_{ 1 } ... b_{ k }$ & $b_{ 1 } \ldots b_{ k }$ \\
    10 & 12 & & $1, ..., k$ & $1, \ldots, k$ \\
    10 & 18 & & $h : A^{ * } \mapsto M$
    & $h : A^{ * } \to M$, gdzie $A^{ * }$ jest wolnym monoidem słownym. \\
    10 & 18 & & $id_{ A } : A \mapsto M$ & $id_{ A } : A \to M$ \\
    10 & 20 & & $a_{ 1 } ... a_{ n }$ & $a_{ 1 } \ldots a_{ n }$ \\
    10 & 20 & & $1, ..., n$ & $1, \ldots, n$ \\
    10 & 22 & & $h( a_{ 1 } ... a_{ n } )$ & $h( a_{ 1 } \ldots a_{ n } )$ \\
    10 & 22 & & $h( a_{ 1 } ) ... h( a_{ n } )$
           & $h( a_{ 1 } ) \ldots h( a_{ n } )$ \\
    10 & 22 & & $a_{ 1 } ... a_{ n }$ & $a_{ 1 } \ldots a_{ n }$ \\
    10 & & 10 & $\{ a^{ n } \quad : \quad n \geq 2 \}$ & $\{ a^{ n } \; : \; n \geq 2 \}$ \\
    10 & & 11 & \textit{generatorów.}
    & generatorów i~jest równa $S \setminus S^{ 2 }$, gdzie
      $S = M \setminus \{ 1_{ M } \}$. \\
    11 &  1 & & $\{ 1 \}$ & $\{ 1_{ M } \}$ \\
    \hline
  \end{tabular}





  \newpage

  \begin{tabular}{|c|c|c|c|c|}
    \hline
    Strona & \multicolumn{2}{c|}{Wiersz} & Jest
                              & Powinno być \\ \cline{2-3}
    & Od góry & Od dołu & & \\
    \hline
    11 &  9 & & $\forall n \in \!\! \Nbb$ & $\forall \, n \in \Nbb$ \\
    11 & 10 & & $( u u^{ -1 } ).....$ & $( u u^{ - 1 } ) \ldots$ \\
    11 & 12 & & $n \in \!\! \Nbb$ & $n \in \Nbb$ \\
    11 & 12 & & $\bigcap_{ n \in \Nbb } \, \mathbf{S}^{ n }$
    & $\bigcap_{ n \in \Nbb } \, S^{ n }$ \\
    11 & 15 & & $\mathbf{S} \supset \mathbf{S}^{ 2 } \supset \mathbf{S}^{ 3 } \ldots
                \supset \mathbf{S}^{ k } \supset \mathbf{S}^{ k + 1 } ...$
    & $S \supset S^{ 2 } \supset S^{ 3 } \supset \ldots \supset S^{ k } \supset S^{ k + 1 } \supset \ldots$ \\
    11 & 17 & & \textit{( 1.2.1)} & (1.2.1) \\
    11 & 18 & & $S^{ k } \; = 0$ & $S^{ k } = 0$ \\
    11 & 19 & & $\mathbf{S} \supsetneq \mathbf{S}^{ 2 } \supsetneq \mathbf{S}^{ 3 } \ldots
                \supsetneq \mathbf{S}^{ k } \supsetneq \mathbf{S}^{ k + 1 } \ldots$
                & $S \supsetneq S^{ 2 } \supsetneq S^{ 3 } \supsetneq \ldots \supsetneq S^{ k } \supsetneq S^{ k + 1 } \supsetneq \ldots$ \\
    11 & & 15 & \textit{pólgrupy}~$S$. & \textit{zbioru}~$A$. \\
    11 & & 16 & $1, ...k$ & $1, \ldots, k$ \\
    11 & & 17 & $\bigcap_{ n \in \Nbb } \mathbf{S}^{ 2 }$
           & $\bigcap_{ n \in \Nbb } S^{ n }$ \\
    11 & & 18 & $k \in \!\! \Nbb$ & $k \in \Nbb$ \\
    12 &  2 & & $1$ & $1_{ M }$ \\
    12 &  3 & & $\{ 1 \}$ & $\{ 1_{ M } \}$ \\
    12 &  6 & & $( M, \cdot, 1 )$ & $( M, \cdot, 1_{ M } )$ \\
    12 &  6 & & \textit{dla pewnego $t \in M$,}
    & dla $t = s_{ p + 2 } \, s_{ p + 3 } \ldots s_{ q }$ \\
    12 &  6 & & $\{ 1 \}$ & $\{ 1_{ M } \}$ \\
    12 &  9 & & $\{ 1 \}$ & $\{ 1_{ M } \}$ \\
    12 &  9 & & $\bigcap S^{ n }$ & $\bigcap_{ n \in \Nbb } S^{ n }$ \\
    12 & 11 & & \textit{z~poprzedniego twierdzenia.}
    & z~twierdzenia 1.2.4. \\
    12 & 11 & & $\{ 1 \}$ & $\{ 1_{ M } \}$ \\
    12 & 11 & & $\bigcap S^{ n }$ & $\bigcap_{ n \in \Nbb } S^{ n }$ \\
    12 & 12 & & $\{ 1 \}$ & $\{ 1_{ M } \}$ \\
    12 & 12 & & $m \!\! \notin \!\! S^{ k + 1 }$ & $m \notin S^{ k + 1 }$ \\
    12 & 18 & & $\bigcap S^{ n }$ & $\bigcap_{ n \in \Nbb } S^{ n }$ \\
    12 & 21 & & $1$ & $1_{ M }$ \\
    12 & &  2 & $\bigcap S^{ n }$ & $\bigcap_{ n \in \Nbb } S^{ n }$ \\
    12 & &  5 & $1$ & $1_{ M }$ \\
    12 & &  6 & $1 =$ & $1_{ M } =$ \\
    12 & &  6 & $= 1$ & $= 1_{ M }$ \\
    % & &   &  &  \\
    % &  & &  &  \\
    % &  & &  &  \\
    % &  & &  &  \\
    % & &  &  &  \\
    % & & &  &  \\
    % & & &  &  \\
    % &  & &  &  \\
    % & &  &  &  \\
    % &   & &  &  \\
    % & & & & \\
    % & & & & \\
    % & & & & \\
    % & & & & \\
    % & & & & \\
    % & & & & \\
    % & & & & \\
    % & & & & \\
    % & & & & \\
    % & & & & \\
    % & & & & \\
    \hline
  \end{tabular}





  \newpage

  \begin{tabular}{|c|c|c|c|c|}
    \hline
    Strona & \multicolumn{2}{c|}{Wiersz} & Jest
                              & Powinno być \\ \cline{2-3}
    & Od góry & Od dołu & & \\
    \hline
    12 & &  9 & $s_{ 1 }, \hspace{0.5em} s_{ 1 } s_{ 2 },$
    & $s_{ 1 }, s_{ 2 } \, s_{ 3 },$ \\
    12 & & 10 & $\bigcap S^{ n }$ & $\bigcap_{ n \in \Nbb } S^{ n }$ \\
    12 & & 11 & $\bigcap S^{ n }$ & $\bigcap_{ n \in \Nbb } S^{ n }$ \\
    12 & & 11 & $M \setminus \{ 1 \} $ & $M \setminus \{ 1_{ M } \} $ \\
    12 & & 14 & $m \in M$,{ } $m \neq 1$ & $m \in M$, $m \neq 1_{ M }$ \\
    13 &  5 & & $m_{ 1 } w \in M$,{ } $w m_{ 2 } \in M$
                & $m_{ 1 } w \in M$, $w m_{ 2 } \in M$ \\
    13 &  7 & & $\{ 1 \}$ & $\{ 1_{ A^{ * } } \}$ \\
    13 &  8 & & $\{ 1 \}$ & $\{ 1_{ A^{ * } } \}$ \\
    13 &  8 & & $\bigcap ( \mathbf{M}^{ + } )^{ n }$
           & $\bigcap_{ n \in \Nbb } ( M^{ + } )^{ n }$ \\
    13 &  9 & & $( \mathbf{M}^{ + } )^{ n }$ & $( M^{ + } )^{ n }$ \\
    13 &  9 & & $m \in ( \mathbf{M}^{ + } )$ & $m \in ( M^{ + } )$ \\
    13 & 13 & & $b_{ 1 }, \ldots b_{ k } \in B$ & $b_{ 1 }, \ldots, b_{ k } \in B$ \\
    13 & 20 & & monoidu $M$
    & monoidu $M$, będącego podmonoidem wolnego monoidu $A^{ * }$, \\
    14  &  2 & & $1$ & $1_{ A^{ * } }$ \\
    14  &  3 & & $1$ & $1_{ A^{ * } }$ \\
    16  &  3 & & N.Chomsky’ego & N.~Chomsky’ego \\
    16  & 10 & & $u \;\;\; \to \;\;\; v \; \in \; P$ & $u \to v \in P$ \\
    16  & 14 & & a{ }{ } $x, y \; \in \; A^{ * }$ & a~$x, y \in A^{ * }$ \\
    16  & 22 & & $w_{ 1 }, ..., w_{ k }$ & $w_{ 1 }, \ldots, w_{ k }$ \\
    16  & 22 & & $w_{ 0 } = x$,{ } $w_{ k } = y$
           & $w_{ 0 } = x$, $w_{ k } = y$ \\
    16  & 23 & & $0, 1, ...k - 1$ & $0, 1, \ldots, k - 1$ \\
    16  & &  3 & $\mapsto^{ * } x \, ,$ & $\mapsto^{ * } x,$ \\
    16  & &  6 & $\mapsto^{ * } w \, ,$ & $\mapsto^{ * } w,$ \\
    16  & & 12 & $w_{ 0 } \mapsto \;\, w_{ 1 } \mapsto \;\, ... \mapsto w_{ k }$
    & $w_{ 0 } \mapsto w_{ 1 } \mapsto \ldots \mapsto w_{ k }$ \\
    16  & & 15 & $w_{ 1 }, ..., w_{ k }$ & $w_{ 1 }, \ldots, w_{ k }$ \\
    17  &  9 & & $V_{ N } \cap V_{ T } = \emptyset${ }{ }{ }i
    & $V_{ N } \cap V_{ T } = \emptyset$~i \\
    17  &  9 & & \textit{każdego}{ }{ }$( u, v ) \in P${ }{ }
               $\#_{ V_{ N } } u \geq 1$
    & \textit{każdego} $( u, v ) \in P$ \textit{zachodzi}
      $\#_{ V_{ N } } u \geq 1$ \\
    17  & &  4 & N.Chomsky’ego & N.~Chomsky’ego \\
    17  & &  7 & $1, 2, ...$ & $1, 2, \ldots$ \\
    17  & & 11 & $1, 2, ...$ & $1, 2, \ldots$ \\
    % & &  &  &  \\
    % & &  &  &  \\
    % & &  & &  \\
    % & &  &  &  \\
    % & & &  &  \\
    % & &  &  &  \\
    % & &  & &  \\
    % & &  & &  \\
    % & &  & & \\
    % & &   &  &  \\
    % & &   &  &  \\
    % &  & &  &  \\
    % &  & &  &  \\
    % &  & &  &  \\
    % & &  &  &  \\
    % & & &  &  \\
    % & & &  &  \\
    % &  & &  &  \\
    % & &  &  &  \\
    % &   & &  &  \\
    % & & & & \\
    % & & & & \\
    % & & & & \\
    % & & & & \\
    % & & & & \\
    % & & & & \\
    % & & & & \\
    % & & & & \\
    % & & & & \\
    % & & & & \\
    % & & & & \\
    \hline
  \end{tabular}





  \newpage

  \begin{tabular}{|c|c|c|c|c|}
    \hline
    Strona & \multicolumn{2}{c|}{Wiersz} & Jest
                              & Powinno być \\ \cline{2-3}
    & Od góry & Od dołu & & \\
    \hline
    18  &  3 & & \textbf{(0) :} & \textbf{(0)} -- \\
    18  &  4 & & \textbf{(1) :} & \textbf{(1)} -- \\
    18  &  8 & & $1$ & $1_{ M }$ \\
    18  &  7 & & $( V_{ N } \cup V_{ T } )^{ * }$,{ } $v \in V_{ N }$,{ }
                 $x \in ( V_{ N } \cup V_{ T } )^{ + }$
                 & $( V_{ N } \cup V_{ T } )^{ * }$, $v \in V_{ N }$,
                   $x \in ( V_{ N } \cup V_{ T } )^{ + }$ \\
    18  &  9 & & $v_{ 0 } \; \to \; 1 \; \in \; P$
    & $v_{ 0 } \to 1_{ M } \in P$ \\
    18  & 11 & & \textbf{(2) :} & \textbf{(2)} -- \\
    18  & 14 & & $V_{ N }$,{ } $x \in ( V_{ N } \cup V_{ T } )$
           & $V_{ N }$, $x \in ( V_{ N } \cup V_{ T } )$ \\
    18  & 15 & & \textbf{(3) :} & \textbf{(3)} -- \\
    18  & 18 & & $V_{ N }$,{ } $x \in V_{ T }^{ * }$
    & $V_{ N }$, $x \in V_{ T }^{ * }$ \\
    18  & 21 & & $\mathbf{\Lcal_{ 0 }}$\textbf{:} & $\mathbf{\Lcal_{ 0 }}$ \\
    18  & 22 & & $\mathbf{\Lcal_{ 1 }}$ \textbf{:} & $\mathbf{\Lcal_{ 1 }}$ \\
    18  & 23 & & $\mathbf{\Lcal_{ 2 }}$ \textbf{:} & $\mathbf{\Lcal_{ 2 }}$ \\
    18  & 24 & & $\mathbf{\Lcal_{ 3 }}$ \textbf{:} & $\mathbf{\Lcal_{ 3 }}$ \\
    19  &  5 & & S.C.Kleene & S.C. Kleene \\
    19  & &  5 & $s \in S$,{ } $a \in A$ & $s \in S$, $a \in A$ \\
    19  & &  5 & $w \in A^{ * }${ } $f( s, wa ) = f( f( s, w ), a )$
    &$w \in A^{ * }$ przyjmujemy $f( s, wa ) = f\big( f( s, w ), a \big)$ \\
    19  & &  6 & $s \in S${ } $f( s, 1 ) = s$
    & $s \in S$ przyjmujemy $f( s, 1_{ A^{ * } } ) = s$ \\
    20  & &  8 & $A^{ * } /_{ \rho }$ & $A^{ * }_{ / \rho }$ \\
    20  & &  8 & kongruencja{ }{ } $\rho \subset ( A^{ * } )^{ 2 }${ }{ } wyznacza
    & kongruencja $\rho \subset ( A^{ * } )^{ 2 }$ wyznacza \\
    \hline
  \end{tabular}





  % \newpage

  % \begin{tabular}{|c|c|c|c|c|}
  %   \hline
  %   Strona & \multicolumn{2}{c|}{Wiersz} & Jest
  %                             & Powinno być \\ \cline{2-3}
  %   & Od góry & Od dołu & & \\
  %   \hline
  %   % & &  &  &  \\
  %   % & &  &  &  \\
  %   % & &  & &  \\
  %   % & &  &  &  \\
  %   % & & &  &  \\
  %   % & &  &  &  \\
  %   % & &  & &  \\
  %   % & &  & &  \\
  %   % & &  & & \\
  %   % & &   &  &  \\
  %   % & &   &  &  \\
  %   % &  & &  &  \\
  %   % &  & &  &  \\
  %   % &  & &  &  \\
  %   % & &  &  &  \\
  %   % & & &  &  \\
  %   % & & &  &  \\
  %   % &  & &  &  \\
  %   % & &  &  &  \\
  %   % &   & &  &  \\
  %   % & & & & \\
  %   % & & & & \\
  %   % & & & & \\
  %   % & & & & \\
  %   % & & & & \\
  %   % & & & & \\
  %   % & & & & \\
  %   % & & & & \\
  %   % & & & & \\
  %   % & & & & \\
  %   % & & & & \\
  %   \hline
  % \end{tabular}





  \newpage

  \begin{tabular}{|c|c|c|c|c|}
    \hline
    Strona & \multicolumn{2}{c|}{Wiersz} & Jest
                              & Powinno być \\ \cline{2-3}
    & Od góry & Od dołu & & \\
    \hline
    52  & & 13 & $[ w_{ 1 } ].....[ w_{ m } ]$ & $[ w_{ 1 } ] \ldots [ w_{ m } ]$ \\
    52  & & 14 & $w_{ 1 }, ....., w_{ m }$ & $w_{ 1 }, \ldots, w_{ m }$ \\
    52  & & 14 & $[ w_{ 1 } ].....[ w_{ m } ]$
           & $[ w_{ 1 } ] \ldots [ w_{ m } ]$ \\
    52  & & 14 & $[ w_{ 1 } ..... w_{ m } ]$ & $[ w_{ 1 } \ldots w_{ m } ]$ \\
    52  & & 14 & $w_{ 1 } ..... w_{ m }$ & $w_{ 1 } \ldots w_{ m }$ \\
    % & &  &  &  \\
    % & &  &  &  \\
    % & &  & &  \\
    % & &  &  &  \\
    % & & &  &  \\
    % & &  &  &  \\
    % & &  & &  \\
    % & &  & &  \\
    % & &  & & \\
    % & &   &  &  \\
    % & &   &  &  \\
    % &  & &  &  \\
    % &  & &  &  \\
    % &  & &  &  \\
    % & &  &  &  \\
    % & & &  &  \\
    % & & &  &  \\
    % &  & &  &  \\
    % & &  &  &  \\
    % &   & &  &  \\
    % & & & & \\
    % & & & & \\
    % & & & & \\
    % & & & & \\
    % & & & & \\
    % & & & & \\
    % & & & & \\
    % & & & & \\
    % & & & & \\
    % & & & & \\
    % & & & & \\
    \hline
  \end{tabular}

\end{center}

\vspace{\spaceTwo}


\noindent
\textbf{Strona otwierająca, wiersz 1 (od dołu).} \\
\Jest  \textit{A.A.Milne} \\
\Powin A.A. Milne \\
\textbf{Str. 8, pierwszy rysunek.} \\
\Jest  $S /_{ Ker_{ h } }$ \\
\Powin $S_{ / Ker_{ h } }$ \\
\textbf{Str. 8, drugi rysunek.} \\
\Jest  $A^{ * } /_{ Ker_{ h } }$ \\
\Powin $M_{ / Ker_{ h } }$ \\
\StrWg{10}{3} \\
\Jest  \textit{to znaczy} $M = B^{ * }$ \\
\Powin czyli $M$ jest kanonicznie izomorficzny z~pewnym wolnym monoidem
słownym~$B^{ * }$, \\
\StrWd{10}{23} \\
\Jest  \textit{Homomorfizm $h$ jest, jak łatwo stwierdzić, izomorfizmem,
  więc monoid $M$ jako izomorficznym z~$A^{ * }$ jest wolny.} \\
\Powin Ponownie, z~faktu że każdy element $m \in S$ można rozłożyć na
  elementy ze zbioru $A$, wynika, że homomorfizm $h$ jest suriekcją.
  Z~jednoznaczności tego rozkładu wynika, że~$h$ jest infekcją, a~więc
  izomorfizmem. Tym samym monoid $M$ jest izomorficzny z~wolnym monoidem
  słownym $A^{ * }$, zgodnie więc z~definicją jest monoidem wolnym. \\
\StrWg{11}{18} \\
\Jest  \textit{Ponieważ, jak uzasadniliśmy, grupa jedynki jest
  trywialna,} \\
\Powin Ponieważ $S$ jest półgrupą, wynika stąd,~że \\








% ############################










% #####################################################################
% #####################################################################
% Bibliografia

\bibliographystyle{plalpha}

\bibliography{MathComScienceBooks}{}





% ############################

% Koniec dokumentu
\end{document}
