% ---------------------------------------------------------------------
% Podstawowe ustawienia i pakiety
% ---------------------------------------------------------------------
\RequirePackage[l2tabu, orthodox]{nag} % Wykrywa przestarzałe i niewłaściwe
% sposoby używania LaTeXa. Więcej jest w l2tabu English version.
\documentclass[a4paper,11pt]{article}
% {rozmiar papieru, rozmiar fontu}[klasa dokumentu]
\usepackage[MeX]{polski} % Polonizacja LaTeXa, bez niej będzie pracował
% w języku angielskim.
\usepackage[utf8]{inputenc} % Włączenie kodowania UTF-8, co daje dostęp
% do polskich znaków.
\usepackage{lmodern} % Wprowadza fonty Latin Modern.
\usepackage[T1]{fontenc} % Potrzebne do używania fontów Latin Modern.



% ------------------------------
% Podstawowe pakiety (niezwiązane z ustawieniami języka)
% ------------------------------
\usepackage{microtype} % Twierdzi, że poprawi rozmiar odstępów w tekście.
\usepackage{graphicx} % Wprowadza bardzo potrzebne komendy do wstawiania
% grafiki.
\usepackage{verbatim} % Poprawia otoczenie VERBATIME.
\usepackage{textcomp} % Dodaje takie symbole jak stopnie Celsiusa,
% wprowadzane bezpośrednio w tekście.
\usepackage{vmargin} % Pozwala na prostą kontrolę rozmiaru marginesów,
% za pomocą komend poniżej. Rozmiar odstępów jest mierzony w calach.
% ------------------------------
% MARGINS
% ------------------------------
\setmarginsrb
{ 0.7in}  % left margin
{ 0.6in}  % top margin
{ 0.7in}  % right margin
{ 0.8in}  % bottom margin
{  20pt}  % head height
{0.25in}  % head sep
{   9pt}  % foot height
{ 0.3in}  % foot sep



% ------------------------------
% Często przydatne pakiety
% ------------------------------
\usepackage{csquotes} % Pozwala w prosty sposób wstawiać cytaty do tekstu.
\usepackage{xcolor} % Pozwala używać kolorowych czcionek (zapewne dużo
% więcej, ale ja nie potrafię nic o tym powiedzieć).



% ------------------------------
% Pakiety do tekstów z nauk przyrodniczych
% ------------------------------
\let\lll\undefined % Amsmath gryzie się z językiem pakietami do języka
% polskiego, bo oba definiują komendę \lll. Aby rozwiązać ten problem
% oddefiniowuję tę komendę, ale może tym samym pozbywam się dużego Ł.
\usepackage[intlimits]{amsmath} % Podstawowe wsparcie od American
% Mathematical Society (w skrócie AMS)
\usepackage{amsfonts, amssymb, amscd, amsthm} % Dalsze wsparcie od AMS
% \usepackage{siunitx} % Dla prostszego pisania jednostek fizycznych
\usepackage{upgreek} % Ładniejsze greckie litery
% Przykładowa składnia: pi = \uppi
\usepackage{slashed} % Pozwala w prosty sposób pisać slash Feynmana.
\usepackage{calrsfs} % Zmienia czcionkę kaligraficzną w \mathcal
% na ładniejszą. Może w innych miejscach robi to samo, ale o tym nic
% nie wiem.



% ##########
% Tworzenie otoczeń "Twierdzenie", "Definicja", "Lemat", etc.
\newtheorem{theorem}{Twierdzenie}  % Komenda wprowadzająca otoczenie
% „theorem” do pisania twierdzeń matematycznych
\newtheorem{definition}{Definicja}  % Analogicznie jak powyżej
\newtheorem{corollary}{Wniosek}



% ---------------------------------------
% Pakiety napisane przez użytkownika.
% Mają być w tym samym katalogu to ten plik .tex
% ---------------------------------------
\usepackage{latexgeneralcommands}
\usepackage{mathcommands}
% \usepackage{calculuscommands}
\newcommand{\conca}{\textrm{conca}}
% \usepackage{SchwartzBooksCommands}  % Pakiet napisany m.in. dla tego pliku.



% ---------------------------------------------------------------------
% Dodatkowe ustawienia dla języka polskiego
% ---------------------------------------------------------------------
\renewcommand{\thesection}{\arabic{section}.}
% Kropki po numerach rozdziału (polski zwyczaj topograficzny)
\renewcommand{\thesubsection}{\thesection\arabic{subsection}}
% Brak kropki po numerach podrozdziału



% ------------------------------
% Ustawienia różnych parametrów tekstu
% ------------------------------
\renewcommand{\arraystretch}{1.2} % Ustawienie szerokości odstępów między
% wierszami w tabelach.



% ------------------------------
% Pakiet "hyperref"
% Polecano by umieszczać go na końcu preambuły.
% ------------------------------
\usepackage{hyperref} % Pozwala tworzyć hiperlinki i zamienia odwołania
% do bibliografii na hiperlinki.










% ---------------------------------------------------------------------
% Tytuł i autor tekstu
\title{Teoria automatów i~języków formalnych \\
  Błędy i~uwagi}

\author{Kamil Ziemian}


% \date{}
% ---------------------------------------------------------------------










% ####################################################################
\begin{document}
% ####################################################################





% ######################################
\maketitle % Tytuł całego tekstu
% ######################################





% ############################
\Work{ % Autorzy i tytuł dzieła
  Maria Foryś, Wit Foryś \\
  \textit{Teoria automatów i~języków formalnych},
  \cite{ForysForysTeoriaAutomatowIJezykowFormalnych2005}}


% ##################
\CenterBoldFont{Uwagi}


\start Zawartość tej książeczki jest pod pewnym względami większa, zaś
pod pewnymi względami mniejsza, niż kursu online \textit{Języki,
  automaty i~obliczania}, autorstwa, notabene, Marii Foryś, Wita
Forysia i~Adama Romana, warto więc przerabiać go równolegle z~tą
książeczką. Można go znaleźć pod tym
\colorhref{https://wazniak.mimuw.edu.pl/index.php?title=J\%C4\%99zyki,\_automaty\_i\_obliczenia}{linkiem}\footnote{Pełna forma linku:
  \href{https://wazniak.mimuw.edu.pl/index.php?title=J\%C4\%99zyki,\_automaty\_i\_obliczenia}
  {https://wazniak.mimuw.edu.pl/index.php?title=J\%C4\%99zyki,\_automaty\_i\_obliczenia}.}.

\vspace{\spaceFour}





\start W~tej książce używany jest cudzysłów w~formie ”cytowany tekst”,
ale polskie standardy typograficzna mówią, że~powinno~się stosować
formę: „cytowany tekst”. Dodatkowo, wyjątek warto uczynić dla ciągów
symboli (liter z~danego alfabetu $A$), nazywanych napisami bądź
stringami (ang. \textit{strings}), które zgodnie z~przyjętą
w~informatyce konwencją będziemy oznaczać jako $\texttt{"} abc \texttt{"}$.

Więcej o~napisach powiemy przy okazji omawiania wolnych monoidów językowych.





% ##################
\CenterBoldFont{Uwagi do konkretnych stron}

% \vspace{\spaceFour}


\start \Str{6} Według mnie symbol $\textrm{mod}_{ 6 }$ wygląda
znacznie lepiej niż używany w tej książce $mod_{ 6 }$.

\vspace{\spaceFour}





\start \Str{6} Na tej stronie znajdujemy informację, że~element
$x \in M$ jest odwracalny w~monoidzie $M$ wtedy i tylko wtedy, gdy
istnieje takie $y \in M$, że $x \cdot y = 1_{ M }$. Jednak monoid nie
musi być przemienny, więc nie widzę powodu by zachodziła też równość
$y \cdot x = 1_{ M }$. Co więcej, wolne monoidy językowe, które
zdefiniujemy dalej, a~które są dla nas szczególnie interesujące, nie
są przemienne, więc problem jest tym poważniejszy.

Wydaje mi, że jest to zwykłe przeoczenie ze strony autorów i~trzeba po
prostu przyjąć, że~element $x \in M$ nazywamy odwracalny, gdy istnieje
taki element $y \in M$, że~zachodzi
\begin{equation}
  \label{eq:Forys-Forys-01}
  x \cdot y = y \cdot x = 1_{ M }.
\end{equation}

\vspace{\spaceFour}





\start \Str{6} Na tej stronie znajdujemy stwierdzenia, że „zbiór
elementów odwracalnych monoidu $M$ stanowi podgrupę tego monoidu”.
Zwykle przez podgrupę rozumie~się podzbiór $G_{ 1 }$ grupy $G_{ 0 }$,
taki że~wraz z~działaniem\footnote{Używając terminologi z~książki
  Białynickiego-Biruli, mówilibyśmy o~trzech różnych działaniach:
  dwuargumentowym działaniu wewnętrznym, jednoargumentowym działaniu brania
  elementu odwrotnego i~zeroargumentowym działaniu zwracającym element
  neutralny. Zob. \cite{BialynickiBirulaZarysAlgebry1987}.} odziedziczonym
z~$G_{ 0 }$ jest grupą. Tutaj
jednak jest mowa, że~grupa jedynki, oznaczmy ją $U$ (od ang.
\textit{unity}), jest podgrupą monoidu, który to monoid nie musi być
podgrupą.

Sens tego zwrotu jest dość oczywisty. Mianowicie, że trójka uporządkowana
$( U, \cdot, 1 )$, gdzie $\cdot$ jest działaniem odziedziczonym po
monoidzie $M$, zaś $1 \in U$, jest elementem neutralnym tego monoidu,
jest grupą. Jednak takie rozciąganie standardowego rozumienia pojęcia podgrupy czyni cały system pojęciowy trochę mniej ścisłym.

Możemy doprecyzować ten formalizm za pomocą pojęcia algebry
ogólnej (zob. \cite{BialynickiBirulaZarysAlgebry1987}) i~wydaje~się, że~w~ostatecznym rozrachunku jest to najbardziej ekonomiczne podejście.

Niech $( A, \varphi_{ 1 }, \varphi_{ 2 }, \ldots, \varphi_{ n } )$ będzie algebrą ogólną typu $\alpha$, np.
monoidem, i~niech $B \subset A$. Jeżeli system
$( B, \varphi_{ 1 }|_{ B }, \varphi_{ 2 }|_{ B }, \ldots, \varphi_{ n }|_{ B } )$ jest algebrą ogólną
typu $\beta$, to mówimy, że~$B$ jest podalgebrą ogólną $A$ typu $\beta$.
Przykładowo, jeśli $( M, \cdot, 1_{ M } )$ jest monoidem, a~$( G, \cdot, 1_{ M } )$,
gdzie $G \subset M$, jest grupą, to mówimy, że~$G$ jest podgrupą monoidu $M$. Tak
jak zazwyczaj, będziemy używali też bardziej zwięzłych wyrażeń językowych
jak „$G$ jest podgrupą $M$”, nie stwierdzając jawnie, że~$M$ jest monoidem.

Idea jaka stoi za tymi pojęciami jest jasna. Po zawężeniu działań
wewnętrznych danej algebry ogólnej $( A, \varphi_{ 1 }, \varphi_{ 2 }, \ldots, \varphi_{ n } )$ do
mniejszego zbioru $B \subset A$, mogą się zmienić własności tych działań i~tym
samym typ algebry. Podzbiór $\Fbb$ pierścienia $\Pcal$ może okazać~się
ciałem, bo wszystkie elementy nieodwracalne, z~wyjątkiem zera, należą do
zbioru $\Pcal \setminus \Fbb$.

Wszystkie te rozważania moglibyśmy ująć w~sposób bardziej sformalizowany,
jednak wydaje~się, że~obecna ich forma jest wystarczająco precyzyjna.

\vspace{\spaceFour}





\start \Str{6} W~dowodzie twierdzenia 1.1.1 brakuje mi zdania typu
„Jeśli $S$ nie jest monoidem, to rozszerzamy go do monoidu, za pomocą
procedury którą zaprezentujemy poniżej.”.

\vspace{\spaceFour}





\start \Str{6} Dowód twierdzenia 1.1.1 zawiera dodatkową informację,
nad którą warto~się zatrzymać. Mianowicie że~każdą półgrupę można
rozszerzyć do monoidu. Wymaga to jedynie
oczywistego/nieoczywistego\footnote{Zależy to od przekonań
  filozoficzno-matematycznych konkretnej osoby.} założenia, że~istnieje
element $1 \notin S$.

W~kontekście samego dowodu twierdzenia 1.1.1, warto~się zastanowić nad
tym dlaczego dokonujemy rozszerzenia półgrupy $( S, \cdot )$ do
monoidu $( S^{ 1 }, \cdot, 1 )$. Dzięki temu mamy rodzinę odwzorowań
$\rho_{ a } : S^{ 1 } \to S^{ 1 }$ i~ponieważ $S^{ 1 }$ jest monoidem,
to mamy $\rho_{ a }( 1 ) = a$, ta zaś równość pozwala pokazać,
że~z~$\rho_{ a_{ 1 } } = \rho_{ a_{ 2 } }$ wynika $a_{ 1 } = a_{ 2 }$.
Z tego zaś od razu wynika, że odwzorowanie
$h : S \to ( \{ \rho_{ a } \}_{ a \in S } )$, $h( a ) = \rho_{ a }$
jest iniektywne. Poza tym rozszerzenie półgrupy do monoidu nie wydaje
się nigdzie indziej potrzebne.

Pytanie, czy istnieje sposób pokazania, że~odwzorowanie $h$ jest
iniektywne, bez rozszerzania półgrupy $S$ do monoidu $S^{ 1 }$? Nawet
jeżeli tak, to prostota przeprowadzonego w~książce dowodu sprawia, że
warto przy nim pozostać.

\vspace{\spaceFour}





\start \Str{7} Tak jak w~przypadku symbolu $\textrm{mod}_{ 6 }$,
wydaje mi~się, że~lepiej wyglądałby symbol $\textrm{Ker}_{ h }$.

\vspace{\spaceFour}





\start \Str{8} Rysunki na tej stronie są zrobione dość niechlujnie.
Wystarczy zwrócić uwagę, że~symbol $h$ leży w~nich w~różnych odległości od
lewego marginesu, podczas, gdy powinien być w~tej samej pozycji.

\vspace{\spaceFour}





\start \Str{8} Wedle podanej tu definicji wolny monoid $M$ nad
alfabetem $A$, który będziemy oznaczać również symbolem $A^{ * }$,
jest obiektem o~pewnej bardzo konkretnej strukturze ontologicznej.
Sprawa ta prowadzi do pewnych niejasności w~dalszym wykładzie
materiału.

Mianowicie bierzemy pewien zbiór $A$, którego elementy noszą nazwę
„liter” lub „symboli” (patrz uwaga do str. 9) następnie tworzymy zbiór
wszystkich ciągów tych symboli, który będziemy oznaczać symbolem
$A^{ * }$.
\begin{equation}
  \label{eq:Forys-Forys-02}
  A^{ * } =
  \{ ( a_{ 1 }, a_{ 2 }, a_{ 3 }, \ldots, a_{ n } ) \;\; : \;\; n \geq 0, \;
  a_{ i } \in A \}.
\end{equation}
Wedle tej definicji do $A^{ * }$ należy ciąg pusty, oznaczany symbolem
$1 \equiv \texttt{""}$. Taki ciąg intuicyjnie jest prosty do pojęcia,
zaś~bardziej formalnie jest to odwzorowanie $f : \emptyset \to A$.

W~zbiorze $A^{ * }$ wprowadzamy też działanie konkatenacjami oznaczane $\cdot$
wzorem
\begin{equation}
  \label{eq:Froys-Forys-03}
  ( a_{ 1 }, a_{ 2 }, \ldots, a_{ n } ) \cdot ( b_{ 1 }, b_{ 2 }, \ldots, b_{ m } ) =
  ( a_{ 1 }, a_{ 2 }, \ldots, a_{ n }, b_{ 1 }, b_{ 2 }, \ldots, b_{ m } ).
\end{equation}

W~dalszym ciągu książki dowiadujemy~się, że klasycznym monoidem wolnym
jest $( \Nbb_{ 0 }, +, 0 )$. Powstaje teraz pytanie, czy aby móc
powiedzieć, że~zbiór liczb naturalnych z~$0$ jest monoidem wolnym, nie
musimy go przedstawić jako zbioru ciągów jakiś symboli, tak jak jest
to przedstawione w~powyższej definicji? To jednak rodzi pewne
problemy.

Zauważmy, że~możemy wprawdzie przyjąć\footnote{Nie będziemy tu
  poruszać niezwykle interesującego i~ważnego zagadnienie, tego
  mianowicie, że~w~pewnych teoriach typów „obiekty izomorficzne są
  identyczne” i~tym samym nie ma dwóch różnych „modeli liczb
  naturalnych” z~zerem czy bez. Zawędrowalibyśmy tym sposobem zbyt
  daleko od właściwego tematu.}, że~zbiór $A = \{ | \}$, liczba $0$ to
pusty ciąg, $1 = ( | )$, $2 = ( |, | )$, $3 = ( |, |, | )$, etc.,
jednak podstawowa intuicja matematyczna mówi, że~wprowadzanie takiej
konstrukcji nie powinno być konieczne. Zanim przejdziemy do dyskusji
tego zagadnienia, zauważmy, że~w tak podanym wolnym monoidzie
$A^{ * }$ konkatenacja ciągów rzeczywiście prowadzi do tego co
intuicyjnie rozpoznajemy jako dodawanie liczb naturalnych. Przykładowo
\begin{equation}
  \label{eq:Forys-Forys-04}
  1 + 2 = ( | ) \cdot ( |, | ) = ( |, |, | ) = 3.
\end{equation}

Wróćmy teraz do problemu, czemu takie podejście do liczby naturalnych
jawi~się intuicji matematycznej jako niekonieczne. Powinno być bowiem
możliwe myślenie o~liczbach naturalnych jako „po prostu o~liczbach”,
jako autonomicznych bytach, nie zaś jako ciągach jakiegoś symbolu.
Dodatkowo, taka definicji to błędne koło, samo bowiem pojęcie ciągu
o~skończonej długości wymaga uznania, że~wiemy czym jest zbiór liczb
$\{ 1, 2, \ldots, n \}$. Skończony ciąg symboli to bowiem nic innego jak
odwzorowanie $f : \{ 1, 2, \ldots, n \} \to A$. Z~tego powodu potrzebujemy
definicji liczb naturalnych, która nie bazuje na pojęciu ciągu, jedną
z~nich jest ta, która bazuje na użyciu zbioru pustego.

Rozważmy teraz inny przykład. Zbiór wszystkich macierzy kwadratowych
wymiaru $n$, który będziemy oznaczać symbolem $\GL( m )$, możemy
rozpatrywać jako monoid $( \GL( m ), \cdot, \matUnit )$, gdzie $\cdot$
to mnożenie macierzy, zaś $\matUnit$ to macierz jednostkowa
$m \times m$. Czy jest więc w~ogóle sens w~pytaniu~się, czy
ten monoid zawiera jakiś podmonoid wolny, rozumiany jako zbiór ciągów
z~działaniem konkatenacji? Wszak mając dwie macierze $A$ i~$B$
należące do $\GL( m )$ musielibyśmy z~jednej strony rozważać ich
iloczyn $A \cdot B$, z~drugiej ciąg dwuelementowy $( A, B )$ i~ustalić
relację między tymi dwoma bytami. Nie jest powiedziane, że~nie można
utożsamić konkretnej macierzy kwadratowej z~ciągiem jakiejś zbioru
macierzy, acz taka konstrukcja wydaje~się znacznie mniej naturalna,
niż przedstawiona powyżej dla liczb naturalnych.

W~świetle tego proponuję następujące wyjście z~sytuacji.
\textbf{Wolnym monoidem językowym $A^{ * }$ nad alfabetem $A$}
będziemy nazywać wcześniej określony zbiór ciągów wraz z~działaniem
konkatenacji. Jest więc to obiekt o~konkretnej strukturze
ontologicznej.

Poza tym zachowujemy prawienie niezmienioną terminologię ze~stron 8--9,
która jest tam wprowadzona dla wolnym monoidów i~stosujemy ją też do
wolnych monoidów językowych. Będziemy więc mówić o~„bazie (alfabecie) $A$”,
elementy zbioru $A$ będziemy nazywać „literami”, bądź jak zostało to
wyjaśnione w~jednym z~dalszych punktów, „symbolami”. Ciągi liter (symboli)
będziemy nazywali „słowami”, a~dowolny podzbiór monoidu $A^{ * }$ będziemy
nazywali „językiem”. W~razie potrzeby będziemy do tego zbioru dodawać
odpowiednie synonimy wszystkich tych pojęć.

Wybiegając trochę w~przyszłość, zdefiniujemy teraz monoid wolny, korzystając
z~pojęcia kanonicznego homomorfizmu. Monoid $( M, \cdot, 1_{ M } )$ jest
\textbf{monoidem wolnym}, jeśli jest istnieje taki zbiór $A \subset M$, że~wolny
monoid językowym $A^{ * }$ nad alfabetem $A$ jest kanonicznie izomorficzny
z~$M$. Nie widać więc powodu by monoid $M$ miał bardziej ustaloną strukturę
ontologiczną, niż np. grupa abelowa.

Na monoidy wolne przenosimy nazewnictwo wprowadzone wcześniej dla wolnych
monoidów językowych. Zbiór $A$ będziemy więc nazywać „bazą (alfabetem)
wolnego monoidu $M$”, elementy $A$ będziemy nazywać „literami (symbolami)”,
etc. Dla większej jasności rozważań element wolnego monidłu językowego
będziemy często zapisywać jako ciągi symboli wewnątrz cudzysłowu
\texttt{" "}. Przykładowo $\texttt{"} abc \texttt{"}$. Mamy nadzieję,
że~pomoże to uniknąć pewnych nieporozumień.

Analogiczna terminologia stosuje~się też do półgrup. Mamy więc \textbf{wolne
  półgrupy językowe nad alfabetem $A$} i~\textbf{wolne półgrupy}, ponieważ
jednak ponieważ ich definicje oraz własności są bardzo podobne do
omówionych powyżej własności monoidów, nie będziemy ich dokładnie opisywać.

????? Zbiór $A$ może być podzbiorem
$M$ lub nie. Jak pokazują twierdzenia udowodnione na stronach 9--12,
jeśli istnieje dowolny zbiór $A$, taki że $M$ jest izomorficzny
z~wolnym monoidem językowym $A^{ * }$, to można znaleźć taki zbiór
$B \subset M$, że~$M$ jest izomorficzny z~wolnym monoidem językowym
$B^{ * }$. Jednak poprawne omówienie tej kwestii wymaga uporządkowania
kilku pojęć.

???? Również z~tych twierdzeń wynika wniosek, że~jeśli dany monoid
$( M, \cdot, 1_{ M } )$ jest monoidem wolnym, to istnieje taki zbiór
$B$, że~każdy element $m \in M$, $m \neq 1_{ M }$ można przedstawić
w~jeden i~tylko jeden sposób jako
\begin{equation}
  \label{eq:Forys-Forys-04}
  b_{ 1 } \cdot b_{ 2 } \cdot \ldots \cdot b_{ n }, \quad
  b_{ i } \in B.
\end{equation}
Zauważmy, że~w~powyższym wyrażeniu może zachodzić $b_{ i } = b_{ j }$
dla $i \neq j$. W~istocie ta własność jest równoważna przyjętej
definicji i~tym samym można by ją przyjąć w~jej miejsce. Nie
zrobiliśmy tego, bo zapisanie tej definicji w~sposób precyzyjny „od
zera” wymagało tyle miejsca, że~nie wygląda na to, byśmy w~ten sposób
coś zyskiwali.

\vspace{\spaceFour}





\start \Str{8} Korzystając z~wprowadzonych wcześniej pojęć, możemy
teraz zająć~się niejasnościami, jakie wynikają, z~użycia w~tej książce
symbolu $A^{ * }$ w~dwóch różnych znaczeniach. Jeżeli bowiem
$A \subset M$, gdzie $M$ jest pewnym monoidem, to symbol $A^{ * }$
może oznaczać zarówno wolny monoid językowy nad alfabetem $A$, jak
i~podmonoid monoidu $M$ generowany przez $A$, a~są to dwa różne
byty.

Jeśli przyjmiemy, że~$M$ to omawiany już wcześniej monoid macierzy
kwadratowych $( \GL( m ), \cdot, \matUnit )$, zaś
$A = \{ B_{ 1 }, B_{ 2 }, B_{ 3 }, \ldots, B_{ n } \}$, to symbol
$A^{ * }$ możne oznaczać zbiór ciągów macierzy kwadratowych,
np.~$( B_{ 1 }, B_{ 2 }, B_{ 3 } )$, jak i~zbiór macierzy kwadratowych
utworzonych jako iloczyny macierzowy macierzy ze zbioru $A$. W~tym
drugim przypadku do tego zbioru należy choćby macierz
$B_{ 1 } \cdot B_{ 2 } \cdot B_{ 3 }$.

By usunąć tego typu dwuznaczności przyjmiemy następującą konwencję.
Jeżeli jest jawnie powiedziane, że~$A$ jest podzbiorem monoidu $M$, to
przez symbol $A^{ * }$ będziemy rozumieli najmniejszy podmonoid $M$
zawierający $A$. Taki obiekt będziemy nazywali \textbf{podmonoidem
  algebraicznym}.

Jeżeli zbiór $A$ jest podany, ale bez wskazania, że
jest to podzbiór pewnego monoidu $M$, to przez symbol $A^{ * }$
będziemy rozumieć wolny monoid językowy nad alfabetem $A$. W~przypadku
gdy $A$ jest podzbiorem monoidu $M$, ale~rozważać będziemy wolny
monoid językowy nad $A$, to zostanie to zaznaczone jawnie. Szczególny
przypadkiem, acz nietrudnym do zrozumienia jest ten, gdy $A \subset M$
i~$M$ jest monoidem językowym. Wówczas obie definicje symbolu
$A^{ * }$~się pokrywają.

W~świetle powyższych ustaleń wprowadzimy odpowiednie poprawki do tej
książeczki.

\vspace{\spaceFour}





\start \Str{8} Dysponując już tak uporządkowanym słownictwem,
przeanalizujmy strukturę wolnego monoidu językowego $A^{ * }$ na
alfabetem $A$. Jeśli potrzebny będzie nam przykład wolnego monoidu
językowego, sięgniemy po niezwykle ważny w~informatyce monoid
$\{ 0, 1 \}^{ * }$. Łatwo spostrzec, że~jest to monoid nieprzemienny,
przykładowo jeśli $u = 00$, $v = 11$, to
\begin{equation}
  \label{eq:Forys-Forys-05}
  u v = 0011 \neq 1100 = vu.
\end{equation}
Nasuwa~się naturalne pytanie: co można powiedzieć o~zbiorze $C$ taki,
że dla wszystkich $u, v \in C$ zachodzi
\begin{equation}
  \label{eq:Forys-Forys-06}
  uv = vu.
\end{equation}
Problem ten wymaga dalszej analizy.

Zauważmy, że~dla monoidu językowego zachodzi równość
\begin{equation}
  \label{eq:Forys-Forys-07}
  | u v | = | u | + | v |.
\end{equation}
Wynika ona w~prosty sposób z~definicji konkatenacji. Z~niej i~z~tego,
że~$| u | \geq 0, u \in M$, wynika, że grupa jedynki w~wolnym
monoidzie językowym jest trywialna. Choć wniosek ten jest prosty do
zrozumienia, bardziej sformalizowany dowód wymagałby użycia
nierówności
\begin{equation}
  \label{eq:Forys-Forys-08}
  | u v | \geq | u |,
\end{equation}
które wynika wprost z~dwóch wspomnianych własności.

\vspace{\spaceFour}





\Str{8} Zwróćmy uwagę, że~jeśli $A \subset M$, gdzie $M$ jest monoidem, to
podmonoid algebraiczny $A^{ * }$ i~wolnym monoid językowy $A^{ * }$
i~alfabetem $A$ mogą być radykalnie różnymi obiektami.

Dla przykładu weźmy monoid $( \Nbb_{ 0 }, +, 0)$ i~niech $A = \{ 0 \}$. Wówczas
jeżeli przez symbol $A^{ * }$ rozumiemy podmonoid algebraiczny, to $A^{ * } =
\{ 0 \}$, bowiem $0 + 0 = 0$. Jeżeli, przez $A^{ * }$ rozumiemy wolny podmonoid
językowy to wówczas zachodzi
\begin{equation}
  \label{eq:Forys-Forys-09}
  A^{ * } = \{ \texttt{""}, \texttt{"} 0 \texttt{"},
  \texttt{"} 00 \texttt{"}, \texttt{"} 000 \texttt{"},
  \texttt{"} 0000 \texttt{"}, \ldots \}.
\end{equation}
Powyższy monoid językowy jest izomorficzny z~całym monoidem
$( \Nbb_{ 0 }, +, 0)$. Ten izomorfizm zadany jest przez relacje
\begin{equation}
  \label{eq:Forys-Forys-10}
  \begin{split}
    h( \texttt{""} ) &= 0, \\
    h( \texttt{"} 0 \texttt{"} ) &= 1, \\
    h( \texttt{"} 0 0 \texttt{"} ) &= 2, \\
    h( \texttt{"} 0 0 0 \texttt{"} ) &= 3, \\
                     &\vdots
  \end{split}
\end{equation}

\vspace{\spaceFour}





\start \Str{8--9} By uniknąć nieporozumień, należy jasno stwierdzić, że~gdy
będziemy mówić o~„zbiorze ciągów liter z~alfabetu $A$” to w domyśle do tego
zbioru zawsze należy też ciąg pusty \texttt{""}.

\vspace{\spaceFour}





\start \Str{9} Wedle przyjętej na tej stronie nomenklatury elementy
alfabetu będziemy nazywać „literami” i~będziemy mówić, że~„słowa są
ciągami liter”. Jednak ze względu na wygodę, warto przyjąć, że słowo
„symbol” jest synonimem słowa „litera”, dzięki czemu będziemy mogli
powiedzieć „elementami alfabetu są symbole” i~„słowa są ciągami
symboli”.

\vspace{\spaceFour}





\start \Str{9} Dowód twierdzenia 1.2.1 byłby znacznie prostszy do
zrozumienia, gdyby został wskazane, że~jeśli $h : A^{ * } \to M$ jest
rozszerzeniem odwzorowania $f : A \to M$ do homomorfizmu musi spełniać
następujące trzy warunki
\begin{subequations}
  \begin{align}
    \label{eq:Forys-Forys-11-A}
    h( 1 ) &= 1_{ M }, \\
    \label{eq:Forys-Forys-11-B}
    h( a ) &= f( a ), \quad \forall \, a \in A, \\
    \label{eq:Forys-Forys-11-C}
    h( a_{ 1 } \ldots a_{ n } )
           &= f( a_{ 1 } ) \cdot \ldots \cdot f( a_{ n } ),
  \end{align}
\end{subequations}
$\forall \, a_{ i } \in A$, $i = 1, 2, \ldots, n$. Dla większej przejrzystości działanie
wewnętrzne w~monoidzie $M$ zapisaliśmy za pomocą symbolu $\cdot$. Równania te
gwarantują nam, że~może istnieć co najwyżej jedno odwzorowanie
$h : A^{ * } \to M$, dla których są one spełnione.

Możemy teraz użyć równań
\eqref{eq:Forys-Forys-11-A}-\eqref{eq:Forys-Forys-11-C}, do
\textit{zdefiniowania} homomorfizmu $h$. Ponieważ sprawdzenie, że~tak zadane
odwzorowanie jest homeomorfizmem jest dość\footnote{Jak to niektórzy
  powiadają: „Nie ma czego dowodzić”.} opuścimy dyskusję tego tematu.

\vspace{\spaceFour}





\start \Str{9} W~celu uproszczenia rozważań wprowadzimy teraz dodatkową
terminologię. Homomorfizm $h : A^{ * } \to M$, $A \subset M$, prowadzący z~wolnego
monoidu językowego $A^{ * }$ do monoidu $M$, jednoznacznie wyznaczony przez
warunki
\begin{subequations}
  \begin{align}
    \label{eq:Forys-Forys-12-A}
    h( \texttt{""} ) &= 1_{ M }, \\
    \label{eq:Forys-Forys-12-B}
    h( \texttt{"} a \texttt{"} ) &= a, \quad
                                   \forall \, a \in A,
  \end{align}
\end{subequations}
będziemy nazywali \textbf{kanonicznym homomorfizmem wolnego monoidu $A^{ * }$
  językowego w~monoid $M$}, lub po prostu \textbf{kanonicznym
  homomorfizmem}, gdy zbiór $A$ i~monoid $M$ będą wynikać z~kontekstu.
Jeżeli dodatkowo $h$ jest izomorfizmem, to będziemy oprócz oczywistych
sformułowań takich jak „kanoniczny izomorfizm”, będziemy również mówili,
że~$A^{ * }$ i~$M$ są kanonicznie izomorficzne.

Nie potrafię powiedzieć, czy ten homomorfizm jest kanoniczny w~sensie teorii
kategorii, jednak w~tym momencie i~tak nie potrafię w~tym momencie wymyślić
dla niego lepszej nazwy.

\vspace{\spaceFour}





\start \Str{9} Warto zwrócić uwagę na to, że~omawiany tu homomorfizm
kanoniczny wolnego monoidu językowego $A^{ * }$ z~monoidem $M$, gdzie
$A \subset M$ jest zbiorem generatorów $M$, nie musi być izomorfizmem. Zanim
przejdziemy do dokładniejszego omówienia tej kwestii, przypomnijmy,
że~homomorfizm kanoniczny jest jednoznacznie wyznaczony przez warunki
\eqref{eq:Forys-Forys-12-A}--\eqref{eq:Forys-Forys-12-B}.

By to pokazać rozważmy
ponownie monoid $( \Nbb_{ 0 }, +, 0 )$ i~wolny monoid językowy
$A^{ * } = \{ 0, 1 \}^{ * }$. Jest
oczywiste, że
\begin{equation}
  \label{eq:Forys-Forys-13}
  A^{ * } =
  \{ \texttt{""}, \texttt{"} 0 \texttt{"}, \texttt{"} 1 \texttt{"},
  \texttt{"} 0 0 \texttt{"}, \texttt{"} 0 1 \texttt{"},
  \texttt{"} 1 1 \texttt{"}, \ldots \}.
\end{equation}
Kanoniczny homomorfizm $h : A^{ * } \to \Nbb_{ 0 }$ na mocy wzorów
\eqref{eq:Forys-Forys-11-A}-\eqref{eq:Forys-Forys-11-C}
i~\eqref{eq:Forys-Forys-12-A}-\eqref{eq:Forys-Forys-12-B}, posiada
następujące własności
\begin{subequations}
  \begin{align}
    \label{eq:Forys-Forys-14-A}
    h( \texttt{""} ) &= 0, \\
    \label{eq:Forys-Forys-14-B}
    h( \texttt{"} 0 \texttt{"} ) &= 0, \quad
    h( \texttt{"} 1 \texttt{"} ) = 1, \\
    \label{eq:Forys-Forys-14-C}
    h( \texttt{"} a_{ 1 } a_{ 2 } \ldots a_{ n } \texttt{"} )
                     &= h( \texttt{"} a_{ 1 } \texttt{"} )
                       + h( \texttt{"} a_{ 2 } \texttt{"} ) + \ldots
                       + h( \texttt{"} a_{ n } \texttt{"} ) = \\
    % \skiplabel
                     &= a_{ 1 } + a_{ 2 } + \ldots + a_{ n },
  \end{align}
\end{subequations}
$\forall \, a_{ i } \in \{ 0, 1 \}$, $i = 1, 2, \ldots, n$. Przykładowo
\begin{equation}
  \label{eq:Forys-Forys-15}
  h( \texttt{"} 0100110 \texttt{"} ) =
  0 + 1 + 0 + 0 + 1 + 1 + 0 = 3.
\end{equation}
Łatwo zauważyć, że~odwzorowanie $h$ nie jest izomorfizmem. Świadczy o~tym
choćby to, że
\begin{equation}
  \label{eq:Forys-Forys-16}
  h( \texttt{""} ) = h( \texttt{"} 0 \texttt{"} ) = 0.
\end{equation}

\vspace{\spaceFour}





\start \Str{9} Potrzebujemy tu poczynić pewną uwagę odnośnie notacji.
Przyjmijmy, że $A$ jest podzbiorem monoidu $M$. W~skutek wybranych oznaczeń
zapis $M = A^{ * }$ może oznaczać, że~podmonoid algebraiczny $A^{ * }$ jest
równy $M$. Jeśli zaś kanoniczny homomorfizm $h : A^{ * } \to M$ jest
izomorfizmem, to wówczas przez $M = A^{ * }$, będzie rozumieć skrótowy zapisem tego faktu. Obie te sytuacje nie są wcale równoważne.

Rozpatrzmy ponownie monoid $( \Nbb_{ 0 }, +, 0 )$ i~wolny monoid językowy
$A^{ * } = \{ 0, 1 \}^{ * }$. Jest oczywiste, że~jeśli symbol $A^{ * }$ rozumiemy
jako podmonoid algebraiczny to oczywiście zachodzi $M = A^{ * }$.
Jeśli jednak przez $A^{ * }$ rozumiemy wolny monoid językowy, to jak zostało
wykazane w~poprzednich komentarzach, homomorfizm $h$ \textit{nie jest}
izomorfizmem, więc nie możemy napisać równości „$M = A^{ * }$”.

Na koniec zauważmy, że~jeśli $A = \{ 1 \}$ to wówczas wolny monoid językowy
$A^{ * }$ jest kanonicznie izomorficzny z~$M$.

\vspace{\spaceFour}





\start \Str{9} Niech $M$ będzie monoidem wolnym, $B \subset M$ jego bazą,
a~$h : B^{ * } \to M$ kanonicznym izomorfizmem. Dla $a \in M$ wprowadzamy
następującą notację
\begin{equation}
  \label{eq:Forys-Forys-17}
  \texttt{"} a \texttt{"} :=
  \texttt{"} b_{ 1 } b_{ 2 } \ldots b_{ n } \texttt{"}, \quad
  a = h( \texttt{"} b_{ 1 } b_{ 2 } \ldots b_{ n } \texttt{"} ), \,
  b_{ i } \in B,\, i = 1, 2, \ldots, n.
\end{equation}
Mamy nadzieję, że ta notacja pozwoli w~przejrzysty sposób zapisywać pewne
dość abstrakcyjne kroki rachunkowe.

\vspace{\spaceFour}





\start \Str{10} Dowód drugiej części twierdzenia 1.2.2 byłby bardziej
zrozumiały, gdyby wzór przedstawiający działania kanonicznego homomorfizmu
zapisać jako
\begin{equation}
  \label{eq:Forys-Forys-18}
  h( \texttt{"} a_{ 1 } a_{ 2 } \ldots a_{ n } \texttt{"} ) =
  h( \texttt{"} a_{ 1 } \texttt{"} ) h( \texttt{"} a_{ 2 } \texttt{"} ) \ldots
  h( \texttt{"} a_{ n } \texttt{"} ) = a_{ 1 } a_{ 2 } \ldots a_{ n }.
\end{equation}

\vspace{\spaceFour}





\start \Str{10} W~dowodzie twierdzenia 1.2.2 użyte jest mowa o~„długości
elementu $m \in M$ w~$B^{ * }$”. Sens tego jest następujący. Dany jest zbiór
$B$, niekoniecznie zawarty w~$M$, i~izomorfizm $f$, nie musi być to
izomorfizm kanoniczny, między wolnym monoidem językowym $B^{ * }$ a~$M$. Dla
każdego elementu $m \in M$ istnieje więc $w \in B^{ * }$, taki że $f( w ) = m$.
Możemy więc przypisać elementowi $m$ liczbę z~$\Nbb_{ 0 }$, którą nazywamy
jego długością, za pomocą wzoru
\begin{equation}
  \label{eq:Forys-Forys-19}
  | m | := | w |.
\end{equation}
Tylko tyle jest nam potrzebne do udowodnienia twierdzenia 1.2.2.

Wybiegnijmy teraz trochę w~przyszłość. Jeśli monoid $M$ jest monoidem
wolnym, to~z~definicji jest on kanonicznie izomorficzny z~wolnym monoidem
językowym $A^{ * }$, dla pewnego $A \subset M$. Dzięki wnioskowi 1.2.2 i~dyskusji
przeprowadzonej w~tych notatkach na jego temat, zbiór $A$ jest jednoznacznie
wyznaczony przez relację
\begin{equation}
  \label{eq:Forys-Forys-20}
  A = S \setminus S^{ 2 }, \quad
  S = M \setminus \{ 1_{ M } \}.
\end{equation}
W~każdym monoidzie wolnym istnieje więc kanoniczne pojęcie długości
elementu $m \in M$. Mianowicie, niech $h : A^{ * } \to M$ będzie kanonicznym
izomorfizmem odpowiednich monoidów i~niech $m = h( w )$. Przez
\textbf{długość elementu $m \in M$} rozumiemy liczbę całkowitą nieujemną
określoną przez wzór \eqref{eq:Forys-Forys-20}.

\vspace{\spaceFour}









??????Drugą rzeczą na którą zwrócimy uwagę jest to, co~się dzieje w~sytuacji, gdy
odwzorowanie $h : A^{ * } \to M$ jest izomorfizmem. Będziemy przy tym rozważać
tylko sytuację, gdy $A \subset M$ i~$A^{ * }$ jest wolny monoidem językowym,
przypadek ogólniejszy pozostawiamy na razie otwarty. Mianowicie możemy wtedy
określić \textbf{długość $| m |$ elementu $m \in M$} za pomocą wzoru
\begin{equation}
  \label{eq:Forys-Forys-21}
  | m | := | w |, \quad
  m = h( w ),\; w \in A^{ * }.
\end{equation}

Jak pokazuje twierdzenie udowodnione na stronie 10, ta definicja jest dobrze
określona. ????

\vspace{\spaceFour}




\start \Str{10--11} W~kontekście podanych tu rozważań, należy zatrzymać~się
zatrzymać nad pewnymi własnościami potęg podzbiorów w~monoidach
i~półgrupach. Niech $S$ będzie dowolną podgrupą, zachodzi wówczas
\begin{equation}
  \label{eq:Forys-Forys-17}
  S \supset S^{ 2 } \supset S^{ 3 } \supset \ldots
\end{equation}
Ogólniej $S^{ k } \supset S^{ k + 1 }$ i~może zachodzić $S^{ k } \neq S^{ k + 1 }$. Choć dowód tego twierdzenia jest bardzo prosty to przeprowadzimy go powoli i~szczegółowo, by dokładnie zrozumieć o~co w~nim chodzi.

Rozpatrzmy przypadek $S \supset S^{ 2 }$. Każdy element jest $S^{ 2 }$ jest postaci $ab$, gdzie $a, b \in S$. Z~definicji działania wewnętrznego mamy $ab = c$, $c \in S$, z~czego wynika, że~$S \supset S^{ 2 }$.

Weźmy teraz przypadek $S^{ 2 } \supset S^{ 3 }$. Rozpatrzymy element
\begin{equation}
  \label{eq:2}
  abc \in S^{ 3 }, \quad a, b, c \in S.
\end{equation}
Znowu, korzystając z~podstawowych własności działania wewnętrznego w~półgrupie mamy
\begin{equation}
  \label{eq:3}
  abc = ( ab ) c.
\end{equation}
Oczywiście zachodzi $ab = a_{ 1 }$, dla pewnego $a_{ 1 } \in S$, więc
\begin{equation}
  \label{eq:4}
  abc = ( ab ) c = a_{ 1 } c,
\end{equation}
a~wyrażenie $a_{ 1 } c$ należy do zbioru $S^{ 2 }$ z~definicji.

Analogicznie przeprowadzamy dowód dla dowolnego $k$. Weźmy iloczyn
\begin{equation}
  \label{eq:5}
  a_{ 1 } a_{ 2 } a_{ 3 } a_{ 4 } \ldots a_{ k + 1 } \in S^{ k }, \quad
  a_{ i } \in S,\, i = 1, 2, \ldots, n.
\end{equation}
Przekształcamy go w~sposób analogiczny co poprzednio.
\begin{equation}
  \label{eq:6}
  a_{ 1 } a_{ 2 } a_{ 3 } a_{ 4 } \ldots a_{ k + 1 } =
  ( a_{ 1 } a_{ 2 } ) a_{ 3 } a_{ 4 } \ldots a_{ k + 1 } =
  b a_{ 3 } a_{ 4 } \ldots a_{ k + 1 } \in S^{ k },
\end{equation}
gdzie $b = a_{ 1 } a_{ 2 } \in S$. Tym samym dowiedliśmy,
że~$S^{ k } \supset S^{ k + 1 }$.

Gdybyśmy rozważali monoid $M$, to nie zaobserwowalibyśmy nic ciekawego.
Wynika to z~tego, że $1_{ M } \in M$, więc $M^{ 2 } = M$. Sprawa robi~się ciekawsza, gdy rozważamy zbiór $S = M \setminus \{ 1_{ M } \}$. Nie możemy zagwarantować, że~$S \supset S^{ 2 }$, bo może~się zdarzyć, że~dla dwóch elementów $a, b \in S$ zachodzi $a b = 1_{ M }$. Możemy jednak zauważyć, że~zachodzi następujące twierdzenie.





% #############
\begin{theorem}
  \label{thm:Forys-Forys-01}

  Niech $M$ będzie monoidem. Zbiór $S = M \setminus \{ 1_{ M } \}$ jest półgrupą wtedy
  i~tylko wtedy, gdy grupa jedynki monoidu $M$ jest trywialna.

\end{theorem}



\begin{proof}

  Dowód tego twierdzenia jest w~istocie bardzo prosty. Po pierwsze
  przypomnijmy, że~z~definicji zbiór $S = M \setminus \{ 1_{ M } \}$ jest półgrupą
  wtedy i~tylko wtedy, gdy $S \supset S^{ 2 }$.

  $\Rightarrow$ Wystarczy pokazać, że~jeśli $a b = 1_{ M }$ to $a = 1_{ M }$ i~tym
  samym $b = 1_{ M }$. Ponieważ $1_{ M } \notin S$ i~$S^{ 2 } \subset S$, więc jeśli
  istniej taka para $a, b \in M$, że~$ab = 1_{ M }$ to $a$ lub $b$ musi
  należeć do zbioru $M \setminus S = \{ 1_{ M } \}$, co kończy tę część dowodu.

  $\Leftarrow$ Dowód przeprowadzimy przez kontrapozycję. Jeżeli grupa jedynki nie
  jest trywialna, to istnieją dwa takie elementy, że~$a b = 1_{ M }$, przy
  czym $a \neq 1_{ M }$ i~$b \neq 1_{ M }$. Tym samym mielibyśmy $a, b \in S$,
  $ab = 1_{ M } \in S^{ 2 }$ i~jednocześnie $ab \notin S$. Dowód został zakończony.

\end{proof}
% #############





\start \Str{10} Zatrzymajmy~się na chwilę nad wnioskiem 1.2.2.
Stwierdzanie, że~baza $B$ wolnego monoidu $M$ jest minimalnym zbiorem
generatorów oznaczy tyle, że dla każdego zbioru generatorów $D$ zachodzi
$B \subset D$. Z~dowodu twierdzenia 1.2.2 wynika, że~$A \subset B$, gdzie
$A = S \setminus S^{ 2 }$. Jeśli ponadto pokazalibyśmy, że~$B \subset D$, to otrzymujemy,
że~$A = B$. W~tym momencie jedyny dowód inkluzji $B \subset D$ jaki potrafię
wymyślić, przebiega w~następujący sposób.

Jeżeli zbiór $D$ zawiera element neutralny, to możemy go usunąć z~niego bez
straty ogólności, por. wzór na stronie~7 omawianej książki. Od tego momentu
przyjmujemy więc, że~$1_{ M } \notin D$.

Każdy element $d$ zbioru $D$ jest w~sposób koniczny izomorficzny z~pewny
ciągiem elementów zbioru $B$:
$\texttt{"} d \texttt{"} = h^{ -1 }( d ) = \texttt{"} b_{ 1 } b_{ 2 } \ldots b_{ n }
\texttt{"} \in B^{ * }$. Tym samym jeśli $b = d_{ 1 } d_{ 2 } \ldots d_{ m }$,
$d_{ i } \in D$, $i = 1, 2, \ldots, m$ to musi zachodzić
\begin{equation}
  \label{eq:8}
  \texttt{"} d_{ 1 } d_{ 2 } \ldots d_{ m } \texttt{"} =
  \texttt{"} b \texttt{"} = h^{ -1 }( b ) \in B^{ * }.
\end{equation}
Ponieważ każdy $\texttt{"} d_{ 1 } \texttt{"}$ jest sam postaci
$\texttt{"} b_{ i,\, 1 } b_{ i,\, 2 } \ldots b_{ i,\, n } \texttt{"}$, gdzie
$b_{ i, j } \in B$, $i = 1, 2, \ldots, n$ jest ustaloną liczbą, $i$ jak poprzednio
przyjmuje wartości od $1$ do $m$, to na mocy tego co zostało powiedziane
przy okazji omawiania wzoru \eqref{eq:Forys-Forys-07}, powyższa równość
może zachodzić wtedy i~tylko wtedy, gdy $m = 1$ i~$d_{ 1 } = \texttt{"} b
\texttt{"}$. Czyli $\texttt{"} b \texttt{"} \in h^{ -1 }( D )$, co jest
równoważne temu, że~$h( \texttt{"} b \texttt{"} ) = b \in D$. Tym samym
dowiedliśmy, że~$B \subset D$.

Tym samym dowiedliśmy, że~baza $B$ jest nie tylko najmniejszym w~sensie
inkluzji zbiorem generatorów wolnego monoidu $M$, ale też, że~jest ona
wyznaczona jednoznacznie przez warunek $B = S \setminus S^{ 2 }$.

\vspace{\spaceFour}



\start \Str{11} W~dowodzie twierdzenia 1.2.4 jest napisane „Jeśli monoid jest wolny, to na mocy twierdzenia 1.2.2 spełnia wszystkich powyższe warunki”, gdzie warunki o~których mowa to to, że~monoid $M$ jest równo podzielny, $S = M \setminus \{ 1_{ M } \}$ jest podgrupą i~$\cap_{ n \in \Nbb } S^{ n } = \emptyset$. Jednak według mnie

% \label{eq:Forys-Forys-07}



% ##################
\CenterBoldFont{Błędy}


\begin{center}

  \begin{tabular}{|c|c|c|c|c|}
    \hline
    & \multicolumn{2}{c|}{} & & \\
    Strona & \multicolumn{2}{c|}{Wiersz} & Jest
                              & Powinno być \\ \cline{2-3}
    & Od góry & Od dołu & & \\
    \hline
    5  & 11 & & $\forall x, y, z \in S$ & $\forall x, y, z \in S,$ \\
    5  & 15 & & $\forall x \in M$ & $\forall x \in M,$ \\
    5  & & 13 & $( \textrm{\textbf{S}}, \cdot )$ & $( S, \cdot )$ \\
    5  & & 12 & $( \textrm{\textbf{M}}, \cdot, 1_{ \textrm{\textbf{M}}} )$
           & $( M, \cdot, 1_{ M } )$ \\
    5  & &  1 & $x \in \textrm{\textbf{M}}$ & $x \in M$ \\
    5  & &  1 & $\exists b \in B\; ,$ & $\exists b \in B,$ \\
    6  &  9 & & $( S, \cdot )${ }, { }{ }$( S', \ast )$
           & $( S, \cdot )$, $( S', \ast )$ \\
    6  & 11 & & $\forall x, y \in S$ & $\forall x, y \in S,$ \\
    6  & 14 & & $\forall x, y \in M$ & $\forall x, y \in M,$ \\
    6  & & 14 & $x \cdot y = 1$ & $x \cdot y = 1_{ M }$ \\
    7  & & 12 & $\forall x, y, z \in S$ & $\forall x, y, z \in S,$ \\
    7  & & 10 & $\forall x, y, z \in S$ & $\forall x, y, z \in S,$ \\
    7  & &  8 & $\forall x, y, z \in S$ & $\forall x, y, z \in S,$ \\
    7  & &  4 & $S / \rho$ & $S_{ / \rho }$ \\
    7  & &  3 & $M / \rho$ & $M_{ / \rho }$ \\
    8  & & 10 & $n \geq 0\;\;,$ & $n \geq 0,$ \\
    8  & &  7 & $n \!\! = \!\! 0$ & $n = 0$ \\
    9  & 22 & & $f : A \mapsto M$ & $f : A \to M$ \\
    9  & 23 & & $h : A^{ * } \mapsto M$ & $h : A^{ * } \to M$ \\
    9  & &  9 & $\to \textrm{\textbf{M}}$ & $\to M$ \\
    % 10 & & 10 &  & \\
    10 & 18 & & $id_{ A } : A \mapsto M$ & $id_{ A } : A \to M$ \\
    % & & & & \\
    % & & & & \\
    % & & & & \\
    % & & & & \\
    \hline
  \end{tabular}

\end{center}


\noindent
\StrWd{6}{1} \\
\Jest  \textit{dla} $x \in S^{ 1 }$ $\rho_{ a }( x )= a x$. \\
\Powin $\rho_{ a }( x ) = a x$ \textit{dla} $x \in S^{ 1 }$. \\
\textbf{Str. 8, pierwszy rysunek.} \\
\Jest  $S /_{ Ker_{ h } }$ \\
\Powin $S_{ / Ker_{ h } }$ \\
\textbf{Str. 8, drugi rysunek.} \\
\Jest  $A^{ * } /_{ Ker_{ h } }$ \\
\Powin $M_{ / Ker_{ h } }$ \\
\StrWg{9}{1} \\
\Jest  wolną półgrupę \\
\Powin wolną półgrupę językową \\
\StrWg{9}{2} \\
\Jest  wolny monoid \\
\Powin wolny monoid językowy \\
\StrWg{9}{2} \\
\Jest  wolna półgrupa \\
\Powin wolna półgrupa językowa \\
\StrWg{9}{3} \\
\Jest  nazywamy \\
\Powin językowego nazywamy \\
\StrWg{9}{4} \\
\Jest  nazywamy \\
\Powin językowego nazywamy \\
\StrWg{9}{6} \\
\Jest  nazywamy \\
\Powin językowego nazywamy \\
\StrWg{9}{6} \\
\Jest  monoid wolny \\
\Powin językowy monoid wolny \\
\StrWg{9}{6} \\
\Jest  półgrupa wolna \\
\Powin językowa półgrupa wolna \\
\StrWg{9}{12} \\
\Jest  wolnych monoidów \\
\Powin wolnych monoidów językowych \\
\StrWg{9}{12} \\
\Jest  wolnych półgrupach \\
\Powin wolnych półgrupach językowych \\
\StrWg{9}{14} \\
\Jest  monoidzie \\
\Powin monoidzie językowym \\
\StrWg{9}{19} \\
\Jest  \textit{monoidu $A^{ * }$} \\
\Powin \textit{wolnego monoidu językowego $A^{ * }$} \\
\StrWg{9}{25} \\
\Jest  \textit{homomorfizm} \\
\Powin \textit{homomorfizm z~wolnego monoidu językowego w~dowolny monoid} \\
\StrWd{9}{6} \\
\Jest  \textit{monoidu} $A^{ * }$ \\
\Powin \textit{monoidu językowego} $A^{ * }$ \\
\StrWd{9}{3} \\
\Jest  wolnych monoidów (półgrup) \\
\Powin wolnych monoidów (półgrup) językowych \\
\StrWg{10}{3} \\
\Jest  \textit{to znaczy} $M = B^{ * }$ \\
\Powin \textit{czyli $M$ jest kanonicznie izomorficzny z~pewnym wolnym
  monoidem językowym $B^{ * }$} \\
\StrWd{10}{11} \\
\Jest  \textit{generatorów.} \\
\Powin \textit{generatorów i~jest równa $S \setminus S^{ 2 }$, gdzie
  $S = M \setminus \{ 1_{ M } \}$}. \\
\StrWg{10}{18} \\
\Jest  $h : A^{ * } \mapsto M$ \\
\Powin $h : A^{ * } \to M$, gdzie $A^{ * }$ jest wolnym monoidem językowym. \\
\StrWd{10}{23} \\
\Jest  \textit{Homomorfizm $h$ jest, jak łatwo stwierdzić, izomorfizmem,
  więc monoid $M$ jako izomorficznym z~$A^{ * }$ jest wolny.} \\
\Powin \textit{Ponownie, z~faktu że każdy element $m \in S$ można rozłożyć na
  elementy ze zbioru $A$, wynika, że homomorfizm $h$ jest suriekcją.
  Z~jednoznaczności tego rozkładu wynika, że~$h$ jest infekcją, a~więc
  izomorfizmem. Tym samym monoid $M$ jest izomorficzny z~wolnym monoidem
  językowym $A^{ * }$, zgodnie więc z~definicją jest monoidem wolnym.}




\vspace{\spaceTwo}
% ############################










% #####################################################################
% #####################################################################
% Bibliografia
\bibliographystyle{plalpha}

\bibliography{MathComScienceBooks}{}





% ############################

% Koniec dokumentu
\end{document}
