% ---------------------------------------------------------------------
% Podstawowe ustawienia i pakiety
% ---------------------------------------------------------------------
\RequirePackage[l2tabu, orthodox]{nag} % Wykrywa przestarzałe i niewłaściwe
% sposoby używania LaTeXa. Więcej jest w l2tabu English version.
\documentclass[a4paper,11pt]{article}
% {rozmiar papieru, rozmiar fontu}[klasa dokumentu]
\usepackage[MeX]{polski} % Polonizacja LaTeXa, bez niej będzie pracował
% w języku angielskim.
\usepackage[utf8]{inputenc} % Włączenie kodowania UTF-8, co daje dostęp
% do polskich znaków.
\usepackage{lmodern} % Wprowadza fonty Latin Modern.
\usepackage[T1]{fontenc} % Potrzebne do używania fontów Latin Modern.



% ------------------------------
% Podstawowe pakiety (niezwiązane z ustawieniami języka)
% ------------------------------
\usepackage{microtype} % Twierdzi, że poprawi rozmiar odstępów w tekście.
\usepackage{graphicx} % Wprowadza bardzo potrzebne komendy do wstawiania
% grafiki.
\usepackage{verbatim} % Poprawia otoczenie VERBATIME.
\usepackage{textcomp} % Dodaje takie symbole jak stopnie Celsiusa,
% wprowadzane bezpośrednio w tekście.
\usepackage{vmargin} % Pozwala na prostą kontrolę rozmiaru marginesów,
% za pomocą komend poniżej. Rozmiar odstępów jest mierzony w calach.
% ------------------------------
% MARGINS
% ------------------------------
\setmarginsrb
{ 0.7in}  % left margin
{ 0.6in}  % top margin
{ 0.7in}  % right margin
{ 0.8in}  % bottom margin
{  20pt}  % head height
{0.25in}  % head sep
{   9pt}  % foot height
{ 0.3in}  % foot sep



% ------------------------------
% Często przydatne pakiety
% ------------------------------
\usepackage{csquotes} % Pozwala w prosty sposób wstawiać cytaty do tekstu.
\usepackage{xcolor} % Pozwala używać kolorowych czcionek (zapewne dużo
% więcej, ale ja nie potrafię nic o tym powiedzieć).



% ------------------------------
% Pakiety do tekstów z nauk przyrodniczych
% ------------------------------
\let\lll\undefined % Amsmath gryzie się z językiem pakietami do języka
% polskiego, bo oba definiują komendę \lll. Aby rozwiązać ten problem
% oddefiniowuję tę komendę, ale może tym samym pozbywam się dużego Ł.
\usepackage[intlimits]{amsmath} % Podstawowe wsparcie od American
% Mathematical Society (w skrócie AMS)
\usepackage{amsfonts, amssymb, amscd, amsthm} % Dalsze wsparcie od AMS
% \usepackage{siunitx} % Dla prostszego pisania jednostek fizycznych
\usepackage{upgreek} % Ładniejsze greckie litery
% Przykładowa składnia: pi = \uppi
\usepackage{slashed} % Pozwala w prosty sposób pisać slash Feynmana.
\usepackage{calrsfs} % Zmienia czcionkę kaligraficzną w \mathcal
% na ładniejszą. Może w innych miejscach robi to samo, ale o tym nic
% nie wiem.



% ##########
% Tworzenie otoczeń "Twierdzenie", "Definicja", "Lemat", etc.
\newtheorem{theorem}{Twierdzenie}  % Komenda wprowadzająca otoczenie
% „theorem” do pisania twierdzeń matematycznych
\newtheorem{definition}{Definicja}  % Analogicznie jak powyżej
\newtheorem{corollary}{Wniosek}



% ---------------------------------------
% Pakiety napisane przez użytkownika.
% Mają być w tym samym katalogu to ten plik .tex
% ---------------------------------------
\usepackage{latexgeneralcommands}
\usepackage{mathcommands}
% \usepackage{calculuscommands}
\newcommand{\conca}{\textrm{conca}}
% \usepackage{SchwartzBooksCommands}  % Pakiet napisany m.in. dla tego pliku.



% ---------------------------------------------------------------------
% Dodatkowe ustawienia dla języka polskiego
% ---------------------------------------------------------------------
\renewcommand{\thesection}{\arabic{section}.}
% Kropki po numerach rozdziału (polski zwyczaj topograficzny)
\renewcommand{\thesubsection}{\thesection\arabic{subsection}}
% Brak kropki po numerach podrozdziału



% ------------------------------
% Ustawienia różnych parametrów tekstu
% ------------------------------
\renewcommand{\arraystretch}{1.2} % Ustawienie szerokości odstępów między
% wierszami w tabelach.



% ------------------------------
% Pakiet "hyperref"
% Polecano by umieszczać go na końcu preambuły.
% ------------------------------
\usepackage{hyperref} % Pozwala tworzyć hiperlinki i zamienia odwołania
% do bibliografii na hiperlinki.










% ---------------------------------------------------------------------
% Tytuł i autor tekstu
\title{Teoria automatów i~języków formalnych \\
  Błędy i~uwagi}

\author{Kamil Ziemian}


% \date{}
% ---------------------------------------------------------------------










% ####################################################################
\begin{document}
% ####################################################################





% ######################################
\maketitle % Tytuł całego tekstu
% ######################################





% ############################
\Work{ % Autorzy i tytuł dzieła
  Maria Foryś, Wit Foryś \\
  \textit{Teoria automatów i~języków formalnych},
  \cite{ForysForysTeoriaAutomatowIJezykowFormalnych2005}}


% ##################
\CenterBoldFont{Uwagi}


\start Zawartość tej książeczki jest pod pewnym względami większa, zaś
pod pewnymi względami mniejsza, niż kursu online \textit{Języki,
  automaty i~obliczania}, autorstwa, nota bene, Marii Foryś, Wita
Forysia i~Adama Romana, warto więc przerabiać go równolegle z~tą
książeczką. Można go znaleźć pod tym\footnote{Pełna forma linku:
  \href{https://wazniak.mimuw.edu.pl/index.php?title=J\%C4\%99zyki,\_automaty\_i\_obliczenia}
  {https://wazniak.mimuw.edu.pl/index.php?title=J\%C4\%99zyki,\_automaty\_i\_obliczenia}.}
\colorhref{https://wazniak.mimuw.edu.pl/index.php?title=J\%C4\%99zyki,\_automaty\_i\_obliczenia}{linkiem}.

\vspace{\spaceFour}



\start W~tej książce używany jest cudzysłów w~formie ”cytowany tekst”,
ale polskie standardy typograficzna mówią, że~powinno~się stosować
formę: „cytowany tekst”. Dodatkowo, wyjątek warto uczynić dla ciągów
symboli (liter z~danego alfabetu $A$), nazywanych napisami bądź
stringami (ang. \textit{strings}), które zgodnie z~przyjętą
w~informatyce konwencją będziemy oznaczać jako \texttt{"abc"}.

Więcej o~napisach powiemy przy okazji omawiania monoidów wolnych.



% ##################
\CenterBoldFont{Uwagi do konkretnych stron}

% \vspace{\spaceFour}


\start \Str{6} Według mnie symbol $\textrm{mod}_{ 6 }$ wygląda
znacznie lepiej, niż używany w tej książce $mod_{ 6 }$.

\vspace{\spaceFour}



\start \Str{6} Na tej stronie znajdujemy informację, że~element
$x \in M$ jest odwracalny w~monoidzie $M$ wtedy i tylko wtedy, gdy
istnieje takie $y \in M$, że $x \cdot y = 1_{ M }$. Jednak monoid nie
musi być przemienny, więc nie widzę powodu by zachodziła też równość
$y \cdot x = 1_{ M }$. Co więcej, wolne monoidy językowe, które
zdefiniujemy dalej, a~które są dla nas szczególnie interesujące, nie
są przemienne, więc problem jest tym poważniejszy.

Wydaje mi, że jest to zwykłe przeoczenie ze strony autorów i~trzeba po
prostu przyjąć, że~element $x \in M$ nazywamy odwracalny, gdy istnieje
taki element $y \in M$, że~zachodzi
\begin{equation}
  \label{eq:Forys-Forys-01}
  x \cdot y = y \cdot x = 1_{ M }.
\end{equation}

\vspace{\spaceFour}



\start \Str{6} Na tej stronie znajdujemy stwierdzenia, że „zbiór
elementów odwracalnych monoidu $M$ stanowi podgrupę tego monoidu”.
Zwykle przez podgrupę rozumie~się podzbiór $G_{ 1 }$ grupy $G_{ 0 }$,
taki że~wraz z~działaniem odziedziczonym z~$G_{ 0 }$ jest grupą. Tutaj
jednak jest mowa, że~grupa jedynki, oznaczmy ją $U$ (od ang.
\textit{unity}), jest podgrupą monoidu, który nie musi być podgrupą.

Sens tego jest dość oczywisty. Mianowicie, że trójka uporządkowana
$( U, \cdot, 1 )$, gdzie $\cdot$ jest działaniem odziedziczonym po
monoidzie $M$, zaś $1 \in U$, jest elementem neutralnym tego monoidu,
jest grupą. Jednak czyni to cały system pojęciowy trochę mniej
ścisłym.

Moglibyśmy doprecyzować ten formalizm za pomocą pojęcia algebry
ogólnej (zob. \cite{BialynickiBirulaZarysAlgebry1987}), lecz zamiast
tego będziemy po prostu mówić, że~$U$ jest podzbiorem $M$ i~jest
grupą. Pod tym kontem dokonamy pewnych poprawek w tekście.

\vspace{\spaceFour}



\start \Str{6} W~dowodzie twierdzenia 1.1.1 brakuje mi zdania typu
„Jeśli $S$ nie jest monoidem, to rozszerzamy go do monoidu, za pomocą
procedury którą zaprezentujemy poniżej.”.

\vspace{\spaceFour}



\start \Str{6} Dowodu twierdzenia 1.1.1 zawiera dodatkową informację,
nad którą warto~się zatrzymać, mianowicie że~każdą półgrupę można
rozszerzyć do monoidu. Wymaga to jedynie
oczywistego/nieoczywistego\footnote{Zależy to od przekonań
  filozoficzno-matematycznych konkretnej osoby.}, że~istnieje element
$1 \notin S$.

W~kontekście samego dowodu twierdzenia 1.1.1, warto~się zastanowić nad
tym dlaczego dokonujemy rozszerzenia półgrupy $( S, \cdot )$ do
monoidu $( S^{ 1 }, \cdot, 1 )$. Dzięki temu mamy rodzinę odwzorowań
$\rho_{ a } : S^{ 1 } \to S^{ 1 }$ i~ponieważ $S^{ 1 }$ jest monoidem,
to mamy $\rho_{ a }( 1 ) = a$, ta zaś równość pozwala pokazać,
że~z~$\rho_{ a_{ 1 } } = \rho_{ a_{ 2 } }$ wynika $a_{ 1 } = a_{ 2 }$.
Z tego zaś od razu wynika, że odwzorowanie
$h : S \to ( \{ \rho_{ a } \}_{ a \in S } )$, $h( a ) = \rho_{ a }$
jest iniektywne. Poza tym rozszerzenie półgrupy do monoidu nie wydaje
się nigdzie indziej potrzebne.

Pytanie, czy istnieje sposób pokazania, że~odwzorowanie $h$ jest
iniektywne, bez rozszerzania półgrupy $S$ do monoidu $S^{ 1 }$? Nawet
jeżeli tak, to prostota przeprowadzonego w~książce dowodu sprawia, że
warto przy nim pozostać.

\vspace{\spaceFour}



\start \Str{7} Tak jak w~przypadku symbolu $\textrm{mod}_{ 6 }$,
wydaje mi~się, że~lepiej wyglądałby symbol $\textrm{Ker}_{ h }$.

\vspace{\spaceFour}



\start \Str{8} Rysunki na tej stronie są zrobione dość niechlujnie.
Wystarczy zwrócić uwagę, że~symbol $h$ leży w~różnych odległości od
lewego marginesu, podczas, gdy powinien być w~tej samej pozycji.

\vspace{\spaceFour}



\start \Str{8} Wedle podanej tu definicji wolny monoid $M$ nad
alfabetem $A$, który będziemy oznaczać również symbolem $A^{ * }$,
jest obiektem o~pewnej bardzo konkretnej strukturze ontologicznej.
Sprawa ta prowadzi do pewnych niejasności w~dalszym wykładzie
materiału.

Mianowicie bierzemy pewien zbiór $A$, którego elementy noszą nazwę
„liter” lub „symboli” (patrz uwaga do str. 9) następnie tworzymy zbiór
wszystkich ciągów tych symboli, który będziemy oznaczać symbolem
$A^{ * }$.
\begin{equation}
  \label{eq:Forys-Forys-02}
  A^{ * } =
  \{ ( a_{ 1 }, a_{ 2 }, a_{ 3 }, \ldots, a_{ n } ) \;\; : \;\; n \geq 0, \;
  a_{ i } \in A \}.
\end{equation}
Wedle tej definicji do $A^{ * }$ należy ciąg pusty, oznaczany symbolem
$1 \equiv \textrm{""}$. Taki ciąg intuicyjnie jest prosty do pojęcia,
zaś~bardziej formalnie jest to odwzorowanie $f : \emptyset \to A$.

W~zbiorze $A^{ * }$ wprowadzamy też działanie $\cdot$ wzorem
\begin{equation}
  \label{eq:Froys-Forys-03}
  ( a_{ 1 }, a_{ 2 }, \ldots, a_{ n } ) \cdot ( b_{ 1 }, b_{ 2 }, \ldots, b_{ m } ) =
  ( a_{ 1 }, a_{ 2 }, \ldots, a_{ n }, b_{ 1 }, b_{ 2 }, \ldots, b_{ m } ).
\end{equation}

W~dalszym ciągu książki dowiadujemy~się, że klasycznym monoidem wolnym
jest $( \Nbb_{ 0 }, +, 0 )$. Powstaje teraz pytanie, czy aby móc
powiedzieć, że~zbiór liczb naturalnych z~$0$ jest monoidem wolnym, nie
musimy go przedstawić jako zbioru ciągów jakiś symboli, tak jak jest
to przedstawione w~powyższej definicji? To jednak rodzi pewne
problemy.

Zauważmy, że~możemy wprawdzie przyjąć\footnote{Nie będziemy tu
  poruszać niezwykle interesującego i~ważnego zagadnienie, tego
  mianowicie, że~w~pewnych teoriach typów „obiekty izomorficzne są
  identyczne” i~tym samym, nie ma dwóch różnych „modeli liczb
  naturalnych” z~zerem czy bez. Zawędrowalibyśmy tym sposobem zbyt
  daleko od właściwego tematu.}, że~zbiór $A = \{ | \}$, liczba $0$ to
pusty ciąg, $1 = ( | )$, $2 = ( |, | )$, $3 = ( |, |, | )$, etc.,
jednak podstawowa intuicja matematyczna mówi, że~wprowadzanie takiej
konstrukcji nie powinno być konieczne. Zanim przejdziemy do dyskusji
tego zagadnienia, zauważmy, że~w tak podanym wolnym monoidzie
$A^{ * }$ konkatenacja ciągów rzeczywiście prowadzi do tego co
intuicyjnie rozpoznajemy jako dodawanie liczb naturalnych. Przykładowo
\begin{equation}
  \label{eq:Forys-Forys-04}
  1 + 2 = ( | ) \cdot ( |, | ) = ( |, |, | ) = 3.
\end{equation}

Wróćmy teraz do problemu, czemu takie podejście do liczby naturalnych
jawi~się intuicji matematycznej jako niekonieczne. Powinno być bowiem
możliwe myślenie o~liczbach naturalnych jako „po prostu o liczbach”,
jako autonomicznych bytach, nie zaś jako ciągach jakiegoś symbolu.
Dodatkowo, taka definicji to błędne koło, samo bowiem pojęcie ciągu
o~skończonej długości wymaga uznania, że~wiemy czym jest zbiór liczb
$\{ 1, 2, \ldots, n \}$. Skończony ciąg symboli to bowiem nic innego jak
odwzorowanie $f : \{ 1, 2, \ldots, n \} \to A$. Z~tego powodu potrzebujemy
definicji liczb naturalnych, która nie bazuje na pojęciu ciągu, jedną
z~nich jest ta, która bazuje na użyciu zbioru pustego.

Rozważmy teraz inny przykład. Zbiór wszystkich macierzy kwadratowych
wymiaru $n$, który będziemy oznaczać symbolem $\GL( m )$, możemy
rozpatrywać jako monoid $( \GL( m ), \cdot, \matUnit )$, gdzie $\cdot$
to mnożenie macierzy, zaś $\matUnit$ to macierz jednostkowa
$m \times m$. Czy jest więc w~ogóle sens w~zadawaniu pytania~się, czy
ten monoid zawiera jakiś podmonoid wolny, rozumiany jako zbiór ciągów
z~działaniem konkatenacji? Wszak mając dwie macierze $A$ i~$B$
należące do $\GL( m )$ musielibyśmy z~jednej strony rozważać ich
iloczyn $A \cdot B$, z~drugiej ciąg dwuelementowy $( A, B )$ i~ustalić
relację między tymi dwoma bytami. Nie jest powiedziane, że~nie można
utożsamić konkretnej macierzy kwadratowej, z~ciągiem jakiejś zbioru
macierzy, acz taka konstrukcja wydaje~się znacznie mniej naturalna,
niż przedstawiona powyżej dla liczb naturalnych.

W~świetle tego proponuję następujące wyjście z~sytuacji.
\textbf{Wolnym monoidem językowym $A^{ * }$ nad alfabetem $A$}
będziemy nazywać wcześniej określony zbiór ciągów wraz z~działaniem
konkatenacji. Jest więc to obiekt o~konkretnej strukturze
ontologicznej. Teraz ustalamy, że~monoid $( M, \cdot, 1_{ M } )$ jest
monoidem wolnym, jeśli jest izomorficzny z~pewnym wolnym monoidem
językowym $A^{ * }$ nad alfabetem $A$. Zbiór $A$ może być podzbiorem
$M$ lub nie. Jak pokazują twierdzenia udowodnione na stronach 9--12,
jeśli istnieje dowolny zbiór $A$, taki że $M$ jest izomorficzny
z~wolnym monoidem językowym $A^{ * }$, to można znaleźć taki zbiór
$B \subset M$, że~$M$ jest izomorficzny z~wolnym monoidem językowym
$B^{ * }$. Jednak poprawne omówienie tej kwestii wymaga uporządkowania
kilku pojęć.

Również z~tych twierdzeń wynika wniosek, że~jeśli dany monoid
$( M, \cdot, 1_{ M } )$ jest monoidem wolnym, to istnieje taki zbiór
$B$, że~każdy element $m \in M$, $m \neq 1_{ M }$ można przedstawić
w~jeden i~tylko jeden sposób jako
\begin{equation}
  \label{eq:Forys-Forys-04}
  b_{ 1 } \cdot b_{ 2 } \cdot \ldots \cdot b_{ n }, \quad
  b_{ i } \in B.
\end{equation}
Zauważmy, że~w~powyższym wyrażeniu może zachodzić $b_{ i } = b_{ j }$
dla $i \neq j$. W~istocie ta własność jest równoważna przyjętej
definicji i~tym samym można by ją przyjąć w~jej miejsce. Nie
zrobiliśmy tego, bo zapisanie tej definicji w~sposób precyzyjny „od
zera” wymagało tyle miejsca, że~nie wygląda na to, byśmy w~ten sposób
coś zyskiwali.

\vspace{\spaceFour}



\start \Str{8} Korzystając z~wprowadzonych wcześniej pojęć, możemy
teraz zająć~się niejasnościami, jakie wynikają, z~użycia w~tej książce
symbolu $A^{ * }$ w~dwóch różnych znaczeniach. Jeżeli bowiem
$A \subset M$, gdzie $M$ jest pewnym monoidem, to symbol $A^{ * }$
może oznaczać zarówno wolny monoid językowy nad alfabetem $A$, jak
i~podmonoid monoidu $M$ generowany przez $A$, są to więc dwa różne
byty.

Jeśli przyjmiemy, że~$M$ to omawiany już wcześniej monoid macierzy
kwadratowych $( \GL( m ), \cdot, \matUnit )$, zaś
$A = \{ B_{ 1 }, B_{ 2 }, B_{ 3 }, \ldots, B_{ n } \}$, to symbol
$A^{ * }$ możne oznaczać zbiór ciągów macierzy kwadratowych,
np.~$( B_{ 1 }, B_{ 2 }, B_{ 3 } )$, jak i~zbiór macierzy kwadratowych
utworzonych jako iloczyny macierzowy macierzy ze zbioru $A$. W~tym
drugim przypadku do tego zbioru należy choćby macierz
$B_{ 1 } \cdot B_{ 2 } \cdot B_{ 3 }$.

By usunąć tego typu dwuznaczności przyjmiemy następującą konwencję.
Jeżeli jest jawnie powiedziane, że~$A$ jest podzbiorem monoidu $M$, to
przez symbol $A^{ * }$ będziemy rozumieli najmniejszy podmonoid $M$
zawierający $A$. Taki obiekt będziemy nazywali \textbf{podmonoidem
  algebraicznym}.

Jeżeli zbiór $A$ jest podany, ale bez wskazania, że
jest to podzbiór pewnego monoidu $M$, to przez symbol $A^{ * }$
będziemy rozumieć wolny monoid językowy nad alfabetem $A$. W~przypadku
gdy $A$ jest podzbiorem monoidu $M$, ale~rozważać będziemy wolny
monoid językowy nad $A$, to zostanie to zaznaczone jawnie. Szczególny
przypadkiem, acz nietrudnym do zrozumienia jest ten, gdy $A \subset M$
i~$M$ jest monoidem językowym. Wówczas obie definicje symbolu
$A^{ * }$~się pokrywają.

W~świetle powyższych ustaleń wprowadzimy odpowiednie poprawki do tej
książeczki.

\vspace{\spaceFour}



\start \Str{8} Dysponując już tak uporządkowanym słownictwem,
przeanalizujmy strukturę wolnego monoidu językowego $A^{ * }$ na
alfabetem $A$. Jeśli potrzebny będzie nam przykład wolnego monoidu
językowego, sięgniemy po niezwykle ważny w~informatyce monoid
$\{ 0, 1 \}^{ * }$. Łatwo spostrzec, że~jest to monoid nieprzemienny,
przykładowo jeśli $u = 00$, $b = 11$, to
\begin{equation}
  \label{eq:Forys-Forys-05}
  u v = 0011 \neq 1100 = vu.
\end{equation}
Nasuwa~się naturalne pytanie: co można powiedzieć o~zbiorze $C$ taki,
że dla wszystkich $u, v \in C$ zachodzi
\begin{equation}
  \label{eq:Forys-Forys-06}
  uv = vu.
\end{equation}
Problem ten wymaga dalszej analizy.

Zauważmy, że~dla monoidu językowego zachodzi równość
\begin{equation}
  \label{eq:Forys-Forys-07}
  | u v | = | u | + | v |.
\end{equation}
Wynika ona w~prosty sposób z~definicji konkatenacji. Z~niej i~z~tego,
że~$| u | \geq 0, u \in M$, wynika, że grupa jedynki w~wolnym
monoidzie językowym jest trywialna. Choć wniosek ten jest prosty do
zrozumienia, bardziej sformalizowany dowód wymagałby użycia
nierówności
\begin{equation}
  \label{eq:Forys-Forys-08}
  | u v | \geq | u |,
\end{equation}
które wynika wprost z~dwóch wspomnianych własności.

\vspace{\spaceFour}



\Str{8} Zwróćmy uwagę, że~jeśli $A \subset M$, gdzie $M$ jest monoidem, to
podmonoid algebraiczny $A^{ * }$ i~wolnym monoid językowy $A^{ * }$
i~alfabetem $A$ mogą być radykalnie różnymi obiektami.

Dla przykładu weźmy monoid $( \Nbb_{ 0 }, +, 0)$ i~niech $A = \{ 0 \}$. Wówczas
jeżeli przez symbol $A^{ * }$ rozumiemy podmonoid algebraiczny, to $A^{ * } =
\{ 0 \}$, bowiem $0 + 0 = 0$. Jeżeli, przez $A^{ * }$ rozumiemy wolny podmonoid
językowy to wówczas zachodzi
\begin{equation}
  \label{eq:Forys-Forys-09}
  A^{ * } = \{ \textrm{""}, 0, 00, 000, 0000, \ldots \}.
\end{equation}
Powyższy monoid językowy jest izomorficzny z~całym monoidem
$( \Nbb_{ 0 }, +, 0)$. Ten izomorfizm zadany jest przez relacje
\begin{subequations}
  \begin{align}
    \label{eq:Forys-Forys-10-A}
    h( \textrm{""} ) &= 0, \\
    h( \textrm{"} 0 \textrm{"} ) &= 1, \\
    h( \textrm{"} 0 0 \textrm{"} ) &= 2, \\
    h( \textrm{"} 0 0 0 \textrm{"} ) &= 3, \\
    &\vdots
  \end{align}
\end{subequations}
Dla większej przejrzystości, element wolnego monidłu językowego zapisujemy
jako ciągi symboli wewnątrz cudzysłowu \texttt{" "}. Tę praktykę będziemy
też stosować w~dalszym ciągu tych notatek, zawsze gdy zajdzie potrzeba
dokładniejszego wyjaśnienia znaczenia danych symboli.

\vspace{\spaceFour}



\start \Str{8--9} By uniknąć nieporozumień, należy jasno stwierdzić, że~gdy
będziemy mówić o~„zbiorze ciągów liter z~alfabetu $A$” to w domyśle do tego
zbioru zawsze należy też ciąg pusty \textrm{""}.

\vspace{\spaceFour}



\start \Str{9} Wedle przyjętej na tej stronie nomenklatury elementy
alfabetu będziemy nazywać „literami” i~będziemy mówić, że~„słowa są
ciągami liter”. Jednak ze względu na wygodę, warto przyjąć, że słowo
„symbol” jest synonimem słowa „litera”, dzięki czemu będziemy mogli
powiedzieć „elementami alfabetu są symbole” i~„słowa są ciągami
symboli”.

\vspace{\spaceFour}



\start \Str{9} Dowód twierdzenia 1.2.1 byłby znacznie prostszy do
zrozumienia, gdyby został wskazane, że~jeśli $h : A^{ * } \to M$ jest
rozszerzeniem odwzorowania $f : A \to M$ do homomorfizmu musi spełniać
następujące trzy warunki
\begin{subequations}
  \begin{align}
    \label{eq:Forys-Forys-11-A}
    h( 1 ) &= 1_{ M }, \\
    \label{eq:Forys-Forys-11-B}
    h( a ) &= f( a ), \quad \forall \, a \in A, \\
    \label{eq:Forys-Forys-11-C}
    h( a_{ 1 } \ldots a_{ n } )
           &= f( a_{ 1 } ) \cdot \ldots \cdot f( a_{ n } ), \quad
             \forall \, a_{ i } \in A^{ * }.
  \end{align}
\end{subequations}
Dla większej przejrzystości działanie wewnętrzne w~monoidzie $M$
zapisaliśmy za pomocą symbolu $\cdot$. Równania te gwarantują nam,
że~może istnieć co najwyżej jedno odwzorowanie $h : A^{ * } \to M$,
dla których są one spełnione.

Możemy teraz użyć równań
\eqref{eq:Forys-Forys-11-A}-\eqref{eq:Forys-Forys-11-C}, do
\textit{zdefiniowania} homomorfizmu $h$. Z~definicji $A^{ * }$ wynika,
że odwzorowanie dane tymi wzorami ma za swoją dziedzinę cały zbiór
$A^{ * }$. Dla każdego $a \in A^{ * }$ zachodzi też $h( a ) \in M$
i~podobnie można sprawdzić pozostałe warunki na to, by $h$ było
homomorfizmem monoidu $A^{ * }$ w~monoid $M$. Ponieważ są to dość
proste\footnote{Jak to niektórzy powiadają „Nie ma czego dowodzić”.}
opuścimy te rozważania.

\vspace{\spaceFour}




\start \Str{9} Warto zwrócić uwagę na to, że~omawiany tu specjalny
homomorfizm wolnego monoidu językowego $A^{ * }$ z~monoidem $M$, gdzie
$A \subset M$ jest zbiorem generatorów $M$, nie musi być izomorfizmem. Zanim
przejdziemy do dokładniejszego omówienia tej kwestii, przypomnijmy, że~ten
homomorfizm jest jednoznacznie wyznaczony przez warunek
\begin{equation}
  \label{eq:Forys-Forys-12}
  h( \textrm{"} a \textrm{"} ) = a, \quad
  a \in A.
\end{equation}

By to pokazać rozważmy
ponownie monoid $( \Nbb_{ 0 }, +, 0 )$ i~wolny monoid językowy
$A^{ * } = \{ 0, 1 \}^{ * }$. Jest
oczywiste, że
\begin{equation}
  \label{eq:Forys-Forys-13}
  A^{ * } =
  \{ \textrm{""}, \textrm{"} 0, \textrm{"} 1 \textrm{"},
  \textrm{"} 0 0 \textrm{"}, \textrm{"} 0 1 \textrm{"},
  \textrm{"} 1 1 \textrm{"}, \ldots \}.
\end{equation}
Homomorfizm $h : A^{ * } \to \Nbb_{ 0 }$ zadajemy za pomocą wzorów
\eqref{eq:Forys-Forys-11-A}-\eqref{eq:Forys-Forys-11-C}
i~\eqref{eq:Forys-Forys-12}, które w~twym przypadku przybierają postać
\begin{subequations}
  \begin{align}
    \label{eq:Forys-Forys-14-A}
    h( \textrm{""} ) &= 0, \\
    \label{eq:Forys-Forys-14-B}
    h( \textrm{"} 0 \textrm{"} ) &= 0, \quad
    h( \textrm{"} 1 \textrm{"} ) = 1, \\
    \label{eq:Forys-Forys-14-C}
    h( \textrm{"} a_{ 1 } a_{ 2 } \ldots a_{ n } \textrm{"} )
                     &= h( \textrm{"} a_{ 1 } \textrm{"} )
                       + h( \textrm{"} a_{ 2 } \textrm{"} ) + \ldots
                       + h( \textrm{"} a_{ n } \textrm{"} ) = \\
    % \nolabel
                     &= a_{ 1 } + a_{ 2 } + \ldots + a_{ n }, \quad
                       a_{ i } \in \{ 0, 1 \}.
  \end{align}
\end{subequations}
Przykładowo
\begin{equation}
  \label{eq:Forys-Forys-15}
  h( \textrm{"} 0100110 \textrm{"} ) =
  0 + 1 + 0 + 0 + 1 + 1 + 0 = 3.
\end{equation}

Łatwo zauważyć, że~odwzorowanie $h$ nie jest izomorfizmem. Świadczy o~tym
choćby poniższy przykład
\begin{equation}
  \label{eq:Forys-Forys-15}
  h( \textrm{""} ) = h( \textrm{"} 0 \textrm{"} ) = 0.
\end{equation}

\vspace{\spaceFour}



\start \Str{9} Potrzebujemy tu poczynić pewną uwagę odnośnie notacji.
Przyjmijmy, że $A$ jest podzbiorem monoidu $M$. W~skutek wybranych oznaczeń
zapis $M = A^{ * }$ może oznaczać, że~podmonoid algebraiczny $A^{ * }$ jest
równy $M$, z~drugiej zaś strony, iż~wolny podmonoid językowy $A^{ * }$ jest
izomorficzny z~$M$, gdzie izomorfizm ustala homomorfizm $h$, wyznaczony
jednoznacznie przez warunek \eqref{eq:Forys-Forys-12}. Obie te sytuacje nie
są wcale równoważne.

Rozpatrzmy ponownie monoid $( \Nbb_{ 0 }, +, 0 )$ i~wolny monoid językowy
$A^{ * } = \{ 0, 1 \}^{ * }$. Jest oczywiste, że~jeśli symbol $A^{ * }$ rozumiemy
jako podmonoid algebraiczny to oczywiście zachodzi $M = A^{ * }$.
Jeśli jednak przez $A^{ * }$ rozumiemy wolny monoid językowy, to jak zostało
wykazane w~poprzednich komentarzach, to homomorfizm $h$ \textit{nie jest}
izomorfizmem, więc nie możemy napisać równości „$M = A^{ * }$”.



???????????

Drugą rzeczą na którą zwrócimy uwagę jest to, co~się dzieje w~sytuacji, gdy
odwzorowanie $h : A^{ * } \to M$ jest izomorfizmem. Będziemy przy tym rozważać
tylko sytuację, gdy $A \subset M$ i~$A^{ * }$ jest wolny monoidem językowym,
przypadek ogólniejszy pozostawiamy na razie otwarty. Mianowicie możemy wtedy określić \textbf{długość $| m |$ elementu $m \in M$} za pomocą wzoru
\begin{equation}
  \label{eq:Forys-Forys-16}
  | m | := | w |, \quad
  m = h( w ),\; w \in A^{ * }.
\end{equation}

Jak pokazuje twierdzenie udowodnione na stronie 10, ta definicja jest dobrze
określona. ????

\vspace{\spaceFour}





% ##################
\CenterBoldFont{Błędy}


\begin{center}

  \begin{tabular}{|c|c|c|c|c|}
    \hline
    & \multicolumn{2}{c|}{} & & \\
    Strona & \multicolumn{2}{c|}{Wiersz} & Jest
                              & Powinno być \\ \cline{2-3}
    & Od góry & Od dołu & & \\
    \hline
    5  & 11 & & $\forall x, y, z \in S$ & $\forall x, y, z \in S,$ \\
    5  & 15 & & $\forall x \in M$ & $\forall x \in M,$ \\
    5  & & 13 & $( \textrm{\textbf{S}}, \cdot )$ & $( S, \cdot )$ \\
    5  & & 12 & $( \textrm{\textbf{M}}, \cdot, 1_{ \textrm{\textbf{M}}} )$
           & $( M, \cdot, 1_{ M } )$ \\
    5  & &  1 & $x \in \textrm{\textbf{M}}$ & $x \in M$ \\
    5  & &  1 & $\exists b \in B\; ,$ & $\exists b \in B,$ \\
    6  &  9 & & $( S, \cdot )${ }, { }{ }$( S', \ast )$
           & $( S, \cdot )$, $( S', \ast )$ \\
    6  & 11 & & $\forall x, y \in S$ & $\forall x, y \in S,$ \\
    6  & 14 & & $\forall x, y \in M$ & $\forall x, y \in M,$ \\
    6  & & 14 & $x \cdot y = 1$ & $x \cdot y = 1_{ M }$ \\
    6  & & 13 & Podgrupa & Grupa \\
    7  & & 12 & $\forall x, y, z \in S$ & $\forall x, y, z \in S,$ \\
    7  & & 10 & $\forall x, y, z \in S$ & $\forall x, y, z \in S,$ \\
    7  & &  8 & $\forall x, y, z \in S$ & $\forall x, y, z \in S,$ \\
    7  & &  4 & $S / \rho$ & $S_{ / \rho }$ \\
    7  & &  3 & $M / \rho$ & $M_{ / \rho }$ \\
    8  & & 10 & $n \geq 0\;\;,$ & $n \geq 0,$ \\
    8  & &  7 & $n \!\! = \!\! 0$ & $n = 0$ \\
    9  & &  9 & $\to \textrm{\textbf{M}}$ & $\to M$ \\
    % & & & & \\
    % & & & & \\
    % & & & & \\
    % & & & & \\
    % & & & & \\
    % & & & & \\
    \hline
  \end{tabular}

\end{center}


\noindent
\StrWd{6}{13} \\
\Jest  jest podgrupą $M$. \\
\Powin jest grupą zawartą, jako zbiór, w~$M$. Grupa ta dziedziczy po $M$
działanie i~element neutralny. \\
\StrWd{6}{1} \\
\Jest  \textit{dla} $x \in S^{ 1 }$ $\rho_{ a }( x )= a x$. \\
\Powin $\rho_{ a }( x ) = a x$ \textit{dla} $x \in S^{ 1 }$. \\
\textbf{Str. 8, pierwszy rysunek.} \\
\Jest  $S /_{ Ker_{ h } }$ \\
\Powin $S_{ / Ker_{ h } }$ \\
\textbf{Str. 8, drugi rysunek.} \\
\Jest  $A^{ * } /_{ Ker_{ h } }$ \\
\Powin $M_{ / Ker_{ h } }$ \\
\StrWg{9}{1} \\
\Jest  wolną półgrupę \\
\Powin wolną półgrupę językową \\
\StrWg{9}{2} \\
\Jest  wolny monoid \\
\Powin wolny monoid językowy \\
\StrWg{9}{2} \\
\Jest  wolna półgrupa \\
\Powin wolna półgrupa językowa \\
\StrWg{9}{3} \\
\Jest  nazywamy \\
\Powin językowego nazywamy \\
\StrWg{9}{4} \\
\Jest  nazywamy \\
\Powin językowego nazywamy \\
\StrWg{9}{6} \\
\Jest  nazywamy \\
\Powin językowego nazywamy \\
\StrWg{9}{6} \\
\Jest  monoid wolny \\
\Powin językowy monoid wolny \\
\StrWg{9}{6} \\
\Jest  półgrupa wolna \\
\Powin językowa półgrupa wolna \\
\StrWg{9}{12} \\
\Jest  wolnych monoidów \\
\Powin wolnych monoidów językowych \\
\StrWg{9}{12} \\
\Jest  wolnych półgrupach \\
\Powin wolnych półgrupach językowych \\
\StrWg{9}{14} \\
\Jest  monoidzie \\
\Powin monoidzie językowym \\
\StrWd{9}{6} \\
\Jest  \textit{monoidu} \\
\Powin \textit{monoidu językowego} \\
\StrWg{9}{25} \\
\Jest  \textit{homomorfizm} \\
\Powin \textit{homomorfizm z~wolnego monoidu językowego w~dowolny monoid} \\
\StrWd{9}{6} \\
\Jest  \textit{monoidu} $A^{ * }$ \\
\Powin \textit{monoidu językowego} $A^{ * }$ \\



\vspace{\spaceTwo}
% ############################










% #####################################################################
% #####################################################################
% Bibliografia
\bibliographystyle{plalpha}

\bibliography{MathComScienceBooks}{}





% ############################

% Koniec dokumentu
\end{document}
