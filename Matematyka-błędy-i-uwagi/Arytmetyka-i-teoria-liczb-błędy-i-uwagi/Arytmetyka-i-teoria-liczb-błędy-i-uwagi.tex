% ---------------------------------------------------------------------
% Podstawowe ustawienia i pakiety
% ---------------------------------------------------------------------
\RequirePackage[l2tabu, orthodox]{nag} % Wykrywa przestarzałe i niewłaściwe
% sposoby używania LaTeXa. Więcej jest w l2tabu English version.
\documentclass[a4paper,11pt]{article}
% {rozmiar papieru, rozmiar fontu}[klasa dokumentu]
\usepackage[MeX]{polski} % Polonizacja LaTeXa, bez niej będzie pracował
% w języku angielskim.
\usepackage[utf8]{inputenc} % Włączenie kodowania UTF-8, co daje dostęp
% do polskich znaków.
\usepackage{lmodern} % Wprowadza fonty Latin Modern.
\usepackage[T1]{fontenc} % Potrzebne do używania fontów Latin Modern.



% ------------------------------
% Podstawowe pakiety (niezwiązane z ustawieniami języka)
% ------------------------------
\usepackage{microtype} % Twierdzi, że poprawi rozmiar odstępów w tekście.
\usepackage{graphicx} % Wprowadza bardzo potrzebne komendy do wstawiania
% grafiki.
\usepackage{verbatim} % Poprawia otoczenie VERBATIME.
\usepackage{textcomp} % Dodaje takie symbole jak stopnie Celsiusa,
% wprowadzane bezpośrednio w tekście.
\usepackage{vmargin} % Pozwala na prostą kontrolę rozmiaru marginesów,
% za pomocą komend poniżej. Rozmiar odstępów jest mierzony w calach.
% ------------------------------
% MARGINS
% ------------------------------
\setmarginsrb
{ 0.7in}  % left margin
{ 0.6in}  % top margin
{ 0.7in}  % right margin
{ 0.8in}  % bottom margin
{  20pt}  % head height
{0.25in}  % head sep
{   9pt}  % foot height
{ 0.3in}  % foot sep



% ------------------------------
% Często przydatne pakiety
% ------------------------------
\usepackage{csquotes} % Pozwala w prosty sposób wstawiać cytaty do tekstu.
\usepackage{xcolor} % Pozwala używać kolorowych czcionek (zapewne dużo
% więcej, ale ja nie potrafię nic o tym powiedzieć).



% ------------------------------
% Pakiety do tekstów z nauk przyrodniczych
% ------------------------------
\let\lll\undefined % Amsmath gryzie się z językiem pakietami do języka
% polskiego, bo oba definiują komendę \lll. Aby rozwiązać ten problem
% oddefiniowuję tę komendę, ale może tym samym pozbywam się dużego Ł.
\usepackage[intlimits]{amsmath} % Podstawowe wsparcie od American
% Mathematical Society (w skrócie AMS)
\usepackage{amsfonts, amssymb, amscd, amsthm} % Dalsze wsparcie od AMS
% \usepackage{siunitx} % Dla prostszego pisania jednostek fizycznych
\usepackage{upgreek} % Ładniejsze greckie litery
% Przykładowa składnia: pi = \uppi
\usepackage{slashed} % Pozwala w prosty sposób pisać slash Feynmana.
\usepackage{calrsfs} % Zmienia czcionkę kaligraficzną w \mathcal
% na ładniejszą. Może w innych miejscach robi to samo, ale o tym nic
% nie wiem.



% ##########
% Tworzenie otoczeń "Twierdzenie", "Definicja", "Lemat", etc.
\newtheorem{theorem}{Twierdzenie}  % Komenda wprowadzająca otoczenie
% „theorem” do pisania twierdzeń matematycznych
\newtheorem{definition}{Definicja}  % Analogicznie jak powyżej
\newtheorem{corollary}{Wniosek}



% ---------------------------------------
% Pakiety napisane przez użytkownika.
% Mają być w tym samym katalogu to ten plik .tex
% ---------------------------------------
\usepackage{latexgeneralcommands}
\usepackage{mathcommands}
% \usepackage{calculuscommands}
% \usepackage{SchwartzBooksCommands}  % Pakiet napisany m.in. dla tego pliku.



% ---------------------------------------------------------------------
% Dodatkowe ustawienia dla języka polskiego
% ---------------------------------------------------------------------
\renewcommand{\thesection}{\arabic{section}.}
% Kropki po numerach rozdziału (polski zwyczaj topograficzny)
\renewcommand{\thesubsection}{\thesection\arabic{subsection}}
% Brak kropki po numerach podrozdziału



% ------------------------------
% Ustawienia różnych parametrów tekstu
% ------------------------------
\renewcommand{\baselinestretch}{1.1}

\renewcommand{\arraystretch}{1.2} % Ustawienie szerokości odstępów między
% wierszami w tabelach.



% ------------------------------
% Pakiet „hyperref”
% Polecano by umieszczać go na końcu preambuły.
% ------------------------------
\usepackage{hyperref} % Pozwala tworzyć hiperlinki i zamienia odwołania
% do bibliografii na hiperlinki.










% ---------------------------------------------------------------------
% Tytuł i autor tekstu
\title{Arytmetyka i~teoria liczb \\
  {\Large Błędy i~uwagi}}

\author{Kamil Ziemian}


% \date{}
% ---------------------------------------------------------------------










% ####################################################################
\begin{document}
% ####################################################################





% ######################################
\maketitle % Tytuł całego tekstu
% ######################################





% ############################
\Work{ % Autor i tytuł dzieła
  Jacek Gancarzewicz \\
  \textit{Arytmetyka}, \cite{GancarzewiczArytmetyka2000}}

\vspace{0em}


% ##################
\CenterBoldFont{Uwagi}

\vspace{0em}


\noindent
W~książce do oznaczenia najważniejszych zbiorów używa~się fontów
pogrubionych: $\mathbf{N}$, $\mathbf{Z}$, $\mathbf{Q}$, etc. W~tych
notatkach będziemy używali standardowych fontów „blackboard bold”: $\Nbb$,
$\Zbb$, $\Qbb$,~etc. Wyjątkiem będzie sytuacja, gdy w~danym miejscu
książki został użyty zły symbol, wtedy będziemy dążyć, by wygląd wzoru który
będziemy poprawiać, był maksymalnie podobny do oryginału w~książce.

\vspace{\spaceFour}





% ##################
\CenterBoldFont{Uwagi do konkretnych stron}

\vspace{0em}


\Str{7} Na tej stronie można zauważyć, że~w~tej książeczce odstępy po kropce
może być bardzo duży. Nie wiem czy jest to wynik użytych ustawień \LaTeX a,
takich jak włączenie lub nie opcji \texttt{\textbackslash frenchspacing},
czy jakiś innych czynników. Niezależnie od tego, tego typu konwekcji
i~błędów edytorskich nie będziemy zapisywać w~tych notatkach, czyniąc
wyjątek dla tych, które są jawnie błędne lub wyglądają bardzo źle.

\vspace{\spaceFour}





\Str{10} Na tej stronie możemy zaobserwować, że~w~książeczce użyta jest
konwencja wedle której cudzysłów zapisujemy jako ”cytowany tekst”, wydaje
mi~się, że~jest to francuska konwencja zapisu. W~naszej ocenie lepiej
wygląda stosowana w~Polsce konwencja zapisu: „cytowany tekst”. Z~tego
powodu, gdy będziemy poprawiać jakiś fragment tekstu, w~którym znajduje~się
cudzysłów, w~poprawionej wersji będziemy stosować tę drugą konwencję.

\vspace{\spaceFour}





\StrWd{20}{8} Duży odstęp po tą linią to chyba błąd edytorski.

\vspace{\spaceFour}





\Str{20} Podane tu wyjaśnienie definicji rekurencyjnej jest niełatwe
w~zrozumieniu i~może nawet nie być poprawne. Warto by ją poprawić.

\vspace{\spaceFour}





\Str{23} Na tej stronie podany jest wzór
\begin{equation}
  \label{eq:Gancarzewicz-Arytmetyka-01}
  a_{ n } =
  \frac{ 1 }{ \sqrt{ 5 } }
  \Big\{ \Big( \frac{ 1 + \sqrt{ 5 } }{ 2 } \Big)^{ n }
  - \Big( \frac{ 1 - \sqrt{5} }{ 2 } \Big)^{ n } \Big\}
\end{equation}
nie wygląda najlepiej, ze względu na to jak wysokość nawiasów ma~się
do wysokości zawartych w~nich wyrażeń. Według mnie, znaczniej lepiej
wyglądały on, gdyby został zapisany w~następujący sposób
\begin{equation}
  \label{eq:Gancarzewicz-Arytmetyka-01}
  a_{ n } =
  \frac{ 1 }{ \sqrt{ 5 } }
  \left\{ \left( \frac{ 1 + \sqrt{ 5 } }{ 2 } \right)^{ n }
  - \left( \frac{ 1 - \sqrt{5} }{ 2 } \right)^{ n } \right\},
\end{equation}
ewentualnie
\begin{equation}
  \label{eq:Gancarzewicz-Arytmetyka-02}
  a_{ n } =
  \frac{ 1 }{ \sqrt{ 5 } }
  \left\{ \left( \tfrac{ 1 + \sqrt{ 5 } }{ 2 } \right)^{ n }
  - \left( \tfrac{ 1 - \sqrt{5} }{ 2 } \right)^{ n } \right\}.
\end{equation}

W~dalszych ciągu tych notatek, poza wyjątkowymi sytuacjami nie będziemy
odnosić~się do tego typu problemów typograficznych.

\vspace{\spaceFour}





\Str{23} W~przeprowadzanym tutaj wyprowadzaniu wzoru (2.6) pojawia się
następujący problem. Czy dopuszczamy by zmienne $a$ i~$b$ w~nim występujące
mogły przyjmować wartość $0$? Jeśli odpowiedź na to pytanie jest twierdząca,
to stajemy przed problemem tego, iż~we wzorze (2.6) pojawiają~się
człony $a^{ 0 }$ i~$b^{ 0 }$. Ponieważ jeśli $b = 0$ to poszukiwana zależność
redukuje się do wyrażenia $a^{ n }$ (analogicznie dla $a = 0$), proponuję by
przyjąć, iż rozważamy tylko sytuację $a \neq 0 \neq b$.

\vspace{\spaceFour}





% \vspace{\spaceFour}





% \vspace{\spaceFour}










% ##################
\newpage

\CenterBoldFont{Błędy}

\vspace{\spaceFive}


\begin{center}

  \begin{tabular}{|c|c|c|c|c|}
    \hline
    Strona & \multicolumn{2}{c|}{Wiersz} & Jest
                              & Powinno być \\ \cline{2-3}
    & Od góry & Od dołu & & \\
    \hline
    10  &  2 & & ” $p$ & „$p$ \\
    10  &  3 & & ” $p$ & „$p$ \\
    10  &  4 & & ” $p$ & „$p$ \\
    10  &  6 & & ” $p$ & „$p$ \\
    10  & 13 & & fałszywe, Są & fałszywe. Są \\
    12  & &  5 & $n_{ n }$ & $n_{ 0 }$ \\
    14  &  5 & & $x \in X \, , y \in Y$ & $x \in X, \, y \in Y$ \\
    15  &  4 & & $f( x' ) \, ,$ & $f( x' ),$ \\
    16  &  2 & & można) & można \\
    20  &  3 & & $a_{ n }$m, & $a_{ n }$, \\
    21  & 17 & & $25, \ldots \ldots$ & $25, \ldots$ \\
    21  & &  7 & $640, \ldots \ldots$ & $640, \ldots$ \\
    24  & 13 & & $b^{ i }$ & $b^{ i }{}'$ \\
    26  & &  6 & $T( 4 ), \ldots \ldots$ & $T( 4 ), \ldots$ \\
    27  &  1 & & $T( n )$ & $T( 2 )$ \\
    53  &  6 & & $\cdots 10^{ k } c_{ k }$ & $\cdots + 10^{ k } \, c_{ k }$ \\
    53  & &  3 & $, 10^{ i } c_{ i }$ & $10^{ i } \, c_{ i }$ \\
    56  & &  8 & $\displaystyle \sum_{ = 0 }^{ s }$
      & $\displaystyle \sum_{ i = 0 }^{ s }$ \\
    % & & & & \\
    % & & & & \\
    % & & & & \\
    \hline
  \end{tabular}

\end{center}

\vspace{\spaceTwo}

\StrWd{24}{4--5} \\
\Jest  jeszcze raz zmienić wskaźnik sumacyjny, tym razem $i$ zamieniamy
na~$j$, \\
\Powin zmienić nazwę wskaźnika sumacyjnego z~$i$ na~$j$ \\
% ############################










% #####################################################################
% #####################################################################
% Bibliografia

\bibliographystyle{plalpha}

\bibliography{MathComScienceBooks}{}





% ############################

% Koniec dokumentu
\end{document}
