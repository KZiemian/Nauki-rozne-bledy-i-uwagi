% ---------------------------------------------------------------------
% Basic configuration and packages
% ---------------------------------------------------------------------
% Package for discovering wrong and outdated usage of LaTeX.
% More information to be found in l2tabu English version.
\RequirePackage[l2tabu, orthodox]{nag}
% Class of LaTeX document: {size of paper, size of font}[document class]
\documentclass[a4paper,11pt]{article}



% ---------------------------------------
% Packages not tied to particular normal language
% ---------------------------------------
% This package should improved spaces in the text.
\usepackage{microtype}
% Add few important symbols, like text Celcius degree
\usepackage{textcomp}



% ---------------------------------------
% Polonization of LaTeX document
% ---------------------------------------
% Basic polonization of the text
\usepackage[MeX]{polski}
% Switching on UTF-8 encoding
\usepackage[utf8]{inputenc}
% Adding font Latin Modern
\usepackage{lmodern}
% Package is need for fonts Latin Modern
\usepackage[T1]{fontenc}



% ---------------------------------------
% Setting margins
% ---------------------------------------
\usepackage[a4paper, total={14cm, 25cm}]{geometry}
% Package for easy settings of margins. Unit of measurement is inch.
% \usepackage{vmargin}
% \setmarginsrb
% { 0.7in}  % left margin
% { 0.6in}  % top margin
% { 0.7in}  % right margin
% { 0.8in}  % bottom margin
% {  20pt}  % head height
% {0.25in}  % head sep
% {   9pt}  % foot height
% { 0.3in}  % foot sep



% ---------------------------------------
% Setting vertical spaces in the text
% ---------------------------------------
% Setting space between lines
\renewcommand{\baselinestretch}{1.1}

% Setting space between lines in tables
\renewcommand{\arraystretch}{1.4}



% ---------------------------------------
% Packages for scientific papers
% ---------------------------------------
% Switching off \lll symbol, that I guess is representing letter ``Ł''.
% It collide with `amsmath' package's command with the same name
\let\lll\undefined
% Basic package from American Mathematical Society (AMS)
\usepackage[intlimits]{amsmath}
% Equations are numbered separately in every section.
\numberwithin{equation}{section}

% Other very useful packages from AMS
\usepackage{amsfonts, amssymb, amscd, amsthm}
% Better looking calligraphy fonts
\usepackage{calrsfs}

% Better looking greek letters
% Example of use: pi -> \uppi
\usepackage{upgreek}
% Improving look of lambda letter
\let\oldlambda\Lambda
\renewcommand{\lambda}{\uplambda}





% ---------------------------------------
% Defining new environments (?)
% ---------------------------------------
% Defining enviroment ``Wniosek''
\newtheorem{corollary}{Wniosek}
\newtheorem{definition}{Definicja}
\newtheorem{theorem}{Twierdzenie}





% ------------------------------
% Private packages
% You need to put them in the same directory as .tex file
% ------------------------------
% Contains various command useful for working with a text
\usepackage{latexgeneralcommands}
% Contains definitions useful for working with mathematical text
\usepackage{mathcommands}

% Package for use in text about ordinary differential equations
\usepackage{ODEcommands}





% ------------------------------
% Package ``hyperref''
% They advised to put it on the end of preambule
% ------------------------------
% It allows you to use hyperlinks in the text
\usepackage{hyperref}










% ------------------------------------------------------------------------------------
% Defining title and author of the text
\title{N.M. Matwiejew \\
  \textit{Metody całkowania równana różniczkowych zwyczajnych},
  \cite{MatwiejewMetodyCalkowaniaRownanRozniczkowychZwyczajnych1986}}

\author{Kamil Ziemian}


% \date{}
% ------------------------------------------------------------------------------------










% ####################################################################
\begin{document}
% ####################################################################





% ######################################
% Title of the text
\maketitle
% ######################################





% ######################################
\section{Uwagi ogólne}

\label{sec:Uwagi-ogolne}
% ######################################



Książka ta podąża za~umową, że~jeśli argumenty danej funkcji są
odpowiednio oczywiste, to można je opuścić. Przykładowo na stronie~7
równanie (1) ma postać
\begin{equation}
  \label{eq:Uwagi-ogolne-01}
  F\left( x, y, y', y'', \ldots, y^{ ( n ) } \right) = 0,
\end{equation}
choć bardziej precyzyjny byłoby zapisanie go jako
\begin{equation}
  \label{eq:Uwagi-ogolne-02}
  F\left( x, y( x ), y'( x ), y''( x ), \ldots, y^{ ( n ) }( x ) \right) = 0,
\end{equation}
W~tych notatkach, by uniknąć nieporozumień, we~wszystkich miejscach gdzie
jest mowa o~wartościach jakie funkcja przyjmuje dla pewnego zestawu
zmiennych, będziemy~się starać podawać jej argumenty w~sposób jawny.

Jeśli więc będziemy rozważali w~sposób ogólny funkcje
$y : \Rbb \to \Rbb$ to będziemy oznaczali ją symbolem~$y$. Natomiast
w~przypadkach takich jak równanie
\begin{equation}
  \label{eq:Uwagi-ogolne-03}
  y( x )^{ 2 } = x,
\end{equation}
będziemy~się starali zawsze wymienić wszystkie argumenty w~sposób jawny.

\VerSpaceFour





\noindent
We~wcześniejszym fragmencie tych notatek napisaliśmy, że~w~książce tej
często pomijane są argumenty funkcji, jeśli uznano, iż~są one odpowiednio
oczywiste. Ta praktyka ma tą wadę, że~pewne wzory nie są tak proste
w~zrozumieniu, jak mogłyby być. Dlatego tego poniżej zapiszemy niektóre
wzory spośród tych obecnych w~książce, podając jawnie argumenty wszystkich
występujących w~nich funkcji, w~nadziei, że~uczyni to jej lekturę odrobinę
łatwiejszą.

Co należy zaznaczyć, nie będziemy stronić od~powtórzeń. Jeśli więc dany
wzór pojawia~się w~kilku miejscach książki, może pojawić się kilka razy
w~liście poniżej. Może~się również zdarzyć, że~wzór obecny już w~innej
części tych notatek w~swojej pełnej formie, pojawi~się na tej liście.
Uważamy, że~zebranie tych wzorów w~jednym miejscu, umożliwi Czytelnikowi
szybkie sprawdzenie, co dokładnie dany wzór symbolizuje. W~naszej ocenie
usprawiedliwia to ewentualne powtórzenia.

\VerSpaceTwo


\noindent
Str. 7, wiersz 3.
\begin{equation}
  \label{eq:Uwagi-ogolne-04}
  F\left( x, y( x ), y'( x ), y''( x ), \ldots, y^{ ( n ) }( x ) \right) = 0
\end{equation}
Str.~7, wiersz 15.
\begin{equation}
  \label{eq:Uwagi-ogolne-05}
  \begin{split}
    \Phi \negHorSpaceEleven \left(
    &x_{ 1 }, x_{ 2 }, \ldots, x_{ n },
      u( x_{ 1 }, x_{ 2 }, \ldots, x_{ n }),
      \tfrac{ \partial u( x_{ 1 }, \, x_{ 2 }, \, \ldots, \, x_{ n } ) }{ \partial x_{ 1 } },
      \tfrac{ \partial u( x_{ 1 }, \, x_{ 2 }, \, \ldots, \, x_{ n } ) }{ \partial x_{ 2 } },
      \ldots, \right. \\
    &\left.
      \tfrac{ \partial u( x_{ 1 }, \, x_{ 2 }, \, \ldots, \, x_{ n } ) }{ \partial x_{ n } }
      \right) = 0
  \end{split}
\end{equation}
Str.~27, wiersz~16 od dołu, wzór~(24).
\begin{equation}
  \label{eq:Uwagi-ogolne-06}
  y( x ) = \varphi( x, C )
\end{equation}
Str.~29, wiersz~9, wzór~(28).
\begin{equation}
  \label{eq:Uwagi-ogolne-07}
  y( x ) = \frac{ y_{ 0 } }{ x_{ 0 } } x
\end{equation}










% ######################################
\section{Czym jest równanie różniczkowe zwyczajne i~jego
  rozwiązanie?}

\label{sec:Czym-jest-rownanie-ETC}
% ######################################



Dla osób obeznajmionych z~analizą matematyczną i~mającą pewne doświadczenie
w~badaniu równań różniczkowych zwyczajnych, samo pojęcie takiego równania
jest na poziomie nieformalnym, intuicyjnym bardzo proste do uchwycenia.
Jednak podanie sformalizowanej definicji, która była wystarczająco pojemna,
by objąć wszystkie przypadki równań, które chcemy objąć tym pojęciem, jest
zadaniem dość trudnym. Te i~inne tego typu uwagi często można odnieść bez
wielkich zmian do pojęcia rozwiązania równania różniczkowego zwyczajnego.
W~dalszym ciągu tekstu zwykle nie będziemy tego zaznaczyć, że~to co
napisaliśmy o~równaniu stosuje~się również do jego rozwiązania, by uniknąć
niepotrzebnych powtórzeń.
% Tu zakończyłem edycję stylu.

Ponadto jak każdy taki ważny problem w~matematyce, podaną definicję
równania różniczkowego zwyczajnego trzeba będzie niewątpliwie nie jeden raz
zmodyfikować i~rozszerzyć, tak aby objąć jej nowymi wariantami kolejne ważne
równania, które w~toku badań będą~się
pojawiać. Przy czym nie chodzi nam np.~o~równania różniczkowe zwyczajne
które
zawierają człon stochastyczny, gdyż problematyka takich równań to w~pewnym
sensie osobna dziedzina badań. Chodzi nam o~sytuacje takie jak omówiona
poniżej analiza równania $y'( x ) = \sqrt{ y( x ) }$, $y( x ) \geq 0$, które
wymaga od nas rozszerzenia podanej wcześniej definicji.

Mimo wszystko, użyteczne jest podanie bardzo formalnej definicji równania
różniczkowego zwyczajnego, która jest na tyle pojemna, że~pozwala objąć
ogromną ilość interesujących równań, z~drugiej strony cechuje~się ona
koncepcyjną lub użytkową prostotą. Przez użytkową prostotę rozumiem to,
że~jeśli dane równanie podpada pod tą definicję, to będą je
cechować bardzo dobre własności z~punktu widzenia teorii lub praktyki
rozwiązywania takich równań lub też nie będzie posiadać pewnych
nieprzyjemnych właściwości, które mogą wystąpić w~ogólniejszym przypadku.
Z~tych samych względów podamy definicję rozwiązania tego równania.

Tekst ten cały czas ewoluuje i~Czytelnik powinien mieć tego świadomość.
W~szczególności, wraz z~postępami w~lekturze książki i~napotykaniem nowych
równań które teoria musi uwzględnić, odpowiednie komentarze będą dodawane,
co~może prowadzić również do~zmiany podanych definicji.

Zdefiniowanie równania różniczkowego zwyczajnego wymaga od nas wprowadzenia
pewnej ilość wstępnych definicji i~oznaczeń. Niech teraz~$O$ będzie
otwartym podzbiorem $\Rbb^{ n + 2 }$, elementy tego zbioru będziemy oznaczać
przez $( x_{ 0 }, x_{ 1 }, \ldots, x_{ n - 1 }, x_{ n }, x_{ n + 1 } )$
lub $x_{ 1 }, x_{ 2 }, \ldots, x_{ n + 1 }, x_{ n + 2 }$. Niech dana będzie funkcja
$F : O \to R$. Przyjmujemy, że~\textbf{zależy ona w~sposób istotny
  od~zmiennej $x_{ n }$} (w~książce używa~się terminu „jawny”), przez co
rozumiemy następującą własność. Istnieją takie liczby
$x_{ 0 }, x_{ 1 }, \ldots, x_{ n }, x_{ n + 1 }, x_{ n + 1 }'$, że~zachodzi
\begin{equation}
  \label{eq:Czym-jest-rownanie-ETC-03}
  F( x_{ 0 }, x_{ 1 }, \ldots, x_{ n }, x_{ n + 1 } ) \neq
  F( x_{ 0 }, x_{ 1 }, \ldots, x_{ n }, x_{ n + 1 }' ).
\end{equation}
Niech teraz $A_{ 0 }$ będzie rzutem zbioru $O$ na oś $x_{ 0 }$, czyli zbiorem
takim że~jeśli $x \in A_{ 0 }$ to istnieją takie liczby
$x_{ 1 }, x_{ 2 }, \ldots, x_{ n }, x_{ n + 1 }$,
że~$( x, x_{ 1 }, x_{ 2 }, \ldots, x_{ n }, x_{ n + 1 } ) \in O$. Zbiór ten
będziemy również oznaczać przez $\proj_{ 0 } O$.

\textbf{Równaniem różniczkowym zwyczajnym rzędu~$n$ w~postaci ogólnej}
nazywamy wyrażenie
\begin{equation}
  \label{eq:Czym-jest-rownanie-ETC-04}
  F\left( x, y( x ), y'( x ), y''( x ), \ldots, y^{ ( n ) }( x ) \right) = 0.
\end{equation}
\textbf{Rozwiązaniem równania \eqref{eq:Czym-jest-rownanie-ETC-04}}
nazywamy funkcję $y_{ 1 } : A_{ 1 } \to \Rbb$, gdzie $A_{ 1 } \subset A_{ 0 }$, która
posiada pochodne do rzędu $n$ włącznie w~każdym punkcie swojej dziedziny
i~dla której zachodzi
\begin{equation}
  \label{eq:Czym-jest-rownanie-ETC-05}
  F\left( x, y_{ 1 }( x ), y_{ 1 }'( x ), \ldots, y_{ 1 }^{ ( n ) }( x ) \right) =
  0, \qquad
  \forall x \in A_{ 1 }.
\end{equation}
Równanie różniczkowe nazywamy \textbf{równaniem różniczkowym
  zwyczajnym rzędu~$n$ w~postaci normalnej} (zob. str.~13 omawianej
książki), jeśli przyjmuje ono postać
\begin{equation}
  \label{eq:Czym-jest-rownanie-ETC-06}
  y^{ ( n ) }( x ) = f\big( x, y( x ), y'( x ), \ldots, y^{ ( n - 1 ) }( x ) \big),
\end{equation}
gdzie $f : O \to \Rbb$ jest funkcją określoną na zbiorze otwartym
$O \subset \Rbb^{ n }$.

Tutaj należy poczynić kilka uwag terminologicznych. Stosunkowo rzadko
będziemy mówić „Dane jest funkcja $F( x_{ 0 }, x_{ 1 }, \ldots, x_{ n + 1 } )$,
która określa równanie różniczkowe
$F\big( x, y( x ), y'( x ), \ldots, y^{ ( n ) }( x ) \big) = 0$.”. Zwykle
stosować będziemy skrót myślowy „Dana jest funkcja
$F\big( x, y( x ), y'( x ), \ldots, y^{ ( n ) }( x ) \big)$ określając równanie
różniczkowe.” lub po prostu „Dane jest równanie różniczkowe
$F\big( x, y( x ), y'( x ), \ldots, y^{ ( n ) }( x ) \big) = 0$.”. Pozwala
to uczynić tekst bardziej zwięzły, a~odczytanie odpowiedniej funkcji~$F$
z~postaci równania różniczkowego zwykle nie nastręcza żadnych problemów.
Analogicznie będziemy raczej mówić „Funkcja
$F\big( x, y( x ), y'( x ), \ldots, y^{ ( n ) }( x ) \big)$ zależy w~sposób
istotny od $y^{ ( n ) }( x )$.”, niż~„Funkcja
$F( x_{ 0 }, x_{ 1 }, x_{ 2 }, \ldots, x_{ n + 1 } )$ zależy w~sposób istotny
od~$x_{ n + 1 }$.”.

Równanie różniczkowe zwyczajne w~postaci normalnej jest
szczególnym przypadkiem równania różniczkowego zwyczajnego w~postaci
ogólnej. Jest ono bowiem równoważne równaniu
\eqref{eq:Czym-jest-rownanie-ETC-04}, gdzie za~funkcję~$F$
przyjmujemy
\begin{equation}
  \label{eq:Czym-jest-rownanie-ETC-06}
  F\big( x, y( x ), y'( x ), \ldots, y^{ ( n ) }( x ) \big) =
  f\big( x, y( x ), y'( x ), \ldots, y^{ ( n - 1 ) }( x ) \big) -
  y^{ ( n - 1 ) }( x ),
\end{equation}
przy czym zgodnie z~podanymi wcześniej zasadami przyjmujemy, że~dziedzina
powyższej funkcji jest równa~$\Rbb \times O$. Funkcja $F$ tej postaci w~oczywisty
sposób zależy istotnie od~$y^{ ( n ) }( x )$. Ponieważ równanie różniczkowe
zwyczajne w~postaci normalnej jest szczególnym przypadkiem równania
różniczkowego zwyczajnego zdefiniowanego wyżej, zasady które w~prowadzimy
w~dalszym ciągu notatek do~pracy z~funkcją $F$, będziemy też stosować, przy
ewentualnych naturalnych zmianach, do funkcji~$f$. Chyba że~jawnie
stwierdzono inaczej.

Przy badaniu równań różniczkowych często spotykamy~się z~sytuacją, że~dane
jest wyrażenie postaci
$F\big( x, y( x ), y'( x ), \ldots, y^{ ( n ) }( x ) \big)$,
ale~jego dziedzina nie została w~sposób jawny podana. Jeśli chodzi
o~wartości jakie może przyjmować $F$, to ze~względu na to co pisze na
stronie siódmej o~funkcjach $y$ jakich będziemy szukać rozwiązując dane
równanie różniczkowe, przyjmujemy, że~zbiór wartości jakie tam funkcja może
przyjmować jest podzbiorem $\Rbb$. Ponieważ prawie nigdy wybór jako
przeciwdziedziny $F$ całego zbioru $\Rbb$ nic nie zmienia, zawsze będziemy
przyjmować iż jest ona równa $\Rbb$.

Jeśli chodzi o~ustalenie dziedziny~$F$ to domyślnie będziemy za
nią wybierać największy zbiór otwarty~$O$, na którym to wyrażenie jest
określone. Jeśli zajdzie będziemy potrzeba rozpatrywać dziedzinę, która
nie jest zbiorem otwartym, wówczas naszym pierwszym wyborem będzie
maksymalny zbiór na którym możemy określić~$F$, niezależnie czy jest on
otwarty, czy nie. Zasady te uzasadniają, czemu
za~dziedzinę funkcji $F$ z~równania \eqref{eq:Czym-jest-rownanie-ETC-06}
przyjęliśmy $\Rbb \times O$. Do zagadnienia ustalania dziedziny funkcji
definiujących równanie różniczkowe będziemy musieli powrócić w~przyszłości.

Jeżeli zbiór ten nie jest otwarty, to otrzymane równanie
nie będzie spełniało podanej wcześniej definicji równania różniczkowego
zwyczajnego. Jest to jeden z~problemów, którego wciąż nie umiemy w~sposób
satysfakcjonujący rozwiązać i~który będziemy szerzej dyskutować w~rozdziale
\eqref{subsec:Definicja-rownania-rozniczkowego-ETC}. Niezależnie od tych
problemów, po ustaleniu dziedziny i~zbioru wartości, wyrażenie~$F$
staje~się dobrze określoną funkcją.

Jeżeli mamy już daną funkcję $F$ z~określoną dziedziną i~przeciwdziedziną,
która definiuje równanie różniczkowe
\begin{equation}
  \label{eq:Czym-jest-rownanie-ETC-07}
  F\left( x, y( x ), y'( x ), y''( x ), \ldots, y^{ ( n ) }( x ) \right) = 0,
\end{equation}
potrzebujemy~się zająć problemem, jakie wstępne warunki ma~spełniać funkcja
$y( x )$, aby można było w~ogóle ją rozważać jako potencjalne rozwiązanie
tego równania? Zacznijmy od problemu dziedziny i~przeciwdziedziny.
Na~stronie siódmej Matwieje podaje, że~jeśli nie jest powiedziane inaczej,
to zakładamy, że~zarówno dziedzina jak i~przeciwdziedzina takiej funkcji są
podzbiorami liczb rzeczywistych. Jeśli chodzi o~przeciwdziedzinę to z~tych
samych powodów co podane wcześniej, jeśli nie jest powiedziane inaczej,
będziemy przyjmować, że jest ona równa $\Rbb$. Kwestia dziedziny jest
jednak bardziej problematyczna.

Niech $A \subset \Rbb$ i~rozważmy funkcję $y : A \to \Rbb$. Podstawowym wymogiem
aby funkcja $y$ było rozwiązaniem równania różniczkowego zwyczajnego jest
to, żeby posiadała pochodną w~każdym punkcie swojej dziedziny. Zbiór $A$
musi więc być dobrany w~taki sposób, by pojęcie pochodnej w~punkcie $x \in A$
miało sens. Wbrew pozorom podanie klasy zbiorów o~tej własności nie jest
takie proste. Zadajmy sobie bowiem pytanie, czy funkcja $f : \Qbb \to \Qbb$
dana wzorem
\begin{equation}
  \label{eq:Czym-jest-rownanie-ETC-08}
  f( x ) = x^{ 2 },
\end{equation}
jest różniczkowalna?

Pomimo problemów z~określeniem klasy tych zbiorów, to w~większości
przypadków domyślnie będziemy szukali funkcji określonej na zbiorze
$A = ( a, b )$. Dopuszczamy przy tym sytuację, że $a$ i~$b$ mogą
przyjmować wartość $\pm \infty$. Jak pokazuje teoria różniczkowania
funkcji jednej zmiennej rzeczywistej, jest to po wieloma względami
najbardziej naturalny wybór, co motywuje szukanie w~pierwszym rzędzie
rozwiązań określonych na takiej dziedzinie. Jak zauważono na~stronie~14
omawianej książki, nie nastręcza również problemu określenie funkcji~$y$
na~odcinkach domkniętych z~jednej, bądź obu stron. W~przypadku odcinka
$[ a, b )$ przez pochodną w~punkcie~$a$ rozumiemy pochodną prawostronną
funkcji $y$ i~analogicznie postępuję w~przypadku odcinka $( a, b ]$. Przy
czym jeśli dany jest odcinek prawostronnie domknięty $( a, b ]$ to musi
zachodzić $-\infty < b < +\infty$, natomiast $a$ może przyjąć wartość $-\infty$.
Analogicznie zasady stosują~się do pozostałych przypadków odcinków
jednostronnie, bądź obustronnie domkniętych.

W~następnej kolejności będziemy szukali rozwiązań określonych na zbiorach
które są $A = \bigcup_{ i = 1 }^{ n } ( a_{ i }, b_{ i } )$. Z~tych samych
powodów co poprzednio dopuszczamy też sytuacje, gdy jeden bądź więcej
przedziałów w~sumie mnogościowej jest jedno- bądź obustronnie domkniętych.
Pierwszy raz równanie, które wymaga szukania rozwiązania o~takiej
dziedzinie, napotykamy na stronie 15. Podane tam równanie
$y'( x ) = y( x )^{ 2 }$
posiada bowiem rozwiązanie postaci
\begin{equation}
  \label{eq:Czym-jest-rownanie-ETC-09}
  y( x ) = \frac{ 1 }{ 1 - x }, \quad
  x \in ( -\infty, 1 ) \cup ( 1, +\infty ).
\end{equation}
Niezależnie czy szukamy rozwiązania mającego za dziedzinę przedział otwarty
$( a, b )$ (domknięty z~jednej, bądź obu stron) czy sumę odpowiednich
przedziałów, to będziemy szukali największego, w~sensie zawierania~się
zbiorów, na~którym możemy określić rozwiązanie danego równania
różniczkowego. Chyba, że~jawnie stwierdzono inaczej.

Pierwszy raz z~konkretnym równaniem różniczkowym stykamy~się na stronie~8.
Równanie o~którym mowa jest postaci
\begin{equation}
  \label{eq:Czym-jest-rownanie-ETC-10}
  y'( x ) - 2 x = 0.
\end{equation}
Z~równania tego w~prosty sposób odczytujemy funkcję $F$:
\begin{equation}
  \label{eq:Czym-jest-rownanie-ETC-11}
  F( x_{ 0 }, x_{ 1 }, x_{ 2 } ) = x_{ 2 } - 2 x_{ 0 }.
\end{equation}
Stosując podane wcześniej zasady przyjmujemy $\Dcal( F ) = \Rbb^{ 3 }$.
Jednocześnie równanie \eqref{eq:Matwiejew-Metody-calkowania-ETC-14} zapisane
jako
\begin{equation}
  \label{eq:Czym-jest-rownanie-ETC-12}
  y'( x ) = 2 x,
\end{equation}
jest przykładem równania pierwszego rzędu w~postaci normalnej. W~prosty
sposób odczytujemy z~niego, że~$f( x_{ 0 }, x_{ 1 } ) = 2 x_{ 0 }$. Stosują
te same reguły co poprzednio, przyjmujemy jego dziedzinę jako równą
$\Rbb^{ 2 }$.

Tutaj należy zwrócić uwagę na pewną matematyczną subtelność. Jeśli
rozważane równanie różniczkowe jest pierwotnie wyrażone w~formie ogólnej
\begin{equation}
  \label{eq:Czym-jest-rownanie-ETC-13}
  F\big( x, y( x ), y'( x ), \ldots, y^{ ( n ) }( x ) \big) = 0,
\end{equation}
i~odwikłamy je do postaci normalnej
\begin{equation}
  \label{eq:Czym-jest-rownanie-ETC-14}
  y^{ ( n ) }( x ) = f\big( x, y( x ), y'( x ), \ldots, y^{ ( n - 1 ) }( x ) \big),
\end{equation}
to ostatni wzór może nie obejmować wszystkich przypadków, dla~których
równanie ma~sens \eqref{eq:Czym-jest-rownanie-ETC-13}. Ponieważ
dziedzina~$F$ jest podzbiorem $\Rbb^{ n + 2 }$, a~$f$ podzbiorem
$\Rbb^{ n + 1 }$, wysłowienie co dokładnie oznacza, że~równanie
\eqref{eq:Czym-jest-rownanie-ETC-13}
jest ogólniejsze od równania \eqref{eq:Czym-jest-rownanie-ETC-14}
może być dość zawiłe, zamiast tego odwołamy~się więc do~przykładu.
Rozpatrzmy równanie w~postaci ogólnej
\begin{equation}
  \label{eq:Czym-jest-rownanie-ETC-15}
  y( x ) \frac{ d y( x ) }{ dx } - x = 0.
\end{equation}
Odczytujemy z~niego funkcję
$F( x_{ 0 }, x_{ 1 }, x_{ 2 } ) = x_{ 2 } x_{ 1 } - x_{ 0 }$, której dziedzina
jest równa $\Rbb^{ 3 }$. Po przekształceniu do postaci normalnej dostajemy
równanie
\begin{equation}
  \label{eq:Czym-jest-rownanie-ETC-16}
  \frac{ d y( x ) }{ d x } = \frac{ x }{ y( x ) }.
\end{equation}
Funkcja $f( x_{ 0 }, x_{ 1 } ) = x_{ 0 } / x_{ 1 }$ ma dziedzinę
$\Rbb \times ( \Rbb \setminus \{ 0 \} )$. Podstawiając $y( x ) = 0$
do~\eqref{eq:Matwiejew-Metody-calkowania-ETC-19} od razu sprawdzamy,
że~funkcja ta nie jest rozwiązaniem tego równania, podczas gdy podstawienie
tej funkcji do \eqref{eq:Matwiejew-Metody-calkowania-ETC-20} prowadzi do
wyrażenie pozbawionego sensu\footnote{Matwiejew podaje, że~w~takiej
  sytuacji należy rozpatrzyć równanie na funkcję $x( y )$, nie ma to
  jednak znaczenie w~obecnym kontekście.}.

Dalszy ciąg niniejszym rozważań na temat tego czym jest równanie
różniczkowe zwyczajne i~jego rozwiązanie, będzie dotyczył przypadków, gdy
podane wcześniej definicji okazują~się zbyt wąskie w~stosunku do wymagań
teorii. Nie będziemy przy tym unikać stosowania pojęć wprowadzony dalej
w~książce, gdyż mają one duże znaczenie dla właściwego zrozumienia czym są
równania różniczkowe. Ta część tekstu cały czas jest rozwijana, więc nie
należy traktować jego obecnej formy, jako zakończonej i~ostatecznej.










% ############################
\subsection{Definicja równania różniczkowego i~jej ograniczenia}

\label{subsec:Definicja-rownania-rozniczkowego-ETC}
% ############################



W~tym ustępie zajmiemy~się problemami, gdy chcemy zastosować definicję
równania różniczkowego zwyczajnego podaną w~poprzedniej części tekstu.
Problem ten prawie zawsze pojawia~się, gdy dziedzina funkcji $F$ nie jest
zbiorem otwartym, co jest jednym z~wymogów podanej tam definicji. Jeżeli
równanie różniczkowe jest w~postaci normalnej
\begin{equation}
  \label{eq:Definicja-rownania-rozniczkowego-ETC-01}
  y^{ ( n ) }( x ) = f\big( x, y( x ), y'( x ), \ldots, y^{ ( n - 1 ) }( x ) \big),
\end{equation}
to oznaczmy dziedzinę funkcji $f$ przez $O$. Jak było wyjaśnione wcześniej
wówczas dziedzin funkcji $F$ jest równa $\Rbb \times O$. Nie umiem teraz tego
udowodnić, ale wydaje mi~się, że~zbiór ten jest otwarty wtedy i~tylko
wtedy, gdy $O$ jest otwarty. Do tego problemu będzie trzeba powrócić, na
razie jednak zajmiemy~się analizą równań w~postaci normalnej, dla których
będzie jasne, że~zbiór $\Rbb \times O$ nie jest otwarty.

Pierwszy raz napotykamy równanie które stawia przed nami ten problem na
stronie 31 tej książki.
\begin{equation}
  \label{eq:Definicja-rownania-rozniczkowego-ETC-02}
  y'( x ) = 2 \sqrt{ y( x ) }.
\end{equation}
Ponieważ interesują nas tylko równania o~wartościach rzeczywistych, więc
przyjmujemy, że~$y( x ) \geq 0$. Funkcja~$f$ definiująca to równanie jest
łatwa do odczytania:
\begin{equation}
  \label{eq:Definicja-rownania-rozniczkowego-ETC-03}
  f( x_{ 0 }, x_{ 1 } ) = \sqrt{ x_{ 1 } }.
\end{equation}
Zgodnie z~przyjętymi wcześniej zasadami, dziedzina $f$ jest równa
$\Rbb \times ( 0, +\infty )$ jeśli szukamy dziedziny który jest zbiorem otwartym, lub
$\Rbb \times [ 0, +\infty )$ jeśli interesuje nas największy zbiór na którym
jesteśmy w~stanie określić. Te dwa przypadki można opisać w~prosty sposób
nierównościami $y > 0$ i~$y \geq 0$, gdzie $y$ oznacza zmienną na
płaszczyźnie, nie funkcję.

Rozpatrzmy przypadek $y > 0$. Wówczas w~każdym punkcie dziedziny spełnione
są założenia o~istnieniu i~jednoznaczności rozwiązania równania
różniczkowego, więc dwie krzywe całkowe nigdy~się nie przecinają.
Rozwiązania są postaci
\begin{equation}
  \label{eq:Definicja-rownania-rozniczkowego-ETC-04}
  y( x ) = \phi( x, C ) = ( x + C )^{ 2 }, \qquad
  x > -C.
\end{equation}

Przejdźmy teraz do przypadku $y \geq 0$. Dla $y = 0$ nie są spełnione
założenia o~istnieniu i~jednoznaczności rozwiązania równania różniczkowego.
Funkcja $y( x ) = 0$, $x \in \Rbb$ jest rozwiązaniem tego równania, tak jak
funkcje dane przez \eqref{eq:Definicja-rownania-rozniczkowego-ETC-04}.
Poprzednim jednak razem nie byliśmy w~stanie określić rozwiązania danego
tym wzorem na zbiorze większym, niż $( -C, +\infty )$, tym razem jest to możliwe.
Stosując więc zasadę, że~szukamy rozwiązania o~największej możliwej
dziedzinie danej jako suma przedziałów, musimy przedłużyć wszystkie
rozwiązania dane przez \eqref{eq:Definicja-rownania-rozniczkowego-ETC-04}
do funkcji danych jako
\begin{equation}
  \label{eq:Definicja-rownania-rozniczkowego-ETC-05}
  y( x ) =
  \begin{cases}
    0, & x \leq -C, \\
    ( x + C )^{ 2 }, & x > -C.
  \end{cases}
\end{equation}
Skutkiem tego, obecnie krzywe całkowe dwóch dowolnych rozwiązań tego
równania~się przecinają. Widzimy więc jak radykalnie mogą~się zmienić
własności rozwiązań danego równania poprzez prostą zmianę dziedziny na
której jest ono określone.














% ######################################
\section{Uwagi do~konkretnych stron}

\label{sec:Uwagi-do-konkrentych-stron}
% ######################################



\VerSpaceFour





\noindent
\Str{10} Przy okazji wyprowadzania równania różniczkowe dla rodziny
wszystkich okręgów na płaszczyźnie $xy$ napotykamy po raz pierwszy na
pewien problem, który powróci do nas w przyszłości. Naszym punktem wyjścia
jest równania
\begin{equation}
  \label{eq:Matwiejew-Metody-calkowania-ETC-22}
  ( x - a )^{ 2 } + ( y - b )^{ 2 } = R^{ 2 }.
\end{equation}
Różniczkując je dwa razy otrzymujemy dwa następujące równania.

\negVerSpaceFour


\begin{subequations}

  \begin{align}
    \label{eq:Matwiejew-Metody-calkowania-ETC-23-A}
    1 + \big( y'( x ) \big)^{ 2 } + \big( y( x ) - b \big) y''( x )
    &= 0, \\
    \label{eq:Matwiejew-Metody-calkowania-ETC-23-B}
    3 y'( x ) y''( x ) + \big( y( x ) - b \big) y'''( x )
    &= 0.
  \end{align}

\end{subequations}


\noindent
Aby usunąć z~równania \eqref{eq:Matwiejew-Metody-calkowania-ETC-23-B}
parametr $b$ przyjmujemy, że~$y''( x ) \neq 0$, więc możemy przepisać równanie
\eqref{eq:Matwiejew-Metody-calkowania-ETC-23-A} jako
\begin{equation}
  \label{eq:Matwiejew-Metody-calkowania-ETC-24}
  y( x ) - b =
  -\frac{ 1 + \big( y'( x ) \big)^{ 2 } }{ y''( x ) }.
\end{equation}
Podstawiając tą zależność do \eqref{eq:Matwiejew-Metody-calkowania-ETC-23-B}
dostajemy
\begin{equation}
  \label{eq:Matwiejew-Metody-calkowania-ETC-25}
  3 y'( x ) y''( x ) -
  \frac{ 1 + \big( y'( x ) \big)^{ 2 } }{ y''( x ) } y'''( x ) = 0.
\end{equation}
Po pomnożeniu obustronnie przez $y''( x )$ dostajemy
\begin{equation}
  \label{eq:Matwiejew-Metody-calkowania-ETC-26}
  3 y'( x ) \big( y''( x ) \big)^{ 2 } -
  \Big( 1 + \big( y'( x ) \big)^{ 2 } \Big) y'''( x ) = 0.
\end{equation}
Choć równanie wyprowadziliśmy przy założeniu, że~$y''( x ) \neq 0$, to końcowa
jego postać jest dobrze określona również, gdy funkcja ta przyjmuje wartość
zero. Powstaje więc pytania, jaki jest zakres obowiązywania tego równania?

Równanie \eqref{eq:Matwiejew-Metody-calkowania-ETC-23-A} możemy przekształci
przy założeniu $y( x ) - b \neq 0$ do postaci
\begin{equation}
  \label{eq:Matwiejew-Metody-calkowania-ETC-27}
  y''( x ) =
  -\frac{ 1 + \big( y'( x ) \big)^{ 2 } }{ y( x ) - b },
\end{equation}
widzimy więc, że~$y''( x ) \neq 0$ jeśli tylko $y( x ) \neq b$, w~przeciwnym
wypadku jest ona nieokreślona. Możemy stąd wyciągnąć wniosek,
że~na mocy swojego wyprowadzenia równanie
\eqref{eq:Matwiejew-Metody-calkowania-ETC-26} obowiązuje dla wszystkich
wartości $x$ dla których $y( x ) \neq b$. To~zaś prowadzi do~kilku problemów
wartych przedyskutowania.

Po~pierwsze, równanie \eqref{eq:Matwiejew-Metody-calkowania-ETC-26} nie
zawiera stałej~$b$, gdyż naszym celem było wyprowadzenie równania
pozbawionego stałych obecnych w~równaniu
\eqref{eq:Matwiejew-Metody-calkowania-ETC-22}. Biorąc więc za~punkt wyjścia
\eqref{eq:Matwiejew-Metody-calkowania-ETC-25} nie jesteśmy w~stanie
sformułować na jego podstawie warunku $y( x ) \neq b$. Możliwe jednak, że~jeśli
rozwiążemy to równanie, to jego rozwiązania będą nieokreślone jeśli $y( x )$
przyjmie pewną wartość $D$, specyficzną dla danego rozwiązania. Już teraz
możemy stwierdzić, iż~jest to prawdą, przynajmniej dla pewnej klasy
rozwiązań tego równania, które w~postaci uwikłanej (zob. str.~15 omawianej
książki\footnote{Na tej stronie omawiane jest rozwiązanie w~postaci
  uwikłanej dla równania pierwszego rzędu, a~równanie
  \eqref{eq:Matwiejew-Metody-calkowania-ETC-25}. Pomimo tego mamy nadzieję,
  że~to w~jakim sensie równanie
  \eqref{eq:Matwiejew-Metody-calkowania-ETC-22} przedstawia rozwiązanie
  równania \eqref{eq:Matwiejew-Metody-calkowania-ETC-26} jest wystarczająco
  zrozumiałe i~nie wymaga dalszych komentarzy.}) jest dane przez naszą
wyjściową zależność
\eqref{eq:Matwiejew-Metody-calkowania-ETC-22}. Nad analizą tej sytuacji
zatrzymamy~się przez chwilę.

Dla $x \in ( a - R, a + R )$ jesteśmy w~stanie je odwikłać na funkcję
$y( x )$ otrzymując dwie funkcje
\begin{equation}
  \label{eq:Matwiejew-Metody-calkowania-ETC-28}
  y_{ \pm }( x ) = \pm\sqrt{ R^{ 2 } - ( x - a )^{ 2 } } + b.
\end{equation}
Nie jest możliwe odwikłanie tego równania na funkcję $y( x )$ w~sposób
matematycznie spójny na~przedziale $( a + R - \delta, a + R ]$, dla pewnego
$\delta > 0$, przy czym analogiczna sytuacja zachodzi dla przedziału
$[ a - R, a - R + \delta )$. W~tematykę tego, czy w~takim razie nie powinniśmy
również rozważać równania różniczkowego na funkcję $x( y )$ (zob. str.~13
tej książki) nie będziemy~się zagłębiać. Niemniej widzimy, że
\begin{equation}
  \label{eq:Matwiejew-Metody-calkowania-ETC-29}
  \lim_{ x \nearrow a + R } y_{ \pm }( x ) = b,
\end{equation}
więc funkcje te możemy przedłużyć do funkcji ciągłych na przedziale
$( a - R, a + R ]$ (analogicznie możemy postąpić dla punktu $a - R$).
Tak przedłużone funkcje będziemy oznaczać przez $\yTilde_{ \pm }( x )$.
Przyjmują one wartość $b$ tylko dla $x = a + R$, ale~żadna z~nich nie jest
różniczkowalna w~punkcie $x = a + R$, bo~zachowuje~się w~ich otoczeniu jak
$\sqrt{ x }$ w~otoczeniu punktu $x = 0$. Powyższe rozważania pomogą nam
naświetlić drugi ważny problem.

Zauważmy, że~choć równanie \eqref{eq:Matwiejew-Metody-calkowania-ETC-26}
obowiązuje dla wszystkich $x$ taki, że~$y( x ) \neq b$, to nawet jeśli ten
warunek jest nam z~góry znany, to zanim nie rozwiążemy tego równania, nie
wiem dla jakich wartości zmiennej $x$ rozwiązanie przyjmie wartość $b$.
Musimy więc najpierw rozwiązać w~odpowiedni sposób to równanie, a~następnie
usunąć z~dziedziny rozwiązania te punkty, dla których $y( x ) = b$. Jeśli
takie punkty rzeczywiście istnieją to powstaje następujące pytanie. Sokor
otrzymaliśmy rozwiązanie również dla tych wartości $x$ dla
których nie jest spełniony warunek $y( x ) \neq b$, to czy nie oznacza to,
że~choć warunek ten był potrzebny do wyprowadzenia
\eqref{eq:Matwiejew-Metody-calkowania-ETC-26}, to jest on ostatecznie
restrykcyjny i~należy przyjąć bardziej ogólną koncepcję rozwiązania?

W~omawianym przykładzie, funkcje $y_{ \pm }( x )$ były zdefiniowane tylko na
przedziale $( a - R, a + R )$, jednak dało~się je przedłużyć do funkcji
$\yTilde_{ \pm }( x )$ ciągłych na $( a - R, a + R ]$, które są jednak
nieróżniczkowalne w~punkcie $a + R$. Czy w~takim wypadku należy uważać
funkcje $\yTilde_{ \pm }( x )$ za rozwiązanie odpowiedniego równania
różniczkowego? Zwróćmy uwagę, że~rozwiązanie w~postaci uwikłanej
\begin{equation}
  \label{eq:Matwiejew-Metody-calkowania-ETC-30}
  ( x - a )^{ 2 } + ( y - b )^{ 2 } = R^{ 2 },
\end{equation}
przedstawia okrąg, czyli krzywą gładką w~każdym swoim punkcie, leżącą
w~płaszczyźnie $xy$, która nie wykazuje żadnej osobliwości w~punkcie
o~współrzędnych $( a + R, b )$. Z~tego
punktu widzenia funkcja $\yTilde_{ + }( x )$ jest obcięciami rozwiązania
danego funkcją uwikłaną do zbioru $( a - R, a + R ]$, które spełnia warunek
$y \geq 0$, więc należy je uważa za pełnoprawne rozwiązanie równania
\eqref{eq:Matwiejew-Metody-calkowania-ETC-26}. Niewątpliwie, fakt,
że~rozwiązania w~postaci uwikłanej pozwalają nam na objęciem pojęciem
rozwiązania równania różniczkowego takich krzywych jak okręgi dane równaniem
\eqref{eq:Matwiejew-Metody-calkowania-ETC-30}, jest jednym z~powodów, dla
których to pojęcie zostało wprowadzone.

Przedstawiona powyżej analiza pozwala uzmysłowić, jak skomplikowanym
zadaniem może być analiza zakresu obowiązywania danego równania
różniczkowego oraz jego rozwiązań. Do omawiania tych zagadnień zapewne
powrócimy jeszcze wielokrotnie w~tych notatkach.

\VerSpaceFour





\noindent
\Str{15} Na tej stronie po raz pierwszy pojawia~się oznaczenie $\Phi_{ x }$
i~$\Phi_{ y }$, gdzie $\Phi$ jest funkcją dwóch zmiennych:
$\Phi( x, y )$. Jest to skrótowy zapis pochodnej cząstkowej
\begin{equation}
  \label{eq:Czym-jest-rownanie-ETC-01}
  \Phi_{ x } \equiv
  \frac{ \partial \Phi( x, y ) }{ \partial x }.
\end{equation}
W~przypadku zmiennej~$y$ dokładny sens tego oznaczenia jest często trochę
inny. Mianowicie należy je rozumieć jako
\begin{equation}
  \label{eq:Czym-jest-rownanie-ETC-01}
  \Phi_{ y } \equiv
  \frac{ \partial \Phi( x, y ) }{ \partial y }\bigg|_{ y = y( x ) }.
\end{equation}
Choć te oznaczenia zwykle są łatwe zrozumienia, to~zdarza~się,
że~mogą powodować niejasności.

\VerSpaceFour





\noindent
\Str{15} Równanie (10) na tej stronie zostało bardzo elegancko wyprowadzone,
przy założeniu, że~funkcja uwikłana zależna od zmiennej $x$ dana związkiem
$\Phi( x, y ) = 0$, spełnia równanie różniczkowe
$y'( x ) = f\big( x, y( x ) \big)$. Nie poruszono jednak kwestii tego, czemu
mamy uważać tą funkcję uwikłaną za rozwiązanie badanego równania?
W~szczególności, jeśli $y( x )$ jest rozwiązanie równania $\Phi( x, y ) = 0$,
to czy spełnia ona równanie $y'( x ) = f\big( x, y( x ) \big)$?
Przy standardowych założeniach o~funkcji $\Phi( x, y )$ odpowiedź na to
ostatnie pytanie jest twierdząca i~dzięki temu możemy uznać funkcję daną
równaniem uwikłanym za rozwiązanie badanego równania.

Dowód tego faktu jest następujący. Załóżmy, że~funkcja $\Phi( x, y )$ spełnia
założenia o~ciągłości i~istnieniu pochodnych, które są wymagane
w~standardowej wersji twierdzenia o~funkcji uwikłanej. Przyjmijmy dodatkowo,
że~$\Phi_{ y }'\big( x, y \big) \neq 0$ w~interesującym nas zakresie wartości
zmiennej~$x$, który oznaczymy~$A$. Tym samym możemy odwikłać funkcję
$y( x )$ dla wszystkich wartości $x \in A$. Przepiszmy teraz równanie (10)
w~bardziej jawnej formie jako
\begin{equation}
  \label{eq:Matwiejew-Metody-calkowania-ETC-31}
  \Phi_{ x }'\big( x, y( x ) \big) +
  \Phi_{ y }'\big( x, y( x ) \big) f\big( x, y( x ) \big) = 0.
\end{equation}
Równanie to przekształcamy w~oczywisty sposób do postaci
\begin{equation}
  \label{eq:Matwiejew-Metody-calkowania-ETC-32}
  -\frac{ \Phi_{ x }'\big( x, y( x ) \big) }
  { \Phi_{ y }'\big( x, y( x ) \big) } =
  f\big( x, y( x ) \big).
\end{equation}
Jak dobrze wiadomo, lewa strona tej równości przedstawia pochodną funkcji
uwikłanej $y'( x )$.

Na koniec możemy zauważyć, że~zarówno wzór
\eqref{eq:Matwiejew-Metody-calkowania-ETC-31} jak i~dobrze znaną zależność
\begin{equation}
  \label{eq:Matwiejew-Metody-calkowania-ETC-33}
  y'( x ) =
  -\frac{ \Phi_{ x }'\big( x, y( x ) \big) }{ \Phi_{ y }'\big( x, y'( x ) \big) },
\end{equation}
można wyprowadzić w~ten sam sposób, poprzez zróżniczkowanie
i~przekształcenie zależności $\Phi\big( x, y( x ) \big) = 0$. Choć z~tego
punktu widzenia równoważność równania
\eqref{eq:Matwiejew-Metody-calkowania-ETC-26}
i~$y'( x ) = f\big( x, y( x ) \big)$ może~się wydawać oczywista, woleliśmy
przedyskutować to zagadnienie możliwie dokładnie.

\VerSpaceFour





\noindent
\Str{16} Należy przedyskutować dwa zagadnienia dotyczące rozwiązania
równania różniczkowego
\begin{equation}
  \label{eq:Matwiejew-Metody-calkowania-ETC-34}
  y'( x ) = f\big( x, y( x ) \big),
\end{equation}
w~postaci parametrycznej. Po pierwsze, czy jeśli $y( x )$ jest rozwiązaniem
tego równania, to czy można podać jego rozwiązanie w~postaci parametrycznej?
Po drugie, jak uzasadnić, że~funkcje $\varphi( t )$ i~$\psi( t )$ spełniające
równanie
\begin{equation}
  \label{eq:Matwiejew-Metody-calkowania-ETC-35}
  \frac{ \psi'( t ) }{ \varphi'( t ) } =
  f\big( \varphi( t ), \psi( t ) \big),
\end{equation}
przedstawiając rozwiązanie równania
\eqref{eq:Matwiejew-Metody-calkowania-ETC-34}?

Zanim odpowiemy na te pytania, zrobimy małą uwagę odnośnie oznaczeń.
W~całym tym punkcie, jeśli nie powiedziano inaczej, symbol prim oznacza
pochodną danej funkcji. W~szczególności we wzorze
\eqref{eq:Matwiejew-Metody-calkowania-ETC-35}, co przyjmowaliśmy domyślnie,
symbol $\psi'( t )$ oznacza pochodną funkcji $\psi( t )$ po~czasie.

Przejdźmy teraz do odpowiedzi na~pierwsze pytanie z postawionych wyżej
pytań. Równanie różniczkowe \eqref{eq:Matwiejew-Metody-calkowania-ETC-34}
jest określone na płaszczyźnie~$xy$, więc funkcje $\varphi( t )$ i~$\psi( t )$ możemy
rozumieć jako odwzorowanie $\Psi( t ) : ( t_{ 0 }, t_{ 1 } ) \to \Rbb^{ 2 }$, dane
przez
\begin{equation}
  \label{eq:Matwiejew-Metody-calkowania-ETC-36}
  \Psi( t ) = \big( \varphi( t ), \psi( t ) \big).
\end{equation}
Korzystając z~interpretacji równania
\eqref{eq:Matwiejew-Metody-calkowania-ETC-34} za pomocą wprowadzonego dalej
na tej stronie i~następnych pojęcia pola kierunków oraz krzywej całkowej,
można stwierdzić, że~funkcje $\varphi( t )$ i~$\psi( t )$ są rozwiązaniem
rozpatrywanego równania w~postaci parametrycznej, jeśli $\Psi( t )$ wyznacza
krzywą całkową. Jest to bardzo ładny atrakcyjny sposób rozumienia
rozwiązania parametrycznego, dla porządku podamy alternatywną interpretacje.

Mianowicie, jeśli $y( x )$ jest rozwiązaniem równania określonym dla
$x \in ( a, b )$, to parę funkcji $\varphi( t )$, $\psi( t )$ określonych
dla~$t \in I = ( t_{ 0 }, t_{ 1 } )$ nazywamy postacią parametryczną tego
równania jeśli zachodzi $\varphi( I ) \subset ( a, b )$ oraz
\begin{equation}
  \label{eq:Matwiejew-Metody-calkowania-ETC-37}
  \psi( t ) = y\big( \varphi( t ) \big).
\end{equation}

Wzór \eqref{eq:Matwiejew-Metody-calkowania-ETC-37}
pozwala nam stwierdzić, że~jeśli mamy dane rozwiązanie $y( x )$, określone
przy tych samych warunkach co powyżej, to zawsze możemy utworzyć odpowiednie
rozwiązanie parametryczne. Wystarczy przyjąć, że~$I = ( a, b )$,
$\varphi( t ) = t$ (zwykle zapisuje to jako $x = t$) i~zdefiniować $\psi( t )$ jako
$y\big( \varphi( t ) \big)$. Zauważmy, że~zachodzi wówczas
\begin{equation}
  \label{eq:Matwiejew-Metody-calkowania-ETC-38}
  \frac{ d \psi( t ) }{ d t } =
  \frac{ d y( x ) }{ d x }\bigg|_{ x = \varphi( t ) }
  \frac{ d \varphi( t ) }{ d t }.
\end{equation}
Oznaczając symbolem prim pochodną po czasie i~zakładając,
że~$\varphi'( t ) \neq 0$ otrzymujemy zależność
\begin{equation}
  \label{eq:Matwiejew-Metody-calkowania-ETC-39}
  \frac{ d y( x ) }{ d x }\bigg|_{ x = \varphi( t ) } =
  \frac{ \psi'( t ) }{ \varphi'( t ) }.
\end{equation}
Razem z~równaniami \eqref{eq:Matwiejew-Metody-calkowania-ETC-34}
i~\eqref{eq:Matwiejew-Metody-calkowania-ETC-37} prowadzi do równania
\begin{equation}
  \label{eq:Matwiejew-Metody-calkowania-ETC-40}
  \frac{ \psi'( t ) }{ \varphi'( t ) } =
  f\big( \varphi( t ), \psi( t ) \big).
\end{equation}
To kończy rozważanie na temat wyrażenia znanego rozwiązania w~postaci
parametrycznej.

Przejdźmy teraz do problemu, dlaczego możemy uważać rozwiązanie równania
\eqref{eq:Matwiejew-Metody-calkowania-ETC-35} za rozwiązania równania
\eqref{eq:Matwiejew-Metody-calkowania-ETC-34}. Przyjmijmy, że~funkcja
$\varphi( t )$ jest różniczkowalna w~sposób ciągły. Na podstawie równania
\eqref{eq:Matwiejew-Metody-calkowania-ETC-35} wiemy, że~musi zachodzić
$\varphi'( t )$. Na podstawie standardowych twierdzeń z~analizy matematycznej
(zob. przykładowo \cite{FichtenholzRachunekRozniczkowyETCVolI2005}) oraz
dobrze znanego wzoru na pochodną funkcji odwrotnej:
\begin{equation}
  \label{eq:Matwiejew-Metody-calkowania-ETC-41}
  \frac{ d f^{ -1 }( y ) }{ dy } =
  \frac{ d f( t ) }{ d t }\bigg|_{ t = f^{ -1 }( y ) },
\end{equation}
w~pewnych otoczeniu wybranego punktu $t$, oznaczmy je
$J = ( t - \delta_{ 1 }, t + \delta_{ 2 } )$, gdzie $\delta_{ 1 }, \delta_{ 2 } > 0$, istniej
funkcja odwrotna do $\varphi( t )$, oznaczać ją będziemy $\eta( x )$, która jest
różniczkowalna i~jej pochodna wynosi
\begin{equation}
  \label{eq:Matwiejew-Metody-calkowania-ETC-42}
  \frac{ d \eta( x ) }{ d x } =
  \frac{ 1 }{ \varphi'\big( \eta( x ) \big) }.
\end{equation}
Dla większej przejrzystość obliczeń użyliśmy tutaj symbolu
$\varphi'\big( \eta( x ) \big)$ na oznaczenia wartości pochodnej $\varphi'( t )$ obliczonej
w~punkcie $\eta( x )$.

Rozpatrzmy teraz funkcję $\psi\big( \eta( x ) \big)$. Jej pochodna po zmiennej $x$
jest równa
\begin{equation}
  \label{eq:Matwiejew-Metody-calkowania-ETC-43}
  \frac{ d \psi\big( \eta( x ) \big) }{ d x } =
  \frac{ d \psi( t ) }{ d t }\bigg|_{ t = \eta( x ) }
  \frac{ d \eta( x ) }{ d x } =
  \frac{ d \psi( t ) }{ d t }\bigg|_{ t = \eta( x ) }
  \frac{ 1 }{ \varphi'\big( \eta( x ) \big)}.
\end{equation}
Jeżeli teraz obliczymy wartość obu stron równania
\eqref{eq:Matwiejew-Metody-calkowania-ETC-35} dla wartości $t = \eta( x )$
otrzymamy
\begin{equation}
  \label{eq:Matwiejew-Metody-calkowania-ETC-44}
  \frac{ \psi'\big( \eta( x ) \big) }{ \varphi'\big( \eta( x ) \big) } =
  f\Big( x, \psi\big( \eta( x ) \big) \Big).
\end{equation}
Korzystając z~\eqref{eq:Matwiejew-Metody-calkowania-ETC-38} możemy przepisać
powyższe równanie jako
\begin{equation}
  \label{eq:Matwiejew-Metody-calkowania-ETC-45}
  \frac{ d \psi\big( \eta( x ) \big) }{ d x } =
  f\Big( x, \psi\big( \eta( x ) \big) \big).
\end{equation}
Widzimy więc, że~funkcja $y( x ) = \psi\big( \eta( x ) \big)$ jest rozwiązaniem
równania \eqref{eq:Matwiejew-Metody-calkowania-ETC-44} określonym na zbiorze
$\varphi( J )$. To wyjaśnia czemu funkcje $\varphi( t )$ i~$\psi( t )$ możemy uznać za
rozwiązanie rozważanego równania różniczkowego.

\VerSpaceFour





\noindent
\Str{16--17} Warto dodać kilka słów komentarza do wprowadzonego tutaj
pojęcia pola kierunków oraz na relacji jaki pojęcie to ma do używanego
przez Władimira Arnolda pojęcia pola wektorowego definiujące równanie
różniczkowe \cite{ArnoldRownaniaRozniczkoweZwyczajne1975}. Samo pole
kierunków oraz inne pojęcia dla niego wprowadzone, takie jak izokliny
(por. str.~17 i~18), w~tych notatkach będziemy rozpatrywali, chyba
że~powiedziano inaczej, tylko w~kontekście najprostszego przypadku, dla
którego są wprowadzone w~tym fragmencie książki. Ten najprostszy przypadek
przedstawia~się następująco. Szukana funkcja $y$ jest odwzorowaniem
$y : A \to \Rbb$, gdzie $A \subset \Rbb$, a~równanie które ma spełniać podane
jest w~standardowej postaci normalnej:
\begin{equation}
  \label{eq:Matwiejew-Metody-calkowania-ETC-46}
  y'( x ) = f\big( x, y( x ) \big).
\end{equation}
W~związku z~tym pole kierunków jest zdefiniowane na płaszczyźnie $xy$,
gdzie $x$ odpowiada zmiennej niezależnej, a~zmienna $y$ wartościom
przyjmowanym przez funkcję $y( x )$.

Zaczniemy od pewnych uwag terminologicznych. O~równaniu
\eqref{eq:Matwiejew-Metody-calkowania-ETC-46} i~różnych jego
uogólnieniach będziemy mówili, że~wyznacza ono
\textbf{ruch zadanego punktu}. Przez ruch punktu będziemy rozumieć
rozwiązanie tego równania $y$. Natomiast przez \textbf{tor ruchu} będziemy
rozumieć zbiór $y( A )$.

Przejdźmy teraz do pola kierunków. W~jego opisie pojawiają~się długości
odcinków i~kąty, stąd wniosek, że~określamy je na płaszczyźnie, na~której
jest zdefiniowany iloczyn skalarny. Inaczej mówiąc, mamy do czynienia
z~dwuwymiarową
przestrzenią euklidesową $\Ebold^{ 2 }$, z~wybranym w~niej układem
współrzędnych. Moglibyśmy więc podać formalizacje pola kierunków za~pomocą
wiązki stycznej $T \Ebold^{ 2 }$, lecz nie widzimy takiej potrzeby.

W~dalszym ciągu potrzebne nam będą następujące oznaczenia. Przez $v$
będziemy oznaczali pole kierunków określone na $\Ebold^{ 2 }$, przez
$\vecv$ będziemy określali pole wektorowe w~takim sensie jak rozumie je
w~cytowanej wyżej pozycji Arnold, po więcej informacji odsyłamy do tej
książki.

Podanie kierunku jest równoważne z~podaniem pewnej prostej, aby~zdefiniować
kierunek wystarczy każdemu punktowi $P$ przyporządkować dowolny wektor
kierunkowy prostej wyznaczający dany kierunek. Tym samym pole wektorowe
$\vecv$ określone w~obszarze $G \subset \Ebold^{ 2 }$ odpowiednie pole kierunków,
jednak w~ogólności pole kierunków nie wyznacza w~sposób jednoznaczny pola
wektorowego. Mając dane pole kierunków możemy zawsze podać nieskończoną
ilość pól wektorowych, takich że wektor $\vecv( x, y )$ ma kierunek równy
$v( x, y )$. Wystarczy bowiem znaleźć jedno takie pole $\vecv( x, y )$,
następnie dla $\lambda \neq 0$ określić pole
\begin{equation}
  \label{eq:Matwiejew-Metody-calkowania-ETC-47}
  \vecv_{ \, \lambda }( x, y ) = \lambda \HorSpaceThree \vecv( x, y ).
\end{equation}
Sposób budowy pola wektorowego odpowiadającego danemu polu kierunków podamy
dalej.

Powstaje pytanie, czemu w~tym kontekście wystarczające jest rozważanie pola
kierunków, podczas gdy Arnold musi rozważać pola wektorowe? Wynika to
z~tego, że~tutaj analizujemy tylko najprostszy przypadek równania, w~którym
szukana jest funkcja $y : A \to \Rbb$, $A \subset \Rbb$, podczas, gdy formalizm
Arnolda obejmuje sytuacje, w~których szukana funkcja jest typu
$y : A \to \Rbb^{ n }$. Przeanalizujemy teraz dokładniej oba te przypadki.

Dla uproszczenia notacji, będziemy od teraz przyjmować, że~we wszystkich
przypadkach mamy $A = ( x_{ 0 }, x_{ 1 } )$. Nie ma to wielkiego znaczenia
dla ogólności przedstawionych rozważań. Niech szukana funkcja będzie postaci
$y : ( x_{ 0 }, x_{ 1 } ) \to \Rbb$. Wówczas krzywa całkowa
zawarta w~$G \subset \Ebold^{ 2 }$ i~styczna do pola kierunków
$v( x, y ) = f( x, y )$ wyznacza nie tylko kształt toru ruchu punktu,
którego równanie ruchu ma postać
\eqref{eq:Matwiejew-Metody-calkowania-ETC-46},
ale też jaki dokładnie sposób w~jaki ten tor pokonuje. Tor ruchu dla
badanego przypadku jest zawsze bardzo prosty, tworzy on bowiem odcinek
\begin{equation}
  \label{eq:Matwiejew-Metody-calkowania-ETC-48}
  y\big( ( x_{ 0 }, x_{ 1 } ) \big) = ( a, b )
\end{equation}
 (ewentualnie jeden z~odcinków
$( a, b ]$, $[ a, b )$, $[ a, b ]$), dla pewnych $a, b \in \Rbb$,
ewentualnie $a = -\infty$, $b = +\infty$. Wynika to z~tego, że~funkcja $y_{ 1 }( x )$
jest ciągła, a~obraz odcinka (zbioru spójnego) przez odwzorowanie ciągłe
jest też odcinek.

Jeśli zaś chodzi o~dokładny przebieg ruchu, to~z~krzywych całkowych pola
kierunków możemy odczytać czy jest to na przykład ruch oscylujący typu
$y_{ 1 }( x ) = A \sin( x )$, $A \in \Rbb$, czy ruch liniowy
$y_{ 1 }( x ) = A x$, etc. Wynika to z~tego, że~krzywa fazowa odpowiada
definicji funkcji jako zbioru par uporządkowanych odpowiedniego iloczynu
kartezjańskiego.

Wskażmy jeszcze na jeden ważny fakt. W~tej konkretnej sytuacji do pełnego
scharakteryzowania krzywych całkowych wystarczy znajomość kierunku do
którego mają one być styczna. Wiemy bowiem, że~krzywa całkowa zawsze
„biegnie od lewej do prawej”, bo to jest kierunek w~których zmienna $x$.
Załóżmy teraz, że~przez punkt $( x, y )$ ma przechodzić gładka krzywa
całkowa, dana przez funkcję $y$. Równaniem tej krzywej jest postaci
\begin{equation}
  \label{eq:Matwiejew-Metody-calkowania-ETC-49}
  K : ( x - \delta, x + \delta ) \to \Rbb^{ 2 }, \quad
  K( x ) = \big( x, y( x ) \big),
\end{equation}
gdzie $\delta > 0$. Wobec tego wektor $\vecv$ styczny do~tej krzywej całkowej
w~tym punkcie jest dany przez
\begin{equation}
  \label{eq:Matwiejew-Metody-calkowania-ETC-50}
  \vecv \HorSpaceThree ( x, y ) =
  \big[ 1, y'( x ) \big].
\end{equation}
Z~analizy matematycznej wiemy, że~jeśli $\alpha$ jest kątem pod jakim styczna do
krzywej w~punkcie $\big( x, y( x ) \big)$, to zachodzi
\begin{equation}
  \label{eq:Matwiejew-Metody-calkowania-ETC-51}
  \tan( \alpha ) = y'( x ).
\end{equation}
Ponieważ zaś $\tan( \alpha )$ jest wyznaczony przez pole kierunków, pozwala nam
to jednoznacznie wyznaczyć wektor styczny do krzywej całkowej. Możemy teraz
skorzystać z~\eqref{eq:Matwiejew-Metody-calkowania-ETC-46} i~otrzymać wzór
na pole wektorowe odpowiadające temu równaniu różniczkowemu:
\begin{equation}
  \label{eq:Matwiejew-Metody-calkowania-ETC-52}
  \vecv \HorSpaceThree ( x, y ) =
  \big[ 1, f( x, y ) \big].
\end{equation}
Możemy teraz odwrócić rozumowanie i~stwierdzić, że~rozwiązanie równania
różniczkowego oznacza znalezienie krzywych całkowych na płaszczyźnie $xy$,
takich że w~każdym ich punkcie ich wektor styczny jest dany przez pole
wektorowe zdefiniowane wzorem \eqref{eq:Matwiejew-Metody-calkowania-ETC-52}.

Przyjrzyjmy~się jeszcze raz temu rozumowaniu. Ponieważ w~omawianym
przypadku wiemy, że~pierwsza pola wektorowego zadanego przez równanie
\eqref{eq:Matwiejew-Metody-calkowania-ETC-47} musi~się zawsze równać jeden:
\begin{equation}
  \label{eq:Matwiejew-Metody-calkowania-ETC-53}
  \vecv_{ x }( x, y ) = 1,
\end{equation}
więc biorąc to pod uwagę, jesteśmy w~stanie znając wartości pola kierunków
w~punkcie $( x, y )$, jesteśmy w~stanie odtworzyć wektor $\vecv( x, y )$.
Warunek ten wyklucza również możliwość zastosowania transformacji takich jak
\eqref{eq:Matwiejew-Metody-calkowania-ETC-47}.

Rozpatrzmy teraz sytuację analizowaną przez Arnolda. Niech więc będzie dane
pole wektorowe $\vecv : G \to \Rbb^{ 2 }$, $G \subset \Rbb^{ 2 }$. Szukać będziemy
funkcji postaci $\vecy : ( t_{ 0 }, t_{ 1 } ) \to G$, spełniający następujący
równanie
\begin{equation}
  \label{eq:Matwiejew-Metody-calkowania-ETC-54}
  \frac{ d \vecy( t ) }{ d t } =
  \vecf \HorSpaceOne \big( t, \vecy( t ) \big) =
  \vecv \HorSpaceThree \big( \vecy( t ) \big),
\end{equation}
będące odpowiednikiem \eqref{eq:Matwiejew-Metody-calkowania-ETC-46}.
Zwyczajowo użyliśmy symbolu $t$ dla oznaczenia parametru od którego zależy
rozwiązanie $\vecy$. Dodatkowo współrzędne kartezjańskie w~zbiorze $G$
będziemy oznaczać $y_{ 1 }$ i~$y_{ 2 }$. Ostatnie równanie można
również zapisać jako
\begin{equation}
  \label{eq:Matwiejew-Metody-calkowania-ETC-55}
  \begin{bmatrix}
    y'_{ 1 }( t ) \\
    y'_{ 2 }( t )
  \end{bmatrix} =
  \begin{bmatrix}
    v_{ 1 }\big( y_{ 1 }( t ), y_{ 2 }( t ) \big) \\
    v_{ 2 }\big( y_{ 1 }( t ), y_{ 2 }( t ) \big)
  \end{bmatrix}\!.
\end{equation}

Jeśli teraz weźmiemy pole wektorowe $\vecv$ i~narysujemy zarówno je samo,
jak i~krzywe których wektory styczne w~każdym punkcie są równe wektorom
pola $\vecv$ w~tym samym punkcie, to tym razem nie otrzymamy pełnej
informacji o~ruchu punktu, który jest opisany równaniem
\eqref{eq:Matwiejew-Metody-calkowania-ETC-53}, bowiem te krzywe będą
obrazować tylko tor danego ruchu. Przykładowo, krzywa widoczna na rysunku
????
pokazuje nam, że~tor ruchu punktu tworzy pewną krzywą zamkniętą, ale nic
nie mówi nam o tym, czy punkt porusza się po niej ze stałą prędkością,
czy też raz zwalnia, raz przyśpiesza?

Aby uzyskać jak poprzednio pełną informację o~ruchu powinniśmy rozpatrywać
krzywe w~zbiorze $( t_{ 0 }, t_{ 1 } ) \times G$, na którym określone jest pole
wektorowe
\begin{equation}
  \label{eq:Matwiejew-Metody-calkowania-ETC-56}
  \vecv_{ \, \textrm{ext} }( t, y_{ 1 }, y_{ 2 } ) =
  \begin{bmatrix}
    1 \\
    v_{ 1 }( y_{ 1 }, y_{ 2 } ) \\
    v_{ 2 }( y_{ 1 }, y_{ 2 } )
  \end{bmatrix}\!.
\end{equation}
Dopiero znalezienie odpowiednika krzywych całkowych dla tego pola da nam
pełną informację o~ruchu, z~tego samego powodu co poprzednio: odpowiednik
krzywej całkowej w~zbiorze $( t_{ 0 }, t_{ 1 } ) \times G$ wyznacza funkcje
wedle definicji danej przez teorię mnogości. Powyżej podaliśmy pole
wektorowe dla tej przestrzeni, kwestią czy można na nią uogólnić pojęcie
pola kierunków, nie będziemy~się zajmować.

Różnica między równaniem \eqref{eq:Matwiejew-Metody-calkowania-ETC-46}
i~\eqref{eq:Matwiejew-Metody-calkowania-ETC-53} wynika głównie z~rodzaju
funkcji, które są ich rozwiązaniami. Dokładniej z~tego, że~funkcje typu
$y : ( t_{ 0 }, t_{ 1 } ) \to \Rbb$ są niezwykle proste, gdyż ich
przeciwdziedzina jest niezwykle prosta. W~tym przypadku torem ruchu jest
zawsze odcinek $( a, b )$, ewentualnie odcinek domknięty z jednej lub obu
stron, podczas gdy w~wypadku funkcji
$\vecy : ( t_{ 0 }, t_{ 1 } ) \to \Rbb^{ 2 }$ torami ruchu są
gładkie krzywe, czyli obiekty geometryczne o~nieporównywalnie większym
bogactwie kształtów i~zachowań niż odcinki.

Na koniec zwróćmy uwagę na rzecz stosunkowo oczywistą. Ze~względu na to,
że~w~poprzednim przypadku płaszczyzna $xy$ zamierała zarówno informacje
o~przebiegu ruchu jak i~o~jego torze, to wystarczyło określić na niej pole
kierunków. W~obecnie rozważanym przypadku, krzywe w~obszarze $G$ zawierają
tylko informacje o~torze ruchu, skutkiem czego gdybyśmy określili pole
kierunków, pole odcinków o~długości jeden, rozwiązania ruchu nie byłyby
jednoznacznie wyznaczone.

Możemy to zilustrować prostym przykładem. Weźmy $G = \Rbb^{ 2 }$ i~pole
wektorowe $\vecv( y_{ 1 }, y_{ 2 } ) = ( 1, 0 )$. Łatwo sprawdzić,
że~$\vecy_{ 1 }( t ) = [ t, 0 ]$ jest rozwiązaniem równania
\eqref{eq:Matwiejew-Metody-calkowania-ETC-49},
ale~$\vecy_{ 2 }( t ) = [ -t, 0 ]$ już nim nie jest. Gdybyśmy zamiast tego
rozpatrywali pole odcinków jednostkowy, takie że~w~punkcie
$( y_{ 1 }, y_{ 2 } )$ odcinek wyznacza ten sam kierunek na którym leży
wektor $\vecv( y_{ 1 }, y_{ 2 } )$ i~żądali od~rozwiązania tylko,
by~wektor $\vecy'( t )$ znajdujący~się w~punkcie $\vecy( t )$ posiadał
kierunek wyznaczony przez pole kierunków, wówczas $\vecy_{ 1 }( t )$
i~$\vecy_{ 2 }( t )$~byłby rozwiązaniami tak postawionego problemu.

Do tych rozważań należy któregoś dnia wrócić, poprawić je i~uzupełnić wedle
tego co jest napisane w~dziele Arnolda
\cite{ArnoldRownaniaRozniczkoweZwyczajne1975}.

\VerSpaceFour





\noindent
\Str{17} W~świetle przykładu 3 ze~strony 19, warto zatrzymać~się dłużej
nad podstawowymi własnościami izoklin. Przyjmijmy, że~izoklina dana jest
przez funkcję $h( x )$ określoną na zbiorze $( x_{ 0 }, x_{ 1 } )$. Tym
samym równanie izokliny jako krzywej jest postaci
\begin{equation}
  \label{eq:Matwiejew-Metody-calkowania-ETC-57}
  K( x ) = \big( x, h( x ) \big).
\end{equation}
Zgodnie z~definicją izokliny, dla pewnego $k \in \Rbb$ zachodzi
\begin{equation}
  \label{eq:Matwiejew-Metody-calkowania-ETC-58}
  k = f\big( x, h( x ) \big).
\end{equation}
Zauważmy, że~aby izoklina była rozwiązaniem danego równania różniczkowego
musiałoby ponadto zachodzi
\begin{equation}
  \label{eq:Matwiejew-Metody-calkowania-ETC-59}
  h'( x ) = k.
\end{equation}
Przykład~3 pokazuje, że~izoklina może być rozwiązaniem równania
różniczkowego które ją definiuje, ale jest to sytuacja dość szczególna.
Przykład~1 ze strony~17 jasno pokazuje, że~żadna z~izoklin równania
\begin{equation}
  \label{eq:Matwiejew-Metody-calkowania-ETC-60}
  y'( x ) = 2 x,
\end{equation}
nie jest jego rozwiązaniem.

\VerSpaceFour




\noindent
\Str{18} Ta strona jest napisana w~trochę bałaganiarski sposób. W~tym samym
paragrafie w~którym analizuje~się czy krzywe całkowe są rosnące czy
malejące, wprowadza~się pojęcie \textbf{linii ekstremów}, za~to
w~paragrafie o~tym czy krzywe całkowe są wypukłe czy wklęsłe, wprowadza~się
pojęcie \textbf{linii punktów przegięcia}. Te pojęcia można byłoby
wprowadzić w~bardziej elegancki i~łatwiejszy w~zrozumieniu sposób.

Najpierw w~jednym paragrafie rozważamy prawą stronę równania
\begin{equation}
  \label{eq:Matwiejew-Metody-calkowania-ETC-61}
  y'( x ) = f\big( x, y( x ) \big),
\end{equation}
dochodząc do wniosku, że~jeśli w~danym obszarze $\Ocal$ płaszczyzny $x y$
jest ona dodatnia (ujemna), to wszystkie krzywe całkowe w~tym obszarze
są skierowane ku górze (ku dołowi). Następnie możemy przeprowadzić
podobną analizę drugiej pochodnej funkcji $y( x )$, różniczkują
po $x$ równania \eqref{eq:Matwiejew-Metody-calkowania-ETC-61}:
\begin{equation}
  \label{eq:Matwiejew-Metody-calkowania-ETC-62}
  y''( x ) =
  \frac{ \partial f\big( x, y( x ) \big)}{ \partial x } +
  \frac{ \partial f\big( x, y \big) }{ \partial y }\Big|_{ y = y( x ) } \,
  f\big( x, y( x ) \big).
\end{equation}
Jeśli prawa strona tego wzoru ma znak dodatni (ujemny) w~obszarze $\Ocal$
płaszczyzny to~funkcja $y( x )$ jest w~tym obszarze wypukła (wklęsła).

W~następnym paragrafie zdefiniowalibyśmy pojęcie linii ekstremów i~linii
punktów przegięcia, teraz zaś podamy tu bardziej sformalizowaną definicję
linii ekstremów, mając nadzieję, że~pomoże to naświetlić lepiej omawiane
problemy. Niech $h( x )$ będzie funkcją określoną na odcinku
$( x_{ 1 }, x_{ 2 } )$.
Mówimy, że~$h( x )$ wyznacza linię ekstremów równania
\eqref{eq:Matwiejew-Metody-calkowania-ETC-61} jeśli przez każdy
punkt $( x_{ 0 }, y_{ 0 } )$, gdzie $y_{ 0 } = h( x_{ 0 } )$,
$x_{ 0 } \in ( x_{ 1 }, x_{ 2 } )$, przechodzi
krzywa całkowa dana funkcją $y( x )$, która to funkcja jest określona
przynajmniej na pewnym przedziale $( x_{ 0 } - \delta, x_{ 0 } + \delta )$, gdzie
$\delta > 0$, która ma w~punkcie $x_{ 0 }$ ekstremu. Jeśli w~jakimś otoczeniu
punktu $x_{ 0 }$ krzywa dana funkcją $y( x )$ pokrywa~się tą daną przez
$h( x )$, to wedle tego co napisano w~książce, nie zachodzi potrzeba by
funkcja $y( x )$ miała w~punkcie $x_{ 0 }$ ekstremu. Trzeba jednak
przyznać, że~fragment książki który mówi o~pokrywaniu się krzywej całkowej
z~linią ekstremów jest trochę niejednoznaczny i~powyższe interpretacja nie
musi być poprawna.

W~przypadku linii punktów przegięcia, nie ma żadnego komentarza odnośnie
przypadku, gdy~krzywa całkowa pokrywa~się tą linią. W~tym momencie nie
umiemy powiedzieć, czy definicja w~obecnej formie jest poprawna, czy
też~jej część została przypadkiem pominięta.

\VerSpaceFour





\noindent
\Str{19} Równanie
\begin{equation}
  \label{eq:Matwiejew-Metody-calkowania-ETC-63}
  \frac{ d y( x ) }{ d x } =
  \frac{ y( x ) }{ x }
\end{equation}
analizowane w~przykładzie~3 na tej stronie można rozwiązać w~inny sposób,
który pod pewnymi względami może być prostszy do~zrozumienia, dlatego
uważamy, iż~warto go tu zaprezentować. Z~drugiej
strony, prezentując poniżej to podejście wchodzimy w~pewne subtelności
pojęcia dziedziny rozwiązań równania różniczkowego i~izoklin, więc niniejsze
rozważania mogą okazać~się trudniejsze do zrozumienia, niż~to co można
znaleźć w~książce.

Mianowicie, zaczynamy od zbadania jego izoklin, czyli szukamy rozwiązań
równania
\begin{equation}
  \label{eq:Matwiejew-Metody-calkowania-ETC-64}
  k = \frac{ y_{ \, \izo }( x ) }{ x }, \quad
  k \in \Rbb.
\end{equation}
Dla zaznaczenia, że~izoklina nie musi być rozwiązanie równania
\eqref{eq:Matwiejew-Metody-calkowania-ETC-63}, szukaną funkcję oznaczyliśmy
$y_{ \, \izo }( x )$, a~nie $y( x )$. Równanie to jest niezmiernie proste,
więc od razu dostajemy
\begin{equation}
  \label{eq:Matwiejew-Metody-calkowania-ETC-65}
  y_{ \, \izo }( x ) = k \, x, \quad
  k \in \Rbb.
\end{equation}
To co wymaga większej uwagi, jest to, że~$x = 0$ nie należy do dziedziny
równania \eqref{eq:Matwiejew-Metody-calkowania-ETC-64}, więc izokliny są
określone tylko dla $x \in ( -\infty, 0 ) \cup ( 0, +\infty )$. Czy jednak powinnyśmy
dopuszczać jako izokliny krzywe reprezentowane przez funkcje, których
dziedziną nie jest przedział $( a, b ) \subset \Rbb$ (ewentualnie przedział
jedno- lub obustronnie domknięty)?

Jak wspomniano wyżej, zwykle chcemy by rozwiązanie równania różniczkowego
było przedziałem otwartym $( a, b ) \subset \Rbb$, a~jak zaraz~się okaże,
izokliny tego równania są również jego rozwiązaniami, więc przyjmiemy dla
nich podobną konwencję. Będziemy więc szukać izoklin, których funkcja
$y_{ \, \izo }( x )$ ma za dziedzinę możliwie największy przedział otwarty
zawarty w~$\Rbb$. Na mocy tego, rozbijamy funkcję daną przez równanie
\eqref{eq:Matwiejew-Metody-calkowania-ETC-65} na dwie funkcje:

\negVerSpaceFour


\begin{subequations}

  \begin{align}
    \label{eq:Matwiejew-Metody-calkowania-ETC-66-A}
    y_{ \, \izo, \, + }( x )
    &=
      k \, x, \quad
      x > 0, \\
    \label{eq:Matwiejew-Metody-calkowania-ETC-66-B}
    y_{ \, \izo, \, - }( x )
    &=
      k \, x, \quad
      x < 0.
  \end{align}

\end{subequations}


\noindent
Tym samy za izokliny równania \eqref{eq:Matwiejew-Metody-calkowania-ETC-63}
będziemy uważać nie proste przechodzące przez początek układu
współrzędnych, lecz odpowiednie półproste wychodzące z~tego punktu. Może to
być zgodne z~intencjami Matwiejewa wyłożonymi w~opisie omawianego
przykładu, lecz fragment o~tym, że~te półproste są zarówno krzywymi
całkowymi jak i~izoklinami, można interpretować na kilka sposobów.

Łatwo zauważyć, że~$y'( x ) = k$ i~$y( x ) / x = k$, więc mogłoby~się
izokliny są zarazem rozwiązaniami równania
\eqref{eq:Matwiejew-Metody-calkowania-ETC-63}. Tak jak w~przypadku izoklin,
rozwiązania równania są określone na zbiorze $( -\infty, 0 )$
albo $( 0, +\infty )$. Zgodność z~tekstem książki jest prawie pewna,
bo~Matwiejew tylko półproste wychodzące z~układu współrzędnych nazywa
krzywymi całkowymi badanego równania. Niemniej warto zaznaczyć, że~funkcja
\begin{equation}
  \label{eq:Matwiejew-Metody-calkowania-ETC-67}
  y_{ - }( x ) = k \, x, \quad
  x < 0,
\end{equation}
opisuje raczej półprostą wchodzącą do~środka układu współrzędnych. Jest
to~jednak drobna szczegół nazewniczy, który nie powinien prowadzić
do~nieporozumień.

Dla porządku przeanalizujemy jeszcze przypadek $x = 0$, $y \neq 0$. W~takiej
sytuacji musimy rozpatrzyć równanie
\begin{equation}
  \label{eq:Matwiejew-Metody-calkowania-ETC-68}
  \frac{ d x( y ) }{ d y } = \frac{ x( y ) }{ y }.
\end{equation}
Równanie izoklin rozwiązujemy tak jak poprzednio dostając

\negVerSpaceFour


\begin{subequations}

  \begin{align}
    \label{eq:Matwiejew-Metody-calkowania-ETC-69-A}
    x_{ \, \izo, \, + }( y )
    &= k_{ 1 } \, y, \quad
      y > 0, \\
    \label{eq:Matwiejew-Metody-calkowania-ETC-69-B}
    x_{ \, \izo, \, - }( y )
    &= k_{ 1 } \, y, \quad
      y < 0.
  \end{align}

\end{subequations}


\noindent
Tak jak poprzednio łatwo sprawdzamy, że~izokliny są też rozwiązaniami
równania \eqref{eq:Matwiejew-Metody-calkowania-ETC-68}. W~szczególności,
rozwiązaniami naszego równania są dwie półproste

\negVerSpaceFour


\begin{subequations}

  \begin{align}
    \label{eq:Matwiejew-Metody-calkowania-ETC-70-A}
    x_{ + }( y )
    &= 0, \quad
      y > 0, \\
    \label{eq:Matwiejew-Metody-calkowania-ETC-70-B}
    x_{ - }( y )
    &= 0, \quad
      y < 0.
  \end{align}

\end{subequations}


Na koniec zwróćmy uwagę na~następujący prosty fakt. Jeśli zapiszemy równania
\eqref{eq:Matwiejew-Metody-calkowania-ETC-66-A}-\eqref{eq:Matwiejew-Metody-calkowania-ETC-66-B}
i~\eqref{eq:Matwiejew-Metody-calkowania-ETC-69-A}-\eqref{eq:Matwiejew-Metody-calkowania-ETC-69-B} jako równania uwikłane na funkcje:

\negVerSpaceFour


\begin{subequations}

  \begin{align}
    \label{eq:Matwiejew-Metody-calkowania-ETC-71-A}
    y - k \, x = 0, \\
    \label{eq:Matwiejew-Metody-calkowania-ETC-71-B}
    x - k_{ 1 } \, y = 0,
  \end{align}

\end{subequations}


\noindent
gdzie $k, k_{ 1 } \in \Rbb$, a~równania te chcemy rozwiązać osobno w~zbiorze
$\Rbb \setminus ( 0, \, 0 )$, rozwiązania zaś mają być określone na zbiorach postaci
$( a, b )$, to możemy zaobserwować, iż~równania
\eqref{eq:Matwiejew-Metody-calkowania-ETC-71-A}
i~\eqref{eq:Matwiejew-Metody-calkowania-ETC-71-B} są sobie w~oczywisty
sposób częściowo równoważne. Jeśli bowiem $k \neq 0$, to równanie
\eqref{eq:Matwiejew-Metody-calkowania-ETC-71-A} przechodzi w~równanie
\eqref{eq:Matwiejew-Metody-calkowania-ETC-71-B}
z~$k_{ 1 } = \frac{ 1 }{ k }$ i~odwrotnie.

Warunek, że~rozwiązania równań
\eqref{eq:Matwiejew-Metody-calkowania-ETC-71-A}
i~\eqref{eq:Matwiejew-Metody-calkowania-ETC-71-B} mają być określone na
zbiorach postaci $( a, b )$ nie jest potrzebny by omawiana równoważność
zachodziła. Dodaliśmy ją, by rozwiązania równania uwikłanego miały tę samą
dziedzinę co izokliny analizowane wcześniej.

\VerSpaceFour





\noindent
\Str{21} Przyjrzyjmy~się następującemu tekstowi jaki możemy znaleźć na tej
stronie.


% ###################
\begin{quote}

  Przede wszystkim z~kursu analizy wiadomo, że~rozwiązanie (19) jest funkcją
  zmiennej niezależnej~$x$ różniczkowalną w~sposób ciągły (${}^{ 2 }$).
  Geometrycznie oznacza to, że~przez punkt $( x_{ 0 }, y_{ 0 } )$ przechodzi
  jedna i~tylko jedna krzywa całkowa.

\end{quote}
% ###################


\noindent
Wzór (19) o~którym tu mowa to
\begin{equation}
  \label{eq:Matwiejew-Metody-calkowania-ETC-72}
  y( x ) =
  \int_{ x_{ 0 } }^{ x } f( x ) \, dx + y_{ 0 }.
\end{equation}
Jakoś nie widzę, żeby z~tego, iż~powyższa funkcja jest różniczkowalna
w~sposób ciągły implikowało to, że~przez każdy punkt $( x_{ 0 }, y_{ 0 } )$
przechodziło tylko jedna krzywa całkowa. Tak w~istocie jest, co wiem na
podstawie twierdzenia Picarda, które w~tej książce pierwszy razy zostanie
wspomniane na stronie~24.

\VerSpaceFour





\noindent
\Str{24} Na tej stronie znajdujemy zdanie


% ###################
\begin{quote}

  W~tej książce rozpatrujemy tylko rozwiązania różniczkowalne w~sposób
  ciągły.

\end{quote}
% ###################


\noindent
Wydaje mi~się, że~Matwiejew chciał za~jego pomocą przekazać następującą
myśl. Jak już wiemy, jeśli w~równaniu
\begin{equation}
  \label{eq:Matwiejew-Metody-calkowania-ETC-73}
  y'( x ) = f\big( x, y( x ) \big)
\end{equation}
prawa strona jest funkcją ciągłą określoną w~obszarze\footnote{Moglibyśmy
  opuścić żądanie, że~$G$ jest obszarem, jednak przekładamy tutaj
  prostotę sformułowania nad jego ogólność.}
$G \subset \Rbb^{ n }$, to na mocy twierdzenia Peanom również rozwiązanie tego
równania jest funkcją ciągłą. Dlatego też od~tego momentu będziemy
zakładać, że~prawa strona równania
\eqref{eq:Matwiejew-Metody-calkowania-ETC-73} jest funkcją ciągłą
co pozwoli nam skupić naszą uwagę na rozwiązaniach równań różniczkowych
zwyczajnych, które spełniają ten warunek.

Ciekawym zagadnienie pozostaje znalezienie takich funkcji~$f$, które
nie są ciągłe, ale~zadane przez nie równanie
\eqref{eq:Matwiejew-Metody-calkowania-ETC-73} posiada rozwiązania
różniczkowalne w~sposób ciągły.

\VerSpaceFour





\noindent
\Str{24} Na tej stronie spotykamy~się po raz pierwszym z~twierdzeniem
Picarda o~istnieniu i~jednoznaczności rozwiązań równania różniczkowego
zwyczajnego. Czy mamy domyślnie przyjmować, że~w~dalszej części książki
założenia tego twierdzenia są spełnione, chyba że~powiedziano inaczej?

\VerSpaceFour





\noindent
\Str{27--28} Na stronach tych znajdujemy definicję, która określa kiedy
funkcja dwóch zmiennych $\varphi( x, C )$ jest rozwiązaniem ogólnym pewnego
równania różniczkowego, co zapisujemy symbolicznie jako
\begin{equation}
  \label{eq:Matwiejew-Metody-calkowania-ETC-74}
  y( x ) = \varphi( x, C ).
\end{equation}
Jednym z~nich jest to, by w~zadanym obszarze można ją było odwikłać
względem~$C$. Dokładny sens tego stwierdzenia jest następujący.
Przekształcamy równanie \eqref{eq:Matwiejew-Metody-calkowania-ETC-73}
z~równania definiującego funkcję $y( x )$ na równanie uwikłane na trzy
zmienne niezależne: $x$, $y$ i~$C$. Równanie to ma postać
\begin{equation}
  \label{eq:Matwiejew-Metody-calkowania-ETC-75}
  y = \varphi( x, C ).
\end{equation}
Następnie żądamy by można je było odwikłać na funkcję $C = \psi( x, y )$.
Ponieważ do~równania tego stosuje~się bez zmian standardowa teorii równań
uwikłanych, dlatego nie będziemy dłużej zatrzymywać~się przy tym
zagadnieniu.

Na koniec zwróćmy jeszcze uwagę na pewną osobliwość zapisu równania
\eqref{eq:Matwiejew-Metody-calkowania-ETC-74}. Po~jego lewej stronie mamy
funkcję jednej zmiennej, a~po jego prawej stronie dwóch zmiennych. Choć nie
jest to w~pełni ścisły zapis, notacja taka jest dobrze uzasadniona
przez praktykę badania i~rozwiązywania równań różniczkowych. Symbolizuje
ono to, że~konkretne rozwiązanie równania różniczkowego otrzymujemy, przez
podstawienie za~drugi argument funkcji $\varphi( x, C )$ konkretnej wartości
zmiennej $C$, w~rezultacie czego otrzymujemy funkcję jednej zmiennej.

\VerSpaceFour





\noindent
\Str{29} Zatrzymamy~się na chwilę nad słowami „Jasne jest wreszcie,
że~funkcja~(24'') jest rozwiązaniem równania~(14) przy wszystkich
wartościach~$C$ danych wzorem~(27), gdy punkt $( x, y )$ przebiega
obszar~(26).” Równanie (14) jest postaci
\begin{equation}
  \label{eq:Matwiejew-Metody-calkowania-ETC-76}
  \frac{ d y( x ) }{ d x } = \frac{ y }{ x },
\end{equation}
natomiast równanie~(24'') to
\begin{equation}
  \label{eq:Matwiejew-Metody-calkowania-ETC-77}
  y( x ) = C x.
\end{equation}
Jeśli chodzi o~równanie (27) to ma ono postać
\begin{equation}
  \label{eq:Matwiejew-Metody-calkowania-ETC-78}
  C = \frac{ y }{ x },
\end{equation}
zaś wzór (26) definiuje półpłaszczyznę $0 < x < +\infty$, $-\infty < y < +\infty$.
W~tym miejscu warto przypomnieć, że~rozpatrujemy tylko funkcje przyjmujące
wartości rzeczywiste.

Należy zwrócić uwagę, że~funkcja
\eqref{eq:Matwiejew-Metody-calkowania-ETC-77} jest rozwiązaniem równania
\eqref{eq:Matwiejew-Metody-calkowania-ETC-76} dla dowolnej wartości
stałej~$C$. Tym samym dowolne wyrażenie postaci $C = \psi( x, y )$, przy
założeniu, że~prowadzi ono do~rzeczywistego~$C$, zwraca nam stałą, taką
że~funkcja $y( x ) = C x$ jest rozwiązaniem rozpatrywanego równania. Z~tego
punktu widzenia nie ma nic specjalnego w~formule
\eqref{eq:Matwiejew-Metody-calkowania-ETC-78}.

Tym co~wyróżnia \eqref{eq:Matwiejew-Metody-calkowania-ETC-78} spośród
wszystkich funkcji $\psi( x, y )$ o~wartościach rzeczywistych jest to,
że~jeśli
\begin{equation}
  \label{eq:Matwiejew-Metody-calkowania-ETC-79}
  C_{ 0 } = \psi( x_{ 0 }, y_{ 0 } ) = \frac{ y_{ 0 } }{ x_{ 0 } },
\end{equation}
to zachodzi
\begin{equation}
  \label{eq:Matwiejew-Metody-calkowania-ETC-80}
  y( x_{ 0 }, C_{ 0 } ) = y_{ 0 }.
\end{equation}
Fakt, że~funkcja $\psi( x, y )$ o~której mowa w~definicji rozwiązania ogólnego
musi spełniać ten warunek, by~$y( x ) = C x$ było rozwiązaniem ogólnym
badanego równania, jest zawarty w~definicji tej funkcji podanej
na~stronie~27, gdzie jest napisane, że~funkcja $\psi( x, y )$ powstaje
poprzez odwikłanie równania
\begin{equation}
  \label{eq:Matwiejew-Metody-calkowania-ETC-81}
  y = \varphi( x, C ),
\end{equation}
na zmienną $C$. O~tym jak należy rozumieć powyższe równanie i~całą procedurę
pisaliśmy już w~komentarzu do stron 27--28.

\VerSpaceFour





\noindent
Czytamy tutaj, że~aby by wyrażenie
\begin{equation}
  \label{eq:Matwiejew-Metody-calkowania-ETC-82}
  \Phi( x, y, C ) = 0,
\end{equation}
lub
\begin{equation}
  \label{eq:Matwiejew-Metody-calkowania-ETC-83}
  \psi( x, y ) = C,
\end{equation}
reprezentowało \textbf{rozwiązanie ogólne w~postaci uwikłanej}, zwane też
\textbf{całką ogólną}, związek ten musi określać rozwiązanie ogólne
$y( x ) = \varphi( x, C )$. Dokładny sens tych słów, wydaje~się być następujący.
Związek \eqref{eq:Matwiejew-Metody-calkowania-ETC-82} lub
\eqref{eq:Matwiejew-Metody-calkowania-ETC-83} można odwikłać względem
zmiennej $y$, w~rezultacie czego dostajemy funkcje $y( x, C ) = \varphi( x, C )$,
która spełnia przedstawioną wcześniej w~książce definicję rozwiązania
ogólnego.

\VerSpaceFour





\noindent
\Str{29} Przykład na tej stronie dobrze ilustruje pojęcia całki ogólna
(rozwiązanie ogólne w~postaci uwikłanej), jak również samego rozwiązania
ogólnego. Warto jednak omówić go odrobinę szerzej.

Równanie które chcemy rozwiązać to
\begin{equation}
  \label{eq:Matwiejew-Metody-calkowania-ETC-84}
  \frac{ d y( x ) }{ dx } = -\frac{ x }{ y( x ) }.
\end{equation}
W~punkcie $( 0, 0 )$ równanie to prowadzi do wyrażenia $0 / 0$, więc
równanie w~tym punkcie jest niezdefiniowane. Dla punktów dla których
$y = 0$ rozpatrywać będziemy równanie
\begin{equation}
  \label{eq:Matwiejew-Metody-calkowania-ETC-85}
  \frac{ d x( y ) }{ dy } = -\frac{ y }{ x( y ) }.
\end{equation}

Wprowadzamy cztery podzbiory płaszczyzny:
$D_{ 1 } = \{ ( x, y ) \, | \, y > 0 \}$,
$D_{ 2 } = \{ ( x, y ) \, | \, y < 0 \}$,
$D_{ 3 } = \{ ( x, y ) \, | \ x > 0 \}$
i~$D_{ 4 } = \{ ( x, y ) \, | \, x < 0 \}$. Jak już wiemy
$y( x ) = \sqrt{ C - x^{ 2 } }$ jest rozwiązaniem ogólnym równania
\eqref{eq:Matwiejew-Metody-calkowania-ETC-84} w~obszarze~$D_{ 1 }$, zaś
$y( x ) = -\sqrt{ C - x^{ 2 } }$ w~obszarze~$D_{ 2 }$. Natomiast funkcja
$x( y ) = \sqrt{ C - y^{ 2 } }$ jest rozwiązaniem ogólnym równania
\eqref{eq:Matwiejew-Metody-calkowania-ETC-85} w~zbiorze $D_{ 3 }$, zaś
$x( y ) = -\sqrt{ C - y^{ 2 } }$ w~obszarze $D_{ 4 }$. Ten zestaw rozwiązań
ogólnych jest w~następującym sensie. Suma zbiorów $D_{ i }$ pokrywa zbiór
$\Rbb \setminus { ( 0, 0 ) }$, czyli zbiór wszystkich punktów, gdzie przynajmniej
jedno z~równań \eqref{eq:Matwiejew-Metody-calkowania-ETC-84} lub
\eqref{eq:Matwiejew-Metody-calkowania-ETC-85} ma sens. Żadnego ze~zbiorów
$D_{ i }$ nie da~się powiększyć w~żaden sposób, nie tracąc przy tym
możliwości określenia rozwiązania ogólnego w~tym zbiorze. Przypomnijmy,
że~określając rozwiązanie ogólne $\varphi( x, C )$ równania
$y'( x ) = f\big( x, y( x ) \big)$ musimy też podać obszar dla którego
to~rozwiązanie obowiązuje.

\VerSpaceFour





\noindent
\Str{31} Jest to dziwne, że~na tej stronie czytamy „W~dalszym ciągu nie
będziemy zajmowali~się rozwiązaniami otrzymywanymi przez sklejanie.”, po
czym następuje przykład, który w~bardzo dobry sposób ilustruje czym są
takie sklejane rozwiązania. Zapewne autorowi chodziło o~to, że~po omówieniu
tego przykładu, rozwiązania otrzymane przez sklejanie rozwiązań szczególnych
i~osobliwych nie będą w~książce rozpatrywane.

\VerSpaceFour





\noindent
\Str{31} Warto zwrócić uwagę na następującą subtelność rachunkową.
Podstawiając funkcję
\begin{equation}
  \label{eq:Matwiejew-Metody-calkowania-ETC-86}
  y( x ) = ( x + C )^{ 2 },
\end{equation}
do równania
\begin{equation}
  \label{eq:Matwiejew-Metody-calkowania-ETC-87}
  y'( x ) = 2 \sqrt{ y( x ) },
\end{equation}
dostajemy
\begin{equation}
  \label{eq:Matwiejew-Metody-calkowania-ETC-88}
  2 ( x + C ) = \absOne{ x + C }.
\end{equation}
Widzimy więc, że~funkcja \eqref{eq:Matwiejew-Metody-calkowania-ETC-86}
jest rozwiązaniem równania \eqref{eq:Matwiejew-Metody-calkowania-ETC-87}
tylko wtedy, gdy~$x \geq -C$.

\VerSpaceFour





\noindent
\Str{34} Niech $P( x, y )$ i~$Q( x, y )$ będą nieprzewiedlnymi wielomianami
zmiennych $x$ i~$y$ o~których mowa w~punkcie $2^{ \circ }$ na tej stronie.
W~jaki sposób można pokazać następującą własność, która jest wykorzystywana
w~dowodzie tego punktu? Mianowicie, że jeśli zachodzi
\begin{equation}
  \label{eq:Matwiejew-Metody-calkowania-ETC-89}
  P( x_{ 0 }, y_{ 0 } ) = Q( x_{ 0 }, y_{ 0 } ) = 0,
\end{equation}
to wówczas istnieje takie otoczenie $O$ punktu $( x_{ 0 }, y_{ 0 } )$,
że~dla każdego punktu $( x_{ 1 }, y_{ 1 } )$ różnego od $( x_{ 0 }, y_{ 0 } )$
zachodzi
\begin{equation}
  P( x_{ 1 }, y_{ 1 } ) \neq 0 \HorSpaceSixteen
  \vee \HorSpaceSixteen
  Q( x_{ 1 }, y_{ 1 } ) \neq 0.
\end{equation}

\VerSpaceFour





\noindent
\Str{35 i~kolejne} Na stronie 35 zaczynają~się pojawiać pojęcia
i~stwierdzenia, które zapewne dobrze były wyjaśnić za pomocą pojęć geometrii
różniczkowej.

{\Large Od strony 35 należy zacząć ponowne czytanie książki.}























% ##################
\newpage

\CenterBoldFont{Błędy}


\begin{center}

  \begin{tabular}{|c|c|c|c|c|}
    \hline
    Strona & \multicolumn{2}{c|}{Wiersz} & Jest
                              & Powinno być \\ \cline{2-3}
    & Od góry & Od dołu & & \\
    \hline
    5   & &  3 & wierdzenia & twierdzenia \\
    15  & & 14 & ono określa & określa ono \\
    32  & & 19 & $2 ( x + C )$ & $2 | x + C |$ \\
    33  & &  8 & $b_{ 1 } > b$ & $b_{ 1 } < b$ \\
    % & & & & \\
    % & & & & \\
    % & & & & \\
    % & & & & \\
    \hline
  \end{tabular}

\end{center}

\VerSpaceSix


\noindent
\StrWierszG{15}{13} \\
\Jest  w~sensie ustępu \\
\Powin w~sensie zdefiniowanym w~ustępie \\
\StrWierszD{20}{2} \\
\Jest  i~nie ma rozwiązania określonego w~tym samym przedziale
nie~identycznego z~rozwiązaniem $y = y( x )$ chociażby w~jednym
punkcie przedziału $| x - x_{ 0 } | \leq h$ różnym od~punktu $x = x_{ 0 }$. \\
\Powin i~nie istnieje rozwiązanie określone w~przedziale
$| x - x_{ 0 } | \leq h_{ 1 } \leq h$, gdzie $h_{ 1 } > 0$, spełniające ten warunek
początkowy, które nie byłoby równe rozwiązaniu $y = y( x )$ w~każdym
punkcie tego przedziału. \\
\StrWierszG{21}{2} \\
\Jest  nie jedno rozwiązanie \\
\Powin ma co najmniej dwa rozwiązania \\
\StrWierszD{26}{9} \\
\Jest  w~pobliżu punktu $x = x_{ 0 }$ \\
\Powin dla każdego $| x - x_{ 0 } | \leq h$, $h > 0$ \\
\StrWierszG{30}{5} \\
\Jest  \textit{w \hspace{0.5em} postaci \hspace{0.5em} parametrycznej} \\
\Powin \textit{w~postaci parametrycznej} \\
\StrWierszD{32}{13} \\
\Jest  nie jedna \\
\Powin więcej niż jedna \\

% ############################










% ######################################
\section{Oznaczenia i~konwencje}

\label{sec:Oznaczenia-i-konwencje}
% ######################################



\noindent
Dziedzinę funkcji $f$ będziemy oznaczać symbolem $\Dcal( f )$.

\VerSpaceFour










% ####################################################################
% ####################################################################
% Bibliography

\bibliographystyle{plalpha}

\bibliography{MathematicsBooks}{}





% ############################

% End of the document
\end{document}
