% ---------------------------------------------------------------------
% Podstawowe ustawienia i pakiety
% ---------------------------------------------------------------------
\RequirePackage[l2tabu, orthodox]{nag} % Wykrywa przestarzałe i niewłaściwe
% sposoby używania LaTeXa. Więcej jest w l2tabu English version.
\documentclass[a4paper,11pt]{article}
% {rozmiar papieru, rozmiar fontu}[klasa dokumentu]
\usepackage[MeX]{polski} % Polonizacja LaTeXa, bez niej będzie pracował
% w języku angielskim.
\usepackage[utf8]{inputenc} % Włączenie kodowania UTF-8, co daje dostęp
% do polskich znaków.
\usepackage{lmodern} % Wprowadza fonty Latin Modern.
\usepackage[T1]{fontenc} % Potrzebne do używania fontów Latin Modern.



% ------------------------------
% Podstawowe pakiety (niezwiązane z ustawieniami języka)
% ------------------------------
\usepackage{microtype} % Twierdzi, że poprawi rozmiar odstępów w tekście.
\usepackage{graphicx} % Wprowadza bardzo potrzebne komendy do wstawiania
% grafiki.
\usepackage{verbatim} % Poprawia otoczenie VERBATIME.
\usepackage{textcomp} % Dodaje takie symbole jak stopnie Celsiusa,
% wprowadzane bezpośrednio w tekście.
\usepackage{vmargin} % Pozwala na prostą kontrolę rozmiaru marginesów,
% za pomocą komend poniżej. Rozmiar odstępów jest mierzony w calach.
% ------------------------------
% MARGINS
% ------------------------------
\setmarginsrb
{ 0.7in}  % left margin
{ 0.6in}  % top margin
{ 0.7in}  % right margin
{ 0.8in}  % bottom margin
{  20pt}  % head height
{0.25in}  % head sep
{   9pt}  % foot height
{ 0.3in}  % foot sep



% ------------------------------
% Często przydatne pakiety
% ------------------------------
\usepackage{csquotes} % Pozwala w prosty sposób wstawiać cytaty do tekstu.
\usepackage{xcolor} % Pozwala używać kolorowych czcionek (zapewne dużo
% więcej, ale ja nie potrafię nic o tym powiedzieć).



% ------------------------------
% Pakiety do tekstów z nauk przyrodniczych
% ------------------------------
\let\lll\undefined % Amsmath gryzie się z językiem pakietami do języka
% polskiego, bo oba definiują komendę \lll. Aby rozwiązać ten problem
% oddefiniowuję tę komendę, ale może tym samym pozbywam się dużego Ł.
\usepackage[intlimits]{amsmath} % Podstawowe wsparcie od American
% Mathematical Society (w skrócie AMS)
\usepackage{amsfonts, amssymb, amscd, amsthm} % Dalsze wsparcie od AMS
% \usepackage{siunitx} % Dla prostszego pisania jednostek fizycznych
\usepackage{upgreek} % Ładniejsze greckie litery
% Przykładowa składnia: pi = \uppi
\usepackage{slashed} % Pozwala w prosty sposób pisać slash Feynmana.
\usepackage{calrsfs} % Zmienia czcionkę kaligraficzną w \mathcal
% na ładniejszą. Może w innych miejscach robi to samo, ale o tym nic
% nie wiem.



% ##########
% Tworzenie otoczeń "Twierdzenie", "Definicja", "Lemat", etc.
\newtheorem{theorem}{Twierdzenie}  % Komenda wprowadzająca otoczenie
% „theorem” do pisania twierdzeń matematycznych
\newtheorem{definition}{Definicja}  % Analogicznie jak powyżej
\newtheorem{corollary}{Wniosek}



% ---------------------------------------
% Pakiety napisane przez użytkownika.
% Mają być w tym samym katalogu to ten plik .tex
% ---------------------------------------
\usepackage{latexgeneralcommands}
\usepackage{mathcommands}
% \usepackage{calculuscommands}
% \usepackage{SchwartzBooksCommands}  % Pakiet napisany m.in. dla tego pliku.



% ---------------------------------------------------------------------
% Dodatkowe ustawienia dla języka polskiego
% ---------------------------------------------------------------------
\renewcommand{\thesection}{\arabic{section}.}
% Kropki po numerach rozdziału (polski zwyczaj topograficzny)
\renewcommand{\thesubsection}{\thesection\arabic{subsection}}
% Brak kropki po numerach podrozdziału



% ------------------------------
% Ustawienia różnych parametrów tekstu
% ------------------------------
\renewcommand{\arraystretch}{1.2} % Ustawienie szerokości odstępów między
% wierszami w tabelach.



% ------------------------------
% Pakiet "hyperref"
% Polecano by umieszczać go na końcu preambuły.
% ------------------------------
\usepackage{hyperref} % Pozwala tworzyć hiperlinki i zamienia odwołania
% do bibliografii na hiperlinki.










% ---------------------------------------------------------------------
% Tytuł i autor tekstu
\title{Teoria automatów i~języków formalnych \\
  Błędy i~uwagi}

\author{Kamil Ziemian}
% \date{}
% ---------------------------------------------------------------------










% ####################################################################
\begin{document}
% ####################################################################





% ######################################
\maketitle % Tytuł całego tekstu
% ######################################





% ############################
\Work{ % Autorzy i tytuł dzieła
  Maria Foryś, Wit Foryś \\
  \textit{Teoria automatów i~języków formalnych},
  \cite{ForysForysTeoriaAutomatowIJezykowFormalnych2005} }


% ##################
\CenterBoldFont{Uwagi}


W~tej książce używany jest cudzysłów w~formie ”cytowany tekst”, ale polskie
standardy typograficzna mówią, że~powinno~się stosować formę: „cytowany
tekst”. Dodatkowo, wyjątek warto uczynić dla ciągów symboli (liter z~danego
alfabetu $A$), nazywanych napisami bądź stringami (ang. \textit{strings}),
które zgodnie z~przyjętą w~informatyce konwencją będziemy oznaczać jako
"abc".

Więcej o~napisach powiemy przy okazji omawiania monoidów wolnych.



% ##################
\CenterBoldFont{Uwagi do konkretnych stron}

% \vspace{\spaceFour}


\start \Str{6} Według mnie symbol $\textrm{mod}_{ 6 }$ wygląda znacznie
lepiej, niż używany w tej książce $mod_{ 6 }$.

\vspace{\spaceFour}



\start \Str{6} Na tej stronie znajdujemy informację, że~element $x \in M$
jest odwracalny w~monoidzie $M$ wtedy i tylko wtedy, gdy istnieje takie
$y \in M$, że $x \cdot y = 1_{ M }$. Jednak monoid nie musi być przemienny, więc
nie widzę powodu by zachodziła też równość $y \cdot x = 1_{ M }$. Trzeba się
dokładniej przyjrzeć temu problemowi.

\vspace{\spaceFour}



\start \Str{6} W~dowodzie twierdzenia 1.1.1 brakuje mi zdania typu „Jeśli
$S$ nie jest monoidem, to rozszerzamy go do monoidu, za pomocą procedury
którą zaprezentujemy poniżej.”.

\vspace{\spaceFour}



\start \Str{6} Dowodu twierdzenia 1.1.1 zawiera dodatkową informację, nad
którą warto~się zatrzymać, mianowicie że~każdą półgrupę można
rozszerzyć do monoidu. Wymaga to jedynie
oczywistego/nieoczywistego\footnote{Zależy to od przekonań
  filozoficzno-matematycznych konkretnej osoby.}, że~istnieje element
$1 \notin S$.

W~kontekście samego dowodu twierdzenia 1.1.1, warto~się zastanowić nad tym
dlaczego dokonujemy rozszerzenia półgrupy $( S, \cdot )$ do monoidu
$( S^{ 1 }, \cdot, 1 )$. Dzięki temu mamy rodzinę odwzorowań
$\rho_{ a } : S^{ 1 } \to S^{ 1 }$ i~ponieważ $S^{ 1 }$ jest monoidem, to
mamy $\rho_{ a }( 1 ) = a$, ta zaś równość pozwala pokazać,
że~z~$\rho_{ a_{ 1 } } = \rho_{ a_{ 2 } }$ wynika $a_{ 1 } = a_{ 2 }$. Z tego zaś od
razu wynika, że odwzorowanie $h : S \to ( \{ \rho_{ a } \}_{ a \in S } )$,
$h( a ) = \rho_{ a }$ jest iniektywne. Poza tym rozszerzenie półgrupy do monoidu
nie wydaje się nigdzie indziej potrzebne.

Pytanie, czy istnieje sposób pokazania, że~odwzorowanie $h$ jest iniektywne,
bez rozszerzania półgrupy $S$ do monoidu $S^{ 1 }$? Nawet jeżeli tak, to
prostota przeprowadzonego w~książce dowodu sprawia, że warto przy nim
pozostać.

\vspace{\spaceFour}



\start \Str{7} Tak jak w~przypadku symbolu $\textrm{mod}_{ 6 }$, wydaje
mi~się, że~lepiej wyglądałby symbol $\textrm{Ker}_{ h }$.








% ##################
\CenterBoldFont{Błędy}


\begin{center}

  \begin{tabular}{|c|c|c|c|c|}
    \hline
    & \multicolumn{2}{c|}{} & & \\
    Strona & \multicolumn{2}{c|}{Wiersz} & Jest
                              & Powinno być \\ \cline{2-3}
    & Od góry & Od dołu & & \\
    \hline
    5  & 11 & & $\forall x, y, z \in S$ & $\forall x, y, z \in S,$ \\
    5  & 15 & & $\forall x \in M$ & $\forall x \in M,$ \\
    5  & & 13 & $( \textrm{\textbf{S}}, \cdot )$ & $( S, \cdot )$ \\
    5  & & 12 & $( \textrm{\textbf{M}}, \cdot, 1_{ \textrm{\textbf{M}}} )$
           & $( M, \cdot, 1_{ M } )$ \\
    5  & &  1 & $x \in \textrm{\textbf{M}}$ & $x \in M$ \\
    5  & &  1 & $\exists b \in B\; ,$ & $\exists b \in B,$ \\
    6  &  9 & & $( S, \cdot )${ }, { }{ }$( S', \ast )$
           & $( S, \cdot )$, $( S', \ast )$ \\
    6  & 11 & & $\forall x, y \in S$ & $\forall x, y \in S,$ \\
    6  & 14 & & $\forall x, y \in M$ & $\forall x, y \in M,$ \\
    6  & & 14 & $x \cdot y = 1$ & $x \cdot y = 1_{ M }$ \\
    6  & & 13 & Podgrupa & Grupa \\
    7  & & 12 & $\forall x, y, z \in S$ & $\forall x, y, z \in S,$ \\
    7  & & 10 & $\forall x, y, z \in S$ & $\forall x, y, z \in S,$ \\
    7  & &  8 & $\forall x, y, z \in S$ & $\forall x, y, z \in S,$ \\
    7  & &  4 & $S / \rho$ & $S_{ / \rho }$ \\
    7  & &  3 & $M / \rho$ & $M_{ / \rho }$ \\
    8  & & 10 & $n \geq 0\;\;,$ & $n \geq 0,$ \\
    8  & &  7 & $n \!\! = \!\! 0$ & $n = 0$ \\
    9  & &  9 & $\to \textrm{\textbf{M}}$ & $\to M$ \\
    % & & & & \\
    % & & & & \\
    % & & & & \\
    % & & & & \\
    % & & & & \\
    % & & & & \\
    \hline
  \end{tabular}

\end{center}


\noindent
\StrWd{6}{13} \\
\Jest  jest podgrupą $M$. \\
\Powin jest grupą zawartą, jako zbiór, w~$M$. \\
\StrWd{6}{1} \\
\Jest  \textit{dla} $x \in S^{ 1 }$ $\rho_{ a }( x )= a x$. \\
\Powin $\rho_{ a }( x ) = a x$ \textit{dla} $x \in S^{ 1 }$. \\


\vspace{\spaceTwo}
% ############################










% #####################################################################
% #####################################################################
% Bibliografia
\bibliographystyle{plalpha}

\bibliography{MathComScienceBooks}{}





% ############################

% Koniec dokumentu
\end{document}
