% ---------------------------------------------------------------------
% Podstawowe ustawienia i pakiety
% ---------------------------------------------------------------------
\RequirePackage[l2tabu, orthodox]{nag} % Wykrywa przestarzałe i niewłaściwe
% sposoby używania LaTeXa. Więcej jest w l2tabu English version.


\documentclass[a4paper,11pt]{article}
% {rozmiar papieru, rozmiar fontu}[klasa dokumentu]
\usepackage[MeX]{polski} % Polonizacja LaTeXa, bez niej będzie pracował
% w języku angielskim.
\usepackage[utf8]{inputenc} % Włączenie kodowania UTF-8, co daje dostęp
% do polskich znaków.
\usepackage[T1]{fontenc} % Potrzebne do używania fontów Latin Modern.
\usepackage{lmodern} % Wprowadza fonty Latin Modern.



% ------------------------------
% Podstawowe pakiety (niezwiązane z ustawieniami języka)
% ------------------------------
\usepackage{microtype} % Twierdzi, że poprawi rozmiar odstępów w tekście.
% \usepackage{graphicx} % Wprowadza bardzo potrzebne komendy do wstawiania
% grafiki.
% \usepackage{verbatim} % Poprawia otoczenie VERBATIME.
% \usepackage{textcomp} % Dodaje takie symbole jak stopnie Celsiusa,
% wprowadzane bezpośrednio w tekście.
\usepackage{vmargin} % Pozwala na prostą kontrolę rozmiaru marginesów,
% za pomocą komend poniżej. Rozmiar odstępów jest mierzony w calach.
% ------------------------------
% MARGINS
% ------------------------------
\setmarginsrb
{ 0.7in}  % left margin
{ 0.6in}  % top margin
{ 0.7in}  % right margin
{ 0.8in}  % bottom margin
{  20pt}  % head height
{0.25in}  % head sep
{   9pt}  % foot height
{ 0.3in}  % foot sep



% ------------------------------
% Często przydatne pakiety
% ------------------------------
\usepackage{csquotes} % Pozwala w prosty sposób wstawiać cytaty do tekstu.
\usepackage{xcolor} % Pozwala używać kolorowych czcionek (zapewne dużo
% więcej, ale ja nie potrafię nic o tym powiedzieć).



% ------------------------------
% Pakiety do tekstów z nauk przyrodniczych
% ------------------------------
\let\lll\undefined % Amsmath gryzie się z pakietami do języka
% polskiego, bo oba definiują komendę \lll. Aby rozwiązać ten problem
% oddefiniowuję tę komendę, ale może tym samym pozbywam się dużego Ł.
\usepackage[intlimits]{amsmath} % Podstawowe wsparcie od American
% Mathematical Society (w skrócie AMS)
\usepackage{amsfonts, amssymb, amscd, amsthm} % Dalsze wsparcie od AMS
% \usepackage{siunitx} % Do prostszego pisania jednostek fizycznych
\usepackage{upgreek} % Ładniejsze greckie litery
% Przykładowa składnia: pi = \uppi
\usepackage{slashed} % Pozwala w prosty sposób pisać slash Feynmana.
\usepackage{calrsfs} % Zmienia czcionkę kaligraficzną w \mathcal
% na ładniejszą. Może w innych miejscach robi to samo, ale o tym nic
% nie wiem.



% ------------------------------
% Tworzenie środowisk (?) „Twierdzenie”, „Definicja”, „Lemat”, etc.
% ------------------------------
% Komenda wprowadzająca otoczenie „theorem” do pisania twierdzeń
% matematycznych.
\newtheorem{theorem}{Twierdzenie}
% Analogicznie jak powyżej
\newtheorem{definition}{Definicja}
\newtheorem{corollary}{Wniosek}



% ------------------------------
% Pakiety napisane przez użytkownika.
% Mają być w tym samym katalogu to ten plik .tex
% ------------------------------
\usepackage{latexgeneralcommands}
\usepackage{mathcommands}
% \usepackage{PDE} % Pakiet napisany między innymi
% % dla tego pliku.





% --------------------------------------------------------------------
% Dodatkowe ustawienia dla języka polskiego
% --------------------------------------------------------------------
\renewcommand{\thesection}{\arabic{section}.}
% Kropki po numerach rozdziału (polski zwyczaj topograficzny)
\renewcommand{\thesubsection}{\thesection\arabic{subsection}}
% Brak kropki po numerach podrozdziału



% ------------------------------
% Ustawienia różnych parametrów tekstu
% ------------------------------
\renewcommand{\baselinestretch}{1.1}

% Ustawienie szerokości odstępów między wierszami w tabelach.
\renewcommand{\arraystretch}{1.4}



% ------------------------------
% Pakiet "hyperref"
% Polecano by umieszczać go na końcu preambuły.
% ------------------------------
\usepackage{hyperref} % Pozwala tworzyć hiperlinki i zamienia odwołania
% do bibliografii na hiperlinki.










% ---------------------------------------------------------------------
% Tytuł, autor, data
\title{Równania różniczkowe cząstkowe \\
  {\Large Błędy i~uwagi}}

\author{Kamil Ziemian}


% \date{}
% ---------------------------------------------------------------------










% ######################################################################
\begin{document}
% ######################################################################





% ######################################
\maketitle % Tytuł całego tekstu
% ######################################





% ############################
\Work{ % Autor i tytuł dzieła
  L.C. Evans \\
  \textit{Równania różniczkowe cząstkowe},
  \cite{EvansRowaniaRozniczoweCzastkowe2008}}

\vspace{0em}


% ##################
\CenterBoldFont{Uwagi do~konkretnych stron}

\vspace{0em}


\noindent
\Str{594}

\vspace{\spaceFour}





\noindent
\Str{597} Nie rozumiem, dlaczego trzeba zakładać we~wzorze całkowania po
włóknach, że~$f$ jest ciągła. Wydaje~się, że~całkowalność i~mierzalność
wystarczają.

\vspace{\spaceFour}






% ##################
\CenterBoldFont{Błędy}


\begin{center}

  \begin{tabular}{|c|c|c|c|c|}
    \hline
    Strona & \multicolumn{2}{c|}{Wiersz} & Jest
                              & Powinno być \\ \cline{2-3}
    & Od góry & Od dołu & & \\
    \hline
    19  &  4 & & równań & tych równań \\
    % & & & & \\
    % & & & & \\
    % & & & & \\
    % & & & & \\
    % & & & & \\
    % & & & & \\
    593 &  5 & & $a, b > 0$ & $a, b \in \Rbb$ \\
    597 &  1 & & \textit{współrzędnych) iloczyn}
           & \textit{współrzędnych)} \\
    597 &  2 & & $x_{ n } >$ & $x_{ n } =$ \\
    600 & & 4 & $x + h_{ i } e_{ i } \in U_{ \epsilon }$
           & $x + h_{ i } e_{ i } \in U$ \\
    %% & & & & \\
    601 & & 10 & $W \: \subset \subset U$ & $W \subset \subset U$ \\
    601 & &  5 & $W \: \subset \subset U$ & $W \subset \subset U$ \\
    614 &  9 & & $u \: \in H$ & $u \in H$\\
    615 & &  9 & miary & miary. \\
    % & & & & \\
    \hline
  \end{tabular}

\end{center}

\vspace{\spaceTwo}









% ############################










% ############################
\newpage

\Work{ % Autor i tytuł dzieła
  Lars H\"{o}rmander \\
  \textit{The Analysis of Linear Partial Differential Operators~I} \\
  \textit{Distributon Theory and~Fourier Analysis},
  \cite{HormanderAnalysisPartialDifferentialOperators1983}}


% ##################
\CenterBoldFont{Błędy}


\begin{center}

  \begin{tabular}{|c|c|c|c|c|}
    \hline
    Strona & \multicolumn{2}{c|}{Wiersz} & Jest
                              & Powinno być \\ \cline{2-3}
    & Od góry & Od dołu & & \\
    \hline
    6   &  5 & & $M >$ & $M =$ \\
    % & & & & \\
    \hline
  \end{tabular}

\end{center}

\vspace{\spaceTwo}


\noindent
\StrWg{1}{3} \\
\Jest  every function is not differentiable. \\
\Powin not every function is differentiable. \\





% ############################










% ############################
\newpage

\Work{ % Autorka i tytuł dzieła
  Hanna Marcinkowska \\
  \textit{Wstęp do~teorii równań różniczkowych cząstkowych},
  \cite{MarcinkowskaWstepRownanRozniczkowychCzastkowych1986}}

\vspace{0em}


% ##################
\CenterBoldFont{Uwagi do~konkretnych stron}


\noindent
\Str{15} Założenie, że~$\vectbold$ jest wektorem jednostkowym, nie
jest w~ogóle wykorzystane w~dowodzie i~jako takie należy je odrzucić. Wzór
\begin{equation}
  \label{eq:MarcinkowskaWDTRRCz-01}
  t_{ n } =
  \sum_{ j = 1 }^{ n - 1 } ( D_{ j } f\left( g_{ 1 }( s_{ 0 } ), \ldots,
  g_{ n - 1 }( s_{ 0 } ) \right) g_{ j }'( s_{ 0 } ),
\end{equation}
wynika nie z~warunku $\absOne{ \vectbold }$, lecz z~tego,
iż $g_{ j }'( s_{ 0 } ) = t_{ j }$, $j = 1, \ldots, n - 1$ i~z~tego, że krzywa
\begin{equation}
  \label{eq:MarcinkowskaWDTRRCz-02}
  I \ni s \mapsto [ g_{ 1 }( s ), \ldots, g_{ n - 1 }( s ),
  f( g_{ 1 }( s ), \ldots, g_{ n - 1 }( s ) ) ] \in \Rbb^{ n }
\end{equation}
leży na~powierzchni $\Sigma$.

\vspace{\spaceFour}








% ##################
\CenterBoldFont{Błędy}


\begin{center}

  \begin{tabular}{|c|c|c|c|c|}
    \hline
    Strona & \multicolumn{2}{c|}{Wiersz} & Jest
                              & Powinno być \\ \cline{2-3}
    & Od góry & Od dołu & & \\
    \hline
    8   &  9 & & $| \vecabold | \;\; | \vecbbold |$
           & $| \vecabold | \, | \vecbbold |$ \\
    % & & & & \\
    % & & & & \\
    % & & & & \\
    % & & & & \\
    % & & & & \\
    \hline
  \end{tabular}

\end{center}

\vspace{\spaceTwo}








% ############################










% ############################
\newpage

\Work{ % Autor i tytuł dzieła
  P. Strzelecki \\
  \textit{Krótkie wprowadzenie do równań różniczkowych cząstkowych},
  \cite{StrzeleckiKrotkieWprowadzenieETC2006}}


% ##################
\CenterBoldFont{Uwagi}

Str. 117.???





% ##################
\CenterBoldFont{Błędy}


\begin{center}

  \begin{tabular}{|c|c|c|c|c|}
    \hline
    Strona & \multicolumn{2}{c|}{Wiersz} & Jest
                              & Powinno być \\ \cline{2-3}
    & Od góry & Od dołu & & \\
    \hline
    117 & & & $\exp\left( \frac{ 1 }{ 1 - | x |^{ 2 } } \right)$
           & $\exp\left( -\frac{ 1 }{ 1 - | x |^{ 2 } } \right)$ \\
           % & & & & \\
           % & & & & \\
           % & & & & \\
    \hline
  \end{tabular}

\end{center}

\vspace{\spaceTwo}






% ############################










% ############################
\newpage

\Work{ % Autorka i tytuł dzieła
  Zofia Szmydt \\
  \textit{Transformacja Fouriera i~równania różniczkowe liniowe},
  \cite{SzmydtTransformacjaFourieraIRownaniaRozniczkoweLiniowe1972}}


% ##################
\CenterBoldFont{Błędy}


\begin{center}

  \begin{tabular}{|c|c|c|c|c|}
    \hline
    Strona & \multicolumn{2}{c|}{Wiersz} & Jest
                              & Powinno być \\ \cline{2-3}
    & Od góry & Od dołu & & \\
    \hline
    13  & 13 & & ta & \emph{ta} \\
    % & & & & \\
    % & & & & \\
    32  & &  3 & $\boldsymbol{ \{ } q_{ k } \}$ & $\{ q_{ k } \}$ \\
    % & & & & \\
    % & & & & \\
    % & & & & \\
    % & & & & \\
    % & & & & \\
    % & & & & \\
    % & & & & \\
    % & & & & \\
    \hline
  \end{tabular}

\end{center}

\vspace{\spaceTwo}


\noindent
\StrWg{308}{6} \\
\Jest  Arytmetyka teoretyczna \\
\Powin \textit{Arytmetyka teoretyczna} \\




% ############################

































% ####################################################################
% ####################################################################
% Bibliografia

\bibliographystyle{plalpha}

\bibliography{MathComScienceBooks}{}





% ############################

% Koniec dokumentu
\end{document}
