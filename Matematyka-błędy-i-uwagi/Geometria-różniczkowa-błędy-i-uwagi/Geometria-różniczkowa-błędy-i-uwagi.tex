% ---------------------------------------------------------------------
% Podstawowe ustawienia i pakiety
% ---------------------------------------------------------------------
\RequirePackage[l2tabu, orthodox]{nag} % Wykrywa przestarzałe i niewłaściwe
% sposoby używania LaTeXa. Więcej jest w l2tabu English version.
\documentclass[a4paper,11pt]{article}
% {rozmiar papieru, rozmiar fontu}[klasa dokumentu]
\usepackage[MeX]{polski} % Polonizacja LaTeXa, bez niej będzie pracował
% w języku angielskim.
\usepackage[utf8]{inputenc} % Włączenie kodowania UTF-8, co daje dostęp
% do polskich znaków.
\usepackage{lmodern} % Wprowadza fonty Latin Modern.
\usepackage[T1]{fontenc} % Potrzebne do używania fontów Latin Modern.



% ------------------------------
% Podstawowe pakiety (niezwiązane z ustawieniami języka)
% ------------------------------
\usepackage{microtype} % Twierdzi, że poprawi rozmiar odstępów w tekście.
\usepackage{graphicx} % Wprowadza bardzo potrzebne komendy do wstawiania
% grafiki.
\usepackage{verbatim} % Poprawia otoczenie VERBATIME.
\usepackage{textcomp} % Dodaje takie symbole jak stopnie Celsiusa,
% wprowadzane bezpośrednio w tekście.
\usepackage{vmargin} % Pozwala na prostą kontrolę rozmiaru marginesów,
% za pomocą komend poniżej. Rozmiar odstępów jest mierzony w calach.
% ------------------------------
% MARGINS
% ------------------------------
\setmarginsrb
{ 0.7in} % left margin
{ 0.6in} % top margin
{ 0.7in} % right margin
{ 0.8in} % bottom margin
{  20pt} % head height
{0.25in} % head sep
{   9pt} % foot height
{ 0.3in} % foot sep



% ------------------------------
% Często przydatne pakiety
% ------------------------------
\usepackage{csquotes} % Pozwala w prosty sposób wstawiać cytaty do tekstu.
\usepackage{xcolor} % Pozwala używać kolorowych czcionek (zapewne dużo
% więcej, ale ja nie potrafię nic o tym powiedzieć).



% ------------------------------
% Pakiety do tekstów z nauk przyrodniczych
% ------------------------------
\let\lll\undefined % Amsmath gryzie się z językiem pakietami do języka
% polskiego, bo oba definiują komendę \lll. Aby rozwiązać ten problem
% oddefiniowuję tę komendę, ale może tym samym pozbywam się dużego Ł.
\usepackage[intlimits]{amsmath} % Podstawowe wsparcie od American
% Mathematical Society (w skrócie AMS)
\usepackage{amsfonts, amssymb, amscd, amsthm} % Dalsze wsparcie od AMS
% \usepackage{siunitx} % Do prostszego pisania jednostek fizycznych
\usepackage{upgreek} % Ładniejsze greckie litery
% Przykładowa składnia: pi = \uppi
\usepackage{slashed} % Pozwala w prosty sposób pisać slash Feynmana.
\usepackage{calrsfs} % Zmienia czcionkę kaligraficzną w \mathcal
% na ładniejszą. Może w innych miejscach robi to samo, ale o tym nic
% nie wiem.



% ##########
% Tworzenie otoczeń "Twierdzenie", "Definicja", "Lemat", etc.
\newtheorem{theorem}{Twierdzenie}  % Komenda wprowadzająca otoczenie
% „theorem” do pisania twierdzeń matematycznych
\newtheorem{definition}{Definicja}  % Analogicznie jak powyżej
\newtheorem{corollary}{Wniosek}



% ---------------------------------------
% Pakiety napisane przez użytkownika.
% Mają być w tym samym katalogu to ten plik .tex
% ---------------------------------------
\usepackage{latexgeneralcommands}
\usepackage{mathcommands}
% \usepackage{calculuscommands}
% \usepackage{SchwartzBooksCommands}  % Pakiet napisany m.in. dla tego pliku.



% ---------------------------------------------------------------------
% Dodatkowe ustawienia dla języka polskiego
% ---------------------------------------------------------------------
\renewcommand{\thesection}{\arabic{section}.}
% Kropki po numerach rozdziału (polski zwyczaj topograficzny)
\renewcommand{\thesubsection}{\thesection\arabic{subsection}}
% Brak kropki po numerach podrozdziału



% ------------------------------
% Ustawienia różnych parametrów tekstu
% ------------------------------
\renewcommand{\arraystretch}{1.2} % Ustawienie szerokości odstępów między
% wierszami w tabelach.



% ------------------------------
% Pakiet „hyperref”
% Polecano by umieszczać go na końcu preambuły.
% ------------------------------
\usepackage{hyperref} % Pozwala tworzyć hiperlinki i zamienia odwołania
% do bibliografii na hiperlinki.










% ---------------------------------------------------------------------
% Tytuł i autor dzieła
\title{Geometria różniczkowa \\
  Błędy i~uwagi}

\author{Kamil Ziemian}


% \date{}
% ---------------------------------------------------------------------










% ####################################################################
% Początek dokumentu
\begin{document}
% ####################################################################





% ######################################
\maketitle % Tytuł całego tekstu
% ######################################





% ############################
\Work{ % Autor i tytuł dzieła
  William M. Boothby \\
  „An~Introduction to~Differentiable Manifolds and~Riemannian
  Geometry”, \cite{BoothbyIntroductionToDifferentiableManifolds1986} }


% ##################
\CenterBoldFont{Uwagi}


\start \StrWd{233}{3} Zamiast $dy/dx$ powinno być $d g( x )/dx$.
To~mogłoby ograniczyć nieporozumienia związane z~tym wzorem.

\vspace{\spaceFour}





% ##################
\CenterBoldFont{Błędy}


\begin{center}

  \begin{tabular}{|c|c|c|c|c|}
    \hline
    & \multicolumn{2}{c|}{} & & \\
    Strona & \multicolumn{2}{c|}{Wiersz} & Jest
                              & Powinno być \\ \cline{2-3}
    & Od góry & Od dołu & & \\
    \hline
    233 &  6 & & $\{ \rho, \theta, \varphi )$ & $\{ ( \rho, \theta, \varphi )$ \\
    % & & & & \\
    % & & & & \\
    % & & & & \\
    % & & & & \\
    % & & & & \\
    \hline
  \end{tabular}

\end{center}

\vspace{\spaceTwo}
% ############################










% ############################
\Work{ % Autorzy i tytuł dzieła
  J. Gancarzewicz, B. Opozda \\
  „Wstęp do geometrii różniczkowej”,
  \cite{GancarzewiczOpozdaWstepDoGeometriiRozniczkowej2003} }


% ##################
\CenterBoldFont{Uwagi}


\start \Str{17} Nazwa \textbf{płaszczyzna styczna} pochodzi stąd,
że~zawiera ona wektor styczny do krzywej $\vectbold$, ponadto jeśli dana
krzywa bez punktów wyprostowania jest płaska, to zawiera~się właśnie
w~tej płaszczyźnie. \textbf{Płaszczyzna prostopadła} jest natomiast
prostopadła do~wektora stycznego do krzywej. Nie potrafię jednak
wyjaśnić skąd~się wzięła nazwa \textbf{płaszczyzna prostująca}.

\vspace{\spaceFour}



\start \Str{31} \textbf{Twierdzenie 3.7.} W~dowodzie faktu, że~z~równości
typu $\vectbold \cdot \vectbold' = 0$ wynik, iż~moduł $\vectbold$ jest stały,
oprócz wzorów (3.5) przyjmowano chyba milcząco, że~dla każdej wartości
parametru $s$ wektory $\vectbold$, $\vecnbold$ i~$\vecbbold$ są ortogonalne,
co jednak nie jest udowodnione. Z~dowodem relacji ortogonalności sprawa
wygląda chyba podobnie. \Dok

\vspace{\spaceFour}



\start \Str{35} W~żadnym z~podanych tu przykładów, nie udowodniono,
że~budowane mapy, jako odwzorowania,~są homomorfizmami między
odpowiednimi przestrzeniami topologicznymi. Pokazanie tego nie~może
być jednak trudne. \Dok

\vspace{\spaceFour}





% ##################
\CenterBoldFont{Błędy}


\begin{center}

  \begin{tabular}{|c|c|c|c|c|}
    \hline
    & \multicolumn{2}{c|}{} & & \\
    Strona & \multicolumn{2}{c|}{Wiersz} & Jest
                              & Powinno być \\ \cline{2-3}
    & Od góry & Od dołu & & \\
    \hline
    19  & 16 & & $\gamma'''( s_{ 0 } )$ & $\gamma^{ (4) }( s_{ 0 } )$ \\
    31  &  3 & & $x_{ o }$ & $x_{ 0 }$ \\
    37  & & 7 & $\left( \frac{ 4 r u_{ 1 } }{ \norm{ u }^{ 2 } }, \ldots,
                \frac{ 4 r u_{ n } }{ \norm{ u }^{ 2 } } \right)$
           & $\left( \frac{ 4 r^{ 2 } u_{ 1 } }{ \norm{ u }^{ 2 } },
             \ldots, \frac{ 4 r^{ 2 } u_{ n } }{ \norm{ u }^{ 2 } }
             \right)$ \\
    41  & & 14 & oraz$( 1, \bar{ z } )$ & oraz $( 1, \bar{ z } )$ \\
    44  & 17 & & $( ( \varphi( x ) )$ & $( \varphi( x ) )$ \\
    49  & &  4 & $\frac{ \partial ( \varphi^{ i } \circ \varphi ) }{ u^{ j } }$
           & $\frac{ \partial ( \varphi^{ i } \circ \varphi ) }{ \partial u^{ j } }$ \\
           % & & & & \\
           % & & & & \\
    \hline
  \end{tabular}

\end{center}


\noindent
\StrWd{40}{3} \\
\Jest
$( \frac { v_{ 1 } }{ u_{ i } }, \ldots, \frac{ v_{ i - 1 } }{ u_{ i } },
\frac{ v_{ i + 1 } }{ u_{ i } }, \ldots, \frac{ v_{ n + 1 } }{ u_{ i } } )$ \\
\Powin
$( \frac{ v_{ 1 } }{ v_{ i } }, \ldots, \frac{ v_{ i - 1 } }{ v_{ i } },
\frac{ v_{ i + 1 } }{ v_{ i } }, \ldots, \frac{ v_{ n + 1 } }{ v_{ i } } )$ \\

\vspace{\spaceTwo}
% ############################










% ############################
\Work{ % Autor i tytuł dzieła
  Bogusław Gdowski \\
  \textit{Elementy geometrii różniczkowej z~zadaniami},
  \cite{GdowskiElementGeometriiRozniczkowejZZadaniami1999}}


% ##################
\CenterBoldFont{Uwagi}


\start \Str{5} Ponieważ autor nie wyjaśnił, co~dokładnie rozumie przez
przestrzeń euklidesową $E_{ 3 }$ (że~jest ona trójwymiarowa, jest
tu~kwestią drugorzędną), nie jest od~razu jasne czym różni~się ona
od~zbioru wszystkich swoich wektorów $E_{ 3 }^{ \;* }$.

\vspace{\spaceFour}



\start \Str{7} Poprawniej i~lepiej tłumaczącym co~się naprawdę dzieje,
byłoby zapisywać równania parametryczne w~formie
$\boldsymbol{ x( u ) } = \boldsymbol{ u }$, analogicznie dla~innych zmiennych,
niż~$\boldsymbol{ x } = \boldsymbol{ u }$. Można to~jednak zrozumieć stosowanie takiej
konwencji jako kwestię uproszczenie notacji i~zwyczaju.

\vspace{\spaceFour}





% ##################
\CenterBoldFont{Błędy}


\begin{center}

  \begin{tabular}{|c|c|c|c|c|}
    \hline
    & \multicolumn{2}{c|}{} & & \\
    Strona & \multicolumn{2}{c|}{Wiersz} & Jest
                              & Powinno być \\ \cline{2-3}
    & Od góry & Od dołu & & \\
    \hline
    6   & 17 & & $X \in E_{ 3 }$ & $X \subset E_{ 3 }$ \\
    11  & 11 & & $\boldsymbol{u_{ 0 }}$ & $\boldsymbol{u_{ 0 }} \in \Rbb$ \\
    15  & &  3 & zbiorem & podzbiorem \\
    18  &  3 & & $\boldsymbol{u_{ 0 }} + + \boldsymbol{h_{ 0 }}$
           & $\boldsymbol{u_{ 0 }} + \boldsymbol{h_{ 0 }}$ \\
           % & & & & \\
           % & & & & \\
           % & & & & \\
           % & & & & \\
           % & & & & \\
    \hline
  \end{tabular}

\end{center}

\noindent
\StrWd{6}{6--5} \\
\Jest  funkcjami jednej zmiennej rzeczywistej \\
\Powin \textbf{\textit{funkcjami jednej zmiennej rzeczywistej}} \\

\vspace{\spaceTwo}
% ############################










% % ############################
% \Work{ % Autor i tytuł dzieła
%   Leszek M.~Sokołowski \\
%   „Elementy analizy tensorowej”,
%   \cite{SokolowskiElementyAnalizyTensorowej2010} }


% % ##################
% \CenterBoldFont{Uwagi}


% \start \StrWd{}{}

% \vspace{\spaceFour}





% % ##################
% \CenterBoldFont{Błędy}


% \begin{center}

%   \begin{tabular}{|c|c|c|c|c|}
%     \hline
%     & \multicolumn{2}{c|}{} & & \\
%     Strona & \multicolumn{2}{c|}{Wiersz} & Jest
%                               & Powinno być \\ \cline{2-3}
%     & Od góry & Od dołu & & \\
%     \hline
%     %     & & & & \\
%     %     & & & & \\
%     %     & & & & \\
%     %     & & & & \\
%     %     & & & & \\
%     \hline
%   \end{tabular}

% \end{center}

% \vspace{\spaceTwo}
% % ############################










% ############################
\newpage

\Work{ % Autor i tytuł dzieła
  Wojciech Wojtyński \\
  \textit{Grupy i~algebry Liego},
  \cite{WojtynskiGrupyIAlgebryLiego1986}}


% ##################
\CenterBoldFont{Uwagi}


\start W~książce do oznaczenia ciał używa się fontów pogrubionych:
$\mathbf{K}$, $\mathbf{R}$, $\mathbf{C}$, etc. W~tych notatkach będziemy
używali standardowych fontów „blackboard bold”: $\Kbb$, $\Rbb$, $\Cbb$, etc.
Wyjątkiem będzie sytuacja, gdy został użyty zły symbol, wtedy będziemy
dążyć, by wzór poprawiany bym maksymalnie podobny do oryginału w~książce.

Podobni będziemy postępować z~symbolami, które obecnie zapisuje się za
pomocą czcionki kaligraficznej. Będziemy więc w~tych notatkach oznaczać
klasę funkcji różniczkowalnych nie symbolem $C^{ k }$, lecz $\Ccal^{ k }$, etc.



% ##################
\CenterBoldFont{Uwagi do konkretnych stron}


\start \Str{13} W~tym miejscu powinniśmy zaznaczyć, że~dla prostoty zapisu
będziemy często opuszczać symbol działania dwuargumentowego i~zamiast pisać
$x \cdot y$, dla $x, y \in X$, będziemy pisali po prostu $x y$.

Reguły jaką stosował Wojtyński decydując~się w~danym miejscu zostawić, bądź
opuścić symbol $\cdot$ pozostają dla mnie zagadką. W~miarę jednak swoich
możliwości będę starał się wskazać w~tych notatkach, w~których miejscach
pisownię wzorów należałoby ujednolicić.

\vspace{\spaceFour}





\start \Str{14} Definicja zbioru $A \cdot B$ byłaby bardziej precyzyjna, gdy
została zapisana w~poniższej postaci.
\begin{equation}
  \label{eq:Wojtynski-01}
  A \cdot B := \{ x \in X : \exists\, y \in A, \exists\, z \in B,\; x = \mu( y, z ) \}.
\end{equation}

\vspace{\spaceFour}





\start \Str{19} Warto zaznaczyć, że~naturalne włożenie $\omega_{ 0 }$ w~algebrę
$A$, dla przypadku gdy $A$ jest algebrą z~jedynką, ma postać
\begin{equation}
  \label{eq:Wojtynski-02}
  \Kbb \ni \lambda \mapsto \omega_{ 0 }( \lambda ) = \lambda \, 1 \in A,
\end{equation}
gdzie $1$ jest elementem neutralnym algebry $A$. Gdy $A$ nie posiada jedynki
przyjmujemy
\begin{equation}
  \label{eq:Wojtynski-03}
  \Kbb \ni \lambda \mapsto \omega_{ 0 }( \lambda ) \equiv 0 \in A.
\end{equation}

\vspace{\spaceFour}





\start \Str{21} Po lewej stronie wzoru (1.6) mamy wyrażenie
$| x_{ i } \, y_{ i } |$, jednak bardziej zgodne ze wzorem (1.5) na stronie 20,
byłoby umieszczenie w~tym miejscu $| x_{ i } \, \bar{y}_{ i } |$. Oczywiście,
oba te wyrażenia są równe, więc nie chodzi o~poprawność podanego wzoru,
tylko o~elegancję matematyczną.

\vspace{\spaceFour}





\start \textbf{Str. 21, wiersze 12, 13.} Wiersze te zostały bardzo brzydko
sformatowane.

\vspace{\spaceFour}





\start \Str{22} We wzorze (1.7) jak zwykle musimy przyjąć, że~$\Vert h \Vert \neq 0$.

\vspace{\spaceFour}





\start \Str{25} Definicja 1.33 byłaby bardziej logiczna, gdyby najpierw
została podana definicja przestrzeni lokalnie łukowo spójnej, a~dopiero
potem lokalnie jednospójnej.

\vspace{\spaceFour}





\start \Str{31} Na początku tej strony czytamy „pierwsza podprzestrzeń ma
wymiar zespolony $( n^{ 2 } - n ) / 2$”. Ponieważ może zachodzić wątpliwość,
którą przestrzeń określamy jako „pierwszą”, zanotujemy, że~przez pierwszą
przestrzeń należy rozumieć przestrzeń macierzy z~zerowymi współczynnikami na
diagonali. Przez „drugą przestrzeń” należy rozumieć przestrzeń macierz
z~zerowymi współczynnikami poza diagonalą.

\vspace{\spaceFour}





\start \Str{31} Dowód tego, że~odwzorowanie
$\GL( n, \Kbb ) \ni A \mapsto \widetilde{A} \in \GL( n, \Kbb )$ jest izomorfizmem,
byłby prostszy, gdyby zostało jawnie powiedziane, że~odwzorowanie to jest
inwolucją.

\vspace{\spaceFour}





\start \Str{31} Aby udowodnić, że~dla macierzy $A \in \OStraight( n, \Rbb )$
zachodzi $| a_{ i, j } | \leq 1$, dla $i, j = 1, 2, \ldots, n$, należy w~równości
\begin{equation}
  \label{eq:Wojtynski-04}
  \sum_{ j = 1 }^{ n } a_{ j i } \, a_{ j k } = \delta_{ k }^{ i }, \quad
  i, k = 1, 2, \ldots, n,
\end{equation}
podstawić $i = k$, do prowadzi do
\begin{equation}
  \label{eq:Wojtynski-05}
  \sum_{ j = 1 }^{ n } a_{ j i } \, a_{ j i } =
  \sum_{ j = 1 }^{ n } ( a_{ j i } )^{ 2 } =
  ( a_{ 1 i } )^{ 2 } + ( a_{ 2 i } )^{ 2 } + \ldots + ( a_{ n i } )^{ 2 } =
  \delta^{ i }_{ i } = 1.
\end{equation}
Ponieważ wszystkie współczynniki $a_{ i j }$, gdzie $i, j = 1, 2, \ldots, n$, są
rzeczywiste, otrzymuje od razu nierówność
\begin{equation}
  \label{eq:Wojtynski-06}
  ( a_{ i j } )^{ 2 } \leq 1.
\end{equation}
Pierwiastkując obustronnie i~biorąc dodatnie wartości pierwiastka
otrzymujemy szukaną nierówność: $| a_{ i j } | \leq 1$.

Całe to rozumowanie jest bardzo proste, jednak uznaliśmy, iż warto je
zapisać w~tak szczegółowy sposób. Czyni to wykład prowadzony w~książce
trochę bardziej jasnym i~zupełnym.

\vspace{\spaceFour}





\start \Str{31} Czy warunek $A A^{ T } = I$ wynika z~warunku $A^{ T } A = I$,
gdy $A$ jest macierzą kwadratową wymiaru $n$?

\vspace{\spaceFour}





\start \Str{32} Bezpośrednie przekształcenie wzoru
$\bar{A} = \widetilde{A} = ( A^{ T } )^{ -1 }$ prowadzi do dwóch zależności:
$A^{ T } \bar{A} = I$, $\bar{A} A^{ T } = I$. Dopiero gdy dokonamy sprzężenia
zespolonego obu stron drugiej z~tych równości dostanie podany w~książce wzór
$A \bar{A}^{ T } = I$. Nie mogę mieć pewności, czemu autor zdecydował~się
zapisać te zależności w~takiej formie, mogę jedynie podejrzewać, że~chodziło
o~to by wyprowadzone z~nich wzory (2.9) przyjęły bardziej symetryczną postać.

Jeżeli teraz dokonamy obustronnego sprzężenia zespolonego równości
$A^{ T } \bar{A} = I$ to dostaniemy $\bar{A}^{ T } A = I$, co wraz
z~$A \bar{A}^{ T } = I$ prowadzi do znanego wniosku: dla macierzy unitarnej
$A$ zachodzi $\bar{A}^{ T } = A^{ -1 }$. Używając operacji sprzężenia
hermitowskiego, możemy tę równość zapisać w~bardzie popularnej formie
$A^{ * } = A^{ -1 }$.

\vspace{\spaceFour}





\start \StrWd{32}{13} Nie mam pojęcia o~co chodzi w~stwierdzeniu
„gdzie $n = \dim \Kbb^{ n }$”.

\vspace{\spaceFour}





\start \StrWd{33}{10} W~niewianie wyglądającym wzorze
$\left( \frac{ e_{ 1 } }{ \sqrt{ \alpha } }, \frac{ e_{ 2 } }{ \sqrt{ \alpha } } \right)$
ukryta jest trudność, wynikająca z~tego, że~pierwiastek kwadratowy na
płaszczyźnie zespolonej jest zawsze funkcją dwuznaczną\footnote{Pierwiastek
  z~zera to jak wiadomo wyjątek.}. Trzeba więc przyjąć jakąś konwencję,
która będzie nam mówiła, ile dokładnie wynosi $\sqrt{ \alpha }$, gdzie
$\alpha \in \Cbb$ i~możemy spokojnie kontynuować rozważania.

\vspace{\spaceFour}





\start \Str{34} W~lewej dolnej i~prawej górnej ćwiartce przedstawione tu
macierzy brakuje symbolu $0$. Kiedyś można by~się zastanowić nad
wygenerowanie w~\LaTeX{} macierzy, która lepiej ilustrowałaby problem.























% ##################
\newpage

\CenterBoldFont{Błędy}


\begin{center}

  \begin{tabular}{|c|c|c|c|c|}
    \hline
    & \multicolumn{2}{c|}{} & & \\
    Strona & \multicolumn{2}{c|}{Wiersz} & Jest
                              & Powinno być \\ \cline{2-3}
    & Od góry & Od dołu & & \\
    \hline
    5   & & 10 & klasa grup & klasa \\
    7   & &  4 & sumę & \textit{sumę} \\
    7   & &  1 & $M_{ 1 } \ldots M_{ k }$ & $M_{ 1 }, \ldots, M_{ k }$ \\
    8   &  3 & & $( m,\; n )$ & $( m, n )$ \\
    8   & 17 & & $n \in N$ & $m \in M$ \\
    8   & 17 & & $f( m ) \in M$ & $f( m ) \in N$ \\
    8   & 18 & & $f( n )$ & $f( m )$ \\
    9   & &  4 & $x_{ 1 },\;\; x_{ 2 } \in X$ & $x_{ 1 },\, x_{ 2 } \in X$ \\
    9   & &  4 & albo & lub \\
    10  &  4 & & $y_{ 0 } \succ x$($y_{ 0 } \prec x$)
           & $x \prec y_{ 0 }$ ($y_{ 0 } \prec x$) \\
    10  & 14 & & \textit{to istnieje w~$X$} & \textit{to w~$X$ istniej} \\
    10  & &  9 & $U \subset \Omega_{ x_{ 0 } }$ & $U \in \Omega_{ x_{ 0 } }$ \\
    11  & 10 & & $\rho( X, Y )$ & $\rho( x, y )$ \\
    11  & &  5 & zbioru & niepustego zbioru \\
    13  &  6 & & $y_{ \varphi } \in X_{ \varphi }$ & $x_{ \varphi } \in X_{ \varphi }$ \\
    14  &  1 & & $x_{ 1 },\;\; x_{ 2 } \in X$ & $x_{ 1 },\, x_{ 2 } \in X$ \\
    14  &  6 & & $x e_{ + }$ & $x \, e_{ + }$ \\
    14  & 18 & & $\mu\big( x, \mu( x, y ) \big)$
           & $\mu\big( x, \mu( y, z ) \big)$ \\
    14  & 21 & & $\mathbf{x} \in \mathbf{X}$ & $x \in X$ \\
    15  & &  5 & $\mu \omega( x_{ 1 }, y )$ & $\mu \, \omega( x_{ 2 }, y )$ \\
    17  &  7 & & $W$ modulo $V$, & $W$, \\
    17  &  7 & & $\tau( x_{ j }, y_{ k } )$
           & $\tau( x_{ j }, y_{ k } ) = \pi( x_{ j } \otimes y_{ k } )$ \\
    18  & &  1 & $X^{ k } \times X^{ m }$ & $X^{ \otimes k } \times X^{ \otimes m }$ \\
    18  & &  1 & $X^{ k } \otimes X^{ m }$ & $X^{ \otimes k } \otimes X^{ \otimes m }$ \\
    20  & 19 & & $\overline{ F( y, x ) }$ & $\overline{ F( y, x ) }$. \\
    20  & 20 & & (tj. & Tj. \\
    20  & 20 & & $\Rbb$)) & $\Rbb$) \\[0.3em]
    21  &  5 & & $\displaystyle
                 \frac{ \overline{ \langle x, y \rangle } }{ | \langle x, y \rangle | }$
           & $\displaystyle
             \frac{ \overline{ \langle x', y \rangle } }{ | \langle x', y \rangle | }$ \\
    21  &  5 & & $\langle x, y \rangle \neq 0$ & $\langle x', y \rangle \neq 0$ \\
    23  &  4 & & $\lambda ( A_{ 1 } x )$ & $\lambda \big( A_{ 1 }( x ) \big)$ \\
    24  & 15 & & $( a, b ) \in \Omega$ & $( a, b ) \in \Omega$, \\
    24  & & 12 & $\Ccal^{ 1 }$ & $\Ccal^{ k }$ \\
    25  &  9 & & $1 / 2 \leq t$ & $1 / 2 < t$ \\
    25  & 12 & & $i_{ x_{ 0 } }( t ) = 0$ & $i_{ x_{ 0 } }( t ) = x_{ 0 }$ \\
    26  & &  4 & 46 & 47 \\
    26  & &  1 & $x \in X$ & $a \in B$ \\
    \hline
  \end{tabular}





  \newpage

  \begin{tabular}{|c|c|c|c|c|}
    \hline
    & \multicolumn{2}{c|}{} & & \\
    Strona & \multicolumn{2}{c|}{Wiersz} & Jest
                              & Powinno być \\ \cline{2-3}
    & Od góry & Od dołu & & \\
    \hline
    27  &  2 & & $= d$ & $= u$ \\
    29  &  9 & & $\textrm{Gl}( n, \Cbb )$ & $\GL( n, \Cbb )$ \\
    29  & &  9 & $( A B )^{ T } = B^{ T } A^{ T }$
           & $( A \cdot B )^{ T } = B^{ T } \cdot A^{ T }$ \\
    29  & &  8 & $\overline{ A B }$ & $\overline{ A \cdot B }$ \\
    29  &  7 & & $( A B )^{ * }$ & $( A \cdot B )^{ * }$ \\
    30  & & 16 & $\oFrak( n, \Cbb )$ & $\oFrak_{ + }( n, \Cbb )$ \\
    30  & & 10 & $*$ -- operacją & $*$ operacją \\
    31  & 13 & & $\widetilde{ ( A B ) }$ & $\widetilde{ ( A \cdot B ) }$ \\
    32  & &  2 & (2.9) & (2.10) \\
    34  & &  6 & \textit{$n$-tą} & \textit{$N$-tą} \\
    %   & 14 & & \textit{to istnieje w~$X$} & \textit{to w~$X$ istniej} \\
    %   & &  9 & $U \subset \Omega_{ x_{ 0 } }$ & $U \in \Omega_{ x_{ 0 } }$ \\
    %   & 10 & & $\rho( X, Y )$ & $\rho( x, y )$ \\
    %   & &  5 & zbioru & niepustego zbioru \\
    %   &  6 & & $y_{ \varphi } \in X_{ \varphi }$ & $x_{ \varphi } \in X_{ \varphi }$ \\
    %   &  1 & & $x_{ 1 },\;\; x_{ 2 } \in X$ & $x_{ 1 },\, x_{ 2 } \in X$ \\
    %   &  6 & & $x e_{ + }$ & $x \, e_{ + }$ \\
    %   & 18 & & $\mu\big( x, \mu( x, y ) \big)$
    %        & $\mu\big( x, \mu( y, z ) \big)$ \\
    %   & 21 & & $\mathbf{x} \in \mathbf{X}$ & $x \in X$ \\
    %   & &  5 & $\mu \omega( x_{ 1 }, y )$ & $\mu \, \omega( x_{ 2 }, y )$ \\
    %   &  7 & & $W$ modulo $V$, & $W$, \\
    %   &  7 & & $\tau( x_{ j }, y_{ k } )$
    %        & $\tau( x_{ j }, y_{ k } ) = \pi( x_{ j } \otimes y_{ k } )$ \\
    %   & &  1 & $X^{ k } \times X^{ m }$ & $X^{ \otimes k } \times X^{ \otimes m }$ \\
    %   & &  1 & $X^{ k } \otimes X^{ m }$ & $X^{ \otimes k } \otimes X^{ \otimes m }$ \\
    %   & 19 & & $\overline{ F( y, x ) }$ & $\overline{ F( y, x ) }$. \\
    %   & 20 & & (tj. & Tj. \\
    %   & 20 & & $\Rbb$)) & $\Rbb$) \\[0.3em]
    %   &  5 & & $\displaystyle
    %              \frac{ \overline{ \langle x, y \rangle } }{ | \langle x, y \rangle | }$
    %        & $\displaystyle
    %          \frac{ \overline{ \langle x', y \rangle } }{ | \langle x', y \rangle | }$ \\
    %   &  5 & & $\langle x, y \rangle \neq 0$ & $\langle x', y \rangle \neq 0$ \\
    %   &  4 & & $\lambda ( A_{ 1 } x )$ & $\lambda \big( A_{ 1 }( x ) \big)$ \\
    %   & 15 & & $( a, b ) \in \Omega$ & $( a, b ) \in \Omega$, \\
    %   & & 12 & $\Ccal^{ 1 }$ & $\Ccal^{ k }$ \\
    %   &  9 & & $1 / 2 \leq t$ & $1 / 2 < t$ \\
    %   & 12 & & $i_{ x_{ 0 } }( t ) = 0$ & $i_{ x_{ 0 } }( t ) = x_{ 0 }$ \\
    %   &  8 & & $e_{ i } ( e_{ j }, e_{ k } )$
    %        & $e_{ i } \cdot ( e_{ j } \cdot e_{ k } )$ \\
    %   &  8 & &
    %        & $( e_{ i } \cdot e_{ j } ) \cdot e_{ k }$ \\
    %   & &  8 & $\bar{q} q$ & $\bar{q} \cdot q$ \\
    %   & &  8 & $e_{ i } ( e_{ j } \cdot e_{ k } )$
    % & $e_{ i } \cdot ( e_{ j } \cdot e_{ k } )$ \\
    37  &  8 & & $e_{ i } ( e_{ j }, e_{ k } )$
           & $e_{ i } \cdot ( e_{ j } \cdot e_{ k } )$ \\
    % & & & & \\
    % & & & & \\
    37  &  8 & &
           & $( e_{ i } \cdot e_{ j } ) \cdot e_{ k }$ \\
    37  & &  8 & $\bar{q} q$ & $\bar{q} \cdot q$ \\
    50  & &  8 & $e_{ i } ( e_{ j } \cdot e_{ k } )$
           & $e_{ i } \cdot ( e_{ j } \cdot e_{ k } )$ \\
& & & & \\
    % & & & & \\
    % & & & & \\
    % & & & & \\
    \hline
  \end{tabular}





  \newpage

  \begin{tabular}{|c|c|c|c|c|}
    \hline
    & \multicolumn{2}{c|}{} & & \\
    Strona & \multicolumn{2}{c|}{Wiersz} & Jest
                              & Powinno być \\ \cline{2-3}
    & Od góry & Od dołu & & \\
    \hline
    %    & & 10 & klasa grup & klasa \\
    %    & &  4 & sumę & \textit{sumę} \\
    %    & &  1 & $M_{ 1 } \ldots M_{ k }$ & $M_{ 1 }, \ldots, M_{ k }$ \\
    %    &  3 & & $( m,\; n )$ & $( m, n )$ \\
    %    & 17 & & $n \in N$ & $m \in M$ \\
    %    & 17 & & $f( m ) \in M$ & $f( m ) \in N$ \\
    %    & 18 & & $f( n )$ & $f( m )$ \\
    %    & &  4 & $x_{ 1 },\;\; x_{ 2 } \in X$ & $x_{ 1 },\, x_{ 2 } \in X$ \\
    %    & &  4 & albo & lub \\
    %   &  4 & & $y_{ 0 } \succ x$($y_{ 0 } \prec x$)
    %        & $x \prec y_{ 0 }$ ($y_{ 0 } \prec x$) \\
    %   & 14 & & \textit{to istnieje w~$X$} & \textit{to w~$X$ istniej} \\
    %   & &  9 & $U \subset \Omega_{ x_{ 0 } }$ & $U \in \Omega_{ x_{ 0 } }$ \\
    %   & 10 & & $\rho( X, Y )$ & $\rho( x, y )$ \\
    %   & &  5 & zbioru & niepustego zbioru \\
    %   &  6 & & $y_{ \varphi } \in X_{ \varphi }$ & $x_{ \varphi } \in X_{ \varphi }$ \\
    %   &  1 & & $x_{ 1 },\;\; x_{ 2 } \in X$ & $x_{ 1 },\, x_{ 2 } \in X$ \\
    %   &  6 & & $x e_{ + }$ & $x \, e_{ + }$ \\
    %   & 18 & & $\mu\big( x, \mu( x, y ) \big)$
    %        & $\mu\big( x, \mu( y, z ) \big)$ \\
    %   & 21 & & $\mathbf{x} \in \mathbf{X}$ & $x \in X$ \\
    %   & &  5 & $\mu \omega( x_{ 1 }, y )$ & $\mu \, \omega( x_{ 2 }, y )$ \\
    %   &  7 & & $W$ modulo $V$, & $W$, \\
    %   &  7 & & $\tau( x_{ j }, y_{ k } )$
    %        & $\tau( x_{ j }, y_{ k } ) = \pi( x_{ j } \otimes y_{ k } )$ \\
    99  &  2 & & $[ a, b ] \in \mathbf{K}$ & $[ a, b ] \in M$ \\
    99  &  4 & & $[ a, b ] \in \mathbf{K}$ & $[ a, b ] \in M$ \\
    99  &  4 & & $[ b, a ] \in \mathbf{K}$ & $[ b, a ] \in M$ \\
    99  & & 13 & $*$ & Działanie $*$ \\
    100 & &  9 & $\mathbf{K} \leq L_{ 2 }$ & $M \leq L_{ 2 }$ \\
    100 & &  9 & $f^{ -1 }( \mathbf{K} )$ & $f^{ - 1}( M )$ \\
    100 & &  6 & $a_{ i } = f^{ -1 }( b_{ i } )$
           & $a_{ i } \in f^{ -1 }( b_{ i } )$ \\
    100 & &  5 & $\mathbf{K}$ & $M$ \\
    100 & &  4 & $f^{ -1 }( K )$ & $f^{ -1 }( M )$ \\
    % & & & & \\
    % & & & & \\
    % & & & & \\
    % & & & & \\
    % & & & & \\
    % & & & & \\
    % & & & & \\
    \hline
  \end{tabular}

\end{center}


\noindent
\StrWg{18}{5} \\
\Jest  $\displaystyle \omega\Big( \big( \sum_{ i } x_{ i } \otimes y_{ i } ),
\sum_{ j } x_{ j } \otimes y_{ j } )$ \\
\Powin $\displaystyle \omega\Big( \sum_{ i } x_{ i } \otimes y_{ i },
\sum_{ j } v_{ j } \otimes w_{ j } \Big)$ \\
\StrWg{18}{5} \\
\Jest  $\displaystyle \mu( x_{ i }, x_{ j } ) \otimes \nu( y_{ i }, y_{ j } )$ \\
\Powin $\displaystyle \mu( x_{ i }, v_{ j } ) \otimes \nu( y_{ i }, w_{ j } )$ \\
\StrWg{19}{11} \\
\Jest  $\rho( x_{ 1 } ) \cdot \rho( x_{ 2 } ) \ldots \rho( x_{ n } )$ \\
\Powin $\rho( x_{ 1 } ) \cdot \rho( x_{ 2 } ) \cdot \ldots \cdot \rho( x_{ n } )$ \\
\StrWd{31}{4} \\
\Jest  $A \in \OStraight( n, \Rbb )$ \\
\Powin $A \in \OStraight( n, \Rbb )$ zachodzi \\
% \StrWd{}{} \\





\vspace{\spaceTwo}
% ############################










% ####################################################################
% ####################################################################
% Bibliografia
\bibliographystyle{plalpha}

\bibliography{MathComScienceBooks}{}





% ############################

% Koniec dokumentu
\end{document}
