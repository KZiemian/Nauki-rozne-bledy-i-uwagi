% --------------------------------------------------------------------
% Podstawowe ustawienia i pakiety
% --------------------------------------------------------------------
\RequirePackage[l2tabu, orthodox]{nag} % Wykrywa przestarzałe i niewłaściwe
% sposoby używania LaTeXa. Więcej jest w l2tabu English version.
\documentclass[a4paper,11pt]{article}
% {rozmiar papieru, rozmiar fontu}[klasa dokumentu]
\usepackage[MeX]{polski} % Polonizacja LaTeXa, bez niej będzie pracował
% w języku angielskim.
\usepackage[utf8]{inputenc} % Włączenie kodowania UTF-8, co daje dostęp
% do polskich znaków.
\usepackage{lmodern} % Wprowadza fonty Latin Modern.
\usepackage[T1]{fontenc} % Potrzebne do używania fontów Latin Modern.



% ----------------------------
% Podstawowe pakiety (niezwiązane z ustawieniami języka)
% ----------------------------
\usepackage{microtype} % Twierdzi, że poprawi rozmiar odstępów w tekście.
\usepackage{graphicx} % Wprowadza bardzo potrzebne komendy do wstawiania
% grafiki.
\usepackage{verbatim} % Poprawia otoczenie VERBATIME.
\usepackage{textcomp} % Dodaje takie symbole jak stopnie Celsiusa,
% wprowadzane bezpośrednio w tekście.
\usepackage{vmargin} % Pozwala na prostą kontrolę rozmiaru marginesów,
% za pomocą komend poniżej. Rozmiar odstępów jest mierzony w calach.
% ----------------------------
% MARGINS
% ----------------------------
\setmarginsrb
{ 0.7in} % left margin
{ 0.6in} % top margin
{ 0.7in} % right margin
{ 0.8in} % bottom margin
{  20pt} % head height
{0.25in} % head sep
{   9pt} % foot height
{ 0.3in} % foot sep



% ------------------------------
% Często przydatne pakiety
% ------------------------------
\usepackage{csquotes} % Pozwala w prosty sposób wstawiać cytaty do tekstu.
\usepackage{xcolor} % Pozwala używać kolorowych czcionek (zapewne dużo
% więcej, ale ja nie potrafię nic o tym powiedzieć).



% ------------------------------
% Pakiety do tekstów z nauk przyrodniczych
% ------------------------------
\let\lll\undefined % Amsmath gryzie się z językiem pakietami do języka
% polskiego, bo oba definiują komendę \lll. Aby rozwiązać ten problem
% oddefiniowuję tę komendę, ale może tym samym pozbywam się dużego Ł.
\usepackage[intlimits]{amsmath} % Podstawowe wsparcie od American
% Mathematical Society (w skrócie AMS)
\usepackage{amsfonts, amssymb, amscd, amsthm} % Dalsze wsparcie od AMS
% \usepackage{siunitx} % Do prostszego pisania jednostek fizycznych
\usepackage{upgreek} % Ładniejsze greckie litery
% Przykładowa składnia: pi = \uppi
\usepackage{slashed} % Pozwala w prosty sposób pisać slash Feynmana.
\usepackage{calrsfs} % Zmienia czcionkę kaligraficzną w \mathcal
% na ładniejszą. Może w innych miejscach robi to samo, ale o tym nic
% nie wiem.



% ##########
% Tworzenie otoczeń "Twierdzenie", "Definicja", "Lemat", etc.
\newtheorem{theorem}{Twierdzenie}  % Komenda wprowadzająca otoczenie
% „theorem” do pisania twierdzeń matematycznych
\newtheorem{definition}{Definicja}  % Analogicznie jak powyżej
\newtheorem{corollary}{Wniosek}



% ---------------------------------------
% Pakiety napisane przez użytkownika.
% Mają być w tym samym katalogu to ten plik .tex
% ---------------------------------------
\usepackage{latexgeneralcommands}
\usepackage{mathcommands}
% \usepackage{calculuscommands}
% \usepackage{SchwartzBooksCommands}  % Pakiet napisany m.in. dla tego pliku.



% --------------------------------------------------------------------
% Dodatkowe ustawienia dla języka polskiego
% --------------------------------------------------------------------
\renewcommand{\thesection}{\arabic{section}.}
% Kropki po numerach rozdziału (polski zwyczaj topograficzny)
\renewcommand{\thesubsection}{\thesection\arabic{subsection}}
% Brak kropki po numerach podrozdziału



% ------------------------------
% Ustawienia różnych parametrów tekstu
% ------------------------------
\renewcommand{\arraystretch}{1.2} % Ustawienie szerokości odstępów między
% wierszami w tabelach.



% ------------------------------
% Pakiet „hyperref”
% Polecano by umieszczać go na końcu preambuły.
% ------------------------------
\usepackage{hyperref} % Pozwala tworzyć hiperlinki i zamienia odwołania
% do bibliografii na hiperlinki.










% ---------------------------------------------------------------------
% Tytuł i autor dzieła
\title{Geometria różniczkowa \\
  Błędy i~uwagi}

\author{Kamil Ziemian}
% \date{}
% ---------------------------------------------------------------------










% ####################################################################
% Początek dokumentu
\begin{document}
% ####################################################################





% ######################################
\maketitle % Tytuł całego tekstu
% ######################################





% ############################
\Work{ % Autor i tytuł dzieła
  William M. Boothby \\
  „An~Introduction to~Differentiable Manifolds and~Riemannian
  Geometry”, \cite{BoothbyIntroductionToDifferentiableManifolds1986} }


% ##################
\CenterBoldFont{Uwagi}


\start \StrWd{233}{3} Zamiast $dy/dx$ powinno być $d g( x )/dx$.
To~mogłoby ograniczyć nieporozumienia związane z~tym wzorem.

\vspace{\spaceFour}





% ##################
\CenterBoldFont{Błędy}


\begin{center}

  \begin{tabular}{|c|c|c|c|c|}
    \hline
    & \multicolumn{2}{c|}{} & & \\
    Strona & \multicolumn{2}{c|}{Wiersz} & Jest
                              & Powinno być \\ \cline{2-3}
    & Od góry & Od dołu & & \\
    \hline
    233 &  6 & & $\{ \rho, \theta, \varphi )$ & $\{ ( \rho, \theta, \varphi )$ \\
    % & & & & \\
    % & & & & \\
    % & & & & \\
    % & & & & \\
    % & & & & \\
    \hline
  \end{tabular}

\end{center}

\vspace{\spaceTwo}
% ############################










% ############################
\Work{ % Autorzy i tytuł dzieła
  J. Gancarzewicz, B. Opozda \\
  „Wstęp do geometrii różniczkowej”,
  \cite{GancarzewiczOpozdaWstepDoGeometriiRozniczkowej2003} }


% ##################
\CenterBoldFont{Uwagi}


\start \Str{17} Nazwa \textbf{płaszczyzna styczna} pochodzi stąd,
że~zawiera ona wektor styczny do krzywej $\vectbold$, ponadto jeśli dana
krzywa bez punktów wyprostowania jest płaska, to zawiera~się właśnie
w~tej płaszczyźnie. \textbf{Płaszczyzna prostopadła} jest natomiast
prostopadła do~wektora stycznego do krzywej. Nie potrafię jednak
wyjaśnić skąd~się wzięła nazwa \textbf{płaszczyzna prostująca}.

\vspace{\spaceFour}



\start \Str{31} \textbf{Twierdzenie 3.7.} W~dowodzie faktu, że~z~równości
typu $\vectbold \cdot \vectbold' = 0$ wynik, iż~moduł $\vectbold$ jest stały,
oprócz wzorów (3.5) przyjmowano chyba milcząco, że~dla każdej wartości
parametru $s$ wektory $\vectbold$, $\vecnbold$ i~$\vecbbold$ są ortogonalne,
co jednak nie jest udowodnione. Z~dowodem relacji ortogonalności sprawa
wygląda chyba podobnie. \Dok

\vspace{\spaceFour}



\start \Str{35} W~żadnym z~podanych tu przykładów, nie udowodniono,
że~budowane mapy, jako odwzorowania,~są homomorfizmami między
odpowiednimi przestrzeniami topologicznymi. Pokazanie tego nie~może
być jednak trudne. \Dok

\vspace{\spaceFour}





% ##################
\CenterBoldFont{Błędy}


\begin{center}

  \begin{tabular}{|c|c|c|c|c|}
    \hline
    & \multicolumn{2}{c|}{} & & \\
    Strona & \multicolumn{2}{c|}{Wiersz} & Jest
                              & Powinno być \\ \cline{2-3}
    & Od góry & Od dołu & & \\
    \hline
    19  & 16 & & $\gamma'''( s_{ 0 } )$ & $\gamma^{ (4) }( s_{ 0 } )$ \\
    31  &  3 & & $x_{ o }$ & $x_{ 0 }$ \\
    37  & & 7 & $\left( \frac{ 4 r u_{ 1 } }{ \norm{ u }^{ 2 } }, \ldots,
                \frac{ 4 r u_{ n } }{ \norm{ u }^{ 2 } } \right)$
           & $\left( \frac{ 4 r^{ 2 } u_{ 1 } }{ \norm{ u }^{ 2 } },
             \ldots, \frac{ 4 r^{ 2 } u_{ n } }{ \norm{ u }^{ 2 } }
             \right)$ \\
    41  & & 14 & oraz$( 1, \bar{ z } )$ & oraz $( 1, \bar{ z } )$ \\
    44  & 17 & & $( ( \varphi( x ) )$ & $( \varphi( x ) )$ \\
    49  & &  4 & $\frac{ \partial ( \varphi^{ i } \circ \varphi ) }{ u^{ j } }$
           & $\frac{ \partial ( \varphi^{ i } \circ \varphi ) }{ \partial u^{ j } }$ \\
           % & & & & \\
           % & & & & \\
    \hline
  \end{tabular}

\end{center}


\noindent
\StrWd{40}{3} \\
\Jest
$( \frac { v_{ 1 } }{ u_{ i } }, \ldots, \frac{ v_{ i - 1 } }{ u_{ i } },
\frac{ v_{ i + 1 } }{ u_{ i } }, \ldots, \frac{ v_{ n + 1 } }{ u_{ i } } )$ \\
\Powin
$( \frac{ v_{ 1 } }{ v_{ i } }, \ldots, \frac{ v_{ i - 1 } }{ v_{ i } },
\frac{ v_{ i + 1 } }{ v_{ i } }, \ldots, \frac{ v_{ n + 1 } }{ v_{ i } } )$ \\

\vspace{\spaceTwo}
% ############################










% ############################
\Work{ % Autor i tytuł dzieła
  Bogusław Gdowski \\
  „Elementy geometrii różniczkowej z~zadaniami”,
  \cite{GdowskiElementGeometriiRozniczkowejZZadaniami1999} }


% ##################
\CenterBoldFont{Uwagi}


\start \Str{5} Ponieważ autor nie wyjaśnił, co~dokładnie rozumie przez
przestrzeń euklidesową $E_{ 3 }$ (że~jest ona trójwymiarowa, jest
tu~kwestią drugorzędną), nie jest od~razu jasne czym różni~się ona
od~zbioru wszystkich swoich wektorów $E_{ 3 }^{ \;* }$.

\vspace{\spaceFour}



\start \Str{7} Poprawniej i~lepiej tłumaczącym co~się naprawdę dzieje,
byłoby zapisywać równania parametryczne w~formie
$\boldsymbol{ x( u ) } = \boldsymbol{ u }$, analogicznie dla~innych zmiennych,
niż~$\boldsymbol{ x } = \boldsymbol{ u }$. Można to~jednak zrozumieć stosowanie takiej
konwencji jako kwestię uproszczenie notacji i~zwyczaju.

\vspace{\spaceFour}





% ##################
\CenterBoldFont{Błędy}


\begin{center}

  \begin{tabular}{|c|c|c|c|c|}
    \hline
    & \multicolumn{2}{c|}{} & & \\
    Strona & \multicolumn{2}{c|}{Wiersz} & Jest
                              & Powinno być \\ \cline{2-3}
    & Od góry & Od dołu & & \\
    \hline
    6   & 17 & & $X \in E_{ 3 }$ & $X \subset E_{ 3 }$ \\
    11  & 11 & & $\boldsymbol{u_{ 0 }}$ & $\boldsymbol{u_{ 0 }} \in \Rbb$ \\
    15  & &  3 & zbiorem & podzbiorem \\
    18  &  3 & & $\boldsymbol{u_{ 0 }} + + \boldsymbol{h_{ 0 }}$
           & $\boldsymbol{u_{ 0 }} + \boldsymbol{h_{ 0 }}$ \\
           % & & & & \\
           % & & & & \\
           % & & & & \\
           % & & & & \\
           % & & & & \\
    \hline
  \end{tabular}

\end{center}

\noindent
\StrWd{6}{6--5} \\
\Jest  funkcjami jednej zmiennej rzeczywistej \\
\Powin \textbf{\textit{funkcjami jednej zmiennej rzeczywistej}} \\

\vspace{\spaceTwo}
% ############################










% % ############################
% \Work{ % Autor i tytuł dzieła
%   Leszek M.~Sokołowski \\
%   „Elementy analizy tensorowej”,
%   \cite{SokolowskiElementyAnalizyTensorowej2010} }


% % ##################
% \CenterBoldFont{Uwagi}


% \start \StrWd{}{}

% \vspace{\spaceFour}





% % ##################
% \CenterBoldFont{Błędy}


% \begin{center}

%   \begin{tabular}{|c|c|c|c|c|}
%     \hline
%     & \multicolumn{2}{c|}{} & & \\
%     Strona & \multicolumn{2}{c|}{Wiersz} & Jest
%                               & Powinno być \\ \cline{2-3}
%     & Od góry & Od dołu & & \\
%     \hline
%     %     & & & & \\
%     %     & & & & \\
%     %     & & & & \\
%     %     & & & & \\
%     %     & & & & \\
%     \hline
%   \end{tabular}

% \end{center}

% \vspace{\spaceTwo}
% % ############################




% ############################
\Work{ % Autor i tytuł dzieła
  Wojciech Wojtyński \\
  „Grupy i~algebry Liego”, \cite{} }


% ##################
\CenterBoldFont{Uwagi}





% ##################
\CenterBoldFont{Błędy}


\begin{center}

  \begin{tabular}{|c|c|c|c|c|}
    \hline
    & \multicolumn{2}{c|}{} & & \\
    Strona & \multicolumn{2}{c|}{Wiersz} & Jest
                              & Powinno być \\ \cline{2-3}
    & Od góry & Od dołu & & \\
    \hline
    8   & & & & $( m, n )$ \\
    8   & & & & $m \in M$ jest $f( m ) \in N$ \\
    8   & & & & $M \ni m \to f( m ) \in N$ \\
    9   & & & & $x_{ 1 }, \, x_{ 2 } \in X$ \\
    25  & & & & $i_{ x_{ 0 } }( t ) = x_{ 0 }$ \\
    % & & & & \\
    % & & & & \\
    99  & & & & także $[ a, b ] \in M$ \\
    99  & & & & $N \leq L_{ 2 }$, \textit{to} $f^{ -1 }( N ) \leq L_{ 1 }$ \\
    100 & & & & $a_{ i } \in f^{ -1 }( b_{ i } )$ \\
    % & & & & \\
    % & & & & \\
    % & & & & \\
    % & & & & \\
    % & & & & \\
    % & & & & \\
    % & & & & \\
    % & & & & \\
    \hline
  \end{tabular}

\end{center}


\noindent
\StrWg{15}{} \\
\Jest \\
\Powin $\omega( \lambda x_{ 1 } + \mu x_{ 2 },\, y ) = \lambda \, \omega( x_{
  1 },\, y ) + \mu \, \omega( x_{ 2 },\, y )$ \\
\StrWg{20}{} \\
\Jest \\
\Powin tj.~$F$ jest formą dwuliniową na $X$ (jako przestrzeni nad~$\Rbb$) \\
\StrWg{30}{} \\
\Jest \\
\Powin $\mathfrak{so}_{ + }( n,\, \Cbb ) = \mathfrak{sl}( n,\, \Cbb ) \cap
\mathfrak{o}_{ + }( n,\, \Cbb ) \textrm{.}$ \\
\StrWg{99}{} \\
\Jest \\
\Powin $[ a, b ] \in M$ jest~równoważne warunkowi $[ b, a ] \in M$ \\

\vspace{\spaceTwo}
% ############################










% ####################################################################
% ####################################################################
% Bibliografia
\bibliographystyle{plalpha}

\bibliography{MathComScienceBooks}{}





% ############################

% Koniec dokumentu
\end{document}
