% ---------------------------------------------------------------------
% Podstawowe ustawienia i pakiety
% ---------------------------------------------------------------------
\RequirePackage[l2tabu, orthodox]{nag} % Wykrywa przestarzałe i niewłaściwe
% sposoby używania LaTeXa. Więcej jest w l2tabu English version.
\documentclass[a4paper,11pt]{article}
% {rozmiar papieru, rozmiar fontu}[klasa dokumentu]
\usepackage[MeX]{polski} % Polonizacja LaTeXa, bez niej będzie pracował
% w języku angielskim.
\usepackage[utf8]{inputenc} % Włączenie kodowania UTF-8, co daje dostęp
% do polskich znaków.
\usepackage{lmodern} % Wprowadza fonty Latin Modern.
\usepackage[T1]{fontenc} % Potrzebne do używania fontów Latin Modern.



% ------------------------------
% Podstawowe pakiety (niezwiązane z ustawieniami języka)
% ------------------------------
\usepackage{microtype} % Twierdzi, że poprawi rozmiar odstępów w tekście.
\usepackage{graphicx} % Wprowadza bardzo potrzebne komendy do wstawiania
% grafiki.
\usepackage{verbatim} % Poprawia otoczenie VERBATIME.
\usepackage{textcomp} % Dodaje takie symbole jak stopnie Celsiusa,
% wprowadzane bezpośrednio w tekście.
\usepackage{vmargin} % Pozwala na prostą kontrolę rozmiaru marginesów,
% za pomocą komend poniżej. Rozmiar odstępów jest mierzony w calach.
% ------------------------------
% MARGINS
% ------------------------------
\setmarginsrb
{ 0.7in} % left margin
{ 0.6in} % top margin
{ 0.7in} % right margin
{ 0.8in} % bottom margin
{  20pt} % head height
{0.25in} % head sep
{   9pt} % foot height
{ 0.3in} % foot sep



% ------------------------------
% Często przydatne pakiety
% ------------------------------
\usepackage{csquotes} % Pozwala w prosty sposób wstawiać cytaty do tekstu.
\usepackage{xcolor} % Pozwala używać kolorowych czcionek (zapewne dużo
% więcej, ale ja nie potrafię nic o tym powiedzieć).



% ------------------------------
% Pakiety do tekstów z nauk przyrodniczych
% ------------------------------
\let\lll\undefined % Amsmath gryzie się z językiem pakietami do języka
% polskiego, bo oba definiują komendę \lll. Aby rozwiązać ten problem
% oddefiniowuję tę komendę, ale może tym samym pozbywam się dużego Ł.
\usepackage[intlimits]{amsmath} % Podstawowe wsparcie od American
% Mathematical Society (w skrócie AMS)
\usepackage{amsfonts, amssymb, amscd, amsthm} % Dalsze wsparcie od AMS
% \usepackage{siunitx} % Do prostszego pisania jednostek fizycznych
\usepackage{upgreek} % Ładniejsze greckie litery
% Przykładowa składnia: pi = \uppi
\usepackage{slashed} % Pozwala w prosty sposób pisać slash Feynmana.
\usepackage{calrsfs} % Zmienia czcionkę kaligraficzną w \mathcal
% na ładniejszą. Może w innych miejscach robi to samo, ale o tym nic
% nie wiem.



% ##########
% Tworzenie otoczeń "Twierdzenie", "Definicja", "Lemat", etc.
\newtheorem{theorem}{Twierdzenie}  % Komenda wprowadzająca otoczenie
% „theorem” do pisania twierdzeń matematycznych
\newtheorem{definition}{Definicja}  % Analogicznie jak powyżej
\newtheorem{corollary}{Wniosek}



% ---------------------------------------
% Pakiety napisane przez użytkownika.
% Mają być w tym samym katalogu to ten plik .tex
% ---------------------------------------
\usepackage{latexgeneralcommands}
\usepackage{mathcommands}
% \usepackage{calculuscommands}
% \usepackage{SchwartzBooksCommands}  % Pakiet napisany m.in. dla tego pliku.



% ---------------------------------------------------------------------
% Dodatkowe ustawienia dla języka polskiego
% ---------------------------------------------------------------------
\renewcommand{\thesection}{\arabic{section}.}
% Kropki po numerach rozdziału (polski zwyczaj topograficzny)
\renewcommand{\thesubsection}{\thesection\arabic{subsection}}
% Brak kropki po numerach podrozdziału



% ------------------------------
% Ustawienia różnych parametrów tekstu
% ------------------------------
\renewcommand{\arraystretch}{1.2} % Ustawienie szerokości odstępów między
% wierszami w tabelach.



% ------------------------------
% Pakiet „hyperref”
% Polecano by umieszczać go na końcu preambuły.
% ------------------------------
\usepackage{hyperref} % Pozwala tworzyć hiperlinki i zamienia odwołania
% do bibliografii na hiperlinki.










% ---------------------------------------------------------------------
% Tytuł i autor dzieła
\title{Geometria różniczkowa \\
  Błędy i~uwagi}

\author{Kamil Ziemian}


% \date{}
% ---------------------------------------------------------------------










% ####################################################################
% Początek dokumentu
\begin{document}
% ####################################################################





% ######################################
\maketitle % Tytuł całego tekstu
% ######################################





% ############################
\Work{ % Autor i tytuł dzieła
  William M. Boothby \\
  „An~Introduction to~Differentiable Manifolds and~Riemannian
  Geometry”, \cite{BoothbyIntroductionToDifferentiableManifolds1986}}


% ##################
\CenterBoldFont{Uwagi do konkretnych stron}


\start \StrWd{233}{3} Zamiast $dy/dx$ powinno być $d g( x )/dx$.
To~mogłoby ograniczyć nieporozumienia związane z~tym wzorem.

\vspace{\spaceFour}





% ##################
\CenterBoldFont{Błędy}


\begin{center}

  \begin{tabular}{|c|c|c|c|c|}
    \hline
    & \multicolumn{2}{c|}{} & & \\
    Strona & \multicolumn{2}{c|}{Wiersz} & Jest
                              & Powinno być \\ \cline{2-3}
    & Od góry & Od dołu & & \\
    \hline
    233 &  6 & & $\{ \rho, \theta, \varphi )$ & $\{ ( \rho, \theta, \varphi )$ \\
    % & & & & \\
    % & & & & \\
    % & & & & \\
    % & & & & \\
    % & & & & \\
    \hline
  \end{tabular}

\end{center}

\vspace{\spaceTwo}
% ############################










% ############################
\Work{ % Autorzy i tytuł dzieła
  J. Gancarzewicz, B. Opozda \\
  „Wstęp do geometrii różniczkowej”,
  \cite{GancarzewiczOpozdaWstepDoGeometriiRozniczkowej2003}}


% ##################
\CenterBoldFont{Uwagi}


\start \Str{17} Nazwa \textbf{płaszczyzna styczna} pochodzi stąd,
że~zawiera ona wektor styczny do krzywej $\vectbold$, ponadto jeśli dana
krzywa bez punktów wyprostowania jest płaska, to zawiera~się właśnie
w~tej płaszczyźnie. \textbf{Płaszczyzna prostopadła} jest natomiast
prostopadła do~wektora stycznego do krzywej. Nie potrafię jednak
wyjaśnić skąd~się wzięła nazwa \textbf{płaszczyzna prostująca}.

\vspace{\spaceFour}





\start \Str{31} \textbf{Twierdzenie 3.7.} W~dowodzie faktu, że~z~równości
typu $\vectbold \cdot \vectbold' = 0$ wynik, iż~moduł $\vectbold$ jest stały,
oprócz wzorów (3.5) przyjmowano chyba milcząco, że~dla każdej wartości
parametru $s$ wektory $\vectbold$, $\vecnbold$ i~$\vecbbold$ są ortogonalne,
co jednak nie jest udowodnione. Z~dowodem relacji ortogonalności sprawa
wygląda chyba podobnie. \Dok

\vspace{\spaceFour}





\start \Str{35} W~żadnym z~podanych tu przykładów, nie udowodniono,
że~budowane mapy, jako odwzorowania,~są homomorfizmami między
odpowiednimi przestrzeniami topologicznymi. Pokazanie tego nie~może
być jednak trudne. \Dok

\vspace{\spaceFour}





% ##################
\CenterBoldFont{Błędy}


\begin{center}

  \begin{tabular}{|c|c|c|c|c|}
    \hline
    & \multicolumn{2}{c|}{} & & \\
    Strona & \multicolumn{2}{c|}{Wiersz} & Jest
                              & Powinno być \\ \cline{2-3}
    & Od góry & Od dołu & & \\
    \hline
    19  & 16 & & $\gamma'''( s_{ 0 } )$ & $\gamma^{ (4) }( s_{ 0 } )$ \\
    31  &  3 & & $x_{ o }$ & $x_{ 0 }$ \\
    37  & & 7 & $\left( \frac{ 4 r u_{ 1 } }{ \norm{ u }^{ 2 } }, \ldots,
                \frac{ 4 r u_{ n } }{ \norm{ u }^{ 2 } } \right)$
           & $\left( \frac{ 4 r^{ 2 } u_{ 1 } }{ \norm{ u }^{ 2 } },
             \ldots, \frac{ 4 r^{ 2 } u_{ n } }{ \norm{ u }^{ 2 } }
             \right)$ \\
    41  & & 14 & oraz$( 1, \bar{ z } )$ & oraz $( 1, \bar{ z } )$ \\
    44  & 17 & & $( ( \varphi( x ) )$ & $( \varphi( x ) )$ \\
    49  & &  4 & $\frac{ \partial ( \varphi^{ i } \circ \varphi ) }{ u^{ j } }$
           & $\frac{ \partial ( \varphi^{ i } \circ \varphi ) }{ \partial u^{ j } }$ \\
           % & & & & \\
           % & & & & \\
    \hline
  \end{tabular}

\end{center}

\vspace{\spaceTwo}


\noindent
\StrWd{40}{3} \\
\Jest
$( \frac { v_{ 1 } }{ u_{ i } }, \ldots, \frac{ v_{ i - 1 } }{ u_{ i } },
\frac{ v_{ i + 1 } }{ u_{ i } }, \ldots, \frac{ v_{ n + 1 } }{ u_{ i } } )$ \\
\Powin
$( \frac{ v_{ 1 } }{ v_{ i } }, \ldots, \frac{ v_{ i - 1 } }{ v_{ i } },
\frac{ v_{ i + 1 } }{ v_{ i } }, \ldots, \frac{ v_{ n + 1 } }{ v_{ i } } )$ \\


% ############################










% ############################
\Work{ % Autor i tytuł dzieła
  Bogusław Gdowski \\
  \textit{Elementy geometrii różniczkowej z~zadaniami},
  \cite{GdowskiElementGeometriiRozniczkowejZZadaniami1999}}


% ##################
\CenterBoldFont{Uwagi}


\start \Str{5} Ponieważ autor nie wyjaśnił, co~dokładnie rozumie przez
przestrzeń euklidesową $E_{ 3 }$ (że~jest ona trójwymiarowa, jest
tu~kwestią drugorzędną), nie jest od~razu jasne czym różni~się ona
od~zbioru wszystkich swoich wektorów $E_{ 3 }^{ \;* }$.

\vspace{\spaceFour}





\start \Str{7} Poprawniej i~lepiej tłumaczącym co~się naprawdę dzieje,
byłoby zapisywać równania parametryczne w~formie
$\boldsymbol{ x( u ) } = \boldsymbol{ u }$, analogicznie dla~innych
zmiennych, niż~$\boldsymbol{ x } = \boldsymbol{ u }$. Można to~jednak
zrozumieć stosowanie takiej konwencji jako kwestię uproszczenie notacji
i~zwyczaju.

\vspace{\spaceFour}





% ##################
\CenterBoldFont{Błędy}


\begin{center}

  \begin{tabular}{|c|c|c|c|c|}
    \hline
    & \multicolumn{2}{c|}{} & & \\
    Strona & \multicolumn{2}{c|}{Wiersz} & Jest
                              & Powinno być \\ \cline{2-3}
    & Od góry & Od dołu & & \\
    \hline
    6   & 17 & & $X \in E_{ 3 }$ & $X \subset E_{ 3 }$ \\
    11  & 11 & & $\boldsymbol{u_{ 0 }}$ & $\boldsymbol{u_{ 0 }} \in \Rbb$ \\
    15  & &  3 & zbiorem & podzbiorem \\
    18  &  3 & & $\boldsymbol{u_{ 0 }} + + \boldsymbol{h_{ 0 }}$
           & $\boldsymbol{u_{ 0 }} + \boldsymbol{h_{ 0 }}$ \\
           % & & & & \\
           % & & & & \\
           % & & & & \\
           % & & & & \\
           % & & & & \\
    \hline
  \end{tabular}

\end{center}

\noindent
\StrWd{6}{6--5} \\
\Jest  funkcjami jednej zmiennej rzeczywistej \\
\Powin \textbf{\textit{funkcjami jednej zmiennej rzeczywistej}} \\

\vspace{\spaceTwo}
% ############################










% % ############################
% \Work{ % Autor i tytuł dzieła
%   Leszek M.~Sokołowski \\
%   „Elementy analizy tensorowej”,
%   \cite{SokolowskiElementyAnalizyTensorowej2010} }


% % ##################
% \CenterBoldFont{Uwagi}


% \start \StrWd{}{}

% \vspace{\spaceFour}





% % ##################
% \CenterBoldFont{Błędy}


% \begin{center}

%   \begin{tabular}{|c|c|c|c|c|}
%     \hline
%     & \multicolumn{2}{c|}{} & & \\
%     Strona & \multicolumn{2}{c|}{Wiersz} & Jest
%                               & Powinno być \\ \cline{2-3}
%     & Od góry & Od dołu & & \\
%     \hline
%     %     & & & & \\
%     %     & & & & \\
%     %     & & & & \\
%     %     & & & & \\
%     %     & & & & \\
%     \hline
%   \end{tabular}

% \end{center}

% \vspace{\spaceTwo}
% % ############################










% ############################
\newpage

\Work{ % Autor i tytuł dzieła
  Wojciech Wojtyński \\
  \textit{Grupy i~algebry Liego},
  \cite{WojtynskiGrupyIAlgebryLiego1986}}


% ##################
\CenterBoldFont{Uwagi}


\start W~książce do oznaczenia ciał używa się fontów pogrubionych:
$\mathbf{K}$, $\mathbf{R}$, $\mathbf{C}$, etc. W~tych notatkach będziemy
używali standardowych fontów „blackboard bold”: $\Kbb$, $\Rbb$, $\Cbb$, etc.
Wyjątkiem będzie sytuacja, gdy został użyty zły symbol, wtedy będziemy
dążyć, by wzór poprawiany bym maksymalnie podobny do oryginału w~książce.

Podobni będziemy postępować z~symbolami, które obecnie zapisuje się za
pomocą czcionki kaligraficznej. Będziemy więc w~tych notatkach oznaczać
klasę funkcji różniczkowalnych nie symbolem $C^{ k }$, lecz $\Ccal^{ k }$, etc.





% ##################
\CenterBoldFont{Uwagi do konkretnych stron}


\start \Str{13} W~tym miejscu powinniśmy zaznaczyć, że~dla prostoty zapisu
będziemy często opuszczać symbol działania dwuargumentowego i~zamiast pisać
$x \cdot y$, dla $x, y \in X$, będziemy pisali po prostu $x y$.

Reguły jaką stosował Wojtyński decydując~się w~danym miejscu zostawić, bądź
opuścić symbol $\cdot$ pozostają dla mnie zagadką. W~miarę jednak swoich
możliwości będę starał się wskazać w~tych notatkach, w~których miejscach
pisownię wzorów należałoby ujednolicić.

\vspace{\spaceFour}





\start \Str{13} Definicja zbiorów $V$ stanowiących bazę otoczeń punktu
$( x_{ \varphi } )$ jest trochę myląca na pierwszy rzut oka, bo mówi tylko
o~zachowaniu tylko skończonej ilości elementów składowych punktu
$( y_{ \phi } )$. Wstałoby ją opisać w~sposób bardziej przejrzysty, w~tym
jednak momencie brakuje mi do tego wiedzy z~topologii.

\vspace{\spaceFour}





\start \Str{14} Definicja zbioru $A \cdot B$ byłaby bardziej precyzyjna, gdy
została zapisana w~poniższej postaci.
\begin{equation}
  \label{eq:Wojtynski-01}
  A \cdot B := \{ x \in X : \exists\, y \in A, \exists\, z \in B,\; x = \mu( y, z ) \}.
\end{equation}

\vspace{\spaceFour}





\start \Str{19} Warto zaznaczyć, że~naturalne włożenie ciała skalarów
$\Kbb$ w~algebrę $A$, które oznaczamy $\omega_{ 0 }$, dla przypadku gdy $A$ jest
algebrą z~jedynką, ma postać
\begin{equation}
  \label{eq:Wojtynski-02}
  \Kbb \ni \lambda \mapsto \omega_{ 0 }( \lambda ) = \lambda \, 1 \in A,
\end{equation}
gdzie $1$ jest elementem neutralnym algebry $A$. Gdy $A$ nie posiada jedynki
przyjmujemy
\begin{equation}
  \label{eq:Wojtynski-03}
  \Kbb \ni \lambda \mapsto \omega_{ 0 }( \lambda ) \equiv 0 \in A.
\end{equation}

\vspace{\spaceFour}





\start \Str{21} Po lewej stronie wzoru (1.6) mamy wyrażenie
$| x_{ i } \, y_{ i } |$, jednak bardziej zgodne ze wzorem (1.5) na stronie 20,
byłoby umieszczenie w~tym miejscu $| x_{ i } \, \bar{y}_{ i } |$. Oczywiście,
oba te wyrażenia są równe, więc nie chodzi o~poprawność podanego wzoru,
tylko o~elegancję matematyczną.

\vspace{\spaceFour}





\start \textbf{Str. 21, wiersze 12, 13.} Wiersze te zostały bardzo brzydko
sformatowane.

\vspace{\spaceFour}





\start \Str{22} We wzorze (1.7) jak zwykle musimy przyjąć, że~$\Vert h \Vert \neq 0$.

\vspace{\spaceFour}





\start \Str{25} Definicja 1.33 byłaby bardziej logiczna, gdyby najpierw
została podana definicja przestrzeni lokalnie łukowo spójnej, a~dopiero
potem lokalnie jednospójnej.

\vspace{\spaceFour}





\start \Str{31} Na początku tej strony czytamy „pierwsza podprzestrzeń ma
wymiar zespolony $( n^{ 2 } - n ) / 2$”. Ponieważ może zachodzić wątpliwość,
którą przestrzeń określamy jako „pierwszą”, zanotujemy, że~przez pierwszą
przestrzeń należy rozumieć przestrzeń macierzy z~zerowymi współczynnikami na
diagonali. Przez „drugą przestrzeń” należy rozumieć przestrzeń macierz
z~zerowymi współczynnikami poza diagonalą.

\vspace{\spaceFour}





\start \Str{31} Dowód tego, że~odwzorowanie
$\GL( n, \Kbb ) \ni A \mapsto \widetilde{A} \in \GL( n, \Kbb )$ jest izomorfizmem,
byłby prostszy, gdyby zostało jawnie powiedziane, że~odwzorowanie to jest
inwolucją.

\vspace{\spaceFour}





\start \Str{31} Aby udowodnić, że~dla macierzy
$A \in \OStraightBold( n, \Rbb )$ zachodzi $| a_{ i, j } | \leq 1$, dla
$i, j = 1, 2, \ldots, n$, należy w~równości
\begin{equation}
  \label{eq:Wojtynski-04}
  \sum_{ j = 1 }^{ n } a_{ j i } \, a_{ j k } = \delta_{ k }^{ i }, \quad
  i, k = 1, 2, \ldots, n,
\end{equation}
podstawić $i = k$, do prowadzi do
\begin{equation}
  \label{eq:Wojtynski-05}
  \sum_{ j = 1 }^{ n } a_{ j i } \, a_{ j i } =
  \sum_{ j = 1 }^{ n } ( a_{ j i } )^{ 2 } =
  ( a_{ 1 i } )^{ 2 } + ( a_{ 2 i } )^{ 2 } + \ldots + ( a_{ n i } )^{ 2 } =
  \delta^{ i }_{ i } = 1.
\end{equation}
Ponieważ wszystkie współczynniki $a_{ i j }$, gdzie $i, j = 1, 2, \ldots, n$, są
rzeczywiste, otrzymuje od razu nierówność
\begin{equation}
  \label{eq:Wojtynski-06}
  ( a_{ i j } )^{ 2 } \leq 1.
\end{equation}
Pierwiastkując obustronnie i~biorąc dodatnie wartości pierwiastka
otrzymujemy szukaną nierówność: $| a_{ i j } | \leq 1$.

Całe to rozumowanie jest bardzo proste, jednak uznaliśmy, iż warto je
zapisać w~tak szczegółowy sposób. Czyni to wykład prowadzony w~książce
trochę bardziej jasnym i~zupełnym.

\vspace{\spaceFour}





\start \Str{31} Czy warunek $A A^{ T } = \matUnit$ wynika z~warunku
$A^{ T } A = \matUnit$, gdy $A$ jest macierzą kwadratową wymiaru $n$?

\vspace{\spaceFour}





\start \Str{32} Bezpośrednie przekształcenie wzoru
$\bar{A} = \widetilde{A} = ( A^{ T } )^{ -1 }$ prowadzi do dwóch zależności:
$A^{ T } \bar{A} = \matUnit$, $\bar{A} A^{ T } = \matUnit$. Dopiero gdy
dokonamy sprzężenia zespolonego obu stron drugiej z~tych równości dostanie
podany w~książce wzór $A \bar{A}^{ T } = \matUnit$. Nie mogę mieć pewności,
czemu autor zdecydował~się zapisać te zależności w~takiej formie, mogę
jedynie podejrzewać, że~chodziło o~to by wyprowadzone z~nich wzory (2.9)
przyjęły bardziej symetryczną postać.

Jeżeli teraz dokonamy obustronnego sprzężenia zespolonego równości
$A^{ T } \bar{A} = \matUnit$ to dostaniemy $\bar{A}^{ T } A = \matUnit$, co
wraz z~$A \bar{A}^{ T } = \matUnit$ prowadzi do znanego wniosku: dla
macierzy unitarnej $A$ zachodzi $\bar{A}^{ T } = A^{ -1 }$. Używając
operacji sprzężenia hermitowskiego, możemy tę równość zapisać w~bardzie
popularnej formie $A^{ * } = A^{ -1 }$.

\vspace{\spaceFour}





\start \Str{32} W~tym miejscu warto się zatrzymać nad wzorem
\begin{equation}
  \label{eq:Wojtynski-07}
  \omega( x, y ) = x^{ T } \Omega \, y,
\end{equation}
który rozpatrywać będziemy tylko w~kontekście przestrzeni $\Cbb^{ n }$,
zwracając tylko uwagę, że~przedstawione tu rozumowanie można uogólnić
na~dowolną skończenie wymiarową przestrzeń wektorową.

Po pierwsze zwróćmy uwagę, że~$x$ po lewej stronie tego wzoru oznacza coś
innego, niż po prawej. Po lewej $x$ oznacza ciąg (kolumnę) liczb $n$ liczb
zespolonych, podczas gdy po prawej stronie oznacza on kolumnę, zapewne inną
od poprzedniej, liczb zespolonych daną zależnościami
\begin{equation}
  \label{eq:Wojtynski-08}
  x =
  \begin{bmatrix}
    x_{ 1 } \\
    x_{ 2 } \\
    \vdots \\
    x_{ n }
  \end{bmatrix}, \quad
  x = \sum_{ i = 1 }^{ n } x_{ i } \, e_{ i },
\end{equation}
gdzie $e_{ 1 }, e_{ 2 }, \ldots, e_{ n }$ jest pewną bazą przestrzeni $\Cbb^{ n }$.
Oznaczenia te są nieporęczne, czego świadectwem jest to, iż~we wzorze
\eqref{eq:Wojtynski-08} ponownie ten sam symbol „$x$” oznacza dwa różne
byty matematyczne.

By uniknąć dalszego zamieszania, wektory należące do
$\Cbb^{ n }$ będziemy oznaczać czcionką pogrubioną, a~kolumny ich
współrzędnych w~pewnej bazie, czcionką prostą. Mamy więc
$\vecxbold, \vecybold, \vecebold_{ 1 }, \vecebold_{ 2 }, \ldots, \vecebold_{ n }
\in \Cbb$, oraz
\begin{equation}
  \label{eq:Wojtynski-09}
  x =
  \begin{bmatrix}
    x_{ 1 } \\
    x_{ 2 } \\
    \vdots \\
    x_{ n }
  \end{bmatrix}, \quad
  \vecxbold = \sum_{ i = 1 }^{ n } x_{ i } \vecebold_{ i }.
\end{equation}
Wzór \eqref{eq:Wojtynski-07} zapiszemy teraz jako
\begin{equation}
  \label{eq:Wojtynski-10}
  \omega( \vecxbold, \vecybold ) = x^{ T } \Omega \, y.
\end{equation}

Weźmy teraz $\vecxbold = \vecebold_{ i }$, $\vecybold = \vecebold_{ j }$.
Z~wzoru \eqref{eq:Wojtynski-09} wynika teraz, że~$x_{ l } = \delta^{ l }_{ i }$,
$y_{ m } = \delta^{ m }_{ j }$ (konwencja jest wprowadzona na stronie~33 książki
Wojtyńskiego) i~tym samym
\begin{equation}
  \label{eq:Wojtynski-11}
  \omega( \vecebold_{ i }, \vecebold_{ j } ) = x^{ T } \Omega \, y =
  \sum_{ l = 1 }^{ n } \sum_{ m = 1 }^{ n } x_{ l } \, \Omega_{ l m } \, y_{ m } =
  \sum_{ l = 1 }^{ n } \sum_{ m = 1 }^{ n } \delta^{ i }_{ l } \, \Omega_{ l m } \, \delta^{ j }_{ m } =
  \Omega_{ i j },
\end{equation}
gdzie $\Omega_{ l m }$ oznacza wyraz macierzy $\Omega$ znajdujący~się na przecięciu
$l$-tego wiersza i~$m$-tej kolumny. Tym samym otrzymaliśmy bardzo przydatny
wzór
\begin{equation}
  \label{eq:Wojtynski-12}
  \Omega_{ i j } = \omega( \vecebold_{ i }, \vecebold_{ j } ).
\end{equation}

W~książce Wojtyńskiego i~w~tych notatkach, przyjęliśmy następującą linię
rozumowanie. Przyjmujemy, iż~istnieje macierz $\Omega$ dla której zachodzi wzór
\eqref{eq:Wojtynski-10} i~na podstawie tego wyprowadziliśmy zależność
\eqref{eq:Wojtynski-12}. Możemy jednak przyjąć za punkt wyjścia własności
formy $\omega( \vecxbold, \vecybold )$, by następnie udowodnić, że~zachodzi wzór
\eqref{eq:Wojtynski-10}, przy czym elementy macierzy $\Omega$ w~nim występującej
spełniają zależność \eqref{eq:Wojtynski-12}.

To oraz inne zagadnienia, takie jak przedstawienie form dwuliniowych,
półtoraliniowych, etc., za pomocą wzoru analogicznego do
\eqref{eq:Wojtynski-10} są wyczerpująco omówione w~wielu książkach do
algebry liniowej, nie będziemy się tutaj dłużej nad nimi
zatrzymywać.\footnote{Warto by napisać w~jakich książkach można znaleźć
  omówienie tych problemów.}

\vspace{\spaceFour}




\start \StrWd{32}{13} Nie mam pojęcia o~co chodzi w~stwierdzeniu
„gdzie $n = \dim \Kbb^{ n }$”.

\vspace{\spaceFour}





\start \StrWd{33}{10} W~niewianie wyglądającym wzorze $\left(
  \frac{ e_{ 1 } }{ \sqrt{ \alpha } }, \frac{ e_{ 2 } }{ \sqrt{ \alpha } } \right)$
ukryta jest trudność, wynikająca z~tego, że~pierwiastek kwadratowy na
płaszczyźnie zespolonej jest zawsze funkcją dwuznaczną\footnote{Pierwiastek
  z~zera to jak wiadomo wyjątek.}. Trzeba więc przyjąć jakąś konwencję,
która będzie nam mówiła, ile dokładnie wynosi $\sqrt{ \alpha }$, gdzie
$\alpha \in \Cbb$ i~możemy spokojnie kontynuować rozważania.

\vspace{\spaceFour}





\start \StrWd{33}{5} Podany tu wzór $\dim Y^{ \perp } = n - 2$ jest niewątpliwie
prawdziwy, w~tym jednak momencie nie potrafię zrozumieć poprawności jego
dowodu. Do problemu tego należy wrócić, po odświeżeniu sobie informacji
z~algebry liniowej o~wymiarach przestrzeni wektorowych i~formach liniowych.

\vspace{\spaceFour}



\start \Str{33-34} Podany tu dowód sprowadzania formy symplektycznej do
postaci kanonicznej zawiera dużą lukę, bowiem w~algorytmie sprowadzania do
postaci kanonicznej został pominięty bardzo ważny krok. Wyjaśnienie jak
należy poprawić przedstawiony algorytm, poprzedzimy ogólną dyskusją problemu.

Przyjmijmy, że~tak jak w~książce, forma symplektyczna $\omega$ jest zadana na
przestrzeni $\Cbb^{ n }$. Niech $n = 2k$, gdzie $k \in \Nbb \setminus \{ 0 \}$. Wiemy
już, że~formy symplektyczne istnieją tylko na przestrzeni o~wymiarze
parzystym więc takie $k$ istnieje. Przejdźmy do momentu, gdy macierz formy symplektycznej ma postać
\begin{equation}
  \label{eq:Wojtynski-13}
  \Omega =
  \begin{pmatrix}
    \hphantom{-} 0 & 1 & \hphantom{-} 0 & \ldots \\
    -1 & 0 & \hphantom{-} 0 & \ldots \\
    \hphantom{-} 0 & 0 & \hphantom{-} 0 & \matUnit \\
    \hphantom{-} 0 & 0 & -\matUnit & 0 \\
  \end{pmatrix}
\end{equation}
Ta macierz nie wygląda najlepiej, acz stworzenie lepszej w~systemie \LaTeX{}
odłożymy na kiedy indziej\footnote{Oby to kiedy indziej kiedyś nadeszło.}.
Na razie zauważymy, że~symbol $\ldots$ oznacza, iż~ostatni podany jawnie symbol
w~macierzy powtarza~się w danym wierszu, aż~do jego końca. Do tego w~trzecim
i~czwartym wierszu każdy symbol oznacza blok macierzowy o~rozmiarach
$( k - 1 ) \times ( k - 1 )$. Symbol $\matUnit$ oznacza macierz jednostkową, zaś
$0$ macierz której elementami są wyłącznie zera. Jak już mówiliśmy, nie
wygląda to ładnie, ale na razie musi wystarczyć.

By ułatwić rachunki, przyjmujemy iż wyrażenie
\begin{equation}
  \label{eq:Wojtynski-14}
  d_{ i } \leftrightarrow d_{ j },
\end{equation}
oznacza przejście do ze starej bazy $d_{ 1 }, d_{ 2 }, \ldots, d_{ n }$ nowej bazy
$d'_{ 1 }, d'_{ 2 }, \ldots, d'_{ n }$ danej jako
\begin{subequations}
  \begin{align}
    \label{eq:Wojtynski-15-A}
    &d'_{ i } = d_{ j }, \\
    \label{eq:Wojtynski-15-B}
    &d'_{ j } = d_{ i }, \\
    \label{eq:Wojtynski-15-C}
    &d'_{ l } = d_{ l },
  \end{align}
\end{subequations}
dla $l \notin \{ i, j \}$. Oprócz tego będziemy zwykle opuszczać prim
w~oznaczeniach nowej bazy, gdyż nie powinno to wprowadzać dwuznaczności,
a~poważnie uprości to zapis.

Wedle tego co pisze w~książce, gdy macierz jest formie
\eqref{eq:Wojtynski-06} to wystarczy wykonać jedną operację zmiany bazy,
$d_{ 2 } \leftrightarrow d_{ k + 1 }$, by uzyskać postać kanoniczną, co jest nieprawdą.
W~istocie potrzebne jest $k - 1$ operacji tego typu, które zaraz opiszemy.
Zwróćmy uwagę, że~dla $n = 2$, $k = 1$ czy potrzebnych jest $1 - 1 = 0$
operacja. Rzeczywiście, macierz \eqref{eq:Wojtynski-06} przyjmuje postać
\begin{equation}
  \label{eq:Wojtynski-16}
  \Omega =
  \begin{pmatrix}
    \hphantom{-} 0 & 1 \\
    -1 & 0
  \end{pmatrix}.
\end{equation}
Gdy $n = 4$, $k = 2$, więc potrzebne jest $k - 1 = 1$ operacja zmiany bazy,
którą jest właśnie operacja $d_{ 2 } \leftrightarrow d_{ k + 1 }$. Widzimy więc, że~dopiero
dla $n = 6$, $k = 3$ przedstawiony w~książce algorytm przestaje dawać
porządny wynik.

Zauważmy teraz, że~ze względu na definicję wyrazów macierzy $\Omega$
\begin{equation}
  \label{eq:Wojtynski-17}
  \Omega_{ l m } = \omega( d_{ l }, d_{ m } ),
\end{equation}
wynika, że~operacja $d_{ i } \leftrightarrow d_{ j }$ na poziomie macierzy sprowadza~się do
\begin{enumerate}

\item Zamiany miejscami $i$-tego wiersza z~$j$-tym.

\item Zamiany miejscami $i$-tej kolumny z~$j$-tą.

\end{enumerate}
By to pokazać, wybierzmy najpierw $l \notin \{ i, j \}$. Wówczas wyrazy
$\Omega_{ l m }$ dla $l \notin \{ i, j \}$ nie ulegają zmianie. Dla $m = i$, mamy zaś
\begin{equation}
  \label{eq:Wojtynski-18}
  \Omega'_{ l i } = \omega( d'_{ l }, d'_{ i } ) = \omega( d_{ l }, d_{ j } ) =
  \Omega_{ l j }.
\end{equation}
Analogiczny rachunek pokazuje, że~$\Omega'_{ k j } = \Omega_{ k i }$. Weźmy teraz
$l = i$ i~$m \notin \{ i, j \}$. Prowadzi to do
\begin{equation}
  \label{eq:Wojtynski-19}
  \Omega'_{ i m } = \omega( d'_{ i }, d'_{ m } ) = \omega( d_{ j }, d_{ m } ) =
  \Omega_{ j m }.
\end{equation}
Przejdźmy teraz do przypadku $k = l$, $l = m$. Mamy wtedy
\begin{equation}
  \label{eq:Wojtynski-20}
  \Omega'_{ i j } = \omega( d'_{ i }, d'_{ j } ) = \omega( d_{ j }, d_{ i } ) =
  \Omega_{ j i }.
\end{equation}
Analogicznie przebiega rachunek pokazujący, że
\begin{equation}
  \label{eq:Wojtynski-21}
  \Omega'_{ j i } = \Omega_{ i j }.
\end{equation}
Wzory \eqref{eq:Wojtynski-18}--\eqref{eq:Wojtynski-21} stanowią indeksowy
zapis reguł 1 i~2.

Jesteśmy teraz gotowi uzupełnić podany w~książce algorytm sprowadzania
formy do postaci kanonicznej. W~momencie w~którym sprawdziliśmy macierz
formy symplektycznej do postaci \eqref{eq:Wojtynski-07} należy wykonać
$k - 1$ operacji zmiany bazy:
\begin{subequations}
  \begin{align}
    \label{eq:Wojtynski-22-A}
    d_{ 2 } &\leftrightarrow d_{ k + 1 }, \\
    \label{eq:Wojtynski-22-B}
    d_{ 2 } &\leftrightarrow d_{ 3 }, \\
    \label{eq:Wojtynski-22-C}
    d_{ 3 } &\leftrightarrow d_{ 4 }, \\
    \label{eq:Wojtynski-22-D}
    &\hspace{0.65em} \vdots \\
    \label{eq:Wojtynski-22-E}
    d_{ k - 1 } &\leftrightarrow d_{ k }.
  \end{align}
\end{subequations}
Nie podamy w~pełni formalnego dowodu tej procedury, gdyż wymagałby on
żmudnego operowania na~wskaźnikach, które zapewne nie wyjaśnia wiele,
zamiast tego, pokażemy jak ta procedure działa dla przypadku $n = 8$.
Zanim jednak to tego przejdziemy, pokażemy jak taki formalny dowód mógłby
wyglądać.

Niech $\Omega^{ 0 }$ oznacza macierz formy symplektycznej w~postaci kanonicznej.
Jej elementy dane są jako
\begin{equation}
  \label{eq:Wojtynski-23}
  \Omega^{ 0 }_{ l m } =
  \begin{cases}
    -\delta^{ l }_{ m - l }, \quad l = 1, 2, \ldots, k, \\
    \hphantom{-} \delta^{ l - k }_{ m }, \quad l = k + 1, k + 2, \ldots, k.
  \end{cases}
\end{equation}
We wzorach powyżej symbol $\delta^{ i }_{ j }$ oznacza symbol Kroneckera, nie zaś
jak w~książce (zob. str.~33) macierz-kolumnę o~wyrazach danych przez
odpowiednie wartości przyjmowane przez ten symbol. Jeżeli teraz zapiszemy
w~notacji wskaźnikowej wyrazy macierzy $\Omega$, wzór \eqref{eq:Wojtynski-07}, to
dzięki wzorom \eqref{eq:Wojtynski-18}--\eqref{eq:Wojtynski-21}, poprzez
dokładną analizę, będziemy w~stanie pokazać, że~ciąg zmian baz
\eqref{eq:Wojtynski-22-A}--\eqref{eq:Wojtynski-22-E} prowadzi od macierzy
$\Omega$ do macierzy $\Omega^{ 0 }$. Na koniec zauważmy, że~ponieważ jak pokazano na
stronie 33, macierz $\Omega$ jest antysymetryczna, wystarczy pokazać, że~ta
procedure prowadzi do dobre postaci wyrazów $\Omega_{ l m }$, dla $l \leq m$ (albo
$l \geq m$).

Przejdźmy teraz to ilustracji działania procedury
\eqref{eq:Wojtynski-22-A}--\eqref{eq:Wojtynski-22-E} dla przypadku $n = 8$,
$k = 4$. W~momencie początkowym nasza macierz ma postać
\begin{equation}
  \label{eq:Wojtynski-24}
  \Omega =
  \begin{pmatrix}
    \hphantom{-} 0 & 1 & \hphantom{-} 0 & \hphantom{-} 0
    & \hphantom{-} 0 & 0 & 0 & 0 \\
    -1 & 0 & \hphantom{-} 0 & \hphantom{-} 0 & \hphantom{-} 0
                         & 0 & 0 & 0 \\
    \hphantom{-} 0 & 0 & \hphantom{-} 0 & \hphantom{-} 0
                     & \hphantom{-} 0 & 1 & 0 & 0 \\
    \hphantom{-} 0 & 0 & \hphantom{-} 0 & \hphantom{-} 0
                     & \hphantom{-} 0 & 0 & 1 & 0 \\
    \hphantom{-} 0 & 0 & \hphantom{-} 0 & \hphantom{-} 0
                     & \hphantom{-} 0 & 0 & 0 & 1 \\
    \hphantom{-} 0 & 0 & -1 & \hphantom{-} 0 & \hphantom{-} 0
                         & 0 & 0 & 0 \\
    \hphantom{-} 0 & 0 & \hphantom{-} 0 & -1 & \hphantom{-} 0
                         & 0 & 0 & 0 \\
    \hphantom{-} 0 & 0 & \hphantom{-} 0 & \hphantom{-} 0 & -1
                         & 0 & 0 & 0
  \end{pmatrix}.
\end{equation}
Musimy teraz dokonać zmian baz $d_{ 2 } \leftrightarrow d_{ 5 }$,
$d_{ 2 } \leftrightarrow d_{ 3 }$, $d_{ 3 } \leftrightarrow d_{ 4 }$, przy czym będziemy korzystali
z~reguł 1~i~2 podanych wcześniej.
\begin{equation}
  \label{eq:Wojtynski-25}
  \begin{split}
    &\begin{pmatrix}
       \hphantom{-} 0 & 1 & \hphantom{-} 0 & \hphantom{-} 0
       & \hphantom{-} 0 & 0 & 0 & 0 \\
       -1 & 0 & \hphantom{-} 0 & \hphantom{-} 0 & \hphantom{-} 0
                        & 0 & 0 & 0 \\
       \hphantom{-} 0 & 0 & \hphantom{-} 0 & \hphantom{-} 0
       & \hphantom{-} 0 & 1 & 0 & 0 \\
       \hphantom{-} 0 & 0 & \hphantom{-} 0 & \hphantom{-} 0
       & \hphantom{-} 0 & 0 & 1 & 0 \\
       \hphantom{-} 0 & 0 & \hphantom{-} 0 & \hphantom{-} 0
       & \hphantom{-} 0 & 0 & 0 & 1 \\
       \hphantom{-} 0 & 0 & -1 & \hphantom{-} 0 & \hphantom{-} 0
                        & 0 & 0 & 0 \\
       \hphantom{-} 0 & 0 & \hphantom{-} 0 & -1 & \hphantom{-} 0
                        & 0 & 0 & 0 \\
       \hphantom{-} 0 & 0 & \hphantom{-} 0 & \hphantom{-} 0 & -1
                        & 0 & 0 & 0
     \end{pmatrix}
      \xrightarrow[ d_{ 2 } \, \leftrightarrow \, d_{ 5 } ]{}
      \begin{pmatrix}
        \hphantom{-} 0 & \hphantom{-} 0 & \hphantom{-} 0 & \hphantom{-} 0
        & 1 & 0 & 0 & 0 \\
        \hphantom{-} 0 & \hphantom{-} 0 & \hphantom{-} 0 & \hphantom{-} 0
        & 0 & 0 & 0 & 1 \\
        \hphantom{-} 0 & \hphantom{-} 0 & \hphantom{-} 0 & \hphantom{-} 0
        & 0 & 1 & 0 & 0 \\
        \hphantom{-} 0 & \hphantom{-} 0 & \hphantom{-} 0 & \hphantom{-} 0
        & 0 & 0 & 1 & 0 \\
        -1 & \hphantom{-} 0 & \hphantom{-} 0 & \hphantom{-} 0 & 0 & 0 & 0
                    & 0 \\
        \hphantom{-} 0 & \hphantom{-} 0 & -1 & \hphantom{-} 0 & 0 & 0 & 0
                    & 0 \\
        \hphantom{-} 0 & \hphantom{-} 0 & \hphantom{-} 0 & -1 & 0 & 0 & 0
                    & 0 \\
        \hphantom{-} 0 & -1 & \hphantom{-} 0 & \hphantom{-} 0 & 0 & 0 & 0
                    & 0 \\
      \end{pmatrix}
      \xrightarrow[ d_{ 2 } \, \leftrightarrow \, d_{ 3 } ]{ } \\[1em]
    &\xrightarrow[ d_{ 2 } \, \leftrightarrow \, d_{ 3 } ]{ }
      \begin{pmatrix}
        \hphantom{-} 0 & \hphantom{-} 0 & \hphantom{-} 0 & \hphantom{-} 0
        & 1 & 0 & 0 & 0 \\
        \hphantom{-} 0 & \hphantom{-} 0 & \hphantom{-} 0 & \hphantom{-} 0
        & 0 & 1 & 0 & 0 \\
        \hphantom{-} 0 & \hphantom{-} 0 & \hphantom{-} 0 & \hphantom{-} 0
        & 0 & 0 & 0 & 1 \\
        \hphantom{-} 0 & \hphantom{-} 0 & \hphantom{-} 0 & \hphantom{-} 0
        & 0 & 0 & 1 & 0 \\
        -1 & \hphantom{-} 0 & \hphantom{-} 0 & \hphantom{-} 0 & 0 & 0 & 0
                    & 0 \\
        \hphantom{-} 0 & -1 & \hphantom{-} 0 & \hphantom{-} 0 & 0 & 0 & 0
                    & 0 \\
        \hphantom{-} 0 & \hphantom{-} 0 & \hphantom{-} 0 & -1 & 0 & 0 & 0
                    & 0 \\
        \hphantom{-} 0 & \hphantom{-} 0 & -1 & \hphantom{-} 0 & 0 & 0 & 0
                    & 0
      \end{pmatrix}
      \xrightarrow[ d_{ 3 } \, \leftrightarrow \, d_{ 4 } ]{}
      \begin{pmatrix}
        \hphantom{-} 0 & \hphantom{-} 0 & \hphantom{-} 0
        & \hphantom{-} 0 & 1 & 0 & 0 & 0 \\
        \hphantom{-} 0 & \hphantom{-} 0 & \hphantom{-} 0
        & \hphantom{-} 0 & 0 & 1 & 0 & 0 \\
        \hphantom{-} 0 & \hphantom{-} 0 & \hphantom{-} 0
        & \hphantom{-} 0 & 0 & 0 & 1 & 0 \\
        \hphantom{-} 0 & \hphantom{-} 0 & \hphantom{-} 0
        & \hphantom{-} 0 & 0 & 0 & 0 & 1 \\
        -1 & \hphantom{-} 0 & \hphantom{-} 0 & \hphantom{-} 0 & 0
                             & 0 & 0 & 0 \\
        \hphantom{-} 0 & -1 & \hphantom{-} 0 & \hphantom{-} 0 & 0
                             & 0 & 0 & 0 \\
        \hphantom{-} 0 & \hphantom{-} 0 & -1 & \hphantom{-} 0 & 0 & 0
                                 & 0 & 0 \\
        \hphantom{-} 0 & \hphantom{-} 0 & \hphantom{-} 0 & -1 & 0 & 0
        & 0 & 0 \\
      \end{pmatrix}
  \end{split}
\end{equation}
Ten przykład powinien dobrze ilustrować, w~jaki sposób działa podany
algorytm sprowadzania formy symplektycznej do postaci kanonicznej.

\vspace{\spaceFour}





\start \StrWg{34}{7--8} Znajdujące~się tu zdanie „Formę mającą postać (2.11)
nazywamy \textit{formą symplektyczną}.” jest niefortunnie sformułowane.
W~istocie formę nazywamy formą symplektyczną, jeśli w~pewnej bazie
przyjmuje postać daną wzorem (2.11).

Stwierdzenie~2.11 pokazuje, że~każda dwuliniowa, antysymetryczna
i~nieosobliwa forma, jest formą symplektyczną. Teraz udowodnimy odwrotną
implikację, mianowicie, iż~każda forma symplektyczna jest antysymetryczna
i~nieosobliwa.

Z~definicji, każda form symplektyczna jest formą dwuliniową i~istnieje baza
przestrzeni $\Cbb^{ n }$, gdzie $n = 2k$, taka że
\begin{equation}
  \label{eq:Wojtynski-26}
  \omega( \vecxbold, \vecybold ) =
  \sum_{ i = 1 }^{ k } x_{ i } \, y_{ k + i } - y_{ i } \, x_{ k + i }, \qquad
  \vecxbold = \sum_{ i = 1 }^{ n } x_{ i } \, \vecebold_{ i }, \;\;
  \vecybold = \sum_{ i = 1 }^{ n } y_{ i } \, \vecebold_{ i }.
\end{equation}
Z~tego od razu wynika, że~$\omega$ jest formą antysymetryczna. Teraz wystarczy
pokazać, że~jeśli $\omega( \vecxbold, \vecybold ) = 0$ dla każdego
$\vecybold \in \Cbb^{ n }$, to $\vecxbold = \vecZeroBold$. Załóżmy, że~dany
jest taki $\vecxbold$ i~weźmy $\vecybold = ( \delta^{ i }_{ j } )$, gdzie
$1 \leq j \leq k$. Obliczmy teraz
\begin{equation}
  \label{eq:Wojtynski-27}
  \omega( \vecxbold, \vecybold ) =
  \sum_{ i = 1 }^{ k } x_{ i } \, y_{ k + i } - y_{ i } \, x_{ k + i } =
  \sum_{ i = 1 }^{ k } x_{ i } \, \delta^{ k + i }_{ j } - \delta^{ i }_{ j } \, x_{ k + i } =
  -x_{ k + j } = 0.
\end{equation}
Widzimy więc, że~$x_{ i } = 0$, dla $k + 1 \leq i \leq n$. Analogicznie biorąc
$\vecebold = ( \delta^{ i }_{ k + j } )$, przy $1 \leq j \leq k$, dowodzimy,
że~$x_{ i } = 0$, dla $1 \leq i \leq k$, tym samym pokazaliśmy,
że~$\vecxbold = \vecZeroBold$. To~kończy dowód twierdzenia, iż~każda forma
symplektyczna, jest dwuliniową, antysymetryczną i~nieosobliwą.

\vspace{\spaceFour}




\start \textbf{Str. 34, wiersze (od dołu) 1--3.} Znajdujemy tu informację,
że~$n$-tą zespoloną (rzeczywistą) liniową grupę symplektyczną, którą
oznaczamy jako $\Sp( n, \Cbb )$ ($\Sp( n, \Rbb )$), definiujemy
jako podgrupę $\GL( 2n, \Cbb )$ ($\GL( 2n, \Rbb )$) składającą~się
z~macierzy nieosobliwych stopnia $2n$ zachowujących macierz
\begin{equation}
  \label{eq:Wojtynski-28}
  \Omega =
  \begin{pmatrix}
    \hphantom{-} 0 & \matUnit \\
    -\matUnit & 0
  \end{pmatrix},
\end{equation}
przy czym $\Omega$ jest macierzą kwadratową stopnia $2n$. Rozumiemy przez to,
że~macierz $A$ należy do $\Sp( n, \Rbb )$ ($\Sp( n, \Cbb )$), jeśli
\begin{equation}
  \label{eq:Wojtynski-29}
  A^{ T } \, \Omega \, A = \Omega.
\end{equation}
Dodatkowe zaznaczanie, że~$A$ musi być macierzą nieosobliwą, kładzie na tą
własność tak silny nacisk, iż~może prowadzić do nieporozumień.

Zauważmy, że~macierz $A$ musi być macierzą nieosobliwą z~dwóch powodów. Po
pierwsze, bo $A \in \GL( n, \Cbb)$ ($A \in \GL( n, \Rbb )$), a~z~definicji
każda macierz tego typu jest nieosobliwa. Po drugie, jak zostało to
zauważone wcześniej na tej stronie książki Wojtyńskiego, jeśli macierz $A$
zachowuje nieosobliwą macierz, to sama jest nieosobliwa, macierz $\Omega$ jest
zaś nieosobliwa. Dowód nieosobliwości macierz $\Omega$ można uzyskać na wiele
sposobów, choćby modyfikując podany w~tych notatkach dowód tego, że~forma
symplektyczna, czyli taka która w~pewnej bazie przyjmuje postać
\eqref{eq:Wojtynski-26}, jest nieosobliwa w~podanym na~32 sensie.

\vspace{\spaceFour}





\start \Str{34} W~lewej dolnej i~prawej górnej ćwiartce przedstawione tu
macierzy brakuje symbolu $0$. Kiedyś można by~się zastanowić nad
wygenerowanie w~\LaTeX{} macierzy, która lepiej ilustrowałaby problem.

\vspace{\spaceFour}





\start \StrWg{34}{8} Aby definicja przestrzeni $\spFrak( n, \Kbb )$ była
ścisła, należy doprecyzować czym jest pojawiająca~się w~niej forma $\omega$.
Należy bowiem przyjąć, że~$\omega$ nie jest dowolną formą symplektyczną, czyli
taką która w~\textit{pewnej} bazie przyjmuje postać
\eqref{eq:Wojtynski-26}, lecz taką która przyjmuje tą postać w~bazie
kanonicznej. Inaczej mówiąc, jeśli
\begin{equation}
  \label{eq:Wojtynski-30}
  \vecxbold = ( x_{ 1 }, x_{ 2 }, \ldots, x_{ 2n } ) \in \Kbb^{ 2n }, \quad
  \vecybold = ( y_{ 1 }, y_{ 2 }, \ldots, y_{ 2n } ) \in \Kbb^{ 2n },
\end{equation}
to zachodzi
\begin{equation}
  \label{eq:Wojtynski-31}
  \omega( \vecxbold, \vecybold ) =
  \sum_{ i = 1 }^{ n } x_{ i } \, y_{ n + i } - y_{ i } \, x_{ n + i }.
\end{equation}

\vspace{\spaceFour}





\start \Str{38} Z~nie do końca jasnych dla mnie powodów, gdy na tej stronie
przedstawiono kwaterniony jako dwuwymiarową przestrzeń wektorową nad nad
$\Cbb$, zdefiniowano kombinację liniową kwaterniowów bazowych $e_{ 0 }$
i~$e_{ 2 }$ w~taki sposób, że~współczynniki zespolone mnożą je z~prawej
strony:
\begin{equation}
  \label{eq:Wojtynski-32}
  q = e_{ 0 } \, z + e_{ 2 } \, w, \quad z, \, w \in \Cbb.
\end{equation}
Podejrzewam, że~chodzi o~to, iż~nieprzemienność mnożenie kwaternionów
sprawia, iż~to jak definiujemy działanie macierzy z~$\lFrak( n, \Hbb )$ na
wektor z~przestrzeni $\Hbb^{ n }$ ma kilka nieoczywistych konsekwencji dla
innych części teorii, co może sprawić iż ta nietypowa notacja okazuje~się
mieć sporo zalet. Zagadnienie to omówimy bardziej szczegółowo w~dalszej
części notatek, tu zaś poprzestaniemy na wskazaniu, iż~w~tej notacji
własność jednorodność odwzorowania $A : V \to V$ zapisujemy jako
\begin{equation}
  \label{eq:Wojtynski-33}
  A ( v \alpha ) = ( A v ) \alpha, \quad \alpha \in \Kbb, \, v \in V,
\end{equation}
nie zaś jak w~standardowej notacji
\begin{equation}
  \label{eq:Wojtynski-34}
  A \alpha v = \alpha A v.
\end{equation}

Ostatecznie, to czy mnożenie zapisujemy z~lewej czy z~prawej strony jest
tylko kwestią konwencji i~całą teorię da~się zdefiniować w~sposób spójny
wybierając jedną z nich. W~tej kwestii warto jednak zauważyć to,
iż~wybierając mnożenie kwaternionów bazowych z~lewej strony, musielibyśmy
zmienić wzór (2.20) z~książki, czyli
\begin{equation}
  \label{eq:Wojtynski-35}
  q =
  e_{ 0 } \, ( t_{ 0 } \, e_{ 0 } + t_{ 1 } \, e_{ 1 } )
  + e_{ 2 } \, ( t_{ 2 } \, e_{ 0 } - t_{ 3 } e_{ 1 } ),
\end{equation}
na
\begin{equation}
  \label{eq:Wojtynski-36}
  q =
  ( t_{ 0 } \, e_{ 0 } + t_{ 1 } \, e_{ 1 } ) \, e_{ 0 }
  + ( t_{ 2 } \, e_{ 0 } + t_{ 3 } \, e_{ 1 } ) \, e_{ 2 }.
\end{equation}
Pod pewnym względami wzór \eqref{eq:Wojtynski-35} wygląda bardziej
naturalnie.

Na koniec zwróćmy uwagę na to, że~skoro liczby zespolone są podciałem
kwaternionów, to niezależnie czy mnożenie kwaternionów bazowych określimy
z~prawej, czy lewej strony, możemy zawsze przenieść liczbę zespoloną
z~prawej strony kwaternionu bazowe na lewą i~na odwrót. W~tym kontekście
warto podać dwa użyteczne wzory:
\begin{subequations}
  \begin{align}
    \label{eq:Wojtynski-37-A}
    &e_{ 0 } \, z = z \, e_{ 0 }, \\
    \label{eq:Wojtynski-38-B}
    &e_{ 2 } \, z = \bar{z} \, e_{ 2 },
  \end{align}
\end{subequations}
gdzie $z \in \Cbb$.

\vspace{\spaceFour}





\start \Str{39} Podane tu wyjaśnienie dlaczego odwzorowanie $S$
przekształca odwzorowania liniowe przestrzeni $\Hbb^{ n }$ w~odwzorowania
liniowe przestrzeni $\Cbb^{ 2n }$ jest niepotrzebnie zagmatwane. Postaram~się
przedstawić dowód tego faktu w~sposób jaśniejszy, najpierw jednak
umieścimy cały problem w~szerszym kontekście. Do opisu tego problemu
można użyć formalizmu teorii kategorii i~choć takie rozwiązanie ma wiele
zalet, to~jednak zastosowanie go tutaj, może nas oddalić od bardzo
konkretnego problemu grup macierzy, którym~się zajmujemy, a~to jest rzeczą
nieporządną.

Zacznijmy od tego, że~główną cechą odróżniającą kwaterniony $\Hbb$ od liczb
rzeczywistych i~zespolonych, jest nieprzemienność ich mnożenia. Mimo tego,
przestrzenie macierzy $\lFrak( n, \Kbb )$, gdzie $\Kbb$ może oznaczać
$\Rbb$, $\Cbb$ lub~$\Hbb$, posiadają wiele wspólnych cech algebraicznych.
Aby je zwięźle opisać, wprowadzimy trochę terminologii. Przyjmiemy,
że~struktura algebraiczna posiadająca wszystkie cechy ciała, ale w~której
mnożenie nie musi być przemienne, będziemy nazywać \textbf{ciałem
  nieprzemiennych} oraz że~zarówno przestrzenie wektorowe jak i~algebry mogą
być zarówno nad ciałem przemiennym jak i~nieprzemiennych. Wówczas
$\lFrak( n, \Rbb )$ i~$\lFrak( n, \Cbb )$ są łącznymi, nieprzemiennymi
algebrami wedle standardowych definicji. Strukturę algebry nieprzemiennej
wyznacza mnożenie przez elementy z~ciała, dodawanie i~mnożenie macierzy.
Opisanie struktur $\lFrak( n, \Hbb )$ wymaga wprowadzenia nowych pojęć.

Niech $V$ będzie przestrzenią wektorową na ciałem $\Kbb$, na której jest
zadane jest działanie wewnętrzne
\begin{equation}
  \label{eq:Wojtynsi-39}
  \cdot : V \times V \ni ( v, w ) \mapsto v \cdot w \in V.
\end{equation}
Rozpatrzmy teraz dwie potencjalne własności tego działania:
\begin{subequations}
  \begin{align}
    \label{eq:Wojtynski-40-A}
    &( \alpha \, v + \beta \, w ) \cdot p  =  \alpha \, ( v \cdot p ) + \beta \, ( w \cdot p ), \\
    \label{eq:Wojtynski-40-B}
    &v \cdot ( \alpha \, w + \beta \, p ) = \alpha \, ( v \cdot w ) + \beta \, ( v \cdot p ),
  \end{align}
\end{subequations}
gdzie $\alpha, \beta \in \Kbb$, $v, w, p \in V$. Jeśli dla danej przestrzeni $V$ wzór
\eqref{eq:Wojtynski-40-A} zachodzi dla wszystkich możliwych przypadków,
to~przestrzeń tą nazywamy \textbf{lewej algebrą}. Definicje \textbf{lewej
  algebry łącznej}, \textbf{lewej algebry przemiennej}, etc., są
prostymi uogólnieniami definicji dla standardowej algebry, nie ma więc
potrzeby byśmy je tutaj podawali. Analogicznie, jeśli dla wszystkich
możliwych przypadków zachodzi wzór \eqref{eq:Wojtynski-40-B} to $V$ nazywamy
\textbf{prawą algebrą}. Zdefiniowanie \textbf{prawej algebry łącznej}
i~innych możliwych jej typów, nie powinno teraz sprawiać żadnego problemu.
Łatwo zauważyć, że jeśli dana przestrzeń jest jednocześnie algebrą lewą
i~prawą, to jest ona algebrą w~standardowym sensie tego słowa.

W~dalszym ciągu tych notatek pokażemy~że $\lFrak( n, \Kbb )$ jest ??? algebrą ???, ale nie jest algebrą bo nie spełnia warunku \eqref{eq:????}

Dokładniej omówimy to zagadnienie poniżej, warto~się jednak teraz
zastanowić, co ta sytuacja nam mówi. Czy powinniśmy zmienić definicję
„algebry nad ciałem nieprzemiennym”, tak by usunąć z~niej jedną wymaganą
własność, tak iż $\lFrak( n, \Hbb )$ stanie~się przykładem takiej algebry?
Czy też należy zostawić definicje takie jakimi są, stwierdzają tylko,
iż~$\lFrak( n, \Hbb )$ jest bytem, któremu niewiele brakuje, by był
odpowiednią algebrą nad ciałem nieprzemiennym? Nie umiem w~tej chwili
odpowiedzieć na to pytanie. Niezależnie od tego, użycie pojęcie „łącznej
nieprzemiennej algebry nad ciałem przemiennym (nieprzemiennym)”, wydaje~się
być dobrym krokiem do zrozumienia problemu.

Naszą analizę rozpoczniemy od usystematyzowania oznaczeń. Za Wojtyńskim
przyjmujemy następującą, prawostronną, konwekcję mnożenia elementów
$\Kbb^{ n }$ przez elementy z~$\Kbb$
\begin{equation}
  % Numeracja wzorów.
  \label{eq:Wojtynski-36}
  v \alpha = ( v_{ 1 } \alpha, v_{ 2 } \alpha, \ldots, v_{ n } \alpha ), \quad
  \alpha \in \Kbb, \, v \in \Kbb^{ n }.
\end{equation}
Ponadto dla macierzy $A \in \lFrak( n, \Kbb )$ o~elementach $a_{ ij }$,
$i, j = 1, 2, \ldots, n$ przyjmujemy dodatkowe oznaczenie
\begin{equation}
  \label{eq:Wojtynski-37}
  A = [ a_{ i j } ].
\end{equation}
Mnożenie macierzy przez element z~$\Kbb$ oraz sumę dwóch macierzy
definiujemy w~zwykły sposób.
\begin{subequations}
  \begin{align}
    \label{eq:Wojtynski-38-A}
    A \alpha &= [ a_{ i j } \, \alpha ], \\
    \label{eq:Wojtynski-38-B}
    A + B &= [ a_{ i j } + b_{ i j } ],
  \end{align}
\end{subequations}
gdzie, $\alpha \in \Kbb$,
$A = [ a_{ i j } ]$, $B = [ b_{ i j } ] \in \lFrak( n, \Kbb )$.
Również podążając za Wojtyńskim, działanie macierzy na wektor z~$\Kbb^{ n }$
definiujemy jako
\begin{equation}
  \label{eq:Wojtynski-39}
  ( A v )_{ i } = \sum_{ j = 1 }^{ n } a_{ i j } \, v_{ j },
\end{equation}
gdzie $v \in \Kbb^{ n }$, $A \in \lFrak( n, \Kbb )$. Teraz operację mnożenia
dwóch macierzy z~$\lFrak( n, \Kbb )$ możemy określić jako
\begin{equation}
  \label{eq:Wojtynski-40}
  A \cdot B = \left[ \sum_{ l = 1 }^{ n } a_{ i l } \, b_{ l j } \right].
\end{equation}
Dla uczynienia rozważań bardziej precyzyjnymi, przez pewien czas mnożenie
macierzy będziemy jawnie oznaczać symbolem $\cdot$.

Teraz przejdziemy do dowodu tego, że~$\lFrak( n, \Kbb )$ stanowi
nieprzemienną algebrę. Z~rozmysłem odkładamy na~później rozważanie
o~związku macierzy $\lFrak( n, \Kbb )$ z~odwzorowaniami liniowymi na
$\Kbb^{ n }$. Jak~się okazuje, ten problem posiada kilka własnych niuansów
wartych osobnej dyskusji. Pierwszym krokiem jest udowodnienie,
że~$\lFrak( n, \Kbb )$ z~działaniami zdefiniowanymi
w~\eqref{eq:Wojtynski-38-A}--\eqref{eq:Wojtynski-38-B} jest przestrzenią
wektorową. Niech $A \in \lFrak( n, \Kbb )$. Własność
\begin{equation}
  \label{eq:Wojtynski-41}
  A \, 1 = A,
\end{equation}
gdzie $1$ jest elementem neutralnym mnożenia w~$\Kbb$ wynika wprost
z~definicji mnożenia macierzy przez element z~ciała $\Kbb$ podanej
w~\eqref{eq:Wojtynski-38-A} oraz własności mnożenia w~ciele $\Kbb$.
Podobnie własność
\begin{equation}
  \label{eq:Wojtynski-42}
  A \, ( \beta \, \alpha ) = ( A \, \beta ) \, \alpha, \quad \alpha, \beta \in \Kbb,
\end{equation}
wynika, jak poprzednio z~definicji mnożenia macierzy przez element z~ciała
oraz łączności mnożenia w~ciele $\Kbb$. W~dalszym ciągu nie będziemy jawnie
przywoływać miejsc w~których korzystamy z~definicji
\eqref{eq:Wojtynski-38-A}--\eqref{eq:Wojtynski-38-B}, pozostaniemy przy
przytaczaniu potrzebnych własności z~ciała $\Kbb$. Z~rozdzielności dodawania
względem mnożenia w~ciele $\Kbb$ wynikają dwie poniższe własności
\begin{subequations}
  \begin{align}
    \label{eq:Wojtynski-43-A}
    &A \, ( \alpha + \beta ) = A \, \alpha + A \, \beta, \\
    \label{eq:Wojtynski-43-B}
    &( A + B ) \, \alpha = A \, \alpha + B \, \beta,
  \end{align}
\end{subequations}
gdzie $A, B \in \lFrak( n, \Kbb )$ i~$\alpha, \beta \in \Kbb$. Korzystając teraz
z~tego, że~dodawanie w~$\Kbb$ jest łączne, przemienne i~dla każdego
elementu $a \in \Kbb$ istnieje element odwrotny względem dodawania $-a$,
dowodzimy
\begin{subequations}
  \begin{align}
    \label{eq:Wojtynski-44-A}
    &( A + B ) + C = A + ( B + C ), \\
    \label{eq:Wojtynski-44-B}
    &A + B = B + A, \\
    \label{eq:Wojtynski-44-C}
    &A + ( -A ) = 0,
  \end{align}
\end{subequations}
gdzie $A, B, C \in \lFrak( n, \Kbb )$, zaś $0 \in \lFrak( n, \Kbb )$ to
macierz zerowa\footnote{Macierz, której wszystkie elementy są równe
  elementowi neutralnemu dodawania w~ciele $\Kbb$.}. Udowodniliśmy więc,
że~$\lFrak( n, \Kbb )$ jest przestrzenią wektorową i~nigdzie w~dowodzie
przemienność, lub jej brak, mnożenia nie odgrywa żadnej roli.

Teraz przejdziemy do dowodu tego, że~$\lFrak( n, \Kbb )$ wraz z~działaniem
mnożenia macierzy \eqref{eq:Wojtynski-40} jest łączną, nieprzemienną
algebrą. Jak powiedzieliśmy wcześniej, nieprzemienność mnożenia macierzy
jest dobrze znana, nie będziemy~się więc nad nią zatrzymywać. Łączność
mnożenia macierzy
\begin{equation}
  \label{eq:Wojtynski-45}
  ( A \cdot B ) \cdot C = A \cdot ( B \cdot C ),
\end{equation}
wynika z~definicji operacji mnożenia macierzy \eqref{eq:Wojtynski-40} oraz
z~łączności mnożenia w~ciele $\Kbb$. Własność
\begin{equation}
  \label{eq:Wojtynski-46}
  A \cdot ( B \, \alpha + C \, \beta ) = A \cdot ( B \, \alpha ) + A \cdot ( C \, \beta ) =
  ( A \cdot B ) \, \alpha + ( A \cdot C ) \, \beta.
\end{equation}
wynika z~definicji mnożenia macierzy oraz łączności i~rozdzielności
mnożenia względem dodawania w~ciele $\Kbb$. Jednak
\begin{equation}
  \label{eq:Wojtynski-47}
  ( A \, \alpha + B \, \beta ) \cdot C \neq
  ( A \cdot C ) \, \alpha + ( B \cdot C ) \, \beta =
  A \cdot ( C \, \alpha ) + B \cdot ( C \, \beta ).
\end{equation}
Równość pomiędzy członem drugim i~trzecim powyższego wyrażenia, wynika
z~definicji wstępujących tam działań oraz łączności i~rozdzielności
mnożenia w~$\Kbb$. By zobaczyć, że~równość między dwoma pierwszy członami
w~ogólności nie zachodzi, wystarczy rozpatrzyć prosty przykład. Niech
$\matUnit \in \lFrak( n, \Kbb )$ będzie macierzą jednostkową zdefiniowaną
w~standardowy sposób. Dla $A = B = \matUnit$, $C = \matUnit \, e_{ 2 }$
i~$\alpha = \beta = \frac{ 1 }{ 2 } e_{ 1 }$ mamy
\begin{subequations}
  \begin{align}
    \label{eq:Wojtynski-48-A}
    &( A \, \alpha + B \, \beta ) \cdot C = \matUnit \, e_{ 1 } \, e_{ 2 }, \\
    \label{eq:Wojtynski-48-B}
    &( A \cdot C ) \, \alpha + ( B \cdot C ) \, \beta = \matUnit \, e_{ 2 } \, e_{ 1 } =
      -\matUnit \, e_{ 1 } \, e_{ 2 }.
  \end{align}
\end{subequations}
Gdyby oprócz wszystkich udowodnionych dotychczas własności, zachodziła
jeszcze równość we wzorze \eqref{eq:Wojtynski-47}, to~$\lFrak( n, \Kbb )$
byłoby łączną, nieprzemienną algebrą. O~problemie tym wspominaliśmy już
na początku naszych rozważań nad strukturą algebraiczną zbiorów
$\lFrak( n, \Kbb )$. Jak widać po tym przykładzie, problem ten pojawia~się
bo mnożenie kwaternionów nie jest przemienne.

Ponieważ ten wynik może być zaskakujący dla osób, dla których liczby
rzeczywiste są synonimem wszystkich „systemów liczbowych”, zatrzymamy~się
dłużej nad pewnym konsekwencjami nieprzemienność mnożenia kwaternionów Dla
wszystkich macierzy $\lFrak( n, \Kbb )$, gdzie $\Kbb$ oznacza $\Rbb$ lub
$\Cbb$, zachodzi własność
\begin{equation}
  \label{eq:Wojtynski-49}
  ( \matUnit \alpha ) A = A ( \matUnit \alpha ), \quad
  \alpha \in \Kbb, \, A \in \lFrak( n, \Kbb ),
\end{equation}
gdzie $\matUnit$ ponownie oznacza macierz jednostkową. Jednak
\begin{equation}
  \label{eq:Wojtynski-50}
  ( \matUnit \alpha ) A \neq A ( \matUnit \alpha ), \quad
  \alpha \in \Hbb, \, A \in \lFrak( n, \Hbb ).
\end{equation}
Prosty przykład ilustrujący tą własność otrzymuje przez wzięcie
$\alpha = e_{ 1 }$, $A = \matUnit \, e_{ 2 }$. Warto zauważyć, że~jeśli
$\alpha = t \, e_{ 0 }$, $t \in \Rbb$, to dla wszystkich macierz
$A \in \lFrak( n, \Hbb )$ zachodzi $( \matUnit \alpha ) A = A ( \matUnit \alpha )$.

Inną własnością odróżniającą macierze z~$\lFrak( n, \Hbb )$ od tych
z~$\lFrak( n, \Rbb )$ i~$\lFrak( n, \Cbb )$ jest to,
że~zachodzi
\begin{equation}
  \label{eq:Wojtynski-51}
  A ( v \, \alpha ) = ( A v ) \, \alpha,
\end{equation}
gdzie $\alpha \in \Hbb,$ $v \in \Hbb^{ n }$, $A \in \lFrak( n, \Hbb )$. Fakt ten
ponownie wynika z~definicji działań oraz łączności i~rozdzielności mnożenia
względem dodawania w~$\Kbb$. Jednakże
\begin{equation}
  \label{eq:Wojtynski-52}
  ( A \, \alpha ) v \neq ( A v ) \, \alpha.
\end{equation}
Prosty przykład dla którego możemy to zobaczyć to $\alpha = e_{ 2 }$,
$v = ( e_{ 3 }, 0, 0, \ldots, 0 )$, $A = \matUnit e_{ 2 }$. Dla podkreślenia tego,
jak bardzo „nienaturalną” jest ta ostatnia własność, zapiszemy ją w~notacji, w~której mnożenie przez liczby obywa~się z~lewej strony.
\begin{equation}
  \label{eq:Wojtynski-53}
  ( \alpha A ) v \neq A ( \alpha v ).
\end{equation}
Prosty przykład dla którego można to zobaczyć to  (tu też używamy
konwencji mnożenia przez liczbę z~lewej strony).

W~tym miejscu musimy zwrócić uwagę no to, że~to wzór \eqref{eq:Wojtynski-51}
wyraża fakt, iż~macierz~$A$ reprezentuje pewno odwzorowanie jednorodnym,
nie zaś wzór \eqref{eq:Wojtynski-52}. Pozorny problem może wyniknąć stąd,
że~mnożenie kwaternionów nie jest przemienne, więc konkretny sposób w~jaki
definiujemy działanie macierzy $A$ na wektor $v \in \Hbb^{ n }$ ma znaczenie
dla naszej teorii. Wojtyński zdefiniował je kładąc
\begin{equation}
  \label{eq:Wojtynski-52}
  ( A v )_{ i } = \sum_{ j }^{ n } a_{ i j } \, v_{ j }, \quad
  A = [ a_{ ij } ] \in \lFrak( n, \Hbb ).
\end{equation}
Takie określenie odwzorowanie $A$ jest jednorodne, gdy mnożenie wektorów
z~$\Hbb^{ n }$ przez elementy z~$\Hbb$ jest zdefiniowane z~prawej strony.
Gdybyśmy zdefiniowali mnożenie przez te elementy z~lewej strony, to
by działanie macierzy na wektor $v$ było jednorodne, musielibyśmy
zdefiniować je jako
\begin{equation}
  \label{eq:Wojtynski-53}
  ( A v )_{ i } = \sum_{ j }^{ n } v_{ j } \, a_{ i j }.
\end{equation}

Wiemy już, że~odwzorowanie zadane przez macierz $A \in \lFrak( n, \Kbb )$
wzorem \eqref{eq:Wojtynski-39} jest jednorodne nad $\Hbb$ i~tym samym
jednorodne nad $\Rbb$ i~$\Cbb$, które są podciałami $\Hbb$\footnote{Użyta
  tu terminologi może być ścisła, ale może być też trochę nieścisła. Nie
  umiem tego teraz ustalić. Niezależnie od tego sama treść matematyczna
  powinna być na tyle jasna, że~możemy na takim sformułowaniu poprzestać.}.
Teraz możemy przejść do dowodu, że~jest ono addytywne i~tym samym liniowe.
Dowód faktu, że
\begin{equation}
  \label{eq:Wojtynski-54}
  A ( v + w ) = A v + A w, \quad v, \, w \in \Hbb^{ n },
\end{equation}
przebiega wedle standardowego schematu, z~wykorzystaniem tego, że~w~ciele
$\Hbb$ dodawania jest rozdzielne względem mnożenia.

W~tym momencie wiemy już, że~każda macierz z~$\lFrak( n, \Hbb )$
reprezentuje odwzorowanie liniowe nad $\Hbb$. Odwrotnie, każde odwzorowanie
liniowe wyznacza pewną macierz. Oznaczmy bowiem
\begin{equation}
  \label{eq:Wojtynski-55}
  E_{ i } = ( 0, 0, \ldots, 0, e_{ 0 }, 0, \ldots, 0 ),
\end{equation}
gdzie $e_{ 0 }$ znajduje~się na $i$-tej pozycji. Wówczas dla dowolnego
$v \in \Hbb^{ n }$ mamy
\begin{equation}
  \label{eq:Wojtynski-56}
  ????
\end{equation}

Przejdźmy teraz do dowodu tego, że~$\lFrak( n, \Kbb )$ tworzy algebrę,
gdzie $\Rbb$, $\Cbb$ lub $\Hbb$. By udowodnić własności
\begin{equation}
  \label{eq:Wojtynski-43}
  ( A + B ) v = A v + B v,
\end{equation}
wystarczy by dodawanie było rozdzielne względem mnożenia, co jest prawdą
we~wszystkich rozważanych typach $\Kbb$. Ponieważ dowód tego faktu jest
prosty, nie będziemy go tu przytaczać jawnie.

\vspace{\spaceFour}














% ##################
\newpage

\CenterBoldFont{Błędy}


\begin{center}

  \begin{tabular}{|c|c|c|c|c|}
    \hline
    & \multicolumn{2}{c|}{} & & \\
    Strona & \multicolumn{2}{c|}{Wiersz} & Jest
                              & Powinno być \\ \cline{2-3}
    & Od góry & Od dołu & & \\
    \hline
    5   & & 10 & klasa grup & klasa \\
    7   & &  4 & sumę & \textit{sumę} \\
    7   & &  1 & $M_{ 1 } \ldots M_{ k }$ & $M_{ 1 }, \ldots, M_{ k }$ \\
    8   &  3 & & $( m,\; n )$ & $( m, n )$ \\
    8   & 17 & & $n \in N$ & $m \in M$ \\
    8   & 17 & & $f( m ) \in M$ & $f( m ) \in N$ \\
    8   & 18 & & $f( n )$ & $f( m )$ \\
    9   & &  4 & $x_{ 1 },\;\; x_{ 2 } \in X$ & $x_{ 1 },\, x_{ 2 } \in X$ \\
    9   & &  4 & albo & lub \\
    10  &  4 & & $y_{ 0 } \succ x$($y_{ 0 } \prec x$)
           & $x \prec y_{ 0 }$ ($y_{ 0 } \prec x$) \\
    10  & 14 & & \textit{to istnieje w~$X$} & \textit{to w~$X$ istniej} \\
    10  & &  9 & $U \subset \Omega_{ x_{ 0 } }$ & $U \in \Omega_{ x_{ 0 } }$ \\
    11  & 10 & & $\rho( X, Y )$ & $\rho( x, y )$ \\
    11  & &  5 & zbioru & niepustego zbioru \\
    13  &  6 & & $y_{ \varphi } \in X_{ \varphi }$ & $x_{ \varphi } \in X_{ \varphi }$ \\
    14  &  1 & & $x_{ 1 },\;\; x_{ 2 } \in X$ & $x_{ 1 },\, x_{ 2 } \in X$ \\
    14  &  6 & & $x e_{ + }$ & $x \, e_{ + }$ \\
    14  & 18 & & $\mu\big( x, \mu( x, y ) \big)$
           & $\mu\big( x, \mu( y, z ) \big)$ \\
    14  & 21 & & $\mathbf{x} \in \mathbf{X}$ & $x \in X$ \\
    15  & &  5 & $\mu \omega( x_{ 1 }, y )$ & $\mu \, \omega( x_{ 2 }, y )$ \\
    17  &  7 & & $W$ modulo $V$, & $W$, \\
    17  &  7 & & $\tau( x_{ j }, y_{ k } )$
           & $\tau( x_{ j }, y_{ k } ) = \pi( x_{ j } \otimes y_{ k } )$ \\
    18  & &  1 & $X^{ k } \times X^{ m }$ & $X^{ \otimes k } \times X^{ \otimes m }$ \\
    18  & &  1 & $X^{ k } \otimes X^{ m }$ & $X^{ \otimes k } \otimes X^{ \otimes m }$ \\
    20  & 19 & & $\overline{ F( y, x ) }$ & $\overline{ F( y, x ) }$. \\
    20  & 20 & & (tj. & Tj. \\
    20  & 20 & & $\Rbb$)) & $\Rbb$) \\[0.3em]
    21  &  5 & & $\displaystyle
                 \frac{ \overline{ \langle x, y \rangle } }{ | \langle x, y \rangle | }$
           & $\displaystyle
             \frac{ \overline{ \langle x', y \rangle } }{ | \langle x', y \rangle | }$ \\
    21  &  5 & & $\langle x, y \rangle \neq 0$ & $\langle x', y \rangle \neq 0$ \\
    23  &  4 & & $\lambda ( A_{ 1 } x )$ & $\lambda \big( A_{ 1 }( x ) \big)$ \\
    23  & &  3 & $f( x_{ 1 }, \ldots x_{ n } )$ & $f( x_{ 1 }, \ldots, x_{ n } )$ \\
    23  & &  3 & $x_{ 1 }^{ \, k_{ 1 } }, \ldots x_{ n }^{ \, k_{ n } }$
           & $x_{ 1 }^{ \, k_{ 1 } } \ldots x_{ n }^{ \, k_{ n } }$ \\
    23  & &  3 & $( x_{ 1 } \ldots x_{ n } )$ & $( x_{ 1 }, \ldots, x_{ n } )$ \\
    24  & 15 & & $( a, b ) \in \Omega$ & $( a, b ) \in \Omega$, \\
    24  & & 12 & $\Ccal^{ 1 }$ & $\Ccal^{ k }$ \\
    25  &  9 & & $1 / 2 \leq t$ & $1 / 2 < t$ \\
    \hline
  \end{tabular}





  \newpage

  \begin{tabular}{|c|c|c|c|c|}
    \hline
    & \multicolumn{2}{c|}{} & & \\
    Strona & \multicolumn{2}{c|}{Wiersz} & Jest
                              & Powinno być \\ \cline{2-3}
    & Od góry & Od dołu & & \\
    \hline
    25  & 12 & & $i_{ x_{ 0 } }( t ) = 0$ & $i_{ x_{ 0 } }( t ) = x_{ 0 }$ \\
    26  & &  4 & 46 & 47 \\
    26  & &  1 & $x \in X$ & $a \in B$ \\
    27  &  2 & & $= d$ & $= u$ \\
    29  &  9 & & $\textrm{Gl}( n, \Cbb )$ & $\GL( n, \Cbb )$ \\
    29  & &  9 & $( A B )^{ T } = B^{ T } A^{ T }$
           & $( A \cdot B )^{ T } = B^{ T } \cdot A^{ T }$ \\
    29  & &  8 & $\overline{ A B }$ & $\overline{ A \cdot B }$ \\
    29  &  7 & & $( A B )^{ * }$ & $( A \cdot B )^{ * }$ \\
    30  & & 16 & $\oFrak( n, \Cbb )$ & $\oFrak_{ + }( n, \Cbb )$ \\
    30  & & 10 & $*$ -- operacją & $*$ operacją \\
    31  &  5 & & $a_{ i i } = -\bar{a}_{ i i }$
           & $-a_{ i i } = \bar{a}_{ i i }$ \\
    31  & 13 & & $\widetilde{ ( A B ) }$ & $\widetilde{ ( A \cdot B ) }$ \\
    31  & &  7 & \big(($A \in \OStraightBold( n, \Rbb )$\big)
           & \big($A \in \OStraightBold( n, \Rbb )$\big) \\
    32  & &  2 & (2.9) & (2.10) \\
    36  &  8 & & $\End( \Kbb^{ 2n } )$ & $\lFrak( 2n, \Kbb )$ \\
    36  &  8 & & $\omega( xAy )$ & $\omega( x, Ay )$ \\
    37  &  9 & & $e_{ i } ( e_{ j }, e_{ k } )$
           & $e_{ i } \cdot ( e_{ j } \cdot e_{ k } )$ \\
    37  &  9 & & $( e_{ i }, e_{ j } ) e_{ k }$
           & $( e_{ i } \cdot e_{ j } ) \cdot e_{ k }$ \\
    37  & &  8 & $q \cdot \frac{ 1 }{ \Vert q^{ 2 } \Vert } \bar{q}$
           & $\bar{q} \cdot \frac{ 1 }{ \Vert q^{ 2 } \Vert } q$ \\
    37  & &  8 & $\bar{q} q$ & $\bar{q} \cdot q$ \\
    37  & &  4 & $\Vert q_{ 1 } q_{ 2 } \Vert$ & $\Vert q_{ 1 } \cdot q_{ 2 } \Vert$ \\
    39  &  8 & & $z_{ n + 1 }$ & $z_{ n + i }$ \\
    39  & &  8 & $\langle s^{ -1 }( z ), \, s^{ -1 }( w ) \rangle_{ \Hbb } =$
           & zachodzi $\langle s^{ -1 }( z ), \, s^{ -1 }( w ) \rangle_{ \Hbb } =$ \\
    50  & &  8 & $e_{ i } ( e_{ j } \cdot e_{ k } )$
           & $e_{ i } \cdot ( e_{ j } \cdot e_{ k } )$ \\
    % & &  &  &  \\
    % & & & & \\
    % & & & & \\
    % & & & & \\
    \hline
  \end{tabular}





  \newpage

  \begin{tabular}{|c|c|c|c|c|}
    \hline
    & \multicolumn{2}{c|}{} & & \\
    Strona & \multicolumn{2}{c|}{Wiersz} & Jest
                              & Powinno być \\ \cline{2-3}
    & Od góry & Od dołu & & \\
    \hline
    % & &  & & \\
    76  & 16 & & $\displaystyle \frac{ \partial ( \varphi \circ f_{ 2 } ) }{ \partial x_{ j } }$
& $\displaystyle \frac{ \partial ( \varphi \circ f_{ 2 } ) }{ \partial x_{ j } }( 0 )$ \\
    99  &  2 & & $[ a, b ] \in \mathbf{K}$ & $[ a, b ] \in M$ \\
    99  &  4 & & $[ a, b ] \in \mathbf{K}$ & $[ a, b ] \in M$ \\
    99  &  4 & & $[ b, a ] \in \mathbf{K}$ & $[ b, a ] \in M$ \\
    99  & & 13 & $*$ & Działanie $*$ \\
    100 & &  9 & $\mathbf{K} \leq L_{ 2 }$ & $M \leq L_{ 2 }$ \\
    100 & &  9 & $f^{ -1 }( \mathbf{K} )$ & $f^{ - 1}( M )$ \\
    100 & &  6 & $a_{ i } = f^{ -1 }( b_{ i } )$
           & $a_{ i } \in f^{ -1 }( b_{ i } )$ \\
    100 & &  5 & $\mathbf{K}$ & $M$ \\
    100 & &  4 & $f^{ -1 }( K )$ & $f^{ -1 }( M )$ \\
    % & & & & \\
    % & & & & \\
    % & & & & \\
    % & & & & \\
    % & & & & \\
    % & & & & \\
    % & & & & \\
    \hline
  \end{tabular}

\end{center}

\vspace{\spaceTwo}


\noindent
\StrWg{18}{5} \\
\Jest  $\displaystyle \omega\Big( \big( \sum_{ i } x_{ i } \otimes y_{ i } ),
\sum_{ j } x_{ j } \otimes y_{ j } )$ \\
\Powin $\displaystyle \omega\Big( \sum_{ i } x_{ i } \otimes y_{ i },
\sum_{ j } v_{ j } \otimes w_{ j } \Big)$ \\
\StrWg{18}{5} \\
\Jest  $\displaystyle \mu( x_{ i }, x_{ j } ) \otimes \nu( y_{ i }, y_{ j } )$ \\
\Powin $\displaystyle \mu( x_{ i }, v_{ j } ) \otimes \nu( y_{ i }, w_{ j } )$ \\
\StrWg{19}{11} \\
\Jest  $\rho( x_{ 1 } ) \cdot \rho( x_{ 2 } ) \ldots \rho( x_{ n } )$ \\
\Powin $\rho( x_{ 1 } ) \cdot \rho( x_{ 2 } ) \cdot \ldots \cdot \rho( x_{ n } )$ \\
\StrWd{31}{4} \\
\Jest  $A \in \OStraightBold( n, \Rbb )$ \\
\Powin $A \in \OStraightBold( n, \Rbb )$ zachodzi \\
\StrWd{34}{6} \\
\Jest  \textit{$n$-tą grupą symplektyczną} (oznaczaną $\Sp( n )$)
nazywamy \\
\Powin Przez \textit{$n$-tą grupę symplektyczną} (oznaczaną $\Sp( n )$)
będziemy rozumieć \\
\StrWd{34}{3} \\
\Jest  \textit{$N$-tą zespoloną} (rzeczywistą) \textit{liniową grupą
  symplektyczną} (oznaczenia $\Sp( n, \Cbb )$, $\Sp( n, \Rbb )$) \\
nazywamy \\
\Powin Przez \textit{$n$-tą zespoloną} (rzeczywistą) \textit{liniową grupę
  symplektyczną} (oznaczenia $\Sp( n, \Cbb )$, $\Sp( n, \Rbb )$) będziemy
rozumieć \\
\StrWd{35}{3} \\
\Jest  czyli $\bar{A}^{ T } \cdot A = \matUnit$ lub inaczej
$\bar{A}^{ T } = A^{ -1 }$, czyli $\widetilde{A} = A^{ -1 }$ \\[0.5em]
\Powin czyli $\bar{A} = \Big( A^{ T } \Big)^{ -1 }$, co można zapisać jako
$\widetilde{A} = \bar{A}$ \\
\StrWg{38}{3} \\[0.5em]
\Jest  $\displaystyle \big( \sum_{ i = 0 }^{ 3 } t_{ i } e_{ i } )$ \\
\Powin $\displaystyle \big( \sum_{ i = 0 }^{ 3 } t_{ i } e _{ i } \big)$ \\[0.5em]
\StrWg{38}{7} \\
\Jest  $e_{ 0 } \, ( t_{ 0 } + t_{ 1 } \, e_{ 1 } )
+ e_{ 2 } \, ( t_{ 2 } - t_{ 3 } \, e_{ 1 } )$ \\[0.5em]
\Powin $e_{ 0 } ( t_{ 0 } \, e_{ 0 } + t_{ 1 } \, e_{ 1 } )
+ e_{ 2 } \, ( t_{ 2 } \, e_{ 0 } - t_{ 3 } \, e_{ 1 } )$ \\

% ############################










% ####################################################################
% ####################################################################
% Bibliografia

\bibliographystyle{plalpha}

\bibliography{MathComScienceBooks}{}





% ############################

% Koniec dokumentu
\end{document}
