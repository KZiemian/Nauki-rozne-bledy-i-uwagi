% ------------------------------------------------------------------------------------------------------------------
% Basic configuration and packages
% ------------------------------------------------------------------------------------------------------------------
% Package for discovering wrong and outdated usage of LaTeX.
% More information to be found in l2tabu English version.
\RequirePackage[l2tabu, orthodox]{nag}
% Class of LaTeX document: {size of paper, size of font}[document class]
\documentclass[a4paper,11pt]{article}



% ------------------------------------------------------
% Packages not tied to particular normal language
% ------------------------------------------------------
% This package should improved spaces in the text
\usepackage{microtype}
% Add few important symbols, like text Celcius degree
\usepackage{textcomp}



% ------------------------------------------------------
% Polonization of LaTeX document
% ------------------------------------------------------
% Basic polonization of the text
\usepackage[MeX]{polski}
% Switching on UTF-8 encoding
\usepackage[utf8]{inputenc}
% Adding font Latin Modern
\usepackage{lmodern}
% Package is need for fonts Latin Modern
\usepackage[T1]{fontenc}



% ------------------------------------------------------
% Setting margins
% ------------------------------------------------------
\usepackage[a4paper, total={14cm, 25cm}]{geometry}



% ------------------------------------------------------
% Setting vertical spaces in the text
% ------------------------------------------------------
% Setting space between lines
\renewcommand{\baselinestretch}{1.1}

% Setting space between lines in tables
\renewcommand{\arraystretch}{1.4}



% ------------------------------------------------------
% Packages for scientific papers
% ------------------------------------------------------
% Switching off \lll symbol, that I guess is representing letter "Ł"
% It collide with `amsmath' package's command with the same name
\let\lll\undefined
% Basic package from American Mathematical Society (AMS)
\usepackage[intlimits]{amsmath}
% Equations are numbered separately in every section
\numberwithin{equation}{section}

% Other very useful packages from AMS
\usepackage{amsfonts}
\usepackage{amssymb}
\usepackage{amscd}
\usepackage{amsthm}

% Package with better looking calligraphy fonts
\usepackage{calrsfs}

% Package with better looking greek letters
% Example of use: pi -> \uppi
\usepackage{upgreek}
% Improving look of lambda letter
\let\oldlambda\Lambda
\renewcommand{\lambda}{\uplambda}




% ------------------------------------------------------
% BibLaTeX
% ------------------------------------------------------
% Package biblatex, with biber as its backend, allow us to handle
% bibliography entries that use Unicode symbols outside ASCII
\usepackage[
language=polish,
backend=biber,
style=alphabetic,
url=false,
eprint=true,
]{biblatex}

\addbibresource{Historia-nauki-Bibliography.bib}





% ------------------------------------------------------
% Defining new environments (?)
% ------------------------------------------------------
% Defining enviroment "Wniosek"
\newtheorem{corollary}{Wniosek}
\newtheorem{definition}{Definicja}
\newtheorem{theorem}{Twierdzenie}





% ------------------------------------------------------
% Local packages
% You need to put them in the same directory as .tex file
% ------------------------------------------------------
% Package containing various command useful for working with a text
\usepackage{./Local-packages/general-commands}
% Package containing commands and other code useful for working with
% mathematical text
\usepackage{./Local-packages/math-commands}





% ------------------------------------------------------
% Package "hyperref"
% They advised to put it on the end of preambule
% ------------------------------------------------------
% It allows you to use hyperlinks in the text
\usepackage{hyperref}










% ------------------------------------------------------------------------------------------------------------------
% Title and author of the text
\title{Historia nauki \\
  {\Large Błędy i~uwagi}}

\author{Kamil Ziemian}

% \date{}
% ------------------------------------------------------------------------------------------------------------------










% ####################################################################
% Początek dokumentu
\begin{document}
% ####################################################################





% ######################################
\maketitle % Tytuł całego tekstu
% ######################################





% ######################################
\section{ % Autor i tytuł dzieła
  Nicolas Bourbaki \\
  \textit{Elementy historii matematyki},
  \parencite{Bourbaki-Elementy-historii-matematyki-Pub-1980}}

\label{sec:Bourbaki-EHM}
% ######################################



% ##################
\CenterBoldFont{Uwagi}

\vspace{0em}


\noindent
\Str{63} Dziwne, że~autorzy formułują tu własność systemów
liczbowych pisząc, że~iloraz $b_{ n } / b_{ n - 1 }$ jest równy tej
samej liczbie~$b$, zamiast po~prostu stwierdzić,
iż~$b_{ n } = b \, b_{ n - 1 }$.

\VerSpaceFour





\noindent
\Str{68} Autorzy piszą tu w~taki sposób, jakby Grecy
i~rachmistrzowie byli dwoma zupełnie różnymi grupami ludzi.
Najprawdopodobniej jednak rachmistrzowie sami byli Grekami.

\VerSpaceFour





\noindent
\StrWierszDol{243}{6} Mniejszy nawias okrągły jest za~duży, przez co źle
wygląda. W~tej chwili nie wiem jednak zmodyfikować tekst, by ten problem
naprawić.





% ##################
\newpage

\CenterBoldFont{Błędy}


\begin{center}

  \begin{tabular}{|c|c|c|c|c|}
    \hline
    Strona & \multicolumn{2}{c|}{Wiersz} & Jest
                              & Powinno być \\ \cline{2-3}
    & Od góry & Od dołu & & \\
    \hline
    \hphantom{0}8 & 11 & & metodzie ” & metodzie” \\
    16 & & 17 & poprzednikom~($^{ 4 }$) & poprzednikom($^{ 4 }$) \\
    18 & & \hphantom{0}3 & „implikuje”) ($^{ 1 }$) & „implikuje”)($^{ 1 }$) \\
    30 & & & 13) ($^{ 1 }$) & 13)($^{ 1 }$) \\
    37 & 15 & & [193c]),obejmujący & [193c]), obejmujący \\
    39 & & \hphantom{0}7 & środka & środka” \\
    41 & 13 & & zdumieniu ($^{ 2 }$) & zdumieniu($^{ 2 }$) \\
    43 & 12 & & stwierdza & stwierdza też \\
    43 & & \hphantom{0}9 & myśl & pomysł \\
    44 & & & analizy ($^{ 2 }$) & analizy($^{ 2 }$) \\
    44 & & 17 & chwili ($^{ 2 }$) & chwili($^{ 2 }$) \\
    45 & 17 & & 448) & 448)) \\
    48 & & \hphantom{0}9 & potocznym & sformalizowanym \\
    60 & 18 & & est & jest \\
    61 & & \hphantom{0}4 & matematyków,a & matematyków, a \\
    65 & \hphantom{0}1 & & niepustego & niepustych \\
    70 & \hphantom{0}2 & & zespolonych~($^{ 1 }$) & zespolonych($^{ 1 }$) \\
    70 & \hphantom{0}8 & & wektorów~($^{ 2 }$) & wektorów($^{ 2 }$) \\
    70 & 15 & & równoważności~($^{ 3 }$) & równoważność($^{ 3 }$) \\
    71 & \hphantom{0}3 & & pierwsza ($^{ 1 }$) & pierwsza($^{ 1 }$) \\
    71 & 15 & & skończonego ($^{ 2 }$) & skończonego($^{ 2 }$) \\
    72 & \hphantom{0}1 & & jeszcze & wciąż \\
    72 & \hphantom{0}3 & & grup”~($^{ 2 }$) & grup”($^{ 2 }$) \\
    72 & \hphantom{0}9 & & abstrakcyjnej~($^{ 2 }$)
    & abstrakcyjnej($^{ 2 }$) \\
    72 & & 16 & 84) ($^{ 3 }$) & 84)($^{ 3 }$) \\
    73 & 10 & & wieku~($^{ 1 }$) & wieku($^{ 1 }$) \\
    73 & & 12 & roli; & roli, \\
    74 & 12 & & Jordana-H\"{o}ldera”~($^{ 1 }$)
    & Jordana-H\"{o}ldera”($^{ 1 }$) \\
    74 & 15 & & ]139b] & [139b] \\
    82 & & \hphantom{0}9 & \textit{metafizycznego}”;
    & \textit{metafizycznego}”); \\
    83 & 12 & & Hamilton ($^{ 1 }$) & Hamilton($^{ 1 }$) \\
    \hline
  \end{tabular}





  \newpage

  \begin{tabular}{|c|c|c|c|c|}
    \hline
    Strona & \multicolumn{2}{c|}{Wiersz} & Jest
                              & Powinno być \\ \cline{2-3}
    & Od góry & Od dołu & & \\
    \hline
    \hphantom{0}84 & & \hphantom{0}6 & a~m.~in. & m.~in. \\
    \hphantom{0}85 & & 16 & Jacobi ($^{ 1 }$) & Jacobi($^{ 1 }$) \\
    \hphantom{0}86 & & 12 & używali & używają \\
    \hphantom{0}89 & & \hphantom{0}4 & niezależne, & niezależne; \\
    \hphantom{0}90 & & 12 & 193)) & 193) \\
    \hphantom{0}90 & & \hphantom{0}4 & niekwadratowych ($^{ 1 }$)
    & niekwadratowych($^{ 1 }$) \\
    \hphantom{0}91 & \hphantom{0}6 & & niewymierna ($^{ 1 }$)
    & niewymierna($^{ 1 }$) \\
    \hphantom{0}91 & & 17 & okręgi ($^{ 2 }$) & okręgi($^{ 2 }$) \\
    \hphantom{0}92 & \hphantom{0}1 & & wiemy ($^{ 1 }$) & wiemy($^{ 1 }$) \\
    \hphantom{0}92 & 14 & & algebraicznych ($^{ 2 }$)
    & algebraicznych($^{ 2 }$) \\
    \hphantom{0}93 & 10 & & niewiele ($^{ 1 }$) & niewiele($^{ 1 }$) \\
    \hphantom{0}93 & 15 & & rachunkach ($^{ 2 }$) & rachunkach($^{ 2 }$) \\
    \hphantom{0}93 & & 17 & kopiować analogiczne wzory
    & szukać analogicznych wzorów \\
    \hphantom{0}93 & & \hphantom{0}3 & skopiowany & wzorowany \\
    \hphantom{0}94 & 11 & & mianowicie ($^{ 1 }$) & mianowicie($^{ 1 }$) \\
    \hphantom{0}94 & & \hphantom{0}9 & w~~którym & z~którego \\
    \hphantom{0}95 & \hphantom{0}4 & & Algebrze & \textit{Algebrze} \\
    \hphantom{0}95 & \hphantom{0}8 & & nowoczesnych($^{ 1 }$).Wreszcie
    & nowoczesnych($^{ 1 }$). Wreszcie \\
    \hphantom{0}95 & 10 & & trzeciego ($^{ 2 }$) & trzeciego($^{ 2 }$) \\
    \hphantom{0}95 & 16 & & dodatnie ($^{ 3 }$) & dodatnie($^{ 3 }$) \\
    \hphantom{0}95 & & 13 & ,meno de & „meno de \\
    \hphantom{0}96 & & 17 & 98) ($^{ 1 }$) & 98)($^{ 1 }$) \\
    \hphantom{0}99 & \hphantom{0}2 & & danego ($^{ 1 }$) & danego($^{ 1 }$) \\
    \hphantom{0}99 & \hphantom{0}3 & & \textit{metafizykę} ($^{ 2 }$)
    & \textit{metafizykę}($^{ 2 }$) \\
    \hphantom{0}99 & \hphantom{0}6 & & wartości ($^{ 3 }$)
    & wartości($^{ 3 }$) \\
    100 & & 18 & pierwiastków ($^{ 1 }$) & pierwiastków($^{ 1 }$) \\
    101 & \hphantom{0}1 & & Gaussa ($^{ 1 }$) & Gaussa($^{ 1 }$) \\
    101 & \hphantom{0}7 & & pierwszego ($^{ 2 }$) & pierwszego($^{ 2 }$) \\
    102 & \hphantom{0}8 & & Abel ($^{ 2 }$) & Abel($^{ 2 }$) \\
    103 & 10 & & równania) ($^{ 2 }$) & równania)($^{ 2 }$) \\
    106 & \hphantom{0}1 & & topologicznej ($^{ 1 }$)
    & topologicznej($^{ 1 }$) \\
    106 & 12 & & koła ($^{ 2 }$) & koła($^{ 2 }$) \\
    \hline
  \end{tabular}





  \newpage

  \begin{tabular}{|c|c|c|c|c|}
    \hline
    Strona & \multicolumn{2}{c|}{Wiersz} & Jest
                              & Powinno być \\ \cline{2-3}
    & Od góry & Od dołu & & \\
    \hline
    106 & & 12 & zespolonych ($^{ 3 }$) & zespolonych($^{ 3 }$) \\
    106 & 16 & & [148a) & [148a] \\
    107 & & 13 & 23) ($^{ 2 }$) & 23)($^{ 2 }$) \\
    109 & 12 & & Euklidesa” ($^{ 1 }$) & Euklidesa”($^{ 1 }$) \\
    110 & \hphantom{0}9 & & ([dane] & [dane] \\
    110 & 15 & & całkowitej ($^{ 1 }$) & całkowitej($^{ 1 }$) \\
    111 & 18 & & kwadratem) ($^{ 2 }$) & kwadratem)($^{ 2 }$) \\
    112 & \hphantom{0}7 & & niewiadomych~~($^{ 1 }$)
    & niewiadomych($^{ 1 }$) \\
    113 & \hphantom{0}8 & & 373) -- ; & 373); \\
    113 & 11 & & $p$-grup ($^{ 2 }$) & $p$-grup($^{ 2 }$) \\
    114 & 18 & & 174) ($^{ 1 }$) & 174)($^{ 1 }$) \\
    115 & \hphantom{0}9 & & rzędu~~($^{ 1 }$) & rzędu($^{ 1 }$) \\
    115 & 18 & & później ($^{ 2 }$) & później($^{ 2 }$) \\
    116 & \hphantom{0}1 & & wieku ($^{ 1 }$) & wieku($^{ 1 }$) \\
    116 & 15 & & jedności ($^{ 2 }$) & jedności($^{ 2 }$) \\
    117 & \hphantom{0}2 & & [150b] ($^{ 1 }$) & [150b]($^{ 1 }$) \\
    118 & & \hphantom{0}9 & algebrze ($^{ 1 }$) & algebrze($^{ 1 }$) \\
    118 & & \hphantom{0}7 & topologia ($^{ 2 }$) & topologia($^{ 2 }$) \\
    119 & \hphantom{0}1 & & jeszcze prawdziwe & prawdziwe \\
    120 & & 13 & 488) ($^{ 1 }$) & 488)($^{ 1 }$) \\
    121 & \hphantom{0}5 & & algebraiczne ($^{ 1 }$)
    & algebraiczne($^{ 1 }$) \\
    121 & \hphantom{0}3 & & $\Zbb[ i |$ & $\Zbb[ i ]$ \\
    122 & \hphantom{0}3 & & Gaussa ($^{ 1 }$) & Gaussa($^{ 1 }$) \\
    123 & & 11 & Galoisa ”czynników  & Galoisa” czynników \\
    % ???????????
    124 & & 10 & \textit{skończona} ($^{ 1 }$)
           & \textit{skończona}($^{ 1 }$) \\
    125 & \hphantom{0}6 & & podstawowe”~~($^{ 1 }$) & podstawowe”($^{ 1 }$) \\
    126 & 18 & & 1880 ($^{ 1 }$) & 1880($^{ 1 }$) \\
    127 & & \hphantom{0}6 & $\Zbb[ \sqrt{ -3 } ]$) ($^{ 1 }$)
           & $\Zbb[ \sqrt{ -3 } ]$)($^{ 1 }$) \\
    128 & \hphantom{0}1 & & $\Zbb[ \theta, \theta' ]$ ($^{ 1 }$)
    & $\Zbb[ \theta, \theta' ]$($^{ 1 }$) \\
    129 & \hphantom{0}8 & & postacią ($^{ 1 }$) & postacią($^{ 1 }$) \\
    135 & \hphantom{0}2 & & topologicznych ($^{ 1 }$)
    & topologicznych($^{ 1 }$) \\
    138 & \hphantom{0}8 & & \textit{prymarnego} ($^{ 1 }$)
    & \textit{prymarnego}($^{ 1 }$) \\
    \hline
  \end{tabular}





  \newpage

  \begin{tabular}{|c|c|c|c|c|}
    \hline
    Strona & \multicolumn{2}{c|}{Wiersz} & Jest
                              & Powinno być \\ \cline{2-3}
    & Od góry & Od dołu & & \\
    \hline
    138 & 10 & & wykazuje ($^{ 2 }$) & wykazuje($^{ 2 }$) \\
    138 & 11 & & pierścieniach ($^{ 3 }$) & pierścieniach($^{ 3 }$) \\
    142 & \hphantom{0}5 & & algebraicznej ($^{ 1 }$)
    & algebraicznej($^{ 1 }$) \\
    142 & & \hphantom{0}7 & wcześniej ($^{ 2 }$) & wcześniej($^{ 2 }$) \\
    143 & & 16 & lokalnemu ($^{ 2 }$) & lokalnemu($^{ 2 }$) \\
    144 & & 16 & bezwzględnych ($^{ 1 }$) & bezwzględnych($^{ 1 }$) \\
    146 & \hphantom{0}3 & & najogólniejszych ($^{ 1 }$)
    & najogólniejszych($^{ 1 }$) \\
    146 & \hphantom{0}7 & & normalizacji ($^{ 2 }$)
    & normalizacji($^{ 2 }$) \\
    149 & & 14 & sformułowań ($^{ 1 }$) & sformułowań($^{ 1 }$) \\
    150 & \hphantom{0}6 & & (31)) ($^{ 1 }$) & (31))($^{ 1 }$) \\
    150 & 12 & & algebr ($^{ 2 }$) & algebr($^{ 2 }$) \\
    151 & \hphantom{0}8 & & algebry ($^{ 1 }$) & algebry($^{ 1 }$) \\
    151 & & 14 & 274) ($^{ 2 }$) & 274)($^{ 2 }$) \\
    152 & 17 & & nilpotentnego ($^{ 1 }$) & nilpotentnego ($^{ 1 }$) \\
    154 & \hphantom{0}3 & & dalej ($^{ 1 }$) & dalej ($^{ 1 }$) \\
    154 & 16 & & podstawowego ($^{ 2 }$) & podstawowego ($^{ 2 }$) \\
    155 & \hphantom{0}1 & & 102) ($^{ 1 }$) & 102) ($^{ 1 }$) \\
    155 & \hphantom{0}3 & & centrum ($^{ 2 }$) & centrum($^{ 2 }$) \\
    155 & & \hphantom{0}7 & teorii ($^{ 3 }$) & teorii($^{ 3 }$) \\
    156 & & \hphantom{0}9 & uwagi ($^{ 1 }$) & uwagi($^{ 1 }$) \\
    157 & \hphantom{0}1 & & algebra ($^{ 1 }$) & algebra($^{ 1 }$) \\
    157 & \hphantom{0}3 & & ówdzie ($^{ 2 }$) & ówdzie($^{ 2 }$) \\
    157 & 13 & & minimalnemu ($^{ 3 }$) & minimalnemu($^{ 3 }$) \\
    157 & 20 & & liniowej ($^{ 4 }$) & liniowej($^{ 4 }$) \\
    158 & & 17 & homologicznej ($^{ 1 }$) & homologicznej($^{ 1 }$) \\
    158 & & 15 & minimalnego ($^{ 2 }$) & minimalnego($^{ 2 }$) \\
    159 & 13 & & dydaktycznych ($^{ 1 }$) & dydaktycznych($^{ 1 }$) \\
    160 & \hphantom{0}6 & & euklidesowego~~($^{ 1 }$)
    & euklidesowego($^{ 1 }$) \\
    160 & 14 & & odróżnienia ($^{ 2 }$) & odróżnienia($^{ 2 }$) \\
    162 & \hphantom{0}2 & & 249)) & 259) \\
    162 & \hphantom{0}6 & & wieku ($^{ 1 }$) & wieku($^{ 1 }$) \\
    162 & 12 & & przeciwnego ($^{ 2 }$) & przeciwnego($^{ 2 }$) \\
    \hline
  \end{tabular}





  \newpage

  \begin{tabular}{|c|c|c|c|c|}
    \hline
    Strona & \multicolumn{2}{c|}{Wiersz} & Jest
                              & Powinno być \\ \cline{2-3}
    & Od góry & Od dołu & & \\
    \hline
    164 & \hphantom{0}1 & & niezmiennikiem ($^{ 1 }$)
    & niezmiennikiem($^{ 1 }$) \\
    164 & \hphantom{0}5 & & oczywistego ($^{ 2 }$) & oczywistego($^{ 2 }$) \\
    166 & \hphantom{0}4 & & 315) ($^{ 1 }$) & 315)($^{ 1 }$) \\
    167 & 14 & & algebraicznej ($^{ 1 }$) & algebraicznej($^{ 1 }$) \\
    167 & & 10 & 371) ($^{ 2 }$) & 371)($^{ 2 }$) \\
    168 & 10 & & U~~Ponceleta & U~Ponceleta \\
    168 & & \hphantom{0}9 & rezultatów ($^{ 1 }$) & rezultatów($^{ 1 }$) \\
    169 & \hphantom{0}3 & & 515) ($^{ 1 }$) & 515)($^{ 1 }$) \\
    170 & \hphantom{0}5 & & jej ($^{ 1 }$) & jej($^{ 1 }$) \\
    171 & & 13 & podobieństw ($^{ 1 }$) & podobieństw($^{ 1 }$) \\
    173 & 14 & & niezmienników ($^{ 2 }$) & niezmienników($^{ 2 }$) \\
    182 & & \hphantom{0}4 & stwarzając & dające \\
    187 & 12 & & poprzedników ($^{ 1 }$) & poprzedników($^{ 1 }$) \\
    187 & 14 & & odkrycia ($^{ 2 }$) & odkrycia($^{ 2 }$) \\
    188 & 16 & & jedności ($^{ 1 }$) & jedności($^{ 1 }$) \\
    189 & & 18 & wielkość ($^{ 2 }$) & wielkość($^{ 2 }$) \\
    190 & \hphantom{0}6 & & wygodnej ($^{ 1 }$) & wygodnej($^{ 1 }$) \\
    191 & \hphantom{0}4 & & niewymierność ($^{ 1 }$)
    & niewymierność($^{ 1 }$) \\
    191 & & 11 & geometrycznej ($^{ 3 }$) & geometrycznej($^{ 3 }$) \\
    196 & & \hphantom{0}8 & rzeczywiste ($^{ 1 }$) & rzeczywiste($^{ 1 }$) \\
    199 & \hphantom{0}9 & & ciągłość) ($^{ 1 }$) & ciągłość)($^{ 1 }$) \\
    201 & \hphantom{0}8 & & wymiaru) ($^{ 1 }$) & wymiaru)($^{ 1 }$) \\
    202 & \hphantom{0}9 & & płaszczyzny ($^{ 1 }$) & płaszczyzny($^{ 1 }$) \\
    202 & & \hphantom{0}6 & 1764) ($^{ 2 }$) & 1764)($^{ 2 }$) \\
    203 & \hphantom{0}8 & & algebry~~($^{ 1 }$) & algebry($^{ 1 }$) \\
    208 & \hphantom{0}4 & & parazwarta ($^{ 1 }$) & parazwarta($^{ 1 }$) \\
    215 & \hphantom{0}3 & & wierze ($^{ 1 }$) & wierze($^{ 1 }$) \\
    227 & \hphantom{0}6 & & całkowaniu~~($^{ 1 }$) & całkowaniu($^{ 1 }$) \\
    242 & \hphantom{0}3 & & 599)) & 599) \\
    252 & & 12 & Euler czasem nawet & nawet Euler czasem \\
    279 & \hphantom{0}2 & & rozszerzenia~($^{ 1 }$)
    & rozszerzenia($^{ 1 }$) \\
    \hline
  \end{tabular}





  % \newpage

  % \begin{tabular}{|c|c|c|c|c|}
  %   \hline
  %   Strona & \multicolumn{2}{c|}{Wiersz} & Jest
  %                             & Powinno być \\ \cline{2-3}
  %   & Od góry & Od dołu & & \\
  %   \hline
  %   % & & & & \\
  %   % & & & & \\
  %   % & & & & \\
  %   % & & & & \\
  %   % & & & & \\
  %   % & & & & \\
  %   % & & & & \\
  %   \hline
  % \end{tabular}

\end{center}

\VerSpaceTwo


\noindent
\StrWierszDol{42}{17} \\
\Jest już od~początku nie przestał  \\
\PowinnoByc od samego początku nie przestawał \\
\StrWierszGora{81}{2--3} \\
\Jest stworzył w~XVII wieku Desargues \\
\PowinnoByc stworzonej w~XVII wieku przez Desargues'a \\
\StrWierszGora{81}{18} \\
\Jest a~wkrótce pod~nazwą zasady dualności \\
\PowinnoByc wkrótce nazwana zasadą dualności \\
\StrWierszDol{97}{5} \\
\Jest w~tym czasie jeszcze \\
\PowinnoByc jeszcze w~tym czasie \\
\StrWierszDol{121}{14} \\
\Jest \textit{wielu innych liczb pierwszych} [niż 5] \\
\PowinnoByc \textit{wielu innych} [niż 5] \textit{liczb pierwszych} \\
\StrWierszGora{122}{8} \\
\Jest którą bez przerwy zajmować~się będzie prawie wyłącznie przez 25 lat \\
\PowinnoByc którą będzie zajmował~się prawie wyłącznym i~niemal bez przerwy
przez 25 lat \\
\StrWierszGora{194}{13} \\
\Jest arytmetyczno-geometrycznej~~($^{ 1 }$) \\
\PowinnoByc arytmetyczno-geometrycznej($^{ 1 }$) \\
\StrWierszGora{240}{10} \\
\Jest pytanie takie postawić \\
\PowinnoByc postawić takie pytanie \\


% ######################################










% ######################################
\newpage

\section{C.B. Boyer \textit{Historia rachunku różniczkowego
    i~całkowego i~rozwój jego pojęć},
  \parencite{Boyer-Historia-rachunku-rozniczkowego-i-calkowego-Pub-1964}}

\label{sec:Boyer-Historia-rachunku-etc}
% ######################################


% ##################
\CenterBoldFont{Uwagi}

\vspace{0em}


\noindent
W~całej książce angielskie zwarte i~treściwe słowo „calculus”
jest zastąpione długim polskim terminem „rachunek różniczkowy
i~całkowy”, co często prowadzi do bardzo niezgrabnych stylistycznie
zdań. Lepiej byłoby wprowadzi do książki, obok powyższego, termin
„analiza matematyczna”, który można ładnie skrócić do „analizy”.

\VerSpaceFour





% ##################
\CenterBoldFont{Uwagi do konkretnych stron}

\vspace{0em}


\noindent
\StrWierszDol{18}{2} Umieszczenie w~tym samym zdaniu stwierdzenia
o~ścisłym sformułowaniu analizy już u~jej początków oraz faktu,
że~matematycy byli niewrażliwi na pewne subtelności, jest dość
karkołomne. Nie wspominając już o~tym, że~te „subtelności” były
często bardzo poważne.

\VerSpaceFour





\noindent
\StrWierszDol{19}{18} Użyte tu określenie „mistycyzm imaginacyjnej
spekulacji” jest wyraźnie niesprawiedliwe w~stosunku do metafizyki,
najważniejszego działu filozofii. Nie~zmienia tego fakt, że~Boyer mógł
mieć na myśli tylko transcendentalną metafizykę ze~szkoły Kanta.

\VerSpaceFour





\noindent
\Str{23} Stwierdzenie, że pewne podstawowe idea zostały usunięte z~analizy
matematycznej, szerzej zaś, z~matematyki, są~mocno wątpliwe.

\VerSpaceFour





\noindent
\StrWierszDol{26}{6} Nazwanie podanych wyżej pojęć „sztucznymi”, ciężko jest
mi nazwać czymś innym, niż nieczułością na piękno matematyki.

\VerSpaceFour





\noindent
\Str{28} Ponieważ drugie wydanie tej książki ukazało~się w~1949~r.,
autor nie mógł wiedzieć, że~w~latach 60 XX w., głównie za sprawą
prac Abrahama Robinsona zostanie sformułowana analiza niestandardowa,
oparta na ścisły pojęciu nieskończenie małych liczb.





% ##################
\newpage

\CenterBoldFont{Błędy}


\begin{center}

  \begin{tabular}{|c|c|c|c|c|}
    \hline
    Strona & \multicolumn{2}{c|}{Wiersz} & Jest
                              & Powinno być \\ \cline{2-3}
    & Od góry & Od dołu & & \\
    \hline
    29  & &  4 & [(376] & ([376] \\
    42  & 13 & & [402 & [402] \\
    % & & & & \\
    % & & & & \\
    % & & & & \\
    % & & & & \\
    % & & & & \\
    % & & & & \\
    \hline
  \end{tabular}

\end{center}

\VerSpaceTwo


\noindent



% ######################################









% ######################################
\newpage

\section{Jeremy J.~Gray \textit{The~Hilbert Challenge}
  \parencite{Gray-The-Hilbert-Challenge-Pub-2000}}

\label{sec:Gray-The-Hilbert-Challenge}
% ######################################


% ##################
% \newpage

\CenterBoldFont{Błędy}


\begin{center}

  \begin{tabular}{|c|c|c|c|c|}
    \hline
    Strona & \multicolumn{2}{c|}{Wiersz} & Jest
    & Powinno być \\ \cline{2-3}
    & Od góry & Od dołu & & \\
    \hline
    \hphantom{0}5 & & \hphantom{0}8 & $6.4$ & $6 \cdot 4$ \\
    24 & & \hphantom{0}2 & $f( x, y ) \!\! := \!\! a x^{ 3 }$
    & $f( x, y ) := a x^{ 3 }$ \\
    25 & & \hphantom{0}2 & $f_{ 2 } \!\! := \!\! ( a c$
    & $f_{ 2 } := ( a c$ \\
    28 & \hphantom{0}6 & & $p_{ 1 }.p_{ 2 }.p_{ n }$
    & $p_{ 1 } \cdot p_{ 2 } \cdot \ldots \cdot p_{ n }$ \\
    36 & & 17 & him & Hilbert \\
    % & & & & \\
    % & & & & \\
    % & & & & \\
    84 & 14 & & $u_{ \HorSpaceOne 1 } \, ( y )$
    & $u_{ \HorSpaceOne 1 }\negHorSpaceOne( y )$ \\

    % & & & & \\
    % & & & & \\
    % & & & & \\
    \hline
  \end{tabular}

\end{center}

\VerSpaceTwo


\noindent
\StrWierszGora{26}{2} \\
\Jest $J \!\!
:= \!\! ( a^{ 2 } d \;\, - \;\, 3 a b c \;\, + \;\, 2 b^{ 3 }) \! x3 \;\; +
\;\; 3( a b d \;\, + \;\, b^{ 2 } c \;\, - \;\, 2 a c^{ 2 } ) x^{ 2 } y$
\\[0.3em]
\PowinnoByc $J := ( a^{ 2 } d - 3 a b c + 2 b^{ 3 } ) x^{ 3 } + 3 ( a b d +
b^{ 2 } c - 2 a c^{ 2 } ) x^{ 2 } y$

% ######################################










% ######################################
\newpage

\section{Red. A.P.~Juszkiewicz
  \textit{Historia matematyki. Tom III: Matematyka XVIII stulecia}
  \parencite{Red-Juszkiewicz-Historia-matematyki-Vol-III-Pub-1977}}

% \vspace{0em}
% ######################################


% % ##################
% \CenterBoldFont{Uwagi}

% \vspace{0em}



% \vspace{\spaceFour}





% % ##################
% \CenterBoldFont{Uwagi do konkretnych stron}


% \noindent
% \Str{}

% \vspace{\spaceFour}





% \noindent
% \StrWd{}{}

% \vspace{\spaceFour}





% \noindent
% \Str{}





% ##################
\CenterBoldFont{Błędy}


\begin{center}

  \begin{tabular}{|c|c|c|c|c|}
    \hline
    Strona & \multicolumn{2}{c|}{Wiersz} & Jest
                              & Powinno być \\ \cline{2-3}
    & Od góry & Od dołu & & \\
    \hline
    264 & & 10 & jej własnymi środkami & środkami spoza niej samej \\
    267 & & \hphantom{0}6 & znał & znał go \\
    283 & & 17 & drugiej. Maclaurin & drugiej, Maclaurin \\
    504 & \hphantom{0}4 & & $z( x, y, y' )$ & $Z( x, y, y' )$ \\
    515 & & \hphantom{0}3 & $( f_{ y' } \, \delta y(_{ b }$
    & $( f_{ y' } \, \delta y )_{ b }$ \\
    % & & & & \\
    % & & & & \\
    % & & & & \\
    \hline
  \end{tabular}

\end{center}

\VerSpaceTwo






% ######################################











% ######################################
\newpage

\section{ % Autor i tytuł dzieła
  R. Rhodes \\
  \textit{Jak powstała bomba atomowa}, \cite{Rhodes00}}

\label{sec:Rhodes-Jak-powstala-etc}
% ######################################


% ##################
\CenterBoldFont{Błędy}


\begin{center}

  \begin{tabular}{|c|c|c|c|c|}
    \hline
    Strona & \multicolumn{2}{c|}{Wiersz} & Jest
                              & Powinno być \\ \cline{2-3}
    & Od góry & Od dołu & & \\
    \hline
    719 & &  6 & 1993 & 1933 \\
    % & & & & \\
    % & & & & \\
    \hline
  \end{tabular}

\end{center}

\VerSpaceTwo


% ######################################










% ######################################
\newpage

\section{A.K. Wróblewski \textit{Historia fizyki},
  \cite{Wroblewski06}}

\label{sec:Wroblewski-Historia-fizyki}
% ######################################


% ##################
\CenterBoldFont{Błędy}


\begin{center}

  \begin{tabular}{|c|c|c|c|c|}
    \hline
    & \multicolumn{2}{c|}{Wiersz} & & \\ \cline{2-3}
    Strona & Od góry & Od dołu & Jest & Powinno być \\
    & (kolumna) & (kolumna) & & \\
    \hline
    203 & 3 (2) & & Jacob 'sGravesande'a & Jacob's Gravesande'a \\
    % & & & & \\
    \hline
  \end{tabular}

\end{center}

\VerSpaceTwo


% ######################################










% ####################################################################
% ####################################################################
% Bibliography

\printbibliography





% ############################
% End of the document

\end{document}
