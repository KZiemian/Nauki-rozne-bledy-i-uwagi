% Autor: Kamil Ziemian

% --------------------------------------------------------------------
% Podstawowe ustawienia i pakiety
% --------------------------------------------------------------------
\RequirePackage[l2tabu, orthodox]{nag} % Wykrywa przestarzałe i niewłaściwe
% sposoby używania LaTeXa. Więcej jest w l2tabu English version.
\documentclass[a4paper,11pt]{article}
% {rozmiar papieru, rozmiar fontu}[klasa dokumentu]
\usepackage[MeX]{polski} % Polonizacja LaTeXa, bez niej będzie pracował
% w języku angielskim.
\usepackage[utf8]{inputenc} % Włączenie kodowania UTF-8, co daje dostęp
% do polskich znaków.
\usepackage{lmodern} % Wprowadza fonty Latin Modern.
\usepackage[T1]{fontenc} % Potrzebne do używania fontów Latin Modern.



% ----------------------------
% Podstawowe pakiety (niezwiązane z ustawieniami języka)
% ----------------------------
\usepackage{microtype} % Twierdzi, że poprawi rozmiar odstępów w tekście.
\usepackage{graphicx} % Wprowadza bardzo potrzebne komendy do wstawiania
% grafiki.
\usepackage{verbatim} % Poprawia otoczenie VERBATIME.
\usepackage{textcomp} % Dodaje takie symbole jak stopnie Celsiusa,
% wprowadzane bezpośrednio w tekście.
\usepackage{vmargin} % Pozwala na prostą kontrolę rozmiaru marginesów,
% za pomocą komend poniżej. Rozmiar odstępów jest mierzony w calach.
% ----------------------------
% MARGINS
% ----------------------------
\setmarginsrb
{ 0.7in} % left margin
{ 0.6in} % top margin
{ 0.7in} % right margin
{ 0.8in} % bottom margin
{  20pt} % head height
{0.25in} % head sep
{   9pt} % foot height
{ 0.3in} % foot sep



% ------------------------------
% Często przydatne pakiety
% ------------------------------
\usepackage{csquotes} % Pozwala w prosty sposób wstawiać cytaty do tekstu.
\usepackage{xcolor} % Pozwala używać kolorowych czcionek (zapewne dużo
% więcej, ale ja nie potrafię nic o tym powiedzieć).



% ------------------------------
% Pakiety do tekstów z nauk przyrodniczych
% ------------------------------
\let\lll\undefined % Amsmath gryzie się z językiem pakietami do języka
% polskiego, bo oba definiują komendę \lll. Aby rozwiązać ten problem
% oddefiniowuję tę komendę, ale może tym samym pozbywam się dużego Ł.
\usepackage[intlimits]{amsmath} % Podstawowe wsparcie od American
% Mathematical Society (w skrócie AMS)
\usepackage{amsfonts, amssymb, amscd, amsthm} % Dalsze wsparcie od AMS
% \usepackage{siunitx} % Do prostszego pisania jednostek fizycznych
\usepackage{upgreek} % Ładniejsze greckie litery
% Przykładowa składnia: pi = \uppi
\usepackage{slashed} % Pozwala w prosty sposób pisać slash Feynmana.
\usepackage{calrsfs} % Zmienia czcionkę kaligraficzną w \mathcal
% na ładniejszą. Może w innych miejscach robi to samo, ale o tym nic
% nie wiem.



% ##########
% Tworzenie otoczeń "Twierdzenie", "Definicja", "Lemat", etc.
\newtheorem{twr}{Twierdzenie} % Komenda wprowadzająca otoczenie
% ,,twr'' do pisania twierdzeń matematycznych
\newtheorem{defin}{Definicja} % Analogicznie jak powyżej
\newtheorem{wni}{Wniosek}



% ----------------------------
% Pakiety napisane przez użytkownika.
% Mają być w tym samym katalogu to ten plik .tex
% ----------------------------
\usepackage{latexshortcuts}
\usepackage{mathshortcuts}



% --------------------------------------------------------------------
% Dodatkowe ustawienia dla języka polskiego
% --------------------------------------------------------------------
\renewcommand{\thesection}{\arabic{section}.}
% Kropki po numerach rozdziału (polski zwyczaj topograficzny)
\renewcommand{\thesubsection}{\thesection\arabic{subsection}}
% Brak kropki po numerach podrozdziału



% ----------------------------
% Ustawienia różnych parametrów tekstu
% ----------------------------
\renewcommand{\arraystretch}{1.2} % Ustawienie szerokości odstępów między
% wierszami w tabelach.



% ----------------------------
% Pakiet "hyperref"
% Polecano by umieszczać go na końcu preambuły.
% ----------------------------
\usepackage{hyperref} % Pozwala tworzyć hiperlinki i zamienia odwołania
% do bibliografii na hiperlinki.





% ####################################################################
% Początek dokumentu
\begin{document}
% ####################################################################





% ######################################## Tytuł całego tekstu
\Main{Geometria różniczkowa --~błędy i~uwagi}
% ########################################




,,P\ldots'' oznacza, że w wydaniu ,,\ldots'' błąd został poprawiony.\\




% ##################
\Work{ % Tytuł i autor dzieła
  William M. Boothby \\
  ,,An~Introduction to~Differentiable Manifolds and~Riemannian
  Geometry'', \cite{BooyhbyIntroductionToDifferentiableManifolds86} }


\CenterTB{Uwagi}

\start \StrWd{233}{3} Zamiast $dy/dx$ powinno być $d g( x )/dx$.
To~mogłoby ograniczyć nieporozumienia związane z~tym wzorem.

\CenterTB{Błędy}
\begin{center}
  \begin{tabular}{|c|c|c|c|c|}
    \hline
    & \multicolumn{2}{c|}{} & & \\
    Strona & \multicolumn{2}{c|}{Wiersz} & Jest
                              & Powinno być \\ \cline{2-3}
    & Od góry & Od dołu & & \\
    \hline
    233 &  6 & & $\{\rho, \theta, \vp)$ & $\{ ( \rho, \theta, \vp )$ \\
    % & & & & \\
    % & & & & \\
    % & & & & \\
    % & & & & \\
    % & & & & \\
    \hline
  \end{tabular}
\end{center}

\vspace{\spaceTwo}















% ##################
\Work{ % Tytuł i autor dzieła
  J. Gancarzewicz, B. Opozda \\
  ,,Wstęp do geometrii różniczkowej'', \cite{GO03} }


\CenterTB{Uwagi}

\start \Str{17} Nazwa \emph{płaszczyzna styczna} pochodzi stąd,
że~zawiera ona wektor styczny do krzywej $\mb{t}$, ponadto jeśli dana
krzywa bez punktów wyprostowania jest płaska, to zawiera~się właśnie
w~tej płaszczyźnie. \emph{Płaszczyzna prostopadła} jest natomiast
prostopadła do~wektora stycznego do krzywej. Nie potrafię jednak
wyjaśnić skąd~się wzięła nazwa \emph{płaszczyzna prostująca}.

\vspace{\spaceFour}


\start \Str{31} \tb{Twierdzenie 3.7.} W~dowodzie faktu, że~z~równości
typu $\mb{t} \cdot \mb{t}' = 0$ wynik, iż~moduł $\mb{t}$ jest stały,
oprócz wzorów (3.5) przyjmowano chyba milcząco, że~dla każdej wartości
parametru $s$ wektory $\mb{t}$, $\mb{n}$ i~$\mb{b}$ są ortogonalne, co
jednak nie jest udowodnione. Z~dowodem relacji ortogonalności sprawa
wygląda chyba podobnie. \Dok

\vspace{\spaceFour}


\start \Str{35} W~żadnym z~podanych tu przykładów, nie udowodniono,
że~budowane mapy, jako odwzorowania,~są homomorfizmami między
odpowiednimi przestrzeniami topologicznymi. Pokazanie tego nie~może
być jednak trudne. \Dok

\CenterTB{Błędy}
\begin{center}
  \begin{tabular}{|c|c|c|c|c|}
    \hline
    & \multicolumn{2}{c|}{} & & \\
    Strona & \multicolumn{2}{c|}{Wiersz} & Jest
                              & Powinno być \\ \cline{2-3}
    & Od góry & Od dołu & & \\
    \hline
    19 & 16 & & $\gamma'''( s_{ 0 } )$ & $\gamma^{ (4) }( s_{ 0 } )$ \\
    31 &  3 & & $x_{ o }$ & $x_{ 0 }$ \\
    37 & & 7 & $\left( \frac{ 4 r u_{ 1 } }{ \norm{ u }^{ 2 } }, \ldots,
               \frac{ 4 r u_{ n } }{ \norm{ u }^{ 2 } } \right)$
           & $\left( \frac{ 4 r^{ 2 } u_{ 1 } }{ \norm{ u }^{ 2 } },
             \ldots, \frac{ 4 r^{ 2 } u_{ n } }{ \norm{ u }^{ 2 } }
             \right)$ \\
    41 & & 14 & oraz$( 1, \bar{ z } )$ & oraz $( 1, \bar{ z } )$ \\
    44 & 17 & & $( ( \varphi( x ) )$ & $( \varphi( x ) )$ \\
    49 & &  4 & $\frac{ \partial ( \varphi^{ i } \circ \varphi ) }
                { u^{ j } }$
           & $\pd{}{ ( \varphi^{ i } \circ \varphi ) }{ u^{ j } }$ \\
           % & & & & \\
           % & & & & \\
    \hline
  \end{tabular}
\end{center}

\noi
\StrWd{40}{3} \\
\Jest
$( \fr{ v_{ 1 } }{ u_{ i } }, \ld, \fr{ v_{ i - 1 } }{ u_{ i } },
\fr{ v_{ i + 1 } }{ u_{ i } }, \ld, \fr{ v_{ n + 1 } }{ u_{ i } } )$ \\
\Pow $( \fr{ v_{ 1 } }{ v_{ i } }, \ld, \fr{ v_{ i - 1 } }{ v_{ i } },
\fr{ v_{ i + 1 } }{ v_{ i } }, \ld, \fr{ v_{ n + 1 } }{ v_{ i } } )$ \\





% ##################
\Work{ % Autor i tytuł dzieła
  Bogusław Gdowski \\
  ,,Elementy geometrii różniczkowej z~zadaniami'',
  \cite{GdowskiElementGeometriiRozniczkowejZZadaniami99} }


\CenterTB{Uwagi}

\start \Str{5} Ponieważ autor nie wyjaśnił, co~dokładnie rozumie przez
przestrzeń euklidesową $E_{ 3 }$ (że~jest ona trójwymiarowa, jest
tu~kwestią drugorzędną), nie jest od~razu jasne czym różni~się ona
od~zbioru wszystkich swoich wektorów $E_{ 3 }^{ \;* }$.

% \vspace{\spaceFour}

\CenterTB{Błędy}
\begin{center}
  \begin{tabular}{|c|c|c|c|c|}
    \hline
    & \multicolumn{2}{c|}{} & & \\
    Strona & \multicolumn{2}{c|}{Wiersz} & Jest
                              & Powinno być \\ \cline{2-3}
    & Od góry & Od dołu & & \\
    \hline
    6   & 17 & & $X \in E_{ 3 }$ & $X \subset E_{ 3 }$ \\
    % & & & & \\
    % & & & & \\
    % & & & & \\
    % & & & & \\
    \hline
  \end{tabular}
\end{center}

\vspace{\spaceTwo}




\Work{
  Wojciech Wojtyński \\
  ,,Grupy i algebry Liego'',\\ wydanie \romannumeral1, GL.}


Uwagi:\\
\begin{itemize}
\item
\item
\item
\item
\item
\item
\item
\item
\item
\item
\end{itemize}

Powinno być:
\begin{itemize}
\item[--] Str. 8. \ldots$( m, n )$\ldots
\item[--] Str. 8. \ldots$m \in M$ jest $f( m ) \in N$\ldots
\item[--] Str. 8. \ldots$M \ni m \rightarrow f( m ) \in N$\ldots
\item[--] Str. 9. \ldots$x_{ 1 }, \, x_{ 2 } \in X$\ldots
\item[--] Str. 15.
  $$\omega( \lambda x_{ 1 } + \mu x_{ 2 }, y ) = \lambda \, \omega(
  x_{ 1 }, y ) + \mu \, \omega( x_{ 2 }, y )$$
\item[--] Str. 20. \ldots tj. $F$ jest formą dwuliniową na $X$ (jako
  przestrzeni nad $\mathbb{R}$)\ldots
\item[--] Str. 25. \ldots$i_{ x_{ 0 } }( t ) = x_{ 0 }$\ldots
\item[--] Str. 30.
  $$\mathfrak{so}_{ + }( n, \mathbb{C} ) = \mathfrak{sl}( n,
  \mathbb{C} ) \cap \mathfrak{o}_{ + }( n, \mathbb{C} ) \textrm{.}$$
\item[--] Str.
\item[--] Str.
\item[--] Str. 99. \ldots także $[ a, b ] \in M$\ldots
\item[--] Str. 99. \ldots $[ a, b ] \in M$ jest równoważne warunkowi
  $[ b, a ] \in M$\ldots
\item[--] Str. 100. \ldots$N \leq L_{ 2 }$, \emph{to}
  $f^{ -1 }( N ) \leq L_{ 1 }$\ldots
\item[--] Str. 100. \ldots$a_{ i } \in f^{ -1 }( b_{ i } )$\ldots
\end{itemize}








% ####################################################################
% ####################################################################
% Bibliografia
\bibliographystyle{alpha} \bibliography{Bibliography}{}



% ############################

% Koniec dokumentu
\end{document}
