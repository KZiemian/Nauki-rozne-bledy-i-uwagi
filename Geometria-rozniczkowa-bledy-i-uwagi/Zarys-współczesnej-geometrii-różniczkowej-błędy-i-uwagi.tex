% Autor: Kamil Ziemian

% --------------------------------------------------------------------
% Podstawowe ustawienia i pakiety
% --------------------------------------------------------------------
\RequirePackage[l2tabu, orthodox]{nag}  % Wykrywa przestarzałe i niewłaściwe
% sposoby używania LaTeXa. Więcej jest w l2tabu English version.
\documentclass[a4paper,11pt]{article}
% {rozmiar papieru, rozmiar fontu}[klasa dokumentu]
\usepackage[MeX]{polski}  % Polonizacja LaTeXa, bez niej będzie pracował
% w języku angielskim.
\usepackage[utf8]{inputenc}  % Włączenie kodowania UTF-8, co daje dostęp
% do polskich znaków.
\usepackage{lmodern}  % Wprowadza fonty Latin Modern.
\usepackage[T1]{fontenc}  % Potrzebne do używania fontów Latin Modern.



% ----------------------------
% Podstawowe pakiety (niezwiązane z ustawieniami języka)
% ----------------------------
\usepackage{microtype}  % Twierdzi, że poprawi rozmiar odstępów w tekście.
\usepackage{graphicx}  % Wprowadza bardzo potrzebne komendy do wstawiania
% grafiki.
\usepackage{verbatim}  % Poprawia otoczenie VERBATIME.
\usepackage{textcomp}  % Dodaje takie symbole jak stopnie Celsiusa,
% wprowadzane bezpośrednio w tekście.
\usepackage{vmargin}  % Pozwala na prostą kontrolę rozmiaru marginesów,
% za pomocą komend poniżej. Rozmiar odstępów jest mierzony w calach.
% ----------------------------
% MARGINS
% ----------------------------
\setmarginsrb
{ 0.7in} % left margin
{ 0.6in} % top margin
{ 0.7in} % right margin
{ 0.8in} % bottom margin
{  20pt} % head height
{0.25in} % head sep
{   9pt} % foot height
{ 0.3in} % foot sep



% ----------------------------
% Często przydatne pakiety
% ----------------------------
\usepackage{csquotes}  % Pozwala w prosty sposób wstawiać cytaty do tekstu.
\usepackage{xcolor}  % Pozwala używać kolorowych czcionek (zapewne dużo
% więcej, ale ja nie potrafię nic o tym powiedzieć).



% ----------------------------
% Pakiety do tekstów z nauk przyrodniczych
% ----------------------------
\let\lll\undefined  % Amsmath gryzie się z językiem pakietami do języka
% polskiego, bo oba definiują komendę \lll. Aby rozwiązać ten problem
% oddefiniowuję tę komendę, ale może tym samym pozbywam się dużego Ł.
\usepackage[intlimits]{amsmath}  % Podstawowe wsparcie od American
% Mathematical Society (w skrócie AMS)
\usepackage{amsfonts, amssymb, amscd, amsthm}  % Dalsze wsparcie od AMS
% \usepackage{siunitx}  % Do prostszego pisania jednostek fizycznych
\usepackage{upgreek}  % Ładniejsze greckie litery
% Przykładowa składnia: pi = \uppi
\usepackage{slashed}  % Pozwala w prosty sposób pisać slash Feynmana.
% \usepackage{calrsfs}  % Zmienia czcionkę kaligraficzną w \mathcal
% na ładniejszą. Może w innych miejscach robi to samo, ale o tym nic
% nie wiem.



% ##########
% Tworzenie otoczeń "Twierdzenie", "Definicja", "Lemat", etc.
\newtheorem{twr}{Twierdzenie}  % Komenda wprowadzająca otoczenie
% ,,twr'' do pisania twierdzeń matematycznych
\newtheorem{defin}{Definicja}  % Analogicznie jak powyżej
\newtheorem{wni}{Wniosek}



% ----------------------------
% Pakiety napisane przez użytkownika.
% Mają być w tym samym katalogu to ten plik .tex
% ----------------------------
\usepackage{geometriarozniczkowa}  % Pakiet napisany konkretnie dla tego
% pliku.
\usepackage{latexshortcuts}
\usepackage{mathshortcuts}



% --------------------------------------------------------------------
% Dodatkowe ustawienia dla języka polskiego
% --------------------------------------------------------------------
\renewcommand{\thesection}{\arabic{section}.}
% Kropki po numerach rozdziału (polski zwyczaj topograficzny)
\renewcommand{\thesubsection}{\thesection\arabic{subsection}}
% Brak kropki po numerach podrozdziału



% ----------------------------
% Ustawienia różnych parametrów tekstu
% ----------------------------
\renewcommand{\arraystretch}{1.25}  % Ustawienie szerokości odstępów między
% wierszami w tabelach.



% ----------------------------
% Pakiet "hyperref"
% Polecano by umieszczać go na końcu preambuły.
% ----------------------------
\usepackage{hyperref}  % Pozwala tworzyć hiperlinki i zamienia odwołania
% do bibliografii na hiperlinki.





% ####################################################################
\begin{document}
% ####################################################################



% ######################################
\Main{ % Tytuł całego tekstu
  Jacek Gancarzewicz \\
  ,,Zarys współczesnej geometrii różniczkowej'', \cite{Gan10} }

\vspace{\spaceTwo} \vspace{\spaceThree}
% ######################################


\CenterTB{Uwagi}

\noi \tb{Rozdział ???} \Dok \start Udowodnić, że pola liniowo
niezależne w jednym punkcie, są liniowo niezależne w pewnym jego
otoczeniu.???

\vspace{\spaceFour}


\start Udowodnić analogiczne twierdzenie dla 1\dywiz form.

\vspace{\spaceFour}


\start Brak definicji modułu.

\vspace{\spaceFour}


\start Brak definicji krzywej całkowej pola wektorowego $X$. Jeżeli
$\varphi_{ t }$ jest lokalny przepływem generującym $X$ na $U$, to
krzywą $\varphi_{ t }( x )$, $x \in U$ nazywamy krzywą całkową pola
$X$.

\vspace{\spaceFour}


\start \Str{18} W książce pada stwierdzenie, że funkcje zdefiniowane
wzorami (1.4), (1.5) i (1.6) są~jedynymi funkcjami liniowymi dla
których odpowiednie diagramy są przemienne. Można to twierdzenie
wzmocnić zauważając, że~funkcje dla których wspomniane diagramy są
przemienne, muszą z~tego względu być liniowe.

\vspace{\spaceFour}


\start \Str{19} Przedstawiony tu diagram jest błędny. Zamiast strzałki
odpowiadającej odwzorowaniu $\pi_{ i }$ wskazującej
od~$X_{ 1 } \oplus \ldots \oplus X_{ k }$ do~$X_{ i }$, powinna być ta
odpowiadająca odwzorowaniu $i_{ i }$ idąca od~$X_{ i }$
do~$X_{ 1 } \oplus \ldots \oplus X_{ k }$ (z tego względu lepiej jest
zmienić tu indeks na $j$).

\vspace{\spaceFour}


\start \Str{20} Ten rozdział jest miejscami zbyt skrótowy. Po
zdefiniowaniu iloczynu tensorowego, odwzorowanie
$f_{ 1 } \otimes \ldots \otimes f_{ n }$ spada z sufitu.

\vspace{\spaceFour}


\start \Str{29} Linie od~11 do~13 licząc od dołu strony są po prostu
skopiowane z książki \cite{Gan04} i~nie mają pokrycia w tekście. To
samo tyczy się linii od 11 i~12 na stronie 31.

\vspace{\spaceFour}


\start \Str{37} Konstrukcja z przykładu 3.24 została już
przeprowadzona, przy innych oznaczeniach, na stronie 21.

\vspace{\spaceFour}


\start \Str{40} Druga część twierdzenia 3.33 i wniosek 3.34
stwierdzają dokładnie to samo.

\vspace{\spaceFour}


\start \Str{48} W diagramie zamiast $p^{ r }_{ s }$ powinno być
$\pi^{ r }_{ s }$.

\vspace{\spaceFour}


\start \Str{61} Drugi akapit można napisać bardziej elegancko. Po
pierwsze zauważmy, że aby zbiór\linebreak
$\{ x \in D_{ f } : f( x ) \in D_{ g } \}$ jest po prostu równy
$f^{ -1 }( D_{ g } )$ i jest otwarty o ile tylko $g$ jest
odwzorowaniem lokalnym, a $f$ odwzorowaniem ciągłym. Wtedy bowiem
$D_{ g }$ jest zbiorem otwartym, a~z~ciągłości $f$ wynika,
że~$f^{ -1 }( D_{ g } )$ jest zbiorem otwartym w $D_{ f }$. Ponieważ
zaś $D_{ f }$ jest otwarty w $X$ więc $f^{ -1 }( D_{ g } )$ również
jest otwarty w $X$. Ta linia rozumowania zdaje się być jednak
zaciemniona przez styl akapitu.

\vspace{\spaceFour}


\start \Str{61} Nie ma ustalonej notacji dla lokalnego homeomorfizmu,
bowiem zamiennie używa się trzech różnych symboli (,,kółko lata po
strzałce'').

\vspace{\spaceFour}


\start \Str{64} Gdyby trzymać się definicji z tej części książki,
$\Gamma'$\dywiz rozmaitość nie byłaby $\Gamma$\dywiz rozmaitością
jeśli $\Gamma' \subsetneq \Gamma$. Jeśli jednak
$\Gamma' \subset \Gamma$ to $\Gamma'$\dywiz struktura ma wszystkie
cechy $\Gamma$\dywiz struktury należy więc rozszerzyć terminologię,
tak że~rozmaitość nazywamy $\Gamma$\dywiz rozmaitością o~ile tylko jej
pseudogrupa zawiera się w~$\Gamma$.

\vspace{\spaceFour}


\start \Str{65} Ten paragraf nie porusza w ogóle kwestii czy na jednej
rozmaitości, jako przestrzeni topologicznej, mogą istnieć różne
$\Gamma$\dywiz struktury. Odpowiedź twierdzącą dają sfery Milnora
(zwane też egzotycznymi sferami).

\vspace{\spaceFour}


\start \Str{65} Brak tu uwagi, że $\mb{C}$ należy traktować jako
$\mathbb{R}^{ 2 }$.

\vspace{\spaceFour}


\start \Str{66} Pusta linia po pierwszym akapicie przykłady 9.6
powinna zostać usunięta.

\vspace{\spaceFour}


Fragment poświęcony zależności pojęć $\Gamma$-struktury i rozmaitości,
oraz wpływowi tego na układ część książki można by napisać dokładniej.
Spróbuj samemu to zrobić.
\\Próba:\\
Niech $\Gamma_{ 1 } \subset \Gamma_{ 2 }$ i ustalmy atlas
$\mathcal{ U }$. Jeżeli rozważymy dane przez ten atlas
$\Gamma$-struktury to w ogólności zachodzi tylko
$\mathcal{ U }^{ * }_{ 1 } \subset \mathcal{ U }^{ * }_{ 2 }$. Przy
przyjętych w tym paragrafie definicjach jeżeli mamy
$\Gamma_{ 2 } = \Gamma^{ \infty }( n )$, to
$\mathcal{ U }^{ * }_{ 1 }$, z reguły nie jest strukturą różniczkową
wyznaczoną przez $\mathcal{ U }$ (bo nie jest maksymalna względem
$\Gamma^{ \infty }( n )$). Z drugiej strony specyfiką geometrii
różniczkowej jest to, że rozważane własności obiektów matematycznych
zależą tylko od tego czy funkcje przejścia danego atlasu należą do
danej pseudogrupy, z tego też względu ,,dziedziczą się w dół''.
Rozsądnie jest więc przyjąć, że dana rozmaitość jest
$\Gamma$-rozmaitością, gdy dla jej pseudogrupy $\Gamma'$, zachodzi
$\Gamma' \subset \Gamma$. Przy takim ujęciu struktura topologiczna
jest różna od różniczkowej, ale rozmaitość różniczkowa
jest również topologiczną.\\

\vspace{\spaceFour}


\start \Str{68} Topologia na wstędze M\"obiusa wprowadzona jest przy
pomocy topologi ilorazowej, choć to pojęcie nie jest nigdzie
zdefiniowane. Podejrzewam, że jest to topologia obrazowa pochodząca od
rzutowania kanonicznego
$\Pi_{ M } : [ 0,1 ] \times \mathbb{ R } \to W$. Należy to jednak
sprawdzić w dobrej książce do~topologii.

\vspace{\spaceFour}


\start \Str{78} Należy podać odnośnik do miejsca gdzie ta własność
przestrzeni parazwartych lub normalnych jest udowodniona.

\vspace{\spaceFour}


\start \Str{78} Brak definicji krzywej. Krzywa na rozmaitości $M$ jest
to odwzorowanie gładkie $\gamma : ( - \epsilon, + \epsilon ) \to M$.

\start \Str{81} Zgodnie za cite{ESTA} suma rozłączna to:
\begin{equation*}
  \bigvee_{ \substack{ x \in M } } T_{ x } M
  = \bigcup_{ \substack{ x \in M } }( x \times T_{ x } M ) \, .
\end{equation*}
Ta definicja wprowadza pewne zmiany w~tekście np.
$\pi^{ -1 }( x ) = x \times T_{ x }M$. Są to jednak nieistotne dla
teorii kwestie formalne.

\vspace{\spaceFour}


\start \Str{83} Przydałby się komentarz odnośnie lematu o klejeniu
struktur topologicznych i różniczkowych.

Próba: \\
Dla dwóch dowolnych punktów $x \neq y$, są możliwe trzy sytuacje. Po
pierwsze może istnieć takie $T_{ \iota } \, ,$ że
$x, \, y \in f_{ \iota }( T_{ \iota } )$. Wtedy, ponieważ
$f_{ \iota }$ jest homeomorfizmem na zbiór otwarty w X i $T_{ \iota }$
jest, jako rozmaitość, przestrzenią Hausdorffa, więc istnieją
rozłączne otoczenia tych punktów w $f_{ \iota } ( T_{ \iota } )$.
Druga sytuacja: istnieją takie $\iota, \kappa$, że
$x \in f_{ \iota } ( T_{ \iota } ), \; y \in f_{ \kappa } ( T_{ \kappa
} )$
i~$f_{ \iota } ( T_{ \iota } ) \cap f_{ \kappa } ( T_{ \kappa } ) =
\emptyset$.
Stanowią więc rozłączne otoczenia dla $x$ i $y$. \\
Problem może więc pojawić się tylko wtedy, gdy nie istnieje takiej
$T_{ \gamma }$, że $x, \, y \in f_{ \gamma }( T_{ \gamma } )$
i~ponadto dla dowolnych $\iota, \kappa$ jest
$f_{ \iota } ( T_{ \iota } ) \cap f_{ \kappa } ( T_{ \kappa } ) \neq
\emptyset$. Wtedy może zajść sytuacja, że dowolne otoczenia $x$ i $y$
zawierają się częściowo w
$f_{ \iota } ( T_{ \iota } ) \cap f_{ \kappa } ( T_{ \kappa } )$ i
zostają one ,,sklejone'' w procedurze wprowadzania topologii. Jest to
dokładnie sytuacja z
przykładu 11.9.\\
Dlaczego więc w omawianych zagadnieniach ten problem się nie
pojawia?\\ We wszystkich występujących przypadkach (czyli wiązek
lokalnie trywialnych) $T_{ \iota }$ jest postaci
$\mc{ U }_{ \iota } \ti F$, gdzie $\mc{ U }$ jest otwartym podzbiorem
rozmaitości, zaś rzutowanie kanoniczne gwarantuje, że jeżeli
$\mc{ U }_{ \iota } \cap \mc{ U }_{ \kappa } = \emptyset$ to
$f_{ \iota } (T_{ \iota }) \cap f_{ \kappa }(T_{ \kappa }) =
\emptyset$. Z tego, że rozmaitość jest przestrzenią Hausdorffa dla
dowolnych $x \neq y$ zawsze możemy znaleźć takie zbiory będące
dziedzinami map.

\vspace{\spaceFour}


\start \Str{101} .???????????????

\vspace{\spaceFour}


\start \Str{103} Dla dowolnej przestrzeni liniowej $V$ wyposażonej
w~topologię ma sens wyrażenie:
\begin{equation*}
  \Lim_{ t \to 0 } \fr{ \vec{ f }( t ) - \vec{ f }( 0 ) }{ t },
\end{equation*}
gdzie $\vec{ f }( t )$ jest funkcją z~$\mb{R}$ w~$V$. Jednak jeżeli
nie założymy, że~ta topologia jest zgodna z~działaniami liniowymi, nie
możemy twierdzić iż~tak zdefiniowana pochodna jest liniowa. W~wypadku
wiązki stycznej jej topologia jest zdefiniowana przez rodzinę
odwzorowań $\widetilde{ \varphi }$, które są liniowe dla każdego
ustalonego $x \in M$, więc definicja pochodnej za pomocą powyższego
wzoru jest poprawna i~jednoznaczna.

\vspace{\spaceFour}


\start Komentarz do procedury liczenia pochodnej funkcji o wartościach
w przestrzeni stycznej.

Zauważmy na początek, że z konstrukcji wiązki stycznej, $T_{ p }M$
jest homeomorficzna z $p \times \mathbb{R}^{ n }$, nie jest więc to
zbiór otwarty w topologi $TM$, możemy na nim jednak wprowadzić
topologię indukowaną i tak będziemy zawsze robili. Trzeba jeszcze
pokazać, tak indukowana topologia na $T_{ p }M$ jest równoważna
wprowadzonej przez liniową funkcję $\widetilde{ \varphi }_{ p }$. Jest
to w istocie zadanie z topologii: mając homomorfizm $h : X \to Y$,
udowodnić,
że jego obcięcie do podzbioru $A \subset X$, z topologią indukowaną, jest homeomorfizmem na zbiór $h( A ) \subset Y$ z topologią indukowaną. Możemy chyba jego dowód w tej chwili pominąć, nie wydaje się on bowiem szczególnie trudny. \\
Teraz liniowość tego homeomorfizmu pozwala nam stwierdzić, że
wprowadzona topologia pochodzi od normy. Takie wprowadzenie normy nie
jest jednak jednoznaczne, niemniej dwa różne homeomorfizmy $T_{ p }M$
z $\mathbb{R}^{ n }$ wprowadzają dwie różne normy, lecz tę samą
topologię. Normy te muszą więc być równoważne. Ponieważ pochodna w
przestrzeni unormowanej jest taka sama dla każdej normy równoważnej,
pewien komentarz do tego faktu można znaleźć w \cite{Sch79}, więc
wynik tego działania jest jednoznaczny.
% \item Str. 104. Dowód wniosku 12.18, można uprościć zauważając, że
%   $( \varphi_{ s } )_{ * }$ możemy traktować jako odwzorowanie
%   liniowe między skończenie wymiarowymi przestrzeniami wektorowymi
%   $T_{ \varphi_{ -s }( x ) }M$ i $T_{ x }M$, z dobrze określoną
%   topologią. Odwzorowanie to jest więc ciągłe, a jak wiemy
%   odwzorowanie liniowe ciągłe jest przemienne z operacją brania
%   pochodnej \cite{LSKAMI}.

\vspace{\spaceFour}


\start \Str{106} W~dowodzie wniosku 12.22 pierwszy raz potrzebne jest
wielokrotnie potem stosowane twierdzenie, a~którego dowód w~książce
nie~jest podany, że~ciągłe pola wektorowe liniowo niezależny w punkcie
$x$ są liniowo niezależne w~pewnym jego otoczeniu. Najprostszy dowód
jaki znam wygląda tak. Jeżeli wybierzemy mapę $( U, \varphi )$ taką,
że~$x \in U$ to ze względu na konstrukcję przestrzeni stycznej,
liniowa niezależność pól jest równoważna liniowej niezależności ich
współrzędnych. Jeśli rozpatrzymy wyznacznik ich współrzędnych to jest
on funkcją ciągłą, jako złożenie funkcji ciągłych, a ponieważ są one
liniowo niezależne w~punkcie $x$ to wyznacznik jest różny od 0,
z~ciągłości wynika zaś, że~jest niezerowy w~pewnym otoczeniu tego
punktu.

\vspace{\spaceFour}


\start \Str{106} Istnienie potrzebnych w~dowodzie wniosku 12.22
liniowo niezależnych pól można teraz wykazać w~następujący sposób. Ze
wektorów $X_{ x_{ 0 } }$, $\partial_{ 1 | { x_{ 0 } } }$, \ldots,
$\partial_{ n | { x_{ 0 } } }$ można wybrać bazę przestrzeni
$T_{ { x_{ 0 } } }M$ zawierająca $X_{ { x_{ 0 } } }$, załóżmy
że~odpowiada ona polom $X$, $\partial_{ 2 }$, \ldots,
$\partial_{ n }$. Na mocy poprzedniego lematu pola te posiadają żądane
własności.

\vspace{\spaceFour}


\start \Str{112} Użyte tu pojęcia $r$\dywiz dżetów są wprowadzone
dopiero na stronie 184.

\vspace{\spaceFour}


\start \Str{112} Nie potrafię powiedzieć w~jaki sposób twierdzenie
13.1 wynika bezpośrednio z~twierdzenia o~funkcji uwikłanej. Pełne
rozwiązanie tego problemu musi się jednak znajdować w~rozdziałach 4
i~5 książki J. Lee \cite{Lee13}.

\vspace{\spaceFour}


\start \Str{127} W~dowodzie twierdzenia 14.13 potrzebne jest
twierdzenie, że~ciągłe przekroje wiązki wektorowej liniowo niezależne
w~jednym punkcie są liniowo niezależne w~pewnym jego otoczeniu. To
jednak można pokazać korzystając z~metody dowodu dla pól wektorowych
przedstawionej w~komentarzu do strony 106. Po zastanowieniu zauważamy,
bowiem że~w~dowodzie korzystaliśmy tylko z~faktu, że~wiązka styczna
jest wiązką wektorową.

\vspace{\spaceFour}


\start \Str{150} Definicja różniczkowalności funktora ze~strony 129,
nie ma chyba sensu w~badanym tu przypadku.

\vspace{\spaceFour}


\start \Str{152} W dowodzie twierdzenia 18.3 korzysta~się
z~różniczkowalności odwzorowania
z~$\mathcal{X}( M ) \to \mathcal{F}_{ x }( M )$, ponieważ jednak
$\mathcal{X}( M )$ jest w~ogólności nienormowalną przestrzenią
nieskończenie wymiarową, to ta pochodna nie wydaje~się być określona.

\vspace{\spaceFour}


\start \Str{159} W dowodzie twierdzenia 18.14 jest luka, nie omówiono mianowicie przypadku gdy rozmaitości $M$ nie można pokryć jedną mapą. Aby to ominąć należy mądrze skorzystać z lokalności. \\Spróbujmy:\\
Pokryjmy całą rozmaitość mapami $U_{ \iota }$, na dziedzinie każdej
mapy możemy zdefiniować odwzorowanie za pomocą wzorów podanych w
dowodzie tego twierdzenia. Czy odwzorowania te można skleić? Tak,
jeżeli bowiem punkt $x$ zawiera się w więcej niż jednej mapie to na
mocy lokalności, wartość $d \omega$ jest taka sama dla dwóch dowolnych
map. Zauważmy, że wynika z tego, iż jest ona jednakowa dla wszystkich
map i wystarczy jeśli użyjemy w rozumowaniu przecięcia dziedzin tylko
dwóch map, a~nie nieskończenie wielu.

\vspace{\spaceFour}


\start \Str{159} W~dowodzie twierdzenia 18.14 brakuje wykazania,
że~zdefiniowane odwzorowanie niezależny od mapy.

\vspace{\spaceFour}


\start \Str{163} W~dowodzie twierdzenia 18.19 należy zauważyć,
że~odwzorowanie spełniające warunki (2)-(4) jest lokalne. Dowód tego
faktu da~się chyba bez problemu przenieść z~poprzednio rozpatrywanych
przypadków odwzorowań lokalnych.

\vspace{\spaceFour}


\start \Str{176} W~dowodzie twierdzenie 20.13 potrzebny jest fakt,
że~podrozmaitości całkowe są spójne, jednak nie ma pojęcia skąd to
wynika.

\vspace{\spaceFour}


\start \Str{176} Nie rozumiem dowodu twierdzenia 2014.

\vspace{\spaceFour}


\start \Str{195} Fakt iż w spójnej składowej grupy Liego $ G $ każdy
element można połączyć krzywą ~ elementem neutralnym nie jest w~żadnym
razie oczywisty.

\vspace{\spaceFour}


\start \Str{202} Uprościć rozumowanie w ostatni akapicie w dowodu
lematu 23.14.

\vspace{\spaceFour}


\start \Str{204} Pierwszy wzór na tej stronie jest niepokojący.

\vspace{\spaceFour}


\start \Str{204} Dowód lematu 24.4 robi niezrozumiały dla mnie użytek
z lematu 24.2.

\vspace{\spaceFour}


\start \Str{207} Coś nie jest tak w~dowodzie twierdzenia 24.5.

\vspace{\spaceFour}


\start \Str{208} Przydałby się komentarz odnośnie sensu i obliczania
wyrażenia $\frac{ d }{ dt } a^{ * }_{ t } a_{ t }$. Ponieważ wszystkie
omawiane w przykładach 24.7-11 rozmaitości są podrozmaitościami
$F^{ n^{ 2 } }$, gdzie $F = \{ \mathbb{ R }, \mathbb{ C } \}$, można
to traktować jako zwykłe różniczkowanie z analizy. Jest to jednak
rozwiązanie mało eleganckie, czy nie można by tu skorzystać z~reguły
Leibniza ze strony 88?

\vspace{\spaceFour}


\start Warto byłoby udowodnić, że różniczka dyfeomorfizmu
$f : M \to N$ jest izomorfizmem między $TM$ i $TN$. Tak samo
warto zaznaczyć, że jest ona izomorfizmem liniowym przestrzeni
$T_{ x }M$ i $T_{ f( x ) }N$.

\vspace{\spaceFour}


\start \Str{219} Na górze strony są błędnie wstawione dwie puste
linie.

\vspace{\spaceFour}


\start \Str{220} W przykładzie 26.5, jest powołanie się na twierdzenie
26.1, jednak na pewno chodzi o~jakieś inne twierdzenie.

\vspace{\spaceFour}


\start \Str{211} Przykład 24.13 został już podany jako przykład 22.5.

\vspace{\spaceFour}


\start \Str{215} Jeżeli w~tej książce używa się symbolu $i_{ G }$
również na oznaczenie odwzorowania $i_{ G }^{ -1 }$, to na tej stronie
trzeba to zaznaczyć.

\vspace{\spaceFour}


\start \Str{220} W~przykładzie 26.5 jest pewien problem. Działanie
grupy jest określone na~bazie, a~nie na przestrzeni totalnej.

\vspace{\spaceFour}


\start \Str{225} Użyte tu odwzorowanie kanoniczne jest zdefiniowane
dopiero za~dwie strony, dokładniej na stronie 227.

\vspace{\spaceFour}


\start \Str{231} W~dowodzie twierdzenia 27.10 brakuje jakiejkolwiek
wzmianki na temat jednoznaczności konstruowanego przekroju.
Uzupełnienie tej luki nie powinno być trudne.

\vspace{\spaceFour}


\start \Str{252} W~tym miejscu po raz pierwszy pojawia~się pochodna
zewnętrzna z~formy o~wartościach wektorowych, jednak nie została ona
zdefiniowana. Najprościej zrobić to chyba w~następujący sposób.



\newpage
\CenterTB{Błędy}
\begin{center}
  \begin{tabular}{|c|c|c|c|c|}
    \hline
    & \multicolumn{2}{c|}{} & & \\
    Strona & \multicolumn{2}{c|}{Wiersz} & Jest
                              & Powinno być \\ \cline{2-3}
    & Od góry & Od dołu & & \\
    \hline
    11  & 11 & & $y b$ & $b y$ \\
    12  &  5 & & kombinacją wektorową &  kombinacją liniową \\
    12  & 10 & & wyrazów ciągu & wektorów \\
    12  & 22 & & oraz & a \\
    15  & & 10 & $y_{ 1 }, \ld , y_{ k }$ & $y_{ 1 }, \ld , y_{ n }$ \\
    17  & &  9 & $\pi_{ 1 }( x ), \ld\ld, u_{ k }$
           & $\pi_{ 1 }( x ), \ld, u_{ k }$ \\
    18  & 12 & & $( 0, \ld, 0, x, 0, \, 0d )$
           & $( 0, \ld, 0, \ub{ x }_{ j }, 0, \ld, 0 )$ \\
    18  & &  1 & $f_{ 1 } \ti \ld \ti f_{ k }$
           & $( f_{ 1 }, \ld, f_{ k } )$ \\
    19  & & 16 & $f : X \ti \ld X \to Y$ & $f : X \ti \ld \ti X \to Y$ \\
    19  & & 13 & $sgn\, \sigma$ & $\sgm\, \sigma$ \\
    19  & &  9 & $X \ti \ld X$ & $X \ti \ld \ti X$ \\
    19  & &  5 & $L^{ k }_{ s } ( X; Y ) = \, L^{ k }_{ s } ( X )$
           & $\dim \, L^{ k }_{ s } ( X; Y ) = \dim \, L^{ k }_{ s } ( X )$ \\
    19  & &  4 & $L^{ k }_{ a } ( X; Y ) = \, L^{ k }_{ a } ( X )$
           & $\dim \, L^{ k }_{ a } ( X; Y ) = \dim \, L^{ k }_{ a } ( X )$ \\
    20  &  7 & & $L_{ a }^{ k }( X )$ & $\dim \, L_{ a }^{ k }( X )$ \\
    24  &  8 & & $X \ot \cdots \ot X$ & $X \ot \ld \ot X$ \\
    24  &  9 & & $R$ & $F$ \\
    24  & 10 & & $X \ot \cdots \ot X$ & $X \ot \ld \ot X$ \\
    26  & &  5 & w twierdzeniu 2.10 & w twierdzeniu 2.9 \\
    27  & 17 & & $v_{ 1 }, \ld v_{ p }$. & $v_{ 1 }, \ld, v_{ p }$. \\
    27  & & 11 & $\la^{ i. k - i }$ & $\la^{ i,\, k - i }$ \\
    29  &  5 & & będą odwzorowaniami & są odwzorowaniami \\
    30  & &  1 & $B_{ n } )$ & $B_{ n } ) )$ \\
    31  & &  5 & $F( B )$ & $\Fc( B )$ \\
    32  &  6 & & kotrakowariantnym & kontrawariantnym \\
    32  &  6 & & kotrakowariantny & kontrawariantny \\
    33  & & 11 & $v$ & $\Vc$ \\
    33  & &  8 & $T_{ 0 }^{ q }( W )$ & $T_{ 0 }^{ p }( W )$ \\
    33  & &  6 & $q = 0$ & $p = 0$ \\
    33  & &  2 & $f^{ * } \ti \ld \ot f^{ * }$
           & $f^{ * } \ot \ld \ot f^{ * }$ \\
    34  & 13 & & $\Kc_{ 1 } \ti \ld \Kc_{ p } \ti \ld$
           & $\Kc_{ 1 } \ti \ld \ti \Kc_{ p } \ti \ld$ \\
    37  & & 19 & $X_{ 1 }, \ld\, , f_{ k }$ & $X_{ 1 }, \ld , f_{ k }$ \\
    38  & &  6 & $( \La^{ k } X )^{ * }$ & $\La^{ k }( X^{ * } )$ \\
    38  & &  1 & $x_{ k }$ & $x_{ k } )$ \\
    40  &  1 & & $X_{ 1 }, \! \ld , f_{ k }$ & $X_{ 1 }, \ld , f_{ k }$ \\
    40  &  4 & & $\Fc( X_{ 1 }, \ld, X_{ k }, Y )( u )$
           & $F( f_{ 1 }, \ld, f_{ k }, g )( u )$ \\
    40  &  6 & & $\Gc( X_{ 1 }, \ld, X_{ k }, Y )( u \ot v )$
           & $\Gc( f_{ 1 }, \ld, f_{ k }, g )( u \ot v )$ \\
    \hline
  \end{tabular}

  \begin{tabular}{|c|c|c|c|c|}
    \hline
    & \multicolumn{2}{c|}{} & & \\
    Strona & \multicolumn{2}{c|}{Wiersz} & Jest
                              & Powinno być \\ \cline{2-3}
    & Od góry & Od dołu & & \\
    \hline
    41  & 14 & & \emph{wymiaru i~ich izomorfizmów} & \emph{wymiaru} \\
    41  & 15 & & $X \ti \ld X$ & $X \ti \ld \ti X$ \\
    41  & & 12 & $\be_{ 1 } : B_{ q }$ & $\be_{ q } : B_{ q }$ \\
    41  & &  4 & $X_{ 1 } = \ld X_{ p }$ & $X_{ 1 } = \ld = X_{ p }$ \\
    41  & &  1 & $( 0. k )$ & $( 0, k )$ \\
    42  &  1 & & $TM \ti_{ M } \ld \ti_{ M } TM$ & $X \ti \ld \ti X$ \\
    42  & &  5 & $R$ & $\R$ \\
    43  &  3 & & \emph{otoczemiem} & \emph{otoczeniem} \\
    43  &  6 & & $1 . \ld, n$ & $1, \ld, n$ \\
    43  &  8 & & $\al. \be$ & $\al, \be$ \\
    46  & 14 & & $\pi_{ k }^{ r } \circ \pi^{ r }_{ s }$
           & $\pi_{ s }^{ r } \circ \pi^{ k }_{ r }$ \\
    48  & &  3 & defin9cję & definicję \\
    49  & 15 & & $a^{ 1 } + ^{ 1 }$ & $a^{ 1 } + b^{ 1 }$ \\
    49  & & 20 & $( \pi^{ 1 } ( b ), \pi^{ 2 } ( b ), \pi^{ 3 } ( b ), \ld )$
           & $( p^{ 1 } ( b ), p^{ 2 } ( b ), p^{ 3 } ( b ), \ld )$ \\
    50  &  6 & & $a_{ 0 }^{ 1 }, a_{ 0 }^{ 2 }, a_{ 0 }^{ 3 }$
           & $x_{ 0 }^{ 1 }, x_{ 0 }^{ 2 }, x_{ 0 }^{ 3 }$ \\
    50  & &  5 & ustala dyfeomorfizm & ustala homeomorfizm \\
    50  & & 1 & $[ X_{ 1 }, Y_{ 1 }], [ X_{ 2 }, Y_{ 2 } ]$
           & $[ X_{ 1 }, X_{ 2 }], [ Y_{ 1 }, Y_{ 2 } ]$ \\
    51  &  4 & & oraz z odwzorowań & oraz odwzorowań \\
    51  & 12 & & pisali & pisali, że \\
    52  & & 12 & ciągłych & odwzorowań ciągłych \\
    52  & &  4 & $A^{ r }$ & $\Ac$ \\
    53  & &  6 & $\Hc$ & $\Hb$ \\
    54  &  1 & & $\Hc$ & $\Hb$ \\
    54  &  3 & & $\Hc$ & $\Hb$ \\
    54  &  5 & & $\ol{ z_{ 1 } }$ & $\ol{ z }_{ 1 }$ \\
    54  &  5 & & $\fr{ z_{ 2 } }{ || z_{ 1 } ||^{ 2 } + || z_{ 2 } ||^{ 2 } } ),.$
           & $-\fr{ z_{ 2 } }{ || z_{ 1 } ||^{ 2 } + || z_{ 2 } ||^{ 2 } } ).$ \\
    54  & &  9 & $j z_{ 2 }$ & $z_{ 2 } j$ \\
    54  & &  6 & $\ol{ z_{ 1 } }$ & $\ol{ z }_{ 1 }$ \\
    55  &  5 & & $\Ac$ & $\Ab$ \\
    55  &  5 & & $\Hc \ti \Hc$ & $\Hb \ti \Hb$ \\
    55  &  8 & & $\Ac$ & $\Ab$ \\
    55  & 15 & & $\Ac$ & $\Ab$ \\
    55  & 17 & & $\Ac$ & $\Ab$ \\
    55  & 17 & & $\fr{ ! }{ || q_{ 1 } ||^{ 2 } + || q_{ 2 } ||^{ 2 } }$
           & $\fr{ 1 }{ || q_{ 1 } ||^{ 2 } + || q_{ 2 } ||^{ 2 } }$ \\
    55  & 17 & & $\ol{ q_{ 1 } }$ & $\ol{ q }_{ 1 }$ \\
    55  & & 16 & $\Ac$ & $\Ab$ \\
    55  & 13 & & $V_{ \Hc }$ & $V_{ \Hb }$ \\
    55  & 13 & & $V_{ \Ac }$ & $V_{ \Ab }$ \\
    \hline
  \end{tabular}

  \begin{tabular}{|c|c|c|c|c|}
    \hline
    & \multicolumn{2}{c|}{} & & \\
    Strona & \multicolumn{2}{c|}{Wiersz} & Jest
                              & Powinno być \\ \cline{2-3}
    & Od góry & Od dołu & & \\
    \hline
    55  & 17 & & $V_{ \Hc }$ & $V_{ \Hb }$ \\
    55  & 17 & & $V_{ \Ac }$ & $V_{ \Ab }$ \\
    55  & 18 & & $V_{ \Hc }$ & $V_{ \Hb }$ \\
    55  & 18 & & $V_{ \Ac }$ & $V_{ \Ab }$ \\
    55  & & 15 & $V_{ \Hc }$ & $V_{ \Hb }$ \\
    55  & & 15 & $V_{ \Ac }$ & $V_{ \Ab }$ \\
    55  & & 13 & $V_{ \Hc }$ & $V_{ \Hb }$ \\
    58  &  3 & & $z$. & $z$, \\
    58  &  5 & & $det f$ & $\det f$ \\
    58  &  9 & & $T( M )$ & $T( X )$ \\
    58  & &  4 & $( b.c )$ & $( b, c )$ \\
    58  & &  2 & $\Ac.\Bc$ & $\Ac, \Bc$ \\
    61  & & 15 & dopuścić do & dopuścić je do \\
    61  & & 11 & homeomorfizm & homeomorfizmem \\
    61  & &  6 & $\Ga$, & $\Ga$; \\
    62  &  3 & & $\id_{ U } \circ \vp$ & $\vp \circ \id_{ U }$ \\
    62  & 20 & & bijekcją & bijekcją klasy $\C^{ \infty }$ \\
    62  & 21 & & różniczkowalne & klasy $\C^{ \infty }$ \\
    63  &  5 & & $i > p$ i $j \leq q$ & $i \leq p$ i $j > p$ \\
    64  & & 21 & przestrzeni & niepustej przestrzeni \\
    64  & &  7 & $\Ga^{ \infty }( n )$-strukturą
           & $\Ga^{ \infty }( n )$-strukturze \\
    65  &  1 & & $\Ga$\dywiz rozmaitością
           & $\Ga$\dywiz rozmaitość \\
    65  &  3 & & $\Ga^{ \infty }( n )$-strukturze
           & $\Ga^{ \infty }( n )$-strukturą \\
    65  &  6 & & $\Ga^{ \infty }( n )$-struktura
           & $\Ga^{ \infty }( n )$-strukturą \\
    65  & &  7 & $\Hom_{ \: F }$ & $\Hom_{ F }$ \\
    67  & &  4 & $\big( \fr{ 4 r u_{ 1 } }{ \norm{ u } }, \ld,
                 \fr{ 4 r u_{ n } }{ \norm{ u } } \big),$
           & $\left( \fr{ 4 r^{ 2 } u_{ 1 } }{ \norm{ u } }, \ld,
             \fr{ 4 r^{ 2 } u_{ n } }{ \norm{ u } } \right)\! ,$ \\
    70  &  2 & & oraz$( 1, \bar{ z } )$ & oraz $( 1, \bar{ z } )$ \\
    70  & & 15 & $\la \circ f$ & $f \circ \la$ \\
    70  & & 15 & $M( q, p + q; Q)$ & $M( q, p + q; q)$ \\
    70  & &  1 & $a$ & $A$ \\
    71  & 21 & & $P.q$ & $P, Q$ \\
    71  & 21 & & $p_{ ) }$ & $p )$ \\
    71  & 18 & & $P' Q'$ & $P', Q'$ \\
    71  & & 16 & $G_{ pq }$ & $G_{ p, q }$ \\
    71  & & 15 & $\P_{ n }[ \R )$ & $\P_{ n }( \R )$ \\
    71  & &  5 & więc topologia na & więc \\
    71  & &  5 & $P_{ n }( \R )$ & $\P_{ n }( \R )$ \\ 
    \hline
  \end{tabular}

  \begin{tabular}{|c|c|c|c|c|}
    \hline
    & \multicolumn{2}{c|}{} & & \\
    Strona & \multicolumn{2}{c|}{Wiersz} & Jest
                              & Powinno być \\ \cline{2-3}
    & Od góry & Od dołu & & \\
    \hline
    72  &  2 & & $\fr{ u_{ 1 } }{ u_{ i } }, \ld. \fr{ u_{ i - 1 } }{ u_{ i } }$
           & $\fr{ u_{ 1 } }{ u_{ i } }, \ld, \fr{ u_{ i - 1 } }{ u_{ i } }$ \\
    72  & & 15 & $\R^{ n }$ & $\R^{ m }$ \\
    72  & &  1 & w & w tym opracowaniu \\
    72  & &  1 & będziemy & będziemy się \\
    74  & &  8 & \emph{klasy klasy} & \emph{klasy} \\
    75  &  9 & & izomorficzna & dyfeomorficzna \\
    76  & 14 & & $\vp^{ -1 }( \supp \, \wt{ \la } )$ & $U$ \\
    76  & &  4 & $\ol{ \, V }$ & $\ol{ V }$ \\
    77  &  3 & & $\ol{\, V}_{ i }$ & $\ol{ V }_{ i }$ \\
    77  &  8 & & $\vp^{ -1 }( \supp \, \la_{ i } )$ & $U_{ i }$ \\
    77  & 20 & & $\supp \, \la$ & $U$ \\
    78  &  2 & & $\ol{ V_{ \kappa } }$ & $\ol{ V }_{ \kappa }$ \\
    78  &  4 & & $\ol{ \wt{ U }_{ \tilde{ \iota } } }$
           & $\ol{ \wt{ U } }_{ \tilde{ \iota } }$ \\
    78  &  6 & & $\wt{ \la_{ \kappa } }$ & $\wt{ \la }_{ \kappa }$ \\
    78  &  7 & & $\wt{ \la_{ \kappa } }$ & $\wt{ \la }_{ \kappa }$ \\
    78  &  7 & & $\ol{ \, V_{ x } }$ & $\ol{ V }_{ x }$ \\
    79  &  5 & & $( ( \vp( x ) )$ & $( \vp( x ) )$ \\
    82  & 10 & & $\wt{ \psi }'_{ x }$
           & $\wt{ \vp }'_{ x }$ \\
    82  & 10 & & $\psi$ & $\vp'$ \\
    82  & & 12 & $f_{ \iota }( U_{ \iota } )$ & $f_{ \iota }( T_{ \iota } )$ \\
    82  & &  1 & dyfeorficzności & dyfeomorficzności \\
    84  &  3 & & $( 0. \fr{ 1 }{ 2 } )$ & $( 0, \fr{ 1 }{ 2 } )$ \\
    84  & & 10 & $FF\Ic$ & $\Ic$ \\
    86  &  5 & & $( ( \vp( x ) )$ & $( \vp( x ) )$ \\
    86  & &  3 & $\R^{ n + 2 }$ & $\R^{ n_{ 2 } }$ \\
    87  &  4 & & ,. & . \\
    88  & &  9 & $( \pr_{ j }|_{ x }$ & $( \pr_{ j }|_{ x } )$ \\
    89  &  7 & & $f_{ x_{ 2 } }^{ 2 }$ & $f_{ x_{ 2 } }^{ ( 2 ) }$ \\
    89  &  7 & & $f_{ x_{ 1 } }^{ 1 }$ & $f_{ x_{ 1 } }^{ ( 1 ) }$ \\
    89  &  7 & & $d f_{ x_{ 2 } }^{ 2 }$ & $d f_{ x_{ 2 } }^{ ( 2 ) }$ \\
    89  &  8 & & $d f_{ x_{ 1 } }^{ 1 }$ & $d f_{ x_{ 1 } }^{ ( 1 ) }$ \\
    90  &  4 & & $\Cc^{ \infty }( M, x )$ & $\Cc^{ \infty }( M, x_{ 0 } )$ \\
    90  & 12 & & $\Cc^{ \infty }( M, x )$
           & $\Cc^{ \infty }( M, x_{ 0 } )$ \\
    90  & 13 & & (11.20) & (11.21) \\
    90  & & 11 & $\fr{ \pr ( \vp^{ i } \circ \vp^{ -1 } ) }{ u^{ j } }$
           & $\fr{ \pr ( \vp^{ i } \circ \vp^{ -1 } ) }{ \pr u^{ j } }$ \\
    90  & &  3 & $\Cc^{ \infty }( M, x )$
           & $\Cc^{ \infty }( M, x_{ 0 } )$ \\
    92  & 13 & & $\xi^{ 1 }(x) \ld$ & $\xi^{ 1 }(x), \ld$ \\
    96  & &  6 & $( f ( \vp^{ -1 }$ & $( f \circ \vp^{ -1 } )$ \\ 
    \hline
  \end{tabular}

  \begin{tabular}{|c|c|c|c|c|}
    \hline
    & \multicolumn{2}{c|}{} & & \\
    Strona & \multicolumn{2}{c|}{Wiersz} & Jest
                              & Powinno być \\ \cline{2-3}
    & Od góry & Od dołu & & \\
    \hline
    96  & &  6 & $t e_{ i } ))$ & $t e_{ i } )$ \\
    103 &  8 & & $\Lim_{ \substack{ t \to 0 } }$
           & $-\Lim_{ \substack{ t \to 0 } }$ \\
    103 & 10 & & $d \vp_{ t }$ & $d_{ \vp_{ -t }( x_{ 0 } ) } \vp_{ t }$ \\
    103 & & 6 & $g_{ -t }$ & $g_{ t }$ \\
    104 &  8 & & $d_{ \vp_{ -t }( x ) } \vp_{ s }$
           & $d_{ \vp_{ -s }( x ) } \vp_{ s }$ \\
    104 &  9 & & $( \big( \vp_{ t } )_{*} Y \big)_{ x }$.
           & $( ( \vp_{ t } )_{ * } Y )_{ x }$ \\
    104 & &  6 & $_{ | t = s }$ & $|_{ t = s }$ \\
    105 & & 12 & $( - \eps, + \eps^{ n } )$ & $( -\eps, +\eps )^{ n }$ \\
    106 &  5 & & $0\ld, 0$ & $0, \ld, 0$ \\
    106 &  6 & & $0\ld, 0$ & $0, \ld, 0$ \\
    107 & 10 & & $( f( \vp_{ t }( x ) )$ & $( f( \vp_{ t }( x ) ) )$ \\
    107 & 10 & & $( \psi_{ t } ( f( x ) )$ & $( \psi_{ t } ( f( x ) ) )$ \\
    107 & 12 & & $f( x ) = \psi_{ t } ( f( x ) )$
           & $\psi_{ t } ( f( x ) )$ \\
    109 & & 15 & $t \to$ & $s \to$ \\
    110 & 13 & & $\dd{}{ c }{ t }( t )$ & $\dd{}{ c }{ t }( 0 )$ \\
    110 & 13 & & $\pd{}{ a }{ t }( t, b( t ) )$
           & $\pd{}{ a }{ t }( 0, b( 0 ) )$ \\
    110 & 13 & & $\pd{}{ a }{ { u^{ i } } }( t, b( t ) )
                 \dd{}{ { b^{ i } } }{ t }( t )$
           & $\pd{}{ a }{ { u^{ i } } }( 0, b( 0 ) )
             \dd{}{ { b^{ i } } }{ t }( 0 )$ \\
    111 &  3 & & $b^{ I }$ & $b^{ i }$ \\
    111 &  4 & & $b^{ I }$ & $b^{ i }$ \\
    111 & &  8 & $\dd{}{ c }{ t }$ & $\dd{}{ g }{ t }$ \\
    111 & &  7 & $\dd{ 2 }{ c }{ t }$ & $\dd{ 2 }{ g }{ t }$ \\
    111 & &  7 & $\wt{ b }$ & $\wt{ b }^{ i }$ \\
    111 & &  6 & $\dd{}{ \wt{ b } }{ t }$
           & $\dd{}{ \wt{ b }^{ i } }{ t }$ \\
    112 & &  5 & $\dd{}{ c }{ t }$ & $\dd{}{ g }{ t }$ \\
    112 & &  6 & $\dd{ 2 }{ c }{ t }$ & $\dd{ 2 }{ g }{ t }$ \\
    118 & & 16 & $(T$ & $T$ \\
    119 &  2 & & $\al\be$ & $\al, \be$ \\
    119 &  4 & & $\al\be$ & $\al, \be$ \\
    120 & 12 & & $( 0. \ld, 0 )$ & $( 0, \ld, 0 )$ \\
    121 & &  5 & $E -$ & $E =$ \\
    122 & 19 & & skończonej & trywialnej \\
    123 &  2 & & $p_{ 1 } \circ f_{ 1 } = p_{ 1 } \circ f_{ 2 }$
           & $f_{ 1 } \circ p_{ 1 } = f_{ 2 } \circ p_{ 1 }$ \\
    123 &  5 & & ze & że \\
    123 &  6 & & $p$ & $p_{ 1 }$ \\
    123 & 20 & & \emph{homomorizmami} & \emph{homomorfizmami} \\
    124 &  6 & & $p_{ 2 }( F( w ) )$ & $p_{ 2 }( w )$ \\
    124 & 14 & & $p_{ 1 }$ & $\vp_{ 1 }$ \\
    124 & 14 & & $p_{ 2 }$ & $\vp_{ 2 }$ \\
    \hline
  \end{tabular}


  \begin{tabular}{|c|c|c|c|c|}
    \hline
    & \multicolumn{2}{c|}{} & & \\
    Strona & \multicolumn{2}{c|}{Wiersz} & Jest
                              & Powinno być \\ \cline{2-3}
    & Od góry & Od dołu & & \\
    \hline
    126 & &  3 & $\im \, F_{ x }$ & $\dim\im \, F_{ x }$ \\
    127 & &  6 & honomorfizmu & homomorfizmu \\
    127 & &  3 & $\ker F_{ t }$ & $\im \, F_{ t }$ \\
    129 & 12 & & $F$ & $\Fc$ \\
    129 & 15 & & $T^{ * }X$ & $X^{ * }$ \\
    129 & 20 & & $X \ti \cdot \ti X$ & $X \ti \ld \ti X$ \\
    129 & 22 & & $X \ti \cdot \ti X$ & $X \ti \ld \ti X$ \\
    129 & 24 & & $X \ti \cdot \ti X$ & $X \ti \ld \ti X$ \\
    129 & 25 & & $( 0, 2 )$ & $( 2, 0 )$ \\
    130 & 17 & & $( \psi_{ 2 } )_{ \pi( w ) })$
           & $( \psi_{ 2 } )_{ \pi( w ) }) \big)$ \\
    131 & 11 & & $( \vp \ti J )$ & $( ( \vp \ti J )$ \\
    131 & 11 & & $R^{ n_{ 1 } }$ & $\R^{ n_{ 1 } }$ \\
    131 & 12 & & $R^{ n_{ 1 } }$ & $\R^{ n_{ 1 } }$ \\
    131 & 13 & & $R^{ n_{ 1 } }$ & $\R^{ n_{ 1 } }$ \\
    131 & 14 & & $E_{ ! }$ & $E_{ 1 }$ \\
    134 &  9 & & $\leftrightsquigarrow$ & $\rightsquigarrow$ \\
    136 &  3 & & contrawariantny & kontrawariantny \\
    136 &  6 & & contrawariantny & kontrawariantny \\
    136 &  7 & & $T^{ k }_{ s }M$ & $T^{ p }_{ q }M$ \\
    136 &  7 & & $\bigotimes^{ k }$ & $\bigotimes^{ p }$ \\
    136 &  7 & & $\bigotimes^{ s }$ & $\bigotimes^{ q }$ \\
    136 & & 13 & trym & tym \\
    136 &  3 & & $\Vc_{ 0 }$ & $\Vc_{ 0 }$; \\
    136 &  6 & & $\Vc_{ 0 }$ & $\Vc_{ 0 }$; \\
    136 & 10 & & $\Vc_{ 0 }$ & $\Vc_{ 0 }$. \\
    137 & 16 & & $T^{ * }_{ x }M$ & $T_{ x }M$ \\
    137 & & 11 & $X^{ ! }$ & $X^{ 1 }$ \\
    140 & & 11 & $\om^{ ! }$ & $\om^{ 1 }$ \\
    140 & &  8 & $M$. & $M$, \\
    141 & 11 & & $X^{ * } M$ & $\Xc^{ * }( M )$ \\
    141 & &  2 & $\al^{ q }$ & $\al^{ p }$ \\
    142 & &  5 & $C^{ I }_{ j } t$ & $C^{ i }_{ j } t$ \\
    143 & 10 & & $t \ot \om^{ 1 } \ld \ot_{ p } \ot \ld$
           & $t \ot \om^{ 1 } \ot \ld \ot \om^{ p } \ot \ld$ \\
    143 & 12 & & $t \ot \om^{ 1 } \ld \ot_{ p } \ot \ld$
           & $t \ot \om^{ 1 } \ot \ld \ot \om^{ p } \ot \ld$ \\
    145 & & 13 & ewentualnie innych & innych \\
    148 &  4 & & $\pr_{ 1 }.\pr_{ 2 }$ & $\pr_{ 1 }, \pr_{ 2 }$ \\
    148 & &  6 & $\supp\, \la_{ \iota }$ & $U_{ \iota }$ \\
    149 & 16 & & $\R^{ q }$ & $\R^{ p }$ \\
    \hline
  \end{tabular}

  \begin{tabular}{|c|c|c|c|c|}
    \hline
    & \multicolumn{2}{c|}{} & & \\
    Strona & \multicolumn{2}{c|}{Wiersz} & Jest
                              & Powinno być \\ \cline{2-3}
    & Od góry & Od dołu & & \\
    \hline
    149 & 17 & & $\R^{ n - q }$ & $\R^{ n - p }$ \\
    149 & & 17 & $R^{ 2k }$ & $\R^{ 2k }$ \\
    150 & & 12 & twierdzenie 18.6 & lemat 18.5 \\
    151 & 20 & & $\Fc( f ) \circ f^{ -1 }$
           & $\Fc( f ) \circ \sigma \circ f^{ -1 }$ \\
    151 & 20 & & \emph{obrazem} & \emph{obrazem przekroju} $\sigma$ \\
    151 & &  8 & $\ul{ F( M ) }$ & $\ul{ \Fc( M ) }$ \\
    151 & &  7 & $\ul{ F( M ) }$ & $\ul{ \Fc( M ) }$ \\
    152 & 13 & & $\ul{ F( M ) }$ & $\ul{ \Fc( M ) }$ \\
    152 & &  5 & $-\vp_{ a t }$ & $= \vp_{ a t }$ \\
    153 & 10 & & $( ( \vp_{ t } )_{ * } \sigma \bigg|_{ t = 0 }$
           & $( ( \vp_{ t } )_{ * } \sigma )\bigg|_{ t = 0 }$ \\
    155 &  3 & & $( \sigma_{ 1 }, \sigma_{ 2 } )$
           & $( \sigma_{ 1 }, \sigma_{ 2 } ) )$ \\
    155 & & 12 & \emph{R} & \emph{$\R$} \\
    155 &  6 & & $\Om$ & $\Om_{ M }$ \\
    155 &  7 & & $( ( \vp_{ t } )_{ * } \sigma_{ 1 } \bigg|_{ t = 0 } )$
           & $( ( \vp_{ t } )_{ * } \sigma_{ 1 } )\bigg|_{ t = 0 }$ \\
    155 &  7 & & $( ( \vp_{ t } )_{ * } \sigma_{ 2 } \bigg|_{ t = 0 } )$
           & $( ( \vp_{ t } )_{ * } \sigma_{ 2 } )\bigg|_{ t = 0 }$ \\
    155 & &  2 & $\Om$ & $\Om_{ M }$ \\
    156 &  9 & & $\Tc( X )$ & $\Tc( M )$ \\
    156 & & 13 & $\om( x )$ & $\om( X )$ \\
    156 & & 13 & $\Lc( C^{ 1 }_{ 1 }( \om \ot X )$
           & $\Lc( C^{ 1 }_{ 1 }( \om \ot X ) )$ \\
    156 & & 12 & $( p, g )$ & $( p, q )$ \\
    156 & &  4 & 18.7(1) & 18.3 \\
    157 & 15 & & $\Xc( M ) \ti \ld \Xc( M )$
           & $\Xc( M ) \ti \ld \ti \Xc( M )$ \\
    157 & &  4 & $X_{ 1 } \ot \ld \ot X_{ n }$ & $X_{ 1 }, \ld, X_{ k }$ \\
    158 &  7 & & $\Xc( M ) \ti \ld \Xc( M )$
           & $\Xc( M ) \ti \ld \ti \Xc( M )$ \\
    158 &  8 & & $( t( X_{ 1 }, \ld, X_{ k } )$
           & $( t( X_{ 1 }, \ld, X_{ k } ) )$ \\
    158 & & 12 & R & $\R$ \\
    161 &  5 & & $\pr_{ i_{ \sigma( 1 ) } } d( g_{ 1 } \ld, \pr_{ i_{ \sigma( \al ) } } g_{ \al } \ld$
           & $d( \pr_{ i_{ \sigma( 1 ) } } g_{ 1 } \ld \pr_{ i_{ \sigma( \al ) } } g_{ \al } \ld$ \\
    161 &  7 & & $\pr_{ i_{ \sigma( 1 ) } } g_{ 1 } \ld, \pr_{ j }\ld$
           & $\pr_{ i_{ \sigma( 1 ) } } g_{ 1 } \ld \pr_{ j }\ld$ \\
    162 & & 10 & $\L\om |_{ U }$ & $\L_{ X }\om|_{ U } $ \\
    163 &  7 & & 18.14, punkty (2) oraz (4) & 18.14(4) oraz 18.16(3) \\
    163 & 16 & & $R$ & $\R$ \\
    163 & &  9 & $\Fc_{ 0 }( M )$ & $\Fc^{ 0 }( M )$ \\
    163 & &  8 & $\Fc_{ 0 }( M )$ & $\Fc^{ 0 }( M )$ \\
    163 & &  7 & (3) & (4) \\
    \hline
  \end{tabular}


  \begin{tabular}{|c|c|c|c|c|}
    \hline
    & \multicolumn{2}{c|}{} & & \\
    Strona & \multicolumn{2}{c|}{Wiersz} & Jest
                              & Powinno być \\ \cline{2-3}
    & Od góry & Od dołu & & \\
    \hline
    164 & & 13 & $X_{ \sigma( k + \bar{ k } ) }$ & $X_{ \sigma( k + \bar{ k } ) } )$ \\
    164 & & 12 & $X_{ \sigma( k + \bar{ k } ) }$ & $X_{ \sigma( k + \bar{ k } ) } )$ \\
    164 & &  5 & $X_{ \sigma( k + \bar{ k } ) }$ & $X_{ \sigma( k + \bar{ k } ) } )$ \\
    164 & &  3 & $X_{ \rho_{ \sigma }( \sigma( k + \bar{ k } ) ) }$
           & $X_{ \rho_{ \sigma }( \sigma( k + \bar{ k } ) ) } )$ \\
    164 & &  1 & $X_{ \al( 2 ) ) }, \ld, X_{ \al( k ) ) }$
           & $X_{ \al( 2 ) }, \ld, X_{ \al( k ) } )$ \\
    164 & &  1 & $X_{ \al( k + 1 ) ) }, \ld, X_{ \al( k + \bar{ k } ) ) }$
           & $X_{ \al( k + 1 ) }, \ld, X_{ \al( k + \bar{ k } ) } )$ \\
    165 & &  3 & $( X, Y )$ & $( X, Y, Z )$ \\
    165 & &  3 & $X( \om( Y, Z ) ) Y( \om( X ) )$
           & $X( \om( Y, Z ) )$ \\
    166 & 10 & & $\L_{ { X_{ 0 } } }$ & $( \L_{ { X_{ 0 } } } \om )$ \\
    166 & 10 & & $\ld i_{ { X_{ 0 } } } ) ( X_{ 1 }, \ld$
           & $\ld i_{ { X_{ 0 } } } ) \om( X_{ 1 }, \ld$ \\
    166 & 11 & & $( X_{ 1 }, \ld, X_{ k } )$ & $( X_{ 1 }, \ld,
                                               X_{ k } ))$ \\
    166 & 13 & & $\L_{ { X_{ 0 } } }$ & $( \L_{ { X_{ 0 } } } \om )$ \\
    166 & &  7 & $[ X_{ i }. X_{ j } ]$ & $[ X_{ i }, X_{ j } ]$ \\
    166 & &  5 & $[ X_{ i }. X_{ j } ]$ & $[ X_{ i }, X_{ j } ]$ \\
    166 & &  3 & $[ X_{ i }. X_{ j } ]$ & $[ X_{ i }, X_{ j } ]$ \\
    167 & &  6 & $X_{ k_{ o } }$ & $X_{ k_{ 0 } }$ \\
    168 & 12 & & $R^{ 3 }$ & $\R^{ 3 }$ \\
    168 & & 13 & $\Dc_{ x_{ o } }$ & $\Dc_{ x_{ 0 } }$ \\
    168 & &  4 & $D$\dywiz polem & $\Dc$\dywiz polem \\
    168 & &  2 & $D$\dywiz polem & $\Dc$\dywiz polem \\
    169 &  8 & & $\supp\, \la$ & $U$ \\
    169 &  9 & & $D$\dywiz polem & $\Dc$\dywiz polem \\
    169 & 12 & & $D$ & $\Dc$ \\
    169 & & 14 & 1-Formę & 1-formę \\
    169 & & 11 & $D$\dywiz pola & $\Dc$\dywiz pola \\
    169 & &  7 & $D$ & $\Dc$ \\
    170 &  5 & & $D$\dywiz polem & $\Dc$\dywiz polem  \\
    170 & &  1 & $D$ & $\Dc$ \\
    171 &  5 & & $D$\dywiz pola & $\Dc$\dywiz pola \\
    171 & 10 & & $D$\dywiz polami & $\Dc$\dywiz polami \\
    171 & 20 & & $D$\dywiz polami & $\Dc$\dywiz polami \\
    171 & & 13 & $D$\dywiz polami & $\Dc$\dywiz polami \\
    171 & &  1 & $D$\dywiz pól & $\Dc$\dywiz pól \\
    172 & & 16 & $D$ & $\Dc$ \\
    173 & 14 & & \emph{$D$\dywiz polem} & \emph{$\Dc$\dywiz polem} \\
    173 & 22 & & $D$\dywiz polem & $\Dc$\dywiz polem \\
    173 & & 16 & $\vp_{ t }$ . & $\vp_{ t }$. \\
    173 & &  6 & $D$ & $\Dc$ \\
    \hline
  \end{tabular}

  \begin{tabular}{|c|c|c|c|c|}
    \hline
    & \multicolumn{2}{c|}{} & & \\
    Strona & \multicolumn{2}{c|}{Wiersz} & Jest
                              & Powinno być \\ \cline{2-3}
    & Od góry & Od dołu & & \\
    \hline
    176 & & 16 & $D$\dywiz pól & $\Dc$\dywiz pól \\
    176 & & 13 & $\vp_{ t_{ 1 } }^{ K } \circ \ld \circ \vp_{ t_{ K } }^{ 1 }$
           & $\vp_{ t_{ 1 } }^{ 1 } \circ \ld \circ \vp_{ t_{ K } }^{ K }$ \\
    176 & & 12 & $D$\dywiz pola & $\Dc$\dywiz pola \\
    176 & & 10 & $[ 0. 1 ]$ & $[ 0, 1 ]$ \\
    177 &  9 & & $\vp_{ \al }( \ga ( \tau_{ \al - 1 } )$
           & $\vp_{ \al }( \ga ( \tau_{ \al - 1 } ) )$ \\
    177 & & 11 & $D$ & $\Dc$ \\
    178 & 13 & & $D$ & $\Dc$ \\
    179 &  8 & & $D$\dywiz adoptowaną & $\Dc$\dywiz adoptowaną \\
    179 & 13 & & $\vp_{ t_{ 1 } }^{ ( 1 ) } \circ \ld \vp_{ t_{ 1 } }^{ ( n ) }$
           & $\vp_{ t_{ 1 } }^{ ( 1 ) } \circ \ld \circ \vp_{ t_{ 1 } }^{ ( n ) }$
    \\
    179 & 18 & & $\pd{}{}{ t_{ 1 } }.$ & $\pd{}{}{ t_{ 1 } },$ \\
    180 &  8 & & $( -\eps. +\eps )$ & $( -\eps, +\eps )$ \\
    180 &  9 & & $( -\eps. +\eps )$ & $( -\eps, +\eps )$ \\
    180 & 11 & & $M - t_{ 1 }$ & $M_{ t_{ 1 } }$ \\
    181 & & 12 & & \\
    181 & &  9 & $\left( \Psi( 0, \hat{ \tau }, 0, \ld, 0) \right.$
           & $\left( \Psi( 0, \hat{ \tau }, 0, \ld, 0) \right)$ \\
    182 &  9 & & & \\
    182 & & 14 & $\left( \Psi( 0, \hat{ \tau }, 0, \ld, 0) \right.$
           & $\left( \Psi( 0, \hat{ \tau }, 0, \ld, 0) \right)$ \\
    183 & 14 & & \emph{takimi} & \emph{takich} \\
    183 & & 15 & $,\, ,$ & $,$ \\
    183 & & 16 & $ij$ & $i, j$ \\
    185 &  1 & & $\nu_{ 1 },\!\ld,\! \mu_{ n }$
           & $\nu_{ 1 }, \ld, \mu_{ n }$ \\
    185 & 10 & & od 1 do $r$ & od 0 do $r$ \\
    186 & & 12 & $\be^{ s } = \be^{ r } \circ \pi^{ r }_{ s }$
           & $\be^{ r } = \be^{ s } \circ \pi^{ r }_{ s }$ \\
    187 & & 11 & $R^{ n }$ & $\R^{ n }$ \\
    187 & &  8 & $\R^{ y }$ & $\R^{ m }$ \\
    187 & &  3 & $\R^{ y }$ & $\R^{ m }$ \\
    188 & 13 & & $( ( \vp( x ) )$ & $( \vp( x ) )$ \\
    188 & 18 & & $\psi \circ \circ f$ & $\psi \circ f$ \\
    188 & & 18 & $\psi \circ \circ f$ & $\psi \circ f$ \\
    188 & & 17 & $| \ga$ & $| \ga |$ \\
    194 & & 11 & $r^{ r }_{ 0 }f$ & $j^{ r }_{ 0 }f$ \\
    195 & & 16 & $S^{ 1 } \ti \ld S^{ 1 }$ & $S^{ 1 } \ti \ld \ti S^{ 1 }$ \\
    196 & 15 & & 1,3 & 1, 3 \\
    196 & 18 & & jest wewnętrzne & jest działanie wewnętrznym \\
    196 & 18 & & ale nie jest łączne & jednak nie jest łączne \\
    196 & & 15 & $i_{ G }$ & $i_{ G }^{ -1 }$ \\
    196 & &  7 & $\vp$ & $\vp_{ t }$ \\
    198 &  8 & & $\Lc( A )$ & $\Lc( f )A$ \\
    \hline
  \end{tabular}

  \begin{tabular}{|c|c|c|c|c|}
    \hline
    & \multicolumn{2}{c|}{} & & \\
    Strona & \multicolumn{2}{c|}{Wiersz} & Jest
                              & Powinno być \\ \cline{2-3}
    & Od góry & Od dołu & & \\
    \hline
    198 & &  9 & $\Lc( Y )$ & $\Lc( f )( Y )$ \\
    198 & &  1 & $( d_{ e }( f \circ \ad_{ a_{ t } } )( Y )$
           & $\big( d_{ e }( f \circ \ad_{ a_{ t } } )( Y ) \big)$ \\
    199 & 2 & & $\L( f )( Y )$ & $\L( f )( X )$ \\
    199 & & 13 & $a_{ -t } B a_{ t }$ & $a_{ t } B a_{ -t }$ \\
    199 & &  8 & $R^{ n }$ & $\R^{ n }$ \\
    199 & &  6 & $R^{ n }$ & $\R^{ n }$ \\
    200 & 9 & & $( d_{ e }\ad_{ a_{ t } }( j^{ r }_{ 0 }Y ) ) )$
           & $( d_{ e }\ad_{ a_{ t } }( j^{ r }_{ 0 }Y ) )$ \\
    200 & 10 & & $( \vp_{ t } )_{ * }Y )$ & $( ( \vp_{ t } )_{ * }Y )$ \\
    200 & 16 & & $a_{ 1 }$ , & $a_{ 1 }$, \\
    201 & & 11 & $d \exp_{ G }$ & $d_{ 0 } \exp_{ G }$ \\
    201 & &  7 & $d \exp_{ G }$ & $d_{ 0 } \exp_{ G }$ \\
    201 & &  4 & $d \exp_{ G }$ & $d_{ 0 } \exp_{ G }$ \\
    205 &  3 & & $\Gc$ & $G$ \\
    205 &  6 & & $d_{ o }\Phi$ & $d_{ 0 }\Phi$ \\
    205 &  7 & & $1.2$ & $1, 2$ \\
    205 &  9 & & $i_{ G }$ & $i_{ G }|_{ V_{ i } }$ \\
    205 & 13 & & $\fr{ O( \norm{ \vp( x ), \vp( y ) } ) }
                 { \norm{ \vp( x ), \vp( y ) } }$
           & $O( \norm{ \vp( x ), \vp( y ) } )$ \\
    205 & & 11 & $\fr{ \eta^{ i } }{ u^{ j } }$
           & $\pd{}{ { \eta^{ i } } }{ { u^{ j } } }$ \\
    205 & & 11 & $\fr{ \eta^{ i } }{ v^{ j } }$
           & $\pd{}{ { \eta^{ i } } }{ { v^{ j } } }$ \\
    206 &  5 & & $t v$ & $\exp( t v )$ \\
    206 &  7 & & $t v$ & $\exp( t v )$ \\
    206 &  8 & & \emph{podgrupy} & \emph{grupy} \\
    206 &  9 & & \} , & \}, \\
    207 &  5 & & w $H$ & $H$ \\
    208 & 18 & & $W.$ & $W,$ \\
    207 & 20 & & $w_{ n } \neq 0$ & $w'_{ n } \neq 0$ \\
    208 & & 12 & $( a^{ * }_{ s } a^{ * }_{ t } a_{ t } a_{ s } |$
           & $( a^{ * }_{ s } a^{ * }_{ t } a_{ t } a_{ s } )$ \\
    208 & &  6 & $\SL( n )$ & $\SL( n, \R )$ \\
    209 &  3 & & $a_{ n \sigma( n ) }$ & $\del_{ n \sigma( n ) }$ \\
    209 & 11 & & $\SO( n )$ & $\SO( n, \R )$ \\
    209 & &  2 & $K$ & $M$ \\
    211 & 14 & & $U$ & $U_{ 0 }$ \\
    211 & 17 & & $W \! \cap \! O_{ 0 }$ & $W \! \cap \! U_{ 0 }$ \\
    211 & & 16 & $L_{ \xi } \circ \psi_{ \xi }^{ -1 }$
           & $\vp_{ \xi } \circ i_{ 0 } \circ \psi_{ \xi }^{ -1 }$ \\
    211 & & 15 & $\id_{ \wt{ V }_{ \xi } }$ & $\id_{ W_{ \xi } }$ \\
    211 & & 11 & $\pi \circ f$ & $f \circ \pi$ \\
    212 &  1 & & $\Z$ & $\R$ \\
    \hline
  \end{tabular}

  \begin{tabular}{|c|c|c|c|c|}
    \hline
    & \multicolumn{2}{c|}{} & & \\
    Strona & \multicolumn{2}{c|}{Wiersz} & Jest
                              & Powinno być \\ \cline{2-3}
    & Od góry & Od dołu & & \\
    \hline
    212 &  1 & & opiosany & opisanym \\
    212 &  2 & & Lioego & Liego \\
    212 & &  9 & $M$ & $G$ \\
    212 & &  3 & $ad_{ \xi^{ -1 } }$ & $\ad_{ \xi^{ -1 } }$ \\
    214 & & 19 & $ad_{ a_{ t } }$ & $\ad_{ a_{ -t } }$ \\
    214 & & 18 & $\Ad_{ a_{ -t } }( B_{ e } )$ & $\Ad_{ a_{ -t } }( B )$ \\
    214 & & 18 & $\Ad_{ a_{ t } }( B_{ e } )$ & $\Ad_{ a_{ t } }( B )$ \\
    214 & & 15 & $-d_{ e }\rho_{ x }$ & $d_{ e }\rho_{ x }$ \\
    214 & &  5 & $X \ti \ld X$ & $X \ti \ld \ti X$ \\
    215 &  2 & & $R$ & $\R$ \\
    215 &  3 & & $R$ & $\R$ \\
    215 & 15 & & $i_{ G }$ & $i_{ G }^{ -1 }$ \\
    215 & 16 & & $\Xc( M )$ & $\Xc( G )$ \\
    216 &  1 & & $h_{ 2 }h_{ 2 }$ & $h_{ 1 }h_{ 2 }$ \\
    216 &  5 & & $i_{ G }$ & $i^{ -1 }_{ G }$ \\
    216 &  6 & & $i_{ G }$ & $i^{ -1 }_{ G }$ \\
    216 & 10 & & $i_{ G }$ & $i^{ -1 }_{ G }$ \\
    216 & 10 & & $i_{ G }$ & $i^{ -1 }_{ G }$ \\
    216 & &  4 & powolnego & dowolnego \\
    216 &  2 & & $\pi_{ 2 }$ & $p_{ 2 }$ \\
    217 & & 18 & takim. że & takim, że \\
    218 & 11 & & $\wh{ \vp }$ & $\wt{ \vp }$ \\
    219 &  5 & & $j^{ r }_{ 0 } \ga_{ 0 }$ & $j^{ r }_{ 0 } \ga$ \\
    219 &  8 & & $.\be^{ -1 }( U )$ & $\be^{ -1 }( U )$ \\
    219 &  8 & & $\fr{ \pr ( \vp^{ i } \circ \ga ) }{ u^{ j } }$
           & $\pd{}{ ( { \vp^{ i } } \circ \ga ) }{ { u^{ j } } }$ \\
    219 &  9 & & $j_{ o }^{ r } \ga_{ 0 }$ & $j_{ 0 }^{ r } \ga$ \\
    219 & & 13 & $j_{ 0 }^{ r } \ga_{ 1 } = j_{ 0 }^{ r } \ga_{ 2 }$
           & $\pi^{ r } ( \ga_{ 1 } ) = \pi^{ r }( \ga_{ 2 } )$ \\
    219 & & 12 & $j^{ r }_{ o } \ga_{ 2 } = j^{ r }_{ o } \ga_{ 1 } \cdot j^{ r } \xi$
           & $j^{ r }_{ 0 } \ga_{ 2 } = j^{ r }_{ 0 } \ga_{ 1 } \cdot j^{ r }_{ 0 } \xi$ \\
    219 & &  7 & $\wt{ \vp }$ & $\wt{ \vp }^{ r }$ \\
    220 &  4 & & $j^{ r } \xi$ & $j^{ r }_{ 0 } \xi$ \\
    220 &  5 & & $j^{ r } \xi$ & $j^{ r }_{ 0 } \xi$ \\
    220 &  6 & & główną & włóknistą główną \\
    220 & & 18 & dwóch kartezjański & kartezjański \\
    220 & &  9 & łacznie & łącznie \\
    222 & & 13 & $f_{ p }( p |$ & $f_{ P }( p )$ \\
    223 & 10 & & $\GL( n )$ & $\GL( n ,\R )$ \\
    223 & 16 & & $F^{ 1 }( M )$ & $L^{ 1 }( M )$ \\
    223 & & 12 & $( \xi( 0, \ld$ & $\xi( 0, \ld$ \\
    \hline
  \end{tabular}

  \begin{tabular}{|c|c|c|c|c|}
    \hline
    & \multicolumn{2}{c|}{} & & \\
    Strona & \multicolumn{2}{c|}{Wiersz} & Jest
                              & Powinno być \\ \cline{2-3}
    & Od góry & Od dołu & & \\
    \hline
    223 & & 12 & $\big( \fr{ \pr \xi^{ 1 } }{ u^{ i } }( 0 ), \ld,
                 \fr{ \pr \xi^{ n } }{ u^{ i } }( 0 ) \big)$
           & $\big( \pd{}{ { \xi^{ 1 } } }{ { u^{ i } } }( 0 ), \ld,
             \pd{}{ { \xi^{ n } } }{ { u^{ i } } }( 0 ) \big)$ \\
    223 & & 12 & $v_{ j }( \ga ) \fr{ \pr \xi ^{ j } }{ u^{ i } }( 0 )$
           & $\wt{ \ga }_{ x }( v_{ j }( \ga ) )
             \pd{}{ { \xi ^{ j } } }{ { u^{ i } } }( 0 )$ \\
    224 & & 15 & $f_{ p }$ & $f_{ P }$ \\
    224 & & 15 & $M;, G'$ & $M, G'$ \\
    224 & & 12 & $f_{ p }$ & $f_{ P }$ \\
    224 & & 12 & $M;, G'$ & $M, G'$ \\
    224 & & 11 & $f_{ p }( p )$ & $f_{ P }( p )$ \\
    224 & & 11 & $f_{ P_{ 0 } }( p_{ 0 } )$ & $f_{ P_{ 0 } }( p )$ \\
    224 & & 10 & $f_{ p }$ & $f_{ P }$ \\
    224 & &  9 & $M;, G'$ & $M, G'$ \\
    225 & &  7 & $p_{ 1 } : U \ti G$ & $p_{ 1 } : U \ti \wt{ G }$ \\
    225 & &  7 & $p_{ 2 } : U \ti G$ & $p_{ 2 } : U \ti \wt{ G }$ \\
    227 & & 15 & $( p. f )$ & $( p, f )$ \\
    227 & &  6 & $( \sigma( x ), f ) )$ & $( \sigma( x ), f )$ \\
    227 & &  6 & $( \sigma( x ), f' ) )$ & $( \sigma( x ), f' )$ \\
    227 & &  5 & $\la_{ \xi }( f )$ & $\la_{ \xi^{ -1 } }( f )$ \\
    228 &  2 & & $\wt{ \sigma }^{ -1 }$ & $\wt{ \sigma }'^{ -1 }$ \\
    228 &  3 & & $\wt{ \sigma' }^{ -1 }$ & $\wt{ \sigma }'^{ -1 }$ \\
    228 & & 10 & $\pi( y )$ & $\pi_{ E }( y )$ \\
    229 & & 16 & $L( M 4 )$ & $L( M )$ \\
    229 & & 10 & $\Phi( p, f' )$ & $\Phi( p', f' )$ \\
    229 & &  8 & $f'^{ i } = f^{ j } A^{ i }_{ j }$
           & $f^{ i } = f'^{ j } A^{ i }_{ j }$ \\
    231 &  8 & & grupy liniowej $G$ & grupy $G$ \\
    231 & 12 & & $\Phi^{ -1 }_{ p }( \sigma( \pi( p ) )$
           & $\Phi^{ -1 }_{ p }( \sigma( \pi( p ) ) )$ \\
    231 & & 17 & $\Phi^{ -1 }_{ p \cdot \xi }( \sigma( \pi( p \cdot \xi ) )$
           & $\Phi^{ -1 }_{ p \cdot \xi }( \sigma( \pi( p \cdot \xi ) ) )$ \\
    231 & & 17 & $\Phi^{ -1 }_{ p \cdot \xi }( \sigma( \pi( p ) )$
           & $\Phi^{ -1 }_{ p \cdot \xi }( \sigma( \pi( p ) ) )$ \\
    231 & & 16 & $\Phi^{ -1 }_{ p \cdot \xi }( \sigma( \pi( p ) )$
           & $\Phi^{ -1 }_{ p \cdot \xi }( \sigma( \pi( p ) ) )$ \\
    231 & & 15 & $\Phi^{ -1 }_{ p }( \sigma( \pi( p ) )$
           & $\Phi^{ -1 }_{ p }( \sigma( \pi( p ) ) )$ \\
    231 & & 15 & $A_{ \sigma } \circ A_{ \xi }$
           & $A_{ \sigma } \circ R_{ \xi }$ \\
    231 & & 10 & \emph{grupy liniowej $G$} & \emph{grupy $G$} \\
    231 & &  5 & $\Psi$ & $\Phi$ \\
    231 & &  1 & $\Psi$ & $\Phi$ \\
    231 & &  1 & $\la_{ \xi^{ -1 } }( A( p ) )$
           & $\la_{ \xi^{ -1 } }( A( p ) ) )$ \\
    232 & 11 & & lematy & lematu \\
    232 & 15 & & $\Psi$ & $\Phi$ \\
    232 & & 8 & $\GL^{ + }( n. \R )$ & $\GL^{ + }( n, \R )$ \\
    \hline
  \end{tabular}

  \begin{tabular}{|c|c|c|c|c|}
    \hline
    & \multicolumn{2}{c|}{} & & \\
    Strona & \multicolumn{2}{c|}{Wiersz} & Jest
                              & Powinno być \\ \cline{2-3}
    & Od góry & Od dołu & & \\
    \hline
    232 & & 3 & $\GL^{ + }( n \R )$ & $\GL^{ + }( n, \R )$ \\
    233 &  7 & & $GL^{ + }( n )$ & $\GL^{ + }( n, \R )$ \\
    234 &  2 & & twierdzenia 28.1 & twierdzenia 28.3 \\
    234 &  8 & & $\Ga^{ \infty }( p, g )$ & $\Ga^{ \infty }( p, q )$ \\
    234 &  9 & & $\Ga^{ \infty }( p, g )$\dywiz atlas
           & $\Ga^{ \infty }( p, q )$\dywiz atlas \\
    234 & 11 & & $\GL( q, p )$ & $\GL( p, q )$ \\
    234 & 14 & & $\GL( g, \R )$ & $\GL( q, \R )$ \\
    236 & 21 & & $i_{ g }( x )$ & $i_{ g }( v )$ \\
    236 & 10 & & $A A^{ * }$ & $A^{ * } A$ \\
    236 & &  1 & $A J_{ q } A^{ * }$ & $A^{ * } J_{ q } A$ \\
    237 & &  8 & $L^{ + }( M )$ & $L^{ + }M$ \\
    237 & &  3 & $L^{ + }( M )$ & $L^{ + }M$ \\
    238 &  3 & & $\Ac^{ * }$ & $Ac^{ * }$ \\
    239 & 15 & & $\GL( n, \C )$ & $\GL( n, \C )$ \\
    239 & 20 & & $\GL( n, \C )$ & $\GL( n, \C )$ \\
    240 & & 12 & $( M. G )$ & $( M, G )$ \\
    240 & & 11 & $( M. G )$ & $( M, G )$ \\
    240 & &  8 & $( M. G )$ & $( M, G )$ \\
    240 & &  8 & $( M. G' )$ & $( M, G' )$ \\
    243 & &  3 & włókna & przestrzeni \\
    244 & 15 & & o o jednym & to o jednym \\
    244 & &  6 & $i_{ G }$ & $i^{ -1 }_{ G }$ \\
    245 &  3 & & $\Ad_{ xi^{ -1 } }$ & $\Ad_{ \xi^{ -1 } }$ \\
    245 & 12 & & $d_{ p } \xi$ & $d_{ p } R_{ \xi }$ \\
    245 & 12 & & $i_{ G }$ & $i_{ G }^{ -1 }$ \\
    245 & 13 & & $i_{ G }$ & $i_{ G }^{ -1 }$ \\
    245 & 13 & & $\left. R_{ \xi } \right)$ & $\left. R_{ \xi } ) \right)$ \\
    245 & & 17 & $i_{ G }$ & $i_{ G }^{ -1 }$ \\
    245 & & 17 & $ad_{ \xi^{ -1 } }$ & $\ad_{ \xi^{ -1 } }$ \\
    245 & & 16 & $i_{ G }$ & $i_{ G }^{ -1 }$ \\
    245 & &  2 & $\vp_{ * } \om$ & $\vp^{ * } \om$ \\
    245 & &  2 & $U \ti G$ & $M \ti G$ \\
    246 &  9 & & $\sup \la_{ \iota }$ & $U_{ \iota }$ \\
    246 & 10 & & $\sup$ & $\supp$ \\
    246 & 19 & & $\sup$ & $\supp$ \\
    247 & & 16 & $\om )_{ \sigma( x ) }$ & $\om_{ \sigma( x ) }$ \\
    247 & & 15 & $\om )_{ \sigma }$ & $\om_{ \sigma }$ \\
    \hline
  \end{tabular}

  \begin{tabular}{|c|c|c|c|c|}
    \hline
    & \multicolumn{2}{c|}{} & & \\
    Strona & \multicolumn{2}{c|}{Wiersz} & Jest
                              & Powinno być \\ \cline{2-3}
    & Od góry & Od dołu & & \\
    \hline
    247 & & 12 & $i_{ G }$ & $i_{ G }^{ -1 }$ \\
    247 & &  4 & $h^{ x }$ & $h^{ * }$ \\
    247 & &  4 & $, .$ & $.$ \\
    248 & 14 & & $( i_{ G } \circ ( \rho_{ \sigma( x ) } )^{ -1 }$
           & $i_{ G }^{ -1 } \circ ( d_{ e }\rho_{ \sigma( x ) } )^{ -1 }$ \\
    248 & 15 & & $( M, g )$ & $( M, G )$ \\
    248 & 17 & & $( M, g )$ & $( M, G )$ \\
    248 & 18 & & $( M, g )$ & $( M, G )$ \\
    249 &  5 & & $TP.\Lc(G)$ & $TP, \Lc(G)$ \\
    250 & &  6 & $R_{ \xi }^{ * }$ & $( R_{ \xi } )_{ * }$ \\
    251 & 15 & & \emph{Lie} & \emph{Liego} \\
    251 & 18 & & $t = o$ & $t = 0$ \\
    251 & &  9 & $M.G$ & $M, G$ \\
    252 & 11 & & $M.G$ & $M, G$ \\
    252 & 12 & & $G):$ & $G).$ \\
    252 & 14 & & $G)$ & $G).$ \\
    252 & 15 & & $G)$ & $G).$ \\
    252 & 18 & & $\rho( \xi^{ -1 } \circ \om$
           & $\rho( \xi^{ -1 } ) \circ \om$ \\
    252 & &  4 & $( LM 0$ & $( LM )$ \\
    253 &  3 & & $T_{ X }$ & $T_{ x }M$ \\
    253 &  6 & & $d_{ p } \pi( v_{ i } ) = X_ { i }$
           & $d_{ p } \pi( X_{ i }) = v_ { i }$ \\
    253 & 15 & & $( LM 0$ & $( LM )$ \\
    253 & & 18 & $T_{ X }$ & $T_{ x }M$ \\
    253 & & 17 & $p( \om_{ p }( X_{ 1 }, \ld, X_{ n } ) )$
           & $\om_{ p }( X_{ 1 }, \ld, X_{ n } )$ \\
    253 & &  9 & $T_{ M }$ & $T_{ x }M$ \\
    253 & &  2 & $( 1, k + 1 )$. & $( 1, k + 1 )$, \\
    254 &  3 & & $ij$ & $i, j$ \\
    254 & 10 & & $T_{ X }$ & $T_{ x }M$ \\
    254 & 14 & & $d_{ p }\pi( V_{ i } ) = X_{ i }$
           & $d_{ p }\pi( X_{ i } ) = V_{ i }$ \\
    254 & & 10 & $K_{ x }( v_{ 1 }, \ld, v_{ k } )$
           & $K_{ x }( V_{ 1 }, \ld, V_{ k } )$ \\
    254 & &  9 & $d_{ p }\pi( X_{ i } ) = v_{ i }$
           & $d_{ p }\pi( X_{ i } ) = V_{ i }$ \\
    254 & &  2 & $K_{ x }( v_{ 1 }, \ld, v_{ k } )$
           & $K_{ x }( V_{ 1 }, \ld, V_{ k } )$ \\
    254 & & 15 & $) )^{ i }_{ j }$ & $)^{ i }_{ j }$ \\
    254 & & 14 & $) )^{ i }_{ j }$ & $)^{ i }_{ j }$ \\
    254 & &  9 & $Ad_{ \xi^{ -1 } }$ & $\Ad_{ \xi^{ -1 } }$ \\
    254 & &  6 & $T_{ X }$ & $T_{ x }M$ \\
    \hline
  \end{tabular}

  \begin{tabular}{|c|c|c|c|c|}
    \hline
    & \multicolumn{2}{c|}{} & & \\
    Strona & \multicolumn{2}{c|}{Wiersz} & Jest
                              & Powinno być \\ \cline{2-3}
    & Od góry & Od dołu & & \\
    \hline
    258 & & 13 & $( dR_{ a_{ t } }( X_{ p }^{ H } ) )$
           & $( dR_{ a_{ t } }( X_{ p }^{ H } ) )$ \\
    258 & & 12 & $( R_{ a_{ t } } )_{ * } \psi( X^{ H } ) )_{ p } ) )$
           & $( ( R_{ a_{ t } } )_{ * } \psi( X^{ H } ) )_{ p }$ \\
    259 & & 14 & $( M. G )$ & $( M, G )$ \\
    260 & &  7 & many & mamy \\
    261 &  8 & & $\wt{ \sigma' }$ & $\wt{ \sigma }'$ \\
    262 & 11 & & $\Phi'( p. f' )$ & $\Phi'( p, f' )$ \\
    262 & 11 & & $\Phi''( p. f'' )$ & $\Phi''( p, f'' )$ \\
    262 & & 17 & $\ul{ E' }$ & $\ul{ E }'$ \\
    262 & & 17 & $\ul{ E'' }$ & $\ul{ E }''$ \\
    262 & & 16 & $( \sigma', \sigma'' )_{ = }$
           & $( \sigma', \sigma'' ) ) =$ \\
    262 & &  8 & $\ul{ E' }$ & $\ul{ E }'$ \\
    262 & &  8 & $\ul{ E'' }$ & $\ul{ E }''$ \\
    261 & &  7 & $( \sigma'. \sigma'' )$ & $( \sigma', \sigma'' )$ \\
    262 & &  6 & $( \wt{ \sigma' }, \wt{ \sigma'' ) }$
           & $( \wt{ \sigma }', \wt{ \sigma }'' )$ \\
    262 & &  6 & $\wt{ \sigma' }$ & $\wt{ \sigma }'$ \\
    262 & &  6 & $\wt{ \sigma'' }$ & $\wt{ \sigma }''$ \\
    266 & 19 & & $[ 0. 1 ]$ & $[ 0, 1 ]$ \\
    266 & 20 & & $[ 0. 1 ]$ & $[ 0, 1 ]$ \\
    266 & & 20 & $[ 0. 1 ]$ & $[ 0, 1 ]$ \\
    267 & 8 & & $ad_{ \xi^{ -1 } }$ & $\ad_{ \xi^{ -1 } }$ \\
    267 & & 17 & $\xi_{ \ga^{ - 1 } \cdot \xi_{ \om } \cdot \xi_{ \ga } }$
           & $\xi_{ \ga^{ - 1 } \cdot \om \cdot \ga }$ \\
    267 & &  9 & $\Psi( p ) / \Psi_{ 0 }( p )$
           & $\Phi( p ) / \Phi_{ 0 }( p )$ \\
    267 & &  7 & $\Psi_{ 0 }( p )$ & $\Phi_{ 0 }( p )$ \\
    267 & &  6 & $\Psi( p ) / \Psi_{ 0 }( p )$
           & $\Phi( p ) / \Phi_{ 0 }( p )$ \\
    267 & &  5 & $\Psi( p ) / \Psi_{ 0 }( p )$
           & $\Phi( p ) / \Phi_{ 0 }( p )$ \\
    267 & &  2 & $\Psi_{ 0 }( p )$ & $\Phi_{ 0 }( p )$ \\
    267 & &  1 & $\Psi( p ) / \Psi_{ 0 }( p )$
           & $\Phi( p ) / \Phi_{ 0 }( p )$ \\
    162 & 13 & & i pozostawiamy & pozostawiamy \\
    518 &  3 & & 1762 & 1962 \\
    519 & 14 & & )1977) & (1977) \\
    520 &  8 & & \emph{oparators} & \emph{operators} \\
    521 &  6 & & 1967, 1967. & 1967. \\
    521 & & 16 & 1985) & 1985 \\
    521 & &  7 & Math\'{e}matiques. ; & Math\'{e}matiques, \\
    522 & 10 & & 1077 & 1977 \\
    524 & 11 & & 1077 & 1977 \\
    524 & 18 & & 1859 & 1959 \\
    524 & 19 & & 1063 & 1963 \\
    524 & &  5 & 1065 & 1965 \\
    524 & &  3 & 1073 & 1973 \\
    % & & & & \\
    \hline
  \end{tabular}

\end{center}
\noi \\
\StrWg{72}{2} \\
\Jest % ?????????
\begin{equation*}
  ( \fr{ v_{ 1 } }{ u_{ i } }, \ld, \fr{ v_{ i - 1 } }{ u_{ i } },
  \fr{ v_{ i + 1 } }{ u_{ i } }, \ld, \fr{ v_{ n } }{ u_{ i } } )  
\end{equation*}
\Pow
\begin{equation*}
  \left( \fr{ v_{ 1 } }{ v_{ i } }, \ld, \fr{ v_{ i - 1 } }{ v_{ i } },
    \fr{ v_{ i + 1 } }{ v_{ i } }, \ld, \fr{ v_{ n } }{ v_{ i } } \right)  
\end{equation*}
\StrWg{103}{9} \\
\Jest szybkością\ldots \\
\Pow  wziętą ze znakiem minus szybkością\ldots \\
\StrWd{92}{13} \\
\Jest \emph{pokrycia mapami rozmaitości $M$ jego współrzędne
  w~mapach pokrycia\ldots} \\
\Pow \emph{atlasu na rozmaitości $M$ jego współrzędne
  w~mapach atlasu\ldots} \\
\StrWg{114}{1} \\
\Jest obie kolumny są liniowo niezależne\ldots \\
\Pow  co najmniej jeden element jest niezerowy\ldots \\
\StrWg{114}{1} \\
\Jest obie kolumny są liniowo niezależne\ldots \\
\Pow  co najmniej jeden element jest niezerowy\ldots \\
\StrWg{121}{13} \\
\Jest sprzeczność z~twierdzenie o wartości średniej.\\
\Pow  to sprzeczność z~twierdzeniem o~wartości pośredniej.\\
\StrWd{129}{16} \\
\Jest Powrócimy do\ldots\\
\Pow  Przejdziemy teraz do\ldots\\
\StrWg{136}{10} \\
\Jest contrawariantny\ldots\\
\Pow  k\dywiz krotnie kowariantny i~s\dywiz krotnie kontrawariantny\ldots \\
\StrWd{140}{3} \\
\Jest \\
\Pow \\
\Str{153}{12} \\
\Jest naturalnymi wiązkami wektorowymi\ld \\
\Pow  wiązkami naturalnymi\ld \\
\StrWd{153}{8} \\
Zupełnie nie rozumiem tego zdania, jednak trzeba je poprawić. \\
\StrWg{154}{7} \\
\Jest
\begin{equation*}
  \wt{ \psi }\Big|_{ \pi^{ -1 }( U ) } \circ \Del \ti \id_{ F_{ 1 } \ti F_{ 2 } }
\end{equation*}
\Pow
\begin{equation*}
  \Del \ti \id_{ F_{ 1 } \ti F_{ 2 } } \circ \wt{ \psi }\Big|_{ \pi^{ -1 }( U ) }
\end{equation*}
\StrWg{154}{8} \\
\Jest $U \ni x \to ( x, x ) \in U \ti U$ \\
\Pow $U \ti U \ni ( x, x ) \to x \in U$ \\
\StrWd{156}{10} \\
\Jest $\om_{ 1 } \ot \ld \ot \om_{ q } \ot X_{ 1 } \ot \ld
\ot X_{ p }$ \\
\Pow $X_{ 1 } \ot \ld \ot X_{ p } \ot \om_{ 1 } \ot \ld \ot
\om_{ q }$ \\
\StrWd{156}{3} \\
\Jest $X_{ 1 } \ot \ld \ot X_{ i - 1 } \ot [ X, X_{ i } ] \ot \ld
\ot X_{ k }$\ld \\
\Pow $X_{ 1 }, \ld, X_{ i - 1 }, [ X, X_{ i } ], \ld, X_{ k }$\ldots \\
\StrWd{156}{3} \\
\Jest $X_{ 1 } \ot \ld \ot X_{ i - 1 } \ot [ X, X_{ i } ] \ot \ld
\ot X_{ k }$\ld \\
\Pow $X_{ 1 }, \ld, X_{ i - 1 }, [ X, X_{ i } ], \ld, X_{ k }$\ld \\
\StrWg{163}{7} \\
\Jest jest generowana prze formy postaci $d \la$, gdzie $\la$
jest funkcją na $N$\ld \\
\Pow jest lokalnie sumą form postaci
$f\, d\la_{ 1 } \we \ldots \we \la_{ k }$, gdzie
$f, \la_{ 1 }, \ld, \la_{ k }$~są funkcjami na $N$\ld \\
\StrWd{163}{3} \\
\Jest jest postaci\ldots \\
\Pow  jest lokalnie sumą wyrażeń postaci\ld \\
\StrWg{211}{4} \\
\Jest różniczkowalne \\
\Pow  odwzorowanie różniczkowalne \\
\StrWg{227}{4} \\
\Jest $U \ti J^{ r }_{ 0 }( \R^{ n }, F ) \pi^{ -1 }( U )$ \\
\Pow  $U \ti J^{ r }_{ 0 }( \R^{ n }, F )$ \\
\StrWg{227}{6} \\
\Jest $( \vp^{ -1 }( y, \rho \circ \tau_{ -\psi( y ) } \circ \psi ) )$ \\
\Pow  $\vp^{ -1 }( y, \rho \circ \tau_{ -\psi( y ) } \circ \psi )$ \\
\StrWg{231}{3} \\
\Jest $v_{ i_{ 1 } } \ot \ld v_{ i - p } \ot { e* }^{ j_{ 1 } } \ot
\ld \ot { e* }^{ j_{ p } }$ \\
\Pow $v_{ i_{ 1 } } \ot \ld \ot v_{ i_{ p } } \ot v^{ * j_{ 1 } } \ot
\ld \ot v^{ * j_{ p } }$ \\
\StrWg{233}{10} \\
\Jest kanoniczny. wyznaczony \\
\Pow  kanoniczny, wyznaczony \\
\StrWg{253}{17} \\
\Jest \emph{$( R, I )$, gdzie reper $p$ interpretujemy jako izomorfizm
  $\R^{ n } \to T_{ \pi( p ) }M$.} \\
\Pow \emph{$( \R, I )$.} \\
\StrWg{253}{15} \\
\Jest $p^{ -1 }( K_{ \pi( p ) }( d_{ p }\pi( X_{ 1 } ), \ld, d_{ p }\pi( X_{ k } ) ) )$ \\
\Pow  $K_{ \pi( p ) }( d_{ p }\pi( X_{ 1 } ), \ld, d_{ p }\pi( X_{ k } ) )$ \\
\StrWd{254}{16} \\
\Jest $( K_{ \pi( p ) }( d_{ p }\pi( X_{ 1 } ), \ld,
d_{ p }\pi( X_{ k } ) )$ \\
\Pow $( K_{ \pi( p ) }( d_{ p }\pi( X_{ 1 } ), \ld,
d_{ p }\pi( X_{ k } ) ) )$ \\
\StrWd{254}{15} \\
\Jest $( K_{ \pi( p ) }( d_{ p }\pi( X_{ 1 } ), \ld,
d_{ p }\pi( X_{ k } ) )$ \\
\Pow $( K_{ \pi( p ) }( d_{ p }\pi( X_{ 1 } ), \ld,
d_{ p }\pi( X_{ k } ) ) )$ \\


111 5 $\wt{ \wt{ A } } = Y$ \\
Str. 135. Teraz omówimy\ld \\
Str. 136. \ldots polega na tym\ld \\
Str. 135. Teraz omówimy\ld \\





% ####################################################################
% ####################################################################
% Bibliografia
\bibliographystyle{alpha} \bibliography{Bibliography}{}


% ############################

% Koniec dokumentu
\end{document}