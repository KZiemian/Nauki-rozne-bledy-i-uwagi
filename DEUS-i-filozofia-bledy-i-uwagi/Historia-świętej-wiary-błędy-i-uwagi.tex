% Autor: Kamil Ziemian

% --------------------------------------------------------------------
% Podstawowe ustawienia i pakiety
% --------------------------------------------------------------------
\RequirePackage[l2tabu, orthodox]{nag} % Wykrywa przestarzałe i niewłaściwe
% sposoby używania LaTeXa. Więcej jest w l2tabu English version.
\documentclass[a4paper,11pt]{article}
% {rozmiar papieru, rozmiar fontu}[klasa dokumentu]
\usepackage[MeX]{polski} % Polonizacja LaTeXa, bez niej będzie pracował
% w języku angielskim.
\usepackage[utf8]{inputenc} % Włączenie kodowania UTF-8, co daje dostęp
% do polskich znaków.
\usepackage{lmodern} % Wprowadza fonty Latin Modern.
\usepackage[T1]{fontenc} % Potrzebne do używania fontów Latin Modern.



% ----------------------------
% Podstawowe pakiety (niezwiązane z ustawieniami języka)
% ----------------------------
\usepackage{microtype} % Twierdzi, że poprawi rozmiar odstępów w tekście.
% \usepackage{graphicx} % Wprowadza bardzo potrzebne komendy do wstawiania
% % grafiki.
% \usepackage{verbatim} % Poprawia otoczenie VERBATIME.
% \usepackage{textcomp} % Dodaje takie symbole jak stopnie Celsiusa,
% wprowadzane bezpośrednio w tekście.
\usepackage{vmargin} % Pozwala na prostą kontrolę rozmiaru marginesów,
% za pomocą komend poniżej. Rozmiar odstępów jest mierzony w calach.
% ----------------------------
% MARGINS
% ----------------------------
\setmarginsrb
{ 0.7in} % left margin
{ 0.6in} % top margin
{ 0.7in} % right margin
{ 0.8in} % bottom margin
{  20pt} % head height
{0.25in} % head sep
{   9pt} % foot height
{ 0.3in} % foot sep



% ------------------------------
% Często używane pakiety
% ------------------------------
% \usepackage{csquotes} % Pozwala w prosty sposób wstawiać cytaty do tekstu.
% \usepackage{xcolor} % Pozwala używać kolorowych czcionek (zapewne dużo
% więcej, ale ja nie potrafię nic o tym powiedzieć).





% --------------------------------------------------------------------
% Dodatkowe ustawienia dla języka polskiego
% --------------------------------------------------------------------
\renewcommand{\thesection}{\arabic{section}.}
% Kropki po numerach rozdziału (polski zwyczaj topograficzny)
\renewcommand{\thesubsection}{\thesection\arabic{subsection}}
% Brak kropki po numerach podrozdziału



% ----------------------------
% Pakiety napisane przez użytkownika.
% Mają być w tym samym katalogu to ten plik .tex
% ----------------------------
\usepackage{latexshortcuts}



% ----------------------------
% Ustawienia różnych parametrów tekstu
% ----------------------------
\renewcommand{\arraystretch}{1.2} % Ustawienie szerokości odstępów między
% wierszami w tabelach.



% ----------------------------
% Pakiet "hyperref"
% Polecano by umieszczać go na końcu preambuły.
% ----------------------------
\usepackage{hyperref} % Pozwala tworzyć hiperlinki i zamienia odwołania
% do bibliografii na hiperlinki.





% --------------------------------------------------------------------
% Tytuł, autor, data
\title{Historia świętej wiary --~błędy i~uwagi}

% \author{}
% \date{}
% --------------------------------------------------------------------





% ####################################################################
% Początek dokumentu
\begin{document}
% ####################################################################



% ######################################
\maketitle  % Tytuł całego tekstu
% ######################################



% ######################################
\section{Historia świętej wiary}

\vspace{\spaceTwo}
% ######################################




% ##################
\Work{ % Autor i tytuł dzieła
  Richard Butterwick \\
  ,,Polska Rewolucja a~Kościół Katolicki 1788--1792'',
  \cite{ButterwickPolskaRewolucjaAKosciolKatolicki2012} }



\CenterTB{Uwagi}

\start \Str{28} Euzebiusz \\


% \CenterTB{Błędy}
% \begin{center}
%   \begin{tabular}{|c|c|c|c|c|}
%     \hline
%     & \multicolumn{2}{c|}{} & & \\
%     Strona & \multicolumn{2}{c|}{Wiersz}& Jest & Powinno być \\ \cline{2-3}
%     & Od góry & Od dołu &  &  \\ \hline
%     & & & & \\
%     & & & & \\
%     & & & & \\
%     & & & & \\
%     %     \hline
%   \end{tabular}


\vspace{\spaceTwo}





% ##################
\newpage
\Work{ % Autor i tytuł dzieła
  Warren H.~Carroll \\
  ,,Historia Chrześcijaństwa. Tom~I: Narodziny Chrześcijaństwa'',
  \cite{CarrollHistoriaChrzecijanstwaTomI2009} }


\CenterTB{Uwagi}

\start \StrWd{17}{4} Gwiazdka w~tej linii jest za~mała.

\vspace{\spaceFour}


\start \StrWd{41}{13} W~tej linii użyto złego rodzaju gwiazdki
i~umieszczono ją trochę krzywo.

\vspace{\spaceFour}


\start \StrWd{61}{20} Gwiazdka w~tej linii jest za~mała.

\vspace{\spaceFour}


\start \StrWd{65}{6} W~tej linii użyto złego rodzaju gwiazdki.

\vspace{\spaceFour}


\start \StrWd{73}{12} Gwiazdka w~tej linii jest trochę za~mała
i~odrobinę krzywo umieszczona.

\vspace{\spaceFour}


\start \StrWd{75}{4} Gwiazdka w~tej linii jest za mała.

\vspace{\spaceFour}


\start \StrWd{112}{2} Gwiazdka w~tej linii jest za mała.

\vspace{\spaceFour}


\start \StrWd{182}{2} Gwiazdka w~tej linii jest za mała.

\vspace{\spaceFour}


\start \tb{Str.~199, wiersze 9, 8 (od~dołu).} Gwiazdki w~tych
liniach~są zbyt małe.

\vspace{\spaceFour}


\start \tb{Str.~200, wiersze 2, 1 (od~dołu).} Gwiazdki w~tych
liniach~są zbyt małe.

\vspace{\spaceFour}


\start \StrWd{249}{1} Gwiazdka w~tej linii jest za mała.

\vspace{\spaceFour}


\start \StrWd{265}{2} Gwiazdka w~tej linii jest za mała.

\vspace{\spaceFour}


\start \StrWd{271}{1} Gwiazdka w~tej linii jest za mała.

\vspace{\spaceFour}


\start \StrWd{282}{14} Gwiazdka w~tej linii jest za mała.

\vspace{\spaceFour}


\start \StrWd{286}{4} Gwiazdka w~tej linii jest za mała.

\vspace{\spaceFour}


\start \tb{Str.~290, wiersze 4, 3 (od~dołu).} Gwiazdki w~tych
liniach~są zbyt małe.

\vspace{\spaceFour}


\start \StrWd{366}{1} Gwiazdka w~tej linii jest za mała.

\vspace{\spaceFour}


\start \StrWg{378}{20} Gwiazdka w~tej linii jest za mała.

\vspace{\spaceFour}


\start \StrWd{495}{1} Gwiazdka w~tej linii jest za mała.

\vspace{\spaceFour}


\start \StrWd{538}{38} Gwiazdka w~tej linii jest za mała.

\vspace{\spaceFour}


\start \StrWd{562}{20} W~tej linii nie podano autorów publikacji. Nie
wiem czy to~błąd, czy~dla tej publikacji nie~trzeba podawać autorów.

\vspace{\spaceFour}


\start \StrWg{575}{4~od~góry i~1 od~dołu} Słowo ,,męczennik'' jest
w~bardzo brzydki sposób podzielone między te dwie linie.

% \vspace{\spaceFour}


% 185???

\CenterTB{Błędy}
\begin{center}
  \begin{tabular}{|c|c|c|c|c|}
    \hline
    & \multicolumn{2}{c|}{} & & \\
    Strona & \multicolumn{2}{c|}{Wiersz} & Jest
                              & Powinno być \\ \cline{2-3}
    & Od góry & Od dołu & & \\
    \hline
    % & & & & \\
    19  & & 14 & \emph{Fossil~~Evidence} & \emph{Fossil Evidence} \\
    19  & &  5 & \emph{Evolution -- the} & \emph{Evolution: The} \\
    20  & & 21 & le & Le \\
    29  & &  2 & \emph{Huyuk, a} & \emph{Huyuk: A} \\
    30  & &  3 & \emph{Sumerians, Their} & \emph{Sumerians: Their} \\
    33  & &  3 & \emph{Babylon, a} & \emph{Babylon: A} \\
    35  & &  5 & \emph{India, a} & \emph{India: A} \\
    42  & &  2 & \emph{Elba, a} & \emph{Elba: A} \\
    44  & &  1 & \emph{Hyksos, a} & \emph{Hyksos: A} \\
    45  & & 18 & \emph{Desert, a} & \emph{Desert: A} \\
    49  & & 20 & \emph{through} & \emph{Through} \\ % A Path Through Genesis
    49  & &  2 & \emph{through} & \emph{Through} \\
    53  & & 15 & \emph{Path through} & \emph{A Path Through} \\
    56  & &  5 & \emph{Akhenaten},,  % ''
           & \emph{Akhenaten}, \\
    61  & &  2 & \emph{Israel,} & \emph{Israel:} \\
    63  & & 12 & \emph{Rames II, a} & \emph{Rames II: A} \\
    73  & &  6 & \emph{into} & \emph{Into} \\
    73  & &  6 & \emph{Past, the} & \emph{Past: The} \\
    81  & &  3 & \emph{Israel, Its} & \emph{Israel: Its} \\
    83  & & 20 & \emph{Hazor, the} & \emph{Hazor: The} \\
    84  & &  2 & \emph{Bible,} & \emph{Bible:} \\
    84  & &  1 & \emph{a Historical} & \emph{A Historical} \\
    91  & &  5 & \emph{Worlds}. 94-100 & \emph{Worlds}, s.~94-100 \\
    92  & &  6 & \emph{Testament}, & \emph{Testament}, [w:] \\
    102 & & 16 & \emph{In} & \emph{in} \\
    102 & &  4 & \emph{Israel, its} & \emph{Israel: Its} \\
    107 & &  3 & \emph{Mesopotamia, Portrait}
           & \emph{Mesopotamia: Portrait} \\
    126 & 18 & & \emph{through} & \emph{Through} \\
    131 & &  9 & Babylon & \emph{Babylon} \\
    132 & &  6 & \emph{Israe}l & \emph{Israel} \\
    145 & & 12 & \emph{Greeks; the} & \emph{Greeks: The} \\
    \hline
  \end{tabular}


  % \begin{tabular}{|c|c|c|c|c|}
  %   \hline
  %   & \multicolumn{2}{c|}{} & & \\
  %   Strona & \multicolumn{2}{c|}{Wiersz} & Jest
  %   & Powinno być \\ \cline{2-3}
  %   & Od góry & Od dołu & & \\
  %   \hline
  %   %   & & & & \\
  %   %   & & & & \\
  %   %   & & & & \\
  %   %   & & & & \\
  %   %   & & & & \\
  %   %   & & & & \\
  %   %   & & & & \\
  %   %   & & & & \\
  %   %   & & & & \\
  %   %   & & & & \\
  %   %   & & & & \\
  %   %   & & & & \\
  %   %   & & & & \\
  %   %   & & & & \\
  %   %   & & & & \\
  %   %   & & & & \\
  %   %   & & & & \\
  %   %   & & & & \\
  %   %   & & & & \\
  %   %   & & & & \\
  %   %   & & & & \\
  %   %   & & & & \\
  %   %   & & & & \\
  %   %   & & & & \\
  %   %   & & & & \\
  %   %   & & & & \\
  %   %   & & & & \\
  %   %   & & & & \\
  %   %   & & & & \\
  %   %   & & & & \\
  %   %   & & & & \\
  %   %   & & & & \\
  %   %   & & & & \\
  %   %   & & & & \\
  %   %   & & & & \\
  %   %   & & & & \\
  %   %   & & & & \\
  %   %   & & & & \\
  %   \hline
  % \end{tabular}


  % \begin{tabular}{|c|c|c|c|c|}
  %   \hline
  %   & \multicolumn{2}{c|}{} & & \\
  %   Strona & \multicolumn{2}{c|}{Wiersz} & Jest
  %   & Powinno być \\ \cline{2-3}
  %   & Od góry & Od dołu & & \\
  %   \hline
  %   %   & & & & \\
  %   %   & & & & \\
  %   %   & & & & \\
  %   %   & & & & \\
  %   %   & & & & \\
  %   %   & & & & \\
  %   %   & & & & \\
  %   %   & & & & \\
  %   %   & & & & \\
  %   %   & & & & \\
  %   %   & & & & \\
  %   %   & & & & \\
  %   %   & & & & \\
  %   %   & & & & \\
  %   %   & & & & \\
  %   %   & & & & \\
  %   %   & & & & \\
  %   %   & & & & \\
  %   %   & & & & \\
  %   %   & & & & \\
  %   %   & & & & \\
  %   %   & & & & \\
  %   %   & & & & \\
  %   %   & & & & \\
  %   %   & & & & \\
  %   %   & & & & \\
  %   %   & & & & \\
  %   %   & & & & \\
  %   %   & & & & \\
  %   %   & & & & \\
  %   %   & & & & \\
  %   %   & & & & \\
  %   %   & & & & \\
  %   %   & & & & \\
  %   %   & & & & \\
  %   %   & & & & \\
  %   %   & & & & \\
  %   %   & & & & \\
  %   \hline
  % \end{tabular}


  \begin{tabular}{|c|c|c|c|c|}
    \hline
    & \multicolumn{2}{c|}{} & & \\
    Strona & \multicolumn{2}{c|}{Wiersz} & Jest
                              & Powinno być \\ \cline{2-3}
    & Od góry & Od dołu & & \\
    \hline
    145 & &  3 & Babylon & \emph{Babylon} \\
    160 & & 21 & \emph{Personality, Its} & \emph{Personality: Its} \\
    166 & &  5 & wspanialej & wspaniałej \\
    171 & &  2 & \emph{Wisdom, the} & \emph{Wisdom: The} \\
    172 & &  4 & \emph{Greeks, the} & \emph{Greeks: The} \\
    178 & &  4 & Agamemnon & \emph{Agamemnon} \\
    182 & &  5 & \emph{Greeks, a} & \emph{Greeks: A} \\
    193 & &  1 & \emph{Great, King} & \emph{Great: King} \\
    203 & &  3 & \emph{B.C} & \emph{B.C.} \\
    218 & &  8 & \emph{146B.C.} & \emph{146~B.C.} \\
    220 & &  2 & \emph{Africanus, Soldier} & \emph{Africanus: Soldier} \\
    223 & &  5 & \emph{Carthage, a} & \emph{Carthage: A} \\
    228 & & 14 & \emph{B. C} & \emph{B.C.} \\
    230 & & 15 & \emph{Empire, Rome's} & \emph{Empire: Rome's} \\
    230 & &  4 & \emph{B.~C.} & \emph{B.C.} \\
    230 & &  1 & \emph{146B.C.} & \emph{146~B.C.} \\
    233 & &  9 & \emph{146B.C.} & \emph{146~B.C.} \\
    233 & &  7 & \emph{146B.C.} & \emph{146~B.C.} \\
    234 & &  5 & \emph{Syria, from} & \emph{Syria: From} \\
    238 & &  7 & \emph{Ezra} & \emph{From Ezra} \\
    238 & &  3 & \emph{Ezra} & \emph{From Ezra} \\
    241 & &  5 & \emph{Ezra} & \emph{From Ezra} \\
    241 & &  4 & \emph{Ezra} & \emph{From Ezra} \\
    241 & &  1 & \emph{Ezra} & \emph{From Ezra} \\
    242 & &  2 & \emph{Ezra} & \emph{From Ezra} \\
    242 & &  2 & \emph{Ezra} & \emph{From Ezra} \\
    % & & & \emph{Ezra} & \emph{From Ezra} \\
    259 & &  3 & \emph{Pompey, the} & \emph{Pompey: The} \\
    261 & &  3 & \emph{Caesar, Politician} & \emph{Caesar: Politician} \\
    262 & &  6 & \emph{Pompey, the} & \emph{Pompey: The} \\
    266 & &  6 & \emph{Pompey, the} & \emph{Pompey: The} \\
    267 & &  2 & \emph{Rule, from} & \emph{Rule: From} \\
    268 & &  3 & \emph{under} & \emph{Under} \\
    268 & &  1 & \emph{under} & \emph{Under} \\
    275 & &  7 & \emph{under} & \emph{Under} \\
    281 & &  3 & \emph{the} & \emph{The} \\
    281 & &  3 & \emph{Joseph, Their} & \emph{Joseph: Their} \\
    293 & &  3 & \emph{Christ, His} & \emph{Christ: His} \\
    294 & &  8 & \emph{St.~Matthew, a} & \emph{St.~Matthew: A} \\
    % & & & & \\
    \hline
  \end{tabular}


  \begin{tabular}{|c|c|c|c|c|}
    \hline
    & \multicolumn{2}{c|}{} & & \\
    Strona & \multicolumn{2}{c|}{Wiersz} & Jest
                              & Powinno być \\ \cline{2-3}
    & Od góry & Od dołu & & \\
    \hline
    295 & &  4 & \emph{Birth, an} & \emph{Birth: An} \\
    300 & &  9 & \emph{Josephus, the} & \emph{Josephus: The} \\
    307 & &  8 & \emph{Birth, an} & \emph{Birth: An} \\
    315 & & 20 & \emph{Bethlehem, an} & \emph{Bethlehem: An} \\
    315 & & 10 & \emph{niemal} & niemal \\
    316 & &  8 & \emph{Dead, Studies} & \emph{Dead: Studies} \\
    320 & & 10 & Jesus Christ, \emph{His} & \emph{Jesus Christ: His} \\
    326 & & 12 & \emph{Antipas, a} & \emph{Antipas: A} \\
    365 & &  9 & 1931 ) & 1931) \\
    374 & & 11 & 549;Belser & 549; Belser \\
    387 & & 10 & \emph{Doctor} & \emph{A~Doctor} \\
    387 & &  9 & \emph{Doctor} & \emph{A~Doctor} \\
    387 & &  5 & \emph{Doctor} & \emph{A~Doctor} \\
    388 & & 17 & \emph{Doctor} & \emph{A~Doctor} \\
    388 & & 13 & \emph{Doctor} & \emph{A~Doctor} \\
    388 & &  1 & \emph{Doctor} & \emph{A~Doctor} \\
    390 & &  7 & \emph{Doctor} & \emph{A~Doctor} \\
    391 & & 11 & \emph{Doctor} & \emph{A~Doctor} \\
    391 & & 10 & \emph{Doctor} & \emph{A~Doctor} \\
    392 & &  3 & \emph{Doctor} & \emph{A~Doctor} \\
    393 & &  8 & \emph{Doctor} & \emph{A~Doctor} \\
    % & & & \emph{Doctor} & \emph{A~Doctor} \\
    402 & &  9 & \emph{Antipas, a} & \emph{Antipas: A }\\
    407 & &  6 & \emph{Greek, a} & \emph{Greek: A} \\
    408 & &  2 & \emph{Claudius, the} & \emph{Claudius: The} \\
    410 & &  3 & \emph{Luke, a} & \emph{Luke: A} \\
    419 & & 24 & \emph{Doctor} & \emph{A~Doctor} \\
    432 & &  2 & \emph{Kerala, a} & \emph{Kerala: A} \\
    436 & &  5 & \emph{Nero, Reality} & \emph{Nero: Reality} \\
    440 & &  4 & \emph{Exile, a} & \emph{Exile: A} \\
    442 & &  9 & \emph{Tertullian, a} & \emph{Tertullian: A} \\
    443 & & 17 & \emph{Jude, Introduction} & \emph{Jude: Introduction} \\
    444 & &  9 & \emph{under} & \emph{Under} \\
    444 & &  2 & \emph{under} & \emph{Under} \\
    449 & &  7 & \emph{under} & \emph{Under} \\
    455 & & 25 & np., & np. \\
    458 & &  4 & \emph{among} & \emph{Among} \\
    458 & &  2 & \emph{Paul, Apostole} & \emph{Paul: Apostole} \\
    459 & &  3 & \emph{Religion; the} & \emph{Religin: The} \\
    \hline
  \end{tabular}


  \begin{tabular}{|c|c|c|c|c|}
    \hline
    & \multicolumn{2}{c|}{} & & \\
    Strona & \multicolumn{2}{c|}{Wiersz} & Jest
                              & Powinno być \\ \cline{2-3}
    & Od góry & Od dołu & & \\
    \hline
    471 & &  9 & \emph{Smyrna, a} & \emph{Smyrna: A} \\
    477 & & 17 & \emph{ofthe} & \emph{of the} \\
    478 & &  7 & \emph{Severus, the} & \emph{Severus: The} \\
    490 & & 15 & \emph{Tertulian, a} & \emph{Tertulian: A} \\
    490 & &  1 & cześć & część \\
    492 & & 13 & \emph{during} & \emph{During} \\
    492 & &  2 & \emph{Mani, a} & \emph{Mani: A} \\
    508 & &  4 & \emph{Edessa, the} & \emph{Edessa: The} \\
    533 & &  4 & \emph{Constantine, a} & \emph{Constantine: A} \\
    561 & & 19 & \emph{Bible, a~Historical}
           & \emph{Bible: A~Historical} \\
    561 & & 16 & T. & T., \\
    561 & & 15 & \emph{Moses, the} & \emph{Moses: The} \\
    562 &  4 & & \emph{Ezekiel: the} & \emph{Ezekiel: The} \\
    562 & 10 & & \emph{Abraham, Loved} & \emph{Abraham: Loved} \\
    562 & 11 & & \emph{Desert, a~History} & \emph{Desert: A~History} \\
    562 & 13 & & \emph{Abraham, Father} & \emph{Abraham: Father} \\
    562 & 15 & & \emph{Canaan: the~Ras} & \emph{Canaan: The~Ras} \\
    562 & & 14 & \emph{Judaea} & \emph{Judea} \\
    562 & & 13 & \emph{Israel, from} & \emph{Israel: From} \\
    562 & & 11 & \emph{Jerusalem; Excavating}
           & \emph{Jerusalem: Excavating} \\
    562 & &  4 & \emph{Covenant, a~Study} & \emph{Covenant: A~Study} \\
    563 &  4 & & \emph{Law; Studies} & \emph{Law: Studies} \\
    563 &  6 & & \emph{1-39, Introduction} & \emph{1-39: Introduction} \\
    563 & & 18 & \emph{Joshua; Biblical} & \emph{Joshua: Biblical} \\
    563 & & 11 & \emph{Qumran, a} & \emph{Qumran: A} \\
    563 & &  8 & \emph{Israel, its} & \emph{Israel: Its} \\
    563 & &  7 & T, & T. \\
    564 &  1 & & \emph{Hazor, the} & \emph{Hazor: The} \\
    564 &  8 & & \emph{Maccabees, with} & \emph{Maccabees: With} \\
    564 & 12 & & \emph{Akhenaten, Pharaoh} & \emph{Akhenaten, Pharaoh} \\
    564 & 13 & & \emph{before} & \emph{Before} \\
    564 & 16 & & \emph{Elba, a} & \emph{Elba: A} \\
    564 & 18 & & \emph{Greeks; the} & \emph{Greeks: The} \\
    564 & & 17 & \emph{Buddhism, its} & \emph{Buddhism: Its} \\
    564 & &  7 & \emph{Darkness; a} & \emph{Darkness: A} \\
    564 & &  3 & \emph{Sumerians; Their} & \emph{Sumerians: Their} \\
    565 &  3 & & \emph{Huyuk, a} & \emph{Huyuk: A} \\
    565 & 13 & & \emph{Mesopotamia, Portrait}
           & \emph{Mesopotamia: Portrait} \\
    \hline
  \end{tabular}


  \begin{tabular}{|c|c|c|c|c|}
    \hline
    & \multicolumn{2}{c|}{} & & \\
    Strona & \multicolumn{2}{c|}{Wiersz} & Jest
                              & Powinno być \\ \cline{2-3}
    & Od góry & Od dołu & & \\
    \hline
    565 & 16 & & \emph{Babylon; a} & \emph{Babylon: A} \\
    565 & 18 & & \emph{Rameses~II, a} & \emph{Ramses~II: A} \\
    565 & 19 & & \emph{India; a} & \emph{India: A} \\
    565 & 17 & & \emph{Hyksos, a} & \emph{Hyksos: A} \\
    565 & 14 & & \emph{Egypt, an} & \emph{Egypt: An} \\
    565 & 13 & & \emph{Personality, Its} & \emph{Personality: Its} \\
    566 &  3 & & \emph{B. C.} & \emph{B.C.} \\
    566 &  7 & & \emph{Carthage, a} & \emph{Carthage: A} \\
    566 &  9 & & \emph{Dead; Studies} & \emph{Dead: Studies} \\
    566 & 10 & & \emph{Empire; Rome's} & \emph{Empire: Rome's} \\
    566 & 11 & & \emph{Greeks, a} & \emph{Greeks: A} \\
    566 & 16 & & \emph{Caesar, Politician} & \emph{Caesar: Politician} \\
    566 & & 20 & \emph{Pompey, the} & \emph{Pompey: The} \\
    566 & & 19 & \emph{Pompey, the} & \emph{Pompey: The} \\
    566 & & 17 & \emph{Great, King} & \emph{Great: King} \\
    566 & & 13 & \emph{lonians} & \emph{Ionians} \\
    566 & &  9 & A.H.M.,\emph{Sparta} & A.H.M., \emph{Sparta} \\
    566 & &  5 & \emph{Past, the} & \emph{Past: The} \\
    566 & &  2 & \emph{B. C.} & \emph{B.C.} \\
    566 & &  1 & \emph{Wisdom; the} & \emph{Wisdom: The} \\
    567 &  5 & & \emph{Library, Glory} & \emph{Library: Glory} \\
    567 & 19 & & \emph{under} & \emph{Under} \\
    567 & 19 & & \emph{Rule, from} & \emph{Rule: From} \\
    567 & 20 & & \emph{Cicero, a} & \emph{Cicero: A} \\
    567 & &  4 & \emph{Passion, Death} & \emph{Passion: Death} \\
    568 &  2 & & \emph{Matthew, a} & \emph{Matthew: A} \\
    568 & 15 & & \emph{Christ, a} & \emph{Christ: A} \\
    568 & 18 & & \emph{Antipas, a} & \emph{Antipas: A} \\
    568 & & 13 & \emph{Birth, an} & \emph{Birth: An} \\
    568 & &  4 & \emph{among} & \emph{Among} \\
    568 & &  4 & \emph{Peo-Pk} & \emph{People} \\
    569 & 12 & & \emph{Tertullian, a} & \emph{Tertullian: A} \\
    569 & & 21 & \emph{Smyrna, a} & \emph{Smyrna: A} \\
    569 & & 13 & \emph{Jude, Introduction} & \emph{Jude: Introduction} \\
    569 & &  1 & \emph{Greek, a} & \emph{Greek: A} \\
    570 & & 17 & \emph{Cecilia, Virgin} & \emph{Cecilia: Virgin} \\
    570 & & 15 & \emph{Exile, a} & \emph{Exile: A} \\
    570 & &  5 & \emph{during} & \emph{During} \\
    \hline
  \end{tabular}


  \begin{tabular}{|c|c|c|c|c|}
    \hline
    & \multicolumn{2}{c|}{} & & \\
    Strona & \multicolumn{2}{c|}{Wiersz} & Jest
                              & Powinno być \\ \cline{2-3}
    & Od góry & Od dołu & & \\
    \hline
    571 &  3 & & \emph{Religion; the} & \emph{Religion: The} \\
    571 & 16 & & \emph{Eusebian; Essay} & \emph{Eusebian: Essay} \\
    571 & 10 & & \emph{Rome; the} & \emph{Rome: The} \\
    571 & &  7 & \emph{Mani; a} & \emph{Mani: A} \\
    571 & &  1 & \emph{before} & \emph{Before} \\
    572 &  2 & & 1959 & 1959. \\
    572 &  6 & & \emph{Edessa, the} & \emph{Edessa: The} \\
    572 & 11 & & \emph{Josephus, the} & \emph{Josephus: The} \\
    572 & 17 & & \emph{Kerala, a} & \emph{Kerala: A} \\
    572 &  7 & & \emph{Severus, the} & \emph{Severus: The} \\
    573 &  1 & & \emph{Aurelius, His} & \emph{Aurelius: His} \\
    573 &  8 & & \emph{Seneca, a} & \emph{Seneca: A} \\
    573 & 12 & & \emph{Constantine, a} & \emph{Constantine: A} \\
    573 & 15 & & 1948 & 1948. \\
    573 & 20 & & \emph{after} & \emph{After} \\
    573 & 18 & & \emph{Claudius, the} & \emph{Claudius: The} \\
    573 & &  9 & \emph{138A.D.} & \emph{138~A.D.} \\
    573 & &  3 & \emph{Meroe, a} & \emph{Meroe: A} \\
    573 & &  2 & \emph{under} & \emph{Under} \\
    573 & &  2 & \emph{Rule, from} & \emph{Rule: From} \\
    574 &  5 & & \emph{Nero, Reality} & \emph{Nero: Reality} \\
    575 &  4 & & \emph{422} & 422 \\
    575 &  6 & & \emph{421} & 421 \\
    575 &  7 & & pne.- 7 & pne. --~7 \\
    % Linia 8, sprawdź czy daty życia Agrypiny są poprawne
    575 &  8 & & \emph{421, 422,} & 421, 422, \\
    575 &  8 & & \emph{508} & 508 \\
    575 & 14 & & \emph{40} & 40 \\
    575 & 14 & & \emph{68} & 68 \\
    575 & 15 & & \emph{91} & 91 \\
    575 & 16 & & \emph{407} & 407 \\
    575 & &  7 & \emph{22} & 22 \\
    575 & &  7 & \emph{517} & 517 \\
    575 & &  5 & \emph{528} & 528 \\
    575 & &  4 & \emph{518} & 518 \\
    575 & &  1 & ii & i \\
    % & & & & \\
    % & & & & \\
    \hline
  \end{tabular}


  % \begin{tabular}{|c|c|c|c|c|}
  %   \hline
  %   & \multicolumn{2}{c|}{} & & \\
  %   Strona & \multicolumn{2}{c|}{Wiersz} & Jest
  %   & Powinno być \\ \cline{2-3}
  %   & Od góry & Od dołu & & \\
  %   \hline
  %   %   & & & & \\
  %   %   & & & & \\
  %   %   & & & & \\
  %   %   & & & & \\
  %   %   & & & & \\
  %   %   & & & & \\
  %   %   & & & & \\
  %   %   & & & & \\
  %   %   & & & & \\
  %   %   & & & & \\
  %   %   & & & & \\
  %   %   & & & & \\
  %   %   & & & & \\
  %   %   & & & & \\
  %   %   & & & & \\
  %   %   & & & & \\
  %   %   & & & & \\
  %   %   & & & & \\
  %   %   & & & & \\
  %   %   & & & & \\
  %   %   & & & & \\
  %   %   & & & & \\
  %   %   & & & & \\
  %   %   & & & & \\
  %   %   & & & & \\
  %   %   & & & & \\
  %   %   & & & & \\
  %   %   & & & & \\
  %   %   & & & & \\
  %   %   & & & & \\
  %   %   & & & & \\
  %   %   & & & & \\
  %   %   & & & & \\
  %   %   & & & & \\
  %   %   & & & & \\
  %   %   & & & & \\
  %   %   & & & & \\
  %   %   & & & & \\
  %   \hline
  % \end{tabular}
\end{center}
\StrWd{575}{4} \\
\Jest \hspace{5pt} nik \\
\Pow  -nik \\

\vspace{\spaceTwo}





% ##################
\Work{ % Autor i tytuł dzieła
  Warren H.~Carroll \\
  ,,Historia Chrześcijaństwa. Tom~II: Budowanie Chrześcijaństwa'',
  \cite{CarrollHistoriaChrzecijanstwaTomII2010} }


\CenterTB{Uwagi}

\start Tłumaczenie podtytułu tego tomu ,,Budowanie Chrześcijaństwa''
jest wyjątkowo niezręczne. Należy zwrócić uwagę, że~Carroll nadał
swojemu cyklowi tytuł ,,History~of Christendom'' nie ,,History~of
Christianity''. ,,Chrisitianity'' tłumaczy się prosto jako
,,chrześcijaństwo'', ,,Chistendom'' nie ma chyba odpowiednika w~języku
polski, w~tym przypadku zaś można jego sens chyba wyjaśnić, jako
wspólnotę ludzi, której sposób życia definiuje chrześcijaństwo.
W~szczególności ,,Christendom'' oznacza również sens polityczny, jako
zbioru państw, które~są połączone wspólną wiarą chrześcijańską i~tym
samym powinny działać jak różne członki jednego ciała.

Jakkolwiek więc tłumaczenie tytułu całego cyklu jako ,,Historia
Chrześcijaństwa'' ma~sens, to tego podtytułu jako ,,Budowanie
Chrześcijaństwa'' już nie. Sugeruje bowiem, że~religia chrześcijańska
była budowana, podczas gdy ona została już wzniesiona przez Chrystusa,
zaś budowane było właśnie ,,Christendom'', wspólnota ludzka żyjąca jej
prawami.

\vspace{\spaceFour}


\start \StrWd{272}{???}


\CenterTB{Błędy}
\begin{center}
  \begin{tabular}{|c|c|c|c|c|}
    \hline
    & \multicolumn{2}{c|}{} & & \\
    Strona & \multicolumn{2}{c|}{Wiersz} & Jest
                              & Powinno być \\ \cline{2-3}
    & Od góry & Od dołu & & \\
    \hline
    14  & 10 & & \emph{Eusebius}. & \emph{Eusebius}, \\
    14  & 11 & & \emph{A.D.324-344} & \emph{A.D. 324-344} \\
    14  & 11 & & \emph{problems} & \emph{Problems} \\
    14  & 12 & & \emph{cordoba} & \emph{Cordoba} \\
    14  & 12 & & \emph{council} & \emph{Council} \\
    14  & 13 & & studies & Studies \\
    14  & & 18 & \emph{church} & \emph{Church} \\
    15  & &  5 & s.90-91 & s.~90-91 \\
    15  & &  4 & s.117 & s.~117 \\
    15  & &  1 & 1971) & 1971 \\
    16  & &  5 & \emph{fourth} & \emph{Fourth} \\
    19  & &  7 & \emph{A.D} & \emph{A.D.} \\
    19  & &  2 & ,,Jerusalem'' & \emph{Jerusalem} \\
    27  & & 17 & \emph{Egipt; the} & \emph{Egipt: The} \\
    27  & & 14 & s.126 & s.~126 \\
    29  & &  2 & 346;Smith & 346; Smith\\
    30  & & 10 & s.285 & s.~285 \\
    31  & &  4 & 341 (Kidd & 341; Kidd \\
    31  & &  3 & 67,71 & 67, 71 \\
    32  & &  6 & s.65 & s.~65 \\
    32  & &  5 & \emph{Antoniego}. & \emph{Antoniego}, \\
    32  & &  4 & 1987) & 1987). \\
    35  & &  5 & 82,380 & 82, 380 \\
    38  & &  1 & \emph{church} & \emph{Church} \\
    45  & & 10 & s.454 & s.~454 \\
    45  & &  9 & 30,52 & 30, 52 \\
    45  & &  9 & 68,76 & 68, 76 \\
    57  & &  7 & \emph{chrześcijaństwie}. & \emph{chrześcijaństwie}, \\
    66  & & 17 & \emph{Jerome, His} & \emph{Jerome: His} \\
    72  & &  4 & \emph{Chrysostom} & \emph{Chryzostom} \\
    74  & &  6 & \emph{saint} & \emph{Saint} \\
    76  & &  6 & Popes & \emph{Popes} \\
    76  & &  6 & Church & \emph{Church} \\
    79  & &  5 & \emph{Claudian; Poetry} & \emph{Claudian: Poetry} \\
    93  & &  6 & \emph{Jerome, His} & \emph{Jerome: His} \\
    106 & &  9 & \emph{Eusebius, bishop} & \emph{Eusebius: Bishop} \\
    114 & &  9 & \emph{Chalcedon, a~Historical}
           & \emph{Chalcedon: A~Historical} \\
    116 & &  4 & \emph{Arthur, a~History} & \emph{Arthur: A~History} \\
    \hline
  \end{tabular}


  \begin{tabular}{|c|c|c|c|c|}
    \hline
    & \multicolumn{2}{c|}{} & & \\
    Strona & \multicolumn{2}{c|}{Wiersz} & Jest
                              & Powinno być \\ \cline{2-3}
    & Od góry & Od dołu & & \\
    \hline
    116 & &  4 & \emph{350-} & \emph{350} \\
    116 & &  2 & 254---257 & 254-257 \\
    124 & &  8 & 266,277,294 & 266, 277, 294 \\
    124 & &  3 & \emph{Populi; Popular} & \emph{Populi: Popular} \\
    124 & &  2 & \emph{ontroversies} & \emph{Controversies} \\
    141 & & 12 & \emph{history} & \emph{History} \\
    141 & & 12 & Stevens. & Stevens, \\
    141 & &  8 & Stevens. & Stevens, \\
    154 & &  3 & \emph{Moddle} & \emph{Middle} \\
    155 & &  6 & \emph{Moddle} & \emph{Middle} \\
    156 & & 16 & \emph{Invasions; the} & \emph{Invasions: The} \\
    160 & &  4 & \emph{I, an~introduction} & \emph{I: An~Introduction} \\
    161 & &  8 & \emph{sixth} & \emph{Sixth} \\
    167 & 13 & & \emph{History}. & \emph{History}, \\
    168 & &  9 & \emph{I, an} & \emph{I:~An} \\
    % & & & & \\
    % & & & & \\
    197 & &  1 & \emph{Great, His} & \emph{Great: His} \\
    201 & & 21 & \emph{God; the} & \emph{God: The} \\
    235 & &  4 & Mann , & Mann, \\
    236 & &  7 & Mann , & Mann, \\
    238 & & 23 & \emph{conquests} & \emph{Conquests} \\
    249 & &  3 & \emph{before} & \emph{Before} \\
    254 & &  2 & V & t.~V \\
    270 & &  3 & 343-344 [ & 343-344. \\
    294 & &  4 & \emph{papal} & \emph{Papal} \\
    308 & &  4 & \emph{Century: a~Study} & \emph{Century: A~Study} \\
    310 & &  6 & \emph{continent} & \emph{Continent} \\
    314 & &  4 & \emph{is.} & s. \\
    315 & & 11 & \emph{during} & \emph{During} \\
    315 & &  3 & \emph{during} & \emph{During} \\
    % & & & & \\
    % & & & & \\
    388 & & 21 & \emph{Great, the} & \emph{Great: The} \\
    388 & & 19 & \emph{Dragon, Alfred} & \emph{Dragon: Alfred} \\
    388 & & 15 & A.Cotarelo & A.~Cotarelo \\
    388 & &  3 & \emph{Magno}) & \emph{Magno} \\
    388 & &  2 & \emph{Great: the} & \emph{Great: The} \\
    \hline
  \end{tabular}


  % \begin{tabular}{|c|c|c|c|c|}
  %   \hline
  %   & \multicolumn{2}{c|}{} & & \\
  %   Strona & \multicolumn{2}{c|}{Wiersz} & Jest
  %   & Powinno być \\ \cline{2-3}
  %   & Od góry & Od dołu & & \\
  %   \hline
  %   %   & & & & \\
  %   %   & & & & \\
  %   %   & & & & \\
  %   %   & & & & \\
  %   %   & & & & \\
  %   %   & & & & \\
  %   %   & & & & \\
  %   %   & & & & \\
  %   %   & & & & \\
  %   %   & & & & \\
  %   %   & & & & \\
  %   %   & & & & \\
  %   %   & & & & \\
  %   %   489 & & 3 & \emph{Religion;} & \emph{Religion:} \\
  %   %   489 & & 2 & \emph{the} & \emph{The} \\
  %   %   & & & & \\
  %   %   & & & & \\
  %   %   & & & & \\
  %   %   & & & & \\
  %   %   & & & & \\
  %   %   & & & & \\
  %   %   & & & & \\
  %   %   & & & & \\
  %   %   & & & & \\
  %   %   & & & & \\
  %   \hline
  % \end{tabular}

  % \begin{tabular}{|c|c|c|c|c|}
  %   \hline
  %   & \multicolumn{2}{c|}{} & & \\
  %   Strona & \multicolumn{2}{c|}{Wiersz} & Jest
  %   & Powinno być \\ \cline{2-3}
  %   & Od góry & Od dołu & & \\
  %   \hline
  %   %   & & & & \\
  %   %   & & & & \\
  %   %   & & & & \\
  %   %   & & & & \\
  %   %   & & & & \\
  %   %   & & & & \\
  %   %   & & & & \\
  %   %   & & & & \\
  %   %   & & & & \\
  %   %   & & & & \\
  %   %   & & & & \\
  %   %   & & & & \\
  %   %   & & & & \\
  %   %   489 & & 3 & \emph{Religion;} & \emph{Religion:} \\
  %   %   489 & & 2 & \emph{the} & \emph{The} \\
  %   %   & & & & \\
  %   %   & & & & \\
  %   %   & & & & \\
  %   %   & & & & \\
  %   %   & & & & \\
  %   %   & & & & \\
  %   %   & & & & \\
  %   %   & & & & \\
  %   %   & & & & \\
  %   %   & & & & \\
  %   \hline
  % \end{tabular}

  \begin{tabular}{|c|c|c|c|c|}
    \hline
    & \multicolumn{2}{c|}{} & & \\
    Strona & \multicolumn{2}{c|}{Wiersz} & Jest
                              & Powinno być \\ \cline{2-3}
    & Od góry & Od dołu & & \\
    \hline
    % & & & & \\
    % & & & & \\
    585 & &  7 & London. & London \\
    586 &  5 & & Struggle & \emph{Struggle} \\
    586 & 15 & & \emph{Chalcedon} & \emph{Chalcedon} \\
    586 & & 15 & \emph{1} & \emph{the~First} \\
    586 & & 10 & Danielou, Jean & Danielou Jean \\
    586 & & 10 & Henri Marrou & Marrou Henri \\
    587 & 13 & & \emph{A.D.} & \emph{A.D.}, \\
    587 & & 17 & \emph{Jerome,} & \emph{Jerome:} \\
    587 & &  8 & London, & London \\
    587 & &  2 & \emph{Moesia, a~History} & \emph{Moesia: History} \\
    588 &  2 & & \emph{Arthur, a~History} & \emph{Arthur: A~History} \\
    588 &  4 & & \emph{Invasion; the Making}
           & \emph{Invasion: The making} \\
    588 & 15 & & \emph{God; the Life} & \emph{God: The Life} \\
    588 & & 18 & \emph{Britain s} & \emph{Britain's} \\
    588 & & 17 & \emph{Chalcedon, a~Historical}
           & \emph{Chalcedon: A~Historical} \\
    590 & 14 & & \emph{Constantinople; Ecclesiastical}
           & \emph{Constantinople: Ecclesiastical} \\
    591 & & 12 & London, & London \\
    592 & 16 & & \emph{Lyons, Churchman} & \emph{Lyons: Churchman} \\
    592 & &  9 & \emph{Great, the~King} & \emph{Great: The~King} \\
    592 & &  8 & \emph{Canterbury; a~Study} & \emph{Canterbury: A~Study} \\
    592 & &  3 & \emph{Slavs; Saints} & \emph{Slavs: Saints} \\
    593 &  7 & & \emph{Empire; the~Arabs} & \emph{Empire: The~Arabs} \\
    593 & 12 & & \emph{Byzantium: the~Imperial}
           & \emph{Byzantium: The~Imperial} \\
    593 & 14 & & \emph{Kings; Their} & \emph{Kings: Their} \\
    593 & 16 & & \emph{England; a~History} & \emph{England: A~History} \\
    593 & 20 & & \emph{Great: the~Truth} & \emph{Great: The~Truth} \\
    593 & & 14 & \emph{State; the~Period} & \emph{State: The~Period} \\
    593 & & 12 & \emph{Dragon; Alfred} & \emph{Dragon: Alfred} \\
    593 & &  5 & \emph{St.~Peter; the~Birth}
           & \emph{St.~Peter: The~Birth} \\
    594 & 12 & & \emph{Dublin: the~History} & \emph{Dublin: The~History} \\
    594 & &  9 & \emph{Desiderius; Montecassino}
           & \emph{Desiderius: Montecassino} \\
    594 & &  1 & \emph{Empire; the~Arabs} & \emph{Empire: The~Arabs} \\
    595 &  1 & & \emph{Rufus; an~Investigation}
           & \emph{Rufus: An~Investigation} \\
    595 &  6 & & \emph{Byzantium: the~Imperial}
           & \emph{Byzantium: The~Imperial} \\
    595 &  8 & & \emph{England; a~History} & \emph{England: A~History} \\
    595 & 11 & & \emph{Kings; Their} & \emph{Kings: Their} \\
    \hline
  \end{tabular}

  \begin{tabular}{|c|c|c|c|c|}
    \hline
    & \multicolumn{2}{c|}{} & & \\
    Strona & \multicolumn{2}{c|}{Wiersz}& Jest
                              & Powinno być \\ \cline{2-3}
    & Od góry & Od dołu & & \\
    \hline
    595 & 16 & & \emph{State: the~Period} & \emph{State: The~Period} \\
    595 & 19 & & \emph{Tancred: a~Study} & \emph{Tancred: A~Study} \\
    595 & & 10 & \emph{Saint Peter; the~Reception}
           & \emph{Saint Peter: The~Reception} \\
    596 &  6 & & \emph{440} & 440 \\
    % Popraw dalsze błędy w indeksie
    % & & & & \\
    % & & & & \\
    % & & & & \\
    % & & & & \\
    % & & & & \\
    % & & & & \\
    % & & & & \\
    % & & & & \\
    % & & & & \\
    % & & & & \\
    % & & & & \\
    % & & & & \\
    % & & & & \\
    % & & & & \\
    % & & & & \\
    % & & & & \\
    % & & & & \\
    % & & & & \\
    % & & & & \\
    % & & & & \\
    % & & & & \\
    % & & & & \\
    % & & & & \\
    % & & & & \\
    % & & & & \\
    % & & & & \\
    % & & & & \\
    % & & & & \\
    % & & & & \\
    % & & & & \\
    % & & & & \\
    % & & & & \\
    % & & & & \\
    \hline
  \end{tabular}
\end{center}
\noi
\StrWd{3}{4} \\
\Jest www. WydawnictwoWektory.pl \\
\Pow  www.WydawnictwoWektory.pl \\
\StrWd{13}{6} \\
\Jest ,,Alexander~of Alexandria'' \\
\Pow  \emph{Alexander~of Alexandria} \\

\vspace{\spaceTwo}





% ##################
\newpage
\Work{ % Autor i tytuł dzieła
  Warren H.~Carroll \\
  ,,Historia Chrześcijaństwa. Tom~IV: Podział Chrześcijaństwa'',
  \cite{CarrollHistoriaChrzecijanstwaTomIV2011} }


\CenterTB{Uwagi}

\CenterTB{Błędy}
\begin{center}
  \begin{tabular}{|c|c|c|c|c|}
    \hline
    & \multicolumn{2}{c|}{} & & \\
    Strona & \multicolumn{2}{c|}{Wiersz} & Jest
                              & Powinno być \\ \cline{2-3}
    & Od góry & Od dołu & & \\
    \hline
    % & & & & \\
    % & & & & \\
    26  & &  2 & \emph{war} & \emph{War} \\
    31  & &  6 & \emph{Blood; a} & \emph{Blood: A} \\
    33  & &  5 & 122=124 & 122-124 \\
    38  & &  2 & \emph{Conquistadors; First-Person}
           & \emph{Conquistadors; First-Person} \\
    39  & &  5 & (cytat) ; & (cytat); \\
    40  & &  2 & \emph{America; the} & \emph{America: The} \\
    41  & &  2 & s.384 & s.~384 \\
    48  & &  3 & 309; Zob. & 309; zob. \\
    49  & &  2 & s.99 & s.~99 \\
    67  & &  3 & s.84 & s.~84 \\
    78  & &  3 & & \\
    97  & &  2 & Marriman,\emph{Suleiman} & Marriman, \emph{Suleiman} \\
    97  & &  2 & 94;von & 94; von\\
    97  & &  2 & X,s.& X, s. \\
    110 & &  2 & \emph{Won; the} & \emph{Won: The} \\
    111 & &  5 & \emph{Cross. A} & \emph{Cross: A} \\
    112 & &  7 & \emph{Enemies. The} & \emph{Enemies: The} \\
    115 & &  5 & \emph{Master. A} & \emph{Master: A} \\
    130 & &  9 & s.278 & s.~278 \\
    139 & &  3 & s.31 & s.~31 \\
    149 & &  3 & \emph{Vasas. A} & \emph{Vasas: A} \\
    160 & &  5 & \emph{America. The} & \emph{America: The} \\
    163 & &  4 & \emph{is} & \emph{Is} \\
    176 & &  1 & Pastor,\emph{History} & Pastor, \emph{History} \\
    180 & &  2 & \emph{Towns. A} & \emph{Towns: A} \\
    181 & &  2 & \emph{Altars. Traditional} & \emph{Altars: Traditional} \\
    217 & & 14 & \emph{Calvin. The} & \emph{Calvin: The} \\
    217 & &  6 & \emph{Calvin, the} & \emph{Calvin: The} \\
    229 & &  1 & \emph{V, King} & \emph{V: King} \\
    230 & &  2 & \emph{V, King} & \emph{V: King} \\
    234 & &  5 & \emph{VI, the} & \emph{VI: The} \\
    236 & &  7 & \emph{VI, the} & \emph{VI: The} \\
    236 & &  5 & \emph{VI, the} & \emph{VI: The} \\
    \hline
  \end{tabular}

  \begin{tabular}{|c|c|c|c|c|}
    \hline
    & \multicolumn{2}{c|}{} & & \\
    Strona & \multicolumn{2}{c|}{Wiersz} & Jest
                              & Powinno być \\ \cline{2-3}
    & Od góry & Od dołu & & \\
    \hline
    236 & &  3 & \emph{VI, the} & \emph{VI: The} \\
    237 & &  6 & \emph{VI, the} & \emph{VI: The} \\
    237 & &  5 & \emph{and} & \emph{and the} \\
    238 & &  5 & \emph{VI, the} & \emph{VI: The} \\
    238 & &  3 & \emph{VI, the} & \emph{VI: The} \\
    239 & &  5 & \emph{VI, the} & \emph{VI: The} \\
    239 & &  3 & \emph{VI, the} & \emph{VI: The} \\
    % & & & & \\
    249 & &  1 & \emph{Mass, and} & \emph{Mass and} \\
    % & & & & \\
    % & & & & \\
    % & & & & \\
    % & & & & \\
    % & & & & \\
    % & & & & \\
    % & & & & \\
    % & & & & \\
    % & & & & \\
    % & & & & \\
    % & & & & \\
    % & & & & \\
    % & & & & \\
    % & & & & \\
    % & & & & \\
    % & & & & \\
    % & & & & \\
    % & & & & \\
    % & & & & \\
    % & & & & \\
    % & & & & \\
    % & & & & \\
    % & & & & \\
    % & & & & \\
    % & & & & \\
    % & & & & \\
    % & & & & \\
    % & & & & \\
    % & & & & \\
    \hline
  \end{tabular}

  % \begin{tabular}{|c|c|c|c|c|}
  %   \hline
  %   & \multicolumn{2}{c|}{} & & \\
  %   Strona & \multicolumn{2}{c|}{Wiersz} & Jest
  %                             & Powinno być \\ \cline{2-3}
  %   & Od góry & Od dołu & & \\
  %   \hline
  %   %   & & & & \\
  %   %   & & & & \\
  %   %   & & & & \\
  %   %   & & & & \\
  %   %   & & & & \\
  %   %   & & & & \\
  %   %   & & & & \\
  %   %   & & & & \\
  %   %   & & & & \\
  %   %   & & & & \\
  %   %   & & & & \\
  %   %   & & & & \\
  %   %   & & & & \\
  %   %   & & & & \\
  %   %   & & & & \\
  %   %   & & & & \\
  %   %   & & & & \\
  %   %   & & & & \\
  %   %   & & & & \\
  %   %   & & & & \\
  %   %   & & & & \\
  %   %   & & & & \\
  %   %   & & & & \\
  %   %   & & & & \\
  %   %   & & & & \\
  %   %   & & & & \\
  %   %   & & & & \\
  %   %   & & & & \\
  %   %   & & & & \\
  %   %   & & & & \\
  %   %   & & & & \\
  %   %   & & & & \\
  %   %   & & & & \\
  %   %   & & & & \\
  %   %   & & & & \\
  %   %   & & & & \\
  %   %   & & & & \\
  %   %   & & & & \\
  %   \hline
  % \end{tabular}

    % \begin{tabular}{|c|c|c|c|c|}
  %   \hline
  %   & \multicolumn{2}{c|}{} & & \\
  %   Strona & \multicolumn{2}{c|}{Wiersz} & Jest
  %                             & Powinno być \\ \cline{2-3}
  %   & Od góry & Od dołu & & \\
  %   \hline
  %   %   & & & & \\
  %   %   & & & & \\
  %   %   & & & & \\
  %   %   & & & & \\
  %   %   & & & & \\
  %   %   & & & & \\
  %   %   & & & & \\
  %   %   & & & & \\
  %   %   & & & & \\
  %   %   & & & & \\
  %   %   & & & & \\
  %   %   & & & & \\
  %   %   & & & & \\
  %   %   & & & & \\
  %   %   & & & & \\
  %   %   & & & & \\
  %   %   & & & & \\
  %   %   & & & & \\
  %   %   & & & & \\
  %   %   & & & & \\
  %   %   & & & & \\
  %   %   & & & & \\
  %   %   & & & & \\
  %   %   & & & & \\
  %   %   & & & & \\
  %   %   & & & & \\
  %   %   & & & & \\
  %   %   & & & & \\
  %   %   & & & & \\
  %   %   & & & & \\
  %   %   & & & & \\
  %   %   & & & & \\
  %   %   & & & & \\
  %   %   & & & & \\
  %   %   & & & & \\
  %   %   & & & & \\
  %   %   & & & & \\
  %   %   & & & & \\
  %   \hline
  % \end{tabular}

  \begin{tabular}{|c|c|c|c|c|}
    \hline
    & \multicolumn{2}{c|}{} & & \\
    Strona & \multicolumn{2}{c|}{Wiersz} & Jest
                              & Powinno być \\ \cline{2-3}
    & Od góry & Od dołu & & \\
    \hline
    816 & &  2 & 223- & 223, \\
    817 &  4 & & \emph{Towns; a} & \emph{Towns: The} \\
    817 &  7 & & \emph{II, King} & \emph{II: King} \\
    817 & 10 & & \emph{Northumberland; the} & \emph{Northumberland: The} \\
    817 & 12 & & Hilaire. & Hilaire, \\
    817 & 13 & & \emph{during} & \emph{During} \\
    817 & 14 & & \emph{Absolutism. A} & \emph{Absolutism: A} \\
    817 & & 11 & \emph{V, King} & \emph{V: King} \\
    817 & &  7 & Henrich. & Henrich, \\
    817 & &  5 & Bradford, & Bradford \\
    817 & &  2 & Anthony. & Anthony, \\
    817 & &  2 & \emph{Magnificent, Scourge}
           & \emph{Magnificent: Scourge} \\
    818 &  3 & & \emph{Bellarmine, Saint} & \emph{Bellarmine: Saint} \\
    818 &  4 & & \emph{Loyola; the} & \emph{Loyola: The} \\
    818 &  8 & & \emph{Darts. The} & \emph{Darts: The} \\
    818 & & 18 & A~\emph{History} & \emph{A~History} \\
    818 & & 12 & \emph{Playground. A} & \emph{Playground: A} \\
    818 & &  6 & \emph{Altars; Traditional} & \emph{Altars: Traditional} \\
    818 & &  6 & \emph{c.~1400-c. 1580} & \emph{1400-1580} \\
    818 & &  4 & E.H. & E.H., \\
    818 & &  1 & Philippe. & Philippe, \\
    % Popraw dalszą część bibliografii
    819 & &  9 & 1913 & 1913. \\
    % & & & & \\
    % & & & & \\
    % & & & & \\
    % & & & & \\
    820 & &  1 & 1992.. & 1992. \\
    823 &  2 & & \emph{1621--9} & \emph{1621--1629} \\
    823 &  6 & & \emph{1520--21} & \emph{1520--1521} \\
    823 & 17 & & (red.). & (red.), \\
    825 &  5 & & Charles. & Charles, \\
    825 & 11 & & John., & John, \\
    825 & & 15 & \emph{World; Our} & \emph{World: Our} \\
    825 & & 14 & \emph{the~Sea; the~Treasure} & \emph{the~Sea:
                                                The~Treasure} \\
    825 & &  5 & Carlos. & Carlos, \\
    825 & &  2 & Parkman, Francis. & Parkman Francis, \\
    825 & &  1 & Francis. & Francis, \\
    \hline
  \end{tabular}

  \begin{tabular}{|c|c|c|c|c|}
    \hline
    & \multicolumn{2}{c|}{} & & \\
    Strona & \multicolumn{2}{c|}{Wiersz} & Jest
                              & Powinno być \\ \cline{2-3}
    & Od góry & Od dołu & & \\
    \hline
    826 &  7 & & \emph{Letters} & \emph{Times} \\
    826 & 12 & & St.~Louis. & St.~Louis \\
    826 & &  8 & \emph{leyasu} & \emph{Ieyasu} \\
    826 & &  4 & R.S. & R.S., \\
    %   & & & & \\
    %   & & & & \\
    %   & & & & \\
    %   & & & & \\
    %   & & & & \\
    %   & & & & \\
    %   & & & & \\
    %   & & & & \\
    %   & & & & \\
    %   & & & & \\
    %   & & & & \\
    %   & & & & \\
    %   & & & & \\
    %   & & & & \\
    %   & & & & \\
    %   & & & & \\
    %   & & & & \\
    %   & & & & \\
    %   & & & & \\
    %   & & & & \\
    %   & & & & \\
    %   & & & & \\
    %   & & & & \\
    %   & & & & \\
    %   & & & & \\
    %   & & & & \\
    %   & & & & \\
    %   & & & & \\
    %   & & & & \\
    %   & & & & \\
    %   & & & & \\
    %   & & & & \\
    %   & & & & \\
    %   & & & & \\
    \hline
  \end{tabular}
\end{center}
\noi
\StrWd{165}{7} \\
\Jest Bruce,\emph{AnneBoleyn},s.293,299-307,313-333;Scarisbrick,\emph{HenryVIII},s.349-350;Ridley,\emph{Cran-} \\
\Pow  Bruce, \emph{Anne Boleyn}, s.~293, 299-307, 313-333; Scarisbrick, \emph{Henry VIII}, s.~349-350; Ridley, \emph{Cran-} \\
\StrWd{165}{6} \\
\Jest
\emph{mer},s.106-111.AnnęBoleynstracono19maja1536roku.Miałazaledwiedwadzieściaosiemlat. \\
\Pow \emph{mer}, s.~106-111. Annę Boleyn stracono 19~maja 1536 roku.
Miała zaledwie dwadzieścia osiem lat. \\

\vspace{\spaceTwo}





% ##################
\Work{ % Autor i tytuł dzieła
  Warren H.~Carroll, Anne W. Carroll \\
  ,,Historia Chrześcijaństwa. Tom~VI: Kryzys Chrześcijaństwa'',
  \cite{CarrollHistoriaChrzecijanstwaTomVI2014} }


\CenterTB{Uwagi}

\start \Str{12} Wcięcia wszystkich akapitów poza pierwszy~są zbyt
duże.

\vspace{\spaceFour}


\start \StrWd{31}{4--2} Zdanie ,,Jestem zobowiązany Jamesowi
H.~Billingtonowi, \emph{Fire In the~Minds~of Man}, wielkiemu
historykowi myśli rewolucyjnej'' po polsku brzmi źle i~jest trochę bez
sensu. Nie wiem jednak jak je~poprawić.

\vspace{\spaceFour}


\start \Str{33} Jest dziwne, że~Lamennais jest tu nazwany ,,wielkim,
choć czasami błądzącym, francuskim duchownym'', skoro sama ta książka
podaje na~43 stronie, że~odrzuci on najpierw wiarę katolicką, potem
zaś chrześcijaństwo. Możliwe, że~ta nielogiczność jest wyniki
pośmiertnej edycji i~uzupełniania tego dzieła oraz pracy tłumacza.

\vspace{\spaceFour}


\start \Str{54} Pisze tu, że~bitwa pod Nowym Orleanem była decydującym
momentem w~Wojnie~1812 roku, powołując~się na książkę Paula Johnsona
\emph{Birth~of the~Modern}. Jednak w~tej pozycji Johnson przedstawia
zupełnie inną wersję wydarzeń. Bitwa ta rozegrała~się już po zawarciu
pokoju w~Londynie \red{Sprawdź miasto}, ale~przed tym jak statek
z~informacją o~tym dotarła do~USA, jej przebieg nie doprowadził jednak
do~kontynuacji działań wojennych. Tym samym, konkluduje Johnson, nie
wpłynęła na zawarcie pokój, ale~bardzo na~jego recepcję. Amerykanie
mogli~się bowiem czuć zwycięzcami wojny jako, że~wygrali ostatnią jej
bitwę.

\vspace{\spaceFour}


\start \Str{63} Możliwe, że~informacje podane na tej i~na następnych
stronach dotyczące Ameryki Łacińskiej są poprawne, jednak napisane są
w~sposób pełen luk i~niejasności. Na~przykład na dole tej strony jest
podane, że~Martin skapitulował przed Monteverdim i~wyjechał
z~Wenezueli, zaraz potem zaś~został zdradzony, aresztowany i~wysłany
przez Bolivara do~Hiszpanii w~zamian za paszport, który umożliwi mu
przyjazd do~Starego Kraju. Wydaje~się mało prawdopodobne, by~Bolivar
mógł aresztować Martina, gdyby ten opuścił już Wenezuelę.

Poza tym, nie ma żadnego jasnego stwierdzenia, że~Bolivar wykorzystał
paszport i~udał~się do~Hiszpanii. Zaraz po~informacji, że~zdobył ten
dokument przenosimy~się do Trujillo dnia 15~czerwca 1813, co może
oznaczać miasto w~Hiszpania, ale~też jedno z~wielu o~takiej nazwie
w~Ameryce Południowej. Pierwszym pewnym miejsce w~którym go potem
widzimy, jest wenezuelska Barcelona.

\vspace{\spaceFour}


\start \StrWd{67}{8} Po~tej linii powinien nastąpić odstęp między
przypisami.

\vspace{\spaceFour}


\start \Str{76} Następcą zmarłego w~1820~roku Jerzego~III
Hanowerskiego był jego najstarszy syn Jerzy~IV Hanowerski panujący
w~latach 1820--1830. Dopiero po~nim panował w~latach 1830--1837
panował Wilhelm~IV, który był młodszym synem Jerzego~III, a~nie jego
dalekim krewnym. Z~tego tej karygodnej pomyłki wszelkie dalsze
odniesienia do~działań tego monarchy mogą być błędnie przypisanymi mu
aktami Jerzego~IV, bądź źle umieszczone w~czasie.

\vspace{\spaceFour}


\start \StrWd{83}{20--17} Zdanie ,,Tak samo było w~przypadku Lenina,
kolejnego wielkiego przywódcy rewolucji, który wychował~się w~pobożnej
chrześcijańskiej rodzinie, a~fakt, że~wedle jego własnego świadectwa,
utracił wiarę w~wieku szesnastu lat, nie miał na~to żadnego wpływu.''
źle brzmi i~bardzo trudno zrozumieć myśl jaką w~tym kontekście miało
przekazywać.

\vspace{\spaceFour}


\start \StrWd{85}{11} Gwiazdka w~tej linii jest za~mała.

\vspace{\spaceFour}


\start \Str{110} Na~tej stronie jest podane, że~gdy~w~1914 roku
zamordowano arcyksięcia Franciszka Ferdynanda i~jego żonę Zofię,
Franciszkowi Józefowi wyrwał~się raz jedyny okrzyk ,,Nie oszczędzono
mi niczego!'', podczas gdy na~stronie~115 jest napisane, iż~wykrzyknął
on ,,Nie oszczędzono mi niczego na~tej ziemni'' w~momencie,
gdy~dowiedział~się o~zamordowaniu swojej żony Elżbiety. Te~dwa
fragmenty zdają~się sobie przeczyć.

\vspace{\spaceFour}


\start \Str{125} W~drugim paragrafie na~tej stronie jest trochę
zamieszani. Na~początku jest mowa o~zebraniu 87 osób szwajcarskim
Vevey. Na~samym jego końcu jest mowa o~głosowaniu w~kortezach i~ilości
głosów jaka tam padła, co~nie ma chyba nic wspólnego z~tym zebraniem
i~ilością osób która na nim była, nie~pamiętam zaś aby w~tej książce
była podana ilość osób zasiadających w~kortezach.

\vspace{\spaceFour}


\start \StrWd{126}{8} Nie wiem czemu w~tej linii umieszczono słowa
\emph{Dios! Patria! Fueros! Rey!}

\vspace{\spaceFour}


\start \Str{135} Fragment utworu poety Grillparzera o~marszałku
Radetzkim jest tu cytowany z~innego źródła niż na~następnej stronie.
Nie jest to żaden błąd, jedynie trochę to dziwne.

\vspace{\spaceFour}


\start \Str{145} Dwa ostatnie paragrafy nie~mają wcięcia w~tekście.

\vspace{\spaceFour}


\start \Str{147} Stwierdzenie, że~to święty Piotr ustanowił papiestwo
i~, ten błąd jest szczególnie karygodny, Kościół jest sprzeczne
z~wiarą katolicką. Zapewne jest to herezja, lecz nie jestem na tyle
kompetentny by~stwierdzić to na 100\%. Jeśli jest to herezja, to
wątpię by obarczała sumienie Carrolla, który zapewne po prostu
popełnił głupi błąd pisząc te słowa.

\vspace{\spaceFour}


\start \Str{151} Przynajmniej w~mojej opinii na~tej stronie panuje
pewne zamieszanie. Nie potrafię na~przykład z~całą pewnością
stwierdzić, które z~wydarzeń opisanych w~ostatnim paragrafie
odnoszą~się do~pierwszego synodu, a~które do drugiego.

\vspace{\spaceFour}


\start \StrWd{165}{14--12} Sens zdania ,,Wielu opuszczało ojczyznę,
wypływając do~USA z~niewielkich portów, a~ich nazwiska przetrwały
tylko w~lokalnej tradycji.'' jest następujący. Pamięć o~tym, kto
wówczas wypłynął do~Stanów Zjednoczonych zachowała~się w~lokalnej
tradycji ustnej, ale~nie w~dokumentach z~tamtej epoki. W~tym sensie
ich nazwiska nie przetrwały w~źródłach, nie należy jednak przez to
rozumieć, że~ich nazwiska zniknęły z~użycia, co taka forma tego zdania
może sugerować.

\vspace{\spaceFour}


\start \Str{173} Mam problem ze zrozumieniem opisanych tu powodów
wybuchu wojny francusko-pruskiej. Dlaczego niby informacja
o~tym, że~Niemcy obrażają Francuzów wysłana do~króla Prus Wilhelma
miała spowodować wypowiedzenie wojny przez Napoleona~III.

\vspace{\spaceFour}


\start \Str{218} Na~dole strony pozostawiono puste miejsce, które
powinien zajmować tekst z~następnej strony.

\vspace{\spaceFour}


\start \StrWd{225}{3} Po tej linii następuje za~duży odstęp.

\vspace{\spaceFour}


\start \Str{264} Dwa pierwsze paragrafy są źle sformatowane.

\vspace{\spaceFour}


\start \Str{274} Na~dole strony pozostawiono puste miejsce, które
powinien zajmować tekst z~następnej strony.

\vspace{\spaceFour}


\start \Str{277} Należy sprawdzić, czy w~czasie Powstania Tajpingów
nie zginęło na~polach bitew więcej osób, niż podczas I~Wojny
Światowej. Uwaga którą tu poczynił Carroll\footnote{Myślę, że~Anne
  W.~Carroll zgodziłaby~się na~przyznanie autorstwa jej mężowi
  Warrenowi.}, należy mieć na uwadze czytając to~co pisze
on~o~I~Wojnie Światowej na~stronach 867 i~873.

\vspace{\spaceFour}


\start \StrWd{299}{1} Czcionka w~tej linii jest za~duża.

\vspace{\spaceFour}


\start \StrWd{305}{4} Imię ojca Rasputina Efima, na~str.~313 jest
pisane ,,Jefim''.

\vspace{\spaceFour}


\start \StrWd{352}{1} Czcionka w~tej linii jest za~duża.

\vspace{\spaceFour}


\start \Str{355} W~pierwszym paragrafie jest mowa o~głosowaniu które
zakończyło się wynikiem siedem do~pięciu, później zaś, że~decyzja
o~pokoju z~Niemcami przeszła stosunkiem siedem do~czterech. Najpewniej
w~obu przypadkach mowa jest o~tym samym głosowaniu i~jeden z~podanych
wyników jest błędny.

\vspace{\spaceFour}


\start \Str{383} Jeśli niczego nie przeoczyłem, to w~tym miejscu
ostatni raz jest mowa o~Denikinie i~jego armii, gdy wycofują~się
na~Kubań i~Krym. Nie dowiadujemy~się tym samym jakie były ich dalsze
losy.

\vspace{\spaceFour}


\start \Str{397} Ponieważ Polska, zapewne tak samo, jak kraje
nadbałtyckie, nie istniała w~1914~r., jest nieprawdopodobne, by
w~memorandum Erzberga była mowa o~nich jako o~sąsiadujących
z~Niemcami. Należy~się domyślać, że~Erzberg chciał włączenia
wszystkich ziem które można było uznać za w~jakimś sensie polskie,
analogicznie dla~państw nadbałtyckich, do~Cesarskich Niemiec po
wygranej wojnie.

\vspace{\spaceFour}


\start \StrWg{416}{22} Po tej linii powinien być większy odstęp.

\vspace{\spaceFour}


\start \StrWd{432}{8} Na~podstawie wcześniejszej części książki nie
jestem w~stanie powiedzieć o~co chodziło w~sprawie nadużyć w~Gruzji.

\vspace{\spaceFour}


\start \Str{435} Na~dole strony pozostawiono puste miejsce, które
powinien zajmować tekst z~następnej strony.

\vspace{\spaceFour}


\start \StrWd{437}{8} Wydaje mi~się, że~spotkałem~się z~wersją,
iż~Trocki został zabity ciosem czekanem. Należy to sprawdzić jeszcze
w~jakiejś innej pracy.

\vspace{\spaceFour}


\start \StrWg{441}{1--2} Szacunki Carrollów, że~w~Chinach żyła jedna
trzecia ludności świata, budzą pewne moje wątpliwości. Po~pierwsze
należałoby ustalić o~jakim okresie czasu mowa, po~drugie należałoby
sprawdzić, jak rzeczywiście przedstawiał~się stosunek ludności Chin do
ludności świata.

\vspace{\spaceFour}


\start \StrWg{445}{17} Deng Xiaoping żył w~latach 1904--1997, zaś za
moment przejęcia jego władzy po~Mao Zedongu, który zmarł w~1976 roku,
należy chyba przyjąć rok~1978. Xiaoping miał więc wtedy nie
dziewięćdziesiąt lecz siedemdziesiąt cztery lata.

\vspace{\spaceFour}


\start \StrWg{451}{8} Tu~można powtórzyć wątpliwości odnośnie podanej
ludności~Chin i~jej udziału w~ludności świata, które~są w~komentarzu
do~strony~441.

\vspace{\spaceFour}


\start \StrWd{451}{6} Ten wiersz jest źle wcięty.

\vspace{\spaceFour}


\start \Str{456} Na~dole strony pozostawiono puste miejsce, które
powinien chyba zajmować tekst z~następnej strony. Choć w~tym wypadku
możliwe jest, że~obecny wybór jest lepszy.

\vspace{\spaceFour}


\start \StrWd{456}{11} Znak ,,*'' jest w~tej linii za mały.

\vspace{\spaceFour}


\start \StrWd{466}{18} Ponieważ ten cytata zaczyna~się z~małej litery,
do~tego zaraz następuje znak~,,\ld'', co sugeruje, że~ten cytat został
błędnie przytoczony. Jednak nie wiem jak go~poprawić.

\vspace{\spaceFour}


\start \StrWd{466}{1} To~odwołanie bibliograficzne jest niedokończone.

\vspace{\spaceFour}


\start \StrWd{478}{5--6} W~mojej opinii te~dwie linie~są źle
sformatowane.

\vspace{\spaceFour}


\start \Str{494} Stwierdzenie, że~Hitler i~Stalin byli w~owym czasie
najpotężniejszymi ludźmi na świecie jest dyskusyjne. Dorównywał ich
sile prezydent USA, można też dywagować, czy ludzie władający Japonią,
nie byli równie potężni\footnote{Nie jestem pewien, czy w~owym czasie
  cała Japońska władza należało do~cesarza Hirohito, stąd takie
  sformułowanie tego punktu.}.

\vspace{\spaceFour}


\start \Str{497} Należy sprawdzić, czy~rzeczywiście największą grupą
etniczną w~Związku Radzieckim byli Ukraińcy.

\vspace{\spaceFour}


\start \Str{497} Należy sprawdzić, jakie naprawdę temperatury~są, czy
może raczej były, zimą na~terenach Syberii. Temperatury rzędu
-90\textcelsius są~w~mojej ocenie mało prawdopodobne. Jeśli jednak
Carrollom chodziło o~-90 stopni Fahrenheita, to~oznaczałoby około
-65\textcelsius, co~jest już znacznie bardziej możliwe.

\vspace{\spaceFour}


\start{499} Gdyby na~31 milinów Ukraińców przypadało 11~milionów ton
zboża to na głowę przypadałoby nie jak piszą Carrollowie
123~kilogramy, lecz~355~kilogramów zboża. Jeśli zaś byłoby to 10
milionów ton zboża, to~na jednego Ukraińca przypadałoby 322 kilogramów
zboża. Ktoś wyraźnie coś pomylił w~rachunkach.

\vspace{\spaceFour}


\start \StrWg{503}{16} W~tym wersie znajduje~się malutkie wcięcie,
którego nie~powinno być. Osobiście uważam też, że~lepiej brzmiałby
w~następującej postaci: ,,Nadejdzie rok czarny, rok krwi i~pożarów''.

\vspace{\spaceFour}


\start \StrWd{503}{2} Znajdujące~się na~końcu tej linii~,,w:'',
wygląda bardzo źle. Należałoby je przenieść na~początek następnej
linii.

\vspace{\spaceFour}


\start \StrWd{507}{4} Ponad tą~linią powinien znajdować~się odstęp.

\vspace{\spaceFour}


\start \StrWd{516}{6} W~tej linii jest wcięcie, którego według mnie
nie~powinno tu~być.

\vspace{\spaceFour}


\start \StrWd{526}{6} Ciężko jest zrozumieć od~razu, kim był
przywoływany w~tej linii Jakub. Jest to zapewne karlistowski następca
tronu, o~którym była mowa w~rozdziale \emph{Zwycięstwo i~klęska
  tradycjonalistów}. \red{Sprawdź kiedyś jak~się on~dokładnie
  nazywał.}

\vspace{\spaceFour}


\start \StrWg{527}{17} Wcięcie tego akapitu jest zbyt duże.

\vspace{\spaceFour}


\start \Str{530} Według podany tu liczb prawica i~centrum zdobyły
razem 210~miejsc w~kortezach więc Front Ludowy z~263 miał 53, nie 26
miejsc przewagi. Albo jakieś ugrupowanie zostało przemilczane, albo~te
liczby zostały źle podane.

\vspace{\spaceFour}


\start \StrWg{537}{6} Skala śmierci w Hiszpańskiej Wojnie Domowej jest
zbyt mała, by~można ją było nazwać holokaustem.

\vspace{\spaceFour}


\start \StrWg{537}{6} Powinno tu~się znaleźć jawne potępienie zbrodni
nacjonalistów. Popełnienie ich w~reakcji nie znosi winy.

\vspace{\spaceFour}


\start \StrWd{540}{7} Wcięcie tego akapitu jest zbyt duże.

\vspace{\spaceFour}


\start \StrWd{543}{1} Do~formy całego przypisu niezbyt pasuje linia
,,Rafael Casa de~la~Vega, \emph{Franco, żołnierz},
tłum.~J.~Chodorowski.''.

\vspace{\spaceFour}


\start \Str{547} Byłoby dziwne, gdyby Churchill ostrzegał Wielką
Brytanię przed Hitlerem w~latach 1924--1932, skoro przez większość
tego czasu był on człowiekiem zupełnie pozbawionym wpływów. Jednak
w~historii zdarzały~się już dziwniejsze rzeczy.

\vspace{\spaceFour}


\start \Str{565} Tekst przypisów~69 i~70 trochę~się ze~sobą nie
zgadzają. W~przypisie 69~autorzy twierdzą, że~praca o~walkach
na~Gudalcana Roberta Leckiego jest pozycją niedoścignioną porównywalną
tylko z~Tukidydesem. Natomiast w~następnym, iż~najlepsza praca w~tym
temacie to~ta autorstwa Samuela Eliota Morisona.

\vspace{\spaceFour}


\start \Str{566} Przypisy od~tłumacza~są źle ponumerowane ilością
gwiazdek.

\vspace{\spaceFour}


\start \StrWg{572}{12} Gwiazdka w~tej linii jest zbyt mała.

\vspace{\spaceFour}


\start \StrWd{585}{17--15} Te~linie~są źle sformatowane.

\vspace{\spaceFour}


\start \StrWd{609}{8} Gwiazdka w~tej linii jest za~mała.

\vspace{\spaceFour}


\start \StrWg{623}{10} Gwiazdka w~tej linii jest zbyt mała.

\vspace{\spaceFour}


\start \StrWd{778}{4--3} Te dwie linie, łącznie z~,,251.'' służącym
za~odnośnik tego przypisu powinny być częścią poprzedniego przypisu.
Samo oznaczenie ,,251.'' jest częścią urwanych w~poprzedniej linii
numerów stron: 250-251.

\vspace{\spaceFour}


\start \StrWd{867}{17} Linia jest źle zedytowana. Drugie zdanie w~tej
linii jest początkiem następnej pozycji w~bibliografii, powinna więc
być zgodnie z~tym sformatowana.

\vspace{\spaceFour}



\newpage
\CenterTB{Błędy}
\begin{center}
  \begin{tabular}{|c|c|c|c|c|}
    \hline
    & \multicolumn{2}{c|}{} & & \\
    Strona & \multicolumn{2}{c|}{Wiersz} & Jest
                              & Powinno być \\ \cline{2-3}
    & Od góry & Od dołu & & \\
    \hline
    7   & &  4 & wszystko$^{ * }$ & wszystko \\
    7   & &  4 & 827 & 837 \\
    7   & &  3 & Rekonkwiście$^{ * }$ & Rekonkwiście \\
    23  & & 10 & \emph{Vhutch} & \emph{Church} \\
    24  & & 25 & Zbawiciela$^{ *^{ * } }$ & Zbawiciela$^{ ** }$ \\
    25  & &  9 & 1919 & 1819 \\
    32  & 11 & & dostosowania & do~stosowania \\
    50  & & 12 & za~panowania & rozpoczęta za~panowania \\
    55  & & 12 & piętnstu & piętnastu \\
    55  & &  7 & interesy & interesy Południa \\
    67  & 17 & & bezbożności'')$^{ 31 }$ & bezbożności''$^{ 31 }$) \\
    67  &  8 & & północy & południa \\
    68  & 21 & & siom & siłom \\
    85  &  1 & & Herald'' Tribune'' & Herald'' \\
    96  & &  2 & W.H. Warren & W.H. Carroll \\
    97  & &  7 & ,,tak uważamy'' & ,,Tak uważamy'' \\
    104 & & 17 & i~związku & i~w~związku \\
    104 & &  2 & W.H.~Warren & W.H.~Carroll \\
    105 & & 11 & Counter-Revolution & Counter-Revolution'' \\
    105 & &  5 & W.H.~Warren & W.H.~Carroll \\
    116 & & 15 & stał~się był & stał~się \\
    117 & & 23 & wyd.3,Boston & wyd.~3, Boston \\
    121 &  3 & & tonizowały & uspokajały \\
    126 & &  4 & Pampelunie. & Pampelunie). \\
    127 & 10 & & aż~przez & potem aż~przez \\
    135 & & 12 & doskonalej & doskonałej \\
    137 &  6 & & go & je \\
    139 & &  5 & \emph{1833} & \emph{1883} \\
    141 & &  8 & tom~VI, rozdział~XIV & rozdział~VIII, \\
    141 & &  7 & P\emph{olitical} & \emph{Political} \\
    141 & &  5 & rozdział zatytułowany & rozdział~II, \\
    150 &  5 & & dogmatach; & dogmatach, \\
    151 & &  8 & torturom... & torturom. \\
    151 & &  7 & miasta.. & miasta. \\
    152 & 22 & & i~i & i \\
    165 & &  7 & 2003) & 2003 \\
    170 & 16 & & potomek & bratanek \\
    \hline
  \end{tabular}

  \begin{tabular}{|c|c|c|c|c|}
    \hline
    & \multicolumn{2}{c|}{} & & \\
    Strona & \multicolumn{2}{c|}{Wiersz} & Jest
                              & Powinno być \\ \cline{2-3}
    & Od góry & Od dołu & & \\
    \hline
    170 & & 10 & potomka, ,,F\"{u}hrera  % ''
           & potomka ,,F\"{u}hrera  %  ''
    \\
    171 & &  7 & skrajnym wręcz & wręcz skrajnym \\
    177 &  3 & & ,,byliście & ,,Byliście \\
    187 & 21 & & roku~Na & roku. Na \\
    190 & &  1 & Karl Marx & \emph{Karl Marx} \\
    194 & 12 & & etc.. & etc. \\
    194 & &  8 & spikerze & mówcy \\
    197 & &  7 & 1987) & 1987 \\
    198 & 16 & & wojny; & wojny \\
    199 & &  2 & 210 & 210. \\
    208 &  7 & & jest & jest natomiast \\
    208 & &  5 & wschodni, wschodni & zachodni, wschodni \\
    210 & &  6 & Pratt,, & Pratt, \\
    210 & &  5 & Carrol & Carroll \\
    223 & & 20 & dna & dnia \\
    228 & &  4 & 2005) & 2005 \\
    229 & & 14 & \emph{Westrn} & \emph{Western} \\
    233 & & 11 & piaty & piąty \\
    235 & 16 & & rzecz & Rzecz \\
    235 & & 15 & Uranu & Urana \\
    237 & &  1 & 1954) & 1954 \\
    239 & 12 & & wyrazili & nie~wyrazili \\
    239 & 15 & & z~zatem & a~zatem \\
    243 &  5 & & uczony; & uczony. \\
    243 &  5 & & roku1743 & roku 1743 \\
    248 & &  1 & 2008) & 2008 \\
    265 & & 13 & si & się \\
    267 & &  3 & (1944 ) & (1944) \\
    269 & &  5 & \emph{Germany, and} & \emph{Germany and} \\
    270 & &  1 & \emph{s.} & s. \\
    273 & &  3 & 1944 & 1994 \\
    294 &  7 & &  % ,,
                 światu''... & światu... \\
    299 & & 11 & Kołłnotaj & Kołłontaj \\
    299 & &  1 & 1988) & 1988 \\
    303 & &  4 &  % ,,
                 \emph{opończa}'' & \emph{opończa} \\
    308 & & 19 & niszczysz & Niszczysz \\
    309 & 22 & & warstw & wszystkich warstw \\
    317 & &  3 & \emph{Kerensky; the} & \emph{Kerensky: The} \\
    \hline
  \end{tabular}

  \begin{tabular}{|c|c|c|c|c|}
    \hline
    & \multicolumn{2}{c|}{} & & \\
    Strona & \multicolumn{2}{c|}{Wiersz} & Jest
                              & Powinno być \\ \cline{2-3}
    & Od góry & Od dołu & & \\
    \hline
    320 & &  3 & Habsburg & \emph{Habsburg} \\
    324 & &  8 & eserowcow & eserowców \\
    327 &  7 & & w~coraz & ludzie w~coraz \\
    330 & &  1 & \emph{Wtnesses} & \emph{Witnesses} \\
    332 & & & Piotrogrodu,. & Piotrogrodu. \\ % Popraw tą linię
    338 & 22 & & roboty!, & roboty! \\
    349 & &  1 & \emph{s.} & s. \\
    351 &  4 & & 1917--1921 & 1914--1922 \\
    365 & &  5 & miasta ; & miasta; \\
    373 & 19 & & rok & rok. \\
    379 & &  2 & 1989) & 1989 \\
    380 &  7 & & zlej & złej \\
    380 & &  6 & destruktywna & destruktywną \\
    381 & &  1 & 1951) & 1951 \\
    385 & 10 & & dopływem & odpływem \\
    387 &  4 & & 1915--1922 & 1914--1922 \\
    387 & &  2 & 1989) & 1989 \\
    392 & &  3 & \emph{s.} & s. \\
    393 & &  6 & \emph{s.} & s. \\
    393 & &  2 & \emph{s.} & s. \\
    394 &  7 & & z & z~dala \\
    394 & &  1 & \emph{XV} , & \emph{XV}, \\
    396 & & 19 & Leonowi XII & Leonowi XIII \\
    408 & &  2 & London1971 & London 1971 \\
    409 & &  1 & \emph{Glory; Poland} & \emph{Glory: Poland} \\
    421 & & 17 & terroru\ld. & terroru\ld \\
    424 & &  5 & Radzieckiej.Trzon & Radzieckiej. Trzon \\
    430 & &  1 & \emph{under} & \emph{Under} \\
    431 & &  2 & \emph{s.} & s. \\
    432 & &  1 & \emph{s.} & s. \\
    433 & &  2 & \emph{s.} & s. \\
    435 &  2 & & krajem.. & krajem. \\
    436 & &  1 & \emph{s.} & s. \\
    442 &  6 & & imperium & imperium Czang \\
    443 &  7 & & doobra & dobra \\
    446 & 10 & & Baun & Braun \\
    448 & 15 & & Hunan Jiangxi. & Huan i~Jangxi \\
    457 & &  3 & \emph{war} & \emph{War} \\
    \hline
  \end{tabular}

  \begin{tabular}{|c|c|c|c|c|}
    \hline
    & \multicolumn{2}{c|}{} & & \\
    Strona & \multicolumn{2}{c|}{Wiersz} & Jest
                              & Powinno być \\ \cline{2-3}
    & Od góry & Od dołu & & \\
    \hline
    457 & &  2 & \emph{war} & \emph{War} \\
    459 & &  2 & \emph{s.} & s. \\
    460 & &  2 & \emph{s.} & s. \\
    461 & & 11 & \emph{s.} & s. \\
    462 & &  4 & \emph{Pro; Modern} & \emph{Pro. Modern} \\
    463 & &  7 & \emph{s.} & s. \\
    464 & &  8 & \emph{s.} & s. \\
    464 & &  3 & \emph{s.} & s. \\
    465 & &  9 & ,,Viva Cristo Rey'' & \emph{Viva Cristo Rey} \\
    465 & &  8 & 194. 199. & 194, 199. \\
    468 &  6 & & \emph{Altars; Baltimore's} & \emph{Altars. Baltimore's} \\
    473 & 14 & & klepsydrze & klepsydrze. \\
    479 & &  6 & \emph{s.} & s. \\
    485 & 12 & & obserwują., wyciągając & obserwują. Wyciągając \\
    487 & & 14 & John a.~Ryan & John A.~Ryan \\
    488 & &  4 & pieniadze & pieniądze \\
    489 & 10 & & prac.$^{151}$ & prac$^{151}$. \\
    497 & 18 & & mniejszością & grupą \\
    498 &  1 & & od & do \\
    502 & 23 & & \emph{od} & \emph{of} \\
    506 & &  5 & zmienić & zmienić zdanie \\
    506 & &  1 & \emph{gwałtownie} & gwałtownie \\
    507 &  3 & & dłużej & długo \\
    507 & &  2 & \emph{Archipelago} III & \emph{Archipelago}, t.~III \\
    511 &  2 & & dotrzeć do~celu & dopłynąć do~celu \\
    514 & 11 & & % ,,
                 Jarosławiu'')$^{ 22 }$  % ,,
           & Jarosławiu''$^{ 22 }$) \\
    514 & &  2 & \emph{labor} & \emph{Labor} \\
    517 & &  5 & Całkowita & całkowita \\
    520 & &  4 & Cronica de~Alfonso~III & \emph{Cronica de~Alfonso~III} \\
    521 & &  1 & wyd & wyd. \\
    522 & 18 & & kardynałem & kardynałem. \\
    523 & & 20 & roku & roku. \\
    525 &  8 & & z~gabinecie & w~gabinecie \\
    527 & & & Toledo$^{ 18 }$\tb{.} & Toledo$^{ 18 }$. \\
    529 & 20 & & lewacki & lewicowy \\
    529 & &  3 & t.~Ivm Mardird & t.~I, Madrid \\
    530 & 21 & &  % ,,
                 \emph{Rey}'' & \emph{Rey} \\
    531 & & 19 & Zamora\tb{:} & Zamora: \\
    \hline
  \end{tabular}

  \begin{tabular}{|c|c|c|c|c|}
    \hline
    & \multicolumn{2}{c|}{} & & \\
    Strona & \multicolumn{2}{c|}{Wiersz} & Jest
                              & Powinno być \\ \cline{2-3}
    & Od góry & Od dołu & & \\
    \hline
    531 & & 18 &  % ,,
                 komunizmu'')$^{ 30 }$\tb{.}
                 % ,,
           & komunizmu'')$^{ 30 }$. \\
    531 & & 13 & W~spólnota & Wspólnota \\
    532 & 14 & & Cywilną$^{ 30 }$ & Cywilną \\
    532 & &  3 & roli,,  % ''
           & roli, \\
    535 & &  2 & \emph{mar tyrs} & \emph{Martyrs} \\
    536 & &  5 & \emph{into} & \emph{Into} \\
    537 & & 22 & Reyes)\tb{.} & Reyes). \\
    537 & &  2 & \emph{1936} , & \emph{1936}, \\
    540 & &  9 & docenili & doceniliby \\
    541 &  8 & & (republikańskie & (Republikańskie \\
    541 & 10 & & komunistom). & komunistom.) \\
    541 & 15 & & faszyzmu''$^{ 55 }$\tb{.} & faszyzmu''$^{ 55 }$. \\
    542 & & 21 & przywódca & Przywódca \\
    542 & & 21 & Boga'' & Boga''. \\
    542 & &  6 & jednym & ,,jednym  % ''
    \\
    % Mogłem źle wyznaczyć początek cytatu.
    545 & &  4 & najzacieklejszych,. & najzacieklejszych \\
    545 & &  1 & red & red. \\
    548 & & 10 & 1987) & 1987 \\
    548 & &  5 & 1969) & 1969 \\
    549 & 18 & & obliczu & w~obliczu \\
    553 & &  3 & \emph{1939--1940} , & \emph{1939--1940}, \\
    555 & & 18 & także teraz & teraz także \\
    555 & &  5 & \emph{s.} & s. \\
    556 & &  4 & \emph{s.} & s. \\
    556 & &  1 & \emph{s.} & s. \\
    557 & &  4 & \emph{s.} & s. \\
    559 & & 12 & \emph{s.} & s. \\
    561 & &  4 & \emph{fate} & \emph{Fate} \\
    562 & &  1 & 1985) & 1985 \\
    563 & 20 & & dwa & trzy \\
    564 &  1 & & samuraje & samurajowie \\
    565 & &  3 & kampanii; & kampanii. \\
    566 &  1 & & wspanialej & wspaniałej \\
    566 & 17 & & zaporami\ld & zaopatrzeniem; \\
    566 & & 15 & Piekle!'' & Piekle!''.'' \\
    566 & & 12 & W~szakżeście& Wszakżeście \\
    570 &  4 & & wschodu & zachodu \\
    571 &  1 & & macDonald & MacDonald \\
    \hline
  \end{tabular}

  \begin{tabular}{|c|c|c|c|c|}
    \hline
    & \multicolumn{2}{c|}{} & & \\
    Strona & \multicolumn{2}{c|}{Wiersz} & Jest
                              & Powinno być \\ \cline{2-3}
    & Od góry & Od dołu & & \\
    \hline
    574 & &  5 & \emph{Preious} & \emph{Precious} \\
    575 & &  3 & 1997) & 1997 \\
    576 & &  1 & \emph{s.} & s. \\
    581 & &  1 & 212,. & 212. \\
    583 & &  7 & \emph{ThePrice} & \emph{The~Price} \\
    587 & & 10 & 1984) & 1984 \\
    591 & & 12 & \emph{Lost; American} & \emph{Lost: American} \\
    % & & & & \\
    % & & & & \\
    % & & & & \\
    601 & &  3 & \emph{1949} & \emph{1949}, \\
    601 & &  2 & \emph{Confoct} & \emph{Conflict} \\
    % & & & & \\
    % & & & & \\
    605 & &  2 & \emph{Hungary from} & \emph{Hungary: From} \\
    607 & &  2 & \emph{between} & \emph{Between} \\
    609 & &  4 & kraju, W & kraju. W \\
    612 & &  3 & \emph{balance} & \emph{Balance} \\
    612 & &  2 & 1998) & 1998 \\
    617 & &  8 & Szpiegostwo & szpiegostwo \\
    % & & & & \\
    634 & &  7 & \emph{s.} & s. \\
    % & & & & \\
    % & & & & \\
    646 & 19 & & \emph{Nie} & \emph{nie} \\
    % & & & & \\
    % & & & & \\
    % & & & & \\
    685 & &  8 & roku Isaacs & R. Isaacs \\
    690 & &  5 & \emph{war} & \emph{War} \\
    693 & &  2 & Kambodży & z~Kambodży \\
    % & & & & \\
    \hline
  \end{tabular}

  % \begin{tabular}{|c|c|c|c|c|}
  %   \hline
  %   & \multicolumn{2}{c|}{} & & \\
  %   Strona & \multicolumn{2}{c|}{Wiersz} & Jest
  %   & Powinno być \\ \cline{2-3}
  %   & Od góry & Od dołu & & \\
  %   \hline
  %   %   & & & & \\
  %   %   & & & & \\
  %   %   & & & & \\
  %   %   & & & & \\
  %   %   & & & & \\
  %   %   & & & & \\
  %   %   & & & & \\
  %   %   & & & & \\
  %   %   & & & & \\
  %   %   & & & & \\
  %   %   & & & & \\
  %   %   & & & & \\
  %   %   & & & & \\
  %   %   & & & & \\
  %   %   & & & & \\
  %   %   & & & & \\
  %   %   & & & & \\
  %   %   & & & & \\
  %   %   & & & & \\
  %   %   & & & & \\
  %   %   & & & & \\
  %   %   & & & & \\
  %   %   & & & & \\
  %   %   & & & & \\
  %   %   & & & & \\
  %   %   & & & & \\
  %   %   & & & & \\
  %   %   & & & & \\
  %   %   & & & & \\
  %   %   & & & & \\
  %   %   & & & & \\
  %   %   & & & & \\
  %   %   & & & & \\
  %   %   & & & & \\
  %   %   & & & & \\
  %   %   & & & & \\
  %   %   & & & & \\
  %   %   & & & & \\
  %   \hline
  % \end{tabular}

  \begin{tabular}{|c|c|c|c|c|}
    \hline
    & \multicolumn{2}{c|}{} & & \\
    Strona & \multicolumn{2}{c|}{Wiersz} & Jest
                              & Powinno być \\ \cline{2-3}
    & Od góry & Od dołu & & \\
    \hline
    709 & &  6 & \emph{of} & \emph{to} \\
    % & & & & \\
    % & & & & \\
    % & & & & \\
    778 & &  5 & 250- & 250-251. \\
    785 & &  3 & Paul\_R\_Ehrlich & Paul\_R\_Ehrlich. \\
    790 & &  8 & Centrulo I~Amy & Centrulo i~Amy \\
    791 & &  4 & html & html. \\
    793 & &  7 & 145 & 145. \\
    793 & &  5 & \emph{s.} & s. \\
    797 & &  6 & Zob. & zob. \\
    797 & &  5 & Zob. & zob. \\
    797 & &  4 & Zob. & zob. \\
    797 & &  1 & Las & Last \\
    \hline
  \end{tabular}

  % \begin{tabular}{|c|c|c|c|c|}
  %   \hline
  %   & \multicolumn{2}{c|}{} & & \\
  %   Strona & \multicolumn{2}{c|}{Wiersz} & Jest
  %   & Powinno być \\ \cline{2-3}
  %   & Od góry & Od dołu & & \\
  %   \hline
  %   %   & & & & \\
  %   %   & & & & \\
  %   %   & & & & \\
  %   %   & & & & \\
  %   %   & & & & \\
  %   %   & & & & \\
  %   %   & & & & \\
  %   %   & & & & \\
  %   %   & & & & \\
  %   %   & & & & \\
  %   %   & & & & \\
  %   %   & & & & \\
  %   %   & & & & \\
  %   %   & & & & \\
  %   %   & & & & \\
  %   %   & & & & \\
  %   %   & & & & \\
  %   %   & & & & \\
  %   %   & & & & \\
  %   %   & & & & \\
  %   %   & & & & \\
  %   %   & & & & \\
  %   %   & & & & \\
  %   %   & & & & \\
  %   %   & & & & \\
  %   %   & & & & \\
  %   %   & & & & \\
  %   %   & & & & \\
  %   %   & & & & \\
  %   %   & & & & \\
  %   %   & & & & \\
  %   %   & & & & \\
  %   %   & & & & \\
  %   %   & & & & \\
  %   %   & & & & \\
  %   %   & & & & \\
  %   %   & & & & \\
  %   %   & & & & \\
  %   \hline
  % \end{tabular}

  \begin{tabular}{|c|c|c|c|c|}
    \hline
    & \multicolumn{2}{c|}{} & & \\
    Strona & \multicolumn{2}{c|}{Wiersz} & Jest
                              & Powinno być \\ \cline{2-3}
    & Od góry & Od dołu & & \\
    \hline
    798 & 21 & & \emph{Kościół} & \emph{Kościół~są} \\
    798 & 22 & & \emph{są~Drogą} & \emph{Drogą} \\
    858 &  5 & & Pio Non (bł.~Pius~IX): & \emph{Pio Non (bł.~Pius~IX):} \\
    858 & 19 & & \emph{ofCatholic} & \emph{of Catholic} \\
    858 & 19 & & \emph{History,}(St.~Louis & \emph{History} (St.~Louis \\
    859 & 13 & & portugalskiej & portugalskiej. \\
    860 &  5 & & DuffDavid. & Duff David \\
    861 &  7 & & państwa.. & państwa. \\
    861 & 18 & & wyd.. & wyd. \\
    862 &  6 & & S. John Brown & S., \emph{John Brown} \\
    862 & &  2 & Jen. & Jen, \\
    864 &  5 & & York, & York \\
    864 & 20 & & \emph{against} & \emph{Against} \\
    864 & 21 & & York, & York \\
    866 & &  2 & FDR & \emph{FDR} \\
    867 & 15 & & York, & York \\
    867 & &  2 & York, & York \\
    868 &  3 & & 2004.. & 2004. \\
    868 & 15 & & York, & York \\
    868 & 23 & & York, & York \\
    868 & 25 & & \emph{kardynał} & \emph{Cardinal} \\
    869 & 24 & & \emph{Denikin} & \emph{Denikin.} \\
    869 & & 12 & wojskowości.. & wojskowości \\
    870 &  8 & & 1937) & 1937). \\
    871 & 13 & & 1939 1961 & 1939, 1961 \\
    871 & 17 & & jedneaj & jednej \\
    871 & & 13 & \emph{Day 1918: World} & \emph{Day, 1918. World} \\
    871 & & 12 & \emph{its} & \emph{Its} \\
    871 & &  7 & \emph{under} & \emph{Under} \\
    873 &  1 & & York, & York \\
    873 &  9 & & York, & York \\
    873 & 10 & & York, & York \\
    873 & &  5 & 1958 1966 & 1958, 1966 \\
    874 & &  6 & Najlepsze I~najbardziej & Najlepsze i~najbardziej \\
    \hline
  \end{tabular}
\end{center}
\noi
\StrWg{103}{5} \\
\Jest mianem krucjaty (\emph{la~cruzada} ) określali \\
\Pow  określali mianem krucjaty (\emph{la~cruzada}) \\
\StrWd{170}{10} \\
\Jest ,,F\"{u}hrera z~Poczdamu'', ojca Fryderyka Wielkiego \\
\Pow ,,F\"{u}hrera z~Poczdamu'', Fryderyka Williama~I, ojca Fryderyka
Wielkiego \\
\StrWd{228}{5} \\
\Jest The~Victory~of Reason: How Christianity Led to Freedom,
Capitalism and~Western Success \\
\Pow \emph{The~Victory~of Reason: How Christianity Led to Freedom,
  Capitalism and~Western Success} \\
\StrWg{234}{18} \\
\Jest zapoczątkowujących teorię indukcji elektromagnetycznej \\
\Pow  które doprowadziły do~powstania teorii indukcji elektromagnetycznej \\
\StrWd{237}{1} \\
\Jest Ford: The~Times, the~Man, and~the~Company \\
\Pow  \emph{Ford: The~Times, the~Man, and~the~Company} \\
\StrWd{246}{9} \\
\Jest Alexander Graham Bell and~the~Passion for~Invention \\
\Pow  \emph{Alexander Graham Bell and~the~Passion for~Invention} \\
\StrWd{299}{2} \\
\Jest Three Who Made a~Revolution \\
\Pow  \emph{Three Who Made a~Revolution} \\
\StrWd{383}{18} \\
\Jest Kołczak \\
\Pow  Kołczak doszedł do wniosku \\
\StrWd{507}{14} \\
\Jest \emph{przeciwko zastosowaniu kary śmierci. Co~więcej,
  przekonał do~swego poglądu Politbiuro.} \\
\Pow przeciwko zastosowaniu kary śmierci. Co~więcej, przekonał
do~swego poglądu Politbiuro. \\
\StrWd{545}{10} \\
\Jest i~dwutomowa \emph{Visions~of Glory} (Boston 1983), \emph{Alone} \\
\Pow w~dwóch tomach: \emph{Visions~of Glory} (Boston 1983)
i~\emph{Alone} \\
\StrWd{566}{5} \\
\Jest South Pacific Combat Air Transport \\
\Pow SCAT (\emph{South Pacific Combat Air Transport}) \\
\StrWd{566}{2} \\
\Jest WAC~~(Women's Army Corps) \\
\Pow WAC~(\emph{Women's Army Corps}) \\
\StrWd{790}{3--1} \\
\Jest ,,Akcja afirmatywna w~orzecznictwie Sądu Najwyższego Stanów
Zjednoczonych'', Z~problemów bezpieczeństwa. Prawa człowieka \\
\Pow \emph{Akcja afirmatywna w~orzecznictwie Sądu Najwyższego Stanów
  Zjednoczonych}, w:~\emph{Z~problemów bezpieczeństwa. Prawa człowieka} \\
\StrWd{791}{3} \\
\Jest \emph{Infant Himicides through} \\
\Pow Bogomir Kuhar, \emph{Infant Homicides Through} \\
\StrWd{797}{6} \\
\Jest ,,Why Can't We~Love Them Both?'' \\
\Pow \emph{Why Can't We~Love Them Both?} \\
\StrWd{799}{5} \\
\Jest ,,Pope John Paul~II's Encyclical \emph{Veritatis Splendor}'' \\
\Pow \emph{Pope John Paul~II's Encyclical ,,Veritatis Splendor''} \\

\vspace{\spaceTwo}





% ##################
\Work{ % Autor i tytuł dzieła
  Łukasz Czarnecki \\
  ,,Konstantynopol~626'', \cite{CzarneckiKonstantynopol2017} }


\CenterTB{Uwagi}

\start \StrWg{58}{18} Mam wątpliwość czy~wszystkie słowa wyróżnione
tu~kursywą~są częścią cytowanego fragmentu.

% \vspace{\spaceFour}


\CenterTB{Błędy}
\begin{center}
  \begin{tabular}{|c|c|c|c|c|}
    \hline
    & \multicolumn{2}{c|}{} & & \\
    Strona & \multicolumn{2}{c|}{Wiersz} & Jest
                              & Powinno być \\ \cline{2-3}
    & Od góry & Od dołu & & \\
    \hline
    19  &  6 & & w~legła gruzach & legła w~gruzach \\
    19  & &  5 & \emph{TheChronicle} & \emph{The~Chronicle} \\
    22  & & 16 & na~bizantyńską & bizantyńską \\
    32  & &  2 & \emph{wiary, islam} & \emph{wiary. Islam} \\
    68  & &  1 & Dzieje Bizancjum & \emph{Dzieje Bizancjum} \\
    121 & & 13 & przed & przed nim \\
    126 & &  2 & The~Armenian & \emph{The~Armenian} \\
    143 & &  4 & wroga$^{ \textrm{\emph{20}} }$. & wroga$^{ 20 }$. \\
    178 &  5 & & tylko & ma~tylko \\
    202 & & 15 & \emph{wiary, islam} & \emph{wiary. Islam} \\
    203 &  3 & & \emph{history} & \emph{History} \\
    \hline
  \end{tabular}
\end{center}

\vspace{\spaceTwo}





% ##################
\Work{ % Autor i tytuł książki
  Red. E. Guerriero, M. Impagliazzo \\
  ,,Najnowsza historia Kościoła. Katolicy i~kościoły chrześcijańskie
  w~czasie pontyfikatu Jana Pawła II (1978--2005)'',
  \cite{GuerrieroImpagliazzoNajnowszaHistoriaKosciola2006} }


% \CenterTB{Uwagi.}

% \start \StrWd{9}{6} Zamieszczony tu komentarz odnośnie słowa

\CenterTB{Błędy}
\begin{center}
  \begin{tabular}{|c|c|c|c|c|}
    \hline
    & \multicolumn{2}{c|}{} & & \\
    Strona & \multicolumn{2}{c|}{Wiersz} & Jest
                              & Powinno być \\ \cline{2-3}
    & Od góry & Od dołu & & \\
    \hline
    6  & 10 & & religia miały & nauka miały \\
    6  & & 10 & do & od \\
    7  & & 11 & dużo & duże \\
    14 & &  3 & zgodne & zgadzające~się \\
    30 & &  4 & Afryki & Ameryki Południowej \\
    51 & & 17 & śś. & św. \\
    63 &  9 & & 1987 & 1986 \\
    \hline
  \end{tabular}
\end{center}

\vspace{\spaceTwo}





% ##################
\Work{ % Autor i tytuł dzieła
  Ks. Bogusław Kumor \\
  ,,Historia Kościoła. Tom~I: Starożytność chrześcijańska'',
  \cite{KumorHistoriaKosciolaTomI2003} }


\CenterTB{Błędy}
\begin{center}
  \begin{tabular}{|c|c|c|c|c|}
    \hline
    & \multicolumn{2}{c|}{} & & \\
    Strona & \multicolumn{2}{c|}{Wiersz} & Jest
                              & Powinno być \\ \cline{2-3}
    & Od góry & Od dołu & & \\
    \hline
    14 & 11 & & rzeciwieństwie & przeciwieństwie \\
    14 & 15 & & jednej formy & jedną formę \\
    % & & & & \\
    % & & & & \\
    % & & & & \\
    % & & & & \\
    % & & & & \\
    % & & & & \\
    \hline
  \end{tabular}
\end{center}

\vspace{\spaceTwo}













% % ######################################
% \section{Dzieła świętych}

% \vspace{\spaceTwo}
% % ######################################








% % ######################################
% \newpage
% \section{Pozostali autorzy}

% \vspace{\spaceTwo}
% % ######################################










% #####################################################################
% #####################################################################
% Bibliografia
\bibliographystyle{plalpha} \bibliography{LibDEUSPhil}{}


% ############################

% Koniec dokumentu
\end{document}
