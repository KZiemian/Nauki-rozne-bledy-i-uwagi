% Autor: Kamil Ziemian

% ---------------------------------------------------------------------
% Podstawowe ustawienia i pakiety
% ---------------------------------------------------------------------
\RequirePackage[l2tabu, orthodox]{nag} % Wykrywa przestarzałe i niewłaściwe
% sposoby używania LaTeXa. Więcej jest w l2tabu English version.
\documentclass[a4paper,11pt]{article}
% {rozmiar papieru, rozmiar fontu}[klasa dokumentu]
\usepackage[MeX]{polski} % Polonizacja LaTeXa, bez niej będzie pracował
% w języku angielskim.
\usepackage[utf8]{inputenc} % Włączenie kodowania UTF-8, co daje dostęp
% do polskich znaków.
\usepackage{lmodern} % Wprowadza fonty Latin Modern.
\usepackage[T1]{fontenc} % Potrzebne do używania fontów Latin Modern.



% ------------------------------
% Podstawowe pakiety (niezwiązane z ustawieniami języka)
% ------------------------------
\usepackage{microtype} % Twierdzi, że poprawi rozmiar odstępów w tekście.
% \usepackage{graphicx} % Wprowadza bardzo potrzebne komendy do wstawiania
% % grafiki.
% \usepackage{verbatim} % Poprawia otoczenie VERBATIME.
% \usepackage{textcomp} % Dodaje takie symbole jak stopnie Celsiusa,
% wprowadzane bezpośrednio w tekście.
\usepackage{vmargin} % Pozwala na prostą kontrolę rozmiaru marginesów,
% za pomocą komend poniżej. Rozmiar odstępów jest mierzony w calach.
% ------------------------------
% MARGINS
% ------------------------------
\setmarginsrb
{ 0.7in} % left margin
{ 0.6in} % top margin
{ 0.7in} % right margin
{ 0.8in} % bottom margin
{  20pt} % head height
{0.25in} % head sep
{   9pt} % foot height
{ 0.3in} % foot sep



% ------------------------------
% Często używane pakiety
% ------------------------------
% \usepackage{csquotes} % Pozwala w prosty sposób wstawiać cytaty do tekstu.
\usepackage{xcolor} % Pozwala używać kolorowych czcionek (zapewne dużo
% więcej, ale ja nie potrafię nic o tym powiedzieć).





% ---------------------------------------------------------------------
% Dodatkowe ustawienia dla języka polskiego
% ---------------------------------------------------------------------
\renewcommand{\thesection}{\arabic{section}.}
% Kropki po numerach rozdziału (polski zwyczaj topograficzny)
\renewcommand{\thesubsection}{\thesection\arabic{subsection}}
% Brak kropki po numerach podrozdziału



% ------------------------------
% Pakiety których pliki *.sty mają być w tym samym katalogu co ten plik
% ------------------------------
\usepackage{latexshortcuts}



% ------------------------------
% Ustawienia różnych parametrów tekstu
% ------------------------------
\renewcommand{\arraystretch}{1.2} % Ustawienie szerokości odstępów między
% wierszami w tabelach



% ------------------------------
% Pakiet „hyperref”
% Polecano by umieszczać go na końcu preambuły
% ------------------------------
\usepackage{hyperref} % Pozwala tworzyć hiperlinki i zamienia odwołania
% do bibliografii na hiperlinki










% ---------------------------------------------------------------------
% Tytuł, autor, data
\title{DEUS~-- błędy i~uwagi, część II}

% \author{}
% \date{}
% ---------------------------------------------------------------------










% ####################################################################
% Początek dokumentu
\begin{document}
% ####################################################################





% ######################################
\maketitle  % Tytuł całego tekstu
% ######################################





% ######################################
\section{Święta wiara i~filozofia}

\vspace{\spaceTwo}
% ######################################



% % ############################
% \subsection{Dzieła powstałe w~XX i~XXI wieku}

% \vspace{\spaceThree}
% % ############################










% ############################
\subsection{Refleksje nad świętą wiarą po 1945~r.}

\vspace{\spaceThree}
% ############################



% ############################
% \newpage
\Work{ % Autor i tytuł dzieła
  Romano Amerio \\
  „Iota unum. Analiza zmian w~Kościele Katolickim”,
  \cite{AmerioIotaUnum} }


\CenterTB{Uwagi}

\start W~tej książce sposób podawania tytułów przywoływanych dzieł
jest niejednolity. Zamiennie używane~są trzy konwencje: „Tytuł”,
\emph{Tytuł}, „\emph{Tytuł}”.

\vspace{\spaceFour}


\start \StrWd{168}{10} W~tej linii odstępy między słowami~są zbyt
duże.

\vspace{\spaceFour}


\start \StrWd{180}{12} Tekst zaczynający~się po~myślniku jest odrobinę
za~mocno wcięty.

\vspace{\spaceFour}


\start \StrWd{231}{2} Nie wiem czy zaznaczanie kursywą pewnych słów
w~zaczynającym~się tu cytacie jest błędem czy nie?

\vspace{\spaceFour}


\start \StrWg{310}{5} Użyte tu obraźliwe słowo „lewactwo” tak nie
pasuje do stylu reszty książki, że~zapewne jest to efekt
niedopuszczalnej swobody tłumacza.

\vspace{\spaceFour}


\start \StrWg{331}{1} Zdanie „Idea \emph{socjalizmu
  chrześcijańskiego} ma~niewątpliwie swoją rację bytu”, brzmi jakoś
dziwnie. Może jego obecna forma jest wynikiem pomyłki edytorskiej?

\vspace{\spaceFour}


\start \Str{425--426} W dialogu \emph{Protagoras} Platon, Sokrates
twierdzi, że~na polityce każdy zna~się tak samo dobrze, choć dialog
ten kończy~się sytuacją, gdy dochodzi on do przeciwnego wniosku.
W~świetle tego należałoby głębiej zbadać dzieła Platona i~inne
traktujące o~Sokratesie, by ustalić, czy rzeczywiście uważał,
iż~o~funkcjonowanie miasta należy pytać polityka.

\vspace{\spaceFour}


\start \StrWd{439}{2} Po ostatnim słowie greckim powinien być
zamykający cudzysłów. W~tym momencie nie umiem pisać w~\LaTeX u po
alfabetem greckim w~taki sposób, by napisać to słowo dobrze. Mam
problem z czcionką i~akcentami.

\vspace{\spaceFour}


\start \StrWg{440}{1} Poza ostatnim słowem, ta linie jest powtórzenie
tekstu z~poprzedniej strony.

\vspace{\spaceFour}


\start \Str{452} Treść akapitu u~góry strony jest dziwna i~ciężka
do~zrozumienia. Należałoby dobrze przemyśleć temat, który porusza.

\vspace{\spaceFour}


\start \StrWd{452}{5} W~tej linii odstępy między wyrazami~są zbyt
duże.

\vspace{\spaceFour}


\start \StrWd{493}{7} W~tej linii odstępy między wyrazami~są zbyt
duże.

\vspace{\spaceFour}


\start \StrWd{496}{4} Ponownie, użyte tu obraźliwe słowo „lewactwo”
tak nie pasuje do stylu reszty książki, że~zapewne jest to efekt
niedopuszczalnej swobody tłumacza.

\vspace{\spaceFour}


\start \StrWd{513}{7} W~tej linii odstępy między wyrazami~są zbyt
duże.

\vspace{\spaceFour}


\start \textbf{Tylna okładka.} Informacje tu podane są w~mojej ocenie
wątpliwe albo dyskusyjne. Dobry tego przykładem jest uznanie
Alessandro Manzoniego za~największego filozofa i~poetę XVII~wieku.

\vspace{\spaceFour}





% ##################
\CenterTB{Błędy}

\begin{center}

  \begin{tabular}{|c|c|c|c|c|}
    \hline
    & \multicolumn{2}{c|}{} & & \\
    Strona & \multicolumn{2}{c|}{Wiersz} & Jest
                              & Powinno być \\ \cline{2-3}
    & Od góry & Od dołu & & \\
    \hline
    30  & &  8 & „franciszkanie” & „franciszkanów” \\
    42  & & 10 & „\emph{O~Rewolucji Francuskiej}”
           & „\emph{O~Rewolucji Francuskiej} \\
    89  & & 10 & SoboruWatykańskiego & Soboru Watykańskiego \\
    92  &  5 & & Bożych & Bożych” \\
    94  &  9 & & lat$^{ 62 }$” & lat”$^{ 62 }$ \\
    103 & &  2 & tytułrm & tytułem \\
    110 & &  4 & P. & R. \\
    112 &  7 & & Liturgii & liturgii \\
    133 & &  9 & (=\emph{ale}) & (= \emph{ale}) \\
    144 &  8 & & Chrystusa$^{ 94 }$” & Chrystusa”$^{ 94 }$\\
    161 & &  4 & [Kościoła]$^{ 114 }$?” & [Kościoła]?”$^{ 114 }$ \\
    166 & &  6 & katolicyzmu$^{ 116 }$” & katolicyzmu”$^{ 116 }$ \\
    170 & & 15 & Rady & rady \\
    172 & &  1 & Bogiem$^{ 121 }$” & Bogiem”$^{ 121 }$ \\
    175 & & 10 & Papieża$^{ 122 }$” & Papieża”$^{ 122 }$ \\
    175 & &  8 & sensu$^{ 123 }$” & sensu”$^{ 123 }$ \\
    183 & &  9 & niegodni$^{ 129 }$” & niegodni”$^{ 129 }$ \\
    232 &  1 & & suos.$^{ 156 }$” & suos.”$^{ 156 }$ \\
    233 & 12 & & regułą$^{ 157 }$” & regułą”$^{ 157 }$ \\
    239 & &  7 & \emph{flecti}$^{ 160 }$” & \emph{flecti}”$^{ 160 }$ \\
    242 &  1 & & \emph{alsit}$^{ 162 }$” & \emph{alsit}”$^{ 162 }$ \\
    242 & & 13 & \emph{sbarro}$^{ 163 }$” & \emph{sbarro}”$^{ 163 }$ \\
    254 & &  6 & \emph{domem}$^{ 165 }$ & \emph{domem}”$^{ 165 }$ \\
    255 &  2 & & \emph{rozwijac się}$^{ 166 }$”
           & \emph{rozwijac się}”$^{ 166 }$ \\
    256 &  1 & & pracy$^{ 168 }$” & pracy”$^{ 168 }$ \\
    283 & 16 & & rasy” & rasy”$^{ 186 }$ \\
    283 & 16 & & 99)$^{ 186 }$ & 99) \\
    293 & 10 & & nagrodę$^{ 191 }$” & nagrodę”$^{ 191 }$ \\
    293 & 15 & & \emph{ciała}$^{ 192 }$” & \emph{ciała}”$^{ 192 }$ \\
    293 & & 10 & duszy$^{ 193 }$” & duszy”$^{ 193 }$ \\
    297 & 10 & & skażona~$^{ 196 }$ & skażona$^{ 196 }$ \\
    301 &  5 & & zaniedbać$^{ 197 }$” & zaniedbać”$^{ 197 }$ \\
    318 & & 14 & socjalistyczną$^{ 202 }$” & socjalistyczną”$^{ 202 }$ \\
    319 & 14 & & Ducha$^{ 203 }$” & Ducha”$^{ 203 }$ \\
    339 & & 10 & absolutnych? & absolutnych?” \\
    363 & &  6 & wydaw\textbf{n}. & wydawn. \\
    381 &  2 & & Notre- Dame & Notre-Dame \\
    382 &  2 & & Świętym$^{ 236 }$” & Świętym”$^{ 236 }$ \\
    382 & & 12 & Pana$^{ 237 }$” & Pana”$^{ 237 }$ \\
    % & & & & \\
    \hline
  \end{tabular}


  \begin{tabular}{|c|c|c|c|c|}
    \hline
    & \multicolumn{2}{c|}{} & & \\
    Strona & \multicolumn{2}{c|}{Wiersz} & Jest
                              & Powinno być \\ \cline{2-3}
    & Od góry & Od dołu & & \\
    \hline
    382 &  2 & & Świętym$^{ 236 }$” & Świętym”$^{ 236 }$ \\
    382 & & 12 & Pana$^{ 237 }$” & Pana”$^{ 237 }$ \\
    383 & 12 & & płaszczyźnie$^{ 240 }$” & płaszczyźnie”$^{ 240 }$ \\
    385 & &  8 & bezpośredni$^{ 243 }$ & bezpośredni \\
    385 & &  7 & Dogmatów” & Dogmatów”$^{ 243 }$ \\
    388 &  8 & & katolika$^{ 245 }$” & katolika”$^{ 245 }$ \\
    394 &  6 & & 190$^{ 251 }$) & 190)$^{ 251 }$ \\
    394 & &  1 & stosowany\ldots & stosowany. \\
    % 395?
    406 & 12 & & przekonany”\ldots & przekonany”. \\
    408 &  3 & & wolę$^{ 261 }$” & wolę”$^{ 261 }$ \\
    413 & &  9 & prawdy$^{ 263 }$” & prawdy”$^{ 263 }$ \\
    415 & &  8 & miejscu$^{ 266 }$ & miejscu” \\
    415 & &  8 & 610).” & 610). \\
    % 423
    425 &  5 & & człowieka & człowieka” \\
    % 425?????
    442 & &  9 & kocha$^{ 280 }$” & kocha”$^{ 280 }$ \\
    % 443??????
    451 & & 14 & widzialne$^{ 283 }$” & widzialne”$^{ 283 }$ \\
    475 & 17 & & \emph{bezpodstawne}$^{ 295 }$”
           & \emph{bezpodstawne}”$^{ 295 }$ \\
           %
    495 & & 11 & człowiek & człowiek” \\
    % 495??????
    505 & &  6 & \emph{Osservatore} & \emph{L'Osservatore} \\
    517 & & 10 & szkodę\ldots & szkodę \\
    % 533??????
    % & & & & \\
    % & & & & \\
    % & & & & \\
    % & & & & \\
    % & & & & \\
    % & & & & \\
    % & & & & \\
    \hline
  \end{tabular}

\end{center}


\noindent
\StrWg{460}{14} \\
\Jest  czyniąc z~nadziei córkę wiary \\
\Powin czyniąc z~wiary córkę nadziei \\

\vspace{\spaceTwo}
% ############################










% ############################
\newpage
\Work{ % Autor i tytuł dzieła
  Dietrich von~Hildebrand \\
  „Koń trojański w~Mieście Boga”, \cite{HildebrandKonTrojanski2006}
}


\CenterTB{Uwagi}

\start \StrWd{80}{5--4} Czy chodzi tu o~to, że~po ogołoceniu z~religii
życie usycha? Czy też, że~po pozbawieniu świętości usycha religia?

\vspace{\spaceFour}


\start \Str{104} Nie wiem czy dało~się to zrobić lepiej, ale~strona
na~której znajduje~się tylko treść jednego przypisu, nie~wygląda
dobrze.

\vspace{\spaceFour}


\start \Str{140} „Jednakże prawda o~istnieniu Boga powinna to
panowanie zdobyć, a~gdy je~zdobędzie, wówczas jej realność społeczna
uzyskuje zupełnie nowych charakter.” Hildebrandowi chodziło chyba
o~to, że~jeśli prawda o~istnieniu Boga zapanuje w~sferze
międzyludzkiej, wówczas to społeczność uzyskuje zupełnie nową jakość
bycia.

\vspace{\spaceFour}


\start \Str{187} Nie potrafię zrozumieć jaki sens miał mieć fragment:
„Liczni spośród tych, co~przenoszą liturgię nad modlitwę prywatną,
ponieważ ta~ostatnia nie sprzyja rzekomo wspólnocie między ludźmi,
sprawiają wrażenie, że~stracili z~pola widzenia ten głęboko
wspólnotowy aspekt modlitwy liturgicznej”.





% ##################
\CenterTB{Błędy}

\begin{center}

  \begin{tabular}{|c|c|c|c|c|}
    \hline
    & \multicolumn{2}{c|}{} & & \\
    Strona & \multicolumn{2}{c|}{Wiersz} & Jest
                              & Powinno być \\ \cline{2-3}
    & Od góry & Od dołu & & \\
    \hline
    62  & &  4 & Kościół & i~Kościół \\
    78  &  4 & & \emph{ludzie} z~\emph{zewnątrz}
           & \emph{ludzie z~zewnątrz} \\
    85  & &  1 & Chicago1977 & Chicago 1977 \\
    106 &  4 & & Tematyczność prawdy$^{ 42 }$
           & Tematyczność$^{ 42 }$ prawdy \\
    224 &  2 & & \emph{]a} & \emph{Ja} \\
    231 & &  7 & \emph{jak} & jak \\
    245 &  8 & & \emph{wiary}~w & \emph{wiary~w} \\
    257 & 11 & & Teil\dywiz harda & Teilharda \\
    294 & & 10 & św.Piotra & św.~Piotra \\
    304 &  6 & & pastorów & proboszczów \\
    374 & &  5 & nowi \emph{moraliści} & \emph{nowi moraliści} \\
    379 &  8 & & Bóg jest & \emph{Bóg jest} \\
    \hline
  \end{tabular}

\end{center}


\noindent
\StrWg{241}{2} \\
\Jest  Hildebrand,\emph{Ethics},Franciscan \\
\Powin Hildebrand, \emph{Ethics}, Franciscan \\
\StrWd{297}{4} \\
\Jest  żywotności,niezależnej od~przypadków,jakie \\
\Powin żywotności, niezależnej od~przypadków, jakie \\
\StrWd{297}{3} \\
\Jest  świecie,przed \\
\Powin świecie, przed \\

\vspace{\spaceTwo}
% ############################










% ############################
\Work{ % Autor i tytuł dzieła
  Dietrich von~Hildebrand \\
  „Spustoszona winnica”, \cite{HildebrandSpustoszonaWinnica2006} }


\CenterTB{Uwagi}

\start \StrWd{99}{7} Zdanie „Wytrwanie w~miłości do~innego człowieka
w~stanie wiecznej niedojrzałości”, nie jest najprostsze
do~zrozumienia. Możliwe, że~powinno ono brzmieć „Wytrwanie w~miłości
do~innego człowieka, będącego w~stanie wiecznej niedojrzałości”.

\vspace{\spaceFour}


\start \Str{114} Stwierdzenie „o~ile Stary Testament uznaje~się
za~prawdziwe objawienie Boga” bardzo odstaje od~treści tej książki.
Hildebrand ciężko posądzać o~podważanie tego, iż~Stary Testament jest
objawieniem Boga, a~to zdanie sprawia wrażenie, że~jest to
dopuszczalne. Prawdopodobnie miało ono brzmieć „o~ile Stary Testament
uznają za~prawdziwe objawienie Boga”, są~bowiem żydzi, którzy
odrzucają Boga oraz~Stary Testament.

\vspace{\spaceFour}


\start \Str{152} Zdanie „Dochodzi do~tego jeszcze jeden problem:
hasło \emph{totus homo} bynajmniej nie znaczy, iż~pozycja Chrystusa
--~Jego posłannictwo, Jego święte kapłaństwo, Jego święty urząd
nauczycielki, Jego charakter jako Króla królów --~odróżnia Go w~sposób
szczególny od~wszystkich ludzi, którzy są tylko ludźmi i~wynosi
Go~ponad nich.” brzmi bardzo dziwnie w~zestawieniu z~resztą książki.
Zapewne został tu~popełniony jakiś błąd.

\vspace{\spaceFour}


\start \Str{177} Użycie słów „odnajdywanie” oraz~„znajdowanie”
w~prowadzonej analizie, jest odrobinę mylące i~należałoby je zmienić
w~taki sposób, by~stała~się ona jaśniejsza.

\vspace{\spaceFour}


\start \Str{216} Paragraf poświęcony relacji jednostki i~wspólnoty
jest napisany w~dziwny sposób, czyniący go dość niejednoznacznym.
Możliwie, że~to wynik pomyłki tłumacza.

\vspace{\spaceFour}


\start \Str{216} Możliwe, że~zamiast „w~połączeniu w~jedną jedyną
substancję” powinno być „połączonych w~jedną jedyną substancję”.
Nie potrafię jednak rozstrzygnąć, która wersja jest poprawna.

\vspace{\spaceFour}





% ##################
\CenterTB{Błędy}

\begin{center}

  \begin{tabular}{|c|c|c|c|c|}
    \hline
    & \multicolumn{2}{c|}{} & & \\
    Strona & \multicolumn{2}{c|}{Wiersz} & Jest
                              & Powinno być \\ \cline{2-3}
    & Od góry & Od dołu & & \\
    \hline
    5   &  2 & & w & \emph{w} \\
    % 71 & & 3 & 24a.25 & 24 a. 25 \\ ???
    81  & &  1 & CLXIX. & CLXIX, \\
    125 &  3 & & z & \emph{z} \\
    129 & & 11 & L'\emph{Osservatore} & \emph{L'Osservatore} \\
    132 & &  1 & 4,20) & 4,20). \\
    137 & &  6 & m.\hspace{1em} in. & m.~in.\\
    271 &  6 & & et & \emph{et} \\
    273 & &  5 & zwraca & zwracamy \\
    282 & &  5 & wydaje~się zmian & zmian wydaje~się \\
    283 & &  8 & Bądźcie & „Bądźcie \\
    % & & & & \\
    \hline
  \end{tabular}

\end{center}


\noindent
\textbf{Grzbiet.} \\
\Jest Hildebrand„Spustoszona \\
\Powin Hildebrand „Spustoszona \\
\StrWd{215}{3} \\
\Jest  Tegoż,\emph{Journuals},TorchBooks/Harper{\&}Row,NewYork(b.d.w.),s.187. \\
\Powin Tegoż, \emph{Journuals}, Torch Books/Harper \& Row, New York
(b.d.w.), s. 187. \\


\vspace{\spaceTwo}
% ############################










% ######################################
\newpage
\section{Święta wiara i~filozofia~-- dzieła podejrzane}

\vspace{\spaceTwo}
% ######################################



% ############################
\subsection{Refleksje nad świętą wiarą po 1945~r.}

\vspace{\spaceThree}
% ############################



% ############################
\Work{ % Autor i tytuł dzieła
  Jacques Maritain \\
  „Wieśniak znad Garonny. Stary świecki chrześcijanin snuje refleksje
  \emph{\`{a}~propos} czasów współczesnych”,
  \cite{MaritainWiesniakZnadGaronny2017} }


% ##################
\CenterTB{Błędy}

\begin{center}

  \begin{tabular}{|c|c|c|c|c|}
    \hline
    & \multicolumn{2}{c|}{} & & \\
    Strona & \multicolumn{2}{c|}{Wiersz} & Jest
                              & Powinno być \\ \cline{2-3}
    & Od góry & Od dołu & & \\
    \hline
    22  & &  3 & słowa. & słowa). \\
    % & & & & \\
    % & & & & \\
    % & & & & \\
    % & & & & \\
    % & & & & \\
    % & & & & \\
    % & & & & \\
    % & & & & \\
    % & & & & \\
    \hline
  \end{tabular}

\end{center}


% \noindent
% \StrWg{241}{2} \\
% \Jest  Hildebrand,\emph{Ethics},Franciscan \\
% \Powin Hildebrand, \emph{Ethics}, Franciscan \\

\vspace{\spaceTwo}
% ############################










% #####################################################################
% #####################################################################
% Bibliografia
\bibliographystyle{plalpha} \bibliography{LibDEUSPhil}{}


% ############################

% Koniec dokumentu
\end{document}
