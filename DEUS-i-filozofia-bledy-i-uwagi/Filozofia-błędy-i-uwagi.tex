% Autor: Kamil Ziemian

% --------------------------------------------------------------------
% Podstawowe ustawienia i pakiety
% --------------------------------------------------------------------
\RequirePackage[l2tabu, orthodox]{nag} % Wykrywa przestarzałe i niewłaściwe
% sposoby używania LaTeXa. Więcej jest w l2tabu English version.
\documentclass[a4paper,11pt]{article}
% {rozmiar papieru, rozmiar fontu}[klasa dokumentu]
\usepackage[MeX]{polski} % Polonizacja LaTeXa, bez niej będzie pracował
% w języku angielskim.
\usepackage[utf8]{inputenc} % Włączenie kodowania UTF-8, co daje dostęp
% do polskich znaków.
\usepackage{lmodern} % Wprowadza fonty Latin Modern.
\usepackage[T1]{fontenc} % Potrzebne do używania fontów Latin Modern.



% ----------------------------
% Podstawowe pakiety (niezwiązane z ustawieniami języka)
% ----------------------------
\usepackage{microtype} % Twierdzi, że poprawi rozmiar odstępów w tekście.
\usepackage{graphicx} % Wprowadza bardzo potrzebne komendy do wstawiania
% grafiki.
\usepackage{verbatim} % Poprawia otoczenie VERBATIME.
\usepackage{textcomp} % Dodaje takie symbole jak stopnie Celsiusa,
% wprowadzane bezpośrednio w tekście.
\usepackage{vmargin} % Pozwala na prostą kontrolę rozmiaru marginesów,
% za pomocą komend poniżej. Rozmiar odstępów jest mierzony w calach.
% ----------------------------
% MARGINS
% ----------------------------
\setmarginsrb
{ 0.7in} % left margin
{ 0.6in} % top margin
{ 0.7in} % right margin
{ 0.8in} % bottom margin
{  20pt} % head height
{0.25in} % head sep
{   9pt} % foot height
{ 0.3in} % foot sep



% ------------------------------
% Często używane pakiety
% ------------------------------
\usepackage{csquotes} % Pozwala w prosty sposób wstawiać cytaty do tekstu.
\usepackage{xcolor} % Pozwala używać kolorowych czcionek (zapewne dużo
% więcej, ale ja nie potrafię nic o tym powiedzieć).



% ------------------------------
% Pakiety do tekstów z nauk przyrodniczych
% ------------------------------
\let\lll\undefined % Amsmath gryzie się z językiem pakietami do języka
% polskiego, bo oba definiują komendę \lll. Aby rozwiązać ten problem
% oddefiniowuję tę komendę, ale może tym samym pozbywam się dużego Ł.
\usepackage[intlimits]{amsmath} % Podstawowe wsparcie od American
% Mathematical Society (w skrócie AMS)
\usepackage{amsfonts, amssymb, amscd, amsthm} % Dalsze wsparcie od AMS
% \usepackage{siunitx} % Dla prostszego pisania jednostek fizycznych
\usepackage{upgreek} % Ładniejsze greckie litery
% Przykładowa składnia: pi = \uppi
\usepackage{slashed} % Pozwala w prosty sposób pisać slash Feynmana.
\usepackage{calrsfs} % Zmienia czcionkę kaligraficzną w \mathcal
% na ładniejszą. Może w innych miejscach robi to samo, ale o tym nic
% nie wiem.





% --------------------------------------------------------------------
% Dodatkowe ustawienia dla języka polskiego
% --------------------------------------------------------------------
\renewcommand{\thesection}{\arabic{section}.}
% Kropki po numerach rozdziału (polski zwyczaj topograficzny)
\renewcommand{\thesubsection}{\thesection\arabic{subsection}}
% Brak kropki po numerach podrozdziału



% ----------------------------
% Pakiety napisane przez użytkownika.
% Mają być w tym samym katalogu to ten plik .tex
% ----------------------------
\usepackage{latexshortcuts}
\usepackage{mathshortcuts}



% ----------------------------
% Ustawienia różnych parametrów tekstu
% ----------------------------
\renewcommand{\arraystretch}{1.2} % Ustawienie szerokości odstępów między
% wierszami w tabelach.



% ----------------------------
% Pakiet "hyperref"
% Polecano by umieszczać go na końcu preambuły.
% ----------------------------
\usepackage{hyperref} % Pozwala tworzyć hiperlinki i zamienia odwołania
% do bibliografii na hiperlinki.





% --------------------------------------------------------------------
% Tytuł, autor, data
\title{Filozofia --~błędy i~uwagi}

% \author{}
% \date{}
% --------------------------------------------------------------------


% ####################################################################
% Początek dokumentu
\begin{document}
% ####################################################################



% ######################################
\maketitle  % Tytuł całego tekstu
% ######################################



% ######################################
\section{Święta wiara i~filozofia}

\vspace{\spaceTwo}
% ######################################



% ##################
\Work{ % Autor i tytuł dzieła
  Jacques Maritain \\
  ,,Trzej reformatorzy. Luter, Kartezjusz, Rousseau'',
  \cite{MaritainTrzejReformatorzy05} }


\CenterTB{Uwagi}

\start \Str{233} Nie jestem w~stanie zrozumieć co~oznacza zdanie
,,Luter do~tego stopnia zapoznaje prawdę, że~miłość jest w~nas wlanym
uczuciem w~samym życiu Boga i~Chrystusa, które otrzymuje w~zasłudze
krwi Chrystusowej.''.

% \vspace{\spaceFour}


\CenterTB{Błędy}
\begin{center}
  \begin{tabular}{|c|c|c|c|c|}
    \hline
    & \multicolumn{2}{c|}{} & & \\
    Strona & \multicolumn{2}{c|}{Wiersz} & Jest
                              & Powinno być \\ \cline{2-3}
    & Od góry & Od dołu & & \\
    \hline
    18  & 13 & & JózefTischner & Józef Tischner \\
    20  & &  9 & dobra & dobra'' \\
    29  & &  4 & deorsum'' & deorsum \\
    35  & &  4 & słabości\ld & słabości. \\
    35  & &  2 & ( dalej & (dalej \\
    39  &  8 & & stanie & stanu \\
    40  & &  2 & X. & X \\
    42  & 13 & & świętej\ld & świętej \\
    42  & &  1 & WA. & WA \\
    42  & &  1 & X. & X \\
    48  & 10 & & teologii?$^{ 27 }$ & teologii$^{ 27 }$? \\
    49  & &  3 & XL. & XL \\
    52  & &  1 & \emph{theol.} & \emph{theol.}, \\
    55  & &  1 & II. --~II & II \\
    65  & &  4 & VI & \emph{VI} \\
    66  & &  6 & XL,P & XL, P \\
    69  &  5 & & wskutek & nie wskutek \\
    70  & & 12 & --~Wystarczy & Wystarczy \\
    % & & & & \\
    % & & & & \\
    89  & &  1 & Pochwała Kartezjusza & \emph{Pochwała Kartezjusza} \\
    % & & & & \\
    % & & & & \\
    % & & & & \\
    133 & &  4 & (Math., IV,3) & Mat. IV, 3 \\
    133 & &  1 & (Teresa) & Teresa \\
    % & & & & \\
    % & & & & \\
    % & & & & \\
    224 &  1 & & sprawiedliwości.''., & sprawiedliwości.''. \\
    224 &  2 & & Paryż & (Paryż \\
    224 & 12 & & z~którego & o~którą \\
    225 &  4 & & X. & X \\
    229 & 13 & & błyskawicami.. & błyskawicami \\
    233 & 11 & & XL & XL, \\
    234 & &  3 & owszem!, & owszem!'', \\
    236 & 11 & & Dobra & dobra \\
    % & & & & \\
    % & & & & \\
    \hline
  \end{tabular}
\end{center}
\noi
\StrWd{223}{9} \\
\Jest GDYBYM NIE BYŁ DOKTOREM \\
\Pow  Gdybym nie był doktorem \\

\vspace{\spaceTwo}




\Work{
  Jacques Maritain \\
  ,,Trzej reformatorzy'' }


% Uwagi:\\
% \begin{itemize}
%
% \item
%
% \item
%
% \end{itemize}

\CenterTB{Błędy}
\begin{center}
  \begin{tabular}{|c|c|c|c|c|}
    \hline
    & \multicolumn{2}{c|}{} & & \\
    Strona & \multicolumn{2}{c|}{Wiersz} & Jest
                              & Powinno być \\ \cline{2-3}
    & Od góry & Od dołu & & \\
    \hline
    135 & 5 & & J ona & Jest ona \\
    143 & 6 & & Dostojewski Andre & Dostojewski, Andre \\
    % ,,
    150 & & 1 & Heloizy'' ,,Stąd  % ''
    % ,,
           & Heloizy''. ,,Stąd \\  % ''
    173 & & 8 & kija obręczy & kija, obręczy \\
    \hline
  \end{tabular}
\end{center}





% ######################################
\newpage
\section{Filozofia i~historia}

\vspace{\spaceTwo}
% ######################################



% #################### \Work{
% R\'{e}mi Brague \\
% ,,Prawo Boga. Filozoficzna historia przymierza'',
% \cite{BraguePrawoBoga14} }


% \CenterTB{Błędy}

% \noi
% \tb{Grzbiet} \\
% \Jest R \'{e}mi {\small Brague} \\
% \Pow R\'{e}mi Brague \\
% \tb{Tylna okładka} \\
% \Jest R\'{e}mi {\small Brague} \\
% \Pow R\'{e}mi Brague \\

% \vspace{\spaceTwo}





% ##################
\Work{
  Eric Voegelin \\
  ,,Izrael i~Objawienie'', \cite{VoegelinIzraelIObjawienie14} }


\CenterTB{Uwagi}

\noi \tb{Konkretne strony}

\vspace{\spaceFour}

\start \StrWd{28}{2} Mowa jest tu o~symbolizacja mikroantropicznej,
wydaje~się jednak, że~Voegelinowi chodziło o~symbolizację
makroantropiczną.

\vspace{\spaceFour}


\start \Str{54} W~przytoczonej tu~inskrypcji jest podane, że~,,Enlil
zwrócił oczy kraju [\emph{kalama}] na~siebie'', podczas gdy na dole
strony Voegelin pisze, że~oczy całego kraju Sumerów zostały skierowane
na~Lugalzagesi. Czy jest to błąd tłumaczenie, czy~też Voegelin
pozwolił sobie na~taką interpretację tego fragmentu? Jeśli to drugie,
to~należy zauważyć, iż~ta interpretacja jest dość odległa od~tekstu,
choć nie oznacza to, że~jest niepoprawna. Ja~w~każdym razie bym jej
w~takiej formie nie przyjął.

\vspace{\spaceFour}


\start \StrWg{57}{2} Jest tu mowa, że~przed stworzeniem ,,niebiańskiej
ziemi'' została stworzona ,,ziemska ziemia'', jednak kontekst
sugeruje, że~kolejność powinna być odwrotna.

\vspace{\spaceFour}


\start \Str{127} Na tej stronie pierwszy raz pojawia~się wspomniana
postać~N, ale~jej imię jest na~przemian pisane ,,N.'' albo~,,N'' i~nie
wiadomo która wersja jest poprawna.

\vspace{\spaceFour}


\start \Str{132} Cytowany tu tekst o~N przychodzącym z~Wyspy Sławy,
wykazuje duże rozbieżność z~brzmieniem, do~którego odwołuje~się
w~swojej analizie Voegelin. Na przykład w~tekście cytowany jest Wyspa
Sławy, a~Voegelin pisze o~Wyspie Płomieni.

\vspace{\spaceFour}


\start \Str{147} Na~tej stronie cytowany jest fragment~V omawianego
tekstu, ale nigdzie nie jest chyba podane, jak ten fragment brzmi.

\vspace{\spaceFour}


\start \Str{180} Ta~strona wprowadziła trochę zamieszania do~mojego
rozumienia tekstu. Przywoływane~są \emph{Hymny do~Amona}, choć wydaje
mi~się do~tej pory w~kontekście egipskim była mowa o~\emph{Hymnach
  do~Atona} (str.~163) i~\emph{Hymnach Echnatona} (str.~171). Poza
tym, są to zapewne dwie różne nazwy tego samego zbioru tekstów.

Do~tego, tekst zdaje~się mówić, że~nowa forma egipskiej religijność,
jakoś związana z~\emph{Hymnami do~Amona}, była monoteistyczna, podczas
gdy na~stronie~173 Voegelin stwierdza, że~nawet stworzenie Atona,
nadal mieściło~się w~obrębie mitu politeistycznego.

\vspace{\spaceFour}


\start \StrWd{197}{10} Zdanie ,,Przestaniemy ufać tablicy~III, lecz
odsuniemy ją na~bok'' brzmi nienaturalnie, to musi być jakaś pomyłka
tłumacza. Możliwe, że~miało być ,,Nie~przestaniemy ufać tablicy~III''.

\vspace{\spaceFour}


\start \StrWd{213}{3} W~wersie tym jest mowa o~autorach badań, lecz
chodzi raczej o~wynik pracy na Pismem~Św. autorów pracujących po
niewoli babilońskiej.

\vspace{\spaceFour}


\start \StrWd{229}{6} Użycie w~tekście polskim angielskiego słowa
,,patchwork'' nie pasuje stylu tłumaczenia. Lepiej byłoby znaleźć
polski odpowiednik tego zwrotu.

\vspace{\spaceFour}


\start \StrWg{241}{12} Rozdział 5~jest wyjątkowo krótki jak na tę
książkę i~nie dzieli się nad podrozdziały, więc odniesienie do
podrozdziału 5.2~jest błędne. Nie umiem jednak ustalić poprawnego
miejsca o~które chodzi Voegelinowi.

\vspace{\spaceFour}


\start \StrWg{312}{5} Po tej linie w~tekście powinie znajdować~się
odstęp.

\vspace{\spaceFour}


\start \StrWg{370}{12} Napisane tu jest, że~po teopolis doszło do
wycofania~się porządku w~formie kosmologicznej, jednak na podstawie
tego co Voegelin pisał wcześniej logiczniejsza byłoby inne
stwierdzenie. Mianowicie, że~po okresie teopolis wraz z~ustanowieniem
królestwa wkracza do~Izraela porządek w~formie kosmologicznej.


\CenterTB{Błędy}
\begin{center}
  \begin{tabular}{|c|c|c|c|c|}
    \hline
    & \multicolumn{2}{c|}{} & & \\
    Strona & \multicolumn{2}{c|}{Wiersz} & Jest
                              & Powinno być \\ \cline{2-3}
    & Od góry & Od dołu & & \\
    \hline
    52  & 13 & & 2923 & 2123 \\
    54  & &  3 & Lugalzaggesi & Lugalzagesi \\
    56  & 13 & & przepływ & na~przepływ \\
    98  & &  4 & forma'' & ,,forma'' \\
    139 & &  7 & agnostyczną metafizykę & agnostycznej metafizyce \\
    165 & & 11 & sa & są \\
    202 & 12 & & mogła & nie mogła \\
    207 &  8 & & Izraela & izraelskiego \\
    278 & &  6 & siedemnastym & dwudziestym pierwszym \\
    291 & 12 & & 13 & 14 \\
    303 & &  8 & wydarł & wydarłem \\
    314 &  6 & & 11.13 & 11, 13 \\
    316 & &  4 & miał zaś & zaś \\
    340 & &  6 & 175 & 1175 \\
    341 &  1 & & wyjaśnić,przyjmując & wyjaśnić, przyjmując \\
    357 & &  1 & 8.17 & 8, 17 \\
    365 & &  8 & nie będziecie & będziecie \\
    375 & 10 & & [zbójeckie]{ } wyprawy & [zbójeckie] wyprawy \\
    377 & & 14 & zarzadzanie & zarządzanie \\
    % & & & & \\
    % & & & & \\
    % & & & & \\
    % & & & & \\
    \hline
  \end{tabular}
\end{center}

\vspace{\spaceTwo}





% ######################################
\newpage
\section{Dzieła bogate w~treść}

\vspace{\spaceTwo}
% ######################################



% ##################
\Work{ % Autor i tytuł dzieła
  Eric Voegelin \\
  ,,Nowa nauka polityki'', \cite{VoegelinNowaNaukaPolityki92} }


\CenterTB{Uwagi}

\start Zapomniałem już na której stronie Voegelin wprowadza pojęcie
gnostyckiego snu, z~którego tłumaczeniem jest pewien problem. Po
angielsku ,,sleep'' określa stan człowieka, gdy ten śpi, lecz sny
które mogą człowieka najść określa~się słowem ,,dream''. Z~tego
względu w~języku angielskim jest jasne, czy Voegelin mówił o~,,gnostic
sleep'', czyli stanie w~którym człowiek ma radykalnie osłabiony
kontakt z~rzeczywistością, czy też o~,,gnostic dream'', czyli życiu
w~świecie dzikich fantazji. Tłumacz nie zaznaczając tej ważnej różnicy
w~tym wydaniu popełnił bardzo poważny błąd. Korzystając z~Internetu
starałem~się sprawdzić która rozumienie jest poprawne, nie mam
rozstrzygających informacji, lecz artykuł anglojęzyczny który
znalazłem, przywołując tę~myśl Voegelina, używał sformułowania
,,gnostic dream''.

\vspace{\spaceFour}


\start \Str{64} Przeczytałem tłumaczenie tu przywoływanych fragmentów
\emph{Państwa} 368 c--d i~nie znalazłem w~nich stwierdzenia,
że~\emph{polis} to wielki człowiek, była za to stwierdzenie,
że~\emph{polis} jest większe niż człowiek, więc i~sprawiedliwość jest
w~nim większa i~łatwiejsza do zauważenia. W~momencie gdy Glaukon mówi,
że~w~pierwotnym państwie jest jedzenie który i~świnie jeść by~mogły,
Sokrates powiada, że~to jest państwo jakby zdrowe, musimy więc teraz
rozważyć państwo w~którym jest bardziej dostatnie, wtedy mówi, że~jest
ono jakby w~gorączce. Są~więc obecne metafory \emph{polis} jako czegoś
żywego, ale~nie wiem czy można~się posunąć do stwierdzenia, że~jest
ono wielki człowiekiem, może jednak coś mi umyka. Opowieść o~państwie
jako o~jednym ciele jest jednak znana już w~starożytności, choćby
w~postaci bajki
\href{https://en.wikipedia.org/wiki/Agrippa_Menenius_Lanatus_(consul_503_BC)}
{Menenius Agrypy}.

\vspace{\spaceFour}


\start \StrWd{144}{} Zdanie ,,W~\emph{Prawach} Platon odsunął prawdę
duszy na~odległość jej objawienia w~\emph{Państwie}.'' jest zupełnie
niezrozumiałe, to musi byś jakiś błąd w~tłumaczeniu.



\CenterTB{Błędy}
\begin{center}
  \begin{tabular}{|c|c|c|c|c|}
    \hline
    & \multicolumn{2}{c|}{} & & \\
    Strona & \multicolumn{2}{c|}{Wiersz} & Jest
                              & Powinno być \\ \cline{2-3}
    & Od góry & Od dołu & & \\
    \hline
    29  & 13 & & warunkowa & warunkową \\
    35  & &  9 & ,, Filozof % ''
           & ,,Filozof \\ % ''
    45  & 10 & & sposob & sposób \\
    64  & 15 & & do do & do \\
    70  &  5 & & ton & Platon \\
    113 &  5 & & pierwszy & drugi \\
    116 & & 11 & w znaczenie & znaczenie \\
    132 & 15 & & wspierać & wspierać się \\
    141 & & 16 & f formami & z~formami \\
    153 & &  3 & wdzięczną & niewdzięczną \\
    166 & & 15 & ładząca & zaprowadzająca ład \\
    % & & & & \\
    \hline
  \end{tabular}
\end{center}

\vspace{\spaceTwo}





% ######################################
\newpage
\section{Analizy filozofii i~kultur}

\vspace{\spaceTwo}
% ######################################



% ##################
\Work{ % Autor i tytuł dzieła
  Reinhart Koselleck \\
  ,,Krytyka i~kryzys. Studium patogenezy świata mieszczańskiego'',
  \cite{KoselleckKrytykaIKryzys15} }

\CenterTB{Uwagi}

\start \StrWg{5}{4} W~tej linii brakuje liczby 39, oznaczającej stronę
na~której rozpoczyna~się wstęp.

\vspace{\spaceFour}


\start \StrWd{114}{15} Wydaje mi~się, że~książka Kosellecka powstała
zbyt wcześnie, aby mogła cytować wydawnictwo z~1993 roku. Możliwe
jednak, że~jest to data publikacji zbioru w~którym praca Tarna
o~Aleksandrze Macedońskim ukazała~się po~raz kolejny i~akurat ten
odnośnik do~tego artykułu został wybrany w~którymś z~kolejnych wydań.
Jednak nie jestem w~stanie rozstrzygnąć tego z~całą pewnością.

\vspace{\spaceFour}


\start \StrWg{116}{10--11} Podzielenie ,,natury?'' na~te dwie linie
jest nie tylko niepoprawne, ale~również bardzo źle wygląda.

\vspace{\spaceFour}


% \start \StrWd{101}{20} Ten fragment pozostawia niejasnym,
% czy~działanie kogoś innego jest manifestowane, jako przedłużenie
% czynności własnego ,,ja'' u~ludzi Zachodu czy~u~Indusów.

% \vspace{\spaceFour}


% \start \StrWg{116}{6} Po~tej linii nie powinno być pionowego odstępu
% w~tekście.

% \vspace{\spaceFour}


% \start \StrWg{559}{6} W~tej linii nie~powinno być wcięcia.


\CenterTB{Błędy}
\begin{center}
  \begin{tabular}{|c|c|c|c|c|}
    \hline
    & \multicolumn{2}{c|}{} & & \\
    Strona & \multicolumn{2}{c|}{Wiersz} & Jest
                              & Powinno być \\ \cline{2-3}
    & Od góry & Od dołu & & \\
    \hline
    65  & &  5 & dalekowzrocznego działania & działania dalekowzrocznego \\
    71  & &  4 & and & \emph{and} \\
    110 & & 14 & jdnym & jednym \\
    115 &  4 & & występują & występując \\
    115 &  5 & & uznając & uznają \\
    137 & 15 & & więc na & więc \\
    184 & & 11 & Neuf & Les Neuf \\
    233 & &  5 & x$\rho\iota\tau\iota\chi\acute{\eta}$
           & $\chi\rho\iota\tau\iota\chi\acute{\eta}$ \\
    260 & & 11 & z~której & w~której \\
    331 & & 12 & wladcy & władcy \\
    % & & & & \\
    % & & & & \\
    % & & & & \\
    \hline
  \end{tabular}
\end{center}

\vspace{\spaceTwo}





% ##################
\Work{ % Autor i tytuł dzieła
  Nakamura Hajime \\
  ,,Systemy myślenia ludów Wschodu. \\
  Indie, Chiny, Tybet, Japonia'',
  \cite{NakamuraSystemyMysleniaLudowWschodu05} }

\CenterTB{Uwagi}

\start \Str{23} Nakamura powinien tu jawnie napisać, co rozumiem przez
,,sposoby myślenia'' i~,,systemy myślenia''.

\vspace{\spaceFour}


\start \StrWd{90}{13} W~polskim tłumaczeniu użyto zwrotu ,,relacja
wyróżniającego~się i~wiedzącego'', który jest dziwny i~trudny
do~zrozumienia.

\vspace{\spaceFour}


\start \StrWd{101}{20} Ten fragment pozostawia niejasnym,
czy~działanie kogoś innego jest manifestowane, jako przedłużenie
czynności własnego ,,ja'' u~ludzi Zachodu czy~u~Indusów.

\vspace{\spaceFour}


\start \StrWg{116}{6} Po~tej linii nie powinno być pionowego odstępu
w~tekście.

\vspace{\spaceFour}


\start \StrWg{559}{6} W~tej linii nie~powinno być wcięcia.


\CenterTB{Błędy}
\begin{center}
  \begin{tabular}{|c|c|c|c|c|}
    \hline
    & \multicolumn{2}{c|}{} & & \\
    Strona & \multicolumn{2}{c|}{Wiersz} & Jest
                              & Powinno być \\ \cline{2-3}
    & Od góry & Od dołu & & \\
    \hline
    69  & &  2 & jedynie rzeczownikowi & rzeczownikowi \\
    106 & 14 & & można & nie można \\
    % & & & & \\
    % & & & & \\
    % & & & & \\
    % & & & & \\
    559 & 20 & & naukowego]. & naukowego].] \\
    \hline
  \end{tabular}
\end{center}

\vspace{\spaceTwo}





% ######################################
\newpage
\section{Historia filozofii}

\vspace{\spaceTwo}
% ######################################



% ##################
\Work{
  Frederick Copleston S. J. \\
  ,,Historia filozofii. Tom~I: Grecja i~Rzym'',
  \cite{CoplestonHistoriaFilozofiiTomI04} }


\CenterTB{Uwagi}

\noi \tb{Ogólne}

\vspace{\spaceFour}

\start Omówienie każde filozofa powinno zawierać listę dzieł
polecanych do przeczytania, aby~czytelnik wiedział, gdzie najlepiej
zetknąć~się z~tym wszystkim co~zawiera~się w~sposobie filozofowania
danego myśliciela. Z~rzeczami takimi jak konkretne sposoby dowodzenia,
używane argumenty, wplatane anegdoty, żarty, etc.

\vspace{\spaceThree}


\noi \tb{Konkretne strony}

\vspace{\spaceFour}

\start \Str{32} W~polskiej tłumaczeniu Diogenesa Laertios, opisuje
styl Anaksymenesa jako prosty i~niewyszukany, zaś w~wersji angielskiej
jako ,,pure unmixed'', co należałoby by przetłumaczyć raczej jako
,,czysty i~pozbawiony obcych naleciałości'', jednak to tłumaczenie
również jest niesatysfakcjonujące.

\vspace{\spaceFour}


\start \Str{33} Według tego co tu napisano, Anaksymanes uważał,
że~istnieje jedne pierścień okalający, zawierający i~ogień i~zimno,
wewnątrz niego zaś~jest powietrze. Nie potrafię sobie wyobrazić, jak
taki pierścień miałby wyglądać.

\vspace{\spaceFour}


\start \Str{33} Należałoby pójść śladem wydania angielskiego i~ostatni
akapit tej strony, który rozpoczyna podsumowanie filozofii jońskiej,
oddzielić graficznie od omówienia myśli Anaksymenesa.





\CenterTB{Błędy}
\begin{center}
  \begin{tabular}{|c|c|c|c|c|}
    \hline
    & \multicolumn{2}{c|}{} & & \\
    Strona & \multicolumn{2}{c|}{Wiersz} & Jest
                              & Powinno być \\ \cline{2-3}
    & Od góry & Od dołu & & \\
    \hline
    16  & &  3 & poglądów & spojrzenia na świat \\
    32  & 16 & & \ld Mówi & \ld mówi \\
    37  & & 18 & zgodę & niezgodę \\
    38  &  5 & & zwykłą & tylko \\
    % & & & & \\
    % & & & & \\
    61 & & 12 & do~Schelling & to~Schelling \\
    82 &  8 & & do~kraju & w~kraju \\
    % & & & & \\
    % & & & & \\
    \hline
  \end{tabular}
\end{center}
\noi
\StrWg{11}{17} \\
\Jest że~podmiotu w~nie większym stopniu nie~można sprowadzić
do~przedmiotu niż~przedmiotu do~podmiotu \\
\Pow że~podmiot nie może być w~większym stopniu sprowadzony
do~przedmiotu niż~przedmiot do~podmiotu \\
\StrWg{32}{3} \\
\Jest Każdy jest zniszczalny, jednakże, jak~się wydaje, istnieje ich
nieograniczone liczba w~tym samy czasie, światów zaczynających istnieć
dzięki wiecznemu ruchowi. \\
\Pow Każdy jest zniszczalny, jednakże, wydaje~się, że~nieskończona ich
liczba istnieje w~tej samej chwili, światów powstających dzięki
wiecznemu ruchowi. \\
\StrWg{32}{25} \\
\Jest poprzedzając koncepcję \\
\Pow dochodząc do koncepcji \\
\StrWd{61}{9} \\
\Jest Nie~mógł raczej być utożsamiany z~Jednym, nie~mogło~się też
zdarzyć,
by~ktoś to czynił zbyt dosłownie. \\
\Pow Nie może być utożsamiony z~Jednym, nie mogło~się też zdarzyć, by
ktoś robił to dosłownie. \\

\vspace{\spaceTwo}





% ######################################
\newpage
\section{Filozofia nauki i~filozofujący naukowcy}

\vspace{\spaceTwo}
% ###################################### 



% ##################
\Work{ % Autor i tytuł dzieła
  Roger Penrose \\
  ,,Moda, wiara i~fantazja w~nowej fizyce Wszechświata'',
  \cite{PenroseModaWiaraIFantazja17} }


\CenterTB{Uwagi}


\start \StrWd{72}{3--2} Ponieważ kwantowa teoria pola zajmuje~się
głównie procesami relatywistycznymi, naturalne jest przyjęcie, że~ruch
wszystkich rozważanych cząstek~są relatywistyczne. W~takim wypadku nie
powinno~się upraszczać sprawy do~tego stopnia, by~przyjmować, iż~pęd
to~prędkość cząstki razy jej masa. Rozumiem, że~zapewne Penrose chciał
uprościć tekst na~potrzebę niespecjalistów, ale~według mnie poszedł
za~daleko.

\vspace{\spaceFour}


\start \Str{78} Należy według mnie zatrzymać~się chwilę nad
stwierdzeniem, że~wartość masy ,,ubranej'' cząstki\footnote{Dla
  prostoty mówię tu o~masie nie zaś o~ładunku, ale~dla niego
  powinno~się dać przeprowadzić analogiczne rozumowanie.}
otrzymuje~się doświadczeń, nie zaś z~teorii. Jeśli weźmiemy teorię
masywnego pola swobodnego, która nie wymaga renormalizacji, to i~tak
zawiera ona parametr masy, który można wyznaczyć tylko z~obserwacji.
Problem teorii kryje~się więc gdzieś indziej.

To~na co liczylibyśmy, to~że znając masę gołej cząstki, jesteśmy
w~stanie policzyć masę cząstki ,,ubranej'', co okazuje~się nieprawdą,
musimy więc wziąć~ją z~danych obserwacyjnych. Właśnie porażka w~tym
miejscu teorii wywołuje spory niesmak. Drugim powodem do~tego niesmaku
jest to, że~należy w~ten sposób zastępować wyrażenia nieskończone,
co~jest zdecydowanie bardzo ,,uwłaczającą'' czynnością.

\vspace{\spaceFour}


\start \StrWd{659}{5} Na~podstawie bibliografii nie jestem w~stanie
zidentyfikować pozycji kryjącej~się za~akronimem DDR, która
wielokrotnie jest przywoływana w~tej książce.

\vspace{\spaceFour}


\start \Str{661} Może to być już przesadny pedantyzm, ale~nie podoba
mi~się stwierdzenie, że~aby otrzymać wykres funkcji odwrotnej
wystarczy zamienić miejscami osie. Ryzyko niezrozumienia jest według
mnie zbyt duże. To co naprawdę trzeba zrobić, to obrócić \emph{cały}
wykres w~taki sposób, by osie zamieniły~się miejscami.

\vspace{\spaceFour}


\start \Str{673} Zamiast liczb całkowitych~$\Z$ należałoby całą
dyskusję oprzeć na~liczbach naturalnych~$\N$. Uczyniłoby to~między
innymi Rys.~A\dywiz 3 bardziej zrozumiałym.

\vspace{\spaceFour}


\start \StrWd{683}{10} Nie powinno~się łamać wzorów matematycznych
na~końcu linii w~taki sposób, jak tu zostało złamane wyrażenie
$\dketo{ \bsym{u} }{ \bsym{v} }$.

\vspace{\spaceFour}


\start \Str{718} Rysunek A\dywiz 26 (b) jest narysowany błędnie.
Po~lewej stronie punktu zgięcia wstęgi M\"{o}biusa, który wypada
na~poziomie cięcia tożsamościowo równego zeru, widać dwa ciąg
przerywanych linii, symbolizujące cięcie. Po~pierwsze nie może być
dwóch takich przerywanych linii, bo rysunek zawiera tylko jedno
cięcie. Po~drugie, cięcia te opuszczają wstęgę M\"{o}biusa, co jest
niemożliwe, bowiem gładkość cięcia tego zabrania, by~powstał taki
skok. Po~trzecie, po~lewej stronie punktu zagięcia, wykres cięcia
znajduje~się na niewidocznej stronie wstęgi, przysłoniętej przez jej
zagięty fragment.

\vspace{\spaceFour}


\start \Str{725} Nie rozumiem, jaka jest logika Penrosa, by~w~tym
miejscu powoływać~się na~wzór~$e^{ i \theta }$, skoro w~dodatku~A.9
na~stronie~735 będzie tłumaczył zupełnie od~zera co~to są liczby
zespolone. Skutkiem tego, osoba, która nie wie czym są liczby
zespolone musiałaby czytać, dodatki nie po~kolei. Skoro jednak taka
osoba nie wie, czym~są liczby zespolone, zapewne widząc ten wzór nie
będzie wiedziała, gdzie szukać jego wyjaśnienia.

Wystarczyłoby dodać na~końcu tego tekstu uwagę ,,Po~informacje o~tym
wzorze zob.~A.9''\footnote{To~może nie jest najlepsze sformułowanie,
  ale~nie wiem jak to~zrobić lepiej.}, ale~niczego takiego tu nie ma.

\vspace{\spaceFour}


\start \Str{747} Pojęcie \tb{funkcji holomorficznej} zdaje~się tu
pojawiać pierwszy raz, w~skutek czego jest zupełnie niezrozumiałe
dla~osoby niezorientowanej w~analizie zespolonej.



\CenterTB{Błędy}
\begin{center}
  \begin{tabular}{|c|c|c|c|c|}
    \hline
    & \multicolumn{2}{c|}{} & & \\
    Strona & \multicolumn{2}{c|}{Wiersz} & Jest
                              & Powinno być \\ \cline{2-3}
    & Od góry & Od dołu & & \\
    \hline
    4   &  5 & & \emph{in~the~Universe} & \emph{of~the~Universe} \\
    107 & &  7 & mieszczącej~się & mieszczących~się \\
    % & & & & \\
    % & & & & \\
    684 &  1 & & $+\, a \dketo{ \bsym{u} }{ \bsym{v} }$
           & $= a \dketo{ \bsym{u} }{ \bsym{ v } }$ \\
    686 &  6 & & $\ldots,$ & $\ldots$ \\
    688 & &  8 & $i \neq j$ & $i \neq j$, \\
    688 & &  3 & $\ldots,$ & $\ldots$ \\
    689 & &  4 & $( \bsym{\rho}_{ 1 },\; \ld, \bsym{\rho}_{ n } )$
           & $( \bsym{\rho}_{ 1 }, \ld, \bsym{\rho}_{ n } )$ \\
    711 & & 10 & A\dywiz 15(c) & A\dywiz 15(b) \\
    731 &  6 & & płaszczyzny & z~płaszczyzny \\
    732 &  2 & & $\infty^{ k \infty }$ & $\infty^{ 2 k \infty }$ \\
    733 &  4 & & $\infty^{ k \infty }$ & $\infty^{ 2 k \infty }$ \\
    733 &  7 & & $s = 1$ & $s = 2$ \\
    733 &  7 & & $s = k$ & $s = 2k$ \\
    733 & 11 & & $\R^{ f + k }$ & $\R^{ f + s }$ \\
    740 &  7 & & $( z - b_{ 2 } )( z - b_{ n } )$
           & $( z - b_{ 2 } ) \cdot \ld \cdot ( z - b_{ n } )$ \\
           % & & & & \\
           % & & & & \\
           % & & & & \\
    \hline
  \end{tabular}
\end{center}
\noi
\StrWg{668}{7} \\
\Jest
\begin{equation*}
  X_{ i } = x_{ i } + A_{ i } ( i = 1, 2, \ld, n ).
\end{equation*}
\Pow
\begin{equation*}
  X_{ i } = x_{ i } + A_{ i }, \quad i = 1, 2, \ld, n.
\end{equation*}
\StrWd{703}{11} \\
\Jest
\begin{equation*}
  \sqrt{ ( X - x )^{ 2 } + ( Y - y )^{ 2 } + ( Z - z )^{ 2 } .) }
\end{equation*}
\Pow
\begin{equation*}
  \sqrt{ ( X - x )^{ 2 } + ( Y - y )^{ 2 } + ( Z - z )^{ 2 } }.)
\end{equation*}

\vspace{\spaceTwo}





% #####################################################################
% #####################################################################
% Bibliografia
\bibliographystyle{alpha} \bibliography{Bibliography}{}


% ############################

% Koniec dokumentu
\end{document}
