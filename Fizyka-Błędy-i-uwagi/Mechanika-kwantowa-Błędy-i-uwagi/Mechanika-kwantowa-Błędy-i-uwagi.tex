% ------------------------------------------------------------------------------------------------------------------
% Basic configuration and packages
% ------------------------------------------------------------------------------------------------------------------
% Package for discovering wrong and outdated usage of LaTeX.
% More information to be found in l2tabu English version.
\RequirePackage[l2tabu, orthodox]{nag}
% Class of LaTeX document: {size of paper, size of font}[document class]
\documentclass[a4paper,11pt]{article}



% ------------------------------------------------------
% Packages not tied to particular normal language
% ------------------------------------------------------
% This package should improved spaces in the text
\usepackage{microtype}
% Add few important symbols, like text Celcius degree
\usepackage{textcomp}



% ------------------------------------------------------
% Polonization of LaTeX document
% ------------------------------------------------------
% Basic polonization of the text
\usepackage[MeX]{polski}
% Switching on UTF-8 encoding
\usepackage[utf8]{inputenc}
% Adding font Latin Modern
\usepackage{lmodern}
% Package is need for fonts Latin Modern
\usepackage[T1]{fontenc}



% ------------------------------------------------------
% Setting margins
% ------------------------------------------------------
\usepackage[a4paper, total={14cm, 25cm}]{geometry}



% ------------------------------------------------------
% Setting vertical spaces in the text
% ------------------------------------------------------
% Setting space between lines
\renewcommand{\baselinestretch}{1.1}

% Setting space between lines in tables
\renewcommand{\arraystretch}{1.4}



% ------------------------------------------------------
% Packages for scientific papers
% ------------------------------------------------------
% Switching off \lll symbol, that I guess is representing letter "Ł"
% It collide with `amsmath' package's command with the same name
\let\lll\undefined
% Basic package from American Mathematical Society (AMS)
\usepackage[intlimits]{amsmath}
% Equations are numbered separately in every section
\numberwithin{equation}{section}

% Other very useful packages from AMS
\usepackage{amsfonts}
\usepackage{amssymb}
\usepackage{amscd}
\usepackage{amsthm}

% Package with better looking calligraphy fonts
\usepackage{calrsfs}

% Package with better looking greek letters
% Example of use: pi -> \uppi
\usepackage{upgreek}
% Improving look of lambda letter
\let\oldlambda\Lambda
\renewcommand{\lambda}{\uplambda}




% ------------------------------------------------------
% BibLaTeX
% ------------------------------------------------------
% Package biblatex, with biber as its backend, allow us to handle
% bibliography entries that use Unicode symbols outside ASCII
\usepackage[
language=polish,
backend=biber,
style=alphabetic,
url=false,
eprint=true,
]{biblatex}

\addbibresource{LogikaITeoriaMnogosciBibliography.bib}





% ------------------------------------------------------
% Defining new environments (?)
% ------------------------------------------------------
% Defining enviroment "Wniosek"
\newtheorem{corollary}{Wniosek}
\newtheorem{definition}{Definicja}
\newtheorem{theorem}{Twierdzenie}





% ------------------------------------------------------
% Private packages
% You need to put them in the same directory as .tex file
% ------------------------------------------------------
% Contains various command useful for working with a text
\usepackage{latexgeneralcommands}
% Contains definitions useful for working with mathematical text
\usepackage{mathcommands}





% ------------------------------------------------------
% Package "hyperref"
% They advised to put it on the end of preambule
% ------------------------------------------------------
% It allows you to use hyperlinks in the text
\usepackage{hyperref}










% ------------------------------------------------------------------------------------------------------------------
% Tytuł tekstu
\title{Mechanika kwantowa \\
  {\Large Błędy i~uwagi}}

\author{Kamil Ziemian}


% \date{}
% ------------------------------------------------------------------------------------------------------------------










% ####################################################################
\begin{document}
% ####################################################################





% ######################################
\maketitle % Tytuł całego tekstu
% ######################################





% ######################################
\section{Mechanika kwantowa}

\VerSpaceTwo
% ######################################



% ######################################
\section{Ramamurti Shankar \\
  \textit{Mechanika kwantowa},
  \parencite{ShankarMechanikaKwantowa2006}}

% ############################

\vspace{0em}


% ##################
\CenterBoldFont{Uwagi}

\vspace{0em}


\noindent
\Str{74} Na rysunku 1.8 b) aby otrzymać poprawny wykres
pochodnej funkcji Gaussa należy odbić wykres przedstawiony względem osi
$y = 0$.

\VerSpaceFour









% ##################
\newpage

\CenterBoldFont{Błędy}


\begin{center}

  \begin{tabular}{|c|c|c|c|c|}
    \hline
    Strona & \multicolumn{2}{c|}{Wiersz} & Jest
                              & Powinno być \\ \cline{2-3}
    & Od góry & Od dołu & & \\
    \hline
    % & & & & \\
    % & & & & \\
    % & & & & \\
    23 & 19 & & antyrównoległą & równoległą \\
    % & & & & \\
    224 & & 6 & $\left\{ -\frac{ 1 }{ i } [ ( y_{ 2  } - y_{ 1 } )^{ 2 }
                + ( y_{ 1 } - y_{ 0 } )^{ 2 } ] \right\}$
           & $\exp\left\{ -\frac{ 1 }{ i } [ ( y_{ 2  } - y_{ 1 } )^{ 2 }
             + ( y_{ 1 } - y_{ 0 } )^{ 2 } ] \right\}$ \\
    225 & 9 & & $N \varepsilon \to t_{ n } - t_{ 0 }$
           & $N \varepsilon = t_{ n } - t_{ 0 }$ \\
    228 & & 8 & $m$ & $\frac{ 1 }{ 2 } m$ \\
    \hline
  \end{tabular}

\end{center}

\VerSpaceTwo


\noindent
Str. 20. Zdanie na dole strony jest mętne. Popraw to.?????

Str. 26. \ldots tylko wtedy, gdy $| V \rangle = 0$\ldots

Str. 78. % \ii czy i?
$$\ldots = i \int_{ a }^{ b } \frac{ dg^{ * } }{ dx } f( x ) \dPL x \, .$$

Str. 81.
$\langle k' | X | k \rangle = \frac{ 1 }{ 2 \pi } \int_{ -\infty
  }^{ \infty } e^{ -i k' x } x e^{ i k x } \dPL x = -i \frac{ d }{ dk } \bigg( \frac{ 1 }{ 2 \pi } \int_{ -\infty }^{ \infty } e^{ i ( k - k' ) x } \bigg) = -i \delta'( k - k' ) \, .$ Wobec tego, jeśli $| g( k ) \rangle$ jest wektorem, którym w bazie $K$
 odpowiada funkcja $g( k )$, to
$X| g( k ) \rangle = \bigg| \frac{ -i d g( k ) }{ d k } \bigg\rangle \, .$
Podsumujmy: w bazie $X$ operator $X$ działa jak $x$, a operator $K$
jak $-i d / d x$ (na funkcje $f( x )$), a w bazie $K$ działa jak
$k$, a operator $X$ jak $-i d / d k$\ldots






% ############################










% ######################################
\section{L.I. Schiff \textit{Mechanika kwantowa},
  \parencite{SchiffMechanikaKwantowe1987}}

% ######################################

\vspace{0em}


% ##################
\CenterBoldFont{Uwagi}

\vspace{0em}


\noindent
Nie rozumiem, i~chyba nie~powinienem rozumieć, eksperymentów
ilustrujących zasadę nieoznaczoności i~kolaps funkcji falowej
(słynny eksperyment z~dwoma szczelinami). We wszystkich tych
zagadnieniach centralną rolę odgrywa foton, który z~natury swojej
jest cząstką relatywistyczną i~nie~można go opisać w~ramach
nierelatywistycznej mechaniki kwantowej której poświęcona jest
większa część książki. Co prawda eksperyment dyfrakcji na dwóch
szczelinach można przeprowadzić dla~cząstek nierelatywistycznych, to
przy pozostałych należy~się chwilę zastanowić. Podejrzewam jednak,
że można znaleźć ich nierelatywistyczne odpowiedniki. Jednak w~samej
nierelatywistycznej mechanice kwantowej foton nie występuje, żadna
też inna cząstka nie~jest~potrzeba, by~relacje
nieoznaczoności~wynikały~z~jej formalizmu matematycznego. Jak
zauważył to Konrad Szymański, wynika to po~prostu z~rozciągłości,
w~szczególnym przypadku, przestrzennej paczki falowej.

\VerSpaceFour



\noindent
\Str{21} Jak zauważył Paweł Duch, nierówność (3.2)
nie~może być prawdziwa, bo istnieją stany własne $J_{ z }$.

\VerSpaceFour



\noindent
\Str{59} Choć odległość między dwoma sąsiednimi
wektorami $\veckbold$ wynosi $\frac{ 2 \pi }{ L }$ i~można ją
uczynić dowolnie małą przez odpowiedni dobór $L$, to analogiczne
stwierdzenie odnośnie energii jest błędne. Przyjmując wektory
sąsiednie jako
$\veckbold_{ 1 } = \frac{ 2 \pi }{ L } [ n_{ x }, n_{ y }, n_{ z } ]$
i~$\veckbold_{ 1 } = \frac{ 2 \pi }{ L } [ n_{ x } + 1, n_{ y }, n_{ z } ]$
różnica ich energii kinetycznej wynosi:
\begin{equation}
  \label{eq:Schiff-01}
  \frac{ \hbar^{ 2 } \veckbold_{ 2 }^{ 2 } }{ 2 m }
  - \frac{ \hbar^{ 2 } \veckbold_{ 1 }^{ 2 } }{ 2 m }
  = \frac{ \hbar^{ 2 } }{ 2 m } \frac{ ( 2 n_{ x } + 1 ) }{ L^{ 2 } }.
\end{equation}
Przy ustalonym $L$ ta wielkość jest dowolnie duża dla odpowiednio
wysokiego $n_{ x }$. Odległości między sąsiednimi stanami można
uważać, za małe tylko jeśli mamy górne ograniczenie na energię,
mówiąc inaczej jeśli w~układzie mamy energię Fermiego.





% ##################
\noindent

\CenterBoldFont{Błędy}


\begin{center}

  \begin{tabular}{|c|c|c|c|c|}
    \hline
    Strona & \multicolumn{2}{c|}{Wiersz} & Jest
                              & Powinno być \\ \cline{2-3}
    & Od góry & Od dołu & & \\
    \hline
    % & & & & \\
    17 & 12 & & 1904 & 1905 \\
    31 & & 4 & jakakolwiek & taka \\
    35 & 9 & & $| \psi( \vecrbold, t |^{ 2 }$ & $| \psi( \vecrbold, t ) |^{ 2 }$ \\
    42 & 2 & & ograniczone & zlokalizowane \\
    54 & 16 & & wartości & dyskretne wartości \\
    54 & & 2 & dwom & dwóm \\
    62 & & 8 & $z + z'$ & $z - z'$ \\
    % & & & & \\
    \hline
  \end{tabular}

\end{center}

\VerSpaceFour


\noindent
\StrWierszGora{21}{16} \\
\Jest orbity**.Równanie(3.3)implikuje\ldots \\
\PowinnoByc orbity**. Równanie (3.3) implikuje\ldots \\
\StrWierszGora{54}{18} \\
\Jest w~odpowiadających im punktach\ldots  \\
\PowinnoByc w~obszarze przez te ścianki zajętym\ldots \\



% ############################










% ######################################
\newpage

\section{Marian Grabowski, Roman S.~Ingarden \\
  \textit{Mechanika kwantowa. Ujęcie w~przestrzeni
    Hilberta},
  \parencite{GrabowskiIngardenMechanikaKwantowa1987}}

% ######################################

\vspace{0em}


% ##################
\CenterBoldFont{Uwagi do konkretnych stron}

\vspace{0em}


\noindent
W~książce powinna być jawnie zamieszczona informacja, że każda skończenie
wymiarowa podprzestrzeń przestrzeni Hilberta (ogólniej: przestrzeni
unormowanej), jest domknięta. Wynika to, choćby z tego, że każda
podprzestrzeń skończenie wymiarowa jest lokalnie zwarta.

\VerSpaceFour





\noindent
Str. Jest tu przykład rozumowania z ogromną dziurą. Nie możemy
korzystać z własności przestrzeni Hilberta dopóki nie udowodnimy, że
jest to przestrzeń Hilberta.

\VerSpaceFour





\noindent
\Str{27} Jest tu pewne zamieszanie odnośnie jednoznaczności
rozkładu. Rozkład na element najbliższy w danej podprzestrzeni i
część ortogonalną musi być jednoznaczny, jeśli istnieje, ze względu
na jednoznaczność rzutu. Nie mniej, nie rozstrzyga to problemu, czy
istnieje alternatywny rzut na te podprzestrzenie. Negatywną odpowiedź
daje nam fakt iż:
$\Mcal \cup \Mcal^{ \bot } = \{ \emptyset \}$.

\VerSpaceFour





\noindent
\Str{27} Przedstawione tu pojęcie zupełności jest trochę mylące. Podana tu
definicja zupełności odpowiada pojęciu \textit{totalności} omówionej
w~książce Waltera Thirring \textit{Fizyka matematyczna. Tom~III}. Przede
wszystkim z podanej definicji zbioru zupełnego nie wynika, że każdy wektor
z~$\Hcal$ można przedstawić jako szereg elementów tego zbioru. Stąd właśnie
Thirring rozróżnia pojęcie totalności i zupełności.

\VerSpaceFour





\noindent
\Str{27} W dowodach Wniosków I oraz II, jest dwa razy użyte
twierdzenie, że wektor ortogonalny do danego zbioru jest też
ortogonalny do jego domknięcia, w dowodzie pierwszego wniosku
wyrażone słownie, w drugim za pomocą wzorów. Warto byłoby zrobić to
bardzie elegancko.

\VerSpaceFour




\noindent
\Str{28} W dowodzie wniosku I.2. jest coś dziwnego. Uwaga, że
należy przyjrzeć się uzyskanym sumom prostym i wywnioskować z nich,
iż $[ \Mcal ]=( \Mcal^{ \bot } )^{ \bot }$, równie
dobrze prowadzi od razu do wniosku
$[ \mathcal{ M } ]^{ \bot } = \mathcal{ M }^{ \bot }$. Zachodzi
bowiem twierdzenie: jeżeli
$A_{ 1 } \oplus A_{ 2 } = B_{ 1 } \oplus B_{ 2 } = X$ i
$B_{ i } \subset A_{ i }$ to $B_{ i } = A_{ i }$. Załóżmy,że tak nie
jest. Wtedy istnieje
$( x_{ 1 }, x_{ 2 } ) \in A_{ 1 } \oplus A_{ 2 }$, taka że
$( x_{ 1 }, x_{ 2 } ) \notin B_{ 1 } \oplus B_{ 2 }$. Teraz istnieje
taki $( y_{ 1 }, y_{ 2 } ) \in B_{ 1 } \oplus B_{ 2 }$, że
$x_{ 1 } + x_{ 2 } = y_{ 1 } + y_{ 2 }$, czyli
$A_{ 1 } \oplus A_{ 2 }$ nie jest sumą prostą.

\VerSpaceFour





\noindent
\Str{38} Pojawia się tu pojęcie operatora ograniczonego, które
jest wprowadzone dopiero na stronie 39.

\VerSpaceFour





\noindent
\Str{40} W dowodzie lematu II.1, gdy mowa jest o udowodnieniu
pierwszej równości w punkcie a), w istocie udowodniono równość:
\begin{equation}
  \label{eq:GrabowskiIngarden-01}
  \Vert A \Vert = \sup_{ \Vert \varphi \Vert = 1 } \Vert A \varphi \Vert \, .
\end{equation}

\VerSpaceFour





\noindent
\Str{46} Punkty twierdzenia są ustawione w dziwnej kolejności,
biorąc pod uwagę logikę dowodu.

\VerSpaceFour





\noindent
\Str{48} Z tego, że dana liczba nie należy do widma, nie wynika
że nie istnieje dla niej rezolwenta.

\VerSpaceFour





\noindent
\Str{49} Drugie stwierdzenie z punktu (\romannumeral4), jest już
  zawarte w punkcie (\romannumeral2).

\VerSpaceFour





\noindent
\Str{52} Użyte tu pojęcie funkcji charakterystycznej, nie jest
chyba nigdzie w książce przedstawione.

\VerSpaceFour





\noindent
\Str{57} Warto byłoby omówić szerszej pojęcie domkniętego rozszerzenia
operatora, domykalności operatora i jego domknięcia. W~szczególności
z~twierdzenia o wykresie domkniętym wynika, że~domknięcie operatora jest
zawsze operatorem ograniczonym.

\VerSpaceFour





\noindent
\Str{58} Uwaga o twierdzeniu II (\romannumeral3) jest zupełnie
niezrozumiała.

\VerSpaceFour





\noindent
\Str{58} Zdefiniowaniu rezolwenty i~widma operatora nieograniczonego powinno
zostać poświęcone więcej miejsca.

\VerSpaceFour





\noindent
\Str{58} Nie dodano, że widmo nieograniczonego operatora
domykalnego, definiujemy jako widmo jego domknięcia.

\VerSpaceFour





\noindent
\Str{58} Jedyność wektora $\eta$ wynika już z~lematu Riesza.

\VerSpaceFour





\noindent
\Str{80} Ustalenie takiej wartości stałej $C$, ani w~ogóle ustalenie jej
wartości, nie jest potrzebne w rozważanym zagadnieniu.

\VerSpaceFour





\noindent
\Str{313} Nie wspomniano tu w jakim sensie dane ciągi funkcji mają być
zbieżne. Osoba znająca teorię całki Lebesgue’a wie, że~wystarczy założyć
zbieżność punktową, a~nawet tylko zbieżność punktową prawie wszędzie.

\VerSpaceFour





\noindent
\Str{315} W twierdzeniu Lebesgue’a o~zbieżności majoryzowanej
brakuje założenia o~zbieżności rozważanego ciągu funkcji.





% ##################
\newpage

\CenterBoldFont{Błędy}


\begin{center}

  \begin{tabular}{|c|c|c|c|c|}
    \hline
    Strona & \multicolumn{2}{c|}{Wiersz} & Jest
                              & Powinno być \\ \cline{2-3}
    & Od góry & Od dołu & & \\
    \hline
    % & & & & \\
    21  &  1 & & $\{ x ,\! \absOne{ \psi( x ) - \varphi( x ) } > 0 \}$
           & $\{ x ;\, | \psi( x ) - \varphi( x )| > 0 \}$ \\
    27  & 12 & & $\Hcal_{ 1 } \otimes \Hcal_{ 2 }$ & $\Hcal_{ 1 } \oplus \Hcal_{ 2 }$ \\
    27  & 13 & & $\Rcal \otimes \Rcal^{ \bot }$ & $\Rcal \oplus \Rcal^{ \bot }$ \\
    30  & & 18 & $\xi - \varphi_{ k } \Vert$ & $\Vert \xi - \varphi_{ k } \Vert$ \\
    32  & & 11 & Teraz$f( a ) = g( 0 )$,$f( b )$
           & Teraz $f( a ) = g( 0 )$, $f( b )$ \\
    34  & & 15 & & $[ a, b ]$ \\
    101 & &  4 & $Px )$ & $P( x )$ \\
    147 &  2 & & bogaci & ubogaci \\
    311 & 10 & & $i$.Jeżeli & $i$. Jeżeli \\
    311 & 15 & & $X$ spełniającą & $X$, spełniającą \\
    312 & & 10 & $\mu$ skończona & $\mu$-skończona \\
    313 &  6 & & $\{ x ;\! f( x ) > a \}$ & $\{ x ;\, f( x ) > a \}$ \\
    313 & 15 & & $0 = a_{ 0 }$ & $0 \leq a_{ 0 }$ \\
    313 & 15 & & $x,$ & $x;$ \\
    313 & 16 & & $A_{ 0 } \ldots \cup A_{ n }$
           & $A_{ 0 } \cup \ldots \cup A_{ n }$ \\
    315 & 3 & & $A^{ 0 } \leq f_{ 1 }( x )$ & $A: 0 \leq f_{ 1 }( x )$ \\
    318 & & 7 & Caucy’ego & Cauchy’ego \\
    % & & & & \\
    \hline
  \end{tabular}

\end{center}

\VerSpaceFour


\noindent
\Str{27} Niech $\varphi \bot [ \Pcal ]$. Wówczas $\varphi \bot \Pcal$ i $\varphi = 0$. \\
\Str{37} \ldots$\psi =
\frac{ \overline{ l ( \varphi_{ 0 } ) } }{ \Vert \varphi_{ 0 } \Vert^{ 2 } } \varphi_{ 0 } \, .$
\Str{40} \ldots punkcie $D( A )$\ldots \\
\Str{40} \ldots określone na $D( A )$\ldots \\
\Str{41} \ldots dużych $n$ mamy $\Vert A_{ n } - A_{ m } \Vert < 2 \epsilon$. \\
\Str{47} \ldots wynika, że $S_{ \lambda } = ( \lambda I - A )^{ -1 } \in B( \Hcal )$. \\
\Str{????} \ldots być równo zbiorowi $\{ 0 \}$ \ldots
\Str{54} \ldots może być równa zbiorowi $\{ 0 \}$\ldots \\
\Str{71} \ldots$A^{ * } A = S U^{ * } U S = S E S$\ldots \\
\Str{74} \ldots interpretacją. Teraz\ldots \\
\Str{85} $\ldots = -i \frac{ d }{ dx } \frac{ 1 }{ \sqrt{ 2 \pi } }
\int\limits_{ -\infty }^{ +\infty } ( e^{ i x s } - 1 ) \varphi( s ) ds \textrm{.}$ \\
\Str{86} $F^{ -1 } Q \varphi = -i \frac{ d }{ dx } \frac{ 1 }{ \sqrt{ 2 \pi } }
\int\limits_{ -\infty }^{ +\infty } ( e^{ i x s } - 1 ) \varphi( s ) d s
= \frac{ 1 }{ \hbar } P F^{ -1 } \varphi$ \\
\Str{311} \ldots dla każdego $i$. Jeżeli\ldots



% ############################










% ######################################
\newpage

\section{Informatyka kwantowa}

\VerSpaceTwo
% ######################################










% ####################################################################
% ####################################################################
% Bibliography

\printbibliography





% ############################

% Koniec dokumentu
\end{document}
