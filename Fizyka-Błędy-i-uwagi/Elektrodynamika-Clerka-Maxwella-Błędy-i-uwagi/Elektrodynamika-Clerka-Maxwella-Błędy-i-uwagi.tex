% ------------------------------------------------------------------------------------------------------------------
% Basic configuration and packages
% ------------------------------------------------------------------------------------------------------------------
% Package for discovering wrong and outdated usage of LaTeX.
% More information to be found in l2tabu English version.
\RequirePackage[l2tabu, orthodox]{nag}
% Class of LaTeX document: {size of paper, size of font}[document class]
\documentclass[a4paper,11pt]{article}



% ------------------------------------------------------
% Packages not tied to particular normal language
% ------------------------------------------------------
% This package should improved spaces in the text.
\usepackage{microtype}
% Add few important symbols, like text Celcius degree
\usepackage{textcomp}



% ------------------------------------------------------
% Polonization of LaTeX document
% ------------------------------------------------------
% Basic polonization of the text
\usepackage[MeX]{polski}
% Switching on UTF-8 encoding
\usepackage[utf8]{inputenc}
% Adding font Latin Modern
\usepackage{lmodern}
% Package is need for fonts Latin Modern
\usepackage[T1]{fontenc}



% ------------------------------------------------------
% Setting margins
% ------------------------------------------------------
\usepackage[a4paper, total={14cm, 25cm}]{geometry}



% ------------------------------------------------------
% Setting vertical spaces in the text
% ------------------------------------------------------
% Setting space between lines
\renewcommand{\baselinestretch}{1.1}

% Setting space between lines in tables
\renewcommand{\arraystretch}{1.4}



% ------------------------------------------------------
% Packages for scientific papers
% ------------------------------------------------------
% Switching off \lll symbol, that I guess is representing letter "Ł"
% It collide with `amsmath' package's command with the same name
\let\lll\undefined
% Basic package from American Mathematical Society (AMS)
\usepackage[intlimits]{amsmath}
% Equations are numbered separately in every section.
\numberwithin{equation}{section}

% Other very useful packages from AMS
\usepackage{amsfonts}
\usepackage{amssymb}
\usepackage{amscd}
\usepackage{amsthm}

% Package with better looking calligraphy fonts
\usepackage{calrsfs}

% Package for writting physical units
\usepackage{siunitx}

% Package with better looking greek letters
% Example of use: pi -> \uppi
\usepackage{upgreek}
% Improving look of lambda letter
\let\oldlambda\Lambda
\renewcommand{\lambda}{\uplambda}




% ------------------------------------------------------
% BibLaTeX
% ------------------------------------------------------
% Package biblatex, with biber as its backend, allow us to handle
% bibliography entries that use Unicode symbols outside ASCII.
\usepackage[
language=polish,
backend=biber,
style=alphabetic,
url=false,
eprint=true,
]{biblatex}

\addbibresource{LogikaITeoriaMnogosciBibliography.bib}





% ------------------------------------------------------
% Defining new environments (?)
% ------------------------------------------------------
% Defining enviroment "Wniosek"
\newtheorem{corollary}{Wniosek}
\newtheorem{definition}{Definicja}
\newtheorem{theorem}{Twierdzenie}





% ------------------------------------------------------
% Local packages
% You need to put them in the same directory as .tex file
% ------------------------------------------------------
% Package containing various command useful for working with a text
\usepackage{./Local-packages/general-commands}
% Package containing commands and other code useful for working with
% mathematical text
\usepackage{./Local-packages/math-commands}





% ------------------------------------------------------
% Package "hyperref"
% They advised to put it on the end of preambule
% ------------------------------------------------------
% It allows you to use hyperlinks in the text
\usepackage{hyperref}










% ------------------------------------------------------------------------------------------------------------------
% Title and author of the text
\title{Elektrodynamika Clerka Maxwella \\
  {\Large Błędy i~uwagi}}

\author{Kamil Ziemian}


% \date{}
% ------------------------------------------------------------------------------------------------------------------










% ####################################################################
% Beginning of the document
\begin{document}
% ####################################################################





% ######################################
\maketitle
% ######################################





% % ######################################
% \section{Standardowe wykłady elektrodynamiki Clerka Maxwella}
% % Tytuł danego działu

% \VerSpaceTwo
% % \vspace{\spaceThree}

% % ######################################



% ######################################
\section{J. D. Jackson \textit{Elektrodynamik klasyczna},
  \parencite{JacksonElektrodynamikaKlasyczna1987}}

% ######################################

\vspace{0em}


% ##################
\CenterBoldFont{Uwagi}

\vspace{0em}


\noindent
Dyskusja elektrostatyki powinna się zacząć od dyskusji problemu układu
odniesienia.

\VerSpaceFour





\noindent
Aby zapewnić fizyczną konsystencje teorii na początku rozdziału I
powinno zostać przyjęte, że w rozważanych przypadkach nie ma obecnych
pól magnetycznych. Nie jest to minimalny warunek konsystencji teorii,
ale najbardziej naturalny.

\VerSpaceFour




% ##################
\CenterBoldFont{Uwagi do~konkretnych stron}


\noindent
\Str{47} Powinna tu być zamieszczona dyskusja problemu określenia wartości
pola elektrycznego, w~punkcie w~którym znajduje się ładunek punktowy.





% ##################
\newpage

\CenterBoldFont{Błędy}


\begin{center}

  \begin{tabular}{|c|c|c|c|c|}
    \hline
    Strona & \multicolumn{2}{c|}{Wiersz} & Jest
                              & Powinno być \\ \cline{2-3}
    & Od góry & Od dołu & & \\
    \hline
    58 & 5 & & $\rho( \bold{ x }' ) \nabla^{ 2 } \bigg(
               \frac{ 1 }{ \sqrt{ r^{ 2 } + a^{ 2 } } } \bigg) \, d^{ 3 } x'$
    & $\int \rho( \bold{ x }' ) \nabla^{ 2 } \bigg(
      \frac{ 1 }{ \sqrt{ r^{ 2 } + a^{ 2 } } } \bigg) \, d^{ 3 } x'$ \\
      % & & & & \\
      % & & & & \\
      % & & & & \\
    \hline
  \end{tabular}

\end{center}

\VerSpaceTwo


\noindent
\StrWierszGora{25}{7} \\
\Jest w~równaniach Maxwella niesymetrycznie jedynie w~pierwszych dwóch
równaniach. \\
\PowinnoByc nie występują symetrycznie w~równaniach Maxwella, są obecne jedynie
w~dwóch pierwszych równaniach. \\

Str. 51. po wewnętrznej stronie, Pristley w analogii

Str. 65. $\displaystyle w = \frac{ q^{ 2 }_{ 1 } }{ 8 \pi | \vecxBold - \vecxBold_{ 1 } |^{ 4 } } + \frac{ q^{ 2 }_{ 1 } }{ 8 \pi | \vecxBold - \vecxBold_{ 2 } |^{ 4 } } + \frac{ q_{ 1 } q_{ 2 } \, ( \vecxBold - \vecxBold_{ 1 } ) \cdot ( \vecxBold - \vecxBold_{ 2 } ) }{ 4 \pi | \vecxBold - \vecxBold_{ 1 } |^{ 3 } | \vecxBold - \vecxBold_{ 2 } |^{ 3 } }$.

Str. 114. $\displaystyle Y_{ l m }( \theta, \varphi ) = \sqrt{ \frac{ ( 2l + 1 ) ( l - m )! }{ 4 \pi ( l + m )! } } P^{ m }_{ l }( \cos \theta ) e^{ i m \varphi }$


% ############################










% % ######################################
% \newpage

% \section{Newtonowsko-einsteinowska fizyka cząstek naładowanych}


% % ######################################



% ######################################
\section{Fritz Rohrlich \textit{Klasyczna teoria cząstek
    naładowanych},
  \parencite{RohrlichKlasycznaTeoriaCzastekNaladowanych1981}}

% ######################################

\vspace{0em}


% ##################
\CenterBoldFont{Uwagi}

\vspace{0em}





% ##################
\CenterBoldFont{Uwagi do~konkretnych stron}





% ##################
\CenterBoldFont{Błędy}


\begin{center}

  \begin{tabular}{|c|c|c|c|c|}
    \hline
    Strona & \multicolumn{2}{c|}{Wiersz} & Jest & Powinno być \\ \cline{2-3}
    & Od góry & Od dołu &  &  \\ \hline
    % & & & & \\
    24 & 7 & & $k \vecrBold$ & $-k \vecrBold$ \\
    25 & & 14 & $\vecEbold \times \frac{ \vecvBold }{ c } \times \vecBbold$
           & $\vecEbold + \frac{ \vecvBold }{ c } \times \vecBbold$ \\
    25 & 4 & & \textit{Electrodynamics},John
           & \textit{Electrodynamics}, John \\
    27 & 10 & & $v / c^{ 2 }$ & $v^{ 2 } / c^{ 2 }$ \\
    % & & & & \\
    % & & & & \\
    \hline
  \end{tabular}

\end{center}

\VerSpaceTwo


% ############################










% ####################################################################
% ####################################################################
% Bibliography

\printbibliography





% ############################
% End of the document

\end{document}
