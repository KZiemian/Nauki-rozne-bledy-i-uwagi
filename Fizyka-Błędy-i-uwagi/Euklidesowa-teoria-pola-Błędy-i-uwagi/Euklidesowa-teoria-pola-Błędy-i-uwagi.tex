% ------------------------------------------------------------------------------------------------------------------
% Basic configuration and packages
% ------------------------------------------------------------------------------------------------------------------
% Package for discovering wrong and outdated usage of LaTeX.
% More information to be found in l2tabu English version.
\RequirePackage[l2tabu, orthodox]{nag}
% Class of LaTeX document: {size of paper, size of font}[document class]
\documentclass[a4paper,11pt]{article}



% ------------------------------------------------------
% Packages not tied to particular normal language
% ------------------------------------------------------
% This package should improved spaces in the text
\usepackage{microtype}
% Add few important symbols, like text Celcius degree
\usepackage{textcomp}



% ------------------------------------------------------
% Polonization of LaTeX document
% ------------------------------------------------------
% Basic polonization of the text
\usepackage[MeX]{polski}
% Switching on UTF-8 encoding
\usepackage[utf8]{inputenc}
% Adding font Latin Modern
\usepackage{lmodern}
% Package is need for fonts Latin Modern
\usepackage[T1]{fontenc}



% ------------------------------------------------------
% Setting margins
% ------------------------------------------------------
\usepackage[a4paper, total={14cm, 25cm}]{geometry}



% ------------------------------------------------------
% Setting vertical spaces in the text
% ------------------------------------------------------
% Setting space between lines
\renewcommand{\baselinestretch}{1.1}

% Setting space between lines in tables
\renewcommand{\arraystretch}{1.4}



% ------------------------------------------------------
% Packages for scientific papers
% ------------------------------------------------------
% Switching off \lll symbol, that I guess is representing letter "Ł"
% It collide with `amsmath' package's command with the same name
\let\lll\undefined
% Basic package from American Mathematical Society (AMS)
\usepackage[intlimits]{amsmath}
% Equations are numbered separately in every section
\numberwithin{equation}{section}

% Other very useful packages from AMS
\usepackage{amsfonts}
\usepackage{amssymb}
\usepackage{amscd}
\usepackage{amsthm}

% Package for tensor notation
\usepackage{tensor}

% Package for writting physical units
\usepackage{siunitx}

% Package with better looking calligraphy fonts
\usepackage{calrsfs}

% Package with better looking greek letters
% Example of use: pi -> \uppi
\usepackage{upgreek}
% Improving look of lambda letter
\let\oldlambda\Lambda
\renewcommand{\lambda}{\uplambda}




% ------------------------------------------------------
% BibLaTeX
% ------------------------------------------------------
% Package biblatex, with biber as its backend, allow us to handle
% bibliography entries that use Unicode symbols outside ASCII
\usepackage[
language=polish,
backend=biber,
style=alphabetic,
url=false,
eprint=true,
]{biblatex}

\addbibresource{LogikaITeoriaMnogosciBibliography.bib}





% ------------------------------------------------------
% Defining new environments (?)
% ------------------------------------------------------
% Defining enviroment "Wniosek"
\newtheorem{corollary}{Wniosek}
\newtheorem{definition}{Definicja}
\newtheorem{theorem}{Twierdzenie}





% ------------------------------------------------------
% Private packages
% You need to put them in the same directory as .tex file
% ------------------------------------------------------
% Contains various command useful for working with a text
\usepackage{latexgeneralcommands}
% Contains definitions useful for working with mathematical text
\usepackage{mathcommands}





% ------------------------------------------------------
% Package "hyperref"
% They advised to put it on the end of preambule
% ------------------------------------------------------
% It allows you to use hyperlinks in the text
\usepackage{hyperref}










% ------------------------------------------------------------------------------------------------------------------
% Title and author of the text
\title{QFT \\
  {\Large Błędy i~uwagi}}

\author{Kamil Ziemian}


% \date{}
% ------------------------------------------------------------------------------------------------------------------










% ####################################################################
% Beginning of the document
\begin{document}
% ####################################################################





% ######################################
\maketitle
% ######################################





% ######################################
\section{N.N. Bogoliubov, D.V. Shirkov \\
  \textit{Introduction to the theory of quantized fields},
  \cite{BogoliubovShirkovIntroductionToTheoryOfQuantiezedFields1959}}

% ######################################

\vspace{0em}


% ##################
\CenterBoldFont{Uwagi}

\vspace{0em}


\noindent
W~książce użyta jest staromodna notacja dla czterowektorów. Ze względu
na wygodę, w~tych notatkach będę używał bardziej współczesnej wersji.





% ##################
\CenterBoldFont{Uwagi do~konkretnych stron}


\noindent
\Str{38} Wyjaśnienie w~tekście czemu składowa $U_{ 0 }$ generuje
ujemną energię nie jest zbyt jasne. W~istocie sprawa jest prosta:
wyrażenie na energie zawiera wyrażenia typu
$-U^{ ( - ) }_{ \mu } U^{ ( + ) \, \mu }$ przy czym każdy wszystkie
iloczyny $U^{ ( - ) }_{ \nu } U^{ ( + )\, \nu }$ (brak sumowania!) są
dodatnie. Stąd wyraz czasowy wchodzi do energii jako ujemny człon
$-U^{ ( - ) }_{ 0 } U^{ ( + )\, 0 }$ podczas gdy człony przestrzenne
dają dodatni wkład. Należy więc wyeliminować człon czasowy.

\VerSpaceFour





\noindent
\Str{45} Warto byłoby jawnie zaznaczyć, że~wektory $\vecebold_{ 1 }$
i~$\vecebold_{ 2 }$ zależą od $k$.

\VerSpaceFour





\noindent
\Str{72}

\VerSpaceFour




\noindent
\Str{106} Warto byłoby dodać komentarz wyjaśniający dlaczego wielkości
termodynamiczne (określane tu jako skalarne) definiuje~się zawsze
w~układzie \textsc{cw} danego fragmentu płynu.

\VerSpaceFour





\noindent
\Str{108}





% ##################
\newpage

\CenterBoldFont{Błędy}


\begin{center}

  \begin{tabular}{|c|c|c|c|c|}
    \hline
    Strona & \multicolumn{2}{c|}{Wiersz} & Jest
                              & Powinno być \\ \cline{2-3}
    & Od góry & Od dołu & & \\
    \hline
    6 & 8 & & begun & began \\
    14 & & 3 & $U$ & $u_{ i }$ \\
    19 & 1 & & $u'^{ k }( x' )$ & $u'_{ k }( x' )$ \\
    21 & 7 & & $\delta u_{ i }$ & $\overline{ \delta } u_{ i }$ \\
    21 & 7 & & $\delta$ & $\overline{ \delta }$ \\
    24 & & 8 & $\sum\limits_{ j,\, k < l }$
           & $\sum\limits_{ j } \sum\limits_{ k < l }$ \\
    24 & & 8 & $( u_{ j }( x )$ & $u_{ j }( x )$ \\
    25 & 2 & & $\sum\limits_{ i,\, i }$ & $\sum\limits_{ i,\, j }$ \\
    26 & & 4 & $-i \alpha u_{ j }$ & $-i \alpha u_{ j }^{ * }$ \\
    27 & 4 & & $\partial u_{ i } / \partial x^{ k }$
           & $\partial u_{ j } / \partial x^{ k }$ \\
    29 & 9 & & (2.19) & (2.20) \\
    30 & & 12 & $\tilde{ \varphi }( x )$ & $\tilde{ \varphi }( k )$ \\
    31 & & 6 & $k_{ 4 } x_{ 4 }$ & $-k_{ 4 } x_{ 4 }$ \\
    33 & 3 & & (3.19) & (3.14) \\
    33 & 3 & & (3.4) & (3.5) \\
    35 & 1 & & $n_{ 3 }^{ 3 }$ & $n_{ 3 }^{ 2 }$ \\
    43 & 13 & & (43) & (4.3) \\
    44 & & 10 & $U^{ *( \mp ) }$ & $U_{ n }^{ *( \mp ) }$ \\
    47 & & 16 & (4.24) & (4.26) \\
    49 & & 14 & $\Gamma^{ m }$ by & by \\
    70 & 1 & & $x^{ 0 }$ & $-x^{ 0 }$ \\
    72 & 8 & & while & but \\
    72 & 8 & & in addition & do not \\
    73 & & 8 & $\theta( k^{ 0 } )$ & $\theta( -k^{ 0 } )$ \\
    74 & & 18 & $\psi^{ ( + ) }( \veckbold )$
           & $\psi^{ ( + ) }( -\veckbold )$ \\
    74 & & 17 & $\psi^{ ( - ) }( -\veckbold )$
           & $\psi^{ ( - ) }( \veckbold )$ \\
           % & & & & \\
           % & & & & \\
           % & & & & \\
           % & & & & \\
           % & & & & \\
    \hline
  \end{tabular}

\end{center}

\VerSpaceTwo

% ############################










% ######################################
\newpage

\section{Henryk Arodź, Leszek Hadasz \textit{Lectures
    on~Classical and~Quantum Theory~of Fields},
  \cite{ArodzHadaszFieldTheory2010} }

% ############################

\vspace{0em}


% ##################
\CenterBoldFont{Uwagi}

\vspace{0em}


\noindent
\Str{4} Związek między omawianymi na tej stronie równaniami
(1.4) i~(1.5) można wyjaśnić w~następujący sposób. Sama postać
(1.4)~sugeruje, że~należy rozpatrzyć następujące równanie
\begin{equation}
  \label{eq:Arodz-Hadasz-LecturesETC-01}
  m R^{ 2 } \frac{ d^{ 2 } \varphi( x_{ i }, t ) }{ dt^{ 2 } } =
  -m g R \sin \varphi( x_{ i }, t )
  + \kappa a \frac{ \partial^{ 2 } \varphi( x, t ) }{ \partial x^{ 2 } }\bigg|_{ x = x_{ i } }.
\end{equation}
Jednak to równanie obowiązuje tylko dla punktów $x_{ i }$ w~których
znajdują wahadła, za to dla dowolnego $t \in \Rbb$. Poprzednio już
założyliśmy istnienie funkcji $\varphi( x , t )$ interpolującej między
tymi punktami. Teraz przyjmujemy, że~również dla każdego
$x \in ( x_{ i }, x_{ i + 1 } )$, $i = -M, \ldots, N - 1$ spełniony jest
związek wyrażony równaniem \eqref{eq:Arodz-Hadasz-LecturesETC-01}, czyli
\begin{equation}
  \label{eq:Arodz-Hadasz-LecturesETC-02}
  m R^{ 2 } \frac{ \partial^{ 2 } \varphi( x_{ i }, t ) }{ \partial^{ 2 } t } =
  -m g R \sin \varphi( x_{ i }, t )
  + \kappa a \frac{ \partial^{ 2 } \varphi( x, t ) }{ \partial x^{ 2 } }.
\end{equation}
A~to jest właśnie równanie (1.5).






% ##################
\newpage

\CenterBoldFont{Błędy}


\begin{center}

  \begin{tabular}{|c|c|c|c|c|}
    \hline
    Strona & \multicolumn{2}{c|}{Wiersz} & Jest
                              & Powinno być \\ \cline{2-3}
    & Od góry & Od dołu & & \\
    \hline
    7   & & 11 & $0, \ldots,\!3$ & $0, \ldots, 3$ \\
    11  &  5 & & $\tensor[]{ \check{ \Lcal } }{ _{ \mu }^{ \nu } }$
           & $\tensor[]{ \check{ \Lcal } }{ ^{ \mu }_{ \nu } }$ \\
    11  &  6 & & $( \Lcal^{ T } )^{ -1 }$ & $\Lcal^{ -1 }$ \\
    % & & & & \\
    % & & & & \\
    % & & & & \\
    % & & & & \\
    \hline
  \end{tabular}

\end{center}

\VerSpaceTwo


% ############################










% ######################################
\newpage

\section{ % Autor i tytuł dzieła
  Michael E. Peskin, Daniel V. Schroeder \\
  \textit{An Introduction to Quantum Field Theory},
  \cite{PeskinSchroederIntroductionToQuantumFieldTheory1995}}

% ######################################

\vspace{0em}


% ##################
\CenterBoldFont{Uwagi do~konkretnych stron}

\vspace{0em}


\noindent
\Str{24} W rzeczywistości to operator $\phi^{ \dagger }( x )$ kreuje cząstkę
w~stanie $| 0 \rangle$, zaś operator $\phi( x )$ niszczy cząstkę w zadanym stanie.
POPRAW.



% ##################
\CenterBoldFont{Błędy}


\begin{center}

  \begin{tabular}{|c|c|c|c|c|}
    \hline
    Strona & \multicolumn{2}{c|}{Wiersz} & Jest
                              & Powinno być \\ \cline{2-3}
    & Od góry & Od dołu & & \\
    \hline
    % & & & & \\
    26  & &  5 & positive-energy & negative-energy \\
    % & & & & \\
    \hline
  \end{tabular}

\end{center}

\VerSpaceTwo

% ############################










% ######################################
\newpage

\section{ % Autor i tytuł dzieła
  Silvan S. Schweber \\
  \textit{An~Introduction to~Relativistic Quantum Field Theory},
  \cite{SchewberAnIntroductionToRelativisticQuantumFieldTheory2005}}

% ######################################

\vspace{0em}


% ##################
\CenterBoldFont{Uwagi do~konkretnych stron}

\vspace{0em}


\noindent
\Str{4} Dyskusja roli pomiaru w~mechanice kwantowej jest
niesatysfakcjonująca, jednak nigdzie jeszcze nie znalazłem dobrej
dyskusji tego problemu, zaś~Schwebera należy pochwalić za zwięzłość
i~klarowność wywodu oraz~jasne stwierdzenie, że~pojęcie pomiaru jest
fundamentalne zarówno dla sformułowania jak i~interpretacji mechaniki
kwantowej.

Jednak problem tego czym jest pomiar nie jest w~ogóle postawiony.
Dlaczego niektóre oddziaływania z~układem wywołują kolaps wektora
stanu, inne nie? Dlaczego w~konsekwencji tego niektóre oddziaływania~są
pomiarem, a~inne nie? Dlaczego, w~końcu, układ zaburzony przez pomiar
który dał wartość obserwabli $a'$, skolapsuje do~wektora
$| a' \rangle \langle a' | \Psi \rangle$, a~nie do jakiegoś innego? Na to pytanie mechanika
kwantowa nie udziela odpowiedzi i~dopóki rozwiązanie tego problemu nie
zostanie znalezione, mechanika kwantowa nie będzie satysfakcjonującą teorią.

Drugim, mniej znaczącym problemem, jest to czy prawie wszystkie
pomiary można na poziomie podstawowym opisać jako zderzenie cząstek?
Których w~takim razie nie można do takiego procesu sprowadzić? Czy na
przykład kiedy czytam ten tekst, to pomiarem jaki wykonałem było
rozproszenie fotonu na cząstkach w~mojej siatkówce? Moja skromna
wiedza z~biologii sugeruje, że~jednak coś jeszcze.

\VerSpaceFour





\noindent
\Str{5} Muszę~się stanowczo nie~zgodzić z~twierdzeniem, że~znaczenie
procesów rozpraszania dla fizyki teoretycznej polega na tym, że~nie jest
konieczne rozumienie sensu fizycznego (lub jak kto woli, interpretacja)
funkcji falowej, gdy cząstki są~blisko siebie i~silnie oddziałują. Głównym
celem nauki jest zrozumienie przyrody, więc poznanie fizycznego znaczenia
funkcji falowej w~takiej procesie jest jednym z~głównych problemów
kwantowej teorii pola.

Stwierdzenie, że~nie musimy tego robić dla procesów rozpraszania
i~dlatego są one ważne, jest tylko dowodem na to, iż~jako fizycy
ponieśliśmy porażkę i~uznaliśmy ją za sukces.

\VerSpaceFour





\noindent
\Str{9}

\VerSpaceFour





\noindent
\Str{10} Ciągłe używane notacji Diraca, które zaczyna~się na~tej stronie
bardzie zaciemnia mi, niż rozjaśnia, zrozumienie działania operatorów
położenia i~pędu w~konkretnej reprezentacji.

\VerSpaceFour





\noindent
\Str{13}

\VerSpaceFour





% ##################
\newpage

\CenterBoldFont{Błędy}


\begin{center}

  \begin{tabular}{|c|c|c|c|c|}
    \hline
    Strona & \multicolumn{2}{c|}{Wiersz} & Jest
                              & Powinno być \\ \cline{2-3}
    & Od góry & Od dołu & & \\
    \hline
    7 & 13 & & $| \; t_{ 0 } )$ & $| t_{ 0 } \rangle$ \\
    7 & & 8 & $U^{ -1 }( t_{ 1 }, t_{ 0 } )$
           & $U^{ -1 }( t_{ 0 }, t_{ 1 } )$ \\
    7 & & 9 & (8\textbf{)} & (8) \\
    8 & & 4 & $V( t ) H_{ S } V( t ) \cdot V^{ -1 }( t )$
           & $V( t ) H_{ S }$ \\
    10 & & 8 & $\delta^{ ( 3 ) }( \vecqbold' - \vecqbold'' )$
           & $\delta^{ ( 3 ) }( \vecqbold'' - \vecqbold' )$ \\
    73 & 10 & & $\epsilon_{ i j }{}^{ 4 } \Gamma_{ k }{}' F \Gamma_{ k }$
           & $\epsilon_{ i j }{}^{ 4 } \Gamma_{ i }{}' F \Gamma_{ i }$ \\
    % & & & & \\
    % & & & & \\
    % & & & & \\
    % & & & & \\
    \hline
  \end{tabular}

\end{center}

\VerSpaceTwo




\noindent
Str. 22. \ldots ($i = 1, 2,\cdots n$)\ldots



% ############################










% ######################################
\newpage

\section{Trylogia Weinberga}

\VerSpaceTwo
% ######################################



% ######################################
\section{Steven Weinberg \\
  \textit{Teoria pól kwantowych} \\
  \textit{Podstawy},
  \cite{WeinbergTeoriaPolKwantowychPodstawy2012}}

% ######################################

\vspace{0em}


% ##################
\CenterBoldFont{Uwagi do~konkretnych stron}

\vspace{0em}


\noindent
\Str{79} Z faktu, że istnieje macierz odwrotna do
$\eta_{ \mu \nu } \Lcal^{ \mu }_{ \rho }$????!!!!!





% ##################
\CenterBoldFont{Błędy}


\begin{center}

  \begin{tabular}{|c|c|c|c|c|}
    \hline
    Strona & \multicolumn{2}{c|}{Wiersz} & Jest
                              & Powinno być \\ \cline{2-3}
    & Od góry & Od dołu & & \\
    \hline
    31 & 4 & & $-c^{ 2 } \hbar^{ 2 }$ & $+c^{ 2 } \hbar^{ 2 }$ \\
    52 & 14 & & $b^{ \dagger }( \veckbold ) \exp( i \omega_{ \veckbold } t )$
           & $b^{ \dagger }( \veckbold ) \exp( +i \omega_{ \veckbold } t )$ \\
    59 & & 4 & Heinsenberg [72[ & Heinsenberg [72] \\ % Zła pisownia.
    60 & 5 & & przyspieszone & przyśpieszone \\
    70 & 2 & & 1961. & 1961). \\
    79 & 3 & & $\eta_{ \mu \nu } dx^{ ' \mu } dx^{ ' \nu }$
           & $\eta_{ \mu \nu } dx'^{ \mu } dx'^{ \nu }$ \\
    79 & 5 & & $\eta_{ \mu \nu } \frac{ \partial x^{ ' \mu } }{ \partial x^{ \rho } }
               \frac{ \partial x^{ ' \nu } }{ \partial x^{ \sigma } }$
           & $\eta_{ \mu \nu } \frac{ \partial x'^{ \mu } }{ \partial x^{ \rho } }
             \frac{ \partial x'^{ \nu } }{ \partial x^{ \sigma } }$ \\
    86 & & 2 & $\Lcal_{ \rho }^{ -1 \mu } P^{ \rho }$
           & $\tensor[]{ ( \Lcal^{ -1 } ) }{ ^{ \mu }_{ \rho } } P^{ \rho }$ \\
    142 & & 14 & powloce & powłoce \\
    \hline
  \end{tabular}





  % \begin{tabular}{|c|c|c|c|c|}
  %   \hline
  %   Strona & \multicolumn{2}{c|}{Wiersz} & Jest
  %   & Powinno być \\ \cline{2-3}
  %   & Od góry & Od dołu & & \\
  %   \hline
  %     %   & & & & \\
  %     %   & & & & \\
  %     %   & & & & \\
  %     %   & & & & \\
  %     %   & & & & \\
  %     %   & & & & \\
  %   \hline
  % \end{tabular}

\end{center}

\VerSpaceTwo


\noindent
\Str{85} \ldots \\
\Str{86} Równania (2.5.1) i\ldots \\
\Str{86} równanie na dole \\
\StrWierszGora{281}{1} \\
\Jest $S_{ \vecpbold_{ 1 }', \sigma_{ 1 }', n_{ 1 }'; \, \vecpbold_{ 2 }',
  \sigma_{ 2 }', n_{ 2 }';\; \cdots, \; \vecpbold_{ 1 }, \sigma_{ 1 },
  n_{ 1 } ; \, \ldots { p }_{ 2 }, \sigma_{ 2 }, n_{ 2 }; \, \cdots }$ \\[0.5em]
\PowinnoByc $S_{ \vecpbold_{ 1 }', \sigma_{ 1 }', n_{ 1 }'; \, \vecpbold_{ 2 }',
  \sigma_{ 2 }', n_{ 2 }';\; \cdots; \; \vecpbold_{ 1 }, \sigma_{ 1 },
  n_{ 1 } ; \, \vecpbold_{ 2 }, \sigma_{ 2 }, n_{ 2 }; \, \cdots }$ \\





% ############################










% ############################
\newpage

\section{ % Autor i tytuł dzieła
  Steven Weinberg \\
  \textit{Teoria pól kwantowych} \\
  \textit{Tom II: Nowoczesne zastosowanie},
  \cite{WeinbergTeoriaPolKwantowychNowoczesneZastosowania1999}}


% ##################
\CenterBoldFont{Uwagi do~konkretnych stron}





% ##################
\CenterBoldFont{Błędy}

\begin{center}

  \begin{tabular}{|c|c|c|c|c|}
    \hline
    Strona & \multicolumn{2}{c|}{Wiersz} & Jest
                              & Powinno być \\ \cline{2-3}
    & Od góry & Od dołu & & \\
    \hline
    & & & & \\
    % & & & & \\
    % & & & & \\
    % & & & & \\
    % & & & & \\
    % & & & & \\
    % & & & & \\
    % & & & & \\
    % & & & & \\
    % & & & & \\
    % & & & & \\
    % & & & & \\
    % & & & & \\
    % & & & & \\
    % & & & & \\
    % & & & & \\
    % & & & & \\
    % & & & & \\
    % & & & & \\
    % & & & & \\
    % & & & & \\
    % & & & & \\
    % & & & & \\
    % & & & & \\
    % & & & & \\
    % & & & & \\
    % & & & & \\
    % & & & & \\
    % & & & & \\
    % & & & & \\
    % & & & & \\
    % & & & & \\
    % & & & & \\
    % & & & & \\
    % & & & & \\
    % & & & & \\
    % & & & & \\
    % & & & & \\
    \hline
  \end{tabular}





  % \begin{tabular}{|c|c|c|c|c|}
  %   \hline
  %   Strona & \multicolumn{2}{c|}{Wiersz} & Jest
  %   & Powinno być \\ \cline{2-3}
  %   & Od góry & Od dołu & & \\
  %   \hline
  %   %   & & & & \\
  %   %   & & & & \\
  %   %   & & & & \\
  %   %   & & & & \\
  %   %   & & & & \\
  %   %   & & & & \\
  %   %   & & & & \\
  %   %   & & & & \\
  %   %   & & & & \\
  %   %   & & & & \\
  %   %   & & & & \\
  %   %   & & & & \\
  %   %   & & & & \\
  %   %   & & & & \\
  %   %   & & & & \\
  %   %   & & & & \\
  %   %   & & & & \\
  %   %   & & & & \\
  %   %   & & & & \\
  %   %   & & & & \\
  %   %   & & & & \\
  %   %   & & & & \\
  %   %   & & & & \\
  %   %   & & & & \\
  %   %   & & & & \\
  %   %   & & & & \\
  %   %   & & & & \\
  %   %   & & & & \\
  %   %   & & & & \\
  %   %   & & & & \\
  %   %   & & & & \\
  %   %   & & & & \\
  %   %   & & & & \\
  %   %   & & & & \\
  %   %   & & & & \\
  %   %   & & & & \\
  %   %   & & & & \\
  %   %   & & & & \\
  %   \hline
  % \end{tabular}

\end{center}

\VerSpaceTwo


\noindent



% ############################










% ######################################
\newpage

\section{Matematyczna kwantowa teoria pola}

\VerSpaceTwo
% ######################################



% ######################################
\section{  % Autor i tytuł dzieła
  Othmar Steinmann \\
  \textit{Perturbative Quantum Eletrodynamics} \\
  \textit{and~Axiomatic Field Theory},
  \cite{SteinmannPerturbativeQEDAndAxiomaticFieldTheory2000}}

% ######################################

\vspace{0em}


% ##################
\CenterBoldFont{Uwagi do~konkretnych stron}

\vspace{0em}




% ##################
\CenterBoldFont{Błędy}


\begin{center}

  \begin{tabular}{|c|c|c|c|c|}
    \hline
    Strona & \multicolumn{2}{c|}{Wiersz} & Jest
                              & Powinno być \\ \cline{2-3}
    & Od góry & Od dołu & & \\
    \hline
    7   & & 11 & $0,\ldots,\!3$ & $0, \ldots, 3$ \\
    11  &  5 & & $\tensor{ \check{ \Lcal } }{ _{ \mu }^{ \nu } }$
           & $\tensor{ \check{ \Lcal } }{ ^{ \mu }_{ \nu } }$ \\
    11  &  6 & & $( \Lcal^{ T } )^{ -1 }$ & $\Lcal^{ -1 }$ \\
    % & & & & \\
    % & & & & \\
    % & & & & \\
    % & & & & \\
    % & & & & \\
    \hline
  \end{tabular}

\end{center}

\VerSpaceTwo



% ############################










% ######################################
\newpage

\section{R.F. Streater, A.S. Wightman \\
  \textit{PCT, spin and statistics, and all that},
  \cite{StreaterWightmanPCT2000}}

% ######################################

\vspace{0em}


% ##################
\CenterBoldFont{Uwagi do~konkretnych stron}

\vspace{0em}


\noindent
\Str{5}

\VerSpaceFour





\noindent
\Str{10} Nie omówiono zagadnienia
$\sgn\tensor{ ( \Lcal_{ 1 } \Lcal_{ 2 } ) }{^{ 0 }_{ 0 }}$ dla iloczynu dwóch
transformacji Lorentza.

\VerSpaceFour





\noindent
\Str{18} W Definicji (2.1,1) powinno być założenie o istnieniu
wektora zerowego.





% ##################
\CenterBoldFont{Błędy}


\begin{center}

  \begin{tabular}{|c|c|c|c|c|}
    \hline
    Strona & \multicolumn{2}{c|}{Wiersz} & Jest
                              & Powinno być \\ \cline{2-3}
    & Od góry & Od dołu & & \\
    \hline
    12 & &  3 & $( \tau^{ 2 } )( \tau^{ \mu } )^{ T }( \tau^{ 2 } )^{ -1 }$
           & $\tau^{ 2 } ( \tau^{ \mu } )^{ T } \tau^{ 2 }$ \\
    12 & & 1 & $\tau^{ 2 } ( x )^{ T }( \tau^{ 2 } )^{ -1 }$
           & $\tau^{ 2 } ( x )^{ T } \tau^{ 2 }$ \\ %Popraw
    13 & 2 & & $\tau^{ 2 } A^{ T }( \tau^{ 2 } )^{ -1 }$
           & $\tau^{ 2 } A^{ T } \tau^{ 2 }$ \\
    % & & & & \\
    % & & & & \\
    % & & & & \\
    % & & & & \\
    \hline
  \end{tabular}

\end{center}

\VerSpaceTwo


% ############################










% ####################################################################
% ####################################################################
% Bibliography

\printbibliography





% ############################

% Koniec dokumentu
\end{document}
