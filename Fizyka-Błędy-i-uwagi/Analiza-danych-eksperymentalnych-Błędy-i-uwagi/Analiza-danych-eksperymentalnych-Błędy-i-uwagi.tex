% ------------------------------------------------------------------------------------------------------------------
% Basic configuration and packages
% ------------------------------------------------------------------------------------------------------------------
% Package for discovering wrong and outdated usage of LaTeX.
% More information to be found in l2tabu English version.
\RequirePackage[l2tabu, orthodox]{nag}
% Class of LaTeX document: {size of paper, size of font}[document class]
\documentclass[a4paper,11pt]{article}



% ------------------------------------------------------
% Packages not tied to particular normal language
% ------------------------------------------------------
% This package should improved spaces in the text.
\usepackage{microtype}
% Add few important symbols, like text Celcius degree
\usepackage{textcomp}



% ------------------------------------------------------
% Polonization of LaTeX document
% ------------------------------------------------------
% Basic polonization of the text
\usepackage[MeX]{polski}
% Switching on UTF-8 encoding
\usepackage[utf8]{inputenc}
% Adding font Latin Modern
\usepackage{lmodern}
% Package is need for fonts Latin Modern
\usepackage[T1]{fontenc}



% ------------------------------------------------------
% Setting margins
% ------------------------------------------------------
\usepackage[a4paper, total={14cm, 25cm}]{geometry}



% ------------------------------------------------------
% Setting vertical spaces in the text
% ------------------------------------------------------
% Setting space between lines
\renewcommand{\baselinestretch}{1.1}

% Setting space between lines in tables
\renewcommand{\arraystretch}{1.4}



% ------------------------------------------------------
% Packages for scientific papers
% ------------------------------------------------------
% Switching off \lll symbol, that I guess is representing letter "Ł"
% It collide with `amsmath' package's command with the same name
\let\lll\undefined
% Basic package from American Mathematical Society (AMS)
\usepackage[intlimits]{amsmath}
% Equations are numbered separately in every section.
\numberwithin{equation}{section}

% Other very useful packages from AMS
\usepackage{amsfonts}
\usepackage{amssymb}
\usepackage{amscd}
\usepackage{amsthm}

% Package for writting physical units
\usepackage{siunitx}

% Package with better looking calligraphy fonts
\usepackage{calrsfs}

% Package with better looking greek letters
% Example of use: pi -> \uppi
\usepackage{upgreek}
% Improving look of lambda letter
\let\oldlambda\Lambda
\renewcommand{\lambda}{\uplambda}




% ------------------------------------------------------
% BibLaTeX
% ------------------------------------------------------
% Package biblatex, with biber as its backend, allow us to handle
% bibliography entries that use Unicode symbols outside ASCII.
\usepackage[
language=polish,
backend=biber,
style=alphabetic,
url=false,
eprint=true,
]{biblatex}

\addbibresource{LogikaITeoriaMnogosciBibliography.bib}





% ------------------------------------------------------
% Defining new environments (?)
% ------------------------------------------------------
% Defining enviroment "Wniosek"
\newtheorem{corollary}{Wniosek}
\newtheorem{definition}{Definicja}
\newtheorem{theorem}{Twierdzenie}





% ------------------------------------------------------
% Private packages
% You need to put them in the same directory as .tex file
% ------------------------------------------------------
% Contains various command useful for working with a text
\usepackage{latexgeneralcommands}
% Contains definitions useful for working with mathematical text
\usepackage{mathcommands}





% ------------------------------------------------------
% Package "hyperref"
% They advised to put it on the end of preambule
% ------------------------------------------------------
% It allows you to use hyperlinks in the text
\usepackage{hyperref}










% ------------------------------------------------------------------------------------------------------------------
% Title and author of the text
\title{Analizy danych eksperymentalnych \\
  {\Large Błędy i~uwagi}}

\author{Kamil Ziemian}


% \date{}
% ------------------------------------------------------------------------------------------------------------------










% ####################################################################
% Beginning of the document
\begin{document}
% ####################################################################





% ######################################
\maketitle
% ######################################





% ######################################
\section{Red. Henryk Szydłowski \textit{Teoria pomiarów},
  \cite{RedSzydlowskiTeoriaPomiarow1981}}

% ######################################

\vspace{0em}


% ##################
\CenterBoldFont{Uwagi}

\vspace{0em}


\noindent
W~przypadku, gdy zajdzie potrzeba przeliczenia cali na~jednostki systemu
\textsc{si}, to będziemy przyjmować, że~$1 \, \si{cal} = 2.54 \, \si{cm}$.
Wszystkie poniższe uwagi do~fragmentów książki w~których pojawiają~się cale,
bazują na tym przeliczniku jednostek.

\VerSpaceFour





% ##################
\CenterBoldFont{Uwagi do~konkretnych stron}

\vspace{0em}


\noindent
\Str{12} Według informacji zawartych w~tej książce, jeśli dokonujemy pomiaru
na~skali przedziałowej, to jedyne operacje jakie są dopuszczone na wynikach
pomiarów to ich dodawanie i~odejmowanie. Jednak jako przykład skali
przedziałowej podana jest skala temperatur, co sugeruje, że~kilka innych
operacji też powinno być dopuszczonych. Przeanalizujemy teraz~tę kwestię.

Zacznijmy od oczywistego stwierdzenia, że~aby operować wielkościami ze skali
przedziałowej, takimi jak temperatura, pomiar nie zawsze jest konieczny.
Całkowicie normalną rzeczą w~fizyce jest rozważanie obiektów o~danej~$T$,
które nie istnieją materialnie, ale~są obiektem naszych rozmyślań. Dotyczy
to nie tylko wielkości które należą do~pewnej skali przedziałowej,
ale~wszystkich innych wielkości, które możemy przypisać obiektom materialnym.
Jest też jasne, że~w~końcu chcemy odnieść nasze rozważania do materialnego
świata i~wtedy pomiar jest niezbędny.

Przejdźmy teraz do operacji, które możemy wykonywać na wielkościach ze skali
przedziałowej, za~wzorcowy ich przykład biorąc temperaturę. Wiemy już,
że~możemy je dodawać i~odejmować, książka zabrania nam jednak wykonywania na
nich operacji mnożenia i~dzielenia. Wydaje~się jednak, że~choć nie ma sensu
fizycznego mnożenie temperatur, to ma sens dzielenie dwóch różnicy
temperatur. Przyjmijmy,
że~$T_{ 1 } = 10 {}^{ \circ }\si{C}$, $T_{ 2 } = 20 {}^{ \circ }\si{C}$,
$T_{ 3 } = 30 {}^{ \circ }\si{C}$, $T_{ 4 } = 50 {}^{ \circ }\si{C}$, wówczas nie widać
przeciwwskazań by obliczyć iloraz
\begin{equation}
  \label{eq:RedSzydlowski-Teoria-pomiarow-01}
  \frac{ T_{ 4 } - T_{ 3 } }{ T_{ 2 } - T_{ 1 } } = 2.
\end{equation}
Zależność ta wyraża prosty fakt, że~różnica temperatur $T_{ 4 } - T_{ 3 }$
jest dwa razy większa, niż różnica $T_{ 2 } - T_{ 1 }$. To, że początek skali
Celsjusza został wybrany w~sposób arbitralny, nie wpływa na ten wynik, bo
różnica temperatur nie zależy od wyboru miejsca na skali któremu
przypisaliśmy wartość~$0$.

Wzór \eqref{eq:RedSzydlowski-Teoria-pomiarow-01} można również przytoczyć na
poparcie tezy, że~różnicę temperatur można pomnożyć przez liczbę
bezwymiarową. Jeżeli mamy liczbę bezwymiarową, powiedzmy $0.5$, to
jak~się wydaje, stwierdzenie, iż~zachodzi
\begin{equation}
  \label{eq:RedSzydlowski-Teoria-pomiarow-02}
  0.5 \cdot ( T_{ 4 } - T_{ 3 } ) = T_{ 2 } - T_{ 1 },
\end{equation}
jest sensowne. Wyraża ona fakt, że~różnica temperatur między $T_{ 2 }$
i~$T_{ 1 }$ jest równa połowie różnicy temperatur między $T_{ 4 }$ i~$T_{ 3 }$.
Ponieważ zaś dzielenie przez liczbę bezwymiarową to mnożenie przez liczbę
do~niej odwrotną, więc jeśli mnożenie różnicy temperatur przez taką liczbę
jest sensowne, to również dzielenie tej różnicy przez taką liczbę posiada
sens.

Należy~się zastanowić, czy przeprowadzone wyżej rozumowanie jest poprawne,
już teraz jednak należy wspomnieć, że~gdyby tak było, to pozwoliłoby wyjaśnić
następujący fakt. Załóżmy, że mamy dwa litry wody, jeden o~temperaturze
$T_{ 1 }$, drugi o~temperaturze $T_{ 2 }$, przy czym przyjmujemy,
że~$T_{ 1 } < T_{ 2 }$. Jeśli zmieszamy je w~taki sposób, żeby nie dostarczyć
ani nie odprowadzić do układu żadnej energii, to jak wiemy z~termodynamiki
(zacytować jakieś źródło????) temperatura $T_{ \textrm{res} }$ takiej
mieszaniny będzie wynosić
\begin{equation}
  \label{eq:RedSzydlowski-Teoria-pomiarow-03}
  T_{ \textrm{res} } = \frac{ T_{ 1 } + T_{ 2 } }{ 2 }.
\end{equation}
Jak można uzasadnić to, że~w~tym wzorze dzielimy \textit{sumę} temperatur
przez liczbę bezwymiarową? Można to uzasadnić tym, że~temperatura tej
mieszaniny jest tak naprawdę dana przez wzór
\begin{equation}
  \label{eq:RedSzydlowski-Teoria-pomiarow-04}
  T_{ \textrm{res} } = T_{ 1 } + 0.5 \cdot ( T_{ 2 } - T_{ 1 } ).
\end{equation}

Można argumentować, że~ten wzór ma więcej sensu fizycznego, niż wzór
\eqref{eq:RedSzydlowski-Teoria-pomiarow-03}, jego sens można bowiem wyjaśnić
w~następujący sposób. Temperatura mieszaniny $T_{ \textrm{res} }$ jest większa
od~mniejszej z~temperatur $T_{ 1 }$, $T_{ 2 }$ o~połowę różnicy między nimi.
Choć wartość numeryczna $T_{ \textrm{res} }$ zależy od wyboru miejsca na skali
temperatur któremu przypisujemy wartość zerową, to we wzorze
\eqref{eq:RedSzydlowski-Teoria-pomiarow-04} jedynie wartość $T_{ 1 }$ zależy
od tego wyboru. Taki więc sposób wyznaczania temperatury mieszaniny traktuje
osobno wielkości arbitralne, zależne od~wyboru punktu zerowego na skali,
a~więc pozbawione głębszego sensu, od tych które są od niego
niezależne i~jako takie mają niepodważalny sens fizyczny.

W~takiej sytuacji wzór \eqref{eq:RedSzydlowski-Teoria-pomiarow-04}
należałoby uważać za przekształcenie wzoru
\eqref{eq:RedSzydlowski-Teoria-pomiarow-03}, które jest dopuszczalne
w~momencie gdy wybierzemy konkretną skalę temperatur, na mocy własności
liczb rzeczywistych. Nie jestem tego pewien, ale wydaje mi~się, że~przejścia
od wzoru \eqref{eq:RedSzydlowski-Teoria-pomiarow-03} do
\eqref{eq:RedSzydlowski-Teoria-pomiarow-04} możemy dokonać, bo~pozwalają na
to własności liczb rzeczywistych. Korzystając więc z~tych własności
przekształcamy wzór który ma więcej treści fizycznej, we wzór który prowadzi
do tych samych numerycznych wartości, ale który jest wygodniejszy przy
przeprowadzaniu obliczeń i~łatwiejszy do zapamiętania.

Na~koniec należy wspomnieć, że~w~języku codziennym mówimy na przykład,
iż~temperatura $20 \, {}^{ \circ }\si{C}$ jest dwa razy większa od temperatury
$10 \, {}^{ \circ }\si{C}$, co nie stanowi żadnego problemu, choć jest
niepoprawne z~punktu widzenia teorii wartości na skali przedziałowej. Język
dnia codziennego ma~swoje własne prawa, odmienne od praw języka teorii
naukowej.

\VerSpaceFour





\noindent
\Str{15} Przy definiowaniu klasy przyrządu może lepiej niż o~„pełnym
wychyleniu miernika” jest mówić o~„pełnym zakresie miernika”?

\VerSpaceFour





{\Large Ponowne czytanie zacząć od strony 12.}

























% ##################
\newpage

\CenterBoldFont{Błędy}


\begin{center}

  \begin{tabular}{|c|c|c|c|c|}
    \hline
    Strona & \multicolumn{2}{c|}{Wiersz} & Jest
                              & Powinno być \\ \cline{2-3}
    & Od góry & Od dołu & & \\
    \hline
    13 & & 10 & $6350$ & $984$ \\
    13 & & \hphantom{0}7 & $6350$ & $984$ \\
    16 & & \hphantom{0}7 & Niepewności przypadkowe
    & \textit{Niepewności przypadkowe} \\
    % & & & & \\
    % & & & & \\
    % & & & & \\
    % & & & & \\
    % & & & & \\
    \hline
  \end{tabular}

\end{center}

\VerSpaceSix


% ############################










% ####################################################################
% ####################################################################
% Bibliography

\printbibliography





% ############################
% End of the document

\end{document}
