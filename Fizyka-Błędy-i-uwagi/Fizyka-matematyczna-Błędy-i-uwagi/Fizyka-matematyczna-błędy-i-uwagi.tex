% Autor: Kamil Ziemian
% Korekta: Wojciech Dyba

% ---------------------------------------------------------------------
% Podstawowe ustawienia i pakiety
% ---------------------------------------------------------------------
\RequirePackage[l2tabu, orthodox]{nag} % Wykrywa przestarzałe i niewłaściwe
% sposoby używania LaTeXa. Więcej jest w l2tabu English version.
\documentclass[a4paper,11pt]{article}
% {rozmiar papieru, rozmiar fontu}[klasa dokumentu]
\usepackage[MeX]{polski} % Polonizacja LaTeXa, bez niej będzie pracował
% w języku angielskim.
\usepackage[utf8]{inputenc} % Włączenie kodowania UTF-8, co daje dostęp
% do polskich znaków.
\usepackage{lmodern} % Wprowadza fonty Latin Modern.
\usepackage[T1]{fontenc} % Potrzebne do używania fontów Latin Modern.



% ------------------------------
% Podstawowe pakiety (niezwiązane z ustawieniami języka)
% ------------------------------
\usepackage{microtype} % Twierdzi, że poprawi rozmiar odstępów w tekście.
\usepackage{graphicx} % Wprowadza bardzo potrzebne komendy do wstawiania
% grafiki.
\usepackage{verbatim} % Poprawia otoczenie VERBATIME.
\usepackage{textcomp} % Dodaje takie symbole jak stopnie Celsiusa,
% wprowadzane bezpośrednio w tekście.
\usepackage{vmargin} % Pozwala na prostą kontrolę rozmiaru marginesów,
% za pomocą komend poniżej. Rozmiar odstępów jest mierzony w calach.
% ------------------------------
% MARGINS
% ------------------------------
\setmarginsrb
{ 0.7in}  % left margin
{ 0.6in}  % top margin
{ 0.7in}  % right margin
{ 0.8in}  % bottom margin
{  20pt}  % head height
{0.25in}  % head sep
{   9pt}  % foot height
{ 0.3in}  % foot sep



% ------------------------------
% Często przydatne pakiety
% ------------------------------
\usepackage{csquotes} % Pozwala w prosty sposób wstawiać cytaty do tekstu.
\usepackage{xcolor} % Pozwala używać kolorowych czcionek (zapewne dużo
% więcej, ale ja nie potrafię nic o tym powiedzieć).



% ------------------------------
% Pakiety do tekstów z nauk przyrodniczych
% ------------------------------
\let\lll\undefined % Amsmath gryzie się z językiem pakietami do języka
% polskiego, bo oba definiują komendę \lll. Aby rozwiązać ten problem
% oddefiniowuję tę komendę, ale może tym samym pozbywam się dużego Ł.
\usepackage[intlimits]{amsmath} % Podstawowe wsparcie od American
% Mathematical Society (w skrócie AMS)
\usepackage{amsfonts, amssymb, amscd, amsthm} % Dalsze wsparcie od AMS
% \usepackage{siunitx} % Dla prostszego pisania jednostek fizycznych
\usepackage{upgreek} % Ładniejsze greckie litery
% Przykładowa składnia: pi = \uppi
\usepackage{slashed} % Pozwala w prosty sposób pisać slash Feynmana.
\usepackage{calrsfs} % Zmienia czcionkę kaligraficzną w \mathcal
% na ładniejszą. Może w innych miejscach robi to samo, ale o tym nic
% nie wiem.



% ------------------------------
% Tworzenie środowisk (?) „Twierdzenie”, „Definicja”, „Lemat”, etc.
% ------------------------------
% Komenda wprowadzająca otoczenie „theorem” do pisania twierdzeń
% matematycznych.
\newtheorem{theorem}{Twierdzenie}
% Analogicznie jak powyżej
\newtheorem{definition}{Definicja}
\newtheorem{corollary}{Wniosek}



% ---------------------------------------
% Pakiety napisane przez użytkownika.
% Mają być w tym samym katalogu to ten plik .tex
% ---------------------------------------
\usepackage{latexgeneralcommands}
\usepackage{mathcommands}
% \usepackage{calculuscommands}
% \usepackage{SchwartzBooksCommands}  % Pakiet napisany m.in. dla tego pliku.



% ---------------------------------------------------------------------
% Dodatkowe ustawienia dla języka polskiego
% ---------------------------------------------------------------------
\renewcommand{\thesection}{\arabic{section}.}
% Kropki po numerach rozdziału (polski zwyczaj topograficzny)
\renewcommand{\thesubsection}{\thesection\arabic{subsection}}
% Brak kropki po numerach podrozdziału



% ------------------------------
% Ustawienia różnych parametrów tekstu
% ------------------------------
\renewcommand{\baselinestretch}{1.1}

% Ustawienie szerokości odstępów między wierszami w tabelach.
\renewcommand{\arraystretch}{1.4}



% ------------------------------
% Pakiet „hyperref”
% Polecano by umieszczać go na końcu preambuły.
% ------------------------------
\usepackage{hyperref} % Pozwala tworzyć hiperlinki i zamienia odwołania
% do bibliografii na hiperlinki.










% ---------------------------------------------------------------------
% Tytuł, autor, data
\title{Fizyka matematyczna \\
  {\Large Błędy i~uwagi}}

\author{Kamil Ziemian}


% \date{}
% ---------------------------------------------------------------------










% ####################################################################
\begin{document}
% ####################################################################





% ######################################
\maketitle % Tytuł całego tekstu
% ######################################





% ####################
\Work{ % Autorzy i tytuł dziełą
  R.~Curant, D.~Hilbert \\
  \textit{Methods~of Mathematical Physics}, \cite{}}


% ##################
\CenterBoldFont{Błędy}


\begin{center}

  \begin{tabular}{|c|c|c|c|c|}
    \hline
    Strona & \multicolumn{2}{c|}{Wiersz} & Jest
                              & Powinno być \\ \cline{2-3}
    & Od góry & Od dołu & & \\
    \hline
    % 15 & & 10 & $\eb_{ p } \lb_{ p }$ & $\eb_{ p } \cdot \lb_{ p }$ \\
    ????? & & & & \\
    % & & & & \\
    % & & & & \\
    % & & & & \\
    % & & & & \\
    % & & & & \\
    % & & & & \\
    % & & & & \\
    % & & & & \\
    % & & & & \\
    \hline
  \end{tabular}

\end{center}

\vspace{\spaceOne}


% ############################










% ##############################
\Work{ % Autor i tytuł dzieła
  Walter Thirring \\
  \textit{Fizyka matematyczna} \\
  \textit{Tom~I: Klasyczne układy dynamiczne},
  \cite{ThirringFizykaMatematycznaVolI1985}}

\vspace{0em}


% ##################
\CenterBoldFont{Uwagi do~konkretnych stron}

\vspace{0em}


\noindent
\Str{18} W~definicji (2.1,1) brakuje założenia o~istnieniu wektora zerowego. ?????

\vspace{\spaceFour}




\noindent
\StrWg{45}{1} Nie wiem jak w~sposób zgrabny napisać, że~wektor
$( 0, 0, \ldots, 0 )$ ma długość $n - 1$.





% ##################
\newpage

\CenterBoldFont{Błędy}


\begin{center}

  \begin{tabular}{|c|c|c|c|c|}
    \hline
    Strona & \multicolumn{2}{c|}{Wiersz} & Jest
                              & Powinno być \\ \cline{2-3}
    & Od góry & Od dołu & & \\
    \hline
    9  &  3 & & $f\!:\;\: =$ & $f :=$ \\
    9  & &  3 & $z^{ 2 }$ & $\absOne{ z }^{ 2 }$ \\
    23 & 17 & & $x_{ 1 }, x_{ 2 } )$ & $( x_{ 1 }, x_{ 2 } )$ \\
    30 &  3 & & $\Phi_{ 1 } = :$ & $\Phi_{ 1 } :$ \\
    37 & 10 & & $x_{ 2 } v_{ 2 }$ & $x_{ 2 } v_{ 2 } )$ \\
    37 & 14 & & $`T( M )$ & $T( M )$ \\
    37 & &  3 & $( q, v ) \rightarrow v$ & $( q, v )$ \\
    43 & &  4 & $x_{ i }( q )$ & $X_{ i }( q )$ \\
    45 & 1 & & $\Rbb^{ n } \setminus \Rbb \times \{ 0, 0, \ldots, 0 \}$
           & $\Rbb^{ n } \setminus ( \Rbb \times ( 0, 0, \ldots, 0 ) )$ \\
    48 & &  9 & $T( T( M )$ & $T( T( M ) )$ \\
    52 & &  1 & $T_{ q }( M )$ & $T_{ q }( M )$\big) \\
    53 & 11 & & $L_{ \Theta^{ -1 }_{ C } }( q ) e_{ j } \, dq^{ i }$
           & $L_{ \Theta^{ -1 }_{ C }( q ) e_{ j } } q^{ i }$ \\
    74 & & 16 & ??? & rozmaitości orientowalnych \\
    75 &  9 & & $\sup \; f_{ i }$ & $\supp \; f_{ i }$ \\
    75 & & 10 & do $\sup \; f$ & od $\supp \; f$ \\
    77 & & 8 & $x^{ 2 } + x^{ 2 }$ & $x^{ 2 } + y^{ 2 }$ \\
    % & & & & \\
    % & & & & \\
    % & & & & \\
    % & & & & \\
    % & & & & \\
    % & & & & \\
    % & & & & \\
    % & & & & \\
    % & & & & \\
    % & & & & \\
    % & & & & \\
    \hline
  \end{tabular}

\end{center}

\vspace{\spaceTwo}


\noindent
\StrWd{10}{7} \\
\Jest  $d( a, a ) = 0$, \\[0.1em]
$d( a,b ) > 0$ dla~$a \neq b$, $d( a, c ) \leq d( a, b ) + d( b,c )$; \\[0.3em]
\Powin $d( a, a ) = 0$, $d( a,b ) > 0$ dla~$a \neq b$,
$d( a, c ) \leq d( a, b ) + d( b,c )$; \\[0.3em]
\StrWd{29}{2} \\[0.3em]
\Jest  $\partial M = \{ a \} \cup \{ b \}$, $U_{ 2 } = ( a, b ]$,
$\Phi_{ 2 } : x \rightarrow b - a$. \\[0.3em]
\Powin $U_{ 2 } = ( a, b ]$, $\Phi_{ 2 } : x \rightarrow b - a$,
$\partial M = \{ a \} \cup \{ b \}$. \\[0.3em]
\StrWg{38}{14} \\
\Jest  Wiązka ta jest trywializowalna, jeśli rozmaitość $X$ jest
paralelyzowalna. \\
\Powin Jeśli wiązka ta jest trywializowalna, to rozmaitość jest
pralelyzowalna. \\
\StrWd{43}{9} \\
\Jest  $\mathbf{ 1 } \, \times$ \textbf{jednostkowy wektor} \\
\Powin $\mathbf{ 1 } \, \equiv$ \textbf{jednostkowy wektor} \\
\StrWd{43}{5} \\
\Jest  $\mathbf{ 1 } \, \times$ wektor jednostkowy \\
\Powin $\mathbf{ 1 } \equiv$ jednostkowy wektor styczny \\






% ############################










% ##############################
\newpage

\Work{ % Autor i tytuł dzieła
  Walter Thirring \\
  \textit{Fizyka matematyczna} \\
  \textit{Tom~III: Mechanika kwantowa atomów} \\
  \textit{i~cząsteczek}, \cite{ThirringFizykaMatematycznaVolIV1987}}

\vspace{0em}


% ##################
\CenterBoldFont{Uwagi do~konkretnych stron}

\vspace{0em}


\noindent
\Str{18} W~definicji (2.1,1) brakuje założenia o~istnieniu wektora zerowego. ?????

\vspace{\spaceFour}





% ##################
\newpage

\CenterBoldFont{Błędy}


\begin{center}

  \begin{tabular}{|c|c|c|c|c|}
    \hline
    Strona & \multicolumn{2}{c|}{Wiersz} & Jest
                              & Powinno być \\ \cline{2-3}
    & Od góry & Od dołu & & \\
    \hline
    19  & 15 & & praktyczne.Moc & praktyczne. Moc \\
    % & & & & \\
    % & & & & \\
    % & & & & \\
    % & & & & \\
    % & & & & \\
    % & & & & \\
    % & & & & \\
    % & & & & \\
    % & & & & \\
    % & & & & \\
    % & & & & \\
    \hline
  \end{tabular}

\end{center}

\vspace{\spaceTwo}



% ############################










% #####################################################################
% #####################################################################
% Bibliografia

\bibliographystyle{plalpha}

\bibliography{PhilNaturBooks}{}





% ############################

% Koniec dokumentu
\end{document}
