% ------------------------------------------------------------------------------------------------------------------
% Basic configuration and packages
% ------------------------------------------------------------------------------------------------------------------
% Package for discovering wrong and outdated usage of LaTeX.
% More information to be found in l2tabu English version.
\RequirePackage[l2tabu, orthodox]{nag}
% Class of LaTeX document: {size of paper, size of font}[document class]
\documentclass[a4paper,11pt]{article}



% ------------------------------------------------------
% Packages not tied to particular normal language
% ------------------------------------------------------
% This package should improved spaces in the text
\usepackage{microtype}
% Add few important symbols, like text Celcius degree
\usepackage{textcomp}



% ------------------------------------------------------
% Polonization of LaTeX document
% ------------------------------------------------------
% Basic polonization of the text
\usepackage[MeX]{polski}
% Switching on UTF-8 encoding
\usepackage[utf8]{inputenc}
% Adding font Latin Modern
\usepackage{lmodern}
% Package is need for fonts Latin Modern
\usepackage[T1]{fontenc}



% ------------------------------------------------------
% Setting margins
% ------------------------------------------------------
\usepackage[a4paper, total={14cm, 25cm}]{geometry}



% ------------------------------------------------------
% Setting vertical spaces in the text
% ------------------------------------------------------
% Setting space between lines
\renewcommand{\baselinestretch}{1.1}

% Setting space between lines in tables
\renewcommand{\arraystretch}{1.4}



% ------------------------------------------------------
% Packages for scientific papers
% ------------------------------------------------------
% Switching off \lll symbol, that I guess is representing letter "Ł"
% It collide with `amsmath' package's command with the same name
\let\lll\undefined
% Basic package from American Mathematical Society (AMS)
\usepackage[intlimits]{amsmath}
% Equations are numbered separately in every section
\numberwithin{equation}{section}

% Other very useful packages from AMS
\usepackage{amsfonts}
\usepackage{amssymb}
\usepackage{amscd}
\usepackage{amsthm}

% Package with better looking calligraphy fonts
\usepackage{calrsfs}

% Package with better looking greek letters
% Example of use: pi -> \uppi
\usepackage{upgreek}
% Improving look of lambda letter
\let\oldlambda\Lambda
\renewcommand{\lambda}{\uplambda}




% ------------------------------------------------------
% BibLaTeX
% ------------------------------------------------------
% Package biblatex, with biber as its backend, allow us to handle
% bibliography entries that use Unicode symbols outside ASCII
\usepackage[
language=polish,
backend=biber,
style=alphabetic,
url=false,
eprint=true,
]{biblatex}

\addbibresource{LogikaITeoriaMnogosciBibliography.bib}





% ------------------------------------------------------
% Defining new environments (?)
% ------------------------------------------------------
% Defining enviroment "Wniosek"
\newtheorem{corollary}{Wniosek}
\newtheorem{definition}{Definicja}
\newtheorem{theorem}{Twierdzenie}





% ------------------------------------------------------
% Local packages
% You need to put them in the same directory as .tex file
% ------------------------------------------------------
% Package containing various command useful for working with a text
\usepackage{general-commands}
% Package containing commands and other code useful for working with
% mathematical text
\usepackage{math-commands}





% ------------------------------------------------------
% Package "hyperref"
% They advised to put it on the end of preambule
% ------------------------------------------------------
% It allows you to use hyperlinks in the text
\usepackage{hyperref}










% ------------------------------------------------------------------------------------------------------------------
% Tytuł, autor, data
\title{Fizyka matematyczna \\
  {\Large Błędy i~uwagi}}

\author{Kamil Ziemian, korekta Wojciech Dyba}


% \date{}
% ------------------------------------------------------------------------------------------------------------------










% ####################################################################
\begin{document}
% ####################################################################





% ######################################
\maketitle % Tytuł całego tekstu
% ######################################





% ######################################
\section{R.~Curant, D.~Hilbert \textit{Methods~of
    Mathematical Physics}, \cite{}}

% ######################################


% ##################
\CenterBoldFont{Błędy}


\begin{center}

  \begin{tabular}{|c|c|c|c|c|}
    \hline
    Strona & \multicolumn{2}{c|}{Wiersz} & Jest
                              & Powinno być \\ \cline{2-3}
    & Od góry & Od dołu & & \\
    \hline
    % 15 & & 10 & $\eb_{ p } \lb_{ p }$ & $\eb_{ p } \cdot \lb_{ p }$ \\
    ????? & & & & \\
    % & & & & \\
    % & & & & \\
    % & & & & \\
    % & & & & \\
    % & & & & \\
    % & & & & \\
    % & & & & \\
    % & & & & \\
    % & & & & \\
    \hline
  \end{tabular}

\end{center}

\VerSpaceTwo


% ######################################










% ######################################
\section{Walter Thirring \\
  \textit{Fizyka matematyczna} \\
  \textit{Tom~I: Klasyczne układy dynamiczne},
  \cite{ThirringFizykaMatematycznaVolI1985}}

% ######################################

\vspace{0em}


% ##################
\CenterBoldFont{Uwagi do~konkretnych stron}

\vspace{0em}


\noindent
\Str{18} W~definicji (2.1,1) brakuje założenia o~istnieniu wektora zerowego. ?????

\VerSpaceFour




\noindent
\StrWierszGora{45}{1} Nie wiem jak w~sposób zgrabny napisać, że~wektor
$( 0, 0, \ldots, 0 )$ ma długość $n - 1$.





% ##################
\newpage

\CenterBoldFont{Błędy}


\begin{center}

  \begin{tabular}{|c|c|c|c|c|}
    \hline
    Strona & \multicolumn{2}{c|}{Wiersz} & Jest
                              & Powinno być \\ \cline{2-3}
    & Od góry & Od dołu & & \\
    \hline
    9  &  3 & & $f\!:\;\: =$ & $f :=$ \\
    9  & &  3 & $z^{ 2 }$ & $\absOne{ z }^{ 2 }$ \\
    23 & 17 & & $x_{ 1 }, x_{ 2 } )$ & $( x_{ 1 }, x_{ 2 } )$ \\
    30 &  3 & & $\Phi_{ 1 } = :$ & $\Phi_{ 1 } :$ \\
    37 & 10 & & $x_{ 2 } v_{ 2 }$ & $x_{ 2 } v_{ 2 } )$ \\
    37 & 14 & & $`T( M )$ & $T( M )$ \\
    37 & &  3 & $( q, v ) \rightarrow v$ & $( q, v )$ \\
    43 & &  4 & $x_{ i }( q )$ & $X_{ i }( q )$ \\
    45 & 1 & & $\Rbb^{ n } \setminus \Rbb \times \{ 0, 0, \ldots, 0 \}$
           & $\Rbb^{ n } \setminus ( \Rbb \times ( 0, 0, \ldots, 0 ) )$ \\
    48 & &  9 & $T( T( M )$ & $T( T( M ) )$ \\
    52 & &  1 & $T_{ q }( M )$ & $T_{ q }( M )$\big) \\
    53 & 11 & & $L_{ \Theta^{ -1 }_{ C } }( q ) e_{ j } \, dq^{ i }$
           & $L_{ \Theta^{ -1 }_{ C }( q ) e_{ j } } q^{ i }$ \\
    74 & & 16 & ??? & rozmaitości orientowalnych \\
    75 &  9 & & $\sup \; f_{ i }$ & $\supp \; f_{ i }$ \\
    75 & & 10 & do $\sup \; f$ & od $\supp \; f$ \\
    77 & & 8 & $x^{ 2 } + x^{ 2 }$ & $x^{ 2 } + y^{ 2 }$ \\
    % & & & & \\
    % & & & & \\
    % & & & & \\
    % & & & & \\
    % & & & & \\
    % & & & & \\
    % & & & & \\
    % & & & & \\
    % & & & & \\
    % & & & & \\
    % & & & & \\
    \hline
  \end{tabular}

\end{center}

\VerSpaceTwo


\noindent
\StrWierszDol{10}{7} \\
\Jest $d( a, a ) = 0$, \\[0.1em]
$d( a,b ) > 0$ dla~$a \neq b$, $d( a, c ) \leq d( a, b ) + d( b,c )$; \\[0.3em]
\PowinnoByc $d( a, a ) = 0$, $d( a,b ) > 0$ dla~$a \neq b$,
$d( a, c ) \leq d( a, b ) + d( b,c )$; \\[0.3em]
\StrWierszDol{29}{2} \\[0.3em]
\Jest $\partial M = \{ a \} \cup \{ b \}$, $U_{ 2 } = ( a, b ]$,
$\Phi_{ 2 } : x \rightarrow b - a$. \\[0.3em]
\PowinnoByc $U_{ 2 } = ( a, b ]$, $\Phi_{ 2 } : x \rightarrow b - a$,
$\partial M = \{ a \} \cup \{ b \}$. \\[0.3em]
\StrWierszGora{38}{14} \\
\Jest Wiązka ta jest trywializowalna, jeśli rozmaitość $X$ jest
paralelyzowalna. \\
\PowinnoByc Jeśli wiązka ta jest trywializowalna, to rozmaitość jest
pralelyzowalna. \\
\StrWierszDol{43}{9} \\
\Jest $\mathbf{ 1 } \, \times$ \textbf{jednostkowy wektor} \\
\PowinnoByc $\mathbf{ 1 } \, \equiv$ \textbf{jednostkowy wektor} \\
\StrWierszDol{43}{5} \\
\Jest $\mathbf{ 1 } \, \times$ wektor jednostkowy \\
\PowinnoByc $\mathbf{ 1 } \equiv$ jednostkowy wektor styczny \\






% ############################










% ######################################
\section{Walter Thirring \\
  \textit{Fizyka matematyczna} \\
  \textit{Tom~III: Mechanika kwantowa atomów} \\
  \textit{i~cząsteczek}, \cite{ThirringFizykaMatematycznaVolIV1987}}

% ######################################

\vspace{0em}


% ##################
\CenterBoldFont{Uwagi do~konkretnych stron}

\vspace{0em}


\noindent
\Str{18} W~definicji (2.1,1) brakuje założenia o~istnieniu wektora zerowego. ?????

\VerSpaceFour





% ##################
\newpage

\CenterBoldFont{Błędy}


\begin{center}

  \begin{tabular}{|c|c|c|c|c|}
    \hline
    Strona & \multicolumn{2}{c|}{Wiersz} & Jest
                              & Powinno być \\ \cline{2-3}
    & Od góry & Od dołu & & \\
    \hline
    19  & 15 & & praktyczne.Moc & praktyczne. Moc \\
    % & & & & \\
    % & & & & \\
    % & & & & \\
    % & & & & \\
    % & & & & \\
    % & & & & \\
    % & & & & \\
    % & & & & \\
    % & & & & \\
    % & & & & \\
    % & & & & \\
    \hline
  \end{tabular}

\end{center}

\VerSpaceTwo



% ############################










% ####################################################################
% ####################################################################
% Bibliography

\printbibliography





% ############################
% End of the document

\end{document}
