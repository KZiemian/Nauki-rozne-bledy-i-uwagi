% ------------------------------------------------------------------------------------------------------------------
% Basic configuration and packages
% ------------------------------------------------------------------------------------------------------------------
% Package for discovering wrong and outdated usage of LaTeX.
% More information to be found in l2tabu English version.
\RequirePackage[l2tabu, orthodox]{nag}
% Class of LaTeX document: {size of paper, size of font}[document class]
\documentclass[a4paper,11pt]{article}



% ------------------------------------------------------
% Packages not tied to particular normal language
% ------------------------------------------------------
% This package should improved spaces in the text
\usepackage{microtype}
% Add few important symbols, like text Celcius degree
\usepackage{textcomp}



% ------------------------------------------------------
% Polonization of LaTeX document
% ------------------------------------------------------
% Basic polonization of the text
\usepackage[MeX]{polski}
% Switching on UTF-8 encoding
\usepackage[utf8]{inputenc}
% Adding font Latin Modern
\usepackage{lmodern}
% Package is need for fonts Latin Modern
\usepackage[T1]{fontenc}



% ------------------------------------------------------
% Setting margins
% ------------------------------------------------------
\usepackage[a4paper, total={14cm, 25cm}]{geometry}



% ------------------------------------------------------
% Setting vertical spaces in the text
% ------------------------------------------------------
% Setting space between lines
\renewcommand{\baselinestretch}{1.1}

% Setting space between lines in tables
\renewcommand{\arraystretch}{1.4}



% ------------------------------------------------------
% Packages for scientific papers
% ------------------------------------------------------
% Switching off \lll symbol, that I guess is representing letter "Ł"
% It collide with `amsmath' package's command with the same name
\let\lll\undefined
% Basic package from American Mathematical Society (AMS)
\usepackage[intlimits]{amsmath}
% Equations are numbered separately in every section
\numberwithin{equation}{section}

% Other very useful packages from AMS
\usepackage{amsfonts}
\usepackage{amssymb}
\usepackage{amscd}
\usepackage{amsthm}

% Package with better looking calligraphy fonts
\usepackage{calrsfs}

% Package with better looking greek letters
% Example of use: pi -> \uppi
\usepackage{upgreek}
% Improving look of lambda letter
\let\oldlambda\Lambda
\renewcommand{\lambda}{\uplambda}




% ------------------------------------------------------
% BibLaTeX
% ------------------------------------------------------
% Package biblatex, with biber as its backend, allow us to handle
% bibliography entries that use Unicode symbols outside ASCII
\usepackage[
language=polish,
backend=biber,
style=alphabetic,
url=false,
eprint=true,
]{biblatex}

\addbibresource{LogikaITeoriaMnogosciBibliography.bib}





% ------------------------------------------------------
% Defining new environments (?)
% ------------------------------------------------------
% Defining enviroment "Wniosek"
\newtheorem{corollary}{Wniosek}
\newtheorem{definition}{Definicja}
\newtheorem{theorem}{Twierdzenie}





% ------------------------------------------------------
% Local packages
% You need to put them in the same directory as .tex file
% ------------------------------------------------------
% Package containing various command useful for working with a text
% \usepackage{general-commands}
% Package containing commands and other code useful for working with
% mathematical text
% \usepackage{math-commands}





% ------------------------------------------------------
% Package "hyperref"
% They advised to put it on the end of preambule
% ------------------------------------------------------
% It allows you to use hyperlinks in the text
\usepackage{hyperref}










% ------------------------------------------------------------------------------------------------------------------
% Tytuł, autor, data
\title{Fizyka fazy skondensowanej \\
  {\Large Błędy i~uwagi}}

\author{Kamil Ziemian}


% \date{}
% ------------------------------------------------------------------------------------------------------------------










% ####################################################################
% Beginning of the document
\begin{document}
% ####################################################################





% ######################################
% Title of the text
\maketitle
% ######################################





% ######################################
\section{Charles Kittle \textit{Wstęp do~fizyki ciała
    stałego},
  \cite{Kittel-Wstep-do-fizyki-ciala-stalego-Pub-1976}}

\VerSpaceTwo
% ######################################



% ##################
\CenterBoldFont{Uwagi do~konkretnych stron}

\vspace{0em}


\noindent
\StrWierszeDol{25}{11--12} Treść zdania „Na rysunku przedstawino zbiór węzłów
sieci, wybór sieci, wybór prostych osi, prostej komórki i~bazy atomów
związanej z~węzłem sieci.” nie odnosi~się do omawianego tu rysunku~c
i~dlatego należy je usunąć.

\VerSpaceFour


















% ######################################
\section{Józef Spałek \textit{Wstęp do~fizyki materii
    skondensowanej},
  \cite{SpalekWstepDoFizykiMateriiSkondensowanej2015}}

\VerSpaceTwo
% ######################################

\vspace{0em}


% ##################
\CenterBoldFont{Uwagi do~konkretnych stron}

\vspace{0em}


\noindent
\textbf{Str.~8, rysunek~1.1.} Obszar zakreskowany jest tak gęsto,
że~jeśli nie przyglądnąć~się z~bliska, to wygląda jak jednolicie
pokryty farbą o~kolorze pomiędzy ciemną czerwienią, a~pomarańczowym.

\VerSpaceFour





\noindent
\textbf{Str.~8, rysunek~1.1.} Jak rozumiem umieszczenie na~osi~$y$
oznaczenia $V( \vecrBold )$, $\varepsilon$ ma symbolizować, że~na tej osi jest
obrazowana jednocześnie energia potencjalna $V( \vecrBold )$ i~energia
elektronu~$\varepsilon$.

\VerSpaceFour





\noindent
\Str{10} Rozróżnienie na~cechy korpuskularne i~falowe, jest nie tylko
XIX-wieczne, ale też bardzo niewygodne i~problematyczne, przynajmniej
w~mojej osobistej opinii. Dlatego należałoby je już odrzucić, jako
przestarzałe, bo jest już wszakże XXI~wiek.

\VerSpaceFour





\noindent
\StrWierszDol{10}{4} Symbol $d / d\veckBold$ należy tu rozumieć w~ten sposób,
że~jest to gradient funkcji skalarnej zależnej od~$\veckBold$.

% \vspace{\spaceFour}





% ##################
\newpage

\CenterBoldFont{Błędy}


\begin{center}

  \begin{tabular}{|c|c|c|c|c|}
    \hline
    Strona & \multicolumn{2}{c|}{Wiersz} & Jest
                              & Powinno być \\ \cline{2-3}
    & Od góry & Od dołu & & \\ \hline
    12 & & 8 & (1.13){ }{ }\textit{wymaga} & (1.13) \textit{wymaga} \\
    % & & & & \\
    % & & & & \\
    % & & & & \\
    % & & & & \\
    \hline
  \end{tabular}

\end{center}

\VerSpaceTwo






% ############################










% ####################################################################
% ####################################################################
% Bibliography

\printbibliography





% ############################
% End of the document

\end{document}
