% ------------------------------------------------------------------------------------------------------------------
% Basic configuration and packages
% ------------------------------------------------------------------------------------------------------------------
% Package for discovering wrong and outdated usage of LaTeX.
% More information to be found in l2tabu English version.
\RequirePackage[l2tabu, orthodox]{nag}
% Class of LaTeX document: {size of paper, size of font}[document class]
\documentclass[a4paper,11pt]{article}



% ------------------------------------------------------
% Packages not tied to particular normal language
% ------------------------------------------------------
% This package should improved spaces in the text
\usepackage{microtype}
% Add few important symbols, like text Celcius degree
\usepackage{textcomp}



% ------------------------------------------------------
% Polonization of LaTeX document
% ------------------------------------------------------
% Basic polonization of the text
\usepackage[MeX]{polski}
% Switching on UTF-8 encoding
\usepackage[utf8]{inputenc}
% Adding font Latin Modern
\usepackage{lmodern}
% Package is need for fonts Latin Modern
\usepackage[T1]{fontenc}



% ------------------------------------------------------
% Setting margins
% ------------------------------------------------------
\usepackage[a4paper, total={14cm, 25cm}]{geometry}



% ------------------------------------------------------
% Setting vertical spaces in the text
% ------------------------------------------------------
% Setting space between lines
\renewcommand{\baselinestretch}{1.1}

% Setting space between lines in tables
\renewcommand{\arraystretch}{1.4}



% ------------------------------------------------------
% Packages for scientific papers
% ------------------------------------------------------
% Switching off \lll symbol, that I guess is representing letter "Ł"
% It collide with `amsmath' package's command with the same name
\let\lll\undefined
% Basic package from American Mathematical Society (AMS)
\usepackage[intlimits]{amsmath}
% Equations are numbered separately in every section
\numberwithin{equation}{section}

% Other very useful packages from AMS
\usepackage{amsfonts}
\usepackage{amssymb}
\usepackage{amscd}
\usepackage{amsthm}

% Package with better looking calligraphy fonts
\usepackage{calrsfs}

% Package with better looking greek letters
% Example of use: pi -> \uppi
\usepackage{upgreek}
% Improving look of lambda letter
\let\oldlambda\Lambda
\renewcommand{\lambda}{\uplambda}




% ------------------------------------------------------
% BibLaTeX
% ------------------------------------------------------
% Package biblatex, with biber as its backend, allow us to handle
% bibliography entries that use Unicode symbols outside ASCII
\usepackage[
language=polish,
backend=biber,
style=alphabetic,
url=false,
eprint=true,
]{biblatex}

\addbibresource{LogikaITeoriaMnogosciBibliography.bib}





% ------------------------------------------------------
% Defining new environments (?)
% ------------------------------------------------------
% Defining enviroment "Wniosek"
\newtheorem{corollary}{Wniosek}
\newtheorem{definition}{Definicja}
\newtheorem{theorem}{Twierdzenie}





% ------------------------------------------------------
% Local packages
% You need to put them in the same directory as .tex file
% ------------------------------------------------------
% Package containing various command useful for working with a text
\usepackage{./Local-packages/general-commands}
% Package containing commands and other code useful for working with
% mathematical text
\usepackage{./Local-packages/math-commands}





% ------------------------------------------------------
% Package "hyperref"
% They advised to put it on the end of preambule
% ------------------------------------------------------
% It allows you to use hyperlinks in the text
\usepackage{hyperref}










% ------------------------------------------------------------------------------------------------------------------
% Title and author of the text
\title{Termostatyka, termodynamika i~fizyka statystyczna \\
  {\Large Błędy i~uwagi}}

\author{Kamil Ziemian}


% \date{}
% ------------------------------------------------------------------------------------------------------------------










% ####################################################################
% Beginning of the document
\begin{document}
% ####################################################################





% ######################################
\maketitle
% ######################################





% % ######################################
% \section{Termostatyka}

% \VerSpaceTwo
% % ######################################



% ####################
\section{Józef Werle \textit{Termodynamika fenomenologiczna},
  \cite{WerleTermodynamikaFenomenologiczna1957}}

\vspace{0em}


% ##################
\CenterBoldFont{Uwagi do~konkretnych stron}

\vspace{0em}


\noindent
\Str{15} Werle wprowadza jak strasznie zamieszanie pojęciowe,
twierdząc na tej samej stronie, że~objętość jest zarówno parametrem
wewnętrzny jak i~zewnętrznym. Powinno~się jawnie stwierdzić, że~mogą
istnieć parametry zarówno wewnętrzne, jak i~zewnętrzne układu,
albo~że~są to wykluczające~się kategorie i~nigdy nie wprowadzać jednej
wielkości do~nich obu.

Dobrego wyjaśnienie podał Kacper Zalewski. Zmienne zewnętrzne to takie
na którymi mamy kontrolę. ????A czy nad objętością metalu mamy
  kontrolę i tym podobne rozterki? Dokończ.???

\VerSpaceFour



\noindent
\Str{17}





% ##################
\noindent

\CenterBoldFont{Błędy}


\begin{center}

  \begin{tabular}{|c|c|c|c|c|}
    \hline
    Strona & \multicolumn{2}{c|}{Wiersz} & Jest
                              & Powinno być \\ \cline{2-3}
    & Od góry & Od dołu & & \\
    \hline
    5   & &  2 & odwracalnych & nieodwracalnych \\
    15  & & 13 & lśedzenia & śledzenia \\
    % & & & & \\
    % & & & & \\
    % & & & & \\
    % & & & & \\
    % & & & & \\
    % & & & & \\
    % & & & & \\
    % & & & & \\
    % & & & & \\
    \hline
  \end{tabular}

\end{center}

\VerSpaceTwo


% ############################










% ######################################
\section{Kerson Huang \textit{Mechanika statystyczna},
  \cite{HuangMechanikaStatystyczna1987}}

% ######################################

\vspace{0em}


% ##################
\CenterBoldFont{Uwagi}

\vspace{0em}


\noindent
???? Brak dyskusji zależności wykładanej teorii od układu odniesienia.

\VerSpaceFour





\noindent
???? Jaka jest struktura matematyczna przestrzeni stanów termodynamicznych
(krótko: przestrzeni termodynamicznej)?

\VerSpaceFour





\noindent
??? Przedstawiona tu dyskusja termodynamiki zostawia ogromną ilość pytań,
zarówno fizycznych jak i matematycznych, bez odpowiedzi.

\VerSpaceFour





\noindent
???? Prawdopodobieństwo zaistnienia danego stanu w części poświęconej fizyce
statystycznej nie zostało wyczerpująco omówione.





% ##################
\CenterBoldFont{Uwagi do~konkretnych stron}


\noindent
\Str{12} To, że równanie
\begin{equation}
  \label{eq:Huang-Mechanika-statystyczna-01}
  f( P, V, T ) = 0
\end{equation}
definiuje nam powierzchnię w~$\Rbb^{ 3 }$ jest intuicyjnie jasne. Trzeba by
oczywiście doprecyzować jakie własności posiada funkcja, czy jest na
przykład różniczkowalna i~wtedy możemy mówić o~powierzchni różniczkowalnej,
etc.

Jednak jeden problem wydaje~się być zupełnie nieporuszony. Czy przy
ustalonej wartości dwóch parametrów, powiedzmy $P_{ 1 }$ i~$V_{ 1 }$ jest
dopuszczalna więcej niż jedna wartość temperatury $T$? Inaczej mówiąc, czy
równanie $f( P_{ 1 }, V_{ 1 }, T ) = 0$ dopuszcza tylko jedno rozwiązanie na
$T$, czy też może ich być więcej: $T_{ 1 }$, $T_{ 2 }$, $T_{ 3 }$, \ldots

Jak powszechnie wiadomo równanie sfery w~trzech wymiarach
\begin{equation}
  \label{eq:Huang-Mechanika-statyczna-02}
  f( x, y, z ) = x^{ 2 } + y^{ 2 } + z^{ 2 } - 1 = 0,
\end{equation}
ma dla $x_{ 1 }$, $y_{ 2 }$ spełniających związek
$( x_{ 1 } )^{ 2 } + ( y_{ 1 } )^{ 2 } = 1$ dokładnie jedno rozwiązanie
$z = 0$. Jeżeli $( x_{ 1 } )^{ 2 } + ( y_{ 1 } )^{ 2 } < 1$ rozwiązania są
dokładnie $z = \sqrt{ ( x_{ 1 } )^{ 2 } + ( y_{ 1 } )^{ 2 } }$
i~$z = -\sqrt{ ( x_{ 1 } )^{ 2 } + ( y_{ 1 } )^{ 2 } }$. W~pozostałych
przypadkach, $( x_{ 1 } )^{ 2 } + ( y_{ 1 } )^{ 2 } > 1$, nie ma żadnych
rozwiązań. Przykład ten pokazuje więc, że~kwestia tego jaką powierzchnię
wyznacza równanie \eqref{eq:Huang-Mechanika-statyczna-01} nie jest wcale
banalna ani z~matematycznego, ani z~fizycznego punktu widzenia.










% ##################
\CenterBoldFont{Błędy}


\begin{center}

  \begin{tabular}{|c|c|c|c|c|}
    \hline
    Strona & \multicolumn{2}{c|}{Wiersz} & Jest
                              & Powinno być \\ \cline{2-3}
    & Od góry & Od dołu & & \\
    \hline
    12  & & 12 & $f( p, V, T )$ & $f( P, V, T )$ \\
    %     & & & & \\
    %     & & & & \\
    \hline
  \end{tabular}

\end{center}

\VerSpaceTwo


Str. 135-136. dla których gęstość zależy od $( p, q )$ tylko przez
hamiltonian \\


% ############################










% ########################################
\section{Józef Werle \textit{Termodynamika fenomenologiczna}
  \cite{Wer57}}

% ########################################

\vspace{0em}


% ##################
\CenterBoldFont{Uwagi do~konkretnych stron}

\vspace{0em}


\noindent
\Str{15} Werle wprowadza jak strasznie zamieszanie pojęciowe,
twierdząc na tej samej stronie, że~objętość jest zarówno parametrem
wewnętrzny jak i~zewnętrznym. Powinno~się jawnie stwierdzić, że~mogą
istnieć parametry zarówno wewnętrzne, jak i~zewnętrzne układu,
albo~że~są to wykluczające~się kategorie i~nigdy nie wprowadzać
jednej wielkości do~nich obu.

Trzeba~się zastanowić, jak jest naprawdę. ????Dok

\VerSpaceFour





\noindent
\Str{17}


% ##################
\CenterBoldFont{Błędy}


\begin{center}

  \begin{tabular}{|c|c|c|c|c|}
    \hline
    Strona & \multicolumn{2}{c|}{Wiersz} & Jest
    & Powinno być \\ \cline{2-3}
    & Od góry & Od dołu &  &  \\ \hline
    5 & & 2 & odwracalnych & nieodwracalnych \\
    15 & & 13 & lśedzenia & śledzenia \\
    % & & & & \\
    % & & & & \\
    % & & & & \\
    % & & & & \\
    % & & & & \\
    % & & & & \\
    % & & & & \\
    % & & & & \\
    & & & & \\ \hline
  \end{tabular}

\end{center}

\VerSpaceTwo




% ############################










% ####################################################################
% ####################################################################
% Bibliography

\printbibliography





% ############################
% End of the document

\end{document}
