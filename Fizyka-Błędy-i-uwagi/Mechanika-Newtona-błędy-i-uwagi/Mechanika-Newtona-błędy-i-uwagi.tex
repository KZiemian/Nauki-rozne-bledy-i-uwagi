% ---------------------------------------------------------------------
% Basic configuration and packages
% ---------------------------------------------------------------------
% Package for discovering wrong and outdated usage of LaTeX.
% More information to be found in l2tabu English version.
\RequirePackage[l2tabu, orthodox]{nag}
% Class of LaTeX document: {size of paper, size of font}[document class]
\documentclass[a4paper,11pt]{article}



% ---------------------------------------
% Packages not tied to particular normal language
% ---------------------------------------
% This package should improved spaces in the text.
\usepackage{microtype}
% Add few important symbols, like text Celcius degree
\usepackage{textcomp}



% ---------------------------------------
% Polonization of LaTeX document
% ---------------------------------------
% Basic polonization of the text
\usepackage[MeX]{polski}
% Switching on UTF-8 encoding
\usepackage[utf8]{inputenc}
% Adding font Latin Modern
\usepackage{lmodern}
% Package is need for fonts Latin Modern
\usepackage[T1]{fontenc}



% ---------------------------------------
% Setting margins
% ---------------------------------------
\usepackage[a4paper, total={14cm, 25cm}]{geometry}
% Package for easy settings of margins. Unit of measurement is inch.
% \usepackage{vmargin}
% \setmarginsrb
% { 0.7in}  % left margin
% { 0.6in}  % top margin
% { 0.7in}  % right margin
% { 0.8in}  % bottom margin
% {  20pt}  % head height
% {0.25in}  % head sep
% {   9pt}  % foot height
% { 0.3in}  % foot sep



% ---------------------------------------
% Setting vertical spaces in the text
% ---------------------------------------
% Setting space between lines
\renewcommand{\baselinestretch}{1.1}

% Setting space between lines in tables
\renewcommand{\arraystretch}{1.4}



% ---------------------------------------
% Packages for scientific papers
% ---------------------------------------
% Switching off \lll symbol, that I guess is representing letter ``Ł''.
% It collide with `amsmath' package's command with the same name
\let\lll\undefined
% Basic package from American Mathematical Society (AMS)
\usepackage[intlimits]{amsmath}
% Equations are numbered separately in every section.
\numberwithin{equation}{section}

% Other very useful packages from AMS
\usepackage{amsfonts, amssymb, amscd, amsthm}

% Package with support of physical units
\usepackage{siunitx}

% Package with better looking calligraphy fonts
\usepackage{calrsfs}

% Package with better looking greek letters
% Example of use: pi -> \uppi
\usepackage{upgreek}
% Improving look of lambda letter
\let\oldlambda\Lambda
\renewcommand{\lambda}{\uplambda}





% ---------------------------------------
% Defining new environments (?)
% ---------------------------------------
% Defining enviroment ``Wniosek''
\newtheorem{corollary}{Wniosek}
\newtheorem{definition}{Definicja}
\newtheorem{theorem}{Twierdzenie}





% ------------------------------
% Private packages
% You need to put them in the same directory as .tex file
% ------------------------------
% Contains various command useful for working with a text
\usepackage{latexgeneralcommands}
% Contains definitions useful for working with mathematical text
\usepackage{mathcommands}





% ------------------------------
% Package ``hyperref''
% They advised to put it on the end of preambule
% ------------------------------
% It allows you to use hyperlinks in the text
\usepackage{hyperref}










% ---------------------------------------------------------------------
% Tytuł, autor, data
\title{Mechanika Newtona \\
  {\Large Błędy i~uwagi}}

\author{Kamil Ziemian}


% \date{}
% ---------------------------------------------------------------------










% ####################################################################
\begin{document}
% ####################################################################





% ######################################
\maketitle % Tytuł całego tekstu
% ######################################





% ######################################
\section{Rozważania ogólne}
% Tytuł danego działu
% ######################################



Omawiają mechanikę Newtona warto przedyskutować następujące zagadnienie.
Występujące w~fizyce wielkości zwykle nie są reprezentowane po prostu
liczbami, lecz liczbami posiadającymi jednostki fizycznie, lub jak to
często mówimy, wymiar fizyczny. Przykładowo wielkość fizyczna może mieć
wymiar sekundy, metra, metra na sekundę,~etc. Fakt ten pozwala
nam pisać równania takie jak
\begin{equation}
  \label{eq:Mechanika-Rozwazania-ogolne-01}
  60 \, \si{s} = 1 \, \si{min},
\end{equation}
gdzie $\si{s}$ oznacza sekundę, a~$\si{min}$ minutę, mimo że ewidentnie
$60$ nie jest równe $1$. Powstaje więc pytanie, jak można matematycznie
sformalizować pojęcie liczby posiadającej jednostki fizyczne? Dla większości
fizyków zajmowanie~się tym byłoby zapewne stratą czasu, gdyż „jest
oczywiste” jak rozumieć wielkość fizyczną posiadającą odpowiedni wymiar.
Jesteśmy jednak przekonani, że~od strony formalizmu matematycznego problem
ten jest interesując, dlatego warto~się nim zająć.

Przedstawimy teraz pewien sposób matematycznego formalizmu wyjaśniającego
ten problem. Zapewne nie jest on optymalny, powinien jednak wystarczyć na
początek. Co należy zaznaczyć, przedstawiona formalizm nie bierze pod uwagę
dwóch ważnych problemów. Po~pierwsze, czy wielkości obecne którymi
operuje~się w~fizyce powinny być reprezentowane zbiorem liczb rzeczywistych
$\Rbb$, czy jakimś innym? Przykładowo, czy dla zastosowań fizycznych
wystarczający jest zbiór liczb wymiernych $\Qbb$?

Drugi ważne pytanie powiązane jest z~pierwszym. Jeśli przyjmiemy,
że~opis jednostek fizycznych ma~się opierać na~liczbach rzeczywistych,
to czy do reprezentowania każdej wielkości fizycznej potrzebny jest cały
zbiór $\Rbb$. Czy może dla niektórych wielkości wystarczający jest np.
zbiór liczb rzeczywistych dodatnich:
$\Rbb_{ + } = \{ x \in \Rbb \, | \, x > 0 \}$? Można bowiem argumentować,
że~wielkości takie jak masa obiektu fizycznego powinny być zawsze dodatnie,
$m > 0$, a~temperatura zawsze większa od zera bezwzględnego, czyli zbiór
liczb rzeczywistych rozciągający~się od minus do plus nieskończoności jest
zbyt duży w~stosunku do naszych realnych potrzeb.

Choć oba te problemy mają swoją wagę, nie będziemy~się wdawać tu w~trudną
dyskusję na ich temat. Dla prostoty przyjmiemy, że~wszystkie wielkości
fizyczne powinny być modelowane na pełnym zbiorze $\Rbb$ i~bazując
na tym założeniu podamy pewną jedną konkretną formalizację liczb
rzeczywistych posiadających jednostki.

W~naszym podejściu przyjmujemy, że~wszystkie rozważane jednostki
opierają~się na trzech podstawowych wielkościach: mierze długości,
mierze czasu i~mierze masy. Uogólnienie podanych tu konstrukcji na
przypadek systemu z~większą ilością podstawowych wielkości nie powinno
stanowić wielkiego problemu. Zaczynamy więc od wprowadzenia trzech symboli
$L$, $T$ i~$M$, które oznaczają odpowiednio koncepcje długości, czasu
i~masy. Podkreślamy, że~chodzi tu o~koncepcje, nie konkretne wielkości
fizyczne. W~szczególności ponieważ $L$ oznacza koncepcję długości, to nie
przypisujemy temu symbolowi, ani wielkości liczbowej, ani jednostek.
Powinno być dość jasne, że~można sensownie mówić o~długości stołu, ale nie
ma sensu pytać~się o~to czy sama koncepcja długości ma długość
$30 \, \si{cm}$ czy nie\footnote{Dla rozwiania potencjalnych wątpliwości,
  można oczywiście pytać~się o~długość symbolu „$L$”, ale to co innego niż
  pytanie~się o~długość tego co symbol ten oznacza.}.

Rozważmy teraz zbiór\footnote{Od ang. \textit{units}.} czterech różnych
symboli
\begin{equation}
  \label{eq:Mechanika-Rozwazania-ogolne-02}
  U_{ 0 } = \{ I, L, T, M \}.
\end{equation}
i~utwórzmy grupę przemienną generowaną przez ten zbiór, przy czym $I$ ma
być jej elementem neutralnym. Grupę tą oznaczmy $U$. Tutaj wchodzimy
w~jeden z~bardziej wątpliwych punktów całej konstrukcji, gdyż nie jestem
pewien, czy dobrze korzystamy z~konstrukcji grupy generowanej przez
dowolny, skończony zbiór. W~tej jednak chwili nie widzę powodu, dla
którego ta przeprowadzona w~trochę naiwny sposób konstrukcja miałaby
być błędna.

Grupa $G$ składa~się ze wszystkich możliwych, skończonych kombinacji
elementów $L$, $T$, $M$ i~elementów do nich odwrotnych $L^{ -1 }$, $T^{ -1 }$
i~$M^{ -1 }$ oraz elementu neutralnego, który oznaczamy symbolem~$I$.
Zawiera więc ona przykładowo elementy
\begin{equation}
  \label{eq:Mechanika-Rozwazania-ogolne-03}
  \begin{split}
    &L, L^{ 2 }, L^{ 3 }, L^{ 4 }, L^{ -1 }, L^{ -2 }, L^{ -3 }, L^{ -4 },
      T^{ \HorSpaceTwo 1 }, T^{ \HorSpaceTwo 2 }, T^{ \HorSpaceTwo 3 },
      T^{ \HorSpaceThree 4 }, T^{ \HorSpaceTwo -1 }, T^{ \HorSpaceTwo -2 },
      T^{ \HorSpaceTwo -3 }, T^{ \HorSpaceTwo -4 }, \\
    &M^{ 1 }, M^{ 2 }, M^{ 3 }, M^{ \HorSpaceOne -1 }, M^{ \HorSpaceOne -2 },
      M^{ \HorSpaceOne -3 }, L T, L T^{ \HorSpaceTwo -1 }, M L, M L^{ -1 },
      M T^{ -2 }, \ldots
  \end{split}
\end{equation}
Wprowadzając notację
\begin{equation}
  \label{eq:Mechanika-Rozwazania-ogolne-04}
  u^{ -1 } = \frac{ 1 }{ u } = 1 / u, \quad u \in U,
\end{equation}
oraz analogiczne jak dla ułamków wymiernych prawa operowania ułamkami
utworzonymi z~elementów grupy $U$, możemy zapisać niektóre wielkości
występujące w~\eqref{eq:Mechanika-Rozwazania-ogolne-03} w~bardziej znajomej
postaci:
\begin{equation}
  \label{eq:Mechanika-Rozwazania-ogolne-05}
  L T^{ -1 } = \frac{ L }{ T }, \quad
  M L^{ -1 } = \frac{ M }{ L }, \quad
  M L^{ -2 } = \frac{ M }{ L^{ 2 } }, \quad \ldots
\end{equation}
Każdemu elementowi $u \in U$ przyporządkowujemy jednowymiarową przestrzeń
wektorową nad $\Rbb$, którą będziemy oznaczać jako $V_{ u }$. Zbiór
wszystkim tych przestrzeni wektorowych będziemy oznaczać przez
$S_{ \HorSpaceOne U }$. W~przestrzeni $V_{ I }$ wyróżniamy jeden konkretny
wektor, którego używamy jako bazy tej przestrzeni, będziemy go oznaczać
przez $e_{ I }$. Pozwala to nam zdefiniować kanoniczny izomorfizm $V_{ I }$
z~$\Rbb$:
\begin{equation}
  \label{eq:Mechanika-Rozwazania-ogolne-06}
  \omega : V_{ I } \ni v = \alpha \HorSpaceOne e_{ I } \mapsto \alpha \in \Rbb.
\end{equation}
Tym samym możemy uznać, iż~$V_{ I }$ reprezentuje przestrzeń wielkości
bezwymiarowych.

Gdy określona jest ogólna struktura bytów matematycznych którymi
będziemy~się posługiwać, potrzebujemy teraz wprowadzić w~ich zbiorze
odpowiednie działania, które odpowiadają dobrze znanym operacjom
wykonywanym na~jednostkach fizycznych. Niech teraz będą dane
$u_{ \HorSpaceOne 1 }, u_{ \HorSpaceOne 2 } \in U$. Każdej parze przestrzeni
$V_{ u_{ 1 } }$, $V_{ u_{ \HorSpaceOne 2 } }$ przyporządkowujemy pewien rodzaj
iloczynu, który będziemy oznaczać
$\psi_{ \HorSpaceOne u_{ \HorSpaceOne 1 }, \, u_{ \HorSpaceOne 2 } }$.
\begin{equation}
  \label{eq:Mechanika-Rozwazania-ogolne-07}
  \psi_{ \HorSpaceOne u_{ \HorSpaceOne 1 }, \, u_{ \HorSpaceOne 2 } } :
  V_{ u_{ 1 } } \times V_{ u_{ 2 } } \to V_{ u_{ 1 } u_{ 2 } },
\end{equation}
gdzie $V_{ u_{ \HorSpaceOne 1 } u _{ \HorSpaceOne 2 } }$ oznacza przestrzeń
wektorową odpowiadającą iloczynowi elementów
$u_{ \HorSpaceOne 1 } u_{ \HorSpaceOne 2 } \in U$.
Od~$\psi_{ \HorSpaceOne u_{ \HorSpaceOne 1 }, \, u_{ \HorSpaceOne 2 } }$ żądamy był on
odwzorowaniem dwuliniowym. Wobec tego odwzorowania te~są w~pełni
określone, jeśli w~każdej przestrzeni $V_{ u }$ wybierzemy jeden
niezerowy wektor, będziemy go oznaczać $v_{ \HorSpaceOne u } \neq 0$,
i~położymy
\begin{equation}
  \label{eq:Mechanika-Rozwazania-ogolne-08}
  \psi_{ \HorSpaceTwo u_{ 1 }, \, u_{ \HorSpaceOne 2 } }(
  v_{ \HorSpaceOne u_{ \HorSpaceOne 1 } }, v_{ u_{ \HorSpaceOne 2 } } ) =
  v_{ \HorSpaceOne u_{ 1 } u_{ \HorSpaceOne 2 } }.
\end{equation}
Konkretny wybór wektorów $v_{ \HorSpaceOne u }$ nie ma znaczenia dla
matematycznej strony formalizmu, nie wydaje~się też mieć znaczenia dla
jego fizycznej fizycznej interpretacji, dlatego nie będziemy~się
nad tym problemem dłużej zatrzymywać.

Rozpatrzmy teraz przestrzeń $V_{ L }$. Wybór konkretnego wektora bazowego
$e_{ L }$ w~tej przestrzeni będziemy rozumieć jako wybór konkretnej
jednostki długości. Przykładowo, jeśli zdecydowaliśmy~się mierzyć
długość w~metrach, to wybrany dla tego przypadku wektor $e_{ L }$
reprezentuje fakt, że~wszystkie długości będziemy od tej pory mierzyć
w~metrach. Dlatego też wektor ten będziemy oznaczać symbolem
$e_{ \HorSpaceTwo \si{m} }$.

Gdy to uczynimy, to tracimy możliwość arbitralnego wyboru wektorów bazowych,
reprezentujących inne jednostki długości. Przykład, ponieważ jak dobrze
wiadomo centymetr jest \textit{zdefiniowany} jako jedna setna metra, więc
odpowiadający mu wektor bazowy dany jest jako:
\begin{equation}
  \label{eq:Mechanika-Rozwazania-ogolne-09}
  e_{ \HorSpaceOne \si{cm} } = \frac{ 1 }{ 100 } e_{ \HorSpaceTwo \si{m} }.
\end{equation}
Analogiczna sytuacja zachodzi dla przestrzeni $V_{ \HorSpaceOne T }$ jeśli
ustaliliśmy jednostkę czasu na, przykładowo, sekundę, a~jako wektor bazowy
tej przestrzeni wybraliśmy $e_{ \HorSpaceOne \si{s} }$. Dla masy i~ewentualnie
innych jednostek fizyczny przedstawiona procedura również przebiega w~ten
sam sposób.

Załóżmy, że~wybraliśmy już wektory bazowe dla podstawowych jednostek
fizycznych: $e_{ \HorSpaceOne L } \in V_{ L }$,
$e_{ \HorSpaceOne T } \in V_{ \HorSpaceOne T }$, $e_{ M } \in V_{ M }$.
Wówczas wektor $e_{ \HorSpaceOne u, \, LTM }$ tworzący bazę przestrzeni
$V_{ u }$, $u \in U$, która to baza odpowiada mierzeniu wielkości fizycznych
w~wybranym układzie jednostek, jest już jednoznacznie wyznaczony przez
zależność
\begin{equation}
  \label{eq:Mechanika-Rozwazania-ogolne-10}
  e_{ \HorSpaceTwo u_{ \HorSpaceOne 1 } u_{ \HorSpaceOne 2 }, \, LTM } :=
  \psi_{ \HorSpaceTwo u_{ \HorSpaceOne 1 }, \, u_{ \HorSpaceOne 2 } }(
  e_{ \HorSpaceTwo u_{ \HorSpaceOne 1 } }, e_{ \HorSpaceTwo u_{  \HorSpaceOne 2 } } ).
\end{equation}
Przykładowo, wektor bazowy przestrzeni $V_{ L T }$ jest wyznaczony przez
zależność
\begin{equation}
  \label{eq:Mechanika-Rozwazania-ogolne-11}
  e_{ \HorSpaceOne L T, \, LTM } :=
  \psi_{ L, \, T }( e_{ L, \, LTM }, e_{ \HorSpaceOne T, \, LTM } ).
\end{equation}
Ponieważ odwzorowanie
$\psi_{ \HorSpaceOne u_{ \HorSpaceOne 1 }, \, u_{ \HorSpaceOne 2 } }$ zostało zdefiniowane
za~pomocą wzoru \eqref{eq:Mechanika-Rozwazania-ogolne-08}, więc wektor
$e_{ \HorSpaceOne u, \, LTM } \in V_{ u }$ zdefiniowany wzorem
\eqref{eq:Mechanika-Rozwazania-ogolne-10} nie zależy od~wyboru
$u_{ 1 }, u_{ \HorSpaceOne 2 } \in U$, takich~że
$u = u_{ 1 } u_{ \HorSpaceOne 2 }$.

W~dalszym ciągu, by uniknąć uciążliwej notacji, będziemy pomijali symbol
$LTM$ w~indeksie dolnym wektora $e_{ \HorSpaceTwo u, \, LTM }$ i~zapisywać
go po prostu jako $e_{ \HorSpaceTwo u }$. Należy jednak pamiętać, że~wektor
ten zależy od~wyboru bazy w~przestrzeniach $V_{ L }$,
$V_{ \HorSpaceOne T }$ i~$V_{ M }$.

Pozwolimy sobie teraz na mała techniczną dygresję. Powyższe rozumowanie
moglibyśmy odwrócić w~następującym sensie. Najpierw wybieramy wektory
bazowe przestrzeni $e_{ L } \in V_{ L }$,
$e_{ \HorSpaceOne T } \in V_{ \HorSpaceOne T }$
i~$e_{ M } \in V_{ M }$, następnie zaś~w~sposób arbitralny wybieramy wektory
$v_{ \HorSpaceOne u } \in V_{ u }$ dla wszystkich pozostałych przestrzeni
i~przyjmujemy, że~tworzą one bazy tych przestrzeni dla układu jednostek
określonego przez $e_{ L }$, $e_{ \HorSpaceOne T }$ i~$e_{ M }$, czyli
\begin{equation}
  \label{eq:Mechanika-Rozwazania-ogolne-12}
  e_{ \HorSpaceOne u } := v_{ \HorSpaceOne u }.
\end{equation}
W~ostatnim zaś kroku określamy odwzorowanie
$\psi_{ \HorSpaceOne u_{ 1 }, \, u_{ \HorSpaceOne 2 } }$ za pomocą wzoru
\begin{equation}
  \label{eq:Mechanika-Rozwazania-ogolne-13}
  \psi_{ \HorSpaceOne u_{ 1 }, \, u_{ \HorSpaceOne 2 } }(
  e_{ \HorSpaceTwo u_{ 1 } }, e_{ \HorSpaceTwo u_{ \HorSpaceOne 2 } } ) =
  e_{ \HorSpaceTwo u_{ 1 } u_{ \HorSpaceOne 2 } }.
\end{equation}

Wybór tego czy najpierw definiujemy odwzorowania
$\psi_{ \HorSpaceOne u_{ 1 }, \, u_{ \HorSpaceOne 2 } }$ za pomocą rodziny wektorów
$u \in V_{ u }$, a~potem za jego pomocą bazę
w~$V_{ u_{ 1 } u_{ \HorSpaceOne 2 } }$, czy na odwrót, nie wpływa w~żaden
realny sposób na~przedstawiony formalizm. Można więc wybrać jedną z~tych
dróg wedle uznania.

Zapisywanie iloczynu wektorów za pomocą funkcji
$\psi_{ \HorSpaceOne u_{ 1 }, \, u_{ \HorSpaceOne 2 } }$ prowadzi do powstania
bardzo precyzyjnych wzorów, z~których można dokładnie odczytać jakie
operacje są wykonywane. Ten sposób zapisu ma jednak tą wadę, że~jest ona
dość trudna w~czytaniu, dlatego wprowadzimy alternatywną notację, którą
będziemy nazywać arytmetyczną. Notacja ta jest doskonale znany każdemu kto
zapoznał~się z~elementarnym wprowadzeniem do~fizyki. Mianowicie będziemy
pisać
\begin{equation}
  \label{eq:Mechanika-Rozwazania-ogolne-14}
  v_{ \HorSpaceOne 1 } v_{ \HorSpaceTwo 2 } :=
  \psi_{ \HorSpaceOne u_{ 1 }, \, u_{ \HorSpaceOne 2 } }(
  v_{ \HorSpaceOne 1 }, v_{ \HorSpaceTwo 2 } ).
\end{equation}

Wprowadzimy teraz operację dzielenia wektora z~przestrzeni
$V_{ u_{ \HorSpaceOne 1 } }$ przez niezerowy element należący do
$V_{ u_{ \HorSpaceOne 2 } }$:
\begin{equation}
  \label{eq:Mechanika-Rozwazania-ogolne-15}
  \varphi_{ \HorSpaceTwo u_{ \HorSpaceOne 1 }, \, u_{ \HorSpaceOne 2 } } :
  V_{ u_{ 1 } } \times ( V_{ u_{ \HorSpaceOne 2 } } \setminus \{ 0 \} ) \to
  V_{ u_{ 1 } / u_{ \HorSpaceOne 2 } }.
\end{equation}
Od funkcji $\varphi_{ \HorSpaceOne u_{ \HorSpaceOne 1 }, \, u_{ \HorSpaceOne 2 } }$ żądamy
by posiadała następujące własności:
\begin{subequations}
  \begin{align}
    \label{eq:Mechanika-Rozwazania-ogolne-16-A}
    \varphi_{ \HorSpaceTwo u_{ \HorSpaceOne 1 }, \, u_{ \HorSpaceOne 2 } }(
    \alpha \HorSpaceOne v_{ \HorSpaceOne 1 } +
    \beta \HorSpaceOne v_{ \HorSpaceOne 2 }, w )
    &= \alpha \HorSpaceOne \varphi_{ \HorSpaceTwo u_{ \HorSpaceOne 1 }, \,
      u_{ \HorSpaceOne 2 } }( v_{ \HorSpaceOne 1 }, w ) +
      \beta \HorSpaceOne \varphi_{ \HorSpaceTwo u_{ \HorSpaceOne 1 }, \,
      u_{ \HorSpaceOne 2 } }( v_{ \HorSpaceOne 2 }, w ), \\[0.2em]
    \label{eq:Mechanika-Rozwazania-ogolne-16-B}
    \varphi_{ \HorSpaceTwo u_{ \HorSpaceOne 1 }, \, u_{ \HorSpaceOne 2 } }( v,
    \lambda \HorSpaceOne w )
    &= \frac{ 1 }{ \lambda } \HorSpaceOne
      \varphi_{ \HorSpaceTwo u_{ \HorSpaceTwo 1 }, \, u_{ \HorSpaceOne 2 } }( v,
      w ), \\[0.2em]
    \label{eq:Mechanika-Rozwazania-ogolne-16-C}
    \varphi_{ \HorSpaceTwo u, \, u }( v_{ \HorSpaceOne u }, v_{ \HorSpaceOne u } )
    &= I \in V_{ I },
  \end{align}
\end{subequations}
gdzie $u, u_{ \HorSpaceOne 1 }, u_{ \HorSpaceOne 2 } \in U$,
$v_{ \HorSpaceOne u } \in V_{ u }$,
$v, v_{ \HorSpaceOne 1 }, v_{ \HorSpaceOne 2 } \in V_{ u_{ \HorSpaceOne 1 } }$,
$w \in V_{ u_{ \HorSpaceOne 2 } }$, $w \neq 0$,
$\alpha, \beta, \lambda \in \Rbb$ i~$\lambda \neq 0$. Niech $v_{ \HorSpaceOne } \in V_{ u }$ będzie
wybraną uprzednio rodziną baz tych przestrzeni. Tak jak dla
$\psi_{ \HorSpaceTwo u_{ 1 }, \, u_{ \HorSpaceTwo 2 }}$ żądamy by zachodził związek
\begin{equation}
  \label{eq:Mechanika-Rozwazania-ogolne-17}
  \varphi_{ \HorSpaceTwo u_{ \HorSpaceOne 1 }, \, u_{ \HorSpaceOne 2 } }(
  v_{ u_{ \HorSpaceOne 1 } },
  v_{ u_{ \HorSpaceOne 2 } } ) =
  v_{ u_{ \HorSpaceOne 1 } / u_{ \HorSpaceOne 2 } },
\end{equation}
lub równoważnie
\begin{equation}
  \label{eq:Mechanika-Rozwazania-ogolne-18}
  \varphi_{ \HorSpaceTwo u_{ \HorSpaceOne 1 }, \, u_{ \HorSpaceOne 2 } }(
  e_{ \HorSpaceOne u_{ \HorSpaceOne 1 } },
  e_{ \HorSpaceTwo u_{ \HorSpaceOne 2 } } ) =
  e_{ \HorSpaceTwo u_{ \HorSpaceOne 1 } / u_{ \HorSpaceOne 2 } }.
\end{equation}

Tak jak dla $\psi_{ \HorSpaceOne u_{ 1 }, \, u_{ \HorSpaceOne 2 } }$ wprowadzamy
uproszczoną notację dla działania
$\varphi_{ \HorSpaceTwo u_{ \HorSpaceOne 1 }, \, u_{ \HorSpaceOne 2 } }$. Zacznijmy
od~przypadku, gdy jedna z~przestrzeni $V_{ u_{ \HorSpaceOne 1 } }$,
$V_{ u_{ \HorSpaceOne 2 } }$ jest równa $V_{ I } = \Rbb$. Wówczas dla
$\alpha \in V_{ I }$ i~$v \in V_{ u }$ będziemy pisali
\begin{equation}
  \label{eq:Mechanika-Rozwazania-ogolne-19}
  \frac{ \alpha }{ v } := \varphi_{ \HorSpaceOne I, \, u }( \alpha, v ), \qquad
  \frac{ 1 }{ \alpha } v := \varphi_{ \HorSpaceOne u, \, I }( v, \alpha ).
\end{equation}



Przystąpimy teraz do udowodnienia własności odwzorowań
$\psi_{ \HorSpaceTwo u_{ \HorSpaceOne 1 }, \, u_{ \HorSpaceOne 2 } }$
i~$\varphi_{ \HorSpaceTwo u_{ \HorSpaceOne 1 }, \, u_{ \HorSpaceOne 2 } }$, które pozwalają
stwierdzić, że~zachowują~się one jak mnożenie i~dzielenie liczb
rzeczywistych. W~pierwszym rzędzie zbadamy własność, która jest analogiczna
do~łączności mnożenia w~ciele liczb rzeczywistych:
\begin{equation}
  \label{eq:Mechanika-Rozwazania-ogolne-20}
  \psi_{ \HorSpaceTwo u_{ \HorSpaceOne 1 } u_{ \HorSpaceOne 2 }, \,
    u_{ \HorSpaceOne 3 } }\big(
  \psi_{ \HorSpaceTwo u_{ \HorSpaceOne 1 }, \, u_{ \HorSpaceOne 2 } }(
  v_{ \HorSpaceOne 1 }, v_{ \HorSpaceOne 2 } ), v_{ \HorSpaceOne 3 } \big) =
  \psi_{ \HorSpaceTwo u_{ \HorSpaceOne 1 }, \, u_{ \HorSpaceOne 2 }
    u_{ \HorSpaceOne 3 } }\big( v_{ \HorSpaceOne 1 },
  \psi_{ \HorSpaceTwo u_{ \HorSpaceOne 2 }, u_{ \HorSpaceOne 3 } }(
  v_{ \HorSpaceOne 2 }, v_{ \HorSpaceOne 3 } ) \big),
\end{equation}
gdzie
$u_{ \HorSpaceOne 1 }, u_{ \HorSpaceOne 2 }, u_{ \HorSpaceOne 3 } \in U$,
$v_{ \HorSpaceOne 1 } \in V_{ u_{ \HorSpaceOne 1 } }$,
$v_{ \HorSpaceOne 2 } \in V_{ u_{ \HorSpaceOne 2 } }$,
$v_{ \HorSpaceOne 3 } \in V_{ u_{ \HorSpaceOne 3 } }$. Do zapisania ostatniej
tożsamości nie użyliśmy notacji arytmetycznej, gdyż ta korzystając
z~funkcji $\psi_{ \HorSpaceTwo u_{ \HorSpaceOne 1 }, \, u_{ \HorSpaceOne 2 } }$ wydobywa
na jaw dokładny sens występujących w~niej operacji. Nie da~się jednak ukryć,
że~jej używamy, to główna przeszkodą w~zrozumieniu dowodu własności
\eqref{eq:Mechanika-Rozwazania-ogolne-19} staje~się zrozumieniu co dokładnie
konkretne symbole oznaczają.

Dowód zależności \eqref{eq:Mechanika-Rozwazania-ogolne-19} zaczniemy od
zauważenia, że~dla pewnych $\alpha, \beta, \gamma \in \Rbb$ zachodzi
$v_{ \HorSpaceOne 1 } =
\alpha \HorSpaceOne e_{ \HorSpaceTwo u_{ \HorSpaceOne 1 } }$,
$v_{ \HorSpaceOne 2 } =
\beta \HorSpaceOne e_{ \HorSpaceTwo u_{ \HorSpaceOne 2 } }$,
$v_{ \HorSpaceOne 3 } =
\gamma \HorSpaceOne e_{ \HorSpaceTwo u_{ \HorSpaceOne 3 } }$. Mamy teraz

\negVerSpaceTwo


\begin{subequations}

  \begin{equation}
    \label{eq:Mechanika-Rozwazania-ogolne-21-A}
    \begin{split}
      \psi_{ \HorSpaceTwo u_{ \HorSpaceOne 1 } u_{ \HorSpaceOne 2 }, \,
      u_{ \HorSpaceOne 3 } }\big(
      \psi_{ \HorSpaceTwo u_{ \HorSpaceOne 1 }, \, u_{ \HorSpaceOne 2 } }(
      v_{ \HorSpaceOne 1 }, v_{ \HorSpaceOne 2 } ), v_{ \HorSpaceOne 3 } \big)
      &=
        \psi_{ \HorSpaceTwo u_{ \HorSpaceOne 1 } u_{ \HorSpaceOne 2 }, \,
        u_{ \HorSpaceOne 3 } }\big(
        \psi_{ \HorSpaceTwo u_{ \HorSpaceOne 1 }, \, u_{ \HorSpaceOne 2 } }(
        \alpha \HorSpaceOne e_{ \HorSpaceTwo u_{ \HorSpaceOne 1 } },
        \beta \HorSpaceOne e_{ \HorSpaceTwo u_{ \HorSpaceOne 2 } } ),
        \gamma \HorSpaceOne e_{ \HorSpaceTwo u_{ \HorSpaceOne 3 } } \big) = \\[0.3em]
      &=
        \alpha \beta \gamma \HorSpaceOne
        \psi_{ \HorSpaceTwo u_{ \HorSpaceOne 1 } u_{ \HorSpaceOne 2 }, \,
        u_{ \HorSpaceOne 3 } }\big(
        \psi_{ \HorSpaceTwo u_{ \HorSpaceOne 1 }, \, u_{ \HorSpaceOne 2 } }(
        e_{ \HorSpaceTwo u_{ \HorSpaceOne 1 } },
        e_{ \HorSpaceTwo u_{ \HorSpaceOne 2 } } ),
        e_{ \HorSpaceTwo u_{ \HorSpaceOne 3 } } \big),
    \end{split}
  \end{equation}

  \negVerSpaceTwo



  \begin{equation}
    \label{eq:Mechanika-Rozwazania-ogolne-21-B}
    \begin{split}
      \psi_{ \HorSpaceTwo u_{ \HorSpaceOne 1 }, \, u_{ \HorSpaceOne 2 }
      u_{ \HorSpaceOne 3 } }\big(
      e_{ \HorSpaceTwo u_{ \HorSpaceOne 1 } },
      \psi_{ \HorSpaceTwo u_{ \HorSpaceOne 2 }, \, u_{ \HorSpaceOne 3 } }(
      e_{ \HorSpaceTwo u_{ \HorSpaceOne 2 } }, e_{ u_{ \HorSpaceOne 3 } } ) \big)
      &=
        \psi_{ \HorSpaceTwo u_{ \HorSpaceOne 1 }, \,
        u_{ \HorSpaceOne 2 } u_{ \HorSpaceOne 3 } }\big(
        \alpha \HorSpaceOne e_{ \HorSpaceTwo u_{ \HorSpaceOne 1 } },
        \psi_{ \HorSpaceTwo u_{ \HorSpaceOne 2 }, \, u_{ \HorSpaceOne 3 } }(
        \beta \HorSpaceOne e_{ \HorSpaceTwo u_{ \HorSpaceOne 2 } },
        \gamma \HorSpaceOne e_{ \HorSpaceTwo u_{ \HorSpaceOne 3 } } )
        \big) = \\[0.3em]
      &=
        \alpha \beta \gamma \HorSpaceOne \psi_{ \HorSpaceTwo u_{ \HorSpaceOne 1 }, \,
        u_{ \HorSpaceOne 2 } u_{ \HorSpaceOne 3 } }\big(
        e_{ \HorSpaceTwo u_{ \HorSpaceOne 1 } },
        \psi_{ \HorSpaceTwo u_{ \HorSpaceOne 2 }, \,
        u_{ \HorSpaceOne 3 } }( e_{ \HorSpaceTwo u_{ \HorSpaceOne 2 } },
        e_{ \HorSpaceTwo u_{ \HorSpaceOne 3 } } ) \big).
    \end{split}
  \end{equation}

\end{subequations}


\noindent
Widzimy więc, że~aby zakończyć dowód tej własności wystarczy wykazać
równości
\begin{equation}
  \label{eq:Mechanika-Rozwazania-ogolne-22}
  \psi_{ \HorSpaceTwo u_{ \HorSpaceOne 1 } u_{ \HorSpaceOne 2 }, \,
    u_{ \HorSpaceOne 3 } }\big(
  \psi_{ \HorSpaceTwo u_{ \HorSpaceOne 1 }, \, u_{ \HorSpaceOne 2 } }(
  e_{ \HorSpaceTwo u_{ \HorSpaceOne 1 } }, e_{ \HorSpaceTwo u_{ \HorSpaceOne 2 } } ),
  e_{ \HorSpaceTwo u_{ \HorSpaceOne 3 } } \big) =
  \psi_{ \HorSpaceTwo u_{ \HorSpaceOne 1 }, \,
    u_{ \HorSpaceOne 2 } u_{ \HorSpaceOne 3 } }\big(
  e_{ \HorSpaceTwo u_{ \HorSpaceOne 1 } },
  \psi_{ \HorSpaceTwo u_{ \HorSpaceOne 2 }, \, u_{ \HorSpaceOne 3 } }(
  e_{ \HorSpaceTwo u_{ \HorSpaceOne 2 } }, e_{ \HorSpaceTwo u_{ \HorSpaceOne 3 } } )
  \big).
\end{equation}
Lewa strona tej zależności wynosi
\begin{equation}
  \label{eq:Mechanika-Rozwazania-ogolne-23}
  \psi_{ \HorSpaceTwo u_{ \HorSpaceOne 1 } u_{ \HorSpaceOne 2 }, \,
    u_{ \HorSpaceOne 3 } }\big(
  \psi_{ \HorSpaceTwo u_{ \HorSpaceOne 1 }, \, u_{ \HorSpaceOne 2 } }(
  e_{ \HorSpaceTwo u_{ \HorSpaceOne 1 } },
  e_{ \HorSpaceTwo u_{ \HorSpaceOne 2 } } ),
  e_{ \HorSpaceTwo u_{ \HorSpaceOne 3 } } \big) =
  \psi_{ \HorSpaceTwo u_{ \HorSpaceOne 1 } u_{ \HorSpaceOne 2 }, \,
    u_{ \HorSpaceOne 3 } }\big(
  e_{ \HorSpaceTwo u_{ \HorSpaceOne 1 } u_{ \HorSpaceOne 2 } },
  e_{ \HorSpaceTwo u_{ \HorSpaceOne 3 } } \big) =
  e_{ \HorSpaceTwo ( u_{ \HorSpaceOne 1 } u_{ \HorSpaceOne 2 } )
    u_{ \HorSpaceOne 3 } },
\end{equation}
natomiast prawa
\begin{equation}
  \label{eq:Mechanika-Rozwazania-ogolne-24}
  \psi_{ \HorSpaceTwo u_{ \HorSpaceOne 1 }, \,
    u_{ \HorSpaceOne 2 } u_{ \HorSpaceOne 3 } }\big(
  e_{ \HorSpaceTwo u_{ \HorSpaceOne 1 } },
  \psi_{ \HorSpaceTwo u_{ \HorSpaceOne 2 }, \, u_{ \HorSpaceOne 3 } }(
  e_{ \HorSpaceTwo u_{ \HorSpaceOne 2 } }, e_{ \HorSpaceTwo u_{ \HorSpaceOne 3 } } )
  \big) =
  \psi_{ \HorSpaceTwo u_{ \HorSpaceOne 1 }, \,
    u_{ \HorSpaceOne 2 } u_{ \HorSpaceOne 3 } }(
  e_{ \HorSpaceTwo u_{ \HorSpaceOne 1 } },
  e_{ \HorSpaceTwo u_{ \HorSpaceOne 2 } u_{ \HorSpaceOne 3 } } ) =
  e_{ \HorSpaceTwo u_{ \HorSpaceOne 1 } ( u_{ \HorSpaceOne 2 }
    u_{ \HorSpaceOne 3 } ) }.
\end{equation}
Tym samym równość \eqref{eq:Mechanika-Rozwazania-ogolne-20} wynika
z~łączności mnożenia w~grupie $U$, to~zaś dowodzi tożsamości
\eqref{eq:Mechanika-Rozwazania-ogolne-18}. Korzystając teraz z~notacji
arytmetycznej możemy tą tożsamość zapisać jako
\begin{equation}
  \label{eq:Mechanika-Rozwazania-ogolne-25}
  ( v_{ \HorSpaceOne 1 } v_{ \HorSpaceOne 2 } ) v_{ \HorSpaceOne 3 } =
  v_{ \HorSpaceOne 1 } ( v_{ \HorSpaceOne 2 } v_{ \HorSpaceOne 3 } ).
\end{equation}

Analogiczny rachunek pokazuje,
że~zachodzi własność analogiczna do~przemienności mnożenia liczb
rzeczywistych:
\begin{equation}
  \label{eq:Mechanika-Rozwazania-ogolne-26}
  \psi_{ \HorSpaceTwo u_{ \HorSpaceOne 1 }, \, u_{ \HorSpaceOne 2 } }(
  v_{ \HorSpaceOne 1 }, v_{ \HorSpaceOne 2 } ) =
  \psi_{ \HorSpaceTwo u_{ \HorSpaceOne 2 }, \, u_{ \HorSpaceOne 1 } }(
  v_{ \HorSpaceOne 2 }, v_{ \HorSpaceOne 1 } ),
\end{equation}
gdzie $u_{ \HorSpaceOne 1 }, u_{ \HorSpaceOne 2 } \in U$,
$v_{ \HorSpaceOne 1 } \in V_{ u_{ \HorSpaceOne 1 } }$
i~$v_{ \HorSpaceOne 2 } \in V_{ u_{ \HorSpaceOne 2 } }$. W~notacji arytmetycznej
mamy po prostu
\begin{equation}
  \label{eq:Mechanika-Rozwazania-ogolne-27}
  v_{ \HorSpaceOne 1 } v_{ \HorSpaceOne 2 } =
  v_{ \HorSpaceOne 2 } v_{ \HorSpaceOne 1 }.
\end{equation}

W~ten sam sposób dowodzimy własności łączącej iloczyn wektorów z~ich
ilorazem
\begin{equation}
  \label{eq:Mechanika-Rozwazania-ogolne-28}
  \varphi_{ \HorSpaceTwo u_{ \HorSpaceOne 1 } / u_{ \HorSpaceOne 2 }, \,
    u_{ \HorSpaceOne 3 } }\big(
  \varphi_{ \HorSpaceTwo u_{ \HorSpaceOne 1 }, \, u_{ \HorSpaceOne 2 } }(
  v_{ \HorSpaceOne 1 }, v_{ \HorSpaceOne 2 } ), v_{ \HorSpaceOne 3 } \big) =
  \varphi_{ \HorSpaceTwo u_{ \HorSpaceOne 1 }, \,
    u_{ \HorSpaceOne 2 } / u_{ \HorSpaceOne 3 } }\big(
  v_{ \HorSpaceOne 1 },
  \psi_{ \HorSpaceTwo u_{ \HorSpaceOne 2 }, \, u_{ \HorSpaceOne 3 } }(
  v_{ \HorSpaceOne 2 }, v_{ \HorSpaceOne 3 } ) \big),
\end{equation}
gdzie
$u_{ \HorSpaceOne 1 }, u_{ \HorSpaceOne 2 }, u_{ \HorSpaceOne 3 } \in U$,
$v_{ \HorSpaceOne 1 } \in V_{ u_{ \HorSpaceOne 1 } }$,
$v_{ \HorSpaceOne 2 } \in V_{ u_{ \HorSpaceOne 2 } }$,
$v_{ \HorSpaceOne 3 } \in V_{ u_{ \HorSpaceOne 3 } }$. W~notacji arytmetycznej
ostatnia tożsamość przyjmuje dobrze znaną postać
\begin{equation}
  \label{eq:Mechanika-Rozwazania-ogolne-29}
  \frac{ v_{ \HorSpaceOne 1 } }{ v_{ \HorSpaceOne 2 } } \div
  v_{ \HorSpaceOne 3 } =
  \frac{ v_{ \HorSpaceOne 1 } }{ v_{ \HorSpaceOne 2 } \HorSpaceOne
    v_{ \HorSpaceOne 3 } }.
\end{equation}
Inną własnością, której dowód przebiega wedle przedstawionego schematu,
jest „prawo rozdzielności mnożenia względem dzielenia i~dzielenia względem
mnożenia”:
\begin{equation}
  \label{eq:Mechanika-Rozwazania-ogolne-30}
  \begin{split}
    &\psi_{ \HorSpaceTwo u_{ \HorSpaceOne 1 } / u_{ \HorSpaceOne 2 }, \,
      q_{ \HorSpaceOne 1 } / q_{ \HorSpaceTwo 2 } }\big(
      \varphi_{ \HorSpaceTwo u_{ \HorSpaceOne 1 }, \, u_{ \HorSpaceOne2 } }(
      v_{ \HorSpaceOne 1 }, v_{ \HorSpaceOne 2 } ),
      \varphi_{ \HorSpaceTwo q_{ \HorSpaceOne 1 }, \, q_{ \HorSpaceTwo 2 } }(
      w_{ \HorSpaceOne 1 }, w_{ \HorSpaceOne 2 } ) \big) = \\
    &=
      \varphi_{ \HorSpaceTwo u_{ \HorSpaceOne 1 } q_{ \HorSpaceOne 1 }, \,
      u_{ \HorSpaceOne 2 } q_{ \HorSpaceTwo 2 } }\big(
      \psi_{ \HorSpaceTwo u_{ \HorSpaceOne 1 }, \, q_{ \HorSpaceOne 1 } }(
      v_{ \HorSpaceOne 1 }, w_{ \HorSpaceOne 1 } ),
      \psi_{ \HorSpaceTwo u_{ \HorSpaceOne 2 }, \, q_{ \HorSpaceTwo 2 } }(
      v_{ \HorSpaceOne 2 }, w_{ \HorSpaceOne 2 } ) \big),
  \end{split}
\end{equation}
gdzie $u_{ \HorSpaceOne 1 }, u_{ \HorSpaceOne 2 }, q_{ \HorSpaceOne 1 },
q_{ \HorSpaceOne 2 } \in U$, $v_{ \HorSpaceOne 1 } \in V_{ u_{ 1 } }$,
$v_{ \HorSpaceOne 2 } \in V_{ u_{ \HorSpaceOne 2 } }$,
$w_{ \HorSpaceOne 1 } \in V_{ q_{ \HorSpaceOne 1 } }$
i~$w_{ \HorSpaceOne 2 } \in V_{ q_{ \HorSpaceTwo 2 } }$. Znaczenie tej tożsamości
staje~się oczywiste, gdy zapiszemy ją w~notacji arytmetycznej
\begin{equation}
  \label{eq:Mechanika-Rozwazania-ogolne-31}
  \frac{ v_{ \HorSpaceOne 1 } }{ v_{ \HorSpaceOne 2 } } \cdot
  \frac{ w_{ \HorSpaceOne 1 } }{ w_{ \HorSpaceOne 2 } } =
  \frac{ v_{ \HorSpaceOne 1 } \HorSpaceOne w_{ \HorSpaceOne 1 } }{
    v_{ \HorSpaceOne 2 } \HorSpaceOne w_{ \HorSpaceOne 2 } }.
\end{equation}

Wiele innych własności funkcji
$\psi_{ \HorSpaceOne u_{ 1 }, \, u_{ \HorSpaceOne 2 } }$
i~$\varphi_{ \HorSpaceTwo u_{ 1 }, \, u_{ \HorSpaceOne 2 } }$ można wykazać,
korzystając z~tych już udowodnionych oraz przedstawionej już metody
przeprowadzania dowodów. Ponieważ zrobienie tego nie powinno nastręczać
żadnych trudności, nie będziemy ich tutaj jawnie analizować, chyba
że~skłoni nas do tego konkretny problem.

Przedstawiony powyżej formalizm powinien pozwolić nam opisać wszystkie
podstawowe operacje wykonywane na~wielkościach fizycznych posiadających
jednostki, acz dopiero test praktyczny pokaże jego prawdziwą wartość. Teraz
zaś przejdziemy do zilustrowania jego działania kilkoma przykładami.
Na~początku należy wybrać wybierać odpowiednie bazy w~przestrzeniach
$V_{ L }$, $V_{ \HorSpaceOne T }$ i~$V_{ M }$. Niech te bazy będą utworzone
przez wektory $e_{ \HorSpaceOne \si{m} } \in V_{ L }$,
$e_{ \HorSpaceOne \si{s} } \in V_{ \HorSpaceOne T }$,
$e_{ \HorSpaceOne \si{kg} } \in V_{ M }$. Wektor $e_{ \HorSpaceOne \si{m} }$ będziemy
interpretować jako oznaczający metr, wektor $e_{ \HorSpaceOne \si{s} }$ jako
reprezentujący sekundę, a~wektor $e_{ \HorSpaceOne \si{kg} }$ jako oznaczający
kilogram.

Jeżeli teraz mamy dwa wektory
$v_{ \HorSpaceOne 1 }, v_{ \HorSpaceOne 2 } \in V_{ L }$,
$v_{ \HorSpaceOne 1 } = 10 \, e_{ \HorSpaceOne \si{m} }$,
$v_{ \HorSpaceOne 2 } = 5 \, e_{ \HorSpaceOne \si{m} }$, to
\begin{equation}
  \label{eq:Mechanika-Rozwazania-ogolne-32}
  v_{ \HorSpaceOne 1 } + v_{ \HorSpaceOne 2 } =
  10 \, e_{ \HorSpaceOne \si{m} } + 5 \, e_{ \HorSpaceOne \si{m} } =
  15 \, e_{ \HorSpaceOne \si{m} } \in V_{ L }.
\end{equation}
W~notacji do której jesteśmy bardziej przyzwyczajeni, choć z~nietypowym
użyciem symbolu $v$, powyższa równość przyjęłaby formę
\begin{equation}
  \label{eq:Mechanika-Rozwazania-ogolne-33}
  v_{ \HorSpaceOne 1 } + v_{ \HorSpaceOne 2 } =
  10 \, \si{m} + 5 \, \si{m} = 15 \, \si{m}.
\end{equation}
Weźmy teraz $v = 10 \, e_{ \HorSpaceOne \si{m} } \in V_{ L }$
i~$w = 5 \, e_{ \HorSpaceOne \si{s} }\in V_{ T }$. Możemy teraz obliczyć
\begin{equation}
  \label{eq:Mechanika-Rozwazania-ogolne-34}
  \begin{split}
    \varphi_{ L, \, T }( v, w )
    &=
      \varphi_{ L, \, T }( 10 \, e_{ \HorSpaceOne \si{m} },
      5 \, e_{ \HorSpaceOne \si{s} } ) =
      10 \varphi_{ L, \, T }( e_{ \HorSpaceOne \si{m} },
      5 \, e_{ \HorSpaceOne \si{s} } ) =
      10 \cdot \frac{ 1 }{ 5 } \varphi_{ L, \, T }( e_{ \HorSpaceOne \si{m} },
      e_{ \HorSpaceOne \si{s} } ) = \\[0.3em]
    &=
      2 \varphi_{ L, \, T }( e_{ \HorSpaceOne \si{m} },
      e_{ \HorSpaceOne \si{s} } ).
  \end{split}
\end{equation}
Wektor $\varphi_{ L, \, T }( e_{ \HorSpaceOne \si{m} }, e_{ \HorSpaceOne \si{s} } )$
będziemy interpretowali jako reprezentujący metr nad sekundę i~zapisywali
za pomocą standardowej notacji jako
\begin{equation}
  \label{eq:Mechanika-Rozwazania-ogolne-29}
  \varphi_{ L, \, T }( e_{ \HorSpaceOne \si{m} }, e_{ \HorSpaceOne \si{s} } ) =
  \frac{ \si{m} }{ \si{s} }.
\end{equation}
Wykazane powyżej „prawa łączności i~rozdzielności”
gwarantują, że~otrzymane wyniki nie zależą od~kolejności wykonywania
działań na~wektorach reprezentujących wielkości fizyczne, tak jak tego
oczekiwaliśmy.

Zauważmy, że~w~omawianym formalizmie, zmiana jednostek w~których wyrażona
jest dana wielkości fizyczna sprowadza~się do zmiany bazy w~przestrzeniach
$V_{ L }$, $V_{ \HorSpaceOne T }$ i~$V_{ M }$. Przykładowo zmiana jednostki
czasu z~sekund na minuty sprowadza~się do przejścia z~bazy
$e_{ \HorSpaceOne \si{s} }$ do $e_{ \HorSpaceOne \si{min} }$:
\begin{equation}
  \label{eq:Mechanika-Rozwazania-ogolne-30}
  e_{ \HorSpaceOne \si{min} } = 60 \, e_{ \HorSpaceOne \si{s} }, \quad
  e_{ \HorSpaceOne \si{s} } = \frac{ 1 }{ 60 } e_{ \HorSpaceOne \si{min} }.
\end{equation}

Podane wyżej rozważania dotyczyły wielkości które mogą być reprezentowane,
przez jedną liczbę rzeczywistą której dodatkowo przypisujemy odpowiednią
jednostkę fizyczną. Zwyczajowo tego typu wielkości określa~się mianem
„wielkości skalarnych” lub „skalarów”. Nie obejmują one choćby taki
obiektów fizycznych posiadających jednostki, jak wektor pola elektrycznego
$\vecE( \vecx, t )$ w~chwili $t$ i~w~punkcie $\vecx$. Jednak uogólnienie
tego formalizmu by obejmował takie przypadki nie powinno stanowić
większego problemu, dlatego nie będziemy poświęcać temu więcej czasu.

Jeśli jakiś konkretny problemy który napotkamy, będą wymagały poważnym zmian
lub uogólnień przedstawionego tu podejścia, wówczas spróbujemy powrócić do
tych rozważań ~się starali do~niego powrócić
i~zmodyfikować w~odpowiedni sposób.










% ######################################
\newpage
% Tytuł danego działu
\section{Kanoniczne prace o~mechanice Newtona}

% \vspace{\spaceTwo}
% ######################################



% ############################
\subsection{ % Autor i tytuł dzieła
  Isaac Newton,
  \textit{Matematyczne zasady filozofii przyrody},
  \cite{NewtonMatematyczneZasadyFilozofiiPrzyrody2011}}


% ##################
\CenterBoldFont{Błędy}


\begin{center}

  \begin{tabular}{|c|c|c|c|c|}
    \hline
    Strona & \multicolumn{2}{c|}{Wiersz} & Jest
                              & Powinno być \\ \cline{2-3}
    & Od góry & Od dołu & & \\
    \hline
    16  & &  2 & Dodajęy & Dodaję \\
    % & & & & \\
    % & & & & \\
    % & & & & \\
    % & & & & \\
    \hline
  \end{tabular}

\end{center}










% ############################










% ######################################
\newpage

\section{Matematyczne ujęcie mechaniki Newtona}
% ######################################



% ############################
\subsection{ % Autor i tytuł dzieła
  Władimir Arnold \\
  \textit{Metody matematyczne mechaniki klasycznej},
  \cite{ArnoldMetodyMatematyczneMechanikiKlasycznej1981}}

\vspace{0em}


% ##################
\CenterBoldFont{Uwagi}

\vspace{0em}


\noindent
\textbf{Rozdział 7.} W~tym rozdziale nie znalazłem dowodu, ani
żadnej wskazówki, że~należy samemu pokazać, iż w~lokalnym układzie
współrzędnych zachodzi dobrze znany wzór:
\begin{equation*}
  \label{eq:Arnold-MetodyMatematyczneETC-01}
  d f = \partial_{ i } f\, d x^{ i }.
\end{equation*}
Zastosowanie tego wzoru znacznie ułatwia rozwiązywanie dalszych zadań
w~tym rozdziale, a~niektóre nie wiem nawet jak zrobić bez niego.

\VerSpaceFour





\noindent
\textbf{Str. 71.} W~twierdzeniu Poincar\'{e}go o~powracaniu
założenie o~ciągłości $g$ wydaje się bardzo nienaturalne. Wydaje się,
że~najlepiej jest je zamienić na żądanie mierzalności tej funkcji.

\VerSpaceFour







% ##################
\newpage

\CenterBoldFont{Błędy}


\begin{center}

  \begin{tabular}{|c|c|c|c|c|}
    \hline
    Strona & \multicolumn{2}{c|}{Wiersz} & Jest
                              & Powinno być \\ \cline{2-3}
    & od góry & od dołu & & \\
    \hline
    12  & &  2 & matematycz netak & matematyczne tak \\
    18  &  3 & & $\Phi( \vecxbold, \dot{ \vecxbold } )$
    & % $\mathbf{F}( \mathbf{ x }, \dot{ \mathbf{ x } } )$
    \\
    19  & &  1 & $\vecgbold\, \vecxbold$ & $-\vecgbold\, \vecxbold$ \\
    23  &  4 & & $f( \dot{ x } )$ & $f( x )$ \\
    24  &  1 & & Narysujem y & Narysujemy \\
    34  & &  2 & zorientowane & zorientowanej \\
    37  & & 10 & $\ddot{ \vecrbold } - r \dot{ \varphi }^{ 2 }$
           & $\ddot{ r } - r \dot{ \varphi }^{ 2 }$ \\
    37  & &  6 & $\dot{ r } -$ & $\ddot{ r } -$ \\
    60  & 10 & & napsać & napisać \\
    61  & &  8 & $\sqrt{ ( \dot{ q }_{ 1 }^{ 2 } + \dot{ q }_{ 2 }^{ 2 }
                 + \dot{ q }_{ 3 }^{ 2 } ) }$
           & $\frac{ 1 }{ 2 } m ( \dot{ q }_{ 1 }^{ 2 } + \dot{ q }_{ 2 }^{ 2 }
             + \dot{ q }_{ 3 }^{ 2 } )$ \\
    62  &  2 & & $m\, \dot{ \vecrbold }$ & $m\, \dot{ r }$ \\
    64  & &  8 & $G( x, p )$ & $G( x_{ 0 }, p )$ \\
    77  & &  9 & $S^{ 2 }$ & $S^{ 1 }$ \\
    81  &  5 & & $TM$ & $TM_{ x }$ \\
    81  &  9 & & $\mathbf{ \eta_{ i } }$ & $\eta_{ i }$ \\
    81  & &  5 & związką & wiązką \\
    81  & &  3 & $\vectbold_{ 0 }$ & $t_{ 0 }$ \\
    82  & 15 & & $m_{ 1 }${  }, & $m_{ 1 }$ \\
    86  &  2 & & \textit{Lagrange’a, to $( M, L )$}
           & \textit{Lagrange’a $( M, L )$, to} \\
    98  & &  3 & $\omega^{ 2 }$ & $\omega_{ 0 }^{ 2 }$ \\
    124 & 13 & & $\mathbf{Q}$ & $Q$ \\
    165 & &  9 & postc & postaci \\
    169 & & 10 & Prykład & Przykład \\
    170 & 8 & & k-wymiaro & k-wymiro- \\
    181 & & 15 & obszru & obszaru \\
    186 & & 11 & $T^{ * } V$ & $T^{ * } V_{ x }$ \\
    188 & 1 & & ednoparametrowa & jednoparametrowa \\
    214 & & 6 & rotacja & rotacją \\
    225 & 13 & & $H( \partial L / \partial \dot{ \vecpbold }, \vecqbold )$
           & $H( \partial L / \partial \dot{ \vecqbold }, \vecqbold )$ \\
    \hline
  \end{tabular}





  \newpage

  \begin{tabular}{|c|c|c|c|c|}
    \hline
    Strona & \multicolumn{2}{c|}{Wiersz} & Jest
                              & Powinno być \\ \cline{2-3}
    & od góry & od dołu & & \\
    \hline
    242 &  3 & & \textit{Jacobiego} & Jacobiego \\
    268 & &  8 & $g$ & $\vecgbold$ \\
    % 291 & & & & \\ % Jak się pisze w LaTeXu cyrlicą?
    351 &  1 & & $P_{ * }TM_{ X }$ & $P_{ * }TM_{ x }$ \\
    351 &  1 & & $T\gFrak_{ p }^{ * })$ & $T\gFrak_{ p }^{ * }$ \\
    373 & 18 & & A.~Arez & A.~Avez \\
    395 & & 11 & \textit{Poincar\'{e}'s} & \textit{Poincar\'{e}s} \\
    % & & & & \\
    \hline
  \end{tabular}

\end{center}

\VerSpaceTwo



\noindent
\StrWierszDol{29}{4} \\
\Jest  tworzy sferę dwuwymiarową. \\
\Powin można przekształcić w~sferę dwuwymiarową. \\
\StrWierszDol{42}{10} \\
\Jest  Słońce znajduje~się nie w~centrum \\
\Powin ale~Słońce nie znajduje~się w~centrum \\
\StrWierszDol{71}{15} \\
\Jest  do swego\ldots \\
\Powin dowolnie blisko swego\ldots \\
\StrWierszGora{215}{4} \\
\Jest  wirowej, a~pole jest bezźródłowe. \\
\Powin wirowej. \\



% ############################










% ############################
\newpage

\section{ % Autorzy i tytuł dzieła
  Roman Stanisław Ingarden, Andrzej Jamiołkowski \\
  \textit{Mechanika klasyczna},
  \cite{IngardenJamiolkowskiMechanikaKlasyczna1980}}

\vspace{0em}


% ##################
\CenterBoldFont{Uwagi do konkretnych stron}

\vspace{0em}


\Str{9--12}

\VerSpaceFour



\noindent
Str. 19. Bardzo ciężko jest zrozumieć uwagę, że w dwóch układach pochodne po czasie są różne, pomimo iż czas płynie tak samo. Proponuję następujące wyjaśnienie tego problemu:

Zauważmy, że w dwóch różnych układach odniesienia $x$ oraz
$\tilde{ x }$ będą różnymi funkcjami czasu (na razie zostawiamy na
boku głębszą dyskusję ontologicznej natury wykonywanych tu operacji).
Wytłumaczmy to na przykładzie: niech $\tilde{ x }$ będzie niezerowym
wektorem i niech układ $\tilde{ \Ocal }$ wykonuje obrót wokół
$\tilde{ p }_{ 0 }$. Teraz w układzie $\Ocal$ $\tilde{ x }$
jest wektorem o stałych współrzędnych, podczas gdy w układzie
$\tilde{ \Ocal }$ dokonuje on obrotu. Podobnie wektory bazy
układu $\tilde{ \Ocal }$ są postrzegane jako nieruchome w tym
układzie, lecz jako obracające się w
układzie $\Ocal$.

(Dyskusja ta wymaga udoskonalenia). Zauważmy, że każda pochodna ma
człon wynikający z różniczkowania współrzędnych i wektorów bazy.
Jeżeli więc mamy dany jakąś funkcje wektorową jako funkcję czasu, to
od wyboru układu odniesienia zależy nie tylko postać funkcyjna
współrzędnych, ale też czy mamy różniczkować dane wektory. W pewnym
sensie (bo do tej pory wszystko to jest niedoprecyzowane) pochodne
konkretnych funkcji skalarnych są takie same w każdym układzie
odniesienia, bo nie wchodzi do nich pochodna wektorów bazy.



Str. 20.
$\frac{ d\vecebold_{ 1 } }{ dt } = \vecomegabold \times \vecebold_{ 1 } \, ,$

Str. 20.
$\frac{ \tilde{ \dPL } \tilde{ \vecxbold } }{ \dPL t }
= \frac{ d\tilde{ x }^{ i } }{ dt } \vecebold_{ i } \, ,$

Str. 20. \ldots także z faktu, że
$\dPL \tilde{ x }^{ i } / \dPL t = \tilde{ \dPL } \tilde{ x
}^{ i } / \dPL t$\ldots

Str. 21.
$\vecvbold = \tilde{ \vecvbold } + \vecvbold_{ 0 } +
\vecomegabold \times \tilde{ \vecxbold } \, ,$

Str. 21. ????
$\frac{ d\vecvbold }{ dt } = \frac{ d\tilde{ \vecvbold } }{ dt }
+ \vecomegabold \times \tilde{ \vecvbold } + \frac{
  d\vecvbold_{ 0 } }{ dt } + \frac{ \vecomegabold }{ t }
\times \tilde{ \vecxbold } + \vecomegabold  \times \bigg(
\frac{ \tilde{ \dPL } \tilde{ \vecxbold } }{ \dPL t } +
\vecomegabold \times \tilde{ \vecxbold } \bigg) \, .$
Sprawdzić.

Str. 24. Obraz odwzorowanie
$X : T \rightarrow E^{ 3N }$\ldots

Str. 28. \ldots chwili $t \in T$ funkcje\ldots

Str. 36. \ldots oraz że nie zależy on od wyboru układu
współrzędnych\ldots


\VerSpaceTwo
% ############################










% ############################
\newpage

\section{ % Autorzy i tytuł dzieła
  J.I. Nejmark, N.A. Fufajew \\
  \textit{Dynamika układów nieholonomicznych},
  \cite{NejmarkFufajewDynamikaUkladowNieholonomicznych1971}}

% \vspace{0em}


% ##################
\CenterBoldFont{Błędy}


\begin{center}

  \begin{tabular}{|c|c|c|c|c|}
    \hline
    & \multicolumn{2}{c|}{} & & \\
    Strona & \multicolumn{2}{c|}{Wiersz}
                            & Jest & Powinno być \\ \cline{2-3}
    & Od góry & Od dołu & & \\
    \hline
    9   &  6 & & i wielu & wielu \\
    9   &  7 & & nczonych & uczonych \\
    11  & &  1 & $\delta$ & $\theta$ \\
    12  &  2 & & prędkość & przyśpieszenie \\
    12  &  3 & & równa & równe \\
    % & & & & \\
    % & & & & \\
    % & & & & \\
    % & & & & \\
    \hline
  \end{tabular}

\end{center}

\VerSpaceTwo

% ############################










% ######################################
\newpage

\section{Książki powstałe po~1945~r.}

\VerSpaceTwo
% ######################################



% ############################
\subsection{ % Autorzy i tytuł dzieła
  Lew D. Landau, Jewginij M. Lifszyc \\
  \textit{Mechanika}, \cite{LandauLifszycMechanika2006}}

\vspace{0em}


% ##################
\CenterBoldFont{Uwagi do konkretnych stron}

\vspace{0em}


\noindent
\Str{14} Podana tu grupa Galileusza składa~się tylko z~pchnięć,
co według mnie tylko zaciemnia strukturę symetrii czasoprzestrzeni
Galileusza. Pełniejsze omówienie tej grupy można znaleźć w~książce
W.~Arnolda \textit{Metody matematyczne mechaniki klasycznej}
\cite{ArnoldMetodyMatematyczneMechanikiKlasycznej1981}.

\VerSpaceFour





\noindent
\Str{13} Przemyślenie jest głębokie, ale przedstawione
stanowczo zbyt krótko, aby było jasne. Spróbuję przedstawić tu pewne
jego rozwinięcie.

Przed wszystkim należy zauważyć, że należy tu rozróżnić jednorodność
i~izotropowość w sensie geometrii przestrzeni i w sensie dynamiki.
Cechy te traktowane jako cechy geometrii czasoprzestrzeni w sensie
geometrii liniowej i różniczkowej, są niezależne od układu
odniesienia. Przejdźmy teraz do problemu dynamiki. Po pierwsze z
doświadczenia wiemy, że możemy przyjąć, iż przestrzeń jest
euklidesowa, jak również że można znaleźć układ odniesienia w którym
cząstki swobodne umieszczone w przestrzeni spoczywają.

\VerSpaceFour





\noindent
\Str{14} $\frac{ \partial L }{ \partial \vecvbold }$ nie jest
funkcją tylko kwadratu prędkości. Jest to wektor o składowych
$( \frac{ \partial L }{ \partial \vecvbold } )_{ i }
= 2 \frac{ \partial L }{ \partial { v^{ 2 } } } v_{ i }$, czyli zależy on
jawnie od składowych prędkości. Widać jednak, że stałość
$\frac{ \partial L }{ \partial \vecvbold }$ wymaga od nas stałości
$\mathbf{ v }$. Jeżeli bowiem rozpatrzymy składową $x$ wektora
(ściślej pola wektorowego)
$\frac{ \partial L }{ \partial \vecvbold }$, mamy warunek na stałość
tego wyrażenia dla dowolnej wartości składowej $x$:
$\frac{ \partial L }{ \partial { v^{ 2 } } } = \frac{ 1 }{ 2 \vecvbold_{ x } }$.
Wyrażenie to należy zakwestionować na paru
poziomach, choćby dlatego, że jest osobliwe dla zerowych prędkości, co
jest niedopuszczalne dla fizycznej teorii. Oczywiście, jeżeli sprawdzimy
również warunek na $y$ składową otrzymamy sprzeczny układ równań.

\VerSpaceFour





\noindent
\Str{14} Należałoby podać większą dyskusję prędkości względnej dwóch
układów inercjalnych.

\VerSpaceFour





\noindent
\Str{15} Jak można ściślej uzasadnić, że rzeczywiście
potrzebujemy liniowej zależności od prędkości prawej strony równania
wyrażającego równoważność między dwoma lagrażjanami? \Dok

\VerSpaceFour




\noindent
\Str{22} Dyskusja ważności addytywnych zasad zachowania, ma
swoją głębię i wagę, zaciemnia ona jednak pewne szczegóły. Autorzy gdy
ją pisali musieli mieć na myśli procesy rozpraszania, nie wspomnieli
jednak, że jeśli znana jest postać oddziaływania między dwoma
cząstkami, również mamy możliwość wyciągnięcia z praw zachowania
ważnych wniosków. Np. jeśli rozpatrujemy układ dwóch cząstek i znamy
energię kinetyczną jednej z nich i energię oddziaływania, to możemy
obliczyć pewne parametry ruchu drugiej.

\VerSpaceFour










% ##################
\newpage

\CenterBoldFont{Błędy}


\begin{center}

  \begin{tabular}{|c|c|c|c|c|}
    \hline
    Strona & \multicolumn{2}{c|}{Wiersz} & Jest
                              & Powinno być \\ \cline{2-3}
    & Od góry & Od dołu & & \\
    \hline
    56  & 12 & & poruszały się z tą samą prędkością & spoczywały \\
    % & & & & \\
    % & & & & \\
    \hline
  \end{tabular}

\end{center}

\VerSpaceTwo


\noindent
\StrWierszDol{27}{2} \\
\Jest  $S = S' + \mu \vecVbold \cdot \vecRbold' + \frac{ 1 }{ 2 } \mu V^{ 2 } t$ \\
\Powin $S = S' + \mu \vecVbold \cdot \vecRbold'( t ) - \mu \vecVbold
\cdot \vecRbold'( 0 ) + \frac{ 1 }{ 2 } \mu V^{ 2 } t$ \\



% ############################










% ############################
\newpage

\section{ % Autor i tytuł dzieła
  Bogdan Skalmierski \\
  \textit{Mechanika}, \cite{SkalmierskiMechanika1998}}

\vspace{0em}


% ##################
\CenterBoldFont{Uwagi}

\vspace{0em}


We wszystkich rozważaniach przestrzeni wektorowych, będziemy~się ograniczać
do dwóch najważniejszych dla fizyki, w~szczególności też dla mechaniki,
typów przestrzeni wektorowych. Mianowicie będziemy rozważać tylko
przestrzenie wektorowe nad ciałem liczby rzeczywistych lub zespolonych.
W~istocie w~mechanice Newtona prawie zawsze wystarczające będzie rozważanie
przestrzeni nad ciałem liczb rzeczywistych.

Dodatkowo przyjmujemy, że~wszystkie rozważane przestrzenie mają skończony
wymiar, chyba że~jest powiedziane inaczej. Wymiar rozważanej przestrzeni
zawsze będziemy oznaczać symbolem~$N$.

\VerSpaceFour





% ##################
\CenterBoldFont{Uwagi do konkretnych stron}


\noindent
\Str{16} W~tym miejscu należałoby dodać następującą uwagę na temat oznaczeń
dla wektorów zaczepionych. Jeżeli wektory zaczepione w~dwóch różnych
punktach $A$ i~$B$ są równoważne, to będziemy je oznaczać tym samym symbolem
np. $\vecabold$. Inaczej mówiąc, jeśli w~punktach $A$ i~$B$ jest zaczepiony
ten sam wektor swobodny, to te dwa wektory swobodne oznaczamy tym samym
symbolem.

W~literaturze zwykle~się nie pisze jawnie, czy dany symbol $\vecabold$
oznacza wektor swobodny, czy wektor zaczepiony w~danym punkcie. Co gorsza
często ten sam symbol oznacza zarówno wektor swobodny $\vecabold$, jak
i~wektor swobodny $\vecabold$ po zaczepieniu w~punkcie $A$. Może to
prowadzić do niejasności i~utrudniać początkującym naukę.

Bardziej przejrzysta byłaby notacja $\vecabold_{ A }$, jednak szansa by to
oznaczenie~się przyjęło jest zaniedbywalnie mała. W~dalszej części
komentarzy będziemy próbowali precyzować co dany wektor dokładnie oznacza.

\VerSpaceFour




\noindent
\Str{16} By uczynić definicję relacji współosiowości wektorów bardziej
ścisłą, przyjmiemy, że~wektor $\vecZeroBold$ jest współosiowy z~dowolnym
innym wektorem.

\VerSpaceFour





\noindent
\Str{17} Warto zatrzymać się na chwilę nad pojęciem \textbf{składowej
  wektora} i~\textbf{współrzędnej wektora}. Niech $V$ będzie rozważaną
przestrzenią wektorową, a~$\vecvbold$ zawartym w~niej wektorem.
\textbf{Współrzędnymi wektora $\vecvbold$ w bazie $\vecebold_{ i }$}
będziemy nazywać ciąg liczb $\alpha_{ i }$ takich, że
\begin{equation}
  \label{eq:SkalmierskiMechanika-01}
  \vecvbold = \sum_{ i = 1 }^{ N } \alpha_{ i } \vecebold_{ i }.
\end{equation}
Gdy nie będzie możliwości nieporozumień, współrzędne wektora $\vecvbold$
będziemy oznaczać symbolami $v_{ i }$, $v_{ i } \equiv \alpha_{ i }$. Analogicznie
współrzędne wektora $\vecabold$ będziemy oznaczać symbolami $a_{ i }$,
wektora $\vecbbold$, $b_{ i }$, etc.

\textbf{Składowymi wektora $\vecvbold$} będziemy nazywać dowolny zbiór
liniowo niezależnych wektorów \\
$\{ \vecwbold_{ 1 }, \vecwbold_{ 2 }, \ldots, \vecwbold_{ k } \}$, $k \geq 1$, taki że
\begin{equation}
  \label{eq:SkalmierskiMechanika-02}
  \vecvbold = \sum_{ i = 1 }^{ k } \vecwbold_{ i }.
\end{equation}

Kilka uwag odnośnie tej definicji. Moglibyśmy przyjąć, że wektory
$\vecwbold_{ i }$ tworzą nie zbiór tylko ciąg, jednak ponieważ dodawanie
(skończonej) liczby wektorów jest przemienne oraz że~w~praktyce rzadko
kiedy podaje~się konkretną numerację zbioru składowych, przyjęliśmy taką
jej . Moglibyśmy też opuścić warunek liniowej niezależności tych
wektorów i~definicja była wciąż użyteczna, jednak to założenie wydaje
nam~się bardzo naturalne w~tym kontekście i~nie powinno sprawiać żadnych
problemu, gdy to pojęcie jest stosowane w literaturze fizycznej.

Wykluczyliśmy możliwość, że $k = 0$, czyli, że zbiór składowych jest pusty.
Ponieważ w~algebrze liniowej przyjmuje się, że suma po pustym zbiorze
wektorów daje nam wektor zerowy, w~skutek tego definicja ta wyklucza rozkład
wektora zerowego na zerową liczbę składowych. Natomiast założenie o~liniowej
niezależność zbioru wyklucza dowolny inny rozkład wektora zerowego na
składowe. Żadna z~tych konsekwencji podanej definicji, nie powinna stanowić
problemu.

Na koniec zauważmy, że~dopuszczamy sytuację gdy $k < N$, co jest przypadkiem
często spotykanym w~praktyce.

\VerSpaceFour





\noindent
\Str{19} Mała uwaga na temat notacji. W~każdej przestrzeni wektorowej $V$
jest określone działanie mnożenia wektora przez liczbę z~zadanego
ciała $\Fbb$: $\textrm{product} : \Fbb \times V \to V$. W sytuacji gdy $\alpha \in \Fbb$,
$\vecabold \in V$, będziemy uznawać, że~następujące wyrażania są sobie
równoważne.
\begin{equation}
  \label{eq:SkalmierskiMechanika-03}
  \textrm{product}( \alpha, \vecabold ) \equiv \alpha \, \vecabold \equiv \vecabold \, \alpha
\end{equation}
Analogiczne, dla $\alpha \neq 0$ uważamy dwa poniższe wyrażenia za równoważne.
\begin{equation}
  \label{eq:SkalmierskiMechanika-04}
  \frac{ 1 }{ \alpha } \vecabold \equiv \frac{ \vecabold }{ \alpha }
\end{equation}

W~przypadku teorii fizycznych w~większości przypadków interesować nas będą
tylko przypadki $\Fbb = \Rbb$ lub $\Fbb = \Cbb$. W~niniejszej książce
wystarczające powinny przestrzenie nad ciałem liczb rzeczywistych.

\VerSpaceFour





\noindent
\Str{19} W tym miejscu powinno być przytoczone „prawo zachowania
wskaźników”, do jego sformułowania potrzebujemy wprowadzić odrobinę
terminologi.

Każdy wskaźnik po którym sumujemy, np. $i$ w wyrażeniu
\begin{equation}
  \label{eq:SkalmierskiMechanika-05}
  \sum_{ i = 1 }^{ N } a^{ i },
\end{equation}
nazywamy \textbf{wskaźnikiem niemym} lub \textbf{martwym}. Taki wskaźnik
jest tylko nazwą zmiennej sumowania, można zmienić go dowolny inny i~treść
matematyczna pozostanie bez zmian. Każdy wskaźnik który nie jest niemym
nazywamy \textbf{wskaźnikiem żywym}.

„Prawo zachowania wskaźników”: każdy wskaźnik żywy występujący po lewej
stronie równości, musi też wystąpić po jej prawej stronie.

\VerSpaceFour





\noindent
\StrWierszDol{19}{10} Jak stwierdziła Iwona Grabska-Gradzińska, lepszym
oznaczeniem od $a_{ xi }$ byłoby $a_{ i,\, x }$. Taka notacja bardzie uwypukla
to, że mamy do czynienia z~iksową składową wektora o~numerze $i$.

\VerSpaceFour





\noindent
\Str{20, 22} Definicja konta między wektorami, choć opierająca~się na tym,
że~„łatwo widać” co to jest kont między wektorami, gdy się je narysuje,
zostawia pewną lukę dla przypadku, gdy jeden wektor jest równy
$\vecZeroBold$.

Aby ją zapełnić przyjmujemy, że~jeśli co najmniej jeden z~wektorów
$\vecabold$ i~$\vecbbold$ jest wektorem zerowym to kąt między nimi
wynosi~$0$ radianów: $\alpha = 0$.

\VerSpaceFour





\noindent
\Str{20} Może nie jest to zaznaczone wyraźnie, ale aby zdefiniować iloczyn
skalarny dwóch wektorów musimy oba przemieścić w~taki sposób, by były
zaczepione w~jednym punkcie. Albo dokonać operacji, która pozwoliłaby je
zaczepić w~wspólnym punkcie. Zauważmy też, że~te same uwagi odnoszą się do
iloczynu wektorowego i~podobnych operacji.

W~definicji „za pomocą obrazka” musimy je zaczepić we wspólnym punkcie, by
zmierzyć kąt jaki jest między nimi. Przykładem innego sposobu liczenia tego
kąta jest traktowanie wektorów jako odcinków skierowanych i~przeniesienie
wektora $\vecbbold$ do punktu końcowego wektora $\vecabold$, obliczenia kąta
$\beta$ między tak utworzonymi odcinkami i~przyjęcie, że kąt między wektorami
jest równy $\alpha = \pi - \beta$ (w~radianach). Widać jednak, że skoro możemy
przesunąć wektor $\vecbbold$ do punktu końcowego wektora $\vecabold$, to
równie dobrze moglibyśmy przenieść go do punktu zaczepienia tego
wektora\footnote{To zagadnienie można podać bardziej ścisłej analizie,
  jednak uważamy, że~obecnej formie jest ono wystarczająco precyzyjnie
  sformułowane w~stosunku do naszych potrzeb.}.

W~przypadku definicji „za pomocą wersorów/współrzędnych”, musimy założyć, że
możemy porównać wersory zaczepione w~dwóch różnych miejscach i~obliczyć
iloczyn skalarny między nimi. Możemy też uznać, że wersor reprezentuje nam
klasę wektorów równoważnych i~w każdym punkcie przestrzeni jest zaczepiony
jeden z nich. Ale tym samym uznaliśmy pojęcie równości „na odległość”,
dzięki czemu możemy sformułować pojęcie przeniesienia wektora

Jakbyśmy więc nie podeszli do zagadnienia, to albo musimy mieć pojęcie
przesuwania wektorów, które ich nie zmienia, albo pojęcie równości wektorów
„na odległość”. W~obecnym kontekście może wydawać~się to drobnostką, jednak
przy dokładnej analizie pewnych wzorów pominięcie tego aspektu może
prowadzić do niejasności i~zbyt pobieżnego rozumienia danego zagadnienia.
Problem ten staje~się czymś bardzo poważnym, gdy przechodzimy do geometrii
różniczkowej.

W~dalszym ciągu będziemy więc przyjmować, że~podczas obliczania wielkości
takich jaki iloczyn skalarny oba wektory zostały przeniesione do pewnego
wspólnego punktu i~dopiero wtedy obliczamy wynik. Dlatego wykonując tą
operację na wektorach $\vecabold$, $\vecbbold$, nie będziemy dodawać im
indeksu dolnego, jak w $\vecabold_{ A }$, by zaznaczyć w~którym miejscu są
one zaczepione.

Powstaje pytanie, skoro rezultatem np. iloczynu wektorowego, jest wektor to
w~którym miejscu jest on zaczepiony? W~takiej sytuacji będziemy pryzmować,
iż~taki wektor jest wektorem swobodnym, chyba że~jest powiedziane inaczej.

\VerSpaceFour





\noindent
\Str{21} Z~formy podanej tu procedury opuszczania wskaźnika, wynika,
że~jeśli $a^{ i }$ są współrzędnymi odpowiedniego wektora, to $a_{ i }$ są
współrzędnymi nie wektora, lecz formy (1-formy). Dopiero korzystając
z~naturalnego izomorfizmu (Czy to jest na pewno naturalny izomorfizm???)
między wektorami i~formami możemy utożsamić te dwa obiekty, co będziemy
w~dalszym ciągu czynić.

\VerSpaceFour





\noindent
\Str{24} Użycie konwencji sumacyjnej we wzorze (1.26) chyba bardzie
zaciemnia, niż rozjaśnia jego treść. Byłby on bardziej zrozumiały, gdyby
został zapisany jako
\begin{equation}
  \label{eq:SkalmierskiMechanika-06}
  \sum_{ j = 1 }^{ 3 } \varepsilon_{ j i k } \, \varepsilon_{ j m n }
  = \delta_{ i m } \, \delta_{ k n } - \delta_{ i n } \, \delta_{ k m }.
\end{equation}

Sam autor ma chyba z~tym problem bo nazywa wyrażenie
$\varepsilon_{ j i k } \, \varepsilon_{ j m n }$ iloczynem, podczas gdy zgodnie z~przyjętą
konwencją należy je uważać za zapis sumy po $j$. Wedle mojej intuicji
językowej wyrażenia arytmetyczne\footnote{Jeśli chodzi o~wyrażenia bardziej
  złożone, np. zawierające funkcję $\sin$, sposób ich nazywania wymaga
  większej dozy refleksji.} nazywamy po wyrażeniu które stoi najniżej
w~hierarchii działań występujących w~danym wyrażeniu. Mówiąc prościej,
to które wykonujemy na końcu. Z tego więc względu sumą nazywamy wyrażenia
$a + b$, $a + bc$, iloczynem $ab$, $a ( b + c )$, etc. Stąd wzór
\eqref{eq:SkalmierskiMechanika-06} nazwalibyśmy raczej sumą, niż iloczynem.

W~skutek tego zamieszania terminologicznego, przedstawiony tu dowód
tożsamości \eqref{eq:SkalmierskiMechanika-06} nie jest całkowicie jasny.
Można go jednak przeprowadzić w następujący sposób\footnote{Dla zupełności
  podamy tu pełny dowód tego twierdzenia.}. Aby uniknąć nieporozumień,
przyjmiemy, że $j_{ 1 }$ oznacza jedną z liczb $1, 2, 3$. Wyrażenie
$\varepsilon_{ j_{ 1 } i k }$ przyjmuje wartość różną od zera wtedy i tylko wtedy, gdy
wszystkie trzy liczby $j_{ 1 }$, $i$, $k$, są między sobą różne. Inaczej
mówiąc, jeśli ciąg $j_{ 1 }, i, k$ stanowi permutacje ciągu $1, 2, 3$.
Jeżeli więc $i = k$ to lewa strona tożsamości
\eqref{eq:SkalmierskiMechanika-06} jest równa 0.
Prawa zaś stronę możemy przepisać jako
\begin{equation}
  \label{eq:SkalmierskiMechanika-07}
  \delta_{ i m } \, \delta_{ i n } - \delta_{ i n } \, \delta_{ i m }.
\end{equation}
Jeżeli $i \neq m$ lub $i \neq n$ to wyrażenie powyżej jest oczywiście równe 0.
Pozostał nam do rozpatrzenia ostatni przypadek: $i = m$ i~$i = n$, bądź
krócej $i = m = n$. W~tym przypadku też oczywiście otrzymujemy 0. Ta sama
analiza stosuje~się do $\varepsilon_{ j m n }$.

Jak powiedziano wyżej, dla ustalonej liczby $j_{ 1 }$ iloczyn
$\varepsilon_{ j_{ 1 } i k }$ z~$\varepsilon_{ j_{ 1 } m n }$ będzie niezerowy wtedy i tylko wtedy,
gdy oba ciągi $j_{ 1 }, i, k$ oraz $j_{ 1 }, m, n$ są różnowartościowe
(inaczej: są permutacjami ciągu $1, 2, 3$). Zachodzi więc
$j_{ 1 } \neq i \neq k \neq j_{ 1 }$ i~analogicznie dla $j_{ 1 } \neq m \neq n \neq j_{ 1 }$.
W~tym przypadku z~sumy \eqref{eq:SkalmierskiMechanika-06} przeżywa tylko
jeden wyraz. Jeśli bowiem $i = 1$, $k = 2$, to niezerowy będzie tylko wyraz
dla którego $j = 3$, analogicznie dla pozostałych przypadków. W takim razie
możliwe są dwie sytuacje: a) $i = m$, $k = n$; b) $i = n$, $k = m$.

W przepadku a) $\varepsilon_{ j_{ 1 } i k } = \varepsilon_{ j_{ 1 } m n }$, więc ich iloczyn jest
równy $1^{ 2 } = 1$ lub $( -1 )^{ 2 } = 1$. Podstawienie zależności między
$j_{ 1 }, i, k, m, n$ do prawej stron wzoru
\eqref{eq:SkalmierskiMechanika-06} pokazuje, że~również jej wartość wynosi
$1$.

Przypadek b). Możliwe są dwie sytuacje:
$\varepsilon_{ j_{ 1 } i k } = 1$, $\varepsilon_{ j_{ 1 } m n } = -1$
lub~$\varepsilon_{ j_{ 1 } i k } = -1$, $\varepsilon_{ j_{ 1 } m n } = 1$. Wobec tego lewa strona wzoru
\eqref{eq:SkalmierskiMechanika-06} wynosi $-1$. Ponownie korzystając
z~relacji między $j_{ 1 }, i, k, m, n$ dostajemy, że~tyle samo wynosi prawa
strona tego wzoru, co kończy dowód.

\VerSpaceFour





\noindent
\Str{24} W~tym miejscu Skalmierski stwierdził, że~prawdziwe jest twierdzenie
które można by wysłowić w~następujący sposób. „Momenty dwóch wektorów
obliczone względem ustalonego punktu są równe wtedy i~tylko wtedy, gdy te
wektory są równoważne i~współosiowe”. Takie twierdzenie jednak nie zachodzi.

Rozważmy najpierw wektory leżące na prostych przechodzących przez punkt
$O$. Momenty wszystkich tych wektorów względem punktu $O$ są równe
$\vecZeroBold$, widzimy więc, że~istnieje nieskończenie wiele wektorów,
które nie muszą być ani równoważne, ani współosiowe, a~momenty ich obu są
równe $\vecZeroBold$. Widzimy też, że o~równoważności i~współosiowości
wektorów których moment wynosi $\vecZeroBold$ nie możemy nic powiedzieć.

Zauważmy też, że~poza przypadkiem wymienionym wyżej, jest tylko jedna inna
możliwość, że~moment wektory jest równy $\vecZeroBold$. Mianowicie gdy
liczymy moment wektora zerowego zaczepionego w~dowolnym punkcie
przestrzeni.

Rozważmy teraz wektory\footnote{Ze względu na wygodę, nie będę w~notacji
  rozróżniał między wektorem, $\alpha$~jego współrzędnymi.}:
$\vecabold = [ 0, 1, 0 ]$ z~wektorem wodzącym jego punktu zaczepienia
$\vecrbold_{ 1 } = [ 1, 0, 0 ]$ i~$\vecbbold = [ -1, 0,  0 ]$ z~wektorem
wodzącym punktu zaczepienia $\vecrbold_{ 2 } = [ 0, 1, 0 ]$. Jak łatwo
sprawdzić zachodzi
\begin{equation}
  \label{eq:SkalmierskiMechanika-07}
  \vecMbold = \vecrbold_{ 1 } \times \vecabold =
  \vecrbold_{ 2 } \times \vecbbold = [ 0, 0, 1 ].
\end{equation}
Wektory $\vecabold$ i~$\vecbbold$ nie są ani równoważne, ani~współliniowe.

Przykład przedstawiony w~poprzednim paragrafie można uogólnić w~następujący
sposób. Mianowicie zauważając, że~iloczyn wektorowy jest niezmiennicze ze
względu na obroty: jeśli dwa wektory $\vecabold$ i~$\vecbbold$ zaczepione
w~punkcie~$A$ obrócimy o~ten sam kąt wokół punktu $A$, to ich iloczyn
wektorowy nie ulegnie zmianie. W~naszym przypadku sytuacja jest trochę
bardziej skomplikowana.

Niech $O$ będzie punktem względem którego liczymy moment wektora, $A_{ 0 }$
punktem w~którym jest zaczepiony wektor $\vecabold_{ 0 }$. Jeśli teraz
obrócimy wektor wodzący punktu $A$, który będziemy oznaczać
$\vecrbold_{ 0 }$, o~kąt $\varphi$ wokół punktu $O$ to otrzymamy nowy wektor
$\vecrbold_{ 1 }$ pokazujący inny punkt. Punkt wskazywany przez wektor
$\vecrbold_{ 1 }$ oznaczmy przez $A_{ 1 }$. Jeśli teraz obrócimy wektor
$\vecabold_{ 0 }$ o~kąt $\varphi$ wokół punktu $A_{ 0 }$, to otrzymany nowy wektor
który oznaczymy przez $\vecabold_{ 1 }$. Teraz wektor $\vecabold_{ 1 }$
zaczepiamy w~punkcie $A_{ 1 }$.

Po tak przeprowadzonej operacji moment wektora $\vecabold_{ 1 }$
zaczepionego w~punkcie $A_{ 1 }$ jest taki sam, jak wektora $\vecabold_{ 0 }$
zaczepionego w~punkcie $A_{ 0 }$. Wynika to z tego, że~tak przeprowadzona
transformacja zachowuje długość wektorów oraz kąt między wektorem
a~wektorem wodzącym jego punktu zaczepienia. Łatwo zauważyć, że~równość
\eqref{eq:SkalmierskiMechanika-07} zachodzi, bo wektor $\vecbbold$ powstaje
przez obrót wektora $\vecabold$ o~kąt $\pi / 2$ zgodnie z~procedurą opisaną
powyżej.

Rozważmy jeszcze jeden przypadek. Niech będzie dany wektor $\vecabold$
o~wektorze wodzącym $\vecrbold_{ 1 }$ i~niech dana będzie liczba rzeczywista
$\alpha \neq 0$. Rozważmy wektor $\vecbbold = \alpha \vecabold$, zaczepiony w~punkcie
danym przez wektor $\vecrbold_{ 2 } = ( 1 / \alpha ) \vecrbold_{ 1 }$. Łatwo
zauważyć, że
\begin{equation}
  \label{eq:SkalmierskiMechanika-08}
  \vecrbold_{ 2 } \times \vecbbold =
  \left( \frac{ 1 }{ \alpha } \vecrbold_{ 1 } \right) \times ( \alpha \vecabold ) =
  \vecrbold_{ 1 } \times \vecabold.
\end{equation}
Otrzymaliśmy całą klasę wektorów, które nie są równoważne, acz są
współosiowe, których momenty względem punktu $O$ są takie same. Zwróćmy
uwagę, że jak w poprzednim przykładzie z~obrotami, tutaj również zmieniamy
nie tylko wektor, ale też jego punkt zaczepienia.

Prawdopodobnie można podać wyczerpującą klasyfikację wektorów posiadających
ten sam moment względem punktu $O$, jedna powyższe przykłady sugerują,
iż~gra nie jest warta świeczki, dlatego poprzestaniemy na udowodnieniu
poniższego twierdzenia.





% #############
\begin{theorem}

  Jeżeli wektory $\vecabold$ i~$\vecbbold$ są równoważne i~współosiowe, to
  ich momenty są równe.

\end{theorem}



\begin{proof}

  Skoro wektory
  $\vecabold$ i~$\vecbbold$ są równoważne i~współosiowe więc różnią się tyko
  punktem zaczepienia na pewnej prostej. Zgodnie z tym co powiedziany
  poprzednio będziemy je więc oznaczać tym samym symbolem $\vecabold$. Tym
  samym mamy
  \begin{equation}
    \label{eq:SkalmierskiMechanika-07}
    \vecMbold_{ 1 } = \vecrbold_{ 1 } \times \vecabold, \quad
    \vecMbold_{ 2 } = \vecrbold_{ 2 } \times \vecabold.
  \end{equation}
  Skoro te wektory różnią się tylko punktem zaczepienia na pewnej prostej,
  zaś na punkt te wskazują wektory $\vecrbold_{ 1 }$ i~$\vecrbold_{ 2 }$ to
  istnieje taki wektor $\vecRbold$ współosiowy z~$\vecabold$, taki że
  $\vecrbold_{ 2 } = \vecrbold_{ 1 } + \vecRbold$. Tym samym mamy
  \begin{equation}
    \label{eq:SkalmierskiMechanika-08}
    \vecMbold_{ 2 } =
    \vecrbold_{ 2 } \times \vecabold =
    ( \vecrbold_{ 1 } + \vecRbold ) \times \vecabold =
    \vecrbold_{ 1 } \times \vecabold + \vecRbold \times \vecabold =
    \vecrbold_{ 1 } \times \vecabold + \vecZeroBold = \vecMbold_{ 1 }.
  \end{equation}

\end{proof}
% #############









% ##################
\newpage

\CenterBoldFont{Błędy}


\begin{center}

  \begin{tabular}{|c|c|c|c|c|}
    \hline
    Strona & \multicolumn{2}{c|}{Wiersz} & Jest
                              & Powinno być \\ \cline{2-3}
    & Od góry & Od dołu & & \\
    \hline
    22 & 13 & & wektor{ }{ }{ }$\vecdbold$ & wektor $\vecdbold$ \\
    22 & &  9 & $a_{ z } b_{ y } \, \hphantom{k} \times \vecjbold$
           & $a_{ z } b_{ y } \, \veckbold \times \vecjbold$ \\
    24 & 17 & &  $( \vecabold \cdot \veccbold ) \cdot \vecbbold
                - ( \vecbbold \cdot \veccbold ) \cdot \vecabold$
           & $( \vecabold \cdot \veccbold ) \vecbbold
             - ( \vecbbold \cdot \veccbold ) \vecabold$ \\
    % & & & & \\
    % & & & & \\
    % & & & & \\
    % & & & & \\
    % & & & & \\
    \hline
  \end{tabular}

\end{center}

\VerSpaceTwo



% ############################










% ############################
\newpage

\subsection{ % Autor i tytuł dzieła
  Bogdan Skalmierski \\
  \textit{Mechanika z~wytrzymałością materiałów},
  \cite{SkalmierskiMechanikaZWytrzymalosciaMaterialow1983}}

\vspace{0em}


% ##################
\CenterBoldFont{Uwagi do konkretnych stron}

\vspace{0em}


\noindent
\Str{21} We~wzorze w~drugiej linii zamiast
\begin{equation}
  \label{eq:Skalmierski-MechanikaZWytrzymalosciaETC-01}
  \sqrt{ 1 - \left( \tfrac{ x }{ x_{ 0 } } \right)^{ 2 } }
\end{equation}
powinno być
\begin{equation}
  \label{eq:SkalmierskiMechanikaZWytrzymaloscia-02}
  \sgn( \cos \varphi ) \,
  \sqrt{ 1 - \left( \tfrac{ x }{ x_{ 0 } } \right)^{ 2 } },
\end{equation}
bo~wykorzystujemy jedynkę trygonometryczną by~wyrazić $\cos$ przez
$\sin$. Ponieważ w~dalszym ciągu obliczeń podnosimy ten człon do
kwadratu, ta niedokładność nie~wpływa na ostateczny wynik.

\VerSpaceFour





\noindent
\Str{36} Aby wyprowadzenie wzoru (3.21) było poprawne,
potrzebujemy by
$| \dot{ \vecebold }_{ 1 } \cdot \vecebold_{ 2 } | = \dot{ \vecebold }_{ 1 }
\cdot \vecebold_{ 2 }$. Oznacza to, że~układ obraca się od~wektora
$\vecebold_{ 1 }$ do~$\vecebold_{ 2 }$.





% ##################
\newpage

\CenterBoldFont{Błędy}


\begin{center}

  \begin{tabular}{|c|c|c|c|c|}
    \hline
    Strona & \multicolumn{2}{c|}{Wiersz} & Jest
                              & Powinno być \\ \cline{2-3}
    & Od góry & Od dołu & & \\
    \hline
    12  & 16 & & można określić & będziemy oznaczali \\
    13  & & 10 & $\vecbbold_{ y }$ & $\vecabold_{ y }$ \\
    17  &  3 & & $( \vecabold \cdot \veccbold ) \cdot \vecbbold
                 - ( \veccbold \cdot \vecbbold ) \cdot \vecabold$
           & $( \vecabold \cdot \veccbold ) \vecbbold
             - ( \veccbold \cdot \vecbbold ) \vecabold$ \\
    21  &  4 & & $\frac{ y }{ { }_{ 0 } }$ & $\frac{ y }{ { y }_{ 0 } }$ \\
    21  &  4 & & $\sqrt{ 1 \:\: \left( \frac{ x }{ x_{ 0 } } \right)^{ 2 } }$
           & $\sqrt{ 1 - \left( \frac{ x }{ x_{ 0 } } \right)^{ 2 } } $ \\
    23  & &  9 & $+2\beta \cos( 2\omega t )$ & $-2\beta \cos( 2\omega t )$ \\
    31  & 13 & & $x_{ 2 } \frac{ \partial x_{ 2 } }{ \partial r }$
           & $\dot{ x }_{ 2 } \frac{ \partial x_{ 2 } }{ \partial r }$ \\
    31  & &  5 & $\vecabold \frac{ \partial \vecrbold }{ \partial q_{ j } }$
           & $\vecabold \cdot \frac{ \partial \vecrbold }{ \partial q_{ j } }$ \\
    32  &  8 & & $\frac{ \partial \vecrbold^{ 2 } }{ \partial { q^{ j } } }$
           & $\frac{ \partial \vecrbold }{ \partial { q^{ j } } }$ \\
    34  &  6 & & $\dot{ \vecebold }_{ i } \vecebold_{ j }$
           & $\dot{ \vecebold }_{ i } \cdot \vecebold_{ j }$ \\
    34  &  8 & & $\vecebold_{ i } \vecebold_{ j }$
           & $\vecebold_{ i } \cdot \vecebold_{ j }$ \\
    34  & 10 & & $\dot{ \vecebold }_{ i } \vecebold_{ j }
                 + \vecebold_{ i } \dot{ \vecebold }_{ j }$
           & $\dot{ \vecebold }_{ i } \cdot \vecebold_{ j }
             + \vecebold_{ i } \cdot \dot{ \vecebold }_{ j }$ \\
    34  & 11 & & $\dot{ \vecebold }_{ i } \vecebold_{ j }$
           & $\dot{ \vecebold }_{ i } \cdot \vecebold_{ j }$ \\
    34  & 12 & & $\dot{ \vecebold }_{ i } \vecebold_{ j }$
           & $\dot{ \vecebold }_{ i } \cdot \vecebold_{ j }$ \\
    36  & 16 & & $( \xi_{ 1 } \dot{ \vecebold }_{ 1 }
                 + \xi_{ 2 } \dot{ \vecebold }_{ 2 }  ) \vecebold_{ 1 }$
           & $( \xi_{ 1 } \dot{ \vecebold }_{ 1 }
             + \xi_{ 2 } \dot{ \vecebold }_{ 2 }  ) \cdot \vecebold_{ 1 }$ \\
    36  & 16 & & $( \xi_{ 1 } \dot{ \vecebold }_{ 1 }
                 + \xi_{ 2 } \dot{ \vecebold }_{ 2 }  ) \vecebold_{ 2 }$
           & $( \xi_{ 1 } \dot{ \vecebold }_{ 1 }
             + \xi_{ 2 } \dot{ \vecebold }_{ 2 }  ) \cdot \vecebold_{ 2 }$ \\
    36  & 18 & & $\dot{ \vecebold }_{ 1 } \vecebold_{ 2 }$
           & $\dot{ \vecebold }_{ 1 } \cdot \vecebold_{ 2 }$ \\
    36  & 20 & & $\dot{ \vecebold }_{ 1 } \vecebold_{ 2 }$
           & $\dot{ \vecebold }_{ 1 } \cdot \vecebold_{ 2 }$ \\
           %        % & & & & \\
    \hline
  \end{tabular}

\end{center}

\VerSpaceTwo


\noindent
\StrWierszGora{31}{13} \\[0.3em]
\Jest
$( \dot{ x }_{ 1 } \vecibold + \dot{ x }_{ 2 } \vecjbold )
\left( \frac{ \partial x_{ 1 } }{ \partial r } \vecibold
  + \frac{ \partial x_{ 2 } }{ \partial r } \vecjbold \right)
\cdot \frac{ 1 }{ \vecOneBold }$ \\[0.5em]
\Powin
$( \dot{ x }_{ 1 } \vecibold + \dot{ x }_{ 2 } \vecjbold )
\cdot \left( \frac{ \partial x_{ 1 } }{ \partial r } \vecibold
  + \frac{ \partial x_{ 2 } }{ \partial r } \vecjbold \right)
\frac{ 1 }{ | \vecOneBold | }$ \\



% ############################










% ############################
\newpage

\subsection{ % Autor i tytuł dzieła
  Bogdan Skalmierski \\
  \textit{Mechanika. Tom~I: Podstawy mechaniki klasycznej},
  \cite{SkalmierskiMechanikPodstawyETCVolI1998}}


% ##################
\CenterBoldFont{Uwagi do konkretnych stron}


\Str{10} Warunek c) w~definicji przestrzeni topologicznej Hausdorffa jest
źle sformułowany, bowiem $\mathrm{R}_{ a }$ jest bez żadnych założeń
otoczeniem punktu B zawartym w~$\mathrm{R}_{ a }$. To co autor chciał tu
podać jest to definicja przestrzeni topologicznej Hausdorffa bazująca na
pojęciu bazy otoczeń (chyba), należy więc zajrzeć do książki do
topologi~by sprawdzić jak należy to poprawić.

\VerSpaceFour





Definicja homeomorfizmu jest trochę nie jasna, można bowiem odczytać ją
tak, że~choć funkcja $f$ musi być ciągła, to żaden zaś warunek nie jest
nałożony na $f^{ -1 }$.

\VerSpaceFour





\StrWierszGora{11}{8} Brak wcięcia akapitu.

\VerSpaceFour





\Str{18}{10} Brak wcięcia akapitu.

\VerSpaceFour






% ##################
\newpage

\CenterBoldFont{Błędy}


\begin{center}

  \begin{tabular}{|c|c|c|c|c|}
    \hline
    & \multicolumn{2}{c|}{} & & \\
    Strona & \multicolumn{2}{c|}{Wiersz} & Jest
                              & Powinno być \\ \cline{2-3}
    & Od góry & Od dołu &  &  \\ \hline
    % & & & & \\
    17 & & 13 & $( \vecabold \times \vecbbold ) \times \vecebold_{ j }$
           & $( \vecabold \times \vecbbold ) \cdot \vecebold_{ j }$ \\
    17 & & 13 & $( \vecebold_{ i } \times \vecebold_{ k } ) \times \vecebold_{ j }$
           & $( \vecebold_{ i } \times \vecebold_{ k } ) \cdot \vecebold_{ j }$ \\
    % & & & & \\
    % & & & & \\
    \hline
  \end{tabular}

\end{center}

\VerSpaceTwo


\noindent
\StrWierszDol{42}{10} \\
\Jest  Słońce znajduje się nie w centrum \\
\Powin ale Słońce nie znajduje się w centrum \\


% ############################










% #####################################################################
% #####################################################################
% Bibliografia

\bibliographystyle{plalpha}

\bibliography{PhilNaturBooks}{}





% ############################

% Koniec dokumentu
\end{document}
