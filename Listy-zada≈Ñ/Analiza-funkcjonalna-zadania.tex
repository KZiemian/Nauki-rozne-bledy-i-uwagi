% ---------------------------------------------------------------------
% Basic configuration and packages
% ---------------------------------------------------------------------
% Package for discovering wrong and outdated usage of LaTeX.
% More information to be found in l2tabu English version.
\RequirePackage[l2tabu, orthodox]{nag}
% Class of LaTeX document: {size of paper, size of font}[document class]
\documentclass[a4paper,11pt]{article}



% ---------------------------------------
% Packages not tied to particular normal language
% ---------------------------------------
% This package should improved spaces in the text.
\usepackage{microtype}
% Add few important symbols, like text Celcius degree
\usepackage{textcomp}



% ---------------------------------------
% Polonization of LaTeX document
% ---------------------------------------
% Basic polonization of the text
\usepackage[MeX]{polski}
% Switching on UTF-8 encoding
\usepackage[utf8]{inputenc}
% Adding font Latin Modern
\usepackage{lmodern}
% Package is need for fonts Latin Modern
\usepackage[T1]{fontenc}



% ---------------------------------------
% Setting margins
% ---------------------------------------
\usepackage[a4paper, total={14cm, 25cm}]{geometry}



% ---------------------------------------
% Setting vertical spaces in the text
% ---------------------------------------
% Setting space between lines
\renewcommand{\baselinestretch}{1.1}

% Setting space between lines in tables
\renewcommand{\arraystretch}{1.4}



% ---------------------------------------
% Packages for scientific papers
% ---------------------------------------
% Switching off \lll symbol, that I guess is representing letter ``Ł''.
% It collide with `amsmath' package's command with the same name
\let\lll\undefined
% Basic package from American Mathematical Society (AMS)
\usepackage[intlimits]{amsmath}
% Equations are numbered separately in every section.
\numberwithin{equation}{section}

% Other very useful packages from AMS
\usepackage{amsfonts}
\usepackage{amssymb}
\usepackage{amscd}
\usepackage{amsthm}

% Package with better looking calligraphy fonts
\usepackage{calrsfs}

% Package with better looking greek letters
% Example of use: pi -> \uppi
\usepackage{upgreek}
% Improving look of lambda letter
\let\oldlambda\Lambda
\renewcommand{\lambda}{\uplambda}





% ---------------------------------------
% Defining new environments (?)
% ---------------------------------------
% Defining enviroment ``Wniosek''
\newtheorem{corollary}{Wniosek}
\newtheorem{definition}{Definicja}
\newtheorem{theorem}{Twierdzenie}





% ------------------------------
% Private packages
% You need to put them in the same directory as .tex file
% ------------------------------
% Contains various command useful for working with a text
% \usepackage{latexgeneralcommands}
% Contains definitions useful for working with mathematical text
% \usepackage{mathcommands}

% Package for use in text about ordinary differential equations
% \usepackage{ODEcommands}





% ------------------------------
% Package ``hyperref''
% They advised to put it on the end of preambule
% ------------------------------
% It allows you to use hyperlinks in the text
\usepackage{hyperref}










% ------------------------------------------------------------------------------------
% Defining title and author of the text
\title{Analiza funkcjonalna, lista zadań}


% \date{}
% ------------------------------------------------------------------------------------










% ####################################################################
\begin{document}
% ####################################################################





% ######################################
% Title of the text
\maketitle
% ######################################





% ######################################
\section{Jacek Chmieliński \textit{Analiza funkcjonalna.
    Notatki do~wykładu}}

\label{sec:Oznaczenia-i-konwencje}
% ######################################


\begin{itemize}

\item[\romannumeral1)] 1, 2, 3, 4, 5, 6, 7, 8, 9, 10, 11, 12, 13, 14, 15,
  16, 17, 18, 19, 20, 21, 22, 23, 24.

\item[\romannumeral2)] 1, 2, 3, 4, 5, 6, 7, 8, 9, 10, 11, 12, 13, 14, 15,
  16, 17, 18, 19, 20, 21, 22, 23, 24, 25, 26, 27, 28, 29 , 30, 31, 32, 33,
  34, 35, 36, 37, 38, 39, 40, 41, 42, 43, 44, 45, 46, 47, 48.

\item[\romannumeral3)] 1, 2, 3, 4, 5, 6, 7, 8, 9, 10, 11, 12, 13, 14, 15,
  16, 17, 18, 19, 20, 21, 22, 23.

\item[\romannumeral4)] 1, 2, 3, 4, 5, 6, 7, 8, 9, 10, 11, 12, 13, 14, 15,
  16, 17, 18, 19, 20, 21, 22, 23, 24, 25, 26, 27, 28, 29.

\item[\romannumeral5)] 1, 2, 3, 4, 5, 6, 7, 8, 9, 10.

\end{itemize}
% ############################










% ######################################
\section{M. Reed, B. Simon \textit{Methods of Modern
    Mathematical Physics~I: Functional Analysis}}

\label{sec:Oznaczenia-i-konwencje}
% ######################################


\begin{itemize}

\item[\romannumeral1)] 1, 3, 4, 5, 6, 7, 8, 9, 10, 11, 12, 14, 27, 29,
  30, 31, 32, 1, 2, 3, 4, 5, 6, 7, 8, 9, 10, 11, 12, 13 , 14, 15, 16, 17,
  18, 19, 20, 21, 22, 23, 24, 25, 26, 27, 28, 29, 30, 31, 32, 33.

\item[\romannumeral1)] 1, 2, 3, 4, 5, 6, 7, 8, 9, 10, 11, 12, 13, 14, 15,
  16, 17. \\
  3: 1, 2, 3, 4, 5, 6, 7, 8, 9, 10, 11, 12, 13, 14, 15. \\
  4: 1, 2, 3, 4, 5, 6, 7, 8, 9, 10, 11, 12, 13, 14, 15, 16. \\
  5: 1, 2, 3, 4, 5, 6, 7, 8, 9, 10, 11, 12, 13, 14, 15. \\
  6: 1, 2, 3, 4, 5, 6, 7, 8, 9, 10, 11, 12, 13, 14, 15, 16, 17.

\item[\romannumeral2)] 1: 1, 2, 3, 4, 5, 6, 7, 8, 9, 10, 11, 12, 13, 14, 15,
  16, 17, 18. \\
  2: 1, 2, 3, 4, 5, 6, 7, 8, 9, 10, 11, 12, 13, 14, 15, 16, 17, 18, 19, 20,
  21, 22, 23. \\
  3: 1, 2, 3, 4, 5, 6, 7, 8, 9, 10, 11, 12, 13, 14, 15, 16. \\
  4: 1, 2, 3, 4, 5, 6, 7, 8, 9, 10, 11, 12, 13, 14, 15, 16, 17, 18, 19. \\
  5: 1, 2, 3, 4, 5, 6, 7, 8, 9, 10, 11, 12, 13, 14.

\item[\romannumeral3)] 1: 1, 2, 3, 4, 5, 6, 7, 8, 9, 10, 11, 12, 13, 14. \\
  2: 1, 2, 3, 4, 5, 6, 7, 8, 9, 10, 11, 12, 13, 14, 15, 16, 17. \\
  3: 1, 2, 3, 4, 5, 6, 7, 8, 9, 10, 11, 12, 13, 14, 15, 16, 17, 18, 19. \\
  4: 1, 2, 3, 4, 5, 6, 7, 8, 9, 10, 11, 12, 13, 14, 15, 16, 17, 18, 19, 20,
  21, 22, 23. \\
  5: 1, 2, 3, 4, 5, 6, 7, 8, 9, 10, 11, 12, 13, 14, 15, 16, 17, 18. \\
  6: 1, 2, 3, 4, 5, 6, 7, 8, 9, 10, 11, 12, 13, 14, 15, 16, 17, 18, 19, 20,
  21.

\item[\romannumeral4)] 1: 1, 2, 3, 4, 5, 6, 7, 8, 9, 10, 11, 12, 13, 14, 15,
  16, 17, 18, 19, 20, 21, 22, 23. \\
  2: 1, 2, 3, 4, 5, 6, 7, 8, 9, 10, 11, 12, 13, 14, 15, 16, 17, 18, 19, 20,
  21. \\
  3: 1, 2, 3, 4, 5, 6, 7, 8, 9, 10, 11, 12, 13, 14, 15, 16, 17, 18, 19, 20,
  21, 22. \\
  4: 1, 2, 3, 4, 5, 6, 7, 8, 9, 10, 11, 12, 13, 14, 15, 16, 17. \\
  5: 1, 2, 3, 4, 5, 6, 7, 8, 9, 10, 11, 12, 13, 14, 15, 16, 17, 18, 19, 20,
  21. \\
  6: 1, 2, 3, 4, 5, 6, 7, 8, 9, 10, 11, 12, 13, 14, 15, 16, 17.

\item[\romannumeral5)] 1: 1, 2, 3, 4, 5, 6, 7, 8, 9, 10, 11. \\
  2: 1, 2, 3, 4, 5, 6, 7, 8, 9, 10, 11, 12, 13, 14, 15, 16, 17, 18, 19, 20,
  21. \\
  3: 1, 2, 3, 4, 5, 6, 7, 8, 9, 10, 11, 12, 13, 14, 15, 16, 17. \\
  4: 1, 2, 3, 4, 5, 6, 7, 8, 9, 10, 11, 12, 13, 14, 15, 16. \\
  5: 1, 2, 3, 4, 5, 6, 7, 8, 9, 10, 11, 12, 13, 14, 15, 16, 17, 18, 19.

\item[\romannumeral6)] 1: 1, 2, 3, 4, 5, 6, 7, 8, 9, 10, 11, 12, 13, 14,
  15, 16, 17, 18, 19, 20, 21, 22, 23, 24. \\
  2: 1, 2, 3, 4, 5, 6, 7, 8, 9, 10, 11, 12, 13. \\
  3: 1, 2, 3, 4, 5, 6, 7, 8, 9, 10, 11. \\
  4: 1, 2, 3, 4, 5, 6, 7, 8, 9, 10, 11, 12, 13, 14, 15, 16. \\
  5: 1, 2, 3, 4, 5, 6, 7, 8, 9, 10, 11, 12, 13, 14, 15. \\
  6: 1, 2, 3, 4, 5, 6, 7, 8, 9, 10, 11, 12, 13, 14, 15.

\item[\romannumeral7)] 1: 1, 2, 3, 4, 5, 6, 7, 8, 9, 10, 11, 12, 13, 14, 15,
  16, 17, 18, 19. \\
  2: 1, 2, 3, 4, 5, 6, 7, 8, 9, 10, 11, 12, 13, 14, 15, 16, 17. \\
  3: 1, 2, 3, 4, 5, 6, 7, 8, 9, 10, 11, 12, 13, 14, 15, 16, 17. \\
  4: 1, 2, 3, 4, 5, 6, 7, 8, 9, 10, 11, 12, 13, 14, 15, 16, 17, 18, 19,
  20. \\
  5: 1, 2, 3, 4, 5, 6, 7, 8, 9, 10, 11, 12, 13, 14, 15.

\item[\romannumeral8)] 1: 1, 2, 3, 4, 5, 6, 7, 8, 9, 10, 11, 12, 13, 14,
  15, 16, 17, 18. \\
  2: 1, 2, 3, 4, 5, 6, 7, 8, 9, 10, 11, 12, 13, 14, 15, 16, 17, 18, 19, 20,
  21. \\
  3: 1, 2, 3, 4, 5, 6, 7, 8, 9, 10, 11, 12. \\
  4: 1, 2, 3, 4, 5, 6, 7, 8, 9, 10, 11.

\item[\romannumeral9)] 1: 1, 2, 3, 4, 5, 6, 7, 8, 9, 10, 11, 12, 13, 14, 15,
  16, 17, 18, 19, 20, 21, 22, 23, 24, 25, 26, 27, 28, 29. \\
  2: 1, 2, 3, 4, 5, 6, 7, 8, 9, 10, 11, 12, 13, 14, 15, 16, 17, 18, 19. \\
  3: 1, 2, 3, 4, 5, 6, 7, 8, 9, 10, 11, 12, 13, 14, 15, 16, 17, 18, 19, 20,
  21. \\
  4: 1, 2, 3, 4, 5, 6, 7, 8, 9, 10, 11, 12, 13, 14, 15, 16.

\end{itemize}
% ############################










% ####################################################################
% ####################################################################
% Bibliography

\bibliographystyle{plalpha}

\bibliography{MathematicsBooks}{}





% ############################

% End of the document
\end{document}
